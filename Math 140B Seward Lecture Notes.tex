% Note for any github stalkers. I am currently in the process
% of learning LaTeX. I don't know what I'm doing yet. Sorry
% if my code absolutely sucks.

\documentclass{book}

\usepackage{fontspec} % used to import Calibri
\usepackage{anyfontsize} % used to adjust font size

% needed for inch and other length measurements
% to be recognized
\usepackage{calc}

% for colors and text effects as is hopefully obvious
\usepackage[dvipsnames]{xcolor}
\usepackage{soul}

% control over margins
\usepackage[margin=1in]{geometry}
\usepackage[strict]{changepage}

\usepackage{mathtools}
\usepackage{amsfonts}
\usepackage{bm}

\usepackage[scr=rsfso, scrscaled=.96]{mathalpha}

\usepackage{amssymb} % originally imported to get the proof square
\usepackage{xfrac}
\usepackage[overcommands]{overarrows} % Get my preferred vector arrows...
\usepackage{relsize}

% Just am using this to get a dashed line in a table...
% Also you apparently want this to be inactive if you aren't
% using it because it slows compilation.
\usepackage{arydshln} \ADLinactivate 
\newenvironment{allowTableDashes}{\ADLactivate}{\ADLinactivate}

\usepackage{graphicx}
\graphicspath{{./140B_images/}}

\usepackage{tikz}
   \usetikzlibrary{arrows.meta}
   \usetikzlibrary{graphs, graphs.standard}

\usepackage{quiver} %commutative diagrams


\newfontfamily{\calibri}{Calibri}
\setlength{\parindent}{0pt}
\definecolor{RawerSienna}{HTML}{945D27}

% ~~~~~~~~~~~~~~~~~~~~~~~~~~~~~~~~~~~~~~~~~~~~~~~~~~
%Arrow Commands:

% Thank you Bernard, gernot, and Sigur who I copied this from:
% https://tex.stackexchange.com/questions/364096/command-for-longhookrightarrow
\newcommand{\hooklongrightarrow}{\lhook\joinrel\longrightarrow}
\newcommand{\hooklongleftarrow}{\longleftarrow\joinrel\rhook}
\newcommand{\hookxlongrightarrow}[2][]{\lhook\joinrel\xrightarrow[#1]{#2}}
\newcommand{\hookxlongleftarrow}[2][]{\xleftarrow[#1]{#2}\joinrel\rhook}

% Thank you egreg who I copied from:
% https://tex.stackexchange.com/questions/260554/two-headed-version-of-xrightarrow
\newcommand{\longrightarrowdbl}{\longrightarrow\mathrel{\mkern-14mu}\rightarrow}
\newcommand{\longleftarrowdbl}{\leftarrow\mathrel{\mkern-14mu}\longleftarrow}

\newcommand{\xrightarrowdbl}[2][]{%
  \xrightarrow[#1]{#2}\mathrel{\mkern-14mu}\rightarrow
}
\newcommand{\xleftarrowdbl}[2][]{%
  \leftarrow\mathrel{\mkern-14mu}\xleftarrow[#1]{#2}
}

% ~~~~~~~~~~~~~~~~~~~~~~~~~~~~~~~~~~~~~~~~~~~~~~~~~~

\newcommand{\hOne}{%
   \color{Black}%
   \fontsize{14}{16}\selectfont%
}
\newcommand{\hTwo}{%
   \color{MidnightBlue}%
   \fontsize{13}{15}\selectfont%
}
\newcommand{\hThree}{%
   \color{PineGreen!85!Orange}
   \fontsize{13}{15}\selectfont%
}
\newcommand{\hFour}{%
   \color{Cerulean}
   \fontsize{12}{14}\selectfont%
}
\newcommand{\myComment}{%
   \color{RawerSienna}%
   \fontsize{12}{14}\selectfont%
}
\newcommand{\pracOne}{
   \color{BrickRed}%
   \fontsize{13}{15}\selectfont%
}
\newcommand{\pracTwo}{
   \color{Orange}%
   \fontsize{12}{14}\selectfont%
}
\newcommand{\teachComment}{
   \color{Orange}%
   \fontsize{12}{14}\selectfont%
}
\newcommand{\exOne}{%
   \color{Purple}%
   \fontsize{14}{16}\selectfont%
}
\newcommand{\exTwo}{%
   \color{RedViolet}%
   \fontsize{13}{15}\selectfont%
}
\newcommand{\exP}{%
   \color{VioletRed}%
   \fontsize{12}{14}\selectfont%
}
% ~~~~~~~~~~~~~~~~~~~~~~~~~~~~~~~~~~~~~~~~~~~~~~~~

\newcommand{\cyPen}[1]{{\vphantom{.}\color{Cerulean}#1}}
\newcommand{\pinkPen}[1]{{\vphantom{.}\color{VioletRed}#1}}

\newenvironment{myIndent}{%
   \begin{adjustwidth}{2.5em}{0em}%
}{%
   \end{adjustwidth}%
}

\newenvironment{myDindent}{%
   \begin{adjustwidth}{5em}{0em}%
}{%
   \end{adjustwidth}%
}

\newenvironment{myTindent}{%
   \begin{adjustwidth}{7.5em}{0em}%
}{%
   \end{adjustwidth}%
}

\newenvironment{myConstrict}{%
   \begin{adjustwidth}{2.5em}{2.5em}%
}{%
   \end{adjustwidth}%
}

\newcommand{\udefine}[1]{{%
   \setulcolor{Red}%
   \setul{0.14em}{0.07em}%
   \ul{#1}%
}}

\newcommand{\uuline}[2][.]{%
{\vphantom{a}\color{#1}%
\rlap{\rule[-0.18em]{\widthof{#2}}{0.06em}}%
\rlap{\rule[-0.32em]{\widthof{#2}}{0.06em}}}%
#2}

\newcommand*{\markDate}[1]{%
   {\huge \color{Black} \textbf{#1} \newline}%
}

\newcommand*{\markHW}[1]{%
   {\huge \color{Black} \textbf{#1} \newline}%
}

\newcommand{\pprime}{{\prime\prime}}
\newcommand{\ppprime}{{\prime\prime\prime}}
\newcommand{\suchthat}{ \hspace{0.5em}s.t.\hspace{0.5em}}
\newcommand{\rea}[1]{\mathrm{Re}(#1)}
\newcommand{\ima}[1]{\mathrm{Im}(#1)}
\newcommand{\comp}{\mathsf{C}}
\newcommand{\myHS}{ \hspace{0.5em}}
\newcommand{\diam}[1]{\mathrm{diam}(#1)}
\newcommand{\domain}[1]{\mathrm{dom}(#1)}


\newcounter{PropNumber}
\setcounter{PropNumber}{82}
\newcommand{\propCount}[1][1]{%
   \addtocounter{PropNumber}{#1}%
   \thePropNumber%
}
\newcounter{SubPropNumber}
\newcommand{\subPropCount}[1][1]{%
   \addtocounter{SubPropNumber}{1}%
   \theSubPropNumber%
}
\newcommand{\resetSubPropCount}{%
   \setcounter{SubPropNumber}{0}%
}

\newcommand{\myId}{\mathrm{Id}}
\newcommand{\myIm}{\mathrm{im}}
\newcommand{\myObj}{\mathrm{Obj}}
\newcommand{\myHom}{\mathrm{Hom}}
\newcommand{\myEnd}{\mathrm{End}}
\newcommand{\myAut}{\mathrm{Aut}}

\newcommand{\mcateg}[1]{{\bm{\mathsf{#1}}}}

% Thank you Gonzalo Medina and Moriambar who wrote this on stack exchange:
%https://tex.stackexchange.com/questions/74125/how-do-i-put-text-over-symbols%
\newcommand{\myequiv}[1]{\stackrel{\mathclap{\mbox{\footnotesize{$#1$}}}}{\equiv}}

% Thank you chs who wrote this on stack exchange:
%https://tex.stackexchange.com/questions/89821/how-to-draw-a-solid-colored-circle%
\newcommand{\filledcirc}[1][.]{\ensuremath{\hspace{0.05em}{\color{#1}\bullet}\mathllap{\circ}\hspace{0.05em}}}

%Thank you blerbl who wrote this on stack exchange:
%https://tex.stackexchange.com/questions/25348/latex-symbol-for-does-not-divide
\newcommand{\ndiv}{\hspace{-0.3em}\not|\hspace{0.35em}}

\newcommand{\mySepOne}[1][.]{%
   {\noindent\color{#1}{\rule{6.5in}{1mm}}}\\%
}
\newcommand{\mySepTwo}[1][.]{%
   {\noindent\color{#1}{\rule{6.5in}{0.5mm}}}\\%
}

\newenvironment{myClosureOne}[2][.]{%
   \color{#1}%
   \begin{tabular}{|p{#2in}|} \hline \\%
}{%
   \\ \hline \end{tabular}%
}

\newcommand{\fillInBlank}[2][.]{{%
   \color{#1}%
   \rule[-0.12em]{#2em}{0.06em}\rule[-0.12em]{#2em}{0.06em}%
   \rule[-0.12em]{#2em}{0.06em}
}}

\newcommand{\retTwo}{\hfill\bigbreak}

\newcounter{LectureNumber}
\newcommand*{\markLecture}[1]{%
   \stepcounter{LectureNumber}%
   {\huge \color{Black} \textbf{Lecture \theLectureNumber: #1} \newline}%
}

\newcommand{\myVS}{\vphantom{$\int_a^b$}}

% Overarrow stuff:
% ~~~~~~~~~~~~~~~~~~~~~~~~~~~~~~~~~~~~~~~~~~~~~~~~~~~~~~~~~~
\NewOverArrowCommand{myVector}{%
   start = {{\smallermathstyle\relbar}},
   middle = {{\smallermathstyle\relbareda}},
   end={{\rightharpoonup}}, space before arrow=0.15em,
   space after arrow=-0.045em,
}

\NewOverArrowCommand{myBar}{%
   start = {{\relbar}},
   middle = {{\relbar}},
   end={{\relbar}}, space before arrow=0.15em,
   space after arrow=-0.025em,
}

% ~~~~~~~~~~~~~~~~~~~~~~~~~~~~~~~~~~~~~~~~~~~~~~~~~~~~~~~~~~~~

\newcommand{\mVec}[1]{\myVector{#1}}
\newcommand{\mVecAst}[1]{\myVector*{#1}}
\newcommand{\mMat}[1]{\mathbf{#1}}

\title{Math 140B Lecture Notes (Professor: Brandon Seward)}
\author{Isabelle Mills}

\begin{document}
\maketitle{}
\setul{0.14em}{0.07em}
\calibri

\hOne
\markLecture{4/1/2024}

Let $f: E \longrightarrow \mathbb{R}$ where $E \subseteq \mathbb{R}$.

\begin{myIndent}\hTwo
   Since $E$ is the domain of $f$, we shall also refer to it as $\domain{f}$.\retTwo
\end{myIndent}

Fix a point $x \in E \cap E^\prime$. Then consider the function $\frac{f(t)-f(x)}{t-x}$ for $t \in \domain{f}\setminus\{x\}$\\[2pt] and define the \udefine{derivative} of $f$ at $x$ to be $f^\prime(x) = \lim\limits_{t\rightarrow x}\left(\frac{f(t)-f(x)}{t-x}\right)$ provided that this\\[2pt] limit exists. When the above limit exists, we say $f$ is differentiable at $x$. \\ [2pt]

We say $f$ is differentiable on $D \subseteq E$ if $f$ is differentiable at every point in $D$,\\ and if $f$ is differentiable on its entire domain, then we call $f$ \udefine{differentiable}.\retTwo

The function $f^\prime(x) = \lim\limits_{t\rightarrow x}\left(\frac{f(t)-f(x)}{t-x}\right)$ is called the \udefine{derivative} of $f$.\retTwo


{\begin{myIndent}\hTwo
   Proposition \propCount: If $f$ is differentiable at $x$, then $f$ is continuous at $x$.\\ [-10pt]
   
   \begin{myIndent}\hThree
      Proof:\\
      Note that $\lim\limits_{t\rightarrow x}\left(f(t)\right) = \lim\limits_{t\rightarrow x}\left((t-x)\frac{f(t)-f(x)}{t-x} + f(x)\right)$.\\

      Now $\lim\limits_{t\rightarrow x}(t-x) = 0$ and we know $\lim\limits_{t\rightarrow x}\frac{f(t)-f(x)}{t-x} = f^\prime(x)$ exists because $f$ is\\ [3pt] differentiable at $x$. Also, obviously $\lim\limits_{t\rightarrow x}f(x) = f(x)$.\retTwo
      
      Thus by proposition 66 {\color{RawerSienna}(check 140A notes)}, we know that:\\
      
      \begin{tabular}{l}
         $\lim\limits_{t\rightarrow x}\left((t-x)\frac{f(t)-f(x)}{t-x} + f(x)\right) = \lim\limits_{t\rightarrow x}(t-x)\lim\limits_{t\rightarrow x}\left(\frac{f(t)-f(x)}{t-x}\right) + \lim\limits_{t\rightarrow x}f(x)$\\ [3pt]

         $\phantom{\lim\limits_{t\rightarrow x}\left((t-x)\frac{f(t)-f(x)}{t-x} + f(x)\right)} = 0\cdot f^\prime(x) + f(x)$ \\ [-4pt]
         $\phantom{\lim\limits_{t\rightarrow x}\left((t-x)\frac{f(t)-f(x)}{t-x} + f(x)\right)} = f(x)$
      \end{tabular}\retTwo

      Thus, $f$ is continuous at $x$.\retTwo
   \end{myIndent}
\end{myIndent}}


{\begin{center} \teachComment
   \begin{myClosureOne}{5}
      Notes:
      \begin{enumerate}
         \item The above proposition says that differentiability is stronger than\newline continuity.
         \item The converse of this proposition is false. For example, the function\newline $f(x) = |x|$ is continuous at $x = 0$ but not differentiable at\newline $x = 0$.
      \end{enumerate}
   \end{myClosureOne}
\end{center}}

\newpage

{\begin{myIndent}\hTwo
   Proposition \propCount: Suppose $f$ and $g$ are real-valued functions with\\ $\domain{f}, \domain{g} \subseteq \mathbb{R}$. Also suppose $f$ and $g$ are differentiable at $x$. Then\\ $f + g$, $fg$, and (when $g(x) \neq 0$) $\frac{f}{g}$ are differentiable at $x$ with:\\
   
   \begin{tabular}{l c c c c c}
      (A)\quad\quad $(f + g)^\prime(x) = f^\prime(x) + g^\prime(x)$ & &&&&{\hFour(sum rule)} \\ [4pt]
      (B)\quad\quad $(fg)^\prime(x) = f^\prime(x)g(x) + f(x)g^\prime(x)$ & &&&& {\hFour(product rule)} \\ [4pt]
      (C)\quad\quad $\left(\dfrac{f}{g}\right)^\prime\hspace{-0.3em}(x) = \dfrac{f^\prime(x)g(x) - f(x)g^\prime(x)}{(g(x))^2}$ & &&&& {\hFour(quotient rule)}
   \end{tabular}\retTwo

   \begin{myIndent}\hThree
      Proof:
      \begin{myIndent}
         \begin{itemize}
            \item[(A)] Since both $f$ and $g$ are differentiable, we know that both\\ $f^\prime(x) = \lim\limits_{t\rightarrow x}\frac{f(t) - f(x)}{t - x}$ and $g^\prime(x) = \lim\limits_{t\rightarrow x}\frac{g(t) - g(x)}{t - x}$ exist. So\\ by proposition 66:\\ [-11pt]
            
            \hspace{-1.5em}${(f + g)^\prime(x) = \lim\limits_{t\rightarrow x}\frac{f(t) + g(t) - f(x) - g(x)}{t-x} = \lim\limits_{t\rightarrow x}\frac{f(t) - f(x)}{t - x} + \lim\limits_{t\rightarrow x}\frac{g(t) - g(x)}{t - x}}$\\ [4pt]

            This means that $(f + g)^\prime(x) = f^\prime(x) + g^\prime(x)$.\\ [-2pt]

            \item[(B)] Note that:\\
            \begin{tabular}{l}
               $(fg)^\prime(x) = \lim\limits_{t\rightarrow x}\frac{f(t)g(t) - f(x)g(x)}{t-x}$ \\

               $\phantom{(fg)^\prime(x)} = \lim\limits_{t\rightarrow x}\frac{f(t)g(t) \cyPen{\vphantom{.} - f(x)g(t) + f(x)g(t)} - f(x)g(x)}{t-x}$ \\

               $\phantom{(fg)^\prime(x)} = \lim\limits_{t\rightarrow x}\left( g(t)\frac{f(t) - f(x)}{t - x} + f(x)\frac{g(t) - g(x)}{t-x} \right)$
            \end{tabular}\\

            By proposition 83, $g(t) \rightarrow g(x)$ as $t \rightarrow x$. Also, since both $f$\\ and $g$ are differentiable, we know $f^\prime(x) = \lim\limits_{t\rightarrow x}\frac{f(t) - f(x)}{t - x}$ and\\[-2pt] $g^\prime(x) = \lim\limits_{t\rightarrow x}\frac{g(t) - g(x)}{t - x}$ exist. So by proposition 66:\\ [-4pt]

            \hspace{-0.5em}$\lim\limits_{t\rightarrow x}\left( g(t)\frac{f(t) - f(x)}{t - x} + f(x)\frac{g(t) - g(x)}{t-x} \right) = f^\prime(x)g(x) + f(x)g^\prime(x)$.\\

            \item[(C)] Note that:\\
            \begin{tabular}{l}
               $\left(\frac{f}{g}\right)^\prime\hspace{-0.3em}(x) = \lim\limits_{t\rightarrow x}\frac{\frac{f(t)}{g(t)} - \frac{f(x)}{g(x)}}{t - x}$ \\ [2pt]

               $\hphantom{\left(\frac{f}{g}\right)^\prime\hspace{-0.3em}(x)} = \lim\limits_{t\rightarrow x}\left(\frac{1}{g(x)g(t)}\frac{f(t)g(x) - f(x)g(t)}{t - x}\right)$ \\ [8pt]

               $\hphantom{\left(\frac{f}{g}\right)^\prime\hspace{-0.3em}(x)} = \lim\limits_{t\rightarrow x}\left(\frac{1}{g(x)g(t)}\frac{f(t)g(x) \cyPen{\vphantom{.} - f(x)g(x) + f(x)g(x)} - f(x)g(t)}{t - x}\right)$ \\ [8pt]

               $\hphantom{\left(\frac{f}{g}\right)^\prime\hspace{-0.3em}(x)} = \lim\limits_{t\rightarrow x}\left(\frac{1}{g(x)g(t)} \left( g(x)\frac{f(t)-f(x)}{t-x} - f(x)\frac{g(t) - g(x)}{t-x} \right)  \right)$
            \end{tabular}\\ [6pt]
            Now, for the same reasons as before, we can use propositions 83\\ and 66 to separate the parts of the above limit to get that the above limit equals:

            {\centering $\frac{1}{(g(x))^2}\left(g(x)f^\prime(x) - f(x)g^\prime(x)\right)$\par}
         \end{itemize}
      \end{myIndent}
   \end{myIndent}
\end{myIndent}}

\newpage

\exOne

If $f(x) = \alpha$ where $\alpha \in \mathbb{R}$ is constant, then trivially $f^\prime(x) = 0$ for all $x$.\\ Meanwhile, if $f(x) = x$, then we can trivially find that $f^\prime(x) = 1$.\retTwo

Claim 1: For all $n \in \mathbb{Z}^+$, if $f(x) = x^n$, then $f^\prime(x) = nx^{n-1}$.\\ [-15pt]

{\begin{myIndent}\exTwo
   Proof: (we proceed by induction)\retTwo

   Base Case: 
   {\begin{myIndent} \exP
      If $n = 1$, then for $f(x) = x^1$, we have that $f^\prime(x) = 1\cdot x^0$.\retTwo
   \end{myIndent}}

   Induction: 
   {\begin{myIndent}\exP
      Now assume $n > 1$, and for $f(x) = x^{n-1}$, we have that ${f^\prime(x) = (n-1)x^{n-2}\text{.}}$\\ For the rest of this proof, I'll abreviate the derivative of $x^n$ as $(x^n)^\prime$ and the\\ derivative of $x^{n-1}$ as $(x^{n-1})^\prime$. Then using product rule, we know that:\\ [-11pt]

      ${\hspace{-2.5em}(x^n)^\prime = x(x^{n-1})^\prime + 1\cdot x^{n-1} = x\cdot (n-1)x^{n-2} + x^{n-1} = ((n-1) + 1)x^{n-1} = nx^{n-1}}$\retTwo
   \end{myIndent}}
\end{myIndent}}

Claim 2: If $f$ is differentiable at $x$ and $\alpha \in \mathbb{R}$, then $(\alpha f)^\prime(x) = \alpha f^\prime(x)$.\\ [-15pt]

{\begin{myIndent}\exTwo
   Proof:\\ By the product rule: $(\alpha f)^\prime(x) = \alpha f^\prime + (\alpha)^\prime f = \alpha f^\prime + 0\cdot f = \alpha f^\prime$.\retTwo
\end{myIndent}}

These combined with proposition 84 tells us that both polynomials and rational\\ functions are differentiable over their domains.

\mySepTwo

\hOne
{\begin{myIndent}\hTwo
   Proposition \propCount: (chain rule)\\
   Let $f$ and $g$ be real-valued functions with $\domain{f}, \domain{g} \subseteq \mathbb{R}$. Let $x \in \mathbb{R}$.\\ Suppose that $f$ is differentiable at $x$ and that $g$ is differentiable at $f(x)$. Then\\ $g \circ f$ is differentiable at $x$ and $(g \circ f)^\prime(x) = g^\prime(f(x))f^\prime(x)$.\\
   
   {\begin{center}\exTwo
      \begin{myClosureOne}{4.5}
         $\hphantom{.}$\\[-24pt] \ul{Intuition}:
         \begin{myIndent}
            $\lim\limits_{t\rightarrow x}\left(\frac{g(f(t)) - g(f(x))}{\pinkPen{f(t) - f(x)}}\cdot \frac{\pinkPen{f(t)-f(x)}}{t-x}\right) = g^\prime(f(t))\cdot f^\prime(t)$.\newline
         \end{myIndent}

         That said, the issue with this intuition is that we need to\\ address the possibility that $f(t) - f(x) = 0$.\\ [-8pt]
      \end{myClosureOne}\retTwo
   \end{center}}

   {\begin{myIndent}\hThree
      Proof:\\
      Set $y = f(x)$ and define $v(s) = \left\{
      \begin{matrix}
         \frac{g(s) - g(y)}{s-y} - g^\prime(y) & \text{ if } s \neq y \\
         0 & \text{ if } s = y
      \end{matrix}\right.$\retTwo

      Note that $v$ is continuous at $y$. This is because $g$ being differentiable\\ at $f(x) = y$ means that:
      
      {\centering $\lim\limits_{s\rightarrow y}v(s) = \lim\limits_{s\rightarrow y}\left(\frac{g(s) - g(y)}{s-y} - g^\prime(y)\right) = g^\prime(y) - g^\prime(y) = 0 = v(y)$.\retTwo\par}

      \newpage

      Also, since $f$ is differentiable at $x$, we know that $f$ is continuous at $x$.\\ Therefore, $v \circ f$ is continuous at $x$ by proposition 68. Additionally, setting\\ $s = f(t)$, we know that $s \rightarrow y$ as $t \rightarrow x$ because $f$ is continuous at $x$. Thus:

      {\center $ \lim\limits_{t\rightarrow x}v(f(t)) = \lim\limits_{s\rightarrow y}v(s) = 0$ \retTwo\par}
      
      Finally, note that $g(s) - g(y) = (s-y)(g^\prime(y) + v(s))$ for all $s$. Thus by\\ substituting that into our limit:\\ [-8pt]
      \begin{myIndent}
         \begin{tabular}{l}
            $(g \circ f)^\prime(x) = \lim\limits_{t\rightarrow x}\frac{g(f(t)) - g(f(x))}{t - x}$ \\ [8pt]
            $\phantom{(g \circ f)^\prime(x) } = \lim\limits_{t\rightarrow x}\frac{f(t) - f(x)}{t - x}(g^\prime(f(x)) + v(f(t)))$ \\ [8pt]
            $\phantom{(g \circ f)^\prime(x) } = f^\prime(x)\left(g^\prime(f(x)) + 0\right)$\quad\quad (by proposition 66)
         \end{tabular}\retTwo
      \end{myIndent}
   \end{myIndent}}
\end{myIndent}}

\markLecture{4/3/2024}
\exOne\mySepTwo\\ [-12pt]
To start off lecture, here is some intuition about the behavior of derivatives. We'll\\ formally define sine and cosine later (on page \_\_) but for this section please take\\ for granted that $(\sin(x))^\prime = \cos(x)$. Additionally, please take for granted that the\\ power rule holds for non-positive integer exponents.\retTwo

{\begin{myIndent}\exTwo
   \begin{enumerate}
      \item Define $f(x) = \left\{
      \begin{matrix}
         x\sin(\frac{1}{x}) & \text{ if } x \neq 0 \\
         0 & \text{ if } x = 0
      \end{matrix}\right.$\\

      When $x \neq 0$, we have by chain rule that $f^\prime(x) = \sin(\frac{1}{x}) - \frac{1}{x}\cos(\frac{1}{x})$.\\ Meanwhile if $x = 0$, then $\frac{f(t) - f(0)}{t - 0} = \frac{t\sin(\frac{1}{t})}{t} = \sin(\frac{1}{t})$ when $t \neq 0$.\\ So $\lim\limits_{t\rightarrow 0}\left(\frac{f(t) - f(0)}{t - 0}\right)$ does not exist, meaning $f$ is not differentiable at $x$.\\

      This shows that $\domain{f^\prime}$ can be a proper subset of $\domain{f}$.\\

      \item Define $g(x) = \left\{
      \begin{matrix}
         x^2\sin(\frac{1}{x}) & \text{ if } x \neq 0 \\
         0 & \text{ if } x = 0
      \end{matrix}\right.$\\
      
      When $x \neq 0$, we have by chain rule that $g^\prime(x) = 2x\sin(\frac{1}{x}) - \cos(\frac{1}{x})$.\\ Meanwhile when $t \neq 0$:

      {\center $\left|\frac{g(t) - g(0)}{t - 0}\right| = \left|\frac{t^2\sin(\frac{1}{t})}{t}\right| = \left|t\sin(\frac{1}{t})\right| \leq |t|$. \retTwo\par}

      Thus $0 = \lim\limits_{t\rightarrow 0}(-t) \leq \lim\limits_{t\rightarrow 0}\left(\frac{g(t) - g(0)}{t - 0}\right) \leq \lim\limits_{t\rightarrow 0}(t) = 0$, meaning $g^\prime(0) = 0$.\\ [2pt] So $\domain{g^\prime} = \domain{g}$. That said, note that $g^\prime$ has a discontinuity of the\\ [2pt] second kind at $0$. Therefore, this shows that the derivative of a function\\ [2pt] does not have to be continuous.
   \end{enumerate}
\end{myIndent}}

\mySepTwo

\newpage

\hOne Let $X$ be a metric space. A function $f: X \longrightarrow \mathbb{R}$ has a \udefine{local maximum} at $p \in X$\\ if $\exists \delta > 0 \suchthat \forall x \in B_\delta(p),\myHS f(x) \leq f(p)$. Similarly, $f$ has a \udefine{local minimum} if\\ $\exists \delta > 0 \suchthat \forall x \in B_\delta(p),\myHS f(x) \geq f(p)$.\retTwo

{\begin{myIndent}\hTwo
   Proposition \propCount: Let $f: (a, b) \longrightarrow \mathbb{R}$. If $f$ has a local maximum at $x$ and $f$ is\\ differentiable at $x$, then $f^\prime(x) = 0$.\retTwo
   {\begin{myIndent} \hThree
      Proof:\\
      Let $\delta > 0$ so that $\forall t \in B_\delta(x),\myHS f(t) \leq f(x)$. Then for all $t \in (x-\delta, x)$,\\ $\frac{f(t) - f(x)}{t-x} \geq 0$. So $f^\prime(x) \geq 0$. Similarly for all $t \in (x, x+\delta)$, we have\\ $\frac{f(t) - f(x)}{t - x} \leq 0$. Thus $f^\prime(x) \leq 0$.\retTwo

      Hence $f^\prime(x) = 0$.\retTwo

      
      {\begin{myTindent} \hFour
         Note that analogous reasoning can show that if $f$ has a local\\ minimum at $x$ and $f$ is differentiable at $x$, then $f^\prime(x) = 0$.\retTwo
      \end{myTindent}}
   \end{myIndent}}

   Proposition \propCount: If $f, g: [a, b] \longrightarrow \mathbb{R}$ are continuous on $[a, b]$ and differentiable\\ on $(a, b)$, then there exists $x \in (a, b)$ with:
   
   {\centering$(f(b) - f(a))g^\prime(x) = (g(b) - g(a))f^\prime(x)$.\retTwo\par}

   {\begin{myIndent} \hThree
      Proof:\\
      Define $h: [a, b] \longrightarrow \mathbb{R}$ by $h(x) = (f(b) - f(a))g(x) - (g(b) - g(a))f(x)$.\\ Then $h(a) = f(b)g(a) - g(b)f(a) = h(b)$.\retTwo

      Notice that $h$ is continuous on $[a, b]$ and differentiable on $(a, b)$ because of\\ propositions 70 and 84. Since $h^\prime(x) = (f(b) - f(a))g^\prime(x) - (g(b) - g(a))f^\prime(x)$,\\ for all $x \in (a, b)$ it now suffices to show that there exists $x \in (a, b)$ with\\ $h^\prime(x) = 0$.\\
      
      Since $h$ is continuous on a compact set $[a, b]$, we know that $h$ attains a\\ maximum value and a minimum value over the interval $[a, b]$.
      \begin{myDindent}
         \begin{itemize}
            \item[Case 1:] If $h$ is constant on $[a, b]$, then $h^\prime(x) = 0$ for all $x \in (a, b)$.\\ [-8pt]
            \item[Case 2:] If there is $t \in (a, b)$ with $h(t) > h(a) = h(b)$, then $h(a)$ and\\ $h(b)$ can't be the max. value that $h$ attains on $[a, b]$. So $h$ has a maximum at some point $x \in (a, b)$. Then by the last theorem,\\ $h^\prime(x) = 0$.
            \item[Case 3:] If there is $t \in (a, b)$ with $h(t) < h(a) = h(b)$, then $h(a)$ and\\ $h(b)$ can't be the min. value that $h$ attains on $[a, b]$. So $h$ has a minimum at some point $x \in (a, b)$. Then by the last theorem,\\ $h^\prime(x) = 0$.
         \end{itemize}
      \end{myDindent}
   \end{myIndent}}

   \newpage

   Proposition \propCount: (Mean Value Theorem)\\
   If $f: [a, b] \longrightarrow \mathbb{R}$ is continuous on $[a, b]$ and differentiable on $(a, b)$, then there is\\ $x \in (a, b)$ with $f(b) - f(a) = (b - a)f^\prime(x)$.\retTwo
   
   {\begin{myIndent} \hThree
      To prove this, apply the previous proposition with $g(x) = x$.\retTwo
   \end{myIndent}}

   Proposition \propCount: Suppose $f (a, b) \longrightarrow \mathbb{R}$ is differentiable. Then:
   \begin{itemize}
      \item If $f^\prime(x) \geq 0$ for all $x \in (a, b)$, then $f$ is monotone increasing.
      \item If $f^\prime(x) \leq 0$ for all $x \in (a, b)$, then $f$ is monotone decreasing.
      \item If $f^\prime(x) = 0$ for all $x \in (a, b)$, then $f$ is constant.\retTwo
   \end{itemize}

   
   \begin{myIndent}\hThree
      Proof:\\
      For all $a < x_1 < x_2 < b$, we know by the mean value theorem that there\\ exists $t \in (x_1, x_2)$ with $f(x_2) - f(x_1) = (x_2 - x_1)f^\prime(t)$. Then since\\ $x_2 - x_1 > 0$, the sign of $f(x_2) - f(x_1)$ depends entirely on $f^\prime(t)$.\retTwo
   \end{myIndent}
\end{myIndent}}

\mySepTwo

\markLecture{4/5/2024}

Even though derivatives are not necessarily continuous, we can show they always\\ satisfy the conclusion of the intermediate value theorem.

{\begin{myIndent}\hTwo
   Proposition \propCount: Suppose $f: [a, b] \rightarrow \mathbb{R}$ is differentiable and $\lambda \in \mathbb{R}$ satisfies\\ that $f^\prime(a) < \lambda < f^\prime(b)$. Then there is $x \in (a, b)$ with $f^\prime(x) = \lambda$.\\
   
   {\begin{myIndent}\hThree
      Proof:\\
      Define $g: [a, b] \rightarrow \mathbb{R}$ by the rule $g(t) = f(t) - \lambda t$. Then $g$ is\\ differentiable with $g^\prime(t) = f^\prime(t) - \lambda$. So, it suffices to find\\ $x \in (a, b)$ with $g^\prime(x) = 0$\retTwo

      Since $g$ is differentiable, we know that $g$ is continuous. Adding in\\ the fact that $[a, b]$ is compact, we know that $g$ achieves a minimum\\ value. So, let $x \in [a, b]$ be such that $g(x)$ is the minimum value of $g$.\retTwo

      Now consider that $f^\prime(a) < \lambda < f^\prime(b) \Longrightarrow g^\prime(a) < 0 < g^\prime(b)$. Since\\ $g^\prime(a) < 0$, there is some $t_1 > a$ near $a$ such that $g(x) \leq g(t_1) < g(a)$.\\ [-11pt]
      {\begin{myIndent}\hFour
         Explanation:\\
         Set $\varepsilon = |g^\prime(a)|$. Then by the definition of limits:
         
         {\centering $\exists \delta > 0 \suchthat \forall t \in (a, a + \delta), \myHS \left|\frac{g(t)- g(a)}{t - a} - g^\prime(a)\right| < \varepsilon$.\retTwo\par}

         Then because $g^\prime(a)$ is negative, we must have that $\frac{g(t)- g(a)}{t - a} < 0$.\\ But as $t - a > 0$, we must have that $g(t)- g(a) < 0$.
         {\begin{myTindent}\begin{myIndent}\teachComment
            This will be a common trick so get used to it.
         \end{myIndent}\end{myTindent}}
      \end{myIndent}}

      \newpage

      Similarly, since $g^\prime(b) > 0$, there is some $t_2 < b$ near $b$ such that\\ $g(x) \leq g(t_2) < g(b)$. Hence, we have shown that $x \neq a$ and $x \neq b$,\\ meaning that $x \in (a, b)$. Then, by applying proposition 86 we know\\ that $g^\prime(x) = 0$.\retTwo


      \begin{myDindent} \teachComment
         We can prove an analogous theorem for when $f^\prime(b) < \lambda < f^\prime(a)$.\retTwo
      \end{myDindent}
   \end{myIndent}}

   \uuline{Corollary}: If $f: [a, b] \longrightarrow \mathbb{R}$ is differentiable, then $f^\prime$ has no simple discontinuities.\\
   
   {\begin{myIndent}\hThree
      Proof:\\
      Assume that $x \in [a, b)$ and $f^\prime(x+)$ exists. Then let $\varepsilon > 0$. By the definition\\ of $f(x+)$:
      
      {\centering$\exists \delta > 0 \suchthat \forall t \in (x, x + \delta),\myHS |f^\prime(t) - f^\prime(x+)| < \sfrac{\varepsilon}{2}$.\retTwo\par}
      
      If $f^\prime(t) = f^\prime(x)$ for all $t \in (x, x+\delta)$, then we automatically have that\\ $f^\prime(x+) = f^\prime(x)$. So assume there exists $t \in (x, x+\delta)$ such that\\ $f^\prime(t) \neq f^\prime(x)$. Then by the previous proposition, there exists $s \in (x, t)$\\ such that $f^\prime(s)$ is between $f^\prime(x)$ and $f^\prime(t)$, and that $|f^\prime(s) - f^\prime(x)| < \sfrac{\varepsilon}{2}$.\\ Finally:

      {\centering $|f^\prime(x) - f^\prime(x+)| \leq |f^\prime(x) - f^\prime(s)| + |f^\prime(s) - f^\prime(x+)| < \sfrac{\varepsilon}{2} + \sfrac{\varepsilon}{2} = \varepsilon$.\retTwo\par}

      So $f^\prime(x)$ must equal $f^\prime(x+)$. Similarly, we can show that if $x \in (a, b]$ and\\ $f^\prime(x-)$ exists, then $f^\prime(x) = f^\prime(x-)$. Thus, it is impossible for $f^\prime$ to have a\\ simple discontinuity.\retTwo

      {\begin{myDindent}\teachComment
         However, we already saw that $f^\prime$ can have discontinuities of the\\ second kind.\retTwo
      \end{myDindent}}
   \end{myIndent}}

   Propositon \propCount: (L'Hôpital's rule)\\
   Suppose $-\infty \leq a \leq b \leq +\infty$, that $f, g: (a, b) \longrightarrow \mathbb{R}$ are differentiable, and that\\ $\forall x \in (a, b), \myHS g^\prime(x) \neq 0$. Then suppose that $\lim\limits_{x\rightarrow a}\frac{f^\prime(x)}{g^\prime(x)} = A \in \mathbb{R} \cup \{-\infty, \infty\}$.\\ If either:
   \begin{itemize}
      \item both $f(x) \rightarrow 0$ and $g(x) \rightarrow 0$ as $x \rightarrow a$
      \item or either $g(x) \rightarrow +\infty$ or $g(x) \rightarrow -\infty$ as $x \rightarrow a$
   \end{itemize}
   then $\lim\limits_{x\rightarrow a}\frac{f(x)}{g(x)}\rightarrow A$.\hspace{11em}{\teachComment (A similar result holds as $x \rightarrow b$.)}\\

   {\begin{myIndent}\hThree
      Proof:\\
      Since $A \in \mathbb{R} \cup \{-\infty, \infty\}$, to show that $\lim\limits_{x\rightarrow a}\frac{f(x)}{g(x)} = A$, it suffices to show:
      \begin{enumerate}
         \item If $A \neq +\infty$, then for every $q \in \mathbb{R}$ with $q > A$, there is $c > a$ with\\ $\forall x \in (a, c),\myHS \frac{f(x)}{g(x)} < q$.
         \item If $A \neq -\infty$, then for every $q \in \mathbb{R}$ with $q < A$, there is $c > a$ with\\ $\forall x \in (a, c),\myHS \frac{f(x)}{g(x)} > q$
      \end{enumerate}

      \newpage

      Let's prove requirement 1. Assume $A \neq +\infty$ and fix $q \in \mathbb{R}$ with $q > A$.\\
      Next pick $r \in \mathbb{R}$ with $A < r < q$. Since $\frac{f^\prime(x)}{g^\prime(x)} \rightarrow A$ as $x \rightarrow a$, there is\\ [-3pt] $c_1 > a$ with $\forall x \in (a, c_1), \myHS \frac{f^\prime(x)}{g^\prime(x)} < r$.\retTwo
      
      Now consider that whenever $a < x < y < c_1$, we have by proposition 87\\ that there exists $t \in (x, y)$ such that:

      {\center $ (f(y) - f(x))g^\prime(t) = (g(y) - g(x))f^\prime(t)$.\retTwo\par}

      By the hypothesis of the theorem, $g^\prime(t)$ can't be zero. Aditionally, because\\ of the mean value theorem, if $g(y) - g(x) = 0$, then there would have to\\ exist $s \in (x, y)$ with $g^\prime(s) = 0$, thus contradicting the hypothesis of the\\ theorem. So, it is safe to rearrange the above expression to get that:

      {\center $\frac{f(y) - f(x)}{g(y) - g(x)} = \frac{f^\prime(t)}{g^\prime(t)} < r$ \retTwo\par}

      
      Case 1: Assume $f(x) \rightarrow 0$ and $g(x) \rightarrow 0$ as $x \rightarrow a$.
      {\begin{myIndent}\hFour
         Then fixing any $y \in (a, c_1)$, we have that $\lim\limits_{x\rightarrow a}\frac{f(y) - f(x)}{g(y) - g(x)} = \frac{f(y)}{g(y)} \leq r < q$.\retTwo
      \end{myIndent}}

      Case 2: Assume $g(x) \rightarrow +\infty$ or $g(x) \rightarrow -\infty$ as $x \rightarrow a$.
      {\begin{myIndent}\hFour
         Then fix any $y \in (a, c_1)$ and pick $c_2 \in (a, c_1)$ such that $\forall x \in (a, c_2)$,\\ $g(x)$ and $g(x) - g(y)$ have the same sign. Then, $\forall x \in (a, c_2)$, we have\\ that $\frac{g(x) - g(y)}{g(x)} > 0$. So:

         {\center $ \frac{f(y) - f(x)}{g(y) - g(x)}\cdot\frac{g(x) - g(y)}{g(x)} < r \cdot \frac{g(x) - g(y)}{g(x)} $ \retTwo\par}

        Note that $\frac{f(y) - f(x)}{g(y) - g(x)}\cdot\frac{g(x) - g(y)}{g(x)} = \frac{f(x) - f(y)}{g(x)} = \frac{f(x)}{g(x)} - \frac{f(y)}{g(x)}$ and\\ $\frac{g(x) - g(y)}{g(x)} = 1 - \frac{g(y)}{g(x)}$. Thus, we can rearrange terms to get that:\\ [-8pt]

         {\center $ \frac{f(x)}{g(x)} < \left(1 - \frac{g(y)}{g(x)}\right)r + \frac{f(y)}{f(x)}$ \retTwo\par}

         Now, $\lim\limits_{x\rightarrow a}\left(\left(1 - \frac{g(y)}{g(x)}\right)r + \frac{f(y)}{f(x)}\right) = (1 - 0)r + 0 = r$. So, there is\\ $c_3 \in (a, c_2)$ such that $\forall x \in (a, c_3), \myHS \left(1 - \frac{g(y)}{g(x)}\right)r + \frac{f(y)}{f(x)} < q$.\retTwo

         Hence, $\forall x \in (a, c_3)$, $\frac{f(x)}{g(x)} < q$.\retTwo
      \end{myIndent}}
      
      Requirement 2 is proved in a similar fashion. $\blacksquare$\retTwo
   \end{myIndent}}
\end{myIndent}}

Let $f$ be a real-valued function with $\domain{f} \subseteq \mathbb{R}$. If $f^\prime$ is defined and is itself\\ differentiable, then the derivative of $f^\prime$ is denoted $f^\pprime$ and called the \udefine{second\\ derivative} of $f$. We similarly define $f^{\ppprime}, f^{(4)}, \ldots, f^{(n)}$.\retTwo

Also, we shall sometimes use $f^{(0)}$ to refer to the original function $f$.

\newpage

\markLecture{4/8/2024}

{\begin{myIndent}\hTwo
   Proposition \propCount: (Taylor's Theorem)\\
   Suppose that $f: [a, b] \longrightarrow \mathbb{R}$, that $f^{(n-1)}$ is continuous on $[a, b]$, and that $f^{(n)}$ is\\ defined on $(a, b)$. Then pick $\alpha \in [a, b]$ and define:

   {\centering $P(t) = \sum\limits_{k=0}^{n-1}\frac{f^{(k)}(\alpha)}{k!}(t - \alpha)^k$\retTwo\par}

   Then for every $\beta \in [a, b] \setminus \{\alpha\}$, there is some $x$ between $\alpha$ and $\beta$ such that\\ $f(\beta) = P(\beta) + \frac{f^{(n)}(x)}{n!}(\beta - \alpha)^n$.\\ [-6pt]

   {\begin{myIndent}\hThree
      Proof:\\
      Set $M = \frac{f(\beta) - P(\beta)}{(\beta - \alpha)^n}$ so that $f(\beta) = P(\beta) + M(\beta - \alpha)^n$. Having done that,\\ [-5pt] our goal is now to find an $x$ between $\alpha$ and $\beta$ such that $\frac{f^{(n)}(x)}{n!} = M$. \retTwo

      Define $g(t) = f(t) - P(t) - M(t - \alpha)^n$. Then, since $P$ is a polynomial\\ of degree $n - 1$, we have that $P^{(n)}(t) = 0$ for all $t$. So:
      
      {\centering $g^{(n)}(t) = f^{(n)}(t) - Mn!$\retTwo\par}

      Thus, it suffices to find an $x$ between $\alpha$ and $\beta$ such that $g^{(n)}(x) = 0$.\retTwo

      Importantly, $P$ is the unique polynomial of degree $n - 1$ satisfying for all\\ $0 \leq k \in \leq k-1$ that $P^{(k)}(\alpha) = f^{(k)}(\alpha)$. Thus, for all $0 \leq k \in \leq k-1$, we\\ have that:
      
      {\centering$g^{(k)}(\alpha) = f^{(k)}(\alpha) - P^{(k)}(\alpha) - M\frac{n!}{(n-k)!}(\alpha - \alpha)^{n-k} = 0$.\retTwo\par}
      
      At the same time, for all $0 \leq k \leq n-1$, we know that $g^{(k)}$ is continuous\\ on $[\alpha, \beta]$ and differentiable on $(\alpha, \beta)$. So, we shall proceed by repeatedly\\ applying the mean value theorem.
      \begin{myIndent}
         \begin{itemize}
            \item $g(\beta) = 0$ and $g(\alpha) = 0$. So, there is $x_1$ between $\alpha$ and $\beta$\\ with $g^\prime(x_1) = 0$.
            \item $g^\prime(x_1) = 0$ and $g^\prime(\alpha) = 0$. So, there is $x_2$ between $\alpha$ and $x_1$\\ with $g^\pprime(x_2) = 0$.
            
            {\centering\fontsize{18}{0}\selectfont $\vdots\phantom{aaaaaaa}$ \retTwo\par}
         \end{itemize}
      \end{myIndent}

      Eventually, you will get an $x_n$ between $\alpha$ and $x_{n-1}$ with $g^{(n)}(x_n) = 0$. $\blacksquare$\retTwo

      \begin{myTindent}\teachComment
         Note that this can be interpretted as a higher order analog\\ of the mean value theorem. In fact, if $n = 1$ then this is\\ just the mean value theorem.
      \end{myTindent}
   \end{myIndent}}
\end{myIndent}}

\newpage

The limit definition of the derivative still makes sense and can be applied to\\ situations where $f$ is a $\mathbb{C}$-valued or $\mathbb{R}^k$-valued function. Although, because\\ this class is called "real" analysis, we shall always require that $\domain{f} \subseteq \mathbb{R}$.
{\begin{myTindent}\begin{myTindent}\teachComment
   (We will talk in 140C about when $\domain{f} \subseteq \mathbb{R}^k$)\retTwo
\end{myTindent}\end{myTindent}}

If $f$ is a $\mathbb{C}$-valued function, then we can write that $f = f_1 + if_2$ where $f_1$ and\\ $f_2$ are real-valued. Then, $f$ is differentiable if and only if $f_1$ and $f_2$ are\\ differentiable. Also, $f^\prime(x) = f_1^\prime(x) + if_2^\prime(x)$.\retTwo

{\begin{myIndent}\exTwo\color{Purple}
   Assume $f^\prime(x)$ exists. Then:
   {\begin{itemize}\exP\color{RedViolet}
      \item $f_1^\prime(x) = \lim\limits_{t\rightarrow x}{\frac{f_1(t) - f_1(x)}{t - x}} = \lim\limits_{t\rightarrow x}{\frac{\rea{f(t)} - \rea{f(x)}}{t - x}} = \lim\limits_{t\rightarrow x}\left(\mathrm{Re}\hspace{-0.24em}\left(\frac{f(t) - f(x)}{t - x}\right)\right)$
      \item $f_2^\prime(x) = \lim\limits_{t\rightarrow x}{\frac{f_2(t) - f_2(x)}{t - x}} = \lim\limits_{t\rightarrow x}{\frac{\ima{f(t)} - \ima{f(x)}}{t - x}} = \lim\limits_{t\rightarrow x}\left(\mathrm{Im}\hspace{-0.24em}\left(\frac{f(t) - f(x)}{t - x}\right)\right)$\\
   \end{itemize}}

   Now $\lim\limits_{t\rightarrow x}\frac{f(t) - f(x)}{t - x} = f^\prime(x)$, $\lim\limits_{t\rightarrow f^\prime(x)}\hspace{-0.6em}\rea{t} = \rea{f^\prime(x)}$, and $\hspace{-0.4em}\lim\limits_{t\rightarrow f^\prime(x)}\hspace{-0.6em}\ima{t} = \ima{f^\prime(x)}$.\\ Thus, $\lim\limits_{t\rightarrow x}\left(\mathrm{Im}\hspace{-0.24em}\left(\frac{f(t) - f(x)}{t - x}\right)\right) = \rea{f^\prime(x)}$ and $\lim\limits_{t\rightarrow x}\left(\mathrm{Im}\hspace{-0.24em}\left(\frac{f(t) - f(x)}{t - x}\right)\right) = \ima{f^\prime(x)}$.
   
   
   {\begin{myIndent}\begin{myTindent}\exP
      You can double check this by considering sequences $(t_n)$ which converge to $x$ but never equal $x$. \\ [14pt]
   \end{myTindent}\end{myIndent}}

   Meanwhile, if $f_1$ and $f_2$ are differentiable, then:\\ [-8pt]

   {\centering\exP\color{RedViolet} 
   \begin{tabular}{l}
      $f^\prime(x) = \lim\limits_{t\rightarrow x}\left(\frac{f_1(t) + if_2(t) - f_1(x) - if_2(x)}{t - x}\right)$\\ [3pt]
      $\phantom{f^\prime(x) } = \lim\limits_{t\rightarrow x}\left(\frac{f_1(t) - f_1(x)}{t - x} + i\frac{f_2(t) - f_2(x)}{t-x}\right) =   f_1^\prime(x) + if_2^\prime(x)$.
   \end{tabular}
   \retTwo\par}
\end{myIndent}}

Similarly, if $\mVec{f}$ is $\mathbb{R}^k$-valued, then we can write $\mVec{f} = (f_1, f_2, \ldots, f_k)$ where\\ $f_1, f_2, \ldots, f_k$ are real-valued. Then $\mVec{f}$ is differentiable if and only if $f_1, f_2, \ldots, f_k$\\ are all differentiable. Also, $\mVec{f}^\prime(x) = (f_1^\prime(x), f_2^\prime(x), \ldots, f^\prime_k(x))$.






% ~~~~~~~~~~~~~~~~~~~~~~~~~~~~~~~~~~~~~~~~~~~~~~~~~~~~~~~~~~~~~~~~~~~~~~ %
\newpage
{\huge \color{Black} \textbf{A List of How The Proposition Numbering in my Notes Lines up With Our Textbook:} \retTwo}
\exOne

\begin{allowTableDashes}
   \begin{tabular}{ c|c||c|c }
      Proposition Number & Label in Textbook & Proposition Number & Label in Textbook \\ \hline
      
      \myVS 83 & 5.2 & 84 & 5.3 \\ \hdashline[10pt/3pt]
      \myVS 85 & 5.5 & 86 & 5.8 \\ \hdashline[10pt/3pt]
      \myVS 87 & 5.9  & 88 & 5.10 \\ \hdashline[10pt/3pt]
      \myVS 89 & 5.11 & 90 & 5.12 \\ \hdashline[10pt/3pt]
      \myVS 91 & 5.13 & 92 & 5.15 \\ \hdashline[10pt/3pt]
      \myVS 93 &  & 94 &  \\ \hdashline[10pt/3pt]
      \myVS 95 &  & 96 &  \\ \hdashline[10pt/3pt]
      \myVS 97 &  & 98 &  \\ \hdashline[10pt/3pt]
      \myVS 99 &  & 100 &  \\ \hdashline[10pt/3pt]
      \myVS 101 &  & 102 &  \\ \hdashline[10pt/3pt]
      \myVS 103 &  & 104 &  \\ \hdashline[10pt/3pt]
   \end{tabular}

\end{allowTableDashes}

\retTwo

Our textbook is \textit{Principles of Mathematical Analysis} by Walter Rudin.

\newpage
\pracOne

\markHW{Homework 1:}

\textbf{Exercise 5.2}: Let $f: (a, b) \longrightarrow \mathbb{R}$ be differentiable with $f^\prime(x) > 0$. Then $f$ is strictly\\ increasing.

{\begin{myIndent}\pracTwo
   For all $a < x_1 < x_2 < b$, we know by the mean value theorem that there exists\\ $t \in (x_1, x_2)$ with $f(x_2) - f(x_1) = (x_2 - x_1)f^\prime(t)$. Since $(x_2 - x_1)$ and $f^\prime(t)$ are\\ positive, we thus have that $f(x_2) - f(x_1) > 0$.\retTwo
\end{myIndent}}

As a consequence of $f$ being strictly increasing, we know $f$ is injective. Thus if we\\ restrict the codomain of $f$ to its image, then $f$ is bijective, meaning there exists a\\ function $g = f^{-1}$ such that $(g \circ f)(x) = x = (f \circ g)(x)$. Now we show that\\ for all $y \in \domain{g},\myHS y \in \domain{g}^\prime$ and $g(z) \rightarrow g(y)$ as $z \rightarrow y$.

{\begin{myIndent}\pracTwo
   Let $y \in \domain{g}$. Then there exists $x \in \domain{f}$ such that $f(x) = y$ and $g(y) = x$. But,\\ since $\domain{f}$ is an open set, we know that $x$ is an interior point of $\domain{f}$. Hence, there\\ [-2pt] exists $r$ such that $[x - r, x + r] \subseteq \domain{f}$. Let $\widehat{f}$ be the restriction of $f$ whose domain is\\ $[x - r, x + r]$ and whose codomain is the image of $[x - r, x + r]$ with respect to $f$.\retTwo
   
   Because $f$ is differentiable, we know $f$ is continuous. Then, note that\\ $\forall x_0 \in \mathrm{dom}\hspace{-0.2em}\left(\widehat{f}\hspace{0.09em}\right) \cap \mathrm{dom}\hspace{-0.2em}\left(\widehat{f}\hspace{0.09em}\right)\hspace{-0.2em}{\displaystyle\vphantom{a}^{\prime}}, \myHS \lim\limits_{t \rightarrow x_0}\widehat{f}(t) = \lim\limits_{t \rightarrow x_0}f(t) = f(x_0) = f(x_0) = \widehat{f}(x_0)$.\\ [-5pt] So, $\widehat{f}$ is a continuous function over its domain. Also, note that $\widehat{f}$ is still bijective.\retTwo

   Meanwhile $[x - r, x + r] = \mathrm{dom}\hspace{-0.2em}\left(\widehat{f}\hspace{0.09em}\right)$ is compact and connected. Firstly, this means\\ [-2pt] that $\widehat{f}^{-1}$ is continuous. Secondly, this tells us that the image of $[x - r, x + r]$ is\\ [2pt]     connected. Because $f$ and thus $\widehat{f}$ is strictly increasing, we know that:\\ [2pt] $\widehat{f}(x - r) < \widehat{f}(x) = f(x) = y < \widehat{f}(x + r)$. So, $[\widehat{f}(x - r), \widehat{f}(x + r)]$ is a subset of the domain of $\widehat{f}^{-1}$ and $y$ is an interior point of that subset of the domain.\\ [6pt]

   $[\widehat{f}(x - r), \widehat{f}(x + r)]$ is perfect, meaning $y$ is a limit point of $[\widehat{f}(x - r), \widehat{f}(x + r)]$.\\ In turn, this means $y$ is a limit point of $\mathrm{dom}\hspace{-0.2em}\left(\widehat{f}\hspace{0.09em}\right)$ and $\domain{g}$ because:
   
   {\centering $[\widehat{f}(x - r), \widehat{f}(x + r)] \subseteq \mathrm{dom}\hspace{-0.2em}\left(\widehat{f}\hspace{0.09em}\right) \subseteq \domain{g}$.\retTwo\par}
   
   Then as $\widehat{f}^{-1}$ is continuous, we know that $\lim\limits_{z\rightarrow y}\widehat{f}^{-1}(z) = \widehat{f}^{-1}(y)$. Additionally, because\\ [-3pt] $y$ is an interior point of ${[\widehat{f}(x - r), \widehat{f}(x + r)]\text{,}}$ there exists $R > 0$ so that\\ [4pt] $B_R(y) \subset [\widehat{f}(x - r), \widehat{f}(x + r)]$.\\ [6pt]
   
   Finally, we show that $g(z) \rightarrow g(y)$ as $z \rightarrow y$.\\ Let $\varepsilon > 0$. Then there exists $\delta > 0$ such that ${|z - y| < \delta \Longrightarrow |\widehat{f}^{-1}(z) - \widehat{f}^{-1}(y)| < \varepsilon\text{.}}$\\ Set $\delta^\prime = \min(\delta, R)$. Then $|z - y| < \delta^\prime \Longrightarrow |g(z) - g(y)| = |\widehat{f}^{-1}(z) - \widehat{f}^{-1}(y)| \leq \varepsilon$.
\end{myIndent}}

\newpage

Finally, we show that $g$ is differentiable. 
{\begin{myIndent}\pracTwo
   Note that $z = f(g(z))$ and $y = f(g(y))$. So, we can write that:
   
   {\centering $\lim\limits_{z \rightarrow y}{\frac{g(z) - g(y)}{z - y}} = \lim\limits_{z \rightarrow y}{\frac{g(z) - g(y)}{f(g(z)) - f(g(y))}}$.\retTwo\par}

   Now let $(z_n)$ be any sequence in the domain of $g$ converging to $y$ such that $z_n \neq y$\\ for any $n$. Since $\lim\limits_{z \rightarrow y}g(z) = g(y)$ for all $y \in \domain{g}$, we know that $g(z_n) \rightarrow g(y)$.\retTwo

   Meanwhile, note that since $f^\prime(x) \neq 0$ for all $x \in \domain{f}$, we can evaluate that:
   
   {\centering $\lim\limits_{t \rightarrow g(y)}{\frac{t - g(y)}{f(t) - f(g(y))}} = \frac{1}{f^\prime(g(y))}$.\retTwo\par}

   So, given any sequence $(t_n)$ in the domain of $f$ converging to $g(y)$ such that $t_n \neq g(y)$\\ for any $n$, we have that $\frac{t_n - g(y)}{f(t_n) - f(g(y))} \rightarrow \frac{1}{f^\prime(g(y))}$. Since $g$ is injective and $z_n \neq y$ for any $n$,\\ $(g(z_n))$ is one such sequence. Hence:
   
   {\centering $\frac{g(z_n) - g(y)}{f(g(z_n)) - f(g(y))} \rightarrow \frac{1}{f^\prime(g(y))}$. \retTwo\par}

   Since this is true for all relevant $(z_n)$, we conclude that:
   
   {\centering $\lim\limits_{z \rightarrow y}{\frac{g(z) - g(y)}{z - y}} = \lim\limits_{z \rightarrow y}{\frac{g(z) - g(y)}{f(g(z)) - f(g(y))}} = \frac{1}{f^\prime(g(y))}$\retTwo\par}

   Thus, $g^\prime(y)$ exists and equals $\frac{1}{f^\prime(g(y))}$.\retTwo
   
   \begin{myDindent}\begin{myDindent}
      \begin{myClosureOne}{3.5}
         \\ [-20pt] Plugging in $y = f(x)$, we get that $g^\prime(f(x)) = \frac{1}{f^\prime(x)}$.\\ [-8pt]
      \end{myClosureOne}\retTwo
   \end{myDindent}\end{myDindent}
\end{myIndent}}

\mySepTwo[Black]

\textbf{Exercise 5.4}: If $C_0 + \frac{C_1}{2} + \ldots + \frac{C_{n-1}}{n} + \frac{C_n}{n+1} = 0$ and $C_0, C_1, \ldots, C_n \in \mathbb{R}$, then we\\  shall prove that the equation $C_0 + C_1x + \ldots + C_{n-1}x^{n-1} + C_n x^n= 0$ has at least\\ [2pt] one real root between $0$ and $1$.

{\begin{myIndent}\pracTwo
   Define the functions:
   \begin{myIndent}
      $f(x) = C_0 + C_1x + \ldots + C_{n-1}x^{n-1} + C_n x^n$\\
      $F(x) = C_0x + \frac{C_1}{2}x + \ldots + \frac{C_{n-1}}{n}x^n + \frac{C_n}{n+1}x^{n+1}$\retTwo
   \end{myIndent}

   Note that $F(0) = 0$ and $F(1) = C_0 + \frac{C_1}{2} + \ldots + \frac{C_{n-1}}{n} + \frac{C_n}{n+1} = 0$ At the same time,\\ $F$ is differentiable with $F^\prime(x) = f(x)$. Therefore, by the mean value theorem there\\ [1pt] exists $t \in (0, 1)$ such that $0 = F^\prime(t) = f(t)$. Thus, that $t$ is a real root between $0$ and\\ [1pt] $1$ for the equation $C_0 + C_1x + \ldots + C_{n-1}x^{n-1} + C_n x^n= 0$.
\end{myIndent}}
\newpage
\textbf{Exercise 5.6}: Suppose the following conditions on $f$:\\ [-22pt]
\begin{itemize}
   \item[(A)] $f$ is continuous for $x \geq 0$\\[-20pt]
   \item[(B)] $f^\prime$ exists for $x > 0$\\[-20pt]
   \item[(C)] $f(0) = 0$\\[-20pt]
   \item[(D)] $f^\prime$ is monotonically increasing\\[-20pt]  
\end{itemize}
Putting $g(x) = \frac{f(x)}{x}$ for $x > 0$, we shall prove that $g$ is monotonically increasing.

{\begin{myIndent}\pracTwo
   Firstly, given any $x > 0$, because of conditions A and B, we can apply the mean value\\ theorem to say that there exists $t \in (0, x)$ such that $f(x) - f(0) = xf^\prime(t)$. Because\\ of condition C, this then simplifies to $f(x) = xf^\prime(t)$. So:
   \begin{center}
      for all $x > 0$, there exists $0 < t < x$ such that $\frac{f(x)}{x} = f^\prime(t)$.\retTwo
   \end{center}

   Meanwhile, because of condition B and the quotient rule, $g$ is differentiable when\\ $x > 0$ with $g^\prime(x) = \frac{f^\prime(x)x - f(x)}{x^2}$. So, consider any $b > a > 0$. By the mean value\\ theorem, there exists $s \in (a, b)$ with $g(b) - g(a) = (b - a)g^\prime(s)$. Obviously, $b - a$\\ [1pt] is positive. Additionally, consider that:
   
   {\center$g^\prime(s) = \frac{f^\prime(s)s - f(s)}{s^2} = \frac{1}{s}\left(f^\prime(s) - \frac{f(s)}{s}\right)$.\retTwo\par}

   Pick $t > 0$ such that $t < s$ and $\frac{f(s)}{s} = f^\prime(t)$. Then $g^\prime(s) = \frac{1}{s}\left(f^\prime(s) - f^\prime(t)\right)$. But,\\ because of condition D, we know that $f^\prime(s) \geq f^\prime(t)$. Hence, $g^\prime(s) \geq 0$.\retTwo

   Therefore, $g(b) - g(a) \geq 0$, meaning $g$ is monotonically increasing.\retTwo
\end{myIndent}}

\mySepTwo[Black]

\textbf{Exercise 5.8}: Consider any real-valued function $f$ which is differentiable on $[a, b]$ with\\ $f^\prime$ being continuous on $[a, b]$. Then we shall prove that:

{\centering $\forall \varepsilon > 0,\myHS \exists \delta > 0\suchthat\forall x, t \in [a, b],\myHS 0 < |t - x| < \delta \Longrightarrow \left|\frac{f(t) - f(x)}{t - x} - f^\prime(x)\right| < \varepsilon$ \retTwo\par}

{\begin{myIndent}\pracTwo
   Because $f^\prime$ is continuous over a compact domain, we know that by theorem 4.19\\ (proposition 76), $f^\prime$ is uniformly continuous. Thus, let $\varepsilon > 0$ and pick $\delta > 0$ such\\ that for all $x, y \in [a, b]$, we have that $|x - y| < \delta \Longrightarrow |f^\prime(x) - f^\prime(y)| < \varepsilon$.\retTwo


   Since $f$ is differentiable on $[a, b]$, we know by the mean value theorem that for any\\ distinct $x$ and $t$ in $[a, b]$, there exists $s$ between $a$ and $b$ such that:
   
   {\center$\frac{f(t) - f(x)}{t - x} = f^\prime(s)$.\retTwo\par}

   Hence, $\left|\frac{f(t) - f(x)}{t - x} - f^\prime(x)\right| = \left| f^\prime(s) - f^\prime(x)\right|$. And since $|s - x| < |t - x|$, we\\ know that if $ 0 < |t - x| < \delta$, then $|f^\prime(s) - f^\prime(x)| < \varepsilon$.\\
\end{myIndent}}

An analogous theorem holds for any vector-valued function $\mVec{f}: [a, b] \longrightarrow \mathbb{R}^k$\\ [-2pt] that is differentiable on $[a, b]$ with $\mVec{f}^\prime$ being continuous on $[a, b]$.\\ [-12pt]

\begin{myIndent}\pracTwo
   Let $\mVec{f}(x) = (f_1(x), f_2(x), \ldots, f_k(x))$. Since $\mVec{f}$ is differentiable on $[a, b]$ and $\mVec{f}^\prime$ is\\ continuous on $[a, b]$, we have for each $i \in \{1, \ldots, k\}$ that $f_i$ is differentiable on $[a, b]$\\ and $f_i^\prime$ is continuous on $[a, b]$.

   \newpage

   Thus, given any $\varepsilon > 0$, we already proved that for each $i \in \{1, \ldots, k\}$, there exists\\ $\delta_i > 0$ such that $\forall t, x \in [a, b], \myHS |t - x| < \delta_i \Longrightarrow \left|\frac{f_i(t) - f_i(x)}{t - x} - f_i^\prime(x)\right| < \frac{1}{\sqrt{k}}\cdot \varepsilon$.\\ Then setting $\delta = \min(\delta_1, \ldots, \delta_k)$, we have that if $0 < |t - x| < \delta$, then:
   \begin{center}
      \begin{tabular}{l}
         $\left\|\frac{\mVec{f}(t) - \mVec{f}(x)}{t - x} - \mVec{f}^\prime(x)\right\| = \left(\left(\left|\frac{f_1(t) - f_1(x)}{t - x} - f_1^\prime(x)\right|\right)^2 + \ldots + \left(\left|\frac{f_k(t) - f_k(x)}{t - x} - f_k^\prime(x)\right|\right)^2\right)^{\frac{1}{2}}$\\

         $\phantom{\left\|\frac{\mVec{f}(t) - \mVec{f}(x)}{t - x} - \mVec{f}^\prime(x)\right\|} < \left( \left(\frac{1}{\sqrt{k}} \cdot \varepsilon\right)^2 + \ldots + \left(\frac{1}{\sqrt{k}} \cdot \varepsilon\right)^2\right)^{\frac{1}{2}} = \sqrt{k \left(\frac{1}{k} \cdot \varepsilon^2\right)} = \varepsilon$
      \end{tabular}\retTwo
   \end{center}
\end{myIndent}

\mySepTwo[Black]

\textbf{Exercise 5.9}: Let $x_0 \in (a, b)$ and $f: (a, b) \longrightarrow \mathbb{R}$ be continuous at $x_0$. If $f^\prime(x)$ exists for\\ all $x \in (a, b) \setminus \{x_0\}$ and $\lim\limits_{t\rightarrow x_0}f^\prime(t) = L$, then $f^\prime(x_0) = L$.\\ [-6pt]

{\begin{myIndent}\pracTwo
   Since $f$ is continuous at $x_0$ and $x_0$ is a limit point of $(a, b)$,  we know that $f(x_0)$ exists\\ [5pt] and that $\lim\limits_{t \rightarrow x_0}f(t) = f(x_0)$. So, define $g(x) = f(x) - f(x_0)$. Then $g^\prime(x) = f^\prime(x)$\\ [-1pt] and $\lim\limits_{t\rightarrow x_0}g(t) = 0$. Additionally, define $h(x) = x - x_0$. Then $h^\prime(x) = 1$ and\\ [-1pt] $\lim\limits_{t\rightarrow x_0}h(t) = 0$.\retTwo

   Importantly, both $g$ and $h$ are differentiable everywhere on $(a, b)\setminus \{x_0\}$. Also,\\ [3pt] $h^\prime(t) \neq 0$ for all $t \in (a, b)$. Thus, we can apply L'hôpital's rule to get that:

   {\center $ \lim\limits_{t\rightarrow x_0}\frac{f(t) - f(x_0)}{t - x_0} = \lim\limits_{t\rightarrow x_0}\frac{g(t)}{h(t)} = \lim\limits_{t\rightarrow x_0}\frac{g^\prime(t)}{h^\prime(t)} = \lim\limits_{t\rightarrow x_0}f^\prime(t) = L$ \retTwo\par}

   Hence $f^\prime(x_0)$ exists and equals $L$.
   
   \begin{myTindent}\begin{myTindent}
      To answer what's actually asked in the book,\\ set $a = -\infty$, $b = +\infty$, $x_0 = 0$ and $L = 3$.\retTwo
   \end{myTindent}\end{myTindent}
\end{myIndent}}

\mySepTwo[Black]

\textbf{Exercise 5.17}: Suppose $f$ is a real, three times differentiable  function on $[-1, 1]$ such\\ that $f(-1) = 0$, $f(0) = 0$, $f(1) = 1$, and $f^\prime(0) = 0$. Then $f^\ppprime(x) \geq 3$ for some\\ $x \in (-1, 1)$.\\ [-6pt]

{\begin{myIndent}\pracTwo
   Since $f$ is three times differentiable on $[-1, 1]$, we know that $f^\pprime$ is continuous on\\ $[-1, 1]$ and that $f^\ppprime(t)$ exists for every $t \in (-1, 1)$. So define:

   {\center $P(t) = f(0) + f^\prime(0)t + \frac{f^\pprime(0)}{2}t^2 = \frac{f^\pprime(0)}{2}t^2$ \retTwo\par}

   Then by Taylor's theorem, we know that there exists $s \in (0, 1)$ such\\ [2pt] that $f(1) = P(1) + \frac{f^\ppprime(s)}{6}x^3 = \frac{f^\pprime(0)}{2} + \frac{f^\ppprime(s)}{6}$. Similarly, we know that\\ [2pt] there exists $t \in (-1, 0)$ such that $f(-1) = \frac{f^\pprime(0)}{2} - \frac{f^\ppprime(t)}{6}$.\retTwo

   Thus, $\frac{f^\ppprime(s)}{6} + \frac{f^\ppprime(t)}{6} = f(1) - f(-1) = 1$, which in turn means that\\ [2pt] $f^\ppprime(s) + f^\ppprime(t) = 6$. If both $f^\ppprime(s)$ and $f^\ppprime(t)$ are less than $3$, then this\\ [2pt] is impossible. So, either $s$ or $t$ must be greater than or equal to $3$.
\end{myIndent}}

\newpage

\textbf{Exercise 5.26}: Suppose $f$ is differentiable on $[a, b]$, $f(a) = 0$, and there is a real number\\ $A$ such that $|f^\prime(x)| \leq A|f(x)|$ for $x \in [a, b]$. Then $f(x) = 0$ for all $x \in [a, b]$.\\ [-8pt]

{\begin{myIndent}\pracTwo
   To start off, note that if $A < 0$, then we automatically have that $f^\prime(x) = f(x) = 0$ for\\ all $x \in [a, b]$. Meanwhile, if $A = 0$, then $f^\prime(x) = 0$ for all $x \in [a, b]$, thus forcing $f$ to\\ be a constant function. Then, as $f(a) = 0$, we have that $f(x) = f(a) = 0$ for all\\ $x \in [a, b]$. \retTwo

   Therefore, we now assume $A > 0$ and observe the following:
   \begin{myIndent}
      Assume $\gamma \in [a, b)$ and $f(\gamma) = 0$. Then let $x_0 \in [\gamma, b]$ and set $M = \hspace{-0.45em}\sup\limits_{\gamma\leq x\leq x_0}\hspace{-0.4em}|f(x)|$.\\ 

      Then for any $x \in (\gamma, x_0]$, we know by the mean value theorem that there\\ exists $t \in (\gamma, x)$ such that $f(x) - f(\gamma) = (x-\gamma)f^\prime(t)$. Since $f(\gamma) = 0$\\ and $x > \gamma$, we thus know that $|f(x)| = (x - \gamma)|f^\prime(t)|$. Hence:

      {\center $|f(x)| = (x - \gamma)|f^\prime(t)| \leq (x - \gamma)A|f(t)| \leq A(x - \gamma)M \leq A(x_0 - \gamma)M$ \retTwo\par}

      Now importantly, since $f$ is continuous on $[\gamma, x_0]$, and $g(x) = |x|$ is continuous\\ on all of $\mathbb{R}$, we know that $(g\circ f)(x) = |f(x)|$ is continuous on $[\gamma, x_0]$. That\\ combined with the fact that $[\gamma, x_0]$ is compact means that we can fix $x \in [\gamma, x_0]$\\ such that $|f(x)| = M$. Then:
      \begin{itemize}
         \item[$\circ$] If $x = \gamma$, then $M = |f(\gamma)| = 0$.\\ [-14pt]
         \item[$\circ$] If $x \neq \gamma$, then $M = |f(x)| \leq A(x_0 - \gamma)M$. Crucially, if $\gamma < x_0 < \gamma + \frac{1}{A}$\\ [1pt] then $0 < A(x_0 - \gamma) < 1$. Therefore, the only way for $M \leq A(x_0 - \gamma)M$ is\\ [2pt] if $M = 0$.\retTwo
      \end{itemize}

      Thus, for $x_0 \in [\gamma, \gamma + \frac{1}{A}) \cap [\gamma, b]$, we have that $\hspace{-0.45em}\sup\limits_{\gamma\leq x\leq x_0}\hspace{-0.4em}|f(x)| = 0$.\retTwo

      In other words, $f(x) = 0$ for all $x \in [\gamma, \gamma + \frac{1}{A}) \cap [\gamma, b]$.\retTwo
   \end{myIndent}

   Still assuming $A > 0$, we have that $0 < \frac{1}{2A} < \frac{1}{A}$. So for any $\gamma \in [a, b]$, we know\\ [2pt] that $[\gamma, \gamma + \frac{1}{2A}) \cap [\gamma, b] \subseteq [\gamma, \gamma + \frac{1}{A}) \cap [\gamma, b]$. Hence, we now proceed by the\\ [2pt] following inductive process:

   \begin{myIndent}
      Start with $\gamma_1 = a$.

      Now do this until told to stop.
      \begin{myIndent}
         If $\gamma_i = b$, then stop. Otherwise, use the above reasoning to show\\ that $f(x) = 0$ for all $x \in [\gamma_i,\hspace{0.1em} \min(\gamma_i + \frac{1}{2A},\hspace{0.1em} b)]$. Then set\\ $\gamma_{i+1} = \min(\gamma_i + \frac{1}{2A},\hspace{0.1em} b)$ and repeat these steps with $\gamma_{i+1}$.\retTwo
      \end{myIndent}
   \end{myIndent}

   This algorithm will terminate in $\left\lceil \frac{b - a}{\frac{1}{A}}\right\rceil$ iterations, thus showing that $f(x) = 0$\\ for all $x \in [a, b]$.
   \retTwo
\end{myIndent}}


\end{document}
