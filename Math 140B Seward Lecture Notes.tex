% Note for any github stalkers. I am currently in the process
% of learning LaTeX. I don't know what I'm doing yet. Sorry
% if my code absolutely sucks.

\documentclass{book}

\usepackage{fontspec} % used to import Calibri
\usepackage{anyfontsize} % used to adjust font size

% needed for inch and other length measurements
% to be recognized
\usepackage{calc}

% for colors and text effects as is hopefully obvious
\usepackage[dvipsnames]{xcolor}
\usepackage{soul}

% control over margins
\usepackage[margin=1in]{geometry}
\usepackage[strict]{changepage}

\usepackage{mathtools}
\usepackage{amsfonts}
\usepackage{bm}

\usepackage[scr=rsfso, scrscaled=.96]{mathalpha}

\usepackage{amssymb} % originally imported to get the proof square
\usepackage{xfrac}
\usepackage[overcommands]{overarrows} % Get my preferred vector arrows...
\usepackage{relsize}

% Just am using this to get a dashed line in a table...
% Also you apparently want this to be inactive if you aren't
% using it because it slows compilation.
\usepackage{arydshln} \ADLinactivate 
\newenvironment{allowTableDashes}{\ADLactivate}{\ADLinactivate}

\usepackage{graphicx}
\graphicspath{{./140B_images/}}

\usepackage{tikz}
   \usetikzlibrary{arrows.meta}
   \usetikzlibrary{graphs, graphs.standard}

\usepackage{quiver} %commutative diagrams


\newfontfamily{\calibri}{Calibri}
\setlength{\parindent}{0pt}
\definecolor{RawerSienna}{HTML}{945D27}

% ~~~~~~~~~~~~~~~~~~~~~~~~~~~~~~~~~~~~~~~~~~~~~~~~~~
%Arrow Commands:

% Thank you Bernard, gernot, and Sigur who I copied this from:
% https://tex.stackexchange.com/questions/364096/command-for-longhookrightarrow
\newcommand{\hooklongrightarrow}{\lhook\joinrel\longrightarrow}
\newcommand{\hooklongleftarrow}{\longleftarrow\joinrel\rhook}
\newcommand{\hookxlongrightarrow}[2][]{\lhook\joinrel\xrightarrow[#1]{#2}}
\newcommand{\hookxlongleftarrow}[2][]{\xleftarrow[#1]{#2}\joinrel\rhook}

% Thank you egreg who I copied from:
% https://tex.stackexchange.com/questions/260554/two-headed-version-of-xrightarrow
\newcommand{\longrightarrowdbl}{\longrightarrow\mathrel{\mkern-14mu}\rightarrow}
\newcommand{\longleftarrowdbl}{\leftarrow\mathrel{\mkern-14mu}\longleftarrow}

\newcommand{\xrightarrowdbl}[2][]{%
  \xrightarrow[#1]{#2}\mathrel{\mkern-14mu}\rightarrow
}
\newcommand{\xleftarrowdbl}[2][]{%
  \leftarrow\mathrel{\mkern-14mu}\xleftarrow[#1]{#2}
}

% ~~~~~~~~~~~~~~~~~~~~~~~~~~~~~~~~~~~~~~~~~~~~~~~~~~

\newcommand{\hOne}{%
   \color{Black}%
   \fontsize{14}{16}\selectfont%
}
\newcommand{\hTwo}{%
   \color{MidnightBlue}%
   \fontsize{13}{15}\selectfont%
}
\newcommand{\hThree}{%
   \color{PineGreen!85!Orange}
   \fontsize{13}{15}\selectfont%
}
\newcommand{\hFour}{%
   \color{Cerulean}
   \fontsize{12}{14}\selectfont%
}
\newcommand{\myComment}{%
   \color{RawerSienna}%
   \fontsize{12}{14}\selectfont%
}
\newcommand{\pracOne}{
   \color{BrickRed}%
   \fontsize{13}{15}\selectfont%
}
\newcommand{\pracTwo}{
   \color{Orange}%
   \fontsize{12}{14}\selectfont%
}
\newcommand{\teachComment}{
   \color{Orange}%
   \fontsize{12}{14}\selectfont%
}
\newcommand{\exOne}{%
   \color{Purple}%
   \fontsize{14}{16}\selectfont%
}
\newcommand{\exTwo}{%
   \color{RedViolet}%
   \fontsize{13}{15}\selectfont%
}
\newcommand{\exP}{%
   \color{VioletRed}%
   \fontsize{12}{14}\selectfont%
}
% ~~~~~~~~~~~~~~~~~~~~~~~~~~~~~~~~~~~~~~~~~~~~~~~~

\newcommand{\cyPen}[1]{{\vphantom{.}\color{Cerulean}#1}}
\newcommand{\pinkPen}[1]{{\vphantom{.}\color{VioletRed}#1}}

\newenvironment{myIndent}{%
   \begin{adjustwidth}{2.5em}{0em}%
}{%
   \end{adjustwidth}%
}

\newenvironment{myDindent}{%
   \begin{adjustwidth}{5em}{0em}%
}{%
   \end{adjustwidth}%
}

\newenvironment{myTindent}{%
   \begin{adjustwidth}{7.5em}{0em}%
}{%
   \end{adjustwidth}%
}

\newenvironment{myConstrict}{%
   \begin{adjustwidth}{2.5em}{2.5em}%
}{%
   \end{adjustwidth}%
}

\newcommand{\udefine}[1]{{%
   \setulcolor{Red}%
   \setul{0.14em}{0.07em}%
   \ul{#1}%
}}

\newcommand{\uuline}[2][.]{%
{\vphantom{a}\color{#1}%
\rlap{\rule[-0.18em]{\widthof{#2}}{0.06em}}%
\rlap{\rule[-0.32em]{\widthof{#2}}{0.06em}}}%
#2}

\newcommand*{\markDate}[1]{%
   {\huge \color{Black} \textbf{#1} \newline}%
}

\newcommand{\pprime}{{\prime\prime}}
\newcommand{\suchthat}{ \hspace{0.5em}s.t.\hspace{0.5em}}
\newcommand{\rea}[1]{\mathrm{Re}(#1)}
\newcommand{\ima}[1]{\mathrm{Im}(#1)}
\newcommand{\comp}{\mathsf{C}}
\newcommand{\myHS}{ \hspace{0.5em}}
\newcommand{\diam}[1]{\mathrm{diam}(#1)}
\newcommand{\domain}[1]{\mathrm{dom}(#1)}


\newcounter{PropNumber}
\setcounter{PropNumber}{82}
\newcommand{\propCount}[1][1]{%
   \addtocounter{PropNumber}{#1}%
   \thePropNumber%
}
\newcounter{SubPropNumber}
\newcommand{\subPropCount}[1][1]{%
   \addtocounter{SubPropNumber}{1}%
   \theSubPropNumber%
}
\newcommand{\resetSubPropCount}{%
   \setcounter{SubPropNumber}{0}%
}

\newcommand{\myId}{\mathrm{Id}}
\newcommand{\myIm}{\mathrm{im}}
\newcommand{\myObj}{\mathrm{Obj}}
\newcommand{\myHom}{\mathrm{Hom}}
\newcommand{\myEnd}{\mathrm{End}}
\newcommand{\myAut}{\mathrm{Aut}}

\newcommand{\mcateg}[1]{{\bm{\mathsf{#1}}}}

% Thank you Gonzalo Medina and Moriambar who wrote this on stack exchange:
%https://tex.stackexchange.com/questions/74125/how-do-i-put-text-over-symbols%
\newcommand{\myequiv}[1]{\stackrel{\mathclap{\mbox{\footnotesize{$#1$}}}}{\equiv}}

% Thank you chs who wrote this on stack exchange:
%https://tex.stackexchange.com/questions/89821/how-to-draw-a-solid-colored-circle%
\newcommand{\filledcirc}[1][.]{\ensuremath{\hspace{0.05em}{\color{#1}\bullet}\mathllap{\circ}\hspace{0.05em}}}

%Thank you blerbl who wrote this on stack exchange:
%https://tex.stackexchange.com/questions/25348/latex-symbol-for-does-not-divide
\newcommand{\ndiv}{\hspace{-0.3em}\not|\hspace{0.35em}}

\newcommand{\mySepOne}[1][.]{%
   {\noindent\color{#1}{\rule{6.5in}{1mm}}}\\%
}
\newcommand{\mySepTwo}[1][.]{%
   {\noindent\color{#1}{\rule{6.5in}{0.5mm}}}\\%
}

\newenvironment{myClosureOne}[2][.]{%
   \color{#1}%
   \begin{tabular}{|p{#2in}|} \hline \\%
}{%
   \\ \hline \end{tabular}%
}

\newcommand{\fillInBlank}[2][.]{{%
   \color{#1}%
   \rule[-0.12em]{#2em}{0.06em}\rule[-0.12em]{#2em}{0.06em}%
   \rule[-0.12em]{#2em}{0.06em}
}}

\newcommand{\retTwo}{\hfill\bigbreak}

\newcounter{LectureNumber}
\newcommand*{\markLecture}[1]{%
   \stepcounter{LectureNumber}%
   {\huge \color{Black} \textbf{Lecture \theLectureNumber: #1} \newline}%
}

\newcommand{\myVS}{\vphantom{$\int_a^b$}}

% Overarrow stuff:
% ~~~~~~~~~~~~~~~~~~~~~~~~~~~~~~~~~~~~~~~~~~~~~~~~~~~~~~~~~~
\NewOverArrowCommand{myVector}{%
   start = {{\smallermathstyle\relbar}},
   middle = {{\smallermathstyle\relbareda}},
   end={{\rightharpoonup}}, space before arrow=0.15em,
   space after arrow=-0.045em,
}

\NewOverArrowCommand{myBar}{%
   start = {{\relbar}},
   middle = {{\relbar}},
   end={{\relbar}}, space before arrow=0.15em,
   space after arrow=-0.025em,
}

% ~~~~~~~~~~~~~~~~~~~~~~~~~~~~~~~~~~~~~~~~~~~~~~~~~~~~~~~~~~~~

\newcommand{\mVec}[1]{\myVector{#1}}
\newcommand{\mVecAst}[1]{\myVector*{#1}}
\newcommand{\mMat}[1]{\mathbf{#1}}

\title{Math 140B Lecture Notes (Professor: Brandon Seward)}
\author{Isabelle Mills}

\begin{document}
\maketitle{}
\setul{0.14em}{0.07em}
\calibri

\hOne
\markLecture{4/1/2024}

Let $f: E \longrightarrow \mathbb{R}$ where $E \subseteq \mathbb{R}$.

\begin{myIndent}\hTwo
   Since $E$ is the domain of $f$, we shall also refer to it as $\domain{f}$.\retTwo
\end{myIndent}

Fix a point $x \in E \cap E^\prime$. Then consider the function $\frac{f(t)-f(x)}{t-x}$ for $t \in \domain{f}\setminus\{x\}$\\[2pt] and define the \udefine{derivative} of $f$ at $x$ to be $f^\prime(x) = \lim\limits_{t\rightarrow x}\left(\frac{f(t)-f(x)}{t-x}\right)$ provided that this\\[2pt] limit exists. When the above limit exists, we say $f$ is differentiable at $x$. \\ [2pt]

We say $f$ is differentiable on $D \subseteq E$ if $f$ is differentiable at every point in $D$,\\ and if $f$ is differentiable on its entire domain, then we call $f$ \udefine{differentiable}.\retTwo

The function $f^\prime(x) = \lim\limits_{t\rightarrow x}\left(\frac{f(t)-f(x)}{t-x}\right)$ is called the \udefine{derivative} of $f$.\retTwo


{\begin{myIndent}\hTwo
   Proposition \propCount: If $f$ is differentiable at $x$, then $f$ is continuous at $x$.\\ [-10pt]
   
   \begin{myIndent}\hThree
      Proof:\\
      Note that $\lim\limits_{t\rightarrow x}\left(f(t)\right) = \lim\limits_{t\rightarrow x}\left((t-x)\frac{f(t)-f(x)}{t-x} + f(x)\right)$.\\

      Now $\lim\limits_{t\rightarrow x}(t-x) = 0$ and we know $\lim\limits_{t\rightarrow x}\frac{f(t)-f(x)}{t-x} = f^\prime(x)$ exists because $f$ is\\ [3pt] differentiable at $x$. Also, obviously $\lim\limits_{t\rightarrow x}f(x) = f(x)$.\retTwo
      
      Thus by proposition 66 {\color{RawerSienna}(check 140A notes)}, we know that:\\
      
      \begin{tabular}{l}
         $\lim\limits_{t\rightarrow x}\left((t-x)\frac{f(t)-f(x)}{t-x} + f(x)\right) = \lim\limits_{t\rightarrow x}(t-x)\lim\limits_{t\rightarrow x}\left(\frac{f(t)-f(x)}{t-x}\right) + \lim\limits_{t\rightarrow x}f(x)$\\ [3pt]

         $\phantom{\lim\limits_{t\rightarrow x}\left((t-x)\frac{f(t)-f(x)}{t-x} + f(x)\right)} = 0\cdot f^\prime(x) + f(x)$ \\ [-4pt]
         $\phantom{\lim\limits_{t\rightarrow x}\left((t-x)\frac{f(t)-f(x)}{t-x} + f(x)\right)} = f(x)$
      \end{tabular}\retTwo

      Thus, $f$ is continuous at $x$.\retTwo
   \end{myIndent}
\end{myIndent}}


{\begin{center} \teachComment
   \begin{myClosureOne}{5}
      Notes:
      \begin{enumerate}
         \item The above proposition says that differentiability is stronger than\newline continuity.
         \item The converse of this proposition is false. For example, the function\newline $f(x) = |x|$ is continuous at $x = 0$ but not differentiable at\newline $x = 0$.
      \end{enumerate}
   \end{myClosureOne}
\end{center}}

\newpage

{\begin{myIndent}\hTwo
   Proposition \propCount: Suppose $f$ and $g$ are real valued functions with\\ $\domain{f}, \domain{g} \subseteq \mathbb{R}$. Also suppose $f$ and $g$ are differentiable at $x$. Then\\ $f + g$, $fg$, and (when $g(x) \neq 0$) $\frac{f}{g}$ are differentiable at $x$ with:\\
   
   \begin{tabular}{l c c c c c}
      (A)\quad\quad $(f + g)^\prime(x) = f^\prime(x) + g^\prime(x)$ & &&&&{\hFour(sum rule)} \\ [4pt]
      (B)\quad\quad $(fg)^\prime(x) = f^\prime(x)g(x) + f(x)g^\prime(x)$ & &&&& {\hFour(product rule)} \\ [4pt]
      (C)\quad\quad $\left(\dfrac{f}{g}\right)^\prime\hspace{-0.3em}(x) = \dfrac{f^\prime(x)g(x) - f(x)g^\prime(x)}{(g(x))^2}$ & &&&& {\hFour(quotient rule)}
   \end{tabular}\retTwo

   \begin{myIndent}\hThree
      Proof:
      \begin{myIndent}
         \begin{itemize}
            \item[(A)] Since both $f$ and $g$ are differentiable, we know that both\\ $f^\prime(x) = \lim\limits_{t\rightarrow x}\frac{f(t) - f(x)}{t - x}$ and $g^\prime(x) = \lim\limits_{t\rightarrow x}\frac{g(t) - g(x)}{t - x}$ exist. So\\ by proposition 66:\\ [-6pt]
            
            \hspace{-1.5em}${(f + g)^\prime(x) = \lim\limits_{t\rightarrow x}\frac{f(t) + g(t) - f(x) - g(x)}{t-x} = \lim\limits_{t\rightarrow x}\frac{f(t) - f(x)}{t - x} + \lim\limits_{t\rightarrow x}\frac{g(t) - g(x)}{t - x}}$\\ [-6pt]

            This means $(f + g)^\prime(x) = f^\prime(x) + g^\prime(x)$.\\ [6pt]

            \item[(B)] Note that:\\
            \begin{tabular}{l}
               $(fg)^\prime(x) = \lim\limits_{t\rightarrow x}\frac{f(t)g(t) - f(x)g(x)}{t-x}$ \\

               $\phantom{(fg)^\prime(x)} = \lim\limits_{t\rightarrow x}\frac{f(t)g(t) \cyPen{\vphantom{.} - f(x)g(t) + f(x)g(t)} - f(x)g(x)}{t-x}$ \\

               $\phantom{(fg)^\prime(x)} = \lim\limits_{t\rightarrow x}\left( g(t)\frac{f(t) - f(x)}{t - x} + f(x)\frac{g(t) - g(x)}{t-x} \right)$
            \end{tabular}\\

            By proposition 83, $g(t) \rightarrow g(x)$ as $t \rightarrow x$. Also, since both $f$\\ and $g$ are differentiable, we know $f^\prime(x) = \lim\limits_{t\rightarrow x}\frac{f(t) - f(x)}{t - x}$ and\\[-2pt] $g^\prime(x) = \lim\limits_{t\rightarrow x}\frac{g(t) - g(x)}{t - x}$ exist. So by proposition 66:\\

            \hspace{-0.5em}$\lim\limits_{t\rightarrow x}\left( g(t)\frac{f(t) - f(x)}{t - x} + f(x)\frac{g(t) - g(x)}{t-x} \right) = f^\prime(x)g(x) + f(x)g^\prime(x)$.\\ [6pt]

            \item[(C)] Note that:\\
            \begin{tabular}{l}
               $\left(\frac{f}{g}\right)^\prime\hspace{-0.3em}(x) = \lim\limits_{t\rightarrow x}\frac{\frac{f(t)}{g(t)} - \frac{f(x)}{g(x)}}{t - x}$ \\ [2pt]

               $\hphantom{\left(\frac{f}{g}\right)^\prime\hspace{-0.3em}(x)} = \lim\limits_{t\rightarrow x}\left(\frac{1}{g(x)g(t)}\frac{f(t)g(x) - f(x)g(t)}{t - x}\right)$ \\ [8pt]

               $\hphantom{\left(\frac{f}{g}\right)^\prime\hspace{-0.3em}(x)} = \lim\limits_{t\rightarrow x}\left(\frac{1}{g(x)g(t)}\frac{f(t)g(x) \cyPen{\vphantom{.} - f(x)g(x) + f(x)g(x)} - f(x)g(t)}{t - x}\right)$ \\ [8pt]

               $\hphantom{\left(\frac{f}{g}\right)^\prime\hspace{-0.3em}(x)} = \lim\limits_{t\rightarrow x}\left(\frac{1}{g(x)g(t)} \left( g(x)\frac{f(t)-f(x)}{t-x} - f(x)\frac{g(t) - g(x)}{t-x} \right)  \right)$
            \end{tabular}\\ [6pt]
            Now, for the same reasons as before, we can use propositions 83\\ and 66 to separate the parts of the above limit to get that the above limit equals:

            {\centering $\frac{1}{(g(x))^2}\left(g(x)f^\prime(x) - f(x)g^\prime(x)\right)$\par}
         \end{itemize}
      \end{myIndent}
   \end{myIndent}
\end{myIndent}}

\newpage

\exOne

If $f(x) = \alpha$ where $\alpha \in \mathbb{R}$ is constant, then trivially $f^\prime(x) = 0$ for all $x$.\\ Meanwhile, if $f(x) = x$, then we can trivially find that $f^\prime(x) = 1$.\retTwo

Claim 1: For all $n \in \mathbb{Z}^+$, if $f(x) = x^n$, then $f^\prime(x) = nx^{n-1}$.\\ [-15pt]

{\begin{myIndent}\exTwo
   Proof: (we proceed by induction)\retTwo

   Base Case: 
   {\begin{myIndent} \exP
      If $n = 1$, then for $f(x) = x^1$, we have that $f^\prime(x) = 1\cdot x^0$.\retTwo
   \end{myIndent}}

   Induction: 
   {\begin{myIndent}\exP
      Now assume $n > 1$, and for $f(x) = x^{n-1}$, we have that ${f^\prime(x) = (n-1)x^{n-2}\text{.}}$\\ For the rest of this proof, I'll abreviate the derivative of $x^n$ as $(x^n)^\prime$ and the\\ derivative of $x^{n-1}$ as $(x^{n-1})^\prime$. Then using product rule, we know that:\\ [-11pt]

      ${\hspace{-2.5em}(x^n)^\prime = x(x^{n-1})^\prime + 1\cdot x^{n-1} = x\cdot (n-1)x^{n-2} + x^{n-1} = ((n-1) + 1)x^{n-1} = nx^{n-1}}$\retTwo
   \end{myIndent}}
\end{myIndent}}

Claim 2: If $f$ is differentiable at $x$ and $\alpha \in \mathbb{R}$, then $(\alpha f)^\prime(x) = \alpha f^\prime(x)$.\\ [-15pt]

{\begin{myIndent}\exTwo
   Proof:\\ By the product rule: $(\alpha f)^\prime(x) = \alpha f^\prime + (\alpha)^\prime f = \alpha f^\prime + 0\cdot f = \alpha f^\prime$.\retTwo
\end{myIndent}}

These combined with proposition 84 tells us that both polynomials and rational\\ functions are differentiable over their domains.

\mySepTwo

\hOne
{\begin{myIndent}\hTwo
   Proposition \propCount: (chain rule)\\
   Let $f$ and $g$ be real-valued functions with $\domain{f}, \domain{g} \subseteq \mathbb{R}$. Let $x \in \mathbb{R}$.\\ Suppose that $f$ is differentiable at $x$ and that $g$ is differentiable at $f(x)$. Then\\ $g \circ f$ is differentiable at $x$ and $(g \circ f)^\prime(x) = g^\prime(f(x))f^\prime(x)$.\\
   
   {\begin{center}\exTwo
      \begin{myClosureOne}{4.5}
         $\hphantom{.}$\\[-24pt] \ul{Intuition}:
         \begin{myIndent}
            $\lim\limits_{t\rightarrow x}\left(\frac{g(f(t)) - g(f(x))}{\pinkPen{f(t) - f(x)}}\cdot \frac{\pinkPen{f(t)-f(x)}}{t-x}\right) = g^\prime(f(t))\cdot f^\prime(t)$.\newline
         \end{myIndent}

         That said, the issue with this intuition is that we need to\\ address the possibility that $f(t) - f(x) = 0$.\\ [-8pt]
      \end{myClosureOne}\retTwo
   \end{center}}

   {\begin{myIndent}\hThree
      Proof:\\
      Set $y = f(x)$ and define $v(s) = \left\{
      \begin{matrix}
         \frac{g(s) - g(y)}{s-y} - g^\prime(y) & \text{ if } s \neq y \\
         0 & \text{ if } s = y
      \end{matrix}\right.$\retTwo

      Note that $v$ is continuous at $y$. This is because $g$ being differentiable\\ at $f(x) = y$ means that:
      
      {\centering $\lim\limits_{s\rightarrow y}v(s) = \lim\limits_{s\rightarrow y}\left(\frac{g(s) - g(y)}{s-y} - g^\prime(y)\right) = g^\prime(y) - g^\prime(y) = 0 = v(y)$.\retTwo\par}

      \newpage

      Also, since $f$ is differentiable at $x$, we know that $f$ is continuous at $x$.\\ Therefore, $v \circ f$ is continuous at $x$ by proposition 68. Additionally, setting\\ $s = f(t)$, we know that $s \rightarrow y$ as $t \rightarrow x$ because $f$ is continuous at $x$. Thus:

      {\center $ \lim\limits_{t\rightarrow x}v(f(t)) = \lim\limits_{s\rightarrow y}v(s) = 0$ \retTwo\par}
      
      Finally, note that $g(s) - g(y) = (s-y)(g^\prime(y) + v(s))$ for all $s$. Thus by\\ substituting that into our limit:\\ [-8pt]
      \begin{myIndent}
         \begin{tabular}{l}
            $(g \circ f)^\prime(x) = \lim\limits_{t\rightarrow x}\frac{g(f(t)) - g(f(x))}{t - x}$ \\ [8pt]
            $\phantom{(g \circ f)^\prime(x) } = \lim\limits_{t\rightarrow x}\frac{f(t) - f(x)}{t - x}(g^\prime(f(x)) + v(f(t)))$ \\ [8pt]
            $\phantom{(g \circ f)^\prime(x) } = f^\prime(x)\left(g^\prime(f(x)) + 0\right)$\quad\quad (by proposition 66)
         \end{tabular}\retTwo
      \end{myIndent}
   \end{myIndent}}
\end{myIndent}}

\markLecture{4/3/2024}
\exOne\mySepTwo\\ [-12pt]
To start off lecture, here is some intuition about the behavior of derivatives. We'll\\ formally define sine and cosine later (on page \_\_) but for this section please take\\ for granted that $(\sin(x))^\prime = \cos(x)$. Additionally, please take for granted that the\\ power rule holds for non-positive integer exponents.\retTwo

{\begin{myIndent}\exTwo
   \begin{enumerate}
      \item Define $f(x) = \left\{
      \begin{matrix}
         x\sin(\frac{1}{x}) & \text{ if } x \neq 0 \\
         0 & \text{ if } x = 0
      \end{matrix}\right.$\\

      When $x \neq 0$, we have by chain rule that $f^\prime(x) = \sin(\frac{1}{x}) - \frac{1}{x}\cos(\frac{1}{x})$.\\ Meanwhile if $x = 0$, then $\frac{f(t) - f(0)}{t - 0} = \frac{t\sin(\frac{1}{t})}{t} = \sin(\frac{1}{t})$ when $t \neq 0$.\\ So $\lim\limits_{t\rightarrow 0}\left(\frac{f(t) - f(0)}{t - 0}\right)$ does not exist, meaning $f$ is not differentiable at $x$.\\

      This shows that $\domain{f^\prime}$ can be a proper subset of $\domain{f}$.\\

      \item Define $g(x) = \left\{
      \begin{matrix}
         x^2\sin(\frac{1}{x}) & \text{ if } x \neq 0 \\
         0 & \text{ if } x = 0
      \end{matrix}\right.$\\
      
      When $x \neq 0$, we have by chain rule that $g^\prime(x) = 2x\sin(\frac{1}{x}) - \cos(\frac{1}{x})$.\\ Meanwhile when $t \neq 0$:

      {\center $\left|\frac{g(t) - g(0)}{t - 0}\right| = \left|\frac{t^2\sin(\frac{1}{t})}{t}\right| = \left|t\sin(\frac{1}{t})\right| \leq |t|$. \retTwo\par}

      Thus $0 = \lim\limits_{t\rightarrow 0}(-t) \leq \lim\limits_{t\rightarrow 0}\left(\frac{g(t) - g(0)}{t - 0}\right) \leq \lim\limits_{t\rightarrow 0}(t) = 0$, meaning $g^\prime(0) = 0$.\\ [2pt] So $\domain{g^\prime} = \domain{g}$. That said, note that $g^\prime$ has a discontinuity of the\\ [2pt] second kind at $0$. Therefore, because $g$ is continuous, this shows that the\\ [2pt] derivative of a continuous function does not have to be continuous.
   \end{enumerate}
\end{myIndent}}

\mySepTwo

\newpage

\hOne Let $X$ be a metric space. A function $f: X \longrightarrow \mathbb{R}$ has a \udefine{local maximum} at $p \in X$\\ if $\exists \delta > 0 \suchthat \forall x \in B_\delta(p),\myHS f(x) \leq f(p)$. Similarly, $f$ has a \udefine{local minimum} if\\ $\exists \delta > 0 \suchthat \forall x \in B_\delta(p),\myHS f(x) \geq f(p)$.\retTwo

{\begin{myIndent}\hTwo
   Proposition \propCount: Let $f: (a, b) \longrightarrow \mathbb{R}$. If $f$ has a local maximum at $x$ and $f$ is\\ differentiable at $x$, then $f^\prime(x) = 0$.\retTwo
   {\begin{myIndent} \hThree
      Proof:\\
      Let $\delta > 0$ so that $\forall t \in B_\delta(x),\myHS f(t) \leq f(x)$. Then for all $t \in (x-\delta, x)$,\\ $\frac{f(t) - f(x)}{t-x} \geq 0$. So $f^\prime(x) \geq 0$. Similarly for all $t \in (x, x+\delta)$, we have\\ $\frac{f(t) - f(x)}{t - x} \leq 0$. Thus $f^\prime(x) \leq 0$.\retTwo

      Hence $f^\prime(x) = 0$.\retTwo

      
      {\begin{myTindent} \hFour
         Note that analogous reasoning can show that if $f$ has a local\\ minimum at $x$ and $f$ is differentiable at $x$, then $f^\prime(x) = 0$.\retTwo
      \end{myTindent}}
   \end{myIndent}}

   Proposition \propCount: If $f, g: [a, b] \longrightarrow \mathbb{R}$ are continuous on $[a, b]$ and differentiable\\ on $(a, b)$, then there exists $x \in (a, b)$ with:
   
   {\centering$(f(b) - f(a))g^\prime(x) = (g(b) - g(a))f^\prime(x)$.\retTwo\par}

   {\begin{myIndent} \hThree
      Proof:\\
      Define $h: [a, b] \longrightarrow \mathbb{R}$ by $h(x) = (f(b) - f(a))g(x) - (g(b) - g(a))f(x)$.\\ Then $h(a) = f(b)g(a) - g(b)f(a) = h(b)$.\retTwo

      Notice that $h$ is continuous on $[a, b]$ and differentiable on $(a, b)$ because of\\ propositions 70 and 84. Since $h^\prime(x) = (f(b) - f(a))g^\prime(x) - (g(b) - g(a))f^\prime(x)$,\\ for all $x \in (a, b)$ it now suffices to show that there exists $x \in (a, b)$ with\\ $h^\prime(x) = 0$.\\
      
      Since $h$ is continuous on a compact set $[a, b]$, we know that $h$ attains a\\ maximum value and a minimum value over the interval $[a, b]$.
      \begin{myDindent}
         \begin{itemize}
            \item[Case 1:] If $h$ is constant on $[a, b]$, then $h^\prime(x) = 0$ for all $x \in (a, b)$.\\ [-8pt]
            \item[Case 2:] If there is $t \in (a, b)$ with $h(t) > h(a) = h(b)$, then $h(a)$ and\\ $h(b)$ can't be the max. value that $h$ attains on $[a, b]$. So $h$ has a maximum at some point $x \in (a, b)$. Then by the last theorem,\\ $h^\prime(x) = 0$.
            \item[Case 3:] If there is $t \in (a, b)$ with $h(t) < h(a) = h(b)$, then $h(a)$ and\\ $h(b)$ can't be the min. value that $h$ attains on $[a, b]$. So $h$ has a minimum at some point $x \in (a, b)$. Then by the last theorem,\\ $h^\prime(x) = 0$.
         \end{itemize}
      \end{myDindent}
   \end{myIndent}}

   \newpage

   Proposition \propCount: (Mean Value Theorem)\\
   If $f: [a, b] \longrightarrow \mathbb{R}$ is continuous on $[a, b]$ and differentiable on $(a, b)$, then there is\\ $x \in (a, b)$ with $f(b) - f(a) = (b - a)f^\prime(x)$.\retTwo
   
   {\begin{myIndent} \hThree
      To prove this, apply the previous proposition with $g(x) = x$.\retTwo
   \end{myIndent}}

   Proposition \propCount: Suppose $f (a, b) \longrightarrow \mathbb{R}$ is differentiable. Then:
   \begin{itemize}
      \item If $f^\prime(x) \geq 0$ for all $x \in (a, b)$, then $f$ is monotone increasing.
      \item If $f^\prime(x) \leq 0$ for all $x \in (a, b)$, then $f$ is monotone decreasing.
      \item If $f^\prime(x) = 0$ for all $x \in (a, b)$, then $f$ is constant.\retTwo
   \end{itemize}

   
   \begin{myIndent}\hThree
      Proof:\\
      For all $a < x_1 < x_2 < b$, we know by the mean value theorem that there\\ exists $t \in (x_1, x_2)$ with $f(x_2) - f(x_1) = (x_2 - x_1)f^\prime(t)$. Then since\\ $x_2 - x_1 > 0$, the sign of $f(x_2) - f(x_1)$ depends entirely on $f^\prime(t)$.\retTwo
   \end{myIndent}
\end{myIndent}}

\mySepTwo\pracOne

\textbf{Exercise 5.2} Let $f: (a, b) \longrightarrow \mathbb{R}$ be differentiable with $f^\prime(x) > 0$. Then $f$ is strictly\\ increasing.

{\begin{myIndent}\pracTwo
   For all $a < x_1 < x_2 < b$, we know by the mean value theorem that there exists\\ $t \in (x_1, x_2)$ with $f(x_2) - f(x_1) = (x_2 - x_1)f^\prime(t)$. Since $(x_2 - x_1)$ and $f^\prime(t)$ are\\ positive, we thus have that $f(x_2) - f(x_1) > 0$.\retTwo
\end{myIndent}}

As a consequence of $f$ being strictly increasing, we know $f$ is injective. Thus if we\\ restrict the codomain of $f$ to $f((a, b))$, then $f$ is bijective, meaning there exists a\\ function $g = f^{-1}$ such that $(g \circ f)(x) = x = (f \circ g)(x)$. Now we show that $g$ is\\ continuous.

{\begin{myIndent}\pracTwo
   Let $y \in \domain{g}$. Then there exists $x \in \domain{f}$ such that $f(x) = y$. But, since $\domain{f}$\\ is an open interval, we know that there exists $\delta$ such that $[x - \delta, x + \delta] \subseteq \domain{f}$.\\ [-2pt] If we let $\hat{f}$ be the restriction of $f$ whose domain is $[x - \delta, x + \delta]$ and whose codomain\\ is the image of $[x - \delta, x + \delta]$
\end{myIndent}}




% We shall show that the\\ domain of $g$ is some open interval $(c, d)$.

% {\begin{myIndent}\pracTwo
%    Since $f$ is differentiable, we know $f$ is continuous. Thus as $(a, b)$ is also connected, by\\ proposition 78 we know that $f((a, b))$ is connected. Thus, $f((a, b)) \subseteq \left[c, d\right]$ where\\ $c = \inf f((a,b))$ and ${d = \sup f((a,b))\text{.}}$\retTwo

%    If $c \in f((a, b))$, then $\exists\hspace{0.05em} x \in (a, b) \suchthat f(x) = c$. However, because $(a, b)$ is open, we\\ know that $x$ is an interior point of $(a, b)$. So for some $\delta > 0, \myHS x - \delta \in (a, b)$. Now\\ $f(x - \delta) \in f((a, b))$. But since $f$ is strictly increasing, we have $f(x - \delta) < f(x) = c$.\\ This contradicts that $c = \inf f((a,b))$. So we conclude that $c \notin f((a,b))$. By similar\\ reasoning, we can also conclude that $d \notin f((a,b))$.\retTwo

%    As a result, we know $f((a, b))$ is the open interval $(c, d)$ meaning $g$ is a function from\\ $(c, d)$ to $(a, b)$.\retTwo
% \end{myIndent}}

% Next, we will show that $g$ is continuous.

% {\begin{myIndent}\pracTwo
%    Consider any $y \in \domain{g} = (c, d)$. Because $y \in f((a, b))$, there exists\\ $x \in \domain{f} = (a,b)$ such that $f(x) = y$. Then, as $x$ is an interior point of\\ $(a, b)$, there exists $\delta > 0$ such that $[x - \delta, x + \delta]$ is a closed subset of $(a, b)$.
   
%    \newpage
   
%    Let $\hat{f}$ be the restriction of $f$ whose domain is $[x - \delta, x + \delta]$ and whose codomain is\\ the image of $[x - \delta, x + \delta]$ with respect to $f$. Note that since $f$ is continuous, $\hat{f}$ is\\ continuous. So by proposition 75, since $\hat{f}$ is a continuous bijection with a compact\\ domain, $\hat{f}^{-1}$ is continuous.\retTwo

%    Next observe that $\hat{f}$ being continuous and $[x - \delta, x + \delta]$ being connected means\\ that $\hat{f}([x - \delta, x + \delta]) = \domain{\hat{f}^{-1}}$ is also connected by proposition 78. Thus,\\ $\domain{\hat{f}^{-1}}\subseteq \mathbb{R}$ is an interval. But since $\hat{f}(x-\delta), \hat{f}(x+\delta) \in \domain{\hat{f}^{-1}}$ and\\ $\hat{f}(x-\delta) = f(x-\delta) < y < f(x + \delta) = \hat{f}(x + \delta)$, we know $y$ is an interior\\ point of that interval. Hence:
%    \begin{itemize}
%       \item $\lim\limits_{t\rightarrow y}\hat{f}^{-1}(t) = \hat{f}^{-1}(y)$ since $y$ is a limit point of $\domain{\hat{f}^{-1}}$
%       \item $\lim\limits_{t\rightarrow y}\hat{f}^{-1}(t) = \lim\limits_{t\rightarrow y}g(t)$ since $y$ is an interior point of $\domain{\hat{f}^{-1}}$.
%    \end{itemize}\retTwo

%    Thus, we finally conclude that $\lim\limits_{t \rightarrow y}\left( f^{-1}(y)\right) = \lim\limits_{t \rightarrow y}\left(\hat{f}^{-1}(y)\right)$

   
% \end{myIndent}}


% ~~~~~~~~~~~~~~~~~~~~~~~~~~~~~~~~~~~~~~~~~~~~~~~~~~~~~~~~~~~~~~~~~~~~~~ %
\newpage
{\huge \color{Black} \textbf{A List of How The Proposition Numbering in my Notes Lines up With Our Textbook:} \retTwo}
\exOne

\begin{allowTableDashes}
   \begin{tabular}{ c|c||c|c }
      Proposition Number & Label in Textbook & Proposition Number & Label in Textbook \\ \hline
      
      \myVS 83 & 5.2 & 84 & 5.3 \\ \hdashline[10pt/3pt]
      \myVS 85 & 5.5 & 86 & 5.8 \\ \hdashline[10pt/3pt]
      \myVS 87 & 5.9  & 88 & 5.10 \\ \hdashline[10pt/3pt]
      \myVS 89 & 5.11 & 90 &  \\ \hdashline[10pt/3pt]
      \myVS 91 &  & 92 &  \\ \hdashline[10pt/3pt]
   \end{tabular}

\end{allowTableDashes}

\retTwo

Our textbook is \textit{Principles of Mathematical Analysis} by Walter Rudin.
\end{document}