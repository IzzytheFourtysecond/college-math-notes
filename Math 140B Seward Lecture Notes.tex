% Note for any github stalkers. I am currently in the process
% of learning LaTeX. I don't know what I'm doing yet. Sorry
% if my code absolutely sucks.

\documentclass{book}

\usepackage{fontspec} % used to import Calibri
\usepackage{anyfontsize} % used to adjust font size

% needed for inch and other length measurements
% to be recognized
\usepackage{calc}

% for colors and text effects as is hopefully obvious
\usepackage[dvipsnames]{xcolor}
\usepackage{soul}

% control over margins
\usepackage[margin=1in]{geometry}
\usepackage[strict]{changepage}

\usepackage{mathtools}
\usepackage{amsfonts}
\usepackage{bm}

\usepackage[scr=rsfso, scrscaled=.96]{mathalpha}

\usepackage{amssymb} % originally imported to get the proof square
\usepackage{xfrac}
\usepackage[overcommands]{overarrows} % Get my preferred vector arrows...
\usepackage{relsize}

% Just am using this to get a dashed line in a table...
% Also you apparently want this to be inactive if you aren't
% using it because it slows compilation.
\usepackage{arydshln} \ADLinactivate 
\newenvironment{allowTableDashes}{\ADLactivate}{\ADLinactivate}

\usepackage{graphicx}
\graphicspath{{./158_Images/}}

\usepackage{tikz}
   \usetikzlibrary{arrows.meta}
   \usetikzlibrary{graphs, graphs.standard}

\usepackage{quiver} %commutative diagrams


\newfontfamily{\calibri}{Calibri}
\setlength{\parindent}{0pt}
\definecolor{RawerSienna}{HTML}{945D27}

% ~~~~~~~~~~~~~~~~~~~~~~~~~~~~~~~~~~~~~~~~~~~~~~~~~~
%Arrow Commands:

% Thank you Bernard, gernot, and Sigur who I copied this from:
% https://tex.stackexchange.com/questions/364096/command-for-longhookrightarrow
\newcommand{\hooklongrightarrow}{\lhook\joinrel\longrightarrow}
\newcommand{\hooklongleftarrow}{\longleftarrow\joinrel\rhook}
\newcommand{\hookxlongrightarrow}[2][]{\lhook\joinrel\xrightarrow[#1]{#2}}
\newcommand{\hookxlongleftarrow}[2][]{\xleftarrow[#1]{#2}\joinrel\rhook}

% Thank you egreg who I copied from:
% https://tex.stackexchange.com/questions/260554/two-headed-version-of-xrightarrow
\newcommand{\longrightarrowdbl}{\longrightarrow\mathrel{\mkern-14mu}\rightarrow}
\newcommand{\longleftarrowdbl}{\leftarrow\mathrel{\mkern-14mu}\longleftarrow}

\newcommand{\xrightarrowdbl}[2][]{%
  \xrightarrow[#1]{#2}\mathrel{\mkern-14mu}\rightarrow
}
\newcommand{\xleftarrowdbl}[2][]{%
  \leftarrow\mathrel{\mkern-14mu}\xleftarrow[#1]{#2}
}

% ~~~~~~~~~~~~~~~~~~~~~~~~~~~~~~~~~~~~~~~~~~~~~~~~~~

\newcommand{\hOne}{%
   \color{Black}%
   \fontsize{14}{16}\selectfont%
}
\newcommand{\hTwo}{%
   \color{MidnightBlue}%
   \fontsize{13}{15}\selectfont%
}
\newcommand{\hThree}{%
   \color{PineGreen!85!Orange}
   \fontsize{13}{15}\selectfont%
}
\newcommand{\hFour}{%
   \color{Cerulean}
   \fontsize{12}{14}\selectfont%
}
\newcommand{\myComment}{%
   \color{RawerSienna}%
   \fontsize{12}{14}\selectfont%
}
% \newcommand{\pracOne}{
%    \color{BrickRed}%
%    \fontsize{13}{15}\selectfont%
% }
\newcommand{\teachComment}{
   \color{Orange}%
   \fontsize{12}{14}\selectfont%
}
\newcommand{\exOne}{%
   \color{Purple}%
   \fontsize{14}{16}\selectfont%
}
\newcommand{\exTwo}{%
   \color{RedViolet}%
   \fontsize{13}{15}\selectfont%
}
\newcommand{\exP}{%
   \color{VioletRed}%
   \fontsize{12}{14}\selectfont%
}
% ~~~~~~~~~~~~~~~~~~~~~~~~~~~~~~~~~~~~~~~~~~~~~~~~

\newcommand{\cyPen}[1]{{\vphantom{.}\color{Cerulean}#1}}

\newenvironment{myIndent}{%
   \begin{adjustwidth}{2.5em}{0em}%
}{%
   \end{adjustwidth}%
}

\newenvironment{myDindent}{%
   \begin{adjustwidth}{5em}{0em}%
}{%
   \end{adjustwidth}%
}

\newenvironment{myTindent}{%
   \begin{adjustwidth}{7.5em}{0em}%
}{%
   \end{adjustwidth}%
}

\newenvironment{myConstrict}{%
   \begin{adjustwidth}{2.5em}{2.5em}%
}{%
   \end{adjustwidth}%
}

\newcommand{\udefine}[1]{{%
   \setulcolor{Red}%
   \setul{0.14em}{0.07em}%
   \ul{#1}%
}}

\newcommand{\uuline}[2][.]{%
{\vphantom{a}\color{#1}%
\rlap{\rule[-0.18em]{\widthof{#2}}{0.06em}}%
\rlap{\rule[-0.32em]{\widthof{#2}}{0.06em}}}%
#2}

\newcommand*{\markDate}[1]{%
   {\huge \color{Black} \textbf{#1} \newline}%
}

\newcommand{\pprime}{{\prime\prime}}
\newcommand{\suchthat}{ \hspace{0.5em}s.t.\hspace{0.5em}}
\newcommand{\rea}[1]{\mathrm{Re}(#1)}
\newcommand{\ima}[1]{\mathrm{Im}(#1)}
\newcommand{\comp}{\mathsf{C}}
\newcommand{\myHS}{ \hspace{0.5em}}
\newcommand{\diam}[1]{\mathrm{diam}(#1)}
\newcommand{\domain}[1]{\mathrm{dom}(#1)}


\newcounter{PropNumber}
\setcounter{PropNumber}{82}
\newcommand{\propCount}[1][1]{%
   \addtocounter{PropNumber}{#1}%
   \thePropNumber%
}
\newcounter{SubPropNumber}
\newcommand{\subPropCount}[1][1]{%
   \addtocounter{SubPropNumber}{1}%
   \theSubPropNumber%
}
\newcommand{\resetSubPropCount}{%
   \setcounter{SubPropNumber}{0}%
}

\newcommand{\myId}{\mathrm{Id}}
\newcommand{\myIm}{\mathrm{im}}
\newcommand{\myObj}{\mathrm{Obj}}
\newcommand{\myHom}{\mathrm{Hom}}
\newcommand{\myEnd}{\mathrm{End}}
\newcommand{\myAut}{\mathrm{Aut}}

\newcommand{\mcateg}[1]{{\bm{\mathsf{#1}}}}

% Thank you Gonzalo Medina and Moriambar who wrote this on stack exchange:
%https://tex.stackexchange.com/questions/74125/how-do-i-put-text-over-symbols%
\newcommand{\myequiv}[1]{\stackrel{\mathclap{\mbox{\footnotesize{$#1$}}}}{\equiv}}

% Thank you chs who wrote this on stack exchange:
%https://tex.stackexchange.com/questions/89821/how-to-draw-a-solid-colored-circle%
\newcommand{\filledcirc}[1][.]{\ensuremath{\hspace{0.05em}{\color{#1}\bullet}\mathllap{\circ}\hspace{0.05em}}}

%Thank you blerbl who wrote this on stack exchange:
%https://tex.stackexchange.com/questions/25348/latex-symbol-for-does-not-divide
\newcommand{\ndiv}{\hspace{-0.3em}\not|\hspace{0.35em}}

\newcommand{\mySepOne}[1][.]{%
   {\noindent\color{#1}{\rule{6.5in}{1mm}}}\\%
}
\newcommand{\mySepTwo}[1][.]{%
   {\noindent\color{#1}{\rule{6.5in}{0.5mm}}}\\%
}

\newenvironment{myClosureOne}[2][.]{%
   \color{#1}%
   \begin{tabular}{|p{#2in}|} \hline \\%
}{%
   \\ \hline \end{tabular}%
}

\newcommand{\fillInBlank}[2][.]{{%
   \color{#1}%
   \rule[-0.12em]{#2em}{0.06em}\rule[-0.12em]{#2em}{0.06em}%
   \rule[-0.12em]{#2em}{0.06em}
}}

\newcommand{\retTwo}{\hfill\bigbreak}

\newcounter{LectureNumber}
\newcommand*{\markLecture}[1]{%
   \stepcounter{LectureNumber}%
   {\huge \color{Black} \textbf{Lecture \theLectureNumber: #1} \newline}%
}

\newcommand{\myVS}{\vphantom{$\int_a^b$}}

% Overarrow stuff:
% ~~~~~~~~~~~~~~~~~~~~~~~~~~~~~~~~~~~~~~~~~~~~~~~~~~~~~~~~~~
\NewOverArrowCommand{myVector}{%
   start = {{\smallermathstyle\relbar}},
   middle = {{\smallermathstyle\relbareda}},
   end={{\rightharpoonup}}, space before arrow=0.15em,
   space after arrow=-0.045em,
}

\NewOverArrowCommand{myBar}{%
   start = {{\relbar}},
   middle = {{\relbar}},
   end={{\relbar}}, space before arrow=0.15em,
   space after arrow=-0.025em,
}

% ~~~~~~~~~~~~~~~~~~~~~~~~~~~~~~~~~~~~~~~~~~~~~~~~~~~~~~~~~~~~

\newcommand{\mVec}[1]{\myVector{#1}}
\newcommand{\mVecAst}[1]{\myVector*{#1}}
\newcommand{\mMat}[1]{\mathbf{#1}}

\title{Math 140B Lecture Notes (Professor: Brandon Seward)}
\author{Isabelle Mills}

\begin{document}
\maketitle{}
\setul{0.14em}{0.07em}
\calibri

\hOne
\markLecture{4/1/2024}

Let $f: E \longrightarrow \mathbb{R}$ where $E \subseteq \mathbb{R}$.

\begin{myIndent}\hTwo
   Since $E$ is the domain of $f$, we shall also refer to it as $\domain{f}$.\retTwo
\end{myIndent}

Fix a point $x \in E \cap E^\prime$. Then consider the function $\frac{f(t)-f(x)}{t-x}$ for $t \in \domain{f}\setminus\{x\}$\\[2pt] and define the \udefine{derivative} of $f$ at $x$ to be $f^\prime(x) = \lim\limits_{t\rightarrow x}\left(\frac{f(t)-f(x)}{t-x}\right)$ provided that this\\[2pt] limit exists. When the above limit exists, we say $f$ is differentiable at $x$. \\ [2pt]

We say $f$ is differentiable on $D \subseteq E$ if $f$ is differentiable at every point in $D$,\\ and if $f$ is differentiable on its entire domain, then we call $f$ \udefine{differentiable}.\retTwo

The function $f^\prime(x) = \lim\limits_{t\rightarrow x}\left(\frac{f(t)-f(x)}{t-x}\right)$ is called the \udefine{derivative} of $f$.\retTwo


{\begin{myIndent}\hTwo
   Proposition \propCount: If $f$ is differentiable at $x$, then $f$ is continuous at $x$.\\ [-10pt]
   
   \begin{myIndent}\hThree
      Proof:\\
      Note that $\lim\limits_{t\rightarrow x}\left(f(t)\right) = \lim\limits_{t\rightarrow x}\left((t-x)\frac{f(t)-f(x)}{t-x} + f(x)\right)$.\\

      Now $\lim\limits_{t\rightarrow x}(t-x) = 0$ and we know $\lim\limits_{t\rightarrow x}\frac{f(t)-f(x)}{t-x} = f^\prime(x)$ exists because $f$ is\\ [3pt] differentiable at $x$. Also, obviously $\lim\limits_{t\rightarrow x}f(x) = f(x)$.\retTwo
      
      Thus by proposition 66 {\color{RawerSienna}(check 140A notes)}, we know that:\\
      
      \begin{tabular}{l}
         $\lim\limits_{t\rightarrow x}\left((t-x)\frac{f(t)-f(x)}{t-x} + f(x)\right) = \lim\limits_{t\rightarrow x}(t-x)\lim\limits_{t\rightarrow x}\left(\frac{f(t)-f(x)}{t-x}\right) + \lim\limits_{t\rightarrow x}f(x)$\\ [3pt]

         $\phantom{\lim\limits_{t\rightarrow x}\left((t-x)\frac{f(t)-f(x)}{t-x} + f(x)\right)} = 0\cdot f^\prime(x) + f(x)$ \\ [-4pt]
         $\phantom{\lim\limits_{t\rightarrow x}\left((t-x)\frac{f(t)-f(x)}{t-x} + f(x)\right)} = f(x)$
      \end{tabular}\retTwo

      Thus, $f$ is continuous at $x$.\retTwo
   \end{myIndent}
\end{myIndent}}


{\begin{center} \teachComment
   \begin{myClosureOne}{5}
      Notes:
      \begin{enumerate}
         \item The above proposition says that differentiability is stronger than\newline continuity.
         \item The converse of this proposition is false. For example, the function\newline $f(x) = |x|$ is continuous at $x = 0$ but not differentiable at\newline $x = 0$.
      \end{enumerate}
   \end{myClosureOne}
\end{center}}

\newpage

{\begin{myIndent}\hTwo
   Proposition \propCount: Suppose $f$ and $g$ are real valued functions with\\ $\domain{f}, \domain{g} \subseteq \mathbb{R}$. Also suppose $f$ and $g$ are differentiable at $x$. Then\\ $f + g$, $fg$, and (when $g(x) \neq 0$) $\frac{f}{g}$ are differentiable at $x$ with:\\
   
   \begin{tabular}{l c c c c c}
      (A)\quad\quad $(f + g)^\prime(x) = f^\prime(x) + g^\prime(x)$ & &&&&{\hFour(sum rule)} \\ [4pt]
      (B)\quad\quad $(fg)^\prime(x) = f^\prime(x)g(x) + f(x)g^\prime(x)$ & &&&& {\hFour(product rule)} \\ [4pt]
      (C)\quad\quad $\left(\dfrac{f}{g}\right)^\prime\hspace{-0.3em}(x) = \dfrac{f^\prime(x)g(x) - f(x)g^\prime(x)}{(g(x))^2}$ & &&&& {\hFour(quotient rule)}
   \end{tabular}\retTwo

   \begin{myIndent}\hThree
      Proof:
      \begin{myIndent}
         \begin{itemize}
            \item[(A)] Since both $f$ and $g$ are differentiable, we know that both\\ $f^\prime(x) = \lim\limits_{t\rightarrow x}\frac{f(t) - f(x)}{t - x}$ and $g^\prime(x) = \lim\limits_{t\rightarrow x}\frac{g(t) - g(x)}{t - x}$ exist. So\\ by proposition 66:\\
            
            \hspace{-1.5em}${(f + g)^\prime(x) = \lim\limits_{t\rightarrow x}\frac{f(t) + g(t) - f(x) - g(x)}{t-x} = \lim\limits_{t\rightarrow x}\frac{f(t) - f(x)}{t - x} + \lim\limits_{t\rightarrow x}\frac{g(t) - g(x)}{t - x}}$\\

            This means $(f + g)^\prime(x) = f^\prime(x) + g^\prime(x)$.\retTwo \retTwo

            \item[(B)] Note that:\\
            \begin{tabular}{l}
               $(fg)^\prime(x) = \lim\limits_{t\rightarrow x}\frac{f(t)g(t) - f(x)g(x)}{t-x}$ \\

               $\phantom{(fg)^\prime(x)} = \lim\limits_{t\rightarrow x}\frac{f(t)g(t) \cyPen{\vphantom{.} - f(x)g(t) + f(x)g(t)} - f(x)g(x)}{t-x}$ \\

               $\phantom{(fg)^\prime(x)} = \lim\limits_{t\rightarrow x}\left( g(t)\frac{f(t) - f(x)}{t - x} + f(x)\frac{g(t) - g(x)}{t-x} \right)$
            \end{tabular}\\

            By proposition 83, $g(t) \rightarrow g(x)$ as $t \rightarrow x$. Also, since both $f$\\ and $g$ are differentiable, we know $f^\prime(x) = \lim\limits_{t\rightarrow x}\frac{f(t) - f(x)}{t - x}$ and\\[-2pt] $g^\prime(x) = \lim\limits_{t\rightarrow x}\frac{g(t) - g(x)}{t - x}$ exist. So by proposition 66:\\

            \hspace{-0.5em}$\lim\limits_{t\rightarrow x}\left( g(t)\frac{f(t) - f(x)}{t - x} + f(x)\frac{g(t) - g(x)}{t-x} \right) = f^\prime(x)g(x) + f(x)g^\prime(x)$.\retTwo\retTwo

            \item[(C)] Note that:\\
            \begin{tabular}{l}
               $\left(\frac{f}{g}\right)^\prime\hspace{-0.3em}(x) = \lim\limits_{t\rightarrow x}\frac{\frac{f(t)}{g(t)} - \frac{f(x)}{g(x)}}{t - x}$ \\

               $\hphantom{\left(\frac{f}{g}\right)^\prime\hspace{-0.3em}(x)} = \lim\limits_{t\rightarrow x}\left(\frac{1}{g(x)g(t)}\frac{f(t)g(x) - f(x)g(t)}{t - x}\right)$ \\

               $\hphantom{\left(\frac{f}{g}\right)^\prime\hspace{-0.3em}(x)} = \lim\limits_{t\rightarrow x}\left(\frac{1}{g(x)g(t)}\frac{f(t)g(x) - f(x)g(t)}{t - x}\right)$ \\
            \end{tabular}\\
         \end{itemize}
      \end{myIndent}
   \end{myIndent}

\end{myIndent}}















% ~~~~~~~~~~~~~~~~~~~~~~~~~~~~~~~~~~~~~~~~~~~~~~~~~~~~~~~~~~~~~~~~~~~~~~ %
\newpage
{\huge \color{Black} \textbf{A List of How The Proposition Numbering in my Notes Lines up With Our Textbook:} \retTwo}
\exOne

\begin{allowTableDashes}
   \begin{tabular}{ c|c||c|c }
      Proposition Number & Label in Textbook & Proposition Number & Label in Textbook \\ \hline
      
      \myVS 83 & 5.2 & 84 & 5.3 \\ \hdashline[10pt/3pt]
   \end{tabular}

\end{allowTableDashes}

\retTwo

Our textbook is \textit{Principles of Mathematical Analysis} by Walter Rudin.
\end{document}