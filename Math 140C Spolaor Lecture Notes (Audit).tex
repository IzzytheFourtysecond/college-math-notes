% Note for any github stalkers. I am currently in the process
% of learning LaTeX. I don't know what I'm doing yet. Sorry
% if my code absolutely sucks.

\documentclass{book}

\usepackage{fontspec} % used to import Calibri
\usepackage{anyfontsize} % used to adjust font size

% needed for inch and other length measurements
% to be recognized
\usepackage{calc}

% for colors and text effects as is hopefully obvious
\usepackage[dvipsnames]{xcolor}
\usepackage{soul}

% control over margins
\usepackage[margin=1in]{geometry}
\usepackage[strict]{changepage}

\usepackage{mathtools}
\usepackage{amsfonts}
\usepackage{bm}

\usepackage[scr=rsfs, scrscaled=.96]{mathalpha}

\usepackage{amssymb} % originally imported to get the proof square
\usepackage{xfrac}
\usepackage[overcommands]{overarrows} % Get my preferred vector arrows...
\usepackage{relsize}

% Just am using this to get a dashed line in a table...
% Also you apparently want this to be inactive if you aren't
% using it because it slows compilation.
\usepackage{arydshln} \ADLinactivate 
\newenvironment{allowTableDashes}{\ADLactivate}{\ADLinactivate}

\usepackage{graphicx}
\graphicspath{{./158_Images/}}

\usepackage{tikz}
   \usetikzlibrary{arrows.meta}
   \usetikzlibrary{graphs, graphs.standard}

\usepackage{quiver} %commutative diagrams


\newfontfamily{\calibri}{Calibri}
\setlength{\parindent}{0pt}
\definecolor{RawerSienna}{HTML}{945D27}

% ~~~~~~~~~~~~~~~~~~~~~~~~~~~~~~~~~~~~~~~~~~~~~~~~~~
%Arrow Commands:

% Thank you Bernard, gernot, and Sigur who I copied this from:
% https://tex.stackexchange.com/questions/364096/command-for-longhookrightarrow
\newcommand{\hooklongrightarrow}{\lhook\joinrel\longrightarrow}
\newcommand{\hooklongleftarrow}{\longleftarrow\joinrel\rhook}
\newcommand{\hookxlongrightarrow}[2][]{\lhook\joinrel\xrightarrow[#1]{#2}}
\newcommand{\hookxlongleftarrow}[2][]{\xleftarrow[#1]{#2}\joinrel\rhook}

% Thank you egreg who I copied from:
% https://tex.stackexchange.com/questions/260554/two-headed-version-of-xrightarrow
\newcommand{\longrightarrowdbl}{\longrightarrow\mathrel{\mkern-14mu}\rightarrow}
\newcommand{\longleftarrowdbl}{\leftarrow\mathrel{\mkern-14mu}\longleftarrow}

\newcommand{\xrightarrowdbl}[2][]{%
  \xrightarrow[#1]{#2}\mathrel{\mkern-14mu}\rightarrow
}
\newcommand{\xleftarrowdbl}[2][]{%
  \leftarrow\mathrel{\mkern-14mu}\xleftarrow[#1]{#2}
}

% ~~~~~~~~~~~~~~~~~~~~~~~~~~~~~~~~~~~~~~~~~~~~~~~~~~

\newcommand{\hOne}{%
   \color{Black}%
   \fontsize{14}{16}\selectfont%
}
\newcommand{\hTwo}{%
   \color{MidnightBlue}%
   \fontsize{13}{15}\selectfont%
}
\newcommand{\hThree}{%
   \color{PineGreen!85!Orange}
   \fontsize{13}{15}\selectfont%
}
\newcommand{\hFour}{%
   \color{Cerulean}
   \fontsize{12}{14}\selectfont%
}
\newcommand{\myComment}{%
   \color{RawerSienna}%
   \fontsize{12}{14}\selectfont%
}
% \newcommand{\pracOne}{
%    \color{BrickRed}%
%    \fontsize{13}{15}\selectfont%
% }
\newcommand{\teachComment}{
   \color{Orange}%
   \fontsize{12}{14}\selectfont%
}
\newcommand{\exOne}{%
   \color{Purple}%
   \fontsize{14}{16}\selectfont%
}
\newcommand{\exTwo}{%
   \color{RedViolet}%
   \fontsize{13}{15}\selectfont%
}
\newcommand{\exP}{%
   \color{VioletRed}%
   \fontsize{12}{14}\selectfont%
}
% ~~~~~~~~~~~~~~~~~~~~~~~~~~~~~~~~~~~~~~~~~~~~~~~~

\newcommand{\cyPen}[1]{{\vphantom{.}\color{Cerulean}#1}}

\newenvironment{myIndent}{%
   \begin{adjustwidth}{2.5em}{0em}%
}{%
   \end{adjustwidth}%
}

\newenvironment{myDindent}{%
   \begin{adjustwidth}{5em}{0em}%
}{%
   \end{adjustwidth}%
}

\newenvironment{myTindent}{%
   \begin{adjustwidth}{7.5em}{0em}%
}{%
   \end{adjustwidth}%
}

\newenvironment{myConstrict}{%
   \begin{adjustwidth}{2.5em}{2.5em}%
}{%
   \end{adjustwidth}%
}

\newcommand{\udefine}[1]{{%
   \setulcolor{Red}%
   \setul{0.14em}{0.07em}%
   \ul{#1}%
}}

\newcommand{\uuline}[2][.]{%
{\vphantom{a}\color{#1}%
\rlap{\rule[-0.18em]{\widthof{#2}}{0.06em}}%
\rlap{\rule[-0.32em]{\widthof{#2}}{0.06em}}}%
#2}

\newcommand*{\markDate}[1]{%
   {\huge \color{Black} \textbf{#1} \newline}%
}

\newcommand{\pprime}{{\prime\prime}}
\newcommand{\suchthat}{ \hspace{0.5em}s.t.\hspace{0.5em}}
\newcommand{\rea}[1]{\mathrm{Re}(#1)}
\newcommand{\ima}[1]{\mathrm{Im}(#1)}
\newcommand{\comp}{\mathsf{C}}
\newcommand{\myHS}{ \hspace{0.5em}}
\newcommand{\diam}[1]{\mathrm{diam}(#1)}
\newcommand{\domain}[1]{\mathrm{dom}(#1)}
\newcommand{\mySpan}{\mathrm{span}}
\newcommand{\myDim}[1]{\mathrm{dim}(#1)}

\newcommand{\rank}[1]{\mathrm{rk}(#1)}
\newcommand{\nullity}[1]{\mathrm{null}(#1)}
\newcommand{\rangeSp}[1]{\mathscr{R}(#1)}
\newcommand{\nullSp}[1]{\mathscr{N}(#1)}


\newcounter{PropNumber}
\setcounter{PropNumber}{82}
\newcommand{\propCount}[1][1]{%
   \addtocounter{PropNumber}{#1}%
   \thePropNumber%
}
\newcounter{SubPropNumber}
\newcommand{\subPropCount}[1][1]{%
   \addtocounter{SubPropNumber}{1}%
   \theSubPropNumber%
}
\newcommand{\resetSubPropCount}{%
   \setcounter{SubPropNumber}{0}%
}

\newcommand{\myId}{\mathrm{Id}}
\newcommand{\myIm}{\mathrm{im}}
\newcommand{\myObj}{\mathrm{Obj}}
\newcommand{\myHom}{\mathrm{Hom}}
\newcommand{\myEnd}{\mathrm{End}}
\newcommand{\myAut}{\mathrm{Aut}}

\newcommand{\mcateg}[1]{{\bm{\mathsf{#1}}}}

% Thank you Gonzalo Medina and Moriambar who wrote this on stack exchange:
%https://tex.stackexchange.com/questions/74125/how-do-i-put-text-over-symbols%
\newcommand{\myequiv}[1]{\stackrel{\mathclap{\mbox{\footnotesize{$#1$}}}}{\equiv}}

% Thank you chs who wrote this on stack exchange:
%https://tex.stackexchange.com/questions/89821/how-to-draw-a-solid-colored-circle%
\newcommand{\filledcirc}[1][.]{\ensuremath{\hspace{0.05em}{\color{#1}\bullet}\mathllap{\circ}\hspace{0.05em}}}

%Thank you blerbl who wrote this on stack exchange:
%https://tex.stackexchange.com/questions/25348/latex-symbol-for-does-not-divide
\newcommand{\ndiv}{\hspace{-0.3em}\not|\hspace{0.35em}}

\newcommand{\mySepOne}[1][.]{%
   {\noindent\color{#1}{\rule{6.5in}{1mm}}}\\%
}
\newcommand{\mySepTwo}[1][.]{%
   {\noindent\color{#1}{\rule{6.5in}{0.5mm}}}\\%
}

\newenvironment{myClosureOne}[2][.]{%
   \color{#1}%
   \begin{tabular}{|p{#2in}|} \hline \\%
}{%
   \\ \hline \end{tabular}%
}

\newcommand{\fillInBlank}[2][.]{{%
   \color{#1}%
   \rule[-0.12em]{#2em}{0.06em}\rule[-0.12em]{#2em}{0.06em}%
   \rule[-0.12em]{#2em}{0.06em}
}}

\newcommand{\retTwo}{\hfill\bigbreak}

\newcounter{LectureNumber}
\newcommand*{\markLecture}[1]{%
   \stepcounter{LectureNumber}%
   {\huge \color{Black} \textbf{Lecture \theLectureNumber: #1} \newline}%
}

\newcommand{\myVS}{\vphantom{$\int_a^b$}}

% Overarrow stuff:
% ~~~~~~~~~~~~~~~~~~~~~~~~~~~~~~~~~~~~~~~~~~~~~~~~~~~~~~~~~~
\NewOverArrowCommand{myVector}{%
   start = {{\smallermathstyle\relbar}},
   middle = {{\smallermathstyle\relbareda}},
   end={{\rightharpoonup}}, space before arrow=0.15em,
   space after arrow=-0.045em,
}

\NewOverArrowCommand{myBar}{%
   start = {{\relbar}},
   middle = {{\relbar}},
   end={{\relbar}}, space before arrow=0.15em,
   space after arrow=-0.025em,
}

% ~~~~~~~~~~~~~~~~~~~~~~~~~~~~~~~~~~~~~~~~~~~~~~~~~~~~~~~~~~~~

\newcommand{\mVec}[1]{\myVector{#1}}
\newcommand{\mVecAst}[1]{\myVector*{#1}}
\newcommand{\mMat}[1]{\mathbf{#1}}

\title{(Audit) Math 140C Lecture Notes (Professor: Luca Spolaor)}
\author{Isabelle Mills}

\begin{document}
\maketitle{}
\setul{0.14em}{0.07em}
\calibri

\hOne
\markLecture{4/2/2024}

A set $X \subseteq \mathbb{R}^n$ where $X \neq \emptyset$ is a \udefine{vector space} if:\\ [-20pt]
\begin{itemize}
   \item $\mVec{x}, \mVec{y} \in X \Longrightarrow \mVec{x} + \mVec{y} \in X$\\ [-20pt]
   \item $\mVec{x} \in X$ and $c \in \mathbb{R} \Longrightarrow c\mVec{x} \in X$.\retTwo
\end{itemize}

If $\phi = \{\mVecAst{x}_1, \ldots, \mVecAst{x}_k\} \subset \mathbb{R}^{n}$, then we define:

{\centering $\mySpan\hspace{0.2em}\phi = \mySpan\{\mVecAst{x}_1, \ldots, \mVecAst{x}_k\} = \left\{ c_1\mVecAst{x}_1 + \ldots + c_k\mVecAst{x}_k \mid c_1, \ldots, c_k \in \mathbb{R} \right\}$.\retTwo\par}

If $E \subseteq \mathbb{R}^n$ and $E = \mySpan\hspace{0.2em}\phi$, then we say $\phi$ \udefine{generates} $E$.\retTwo

{\begin{myIndent}\exOne
   Note that $\mySpan\{\mVecAst{x}_1, \ldots, \mVecAst{x}_2\}$ forms a vector space (this is trivial to check).\retTwo
\end{myIndent}}

$\{\mVecAst{x}_1, \ldots, \mVecAst{x}_k\} \subseteq \mathbb{R}^n$ is called \udefine{linearly independent} if:

{\centering $\sum\limits_{i=1}^{k}{c_i\mVecAst{x}_i} = 0 \Longrightarrow \forall i \in \{1, \ldots, k\},\myHS c_i = 0$.\retTwo\par}

If the above implication does not hold, then we call the set \udefine{linearly dependent}.\retTwo

If $X \subseteq \mathbb{R}^n$ is a vector space, then we define the \udefine{dimension} of $X$ as: 

{\centering ${\myDim{X} = \sup\{k \in \mathbb{N} \cup \{0\} \mid \exists \{\mVecAst{x}_1, \ldots, \mVecAst{x}_k\}\subset X \text{ which is linearly independent}\}\text{.}}$\par}

{\begin{myDindent} \teachComment
   Also, we define any set containing $\mVec{0}$ to be automatically linearly dependent.\\
   This includes the singleton: $\{\mVec{0}\}$.\retTwo
\end{myDindent}}

$Q = \{\mVecAst{x}_1, \ldots, \mVecAst{x}_k\}$ is a \udefine{basis} for $X$ if:\\ [-18pt]
\begin{itemize}
   \item $Q$ is linearly independent.\\ [-20pt]
   \item $\mySpan\hspace{0.2em}Q = X$\retTwo
\end{itemize}

\begin{myIndent}\exOne
   As an example of a basis, for $\mathbb{R}^n$ we define \udefine{the standard basis} as the set\\ $\{e_1, e_2, \ldots, e_n\}$ where $e_i$ is the vector whose $i$th element is $1$ and whose\\ other elements are $0$. It is pretty trivial to check that this set is in fact a\\ basis of $\mathbb{R}^n$.\retTwo
\end{myIndent}

\mySepTwo

{\begin{myIndent} \hTwo
   \uuline{Proposition}: If $B = \{\mVecAst{x}_1, \ldots, \mVecAst{x}_k\}$ is a basis of a vector space $X$, then:\\ [-14pt]
   \begin{enumerate}
      \item $\forall \mVec{v} \in X,\myHS c_1, \ldots, c_k \in \mathbb{R} \suchthat \mVec{v} = \sum\limits_{i=1}^{k}{c_i\mVecAst{x}_i}$
      
      {\begin{myIndent} \hThree
         This is true because $X = \mySpan\hspace{0.2em}B$. So by definition of a span, $\mVec{v}$ can\\ be expressed as a linear combination of the vectors of $B$.
      \end{myIndent}}

      \newpage

      \item The $c_i$ such that $\mVec{v} = \sum\limits_{i=1}^{k}{c_i\mVecAst{x}_i}$ are unique.
      
      {\begin{myIndent}\hThree
         Suppose that $\mVec{v} = \sum c_i\mVecAst{x}_i = \sum \alpha_i\mVecAst{x}_i$. Then $\mVec{0} = \sum(c_i - \alpha_i)\mVecAst{x}_i$.\\ Then since $\{\mVecAst{x}_1, \ldots, \mVecAst{x}_k\}$ are linearly independent, we know for all $i$\\ that $c_i - \alpha_i = 0$. Hence, $c_i = \alpha_i$ for each $i$.\\
      \end{myIndent}}
   \end{enumerate}

   \uuline{Theorem 9.2}: Let $k \in \mathbb{N} \cup \{0\}$. If $X = \mySpan\{\mVecAst{x}_1, \ldots, \mVecAst{x}_k\}$, then $\myDim{X} \leq k$.\retTwo
   
   {\begin{myIndent}\hThree
      Proof:\\
      Suppose for the sake of contradiction that there exists a linearly\\ independent set $Q = \{\mVecAst{y}_1, \ldots, \mVecAst{y}_{k+1}\} \subset X$ which spans $X$. Then,\\ define $S_0 = \{\mVecAst{x}_1, \ldots, \mVecAst{x}_k\}$ and note that $S_0$ spans $X$.\retTwo

      Now by induction, assume for $i \in \{0, 1, \ldots, k-1\}$, that $S_i$ contains the\\ first $i$ vectors of $Q$ in addition to $k - i$ vectors of $S_0$, and that $\mySpan\hspace{0.2em}{S_i} = X$.\\
      Then since $S_i$ spans $X$, we know that $y_{i+1} \in X$ is in the span of $S_i$. So,\\ letting $\mVecAst{x}_{n_1}, \ldots, \mVecAst{x}_{n_{k-i}}$ be the elements from $S_0$ in $S_i$, we know that there\\ exists scalars $a_1, \ldots, a_{i+1}, b_1, \ldots, b_{k-i} \in \mathbb{R}$ where $a_{i+1} = 1$ such that:

      {\centering $\sum\limits_{j=1}^{i+1}a_j\mVecAst{y}_j +  \sum\limits_{j=1}^{k-i}b_j\mVecAst{x}_{n_j} = \mVec{0}$ \retTwo\par}

      If all $b_j = 0$, then we have a contradiction. This is because $\{\mVecAst{y}_1, \ldots, \mVecAst{y}_{k+1}\}$\\ is assumed to be linearly independent. So, having all $b_j = 0$ implies that:
      
      {\centering $\sum\limits_{j=1}^{i+1}a_j\mVecAst{y}_j = \sum\limits_{j=1}^{i+1}a_j\mVecAst{y}_j + \sum\limits_{j=i+2}^{k+1}0\cdot\mVecAst{y}_j = \mVec{0}$\par}
      
      In turn this means that all $a_j = 0$, which contradicts that $a_{i+1} = 1$.\retTwo

      So, not all $b_j = 0$. This means that for some $j$ we must have that $\mVecAst{x}_{n_{j}}$ is in\\ the span of $(S_i \setminus \{\mVecAst{x}_{n_{j}}\}) \cup \{\mVecAst{y}_{i+1}\}$. Call this set $S_{i+1}$. Clearly, $S_{i+1}$ contains\\ the first $i + 1$ vectors of $Q$. Also:
      
      {\centering$\mySpan\hspace{0.2em}{S_{i+1}} = \mySpan\hspace{0.2em}(S_{i} \cup \{\mVecAst{y}_{i+1}\}) = \mySpan\hspace{0.2em}S_i = X$.\retTwo\par}

      So $S_{i+1}$ satisfies the same conditions $S_i$ did.\retTwo

      Now we get to the contradiction. Using the above reasoning, we will\\ eventually construct $S_k = \{\mVecAst{y}_1, \ldots, \mVecAst{y}_k\}$ which still spans $X$. However, since\\ $\mVecAst{y}_{k+1} \in X$, that means that $\mVecAst{y}_{k+1}$ equals some linear combination of the other\\ $\mVec{y}$ in $Q$. This contradicts that $Q$ is linearly independent. $\blacksquare$\retTwo
   \end{myIndent}}

   \uuline{Corollary}: If $B = \{\mVecAst{x}_1, \ldots, \mVecAst{x}_k\}$ is a basis for $X$, then $\myDim{X} = k$.\retTwo

   {\begin{myIndent}\hThree
      Proof:\\
      Since $B$ is linearly independent, by definition $\myDim{X} \geq k$. Meanwhile,\\ since $B$ spans $X$, we know by the above theorem that $\myDim{X} \leq k$.\\ So $\myDim{X} = k$.
   \end{myIndent}}

   \newpage

   \uuline{Theorem 9.3}: Suppose $X$ is a vector space and $\myDim{X} = n$. Then:
   \begin{itemize}
      \item[(A)] For $E = \{\mVecAst{x}_1, \ldots, \mVecAst{x}_n\} \subset X$, we have that $X = \mySpan\hspace{0.2em}E$ implies that $E$ is\\ linearly independent.
      
      {\begin{myIndent}\hThree
         Proof:\\
         First, assume $E$ is linearly independent. Then, note that for any\\ $\mVecAst{y} \in X$, we must have that $E \cup \{\mVecAst{y}^{\phantom{|}}\}$ is linearly dependent because\\ $|E \cup \{\mVecAst{y}^{\phantom{|}}\}| > \myDim{X}$. So, there exists $c_1, \ldots, c_n, c_{n+1} \in \mathbb{R}$ such\\ that at least one $c_i$ is nonzero and:

         {\centering $\sum\limits_{i=1}^n{c_i\mVecAst{x}_i} + c_{n+1}\mVecAst{y} = \mVec{0}$\retTwo\par}

         Now if $c_{n+1} = 0$, we have a contradiction because $E$ is linearly\\ independent. So, we conclude that $c_{n+1} \neq 0$. Thus, by rearranging\\ terms we can express $y$ as a linear combination of the vectors of $E$.\\ Therefore, $\mySpan\hspace{0.2em}E = X$ since $y$ can be any vector in $X$.\retTwo

         Secondly, assume $E$ is not linearly independent. Then for some $\mVecAst{x}_i \in E$, we have that $\mySpan\hspace{0.2em}E = \mySpan(E \setminus \{\mVecAst{x}_i\})$. However, $|E \setminus \{\mVecAst{x}_i\}| = n-1$.\\ So if $X = \mySpan\hspace{0.2em}E$, then $\myDim{X} \leq |E \setminus \{\mVecAst{x}_i\}| = n-1$, which contradicts\\ our assumption that $\myDim{X} = n$. Hence, $X \neq \mySpan\hspace{0.2em}E$.\\
      \end{myIndent}}

      \item[(B)] $X$ has a basis and every basis of $X$ consists of $n$ vectors.
      {\begin{myIndent}\hThree
         Proof:\\
         By the definition of $\myDim{X}$, we know that there exists a linearly\\ independent set of $n$ vectors. By the previous part of this theorem,\\ we also know that that set spans $X$. So, it is a basis of $X$. Meanwhile, by\\ the corollary to theorem 9.2, we know that the number of vectors in a\\ basis of $X$ equals the dimension of $X$. Hence, all bases of $X$ must have $n$ vectors.\\
      \end{myIndent}}

      \item[(C)] If $1 \leq m \leq n$ and $\{\mVecAst{y}_1, \ldots, \mVecAst{y}_m\} \subset X$ is linearly independent, then $X$ has a\\ basis that contains $\mVecAst{y}_1, \ldots, \mVecAst{y}_m$.
      {\begin{myIndent}\hThree
         Proof:\\
         Let $S_0 = \{\mVecAst{x}_1, \ldots, \mVecAst{x}_n\}$ be a basis of $X$ and $Q = \{\mVecAst{y}_1, \ldots, \mVecAst{y}_m\}$. Then\\ by the same induction which we used to prove theorem 9.2, we can\\ construct a basis:  $S_m$, of $X$ which contains $\mVecAst{y}_1, \ldots, \mVecAst{y}_m$.
      \end{myIndent}}
   \end{itemize}
\end{myIndent}}

\mySepTwo

Let $X$ and $Y$ be vector spaces. A map $\mMat{A}: X \longrightarrow Y$ is \udefine{linear} if\\ $\mMat{A}(c_1\mVecAst{x}_1 + c_2\mVecAst{x}_2) = c_1\mMat{A}(\mVecAst{x}_1) + c_2\mMat{A}(\mVecAst{x}_2)$ for all $\mVecAst{x}_1, \mVecAst{x}_2 \in X$ and $c_1, c_2 \in \mathbb{R}$.

\newpage

{\begin{myIndent}\exOne
   Observations:
   \begin{enumerate}
      \item A linear map sends $\mVec{0}$ to $\mVec{0}$. This is because: 
      
      {\centering$\mMat{A}(\mVec{0}) = \mMat{A}(\mVec{v} - \mVec{v}) = \mMat{A}(\mVec{v}) - \mMat{A}(\mVec{v}) = \mVec{0}$.\retTwo\par}

      \item If $\mMat{A}: X \longrightarrow Y$ is a linear map and $B = \{\mVecAst{x}_1, \ldots, \mVecAst{x}_k\}$ is a basis of $X$,\\
      then $\mMat{A}\left(\sum\limits_{i=1}^k(c_i\mVecAst{x}_i)\right) = \sum\limits_{i=1}^kc_i\mMat{A}(\mVecAst{x}_i)$ for all $c_1, \ldots, c_k \in \mathbb{R}$.\retTwo
   \end{enumerate}
\end{myIndent}}

Given two vector spaces $X$ and $Y$, we define $L(X, Y)$ to be the set of all linear transformations from $X$ into $Y$. Also, we shall abbreviate $L(X, X)$ as $L(X)$.\retTwo

$\nullSp{\mMat{A}} = $ "\udefine{null space} of $\mMat{A}$" $ = \{\mVec{x} \in X \mid \mMat{A}(\mVec{x}) = \mVec{0}\}$.\retTwo

$\rangeSp{\mMat{A}} =$ "range of $\mMat{A}$" $ = \{\mVec{y} \in Y \mid \exists \mVec{x} \in X \suchthat \mMat{A}\mVec{x} = \mVec{y}\}$.\retTwo

{\begin{myIndent}\hTwo
   \uuline{Proposition}: For any linear map $\mMat{A}: X \longrightarrow Y$, $\nullSp{\mMat{A}}$ and $\rangeSp{\mMat{A}}$ are vector spaces.\\
   \begin{myIndent}\hThree
      Proof:
      \begin{itemize}
         \item Assume $\mVecAst{x}_1, \mVecAst{x}_2 \in \nullSp{\mMat{A}} \subset X$ and $c \in \mathbb{R}$. Then:
         \begin{itemize}
            \item[$\circ$] $\mMat{A}(\mVecAst{x}_1 + \mVecAst{x}_2) = \mMat{A}(\mVecAst{x}_1) + \mMat{A}(\mVecAst{x}_2) = \mVec{0} + \mVec{0} = \mVec{0}$, which means\\ that $\mVecAst{x}_1 + \mVecAst{x}_2 \in \nullSp{\mMat{A}}$.
            
            \item[$\circ$] $\mMat{A}(c\mVecAst{x}_1) = c\mMat{A}(\mVecAst{x}_1) = c\mVec{0} = \mVec{0}$. So $c\mVecAst{x}_1 \in \nullSp{\mMat{A}}$.
         \end{itemize}
         This shows that $\nullSp{\mMat{A}}$ is a vector space.\retTwo

         \item Assume $\mVecAst{y}_1, \mVecAst{y}_2 \in \rangeSp{\mMat{A}} \subset Y$ and $c \in \mathbb{R}$. Then:
         \begin{itemize}
            \item[$\circ$] We know there exists $\mVecAst{x}_1, \mVecAst{x}_2 \in X$ such that $\mMat{A}(\mVecAst{x}_1) = \mVecAst{y}_1$ and\\ $\mMat{A}(\mVecAst{x}_2) = \mVecAst{y}_2$. In turn, $\mMat{A}(\mVecAst{x}_1 + \mVecAst{x}_2) = \mMat{A}(\mVecAst{x}_1) + \mMat{A}(\mVecAst{x}_2) = \mVecAst{y}_1 + \mVecAst{y}_2$.\\
            So $\mVecAst{y}_1 + \mVecAst{y}_2 \in \rangeSp{\mMat{A}}$.

            \item[$\circ$] Now continue letting $\mVecAst{x}_1 \in X$ be a vector such that $\mMat{A}(\mVecAst{x}_1) = \mVecAst{y}_1$.\\ Then $\mMat{A}(c\mVecAst{x}_1) = c\mMat{A}(\mVecAst{x}_1) = c\mVecAst{y}_1$. So $c\mVecAst{y}_1 \in \rangeSp{\mMat{A}}$.
         \end{itemize}
         This shows that $\rangeSp{\mMat{A}}$ is a vector space. \retTwo
      \end{itemize}
   \end{myIndent}
\end{myIndent}}

$\rank{\mMat{A}} = $ "\udefine{rank} of $\mMat{A}$" $ = \myDim{\rangeSp{\mMat{A}}}$.\retTwo

$\nullity{\mMat{A}} = $ "\udefine{nullity} of $\mMat{A}$" $ = \myDim{\nullSp{\mMat{A}}}$.\retTwo

{\begin{myIndent}\hTwo
   \uuline{Rank-Nullity Theorem}: Given any $\mMat{A} \in L(X, Y)$, we have that
   
   {\centering $\myDim{X} = \rank{\mMat{A}} + \nullity{\mMat{A}}$.\retTwo\par}

   {\begin{myIndent} \hThree
      Proof:\\
      
   \end{myIndent}}
\end{myIndent}}


\end{document}

