% Note for any github stalkers. I am currently in the process
% of learning LaTeX. I don't know what I'm doing yet. Sorry
% if my code absolutely sucks.


\documentclass{book}

\usepackage{fontspec} % used to import Calibri
\usepackage{anyfontsize} % used to adjust font size

% needed for inch and other length measurements
% to be recognized
\usepackage{calc}

% for colors and text effects as is hopefully obvious
\usepackage[dvipsnames]{xcolor}
\usepackage{soul}

% control over margins
\usepackage[margin=1in]{geometry}
\usepackage[strict]{changepage}

\usepackage{mathtools}
\usepackage{amsfonts}
\usepackage{amssymb} % originally imported to get the proof square

\setmainfont{Calibri}
\setlength{\parindent}{0pt}
\definecolor{RawerSienna}{HTML}{945D27}

\newcommand{\hOne}{%
   \color{Black}%
   \fontsize{14}{14}\selectfont%
}
\newcommand{\hTwo}{%
   \color{MidnightBlue}%
   \fontsize{13}{13}%
}
\newcommand{\hThree}{%
   \color{PineGreen}
   \fontsize{13}{13}
}
\newcommand{\myComment}{%
   \color{RawerSienna}%
   \fontsize{12}{12}%
}

\newenvironment{myIndent}{%
   \begin{adjustwidth}{2.5em}{0em}%
}{%
   \end{adjustwidth}%
}

\newcommand{\udefine}[1]{%
   \setulcolor{Red}%
   \setul{0.1ex}{0.15ex}%
   \ul{#1}%
}

\newcounter{LectureNumber}
\newcommand*{\markLecture}[1]{%
   \stepcounter{LectureNumber}%
   {\huge \color{Black} \textbf{Lecture \theLectureNumber: #1} \newline}%
}

\newcommand{\pprime}{\prime\prime}

\newcounter{PropNumber}
\newcommand{\propCount}{%
   \stepcounter{PropNumber}%
   \thePropNumber%
}



\title{Math 140A Lecture Notes (Professor: Brandon Seward)}
\author{Isabelle Mills}


\begin{document}
   \maketitle

   \markLecture{1/8/2024}

   \hOne
   An \udefine{order} on a set $S$, typically denoted as $<$, is
   a binary relation satisfying:
   \begin{enumerate}
      \item $\forall x, y \in S$, exactly one of the following is true:
      \begin{itemize}
         \item $x<y$ \item $x=y$ \item $y<x$
      \end{itemize}

      \item given $x, y, z \in S$, we have that $x<y<z\Rightarrow x<z$
   \end{enumerate}
   \hfill \bigbreak

   As a shorthand, we will specify that
   \begin{itemize}
      \item $x>y \Leftrightarrow y<x$
      \item $x\leq y \Leftrightarrow x<y \text{ or } x=y$
      \item $x\geq y \Leftrightarrow x>y \text{ or } x=y$
   \end{itemize}

   An \udefine{ordered set} is a set with a specified ordering. Let
   $S$ be an ordered set and $E$ be a nonempty subset of $S$. If
   $b \in S$ has the property that $\forall x \in E, \hspace{0.25em}
   x \leq b$, then we call $b$ an \udefine{upperbound} to $E$ and 
   say that $E$ is \udefine{bounded above} by $b$. Similarly, if $b$ 
   has the property that $\forall x \in E, \hspace{0.25em}
   x \geq b$, then we call $b$ an \udefine{lower bound} to $E$ and 
   say that $E$ is \udefine{bounded below} by $b$.
   \hfill \bigbreak

   We call $\beta \in S$ the \udefine{least upperbound} to $E$ if
   $\beta$ is an upper bound to $E$ and $\beta$ is the least of all
   upperbounds to $E$. In this case, we also commonly call $\beta$ 
   the \udefine{supremum} of $E$ and denote it as $\sup{E}$.

\end{document}