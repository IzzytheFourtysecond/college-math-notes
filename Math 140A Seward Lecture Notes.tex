% Note for any github stalkers. I am currently in the process
% of learning LaTeX. I don't know what I'm doing yet. Sorry
% if my code absolutely sucks.

% Additional note for gihub stalkers. I'm sorry I haven't
% learned how to split this into multiple files yet. Sorry.


\documentclass{book}

\usepackage{fontspec} % used to import Calibri
\usepackage{anyfontsize} % used to adjust font size

% needed for inch and other length measurements
% to be recognized
\usepackage{calc}

% for colors and text effects as is hopefully obvious
\usepackage[dvipsnames]{xcolor}
\usepackage{soul}

% control over margins
\usepackage[margin=1in]{geometry}
\usepackage[strict]{changepage}

\usepackage{mathtools}
\usepackage{amsfonts}
\usepackage{amssymb} % originally imported to get the proof square
\usepackage{xfrac}

% Just am using this to get a dashed line in a table...
% Also you apparently want this to be inactive if you aren't
% using it because it slows compilation.
\usepackage{arydshln} \ADLinactivate 
\newenvironment{allowTableDashes}{\ADLactivate}{\ADLinactivate}

\usepackage{graphicx}
\graphicspath{{./140A_images/}}


\newfontfamily{\calibri}{Calibri}


\setlength{\parindent}{0pt}
\definecolor{RawerSienna}{HTML}{945D27}

\newcommand{\hOne}{%
   \color{Black}%
   \fontsize{14}{14}\selectfont%
}
\newcommand{\hTwo}{%
   \color{MidnightBlue}%
   \fontsize{13}{13}\selectfont%
}
\newcommand{\hThree}{%
   \color{PineGreen}
   \fontsize{13}{13}\selectfont%
}
\newcommand{\myComment}{%
   \color{RawerSienna}%
   \fontsize{12}{12}\selectfont%
}
\newcommand{\teachComment}{
   \color{Orange}%
   \fontsize{12}{12}\selectfont%
}
\newcommand{\exOne}{%
   \color{Purple}%
   \fontsize{14}{14}\selectfont%
}
\newcommand{\exP}{%
   \color{VioletRed}%
   \fontsize{12}{12}\selectfont%
}

\newenvironment{myIndent}{%
   \begin{adjustwidth}{2.5em}{0em}%
}{%
   \end{adjustwidth}%
}

\newcommand{\udefine}[1]{%
   \setulcolor{Red}%
   \setul{0.1ex}{0.15ex}%
   \ul{#1}%
}

\newcommand{\uuline}[2][.]{%
{\vphantom{a}\color{#1}%
\rlap{\rule[-0.18em]{\widthof{#2}}{0.06em}}%
\rlap{\rule[-0.32em]{\widthof{#2}}{0.06em}}}%
#2}

\newcommand{\fillInBlank}[2][.]{{%
   \color{#1}%
   \rule[-0.12em]{#2em}{0.06em}\rule[-0.12em]{#2em}{0.06em}%
   \rule[-0.12em]{#2em}{0.06em}
}}

\newcommand{\retTwo}{\hfill\bigbreak}

\newcounter{LectureNumber}
\newcommand*{\markLecture}[1]{%
   \stepcounter{LectureNumber}%
   {\huge \color{Black} \textbf{Lecture \theLectureNumber: #1} \newline}%
}

\newcommand{\pprime}{\prime\prime}


\newcounter{PropNumber}
\newcommand{\propCount}[1][1]{
   \addtocounter{PropNumber}{#1}%
   \thePropNumber%
}
\newcounter{SubPropNumber}
\newcommand{\subPropCount}[1][1]{%
   \addtocounter{SubPropNumber}{1}%
   \theSubPropNumber%
}
\newcommand{\resetSubPropCount}{%
   \setcounter{SubPropNumber}{0}%
}


\newcommand{\mySep}[1]{%
   {\noindent\color{#1}{\rule{6.5in}{1mm}}}\\%
}


\title{Math 140A Lecture Notes (Professor: Brandon Seward)}
\author{Isabelle Mills}


\begin{document}
   \maketitle
   \calibri

   \markLecture{1/8/2024}

   \hOne
   An \udefine{order} on a set $S$, typically denoted as $<$, is
   a binary relation satisfying:
   \begin{enumerate}
      \item $\forall x, y \in S$, exactly one of the following is true:
      \begin{itemize}
         \item $x<y$ \item $x=y$ \item $y<x$
      \end{itemize}

      \item given $x, y, z \in S$, we have that $x<y<z\Rightarrow x<z$
   \end{enumerate}
   \hfill \bigbreak

   As a shorthand, we will specify that
   \begin{itemize}
      \item $x>y \Leftrightarrow y<x$
      \item $x\leq y \Leftrightarrow x<y \text{ or } x=y$
      \item $x\geq y \Leftrightarrow x>y \text{ or } x=y$
   \end{itemize}

   An \udefine{ordered set} is a set with a specified ordering. Let
   $S$ be an ordered set and $E$ be a nonempty subset of $S$. 

   \hTwo
   \begin{myIndent}
   \begin{itemize}
      \item If $b \in S$ has the property that $\forall x \in E, \hspace{0.25em}
      x \leq b$, then we call $b$ an\\ \udefine{upperbound} to $E$ and 
      say that $E$ is \udefine{bounded above} by $b$.
      \hfill \bigbreak
      
      \item if $b \in S$ has the property that $\forall x \in E, 
      \hspace{0.25em} x \geq b$, then we call $b$ an 
      \udefine{lower bound} to $E$ and say that $E$ is 
      \udefine{bounded below} by $b$.
      \hfill \bigbreak
   
      \item We call $\beta \in S$ the \udefine{least upperbound} to $E$ if
      $\beta$ is an upper bound to $E$ and $\beta$ is the least of all
      upperbounds to $E$. In this case, we also commonly call $\beta$ 
      the \udefine{supremum} of $E$ and denote it as $\sup{E}$.
      \hfill \bigbreak
   
      \item We call $\beta \in S$ the \udefine{greatest lower bound} to $E$ if
      $\beta$ is an lower bound to $E$ and $\beta$ is the greatest of all
      lower bounds to $E$. In this case, we also commonly call $\beta$ 
      the \udefine{infimum} of $E$ and denote it as $\inf{E}$.
      \hfill \bigbreak

      \item We call $e \in E$ the \udefine{maximum} of E if $\forall 
      x \in E, \hspace{0.25em} x \leq e$
      \hfill \bigbreak

      \item We call $e \in E$ the \udefine{minimum} of E if $\forall 
      x \in E, \hspace{0.25em} x \geq e$
      \hfill \bigbreak
   \end{itemize}
   \end{myIndent}

   \hOne
   \uuline{Fact}: For an ordered set $S$ and nonempty $E\subseteq S$,
   either:
   \begin{itemize}
      \item neither $\max{E}$ nor $\sup{E}$ exists
      \item $\sup{E}$ exists but $\max{E}$ does not exist
      \item $\max{E}$ exists and $\sup{E}=\max{E}$
   \end{itemize}

   \pagebreak
   \exOne
   Using $\mathbb{Q}$ as our ordered set...
   \begin{itemize}
      \item For $E = \{q \in \mathbb{Q} \mid 0<q<1\}$,
      $\max{E}$ does not exist but $\sup{E}$ exists and equals $1$.
      \exP
      \begin{myIndent}
         To understand why, note that the set of all upper bounds
         of $E$ is equal to\\ $\{q \in \mathbb{Q} \mid q \geq 1\}$ and 
         $1$ is obviously the smallest element of that set. Thus, $1$
         is the supremum of $E$. However, $1 \notin E$. Thus, if
         $\max{E}$ did exist, it would have to not equal $1$. But that
         would contradict $1$ being the least greatest bound.
      \end{myIndent}
      \hfill \bigbreak

      \exOne
      \item For $E = \{q \in \mathbb{Q} \mid 0<q \leq 1\}$,
      $\max{E}$ and $\sup{E}$ exist and they both are equal to $1$
      \exP
      \begin{myIndent}
         The reasoning for this is similar to that for the previous set.
      \end{myIndent}
      \hfill \bigbreak
      
      \exOne
      \item For $E = \{q \in \mathbb{Q} \mid q^2 < 2\}$, neither
      $\max{E}$ and $\sup{E}$ exist.
      \exP
      \begin{myIndent}
         To prove this, we can show that there exists a function $f: 
         \mathbb{Q}^+ \rightarrow \mathbb{Q}^+$ such that
         $\forall q \in \mathbb{Q}^+$, we have that $q^2 < 2
         \Rightarrow q^2 < (f(q))^2 < 2 \text{ and } 2 < q^2
         \Rightarrow 2 < (f(q))^2 < q^2$. Thus, we can show that
         the set of upper bounds to $E$ has no minimum element (meaning
         $\sup{E}$ is undefined) and $E$ itself has no maximum element.
         \hfill \bigbreak


         Now instead of being like Rudin and simply providing the desired
         function, I want to present how one may come up with a function 
         that works for this proof themselves.
         \hfill \bigbreak

         Firstly, note that for the following reasons, we know our 
         desired function must be a rational function:
         \begin{itemize}
            \item[$\diamond$] $\forall q \in \mathbb{Q}, \hspace{0.25em} 
            f(q) \in \mathbb{Q}$. Based on this, we can't use any radicals,
            trig functions, logarithms, or exponentials in our desired
            function. \hfill \bigbreak

            \item[$\diamond$] $q^2 > 2 \Rightarrow f(q) < q$. In other
            words, $f$ needs to grow slower than a linear function. Thus,
            we can rule out the possibility of $f$ being a polynomial.
            \hfill \bigbreak

            \item[$\diamond$] If we wanted $f$ to be a linear function,
            it would have to have the form\\ $f(q) = \alpha(q - \sqrt{2})
            + \sqrt{2}$ where $\alpha$ is some constant. This is
            because when $q^2 = 2, \hspace{0.25em} f(q) = q$. However, there
            is no value one can set $\alpha$ to which both eliminates the
            presence of irrational numbers in that function while
            simultaneously making $f(q) \neq q$ when $q^2 \neq 2$. So
            no linear function can possibly work for this proof.
            \hfill \bigbreak
         \end{itemize}

         Having narrowed our search, let's now pick some convenient
         properties we would wish our proof function to have. Specifically,
         let's force $f$ to be constantly increasing, have a $y$-intercept
         of $1$, and approach a horizontal asymptote of $y = 2$. Doing this,
         we can now say that an acceptable function will have the following
         form where $\alpha$ is an unknown constant:
         \[f(q) = 1 + \frac{q}{q + \alpha}\]
         
         And finally, we can solve for $\alpha$ using the following system
         of equations:
         \[\begin{matrix}
            (1 + \frac{q}{q + \alpha})^2 = 2 \\
            \\
            1 + \frac{q}{q + \alpha} = q
         \end{matrix}\]

         Now here's where a graphing calculator like Desmos can be very
         useful. Instead of painstakely having to solve for $\alpha$,
         we can use a graphing calculator to approximate the value of
         $\alpha$ that satisfies our system of equations.
         \begin{center}
         \includegraphics[scale=0.75]{Finding_Equation_Demonstration_1.png}
         \end{center}

         Based on the graph above, it looks like $f(q) = 1 +
         \frac{q}{q + 2}$ will work for our proof. And sure enough 
         it does. Furthermore, we can verify that the function we came
         up with is equivalent to that which Rudin presents.
      \end{myIndent}
   \end{itemize}

   \mySep{Purple}
   \hOne\hfill \break
   We say an ordered set $S$ has the \udefine{least upperbound
   property} if and only if when $E \subseteq S$ is nonempty and 
   bounded above, then the supremum of $E$ exists in $S$. Additionally, 
   we say an ordered set $S$ has the \udefine{greatest lower bound 
   property} if and only if when $E \subseteq S$ is nonempty and 
   bounded below, then the infimum of $E$ exists in $S$.
   
   \begin{myIndent}\begin{myIndent}\begin{myIndent}
   \begin{myIndent}\begin{myIndent}
      \teachComment
      When we define the set of real numbers, this will be one of the
      fundamental properties of that set. \hfill \bigbreak
   \end{myIndent}\end{myIndent}\end{myIndent}
   \end{myIndent}\end{myIndent}

   \markLecture{1/10/2024}
   \begin{myIndent}
      \hTwo
      Proposition \propCount: $S$ has the least upperbound property if
      and only if $S$ has the greatest lower bound property.
      
      \hThree
      \begin{myIndent}
         Proof: Let's say we have an ordered set $S$
         \hfill \bigbreak

         Assume $S$ has the least upperbound property. Then, let
         $B \subseteq S$ be a nonempty subset which is bounded below.
         Additionally, let $A \subseteq S$ be the set of all lower bounds
         of $B$. \newpage
         
         \begin{myIndent}
            We know that $A \neq \emptyset$ because we assumed that $B$ is
            bounded below. Thus, at least one lower bound to $B$ exists and
            belongs to $A$. \\Additionally, because we assumed $B$ is nonempty,
            we can say that each $b \in B$ is an upper bound to $A$. Thus, $A$
            is bounded above. Because of these two facts, we can apply the
            greatest lower bound property to say that the supremum of $A$ 
            exists. \hfill \bigbreak
   
            Let's define $\alpha \coloneq \sup{A}$. With that, our goal is 
            now to show that\\ $\alpha = \inf{B}$. To do this, we need to 
            show firstly that $\alpha$ is a lower bound to $B$ and 
            secondly that it is greater than all other lower bounds of $B$.
            \retTwo
            \begin{enumerate}
               \item For each $b \in B$, we have that $b$ is an 
                  upperbound to $A$. And since $\alpha = \sup{A}$ is the
                  least upperbound to $A$, we must have that $\alpha \leq
                  b$. Thus $\alpha$ is a lower bound to $B$. \retTwo
               
               \item If $x \in S$ is a lower bound to $B$, then $x \in A$.
                  And since\\ $\alpha = \sup{A}$, $x \leq \alpha$. This shows
                  that $\alpha$ is greater than or equal to all other lower
                  bounds. \retTwo
            \end{enumerate}
            
            Hense, $\alpha$ is the infimum of $B$. And since we did this
            for a general $B \subseteq S$, we can thus say that $S$ has
            the greatest lower bound property. \retTwo
         \end{myIndent}

         Now we skipped doing the reverse direction proof because it
         is almost completely identical to the foward direction proof.
         However, just know that the above proposition is an
         \uuline{if and only if} statement. $\blacksquare$
      \end{myIndent}
   \end{myIndent}

   \mySep{PineGreen}
   \hOne\hfill \break
   A \udefine{field} is a set $F$ equipped with $2$ binary operations,
   denoted $+$ and $\cdot$, and containing two elements $0 \neq 1 \in F$
   satisfying the following conditions for all $x, y, z \in F$:

   \hTwo
   \begin{myIndent}
      
      \begin{tabular}{ l c }%
         {\large \textbullet} \hspace{1ex} Associativity: &
            $\begin{matrix} (x + y) + z = x + (y + z) \\ (x \cdot y) 
               \cdot z = x \cdot (y \cdot z) \end{matrix}$ \\ \\
         
         {\large \textbullet} \hspace{1ex} Commutativity: &
            $\begin{matrix} x+y=y+x\\x\cdot y = y \cdot x\end{matrix}$
               \\ \\
         
         {\large \textbullet} \hspace{1ex} Identity: &
            $\begin{matrix}0+x=x\\1\cdot x=x\end{matrix}$ \\ \\
         
         {\large \textbullet} \hspace{1ex} Inverses: &
            $\begin{matrix} \forall x\in F, \hspace{0.5em}
               \exists {-x} \in F \hspace{0.50em} s.t.
               \hspace{0.50em} x + {-x} = 0 \\ \forall x\neq 0\in F, 
               \hspace{0.5em} \exists {\displaystyle 
               \frac{1}{x}} \in F \hspace{0.50em} s.t. \hspace{0.50em} 
               x \cdot {\displaystyle \frac{1}{x}}=1\end{matrix}$ \\ \\
         
         {\large \textbullet} \hspace{1ex} Distributivity: &
            $x \cdot (y + z) = (x \cdot y) + (x \cdot z)$
      \end{tabular}
   \end{myIndent}

   \newpage
   \hOne
   We shall assign the following notation: 
   
   {
      \hTwo % Ok so I now know that lines on the table
            % at least by default have the same color as the
            % text color.
      \centering
      \renewcommand{\arraystretch}{2.2}
      \begin{allowTableDashes}
         \begin{tabular}{ c;{10pt/3pt}c }
         
            We write \fillInBlank{1} \hspace{0.75em} & 
               \hspace{0.75em} to mean \fillInBlank{1} \\ \hline
            
            $x-y$ & $x + {-y}$ \\ \hdashline[10pt/3pt]

            ${\displaystyle \frac{x}{y}}$ & $x \cdot 
                  {\displaystyle \frac{1}{y}}$\\[8pt] \hdashline[10pt/3pt]
            
            $2$ & $1 + 1$\\ \hdashline[10pt/3pt]

            $2x$ & $x + x$\\ \hdashline[10pt/3pt]

            $x^2$ & $x \cdot x$
         
         \end{tabular}
      \end{allowTableDashes}
      \par
   }
   \retTwo
   
   \hOne
   Now what follows is a number of propositions concerning the 
   arithmetic properties of a field...
   
   \begin{myIndent}
      \hTwo
      \resetSubPropCount
      For a field $F$ and elements $x,y,z\in F$, we have the
      following propositions:
      
      Proposition \propCount.\subPropCount: $x+y=x+z \Rightarrow y=z$
      {\hThree
      \begin{myIndent}
         Proof: Assume $x+y=x+z$. Then...
            \begin{myIndent}
               \renewcommand{\arraystretch}{1.4}
               \begin{tabular}{ l c }
                  $y=0+y$ & (addition identity property)\\
                  $\hphantom{y}=({-x}+x)+y$ & (addition inverse property)\\
                  $\hphantom{y}={-x}+(x+y)$ & (addition associative property)\\
                  $\hphantom{y}={-x}+(x+z)$ & (by our assumption)\\
                  $\hphantom{y}=({-x}+x)+z$ & (addition associative property)\\
                  $\hphantom{y}=0+z$ & (addition inverse property)\\
                  $\hphantom{y}=z$ & (addition identity property)
               \end{tabular}
            \end{myIndent}
      \end{myIndent} \retTwo}

      Proposition \propCount[0].\subPropCount: $x+y=x \Rightarrow y=0$
      {\hThree
      \begin{myIndent}
         Proof: Plug in $z=0$ into proposition 2.1. in order to get
         that $y=z=0$.
      \end{myIndent} \retTwo}

      Proposition \propCount[0].\subPropCount: $x+y=0 \Rightarrow y={-x}$
      {\hThree
      \begin{myIndent}
         Proof: Plug in $z={-x}$ into proposition 2.1. in order to get
         that\\ $y=z={-x}$.
      \end{myIndent} \retTwo}

      Proposition \propCount[0].\subPropCount: ${-(-x)}=x$
      {\hThree
      \begin{myIndent}
         Proof: Observe that $x+{-x}={-x}+x=0$ by the inverse and\\ 
         commutative properties of addition. Then, by proposition
         2.3, we know that ${-x}+x=0 \Rightarrow x={-(-x)}$.
      \end{myIndent}}
      \newpage
      Proposition \propCount[0].\subPropCount: $x\cdot y=x\cdot z 
                                    \text{ and } x\neq0\Rightarrow y=z$
      {\hThree
      \begin{myIndent}
         Proof: Assume $x\cdot y=x\cdot z$ and $x\neq0$. Then...
            \begin{myIndent}
               \renewcommand{\arraystretch}{1.4}
               \begin{tabular}{ l c }
                  $y=1\cdot y$ & (multiplication identity property)\\
                  $\hphantom{y}=(\frac{1}{x}\cdot x)\cdot y$ & 
                                 (multiplication inverse property)\\
                  $\hphantom{y}=\frac{1}{x}\cdot(x\cdot y)$ & 
                                 (multiplication associative property)\\
                  $\hphantom{y}=\frac{1}{x}\cdot(x\cdot z)$ & 
                                 (by our assumption)\\
                  $\hphantom{y}=(\frac{1}{x}\cdot x)\cdot z$ & 
                                 (multiplication associative property)\\
                  $\hphantom{y}=1\cdot z$ & 
                                 (multiplication inverse property)\\
                  $\hphantom{y}=z$ & 
                                 (multiplication identity property)
               \end{tabular}
               \retTwo
            
         \begin{myIndent}\begin{myIndent}\begin{myIndent}
            \teachComment
            Note that to use the multiplication inverse\\ property,
            we have to assume $x \neq 0$ !!
         \end{myIndent}\end{myIndent}\end{myIndent}

         \end{myIndent}
      \end{myIndent} \retTwo}

      Proposition \propCount[0].\subPropCount: $x\cdot y=x \Rightarrow y=1$
      {\hThree
      \begin{myIndent}
         Proof: Plug in $z=1$ into proposition 2.5. in order to get
         that $y=z=1$.
      \end{myIndent} \retTwo}

      Proposition \propCount[0].\subPropCount: $x\cdot y=1 \Rightarrow 
               y={\displaystyle\frac{1}{x}}$
      {\hThree
      \begin{myIndent}
         Proof: Plug in $z={\displaystyle\frac{1}{x}}$ into proposition 2.5. in order to get
         that\\ $y=z={\displaystyle\frac{1}{x}}$.
      \end{myIndent} \retTwo}

      Proposition \propCount[0].\subPropCount:
         ${\displaystyle \frac{\hspace{0.15em}1\hspace{0.15em}}{
            \frac{1}{x}}} = x$
      {\hThree
      \begin{myIndent}
         Proof: Observe that $x\cdot{\displaystyle\frac{1}{x}}=
         {\displaystyle\frac{1}{x}}\cdot x=0$ 
         by the inverse and commutative \\[3pt] properties of multiplication. 
         Then, by proposition 2.7, we know that \\[3pt]
         ${\displaystyle\frac{1}{x}}\cdot x=0 \Rightarrow x={\displaystyle 
         \frac{\hspace{0.15em}1\hspace{0.15em}}{\frac{1}{x}}}$.
      \end{myIndent} \retTwo}

      Proposition \propCount[0].\subPropCount: $0\cdot x = 0$
      {\hThree
      \begin{myIndent}
         Proof: $(0\cdot x) + (0\cdot x) = (0+0)\cdot x = 0\cdot x$.
         Thus we have an expression of the form $a+b=a$ which we can use
         proposition 2.2 on. Hence, we can conclude $0\cdot x=0$.
      \end{myIndent} \retTwo}

      Proposition \propCount[0].\subPropCount: 
         $x\neq0 \text{ and } y\neq0 \Rightarrow x\cdot y \neq 0$
      {\hThree
      \begin{myIndent}
         Proof: since $x, y\neq0$, we can say that $x\cdot y \cdot
         {\displaystyle\frac{1}{x}} \cdot {\displaystyle\frac{1}{y}}
         = 1 \neq 0$. Now by proposition 2.9, $x\cdot y=0 \Rightarrow
         (x\cdot y)\cdot({\displaystyle\frac{1}{x}} \cdot 
         {\displaystyle\frac{1}{y}})=0$. However, we know that is not the
         case. So $x\cdot y$ can't equal zero.
      \end{myIndent} \retTwo}
   \end{myIndent}
   \newpage
   
   \markLecture{1/12/2024}


\end{document}