\documentclass{book}

\usepackage{fontspec} % used to import Calibri
\usepackage{anyfontsize} % used to adjust font size

% needed for inch and other length measurements
% to be recognized
\usepackage{calc}

% for colors and text effects as is hopefully obvious
\usepackage[dvipsnames]{xcolor}
\usepackage{soul}

% control over margins
\usepackage[margin=1in]{geometry}
\usepackage[strict]{changepage}

\usepackage{mathtools}
\usepackage{amsfonts}
\usepackage{bm}

\usepackage[scr=rsfso, scrscaled=.96]{mathalpha}

% This is how I'm getting the nice caligraphy font :(
\DeclareMathAlphabet{\eulerscr}{U}{eus}{m}{n}
\newcommand{\mathcalli}[1]{\text{\scalebox{1.11}{$\eulerscr{#1}$}}}


\usepackage{amssymb} % originally imported to get the proof square
\usepackage{xfrac}
\usepackage[overcommands]{overarrows} % Get my preferred vector arrows...
\usepackage{relsize}

% Just am using this to get a dashed line in a table...
% Also you apparently want this to be inactive if you aren't
% using it because it slows compilation.
\usepackage{arydshln} \ADLinactivate 
\newenvironment{allowTableDashes}{\ADLactivate}{\ADLinactivate}

\usepackage{graphicx}
\graphicspath{{./158_Images/}}

\usepackage{tikz}
   \usetikzlibrary{arrows.meta}
   \usetikzlibrary{graphs, graphs.standard}

\usepackage{quiver} %commutative diagrams






\usepackage[hidelinks]{hyperref}
\newcommand{\inLinkRap}[2]{{\color{blue}\hyperlink{#1}{\textit{#2}}}}







\newfontfamily{\calibri}{Calibri}
\setlength{\parindent}{0pt}
\definecolor{RawerSienna}{HTML}{945D27}

% ~~~~~~~~~~~~~~~~~~~~~~~~~~~~~~~~~~~~~~~~~~~~~~~~~~
%Arrow Commands:

% Thank you Bernard, gernot, and Sigur who I copied this from:
% https://tex.stackexchange.com/questions/364096/command-for-longhookrightarrow
\renewcommand{\hookrightarrow}{\lhook\joinrel\rightarrow}
\renewcommand{\hookleftarrow}{\leftarrow\joinrel\rhook}
\newcommand{\hooklongrightarrow}{\lhook\joinrel\longrightarrow}
\newcommand{\hooklongleftarrow}{\longleftarrow\joinrel\rhook}
\newcommand{\hookxlongrightarrow}[2][]{\lhook\joinrel\xrightarrow[#1]{#2}}
\newcommand{\hookxlongleftarrow}[2][]{\xleftarrow[#1]{#2}\joinrel\rhook}

% Thank you egreg who I copied from:
% https://tex.stackexchange.com/questions/260554/two-headed-version-of-xrightarrow
\newcommand{\longrightarrowdbl}{\longrightarrow\mathrel{\mkern-14mu}\rightarrow}
\newcommand{\longleftarrowdbl}{\leftarrow\mathrel{\mkern-14mu}\longleftarrow}

\newcommand{\xrightarrowdbl}[2][]{%
  \xrightarrow[#1]{#2}\mathrel{\mkern-14mu}\rightarrow
}
\newcommand{\xleftarrowdbl}[2][]{%
  \leftarrow\mathrel{\mkern-14mu}\xleftarrow[#1]{#2}
}

\newcommand{\mRoman}[1]{%
   \textrm{\MakeUppercase{\romannumeral #1}}%
}



% ~~~~~~~~~~~~~~~~~~~~~~~~~~~~~~~~~~~~~~~~~~~~~~~~~~

\newcommand{\hOne}{%
   \color{Black}%
   \fontsize{14}{16}\selectfont%
}
\newcommand{\hTwo}{%
\color{Black}%
   \fontsize{13}{15}\selectfont%
}
% \newcommand{\scratchWork}{%
%    \color{PineGreen!85!Orange}
%    \fontsize{12}{14}\selectfont%
% }
\newcommand{\hThree}{%
   \color{Black}%
   \fontsize{12}{14}\selectfont%
}
\newcommand{\myComment}{%
   \color{RawerSienna}%
   \fontsize{12}{14}\selectfont%
}
\newcommand{\pracOne}{
   \color{BrickRed}%
   \fontsize{13}{15}\selectfont%
}
\newcommand{\pracTwo}{
   \color{Orange}%
   \fontsize{12}{14}\selectfont%
}
\newcommand{\why}{%
   \color{Orange}%
   \fontsize{12}{14}\selectfont%
	Why:
}
\newcommand{\exOne}{%
   \color{Purple}%
   \fontsize{14}{16}\selectfont%
}
\newcommand{\exTwo}{%
   \color{Purple}%
   \fontsize{13}{15}\selectfont%
}
\newcommand{\exThree}{%
   \color{Purple}%
   \fontsize{12}{14}\selectfont%
}
\newcommand{\exP}{%
   \color{Purple}%
   \fontsize{12}{14}\selectfont%
}
\newcommand{\exTwoP}{%
   \color{RedViolet}%
   \fontsize{13}{15}\selectfont%
}
\newcommand{\exThreeP}{%
   \color{RedViolet}%
   \fontsize{12}{14}\selectfont%
}
\newcommand{\exPP}{%
   \color{RedViolet}%
   \fontsize{12}{14}\selectfont%
}
\newcommand{\exPPP}{%
   \color{VioletRed}%
   \fontsize{12}{14}\selectfont%
}

% Homework standard below (God the bloat in the header is absurd...)
% ~~~~~~~~~~~~~~~~~~~~~~~~~~~~~~~~~~~~~~~~~~~~~~~~
\newcommand{\Hstatement}{%
   \color{MidnightBlue!90!Black}%
   \fontsize{12}{13}\selectfont%
}
\newcommand{\HexOne}{%
   \color{Purple}%
   \fontsize{12}{13}\selectfont%
}
\newcommand{\HexTwoP}{%
   \color{RedViolet}%
   \fontsize{12}{13}\selectfont%
}
\newcommand{\HexPPP}{%
   \color{VioletRed}%
   \fontsize{11}{12}\selectfont%
}

% ~~~~~~~~~~~~~~~~~~~~~~~~~~~~~~~~~~~~~~~~~~~~~~~~

\newcommand{\cyPen}[1]{{\vphantom{.}\color{Cerulean}#1}}
\newcommand{\redPen}[1]{{\vphantom{.}\color{Red}#1}}

\newenvironment{myIndent}{%
   \begin{adjustwidth}{2.5em}{0em}%
}{%
   \end{adjustwidth}%
}

\newenvironment{myDindent}{%
   \begin{adjustwidth}{5em}{0em}%
}{%
   \end{adjustwidth}%
}

\newenvironment{myTindent}{%
   \begin{adjustwidth}{7.5em}{0em}%
}{%
   \end{adjustwidth}%
}

\newenvironment{myConstrict}{%
   \begin{adjustwidth}{2.5em}{2.5em}%
}{%
   \end{adjustwidth}%
}

\newcommand{\udefine}[1]{{%
   \setulcolor{Red}%
   \setul{0.14em}{0.07em}%
   \ul{#1}%
}}

\newcommand{\uprop}[1]{{%
   \setulcolor{Purple}%
   \setul{0.14em}{0.07em}%
   \ul{#1} 
}}

\newcommand{\blab}[1]{\textbf{#1}}

\newcommand{\uuline}[2][.]{%
{\vphantom{a}\color{#1}%
\rlap{\rule[-0.18em]{\widthof{#2}}{0.06em}}%
\rlap{\rule[-0.32em]{\widthof{#2}}{0.06em}}}%
#2}

\newcommand{\pprime}{{\prime\prime}}
\newcommand{\suchthat}{ \hspace{0.3em}s.t.\hspace{0.3em}}
\newcommand{\rea}[1]{\mathrm{Re}(#1)}
\newcommand{\ima}[1]{\mathrm{Im}(#1)}
\newcommand{\comp}{\mathsf{C}}
\newcommand{\trans}{\mathsf{T}}
\newcommand{\myHS}{ \hspace{0.5em}}
\newcommand{\gap}{\phantom{2}}

\newcommand{\myId}{\mathrm{Id}}
\newcommand{\myIm}{\mathrm{im}}
\newcommand{\Obj}{\mathrm{Obj}}
\newcommand{\Hom}{\mathrm{Hom}}
\newcommand{\End}{\mathrm{End}}
\newcommand{\Aut}{\mathrm{Aut}}

\newcommand{\df}{\mathrm{d}}
\newcommand{\Df}{\mathrm{D}}

\newcommand{\mcateg}[1]{{\bm{\mathsf{#1}}}}

\newcommand{\mdeg}{\mathrm{mdeg}\phantom{.}}

\newcommand{\card}{\mathrm{card}}
\newcommand{\supp}{\mathrm{supp}}
\newcommand{\diam}{\mathrm{diam}}
\newcommand{\opnorm}{\mathrm{op}}
\newcommand{\sgn}{\mathrm{sgn}}
\newcommand{\mSpan}{\mathrm{span}}
\newcommand{\Interior}{\mathop{\mathrm{Int}}}

\newcommand{\mMat}[1]{\mathbf{#1}}

\newcommand{\NBV}{\ensuremath{\mathrm{NBV}}}
\newcommand{\Acc}{\mathrm{Acc}}
\newcommand{\BV}{\ensuremath{\mathrm{BV}}}
\newcommand{\Var}{\ensuremath{\mathrm{Var}}}

\newcommand{\Alt}{\mathrm{Alt}}
\newcommand{\Sym}{\mathrm{Sym}}

\newcommand{\weakst}{weak$^*$ }

\newcommand{\radtimes}{\mathop{\widehat{\times}}}

% Thank you Gonzalo Medina and Moriambar who wrote this on stack exchange:
%https://tex.stackexchange.com/questions/74125/how-do-i-put-text-over-symbols%
\newcommand{\myequiv}[1]{\stackrel{\mathclap{\mbox{\footnotesize{$#1$}}}}{\equiv}}

% Thank you chs who wrote this on stack exchange:
%https://tex.stackexchange.com/questions/89821/how-to-draw-a-solid-colored-circle%
\newcommand{\filledcirc}[1][.]{\ensuremath{\hspace{0.05em}{\color{#1}\bullet}\mathllap{\circ}\hspace{0.05em}}}

%Thank you blerbl who wrote this on stack exchange:
%https://tex.stackexchange.com/questions/25348/latex-symbol-for-does-not-divide
\newcommand{\ndiv}{\hspace{-0.3em}\not|\hspace{0.35em}}

\newcommand{\mySepOne}[1][.]{%
   {\noindent\color{#1}{\rule{6.5in}{1mm}}}\\%
}
\newcommand{\mySepTwo}[1][.]{%
   {\noindent\color{#1}{\rule{6.5in}{0.5mm}}}\\%
}
\newcommand{\mySepThree}[1][.]{%
   {\noindent\color{#1}{\rule{6in}{0.25mm}}}\\%
}

\newenvironment{myClosureOne}[2][.]{%
   \color{#1}%
   \begin{tabular}{|p{#2in}|} \hline \\%
}{%
   \\ \hline \end{tabular}%
}

\newcommand{\retTwo}{\hfill\bigbreak}

\newcommand{\dispDate}[1]{{
   \color{Black}%
   \fontsize{20}{18}\selectfont%
   #1\retTwo
}}


\title{Math Journal}
\author{Isabelle Mills}


\begin{document}
   \maketitle{}
   \setul{0.14em}{0.07em}
   \calibri\hOne
   
   \dispDate{8/31/2024}
   My goal for today is to work through the appendix to chapter 1 in Baby Rudin. This appendix focuses on constructing the real numbers using Dedikind cuts.\retTwo
   
   \hTwo
   \begin{myIndent}
      We define a \udefine{cut} to be a set $\alpha \subset \mathbb{Q}$ such that:
      \begin{enumerate}
         \item $\alpha \neq \emptyset$
         \item If $p \in \alpha$,\myHS $q \in \mathbb{Q}$, and $q < p$, then $q \in \alpha$.
         \item If $p \in \alpha$, then $p < r$ for some $r \in \alpha$\newline
      \end{enumerate}

      Point 3 tells us that $\alpha$ doesn't have a max element. Also, point 2 directly implies the following facts:
      \begin{itemize}
         \item[a.] If $p \in \alpha$,\myHS $q \in \mathbb{Q}$, and $q \notin \alpha$, then $q > p$.
         \item[b.] If $r \notin \alpha$,\myHS $r, s \in \mathbb{Q}$, and $r < s$, then $s \notin \alpha$.\newline
      \end{itemize}

      As a shorthand, I shall refer to the set of all cuts as $R$.
      \begin{myIndent}\myComment
         An example of a cut would be the set of rational numbers less than $2$.\\
      \end{myIndent}

      Firstly, we shall assign an ordering to $R$. Specifically, given any $\alpha, \beta \in R$, we say that $\alpha < \beta$ if $\alpha \subset \beta$ (a proper subset).

      \begin{myIndent}\exTwo
         Here we prove that $<$ satisfies the definition of an ordering.
         \begin{itemize}
            \item[\mRoman{1}.] It's obvious from the definition of a proper subset that at most one of the following three things can be true: $\alpha < \beta$,\myHS $\alpha = \beta$, and $\beta < \alpha$.\retTwo
            
            Now let's assume that $a \not< \beta$ and $\alpha \not= \beta$. Then $\exists p \in \alpha$ such that $p \notin \beta$. But then for any $q \in \beta$, we must have by fact b. above that $q < p$. Hence $q \in \alpha$, meaning that $\beta \subset \alpha$. This proves that at least one of the following has to be true: $\alpha < \beta$,\myHS $\alpha = \beta$, and $\beta < \alpha$.\retTwo

            \item[\mRoman{2}.] If for $\alpha, \beta, \gamma \in R$ we have that $\alpha < \beta$ and $\beta < \gamma$, then clearly $\alpha < \gamma$ becuase $\alpha \subset \beta \subset \gamma$.\retTwo
         \end{itemize}
      \end{myIndent}

      Now we claim that $R$ equipped with $<$ has the least-upper-bound property.
      \begin{myIndent}\exTwo
         Proof:\\
         Let $A \subset R$ be nonempty and $\beta \in R$ be an upper bound of $A$. Then set\\ $\gamma = \hspace{-0.2em}\bigcup\limits_{\alpha \in A}\hspace{-0.2em}\alpha$. Firstly, we want to show that $\gamma \in R$\retTwo

         Since $A \neq \emptyset$, there exists $\alpha_0 \in A$. And as $\alpha_0 \neq \emptyset$ and $\alpha_0 \subseteq \gamma$ by definition, we know that $\gamma \neq \emptyset$. At the same time, we know that $\gamma \subset \beta$ since $\forall \alpha \in A$,\myHS $\alpha \subset \beta$. Hence, $\gamma \neq \mathbb{Q}$, meaning that $\gamma$ satisfies property 1$.$ of cuts.\retTwo

         Next, let $p \in \gamma$ and $q \in \mathbb{Q}$ such that $q < p$. We know that for some $\alpha_1 \in A$, we have that $p \in \alpha_1$. Hence by property 2$.$ of cuts, we know that $q \in \alpha_1 \subset \gamma$, thus showing that $\gamma$ satisfies property 2$.$ of cuts.\newpage
         
         Thirdly, by property 3$.$ we can pick $r \in \alpha_1$ such that $p < r$ and $r \in \alpha_1 \subset \gamma$. So, $\gamma$ satisfies property 3$.$ of cuts.\retTwo

         With that, we've now shown that $\gamma \in R$. Clearly, $\gamma$ is an upper bound of $A$ since $\alpha \subset \gamma$ for all $\alpha \in A$. Meanwhile, consider any $\delta < \gamma$. Then $\exists s \in \gamma$ such that $s \notin \delta$. And since $s \in \gamma$, we know that $s \in \alpha$ for some $\alpha \in A$. Hence, $\delta < \alpha$, meaning that $\delta$ is not an upper bound of $A$. This shows that $\gamma = \sup A$.\\ [6pt]
      \end{myIndent}

      Secondly, we want to assign $+$ and $\hspace{0.1em}\cdot\hspace{0.1em}$ operations to $R$ so that $R$ is an ordered field.\retTwo
      
      To start, given any $\alpha, \beta \in R$, we shall define $\alpha + \beta$ to be the set of all sums $r + s$ such that $r \in \alpha$ and $s \in \beta$.
      \begin{myIndent}\exTwo
         Here we show that $\alpha + \beta \in R$.
         \begin{enumerate}
            \item Clearly, $\alpha + \beta \neq \emptyset$. Also, take $r^\prime \notin \alpha$ and $s^\prime \notin \beta$. Then $r^\prime + s^\prime > r + s$ for all $r \in \alpha$ and $s \in \beta$. Hence, $r^\prime + s^\prime \notin \alpha + \beta$, meaning that $\alpha + \beta \neq \mathbb{Q}$.\\ [-9pt]
         \end{enumerate}

         Now let $p \in \alpha + \beta$. Thus there exists $r \in \alpha$ and $s \in \beta$ such that $p = r + s$.\\ [-9pt]

         \begin{enumerate}
            \item[2.] Suppose $q < p$. Then $q - s < r$, meaning that $q - s \in \alpha$. Hence,\\ $q = (q - s) + s \in \alpha + \beta$.\retTwo
            
            \item[3.] Let $t \in \alpha$ so that $t > r$. Then $p = r + s < t + s$ and $t + s \in \alpha + \beta$.\retTwo
         \end{enumerate}
      \end{myIndent}

      Also, we shall define $0^*$ to be the set of all negative rational numbers. Clearly, $0^*$ is a cut. Furthermore, we claim that $+$ satisfies the addition requirements of a field with $0^*$ as its $0$ element.

      \begin{myIndent}\exTwo
         Commutativity and associativity of $+$ on $R$ follows directly from the\\ commutativity and associativity of addition on the rational numbers.\retTwo

         Also, for any $\alpha \in R$,\myHS $\alpha + 0^* = \alpha$.
         \begin{myIndent}\exP
            If $r \in \alpha$ and $s \in 0^*$, then $r + s < r$. Hence $r + s \in \alpha$, meaning that $\alpha + 0^* \subseteq \alpha$. Meanwhile, if $p \in \alpha$, then we can pick $r \in \alpha$ such that $r > p$. Then, $p - r \in 0^*$ and $p = r + (p - r) \in \alpha + 0^*$. So, $\alpha \subseteq \alpha + 0^*$.\retTwo
         \end{myIndent}

         Finally, given any $\alpha \in R$, let $\beta = \{p \in \mathbb{Q} \mid \exists\hspace{0.1em} r \in \mathbb{Q}^+ \suchthat -p-r \notin \alpha\}$.
         \begin{myIndent}\myComment
            To give some intuition on this definition, firstly we want to guarentee that for all $p \in \beta$, $-p$ is greater than all elements of $\alpha$. Secondly, we add the $-r$ term to guarentee that $\beta$ doesn't have a maximum element.\\
         \end{myIndent}

         We claim that $\beta \in R$ and $\beta + \alpha = 0^*$. Hence, we can define $-\alpha = \beta$.

         \begin{myIndent}\exP
            To start, we'll show that $\beta \in R$:
            \begin{enumerate}
               \item For $s \notin \alpha$ and $p = -s - 1$, we have that $-p - 1 \notin \alpha$. Hence, $p \in \beta$, meaning that $\beta \neq \emptyset$. Meanwhile, if $q \in \alpha$, then $-q \notin \beta$ because there does not exist $r > 0$ such that $-(-q) - r = q - r \notin \alpha$. So $\beta \neq \mathbb{Q}$.\\ [-6pt]
            \end{enumerate}

            Now let $p \in \beta$ and pick $r > 0$ such that $-p -r \notin \alpha$.\newpage

            \begin{enumerate}
               \item[2.] Suppose $q < p$. Then $-q - r > -p - r$, meaning that $-q - r \notin \alpha$. Hence, $q \in \beta$.\retTwo
               
               \item[3.] Let $t = p + \frac{r}{2}$. Then $t > p$ and $-t - \frac{r}{2} = -p - r \notin \alpha$, meaning $t \in \beta$.\retTwo
            \end{enumerate}

            Now that we've proved $\beta \in R$, we next prove that $\beta$ is the additive inverse of $\alpha$. To start, suppose $r \in \alpha$ and $s \in \beta$. Then $-s \notin \alpha$, meaning that $r < -s$. So $r + s < 0$, thus showing that $\alpha + \beta \subseteq 0^*$.\retTwo

            As for the other inclusion, pick any $v \in 0^*$ and set $w = -\frac{v}{2}$. Then because $w > 0$, we can use the archimedean property of $\mathbb{Q}$ to say that there exists $n \in \mathbb{Z}$ such that $nw \in \alpha$ but $(n+1)w \notin \alpha$. Put $p = -(n + 2)w$. Then $p \in \beta$ because $-p - w = (n+1)w
            \notin \alpha$. And finally, $v = nw + p \in \alpha + \beta$. Thus, $0^* \subseteq \alpha + \beta$.\retTwo
         \end{myIndent}
      \end{myIndent}
   \end{myIndent}
   \dispDate{9/1/2024}
   \begin{myIndent}\hTwo
      Based on the definition of $+$, it's also hopefully clear that for any $\alpha, \beta, \gamma \in R$ such that $\alpha < \beta$, we have that $\alpha + \gamma < \beta + \gamma$.\retTwo

      Next, we shall define multiplication on $R$. Except, first we're going to limit ourselves to the set $R^+$ of all cuts greater than $0^*$. So, given any $\alpha, \beta \in R^+$, we shall define $\alpha \beta$ to be the set of all $p \in \mathbb{Q}$ such that $p \leq rs$ where $r \in \alpha$,\myHS $s \in \beta$,\myHS $r > 0$, and $s > 0$.

      \begin{myIndent}\exTwo
         Here we show that $\alpha\beta \in R^+$.
         \begin{enumerate}
            \item Clearly $\alpha\beta \neq \emptyset$. Also, take any $r^\prime \notin \alpha$ and $s^\prime \notin \beta$. Then $r^\prime s^\prime > rs$ for all $r \in \alpha \cap \mathbb{Q}^+$ and $s \in \beta \cap \mathbb{Q}^+$ since all four rational numbers are positive. By extension, $r^\prime s^\prime$ is greater than all the elements (both positive and negative) of $\alpha\beta$. So, $r^\prime s^\prime \notin \alpha\beta$, meaning that $\alpha\beta \neq \mathbb{Q}$.\\ [-9pt]
         \end{enumerate}

         Now let $p \in \alpha\beta$. Based on our definition of $\alpha\beta$, we know that the conditions of a cut trivially hold for any negative $p$. So, we'll assume from now on that $p > 0$. (Also note that a positive choice of $p$ must exist because both $\alpha$ and $\beta$ by assumption have positive elements.)\retTwo

         Since $p \in \alpha\beta \cap \mathbb{Q}^+$, we know there exists $r \in \alpha$ and $s \in \beta$ such that $p = rs$ and $r, s > 0$.

         \begin{enumerate}
            \item[2.] Suppose $0 < q < p$ (the case where $q \leq 0$ is trivial). Then $\frac{q}{s} < r$, meaning that $\frac{q}{s} \in \alpha$. So, $q = \frac{q}{s} \cdot s \in \alpha\beta$.\retTwo
            
            \item[3.] Let $t \in \alpha$ so that $t > r$. Then $p = rs < ts$ and $ts \in \alpha\beta$.\retTwo
         \end{enumerate}
      \end{myIndent}

      Also, we shall define $1^*$ to be the set of all rational numbers less than $1$. Clearly, $1^*$ is a cut. And we claim that $\hspace{0.1em}\cdot\hspace{0.1em}$ satisfies the multiplication requirements of a field with $1^*$ as its $1$ element.\newpage

      \begin{myIndent}\exTwo
         As before, commutativity and associativity of $\hspace{0.1em}\cdot\hspace{0.1em}$ on $R^+$ follows directly from commutativity and associativity of multiplication on the rational numbers.\retTwo

         Next, for any $\alpha \in R^+$, we have that $\alpha 1^* = \alpha$.
         \begin{myIndent}\exP
            It's clear that for any rational number $r \leq 0$, we have that $r \in \alpha 1^*$ and $r \in \alpha$. So we can exclusively focus on positive rational numbers.\retTwo
            
            Now suppose $r \in \alpha \cap \mathbb{Q}^+$ and $s \in 1^*$. Then $rs < r$, meaning that $rs \in \alpha$. So $\alpha 1^* \subseteq \alpha$. Meanwhile, if $p \in \alpha \cap \mathbb{Q}^+$, then we can pick $r \in \alpha$ such that $r > p$. Then $\frac{p}{r} \in 1^*$ and $p = \frac{p}{r} \cdot r \in \alpha 1^*$. So, $\alpha \subseteq \alpha 1^*$.\retTwo
         \end{myIndent}

         Thirdly, given any $\alpha \in R^+$, define:

         \begin{centering}
            $\beta = \{p \in \mathbb{Q} \mid p \leq 0\} \cup \{p \in \mathbb{Q}^+ \mid \exists r \in \mathbb{Q}^+ \suchthat \frac{1}{q} - r \notin \alpha\}$\retTwo\par
         \end{centering}

         \begin{myIndent}\exP
            Here we show that $\beta \in R^+$.
            \begin{enumerate}
               \item Clearly $\beta \neq \emptyset$. Also, if $q \in \alpha$, then $\frac{1}{q} \notin \beta$. Hence, $\beta \neq \mathbb{Q}$.\retTwo
            \end{enumerate}

            Now let $p \in \beta$ and pick $r > 0$ such that $\frac{1}{p} - r \notin \alpha$. Also, assume $p > 0$ because the proof is trivial if $p \leq 0$. (The fact that $p > 0$ in $\beta$ exists is trivial to show.)\retTwo

            \begin{enumerate}
               \item[2.] If $q \leq 0 < p$, then trivially $q \in \beta$. Meanwhile, if $0 < q < p$, then\\ [2pt] $\frac{1}{q} - r > \frac{1}{p} - r$, meaning that $\frac{1}{q} - r \notin \alpha$. Hence, $q \notin \beta$.\retTwo
               
               \item[3.] Let $t = \frac{1}{\frac{1}{p} - \frac{r}{2}}$. Then since $\frac{1}{p} - r \notin \alpha$, we know that $\frac{1}{p} - \frac{r}{2} > 0$. Also since $\frac{1}{t} = \frac{1}{p} - \frac{r}{2} < \frac{1}{p}$, we have that $t > p$. But note that $\frac{1}{t} - \frac{r}{2} = \frac{1}{p} - r \notin \alpha$.\\[2pt] Hence $t \notin \beta$.\retTwo
            \end{enumerate}
         \end{myIndent}

         We claim that $\beta\alpha = 1^*$. Hence, we can define $\frac{1}{\alpha} = \beta$.

         \begin{myIndent}\exP
            To start, suppose $r \in \alpha \cap \mathbb{Q}^+$ and $s \in \beta \cap \mathbb{Q}^+$. Then $\frac{1}{s} \notin \alpha$, meaning that\\ $r < \frac{1}{s}$. So $rs < 1$, thus showing that $\alpha\beta \subseteq 1^*$.\retTwo

            The other inclusion has a more complicated proof. Firstly, take any\\ $v \in 1^* \cap \mathbb{Q}^+$ (the proof is trivial if $v \leq 0$). Then set $w = \frac{1}{v}$, meaning\\ that $w > 1$. Now since $\alpha \in R^+$, we know there exists $n \in \mathbb{Z}$ such that\\ $w^n \in \alpha$ but $w^{n+1} \notin \alpha$. Then as $w^{n+2} > w^{n+1}$, we know that $\frac{1}{w^{n+2}} \in \beta$.\\ Hence, $v^2 = w^n \frac{1}{w^{n+2}} \in \alpha\beta$.\retTwo

            Now that we've shown that the square of every $v \in 1^* \cap \mathbb{Q}^+$ is also in $\alpha\beta$,\\ [2pt] we next show that there exists $z \in 1^* \cap \mathbb{Q}^+$ such that $z^2 > v$. Suppose $v = \frac{p}{q}$ where $p, q \in \mathbb{Z}^+$. Then set $z = \frac{p + q}{2q}$. Importantly, since $p < q$, we still have that $z \in 1^*$. But also note that:

            \begin{centering}
               $z^2 - v = \frac{p^2 + 2pq + q^2}{4q^2} - \frac{4pq}{4q^2} = \frac{p^2 - 2pq + q^2}{4q^2} = \left(\frac{p - q}{2q}\right)^2 \geq 0$\retTwo\par
            \end{centering}

            Thus as $v \leq z^2$ and $z^2 \in \alpha\beta$, we have that $v \in \alpha\beta$ as well. So $1^* \subseteq \alpha\beta$.\newpage
         \end{myIndent}

         Finally, so long as $\alpha, \beta, \gamma \in R^+$, we have that  $\alpha(\beta + \gamma) = \alpha\beta + \alpha\gamma$ because the rational numbers satisfy the distributive property.\retTwo
      \end{myIndent}

      Notably, in proving that $\alpha\beta \in R^+$ before, we also guarenteed that for $\alpha, \beta > 0$, we have that $\alpha\beta > 0$.\retTwo
   \end{myIndent} 

   \dispDate{9/7/2024}

   \begin{myIndent}

      Now we still need to extend our definition of multiplication from $R^+$ to all of $R$. To do this, set $\alpha 0^* = 0^*\alpha = 0^*$ and define:

      {\centering $\alpha \beta = \left\{
      \begin{matrix}
         (-\alpha)(-\beta) & \text{ if } \alpha < 0^*, \beta < 0^* \\
         -((-\alpha)\beta) & \text { if } \alpha < 0^*, \beta > 0^* \\
         -(\alpha(-\beta)) & \text{ if } \alpha > 0^*, \beta < 0^*
      \end{matrix}\right.$ \retTwo\par}

      Having done that, reproving those properties of multiplication on all of $R$ just\\ becomes a matter of addressing many cases and using the identity that\\ $(-(-\alpha)) = \alpha$.

      \begin{myIndent}\myComment
         Note that that identity can be proven just from the addition properties of a field.\retTwo
      \end{myIndent}

      Because I'm bored with this construction at this point, I'm going to skip reproving those properties.\retTwo

      So now that we've established that $R$ is a field, all we have left to do is to show that all numbers $r, s \in \mathbb{Q}$ are represented by cuts $r^*, s^* \in R$ such that:
      
      \begin{itemize}
         \item $(r + s)^* = r^* + s^*$
         \item $(rs)^* = r^*s^*$
         \item $r < s \Longleftrightarrow r^* < s^*$\retTwo
      \end{itemize}

      Again, I'm super bored and demotivated at this point. So, I'm going to skip showing this.\retTwo

      With all that done, we've now shown that $R$ satisfies all of the properties of real numbers. That concludes the proof of the existence theorem of the real numbers.
      \newpage
   \end{myIndent}

   \dispDate{9/9/2024}

   \hOne
   Today I'm just looking at James Munkres' book \textit{Topology}. Now while I'm done with the era of my life of taking exhaustive notes on a textbook, I still want to write down some interesting proofs. I also hope to do some exercises.\retTwo

   \blab{Theorem 7.8:} Let $A$ be a nonempty set. There is no injective map $f: \mathcal{P}(A) \longrightarrow A$ and there is no surjective map $g: A \longrightarrow \mathcal{P}(A)$.

   \begin{myDindent}\myComment
      In other words, the power set of a set has strictly greater cardinality.\retTwo
   \end{myDindent}

   
   \begin{myIndent}\hTwo
      Proof:\\
      If such an injective $f$ existed, then that would imply a surjective $g$ exists. So, we just need to show that any function $g: A \longrightarrow \mathcal{P}(A)$ isn't surjective.\retTwo

      Let $g: A \longrightarrow \mathcal{P}(A)$ be any function and define $B = \{a \in A \mid a \in A - g(a)\}$.\\ Clearly, $B \subseteq A$. However, $B$ cannot be in the image of $g$. After all, suppose there exists $a_0 \in A$ such that $g(a_0) = B$. Then we get a contradiction because:

      {\centering $a_0 \in B \Longleftrightarrow a_0 \in A - g(a_0) \Longleftrightarrow a_0 \in A - B$ \retTwo\par}

      Hence, $g(A) \neq \mathcal{P}(A)$ and we conclude that $g$ can't be surjective. $\blacksquare$\retTwo\retTwo
   \end{myIndent}

   \exOne
   \blab{Exercise 7.3:} Let $X= \{0, 1\}$. Show there is a bijective correspondence between the set $\mathcal{P}(\mathbb{Z}_+)$ and the Cartesian product $X^\omega$.\retTwo

   
   \begin{myIndent}\exTwo
      For any set $A \in \mathcal{P}(\mathbb{Z}_+)$, define $f(A)$ to be the $\omega$-tuple $\mathbf{x}$ such that for all\\ $i \in \mathbb{Z}^+$, $\mathbf{x}_i = 1$ if $i \in A$ and $\mathbf{x}_i = 0$ if $i \notin A$. Then clearly $f$ is injective as\\ $\forall A, B \in \mathcal{P}(\mathbb{Z}_+)$, $f(A) = f(B) \Longrightarrow A = B$. Also, given any $\mathbf{x} \in X^\omega$, we\\ know that the set $A = \{i \in \mathbb{Z}_+ \mid \mathbf{x}_i = 1\}$ satisfies that $f(A) = \mathbf{x}$,\\ meaning $f$ is surjective.
      
      \retTwo Hence, $f$ is a bijective function between $\mathcal{P}(\mathbb{Z}_+)$ and $X^\omega$.
      \begin{myTindent}\myComment
         Note that this construction still works if $\mathbb{Z}_+$ is replaced with any\\ countably infinite set.\retTwo\retTwo
      \end{myTindent}
   \end{myIndent}

   \blab{Exercise 7.5:} Determine whether the following sets are countable or not.
   \begin{itemize}
      \item[(f)] The set $F$ of all functions $f: \mathbb{Z}_+ \longrightarrow \{0, 1\}$ that are "eventually zero", meaning there is a positive integer $N$ such that $f(n) = 0$ for all $n \geq N$.
      
      \begin{myIndent}\exTwo
         $F$ is countable. To see why, let:
         
         {\centering $A_n = \{f: \mathbb{Z}_+ \longrightarrow \{0, 1\} \mid \forall i \geq n, \myHS f(i) = 0\}$\retTwo\par}
         
         Thus each $A_n$ is finite (with $2^n$ elements) and $F = \bigcup\limits_{n = 1}^\infty A_n$.\newpage
      \end{myIndent}

      \item[(g)] The set $G$ of all functions $f: \mathbb{Z}_+ \longrightarrow \mathbb{Z}_+$ that are eventually $1$.
      
      \begin{myIndent}\exTwo
         $G$ is countable. To see why, let:

         {\centering $A_n = \{f: \mathbb{Z}_+ \longrightarrow \mathbb{Z}_+ \mid \forall i \geq n, \myHS f(i) = 1\}$\retTwo\par}

         Then each $A_n$ has a bijective correspondence with $(\mathbb{Z}_+)^n$, meaning each $A_n$ is countable, and $G = \bigcup\limits_{n = 1}^\infty A_n$.
         
         \begin{myTindent}\myComment
            The same argument applies to all functions $f: \mathbb{Z}_+ \longrightarrow \mathbb{Z}_+$ that are eventually any constant.\retTwo
         \end{myTindent}
      \end{myIndent}

      \item[(h)] The set $H$ of all functions $f: \mathbb{Z}_+ \longrightarrow \mathbb{Z}_+$ that are eventually constant.
      
      \begin{myIndent}\exTwo
         $H$ is countable. To see why, let $A_n$ be the set of all functions\\ $f: \mathbb{Z}_+ \longrightarrow \mathbb{Z}_+$ that are eventually $n$. Because of part g of\\ [-6pt] this exercise, we know that each $A_n$ is countable. Also, $H = \bigcup\limits_{n = 1}^\infty A_n$.\retTwo
      \end{myIndent}

      \item[(i)] The set $I$ of all two-element subsets of $\mathbb{Z}_+$
      \item[(j)] The set $J$ of all finite subsets of $\mathbb{Z}_+$.
      
      \begin{myIndent}\exTwo
         Both $I$ and $J$ are countably infinite. We know this because we can define\\ surjections from $(\mathbb{Z_+})^2$ to $I$ and $\bigcup\limits_{n = 1}^\infty (\mathbb{Z}_+)^n$ to $J$.
         
         \begin{myIndent}
            (Finite cartesian products of countable sets and unions of countably many countable sets are countable.)\retTwo\retTwo
         \end{myIndent}
      \end{myIndent}
   \end{itemize}

   \blab{Exercise 7.6.a:} Show that if $B \subset A$ and there is an injection $f: A \longrightarrow B$, then $|A| = |B|$.
   
   \begin{myIndent}\exTwo
      According to the hint, we set $A_1 = A$ and $A_n = f(A_{n-1})$ for all $n > 1$. Similarly, we set $B_1 = B$ and $B_n = f(B_{n-1})$ for all $n > 1$.\retTwo

      We can assume $A_2$ is a proper subset of $B_1$ because if $A_2 = B_1$, then we already have that $f$ is a bijection. Also, as $f$ is an injection, we know that $B_2 \subset A_2$. Thus by induction, we can conclude that:

      {\centering $ A_1 \supset B_1 \supset A_2 \supset B_2 \supset A_3 \supset B_3 \supset \cdots $\retTwo\par}

      Now, the textbook recommends defining $h: A \longrightarrow B$ by:

      \begin{center}
         $h(x) = \left\{
         \begin{matrix}
            f(x) & & \text{ if } x \in A_n - B_n \text{ for any } n \in \mathbb{Z}_+ \\
            x & & \text{ otherwise }
         \end{matrix}\right.$\newpage
      \end{center}

      
      \begin{myIndent}
         \myComment I want to ask a professor about this definition because it urks me. My issue with\\ this definition of $h$ is that I feel like it should be possible for:
         $$\bigcap\limits_{n=1}^\infty (A_n \cap B_n) \neq \emptyset.$$
         
         However, we wouldn't be able to know that some $x$ is in that intersection and\\ thus falls into case 2 until after an infinite number of steps.\retTwo
         
         On the other hand, $S_1 = \bigcup\limits_{n = 1}^\infty (A_n - B_n)$ is a valid definition for a set, as is\\ $S_2 = A - S_1$. So the definition $h$ is valid because it's saying that $h(x) = f(x)$\\ [6pt] if $x \in S_1$ and $h(x) = x$ if $x \in S_2$.\retTwo

         Maybe my issue is just that I have trouble trusting the validity of a function definition if I can't actually evaluate that function myself. Although, there are lots of functions like that that I don't have any problem with. For example, given $g(x) = 0$ if $x$ is rational and $g(x) = 1$ if $x$ is irrational, what is $g(\pi^2)$?\retTwo
      \end{myIndent}

      Hopefully it is clear that $h$ is in fact a valid function from $A$ to $B$. Now firstly, we shall show that $h$ is injective.

      \begin{myIndent}\exP
         Let $x, y \in A$ such that $x \neq y$. If there are integers $n$ and $m$ such that $x \in A_n - B_n$ and $y \in A_m - B_m$, then $h(x) \neq h(y)$ because $f$ is injective. Meanwhile, if no such $n$ or $m$ exists, then $h(x) \neq h(y)$ because $x \neq y$.\retTwo

         This leaves the case that there exists $n \in \mathbb{Z}_+$ such that $x \in A_n - B_n$ but for\\ all $m \in \mathbb{Z}_+,\myHS y \notin A_m - B_m$. Then, note that $f(x) \in f(A_{n} - B_{n})$. And since\\ $f$ is injective, we thus have that $f(x) \in f(A_{n}) - f(B_{n}) = A_{n+1} - B_{n+1}$.\\ Therefore, as $y \notin A_{n+1} - B_{n+1}$, we know that $h(x) \neq y = h(y)$.\retTwo
      \end{myIndent}

      Next, we show $h$ is surjective.

      \begin{myIndent}\exP
         Let $y \in B$.\retTwo
         
         Suppose there exists $n \in \mathbb{Z}_+$ such that $y \in A_n - B_n$. We know that $n \neq 1$ since $y \in B$. Thus, there must exist $x \in A_{n-1}$ such that $y = f(x) \in f(A_{n-1}) = A_n$. Furthermore, this $x$ can't be in $B_{n-1}$ because otherwise $y$ would be in $B_n$ which we know isn't true. So, $x \in A_{n-1} - B_{n-1}$, meaning that $h(x) = f(x) = y$.\retTwo

         Meanwhile, if no such $n$ exists, then we simply have that $h(y) = y$. Hence,\\ $h(A) = B$.\retTwo
      \end{myIndent}

      Thus, we've shown that $h$ is a bijection, meaning that $|A| = |B|$.\newpage
   \end{myIndent}

   \blab{Exercise 7.7:} Show that $|\{0, 1\}^\omega| = |(\mathbb{Z}_+)^\omega|$.

   
   \begin{myIndent}\exTwo
      Firstly, obviously a bijection exists between $\{0, 1\}^\omega$ and $\{1, 2\}^\omega$. Also,\\ $\{1, 2\}^\omega \subset (\mathbb{Z}_+)^\omega$. So, if we can construct an injective function from $(\mathbb{Z}_+)^\omega$\\ to $\{1, 2\}^\omega$, then we can apply the result of exercise 7.6.a to prove this\\ exercise's claim.\retTwo

      We shall create this injection using a diagonalization argument. Let $x \in (\mathbb{Z}_+)^\omega$.\\ Then we define $f(x) = y \in \{1, 2\}^\omega$ as follows:
      
      \begin{center}
         \begin{tabular}{c}
            $y(1) = 2$ if $x(1) = 1$. Otherwise $y(1) = 1$.\\ [6pt]
            $y(2) = 2$ if $x(1) = 2$. Otherwise $y(2) = 1$.\\
            $y(3) = 2$ if $x(2) = 1$. Otherwise $y(3) = 1$.\\ [6pt]
            $y(4) = 2$ if $x(1) = 3$. Otherwise $y(4) = 1$.\\
            $y(5) = 2$ if $x(2) = 2$. Otherwise $y(5) = 1$.\\
            $y(6) = 2$ if $x(3) = 1$. Otherwise $y(6) = 1$. \\ [6pt]

            $y(7) = 2$ if $x(1) = 4$. Otherwise $y(7) = 1$.\\
            $\vdots$\\ [12pt]
         \end{tabular}
      \end{center}

      Clearly $f$ is an injection since $f(x_1) = f(x_2)$ implies that $x_1$ and $x_2$ have the same integers at all indices.\retTwo\retTwo
   \end{myIndent}

   \blab{Exercise 7.6.b: (Schroeder-Bernstein theorem)} If there are injections $f: A \longrightarrow C$ and $g: C\longrightarrow A$, then $A$ and $C$ have the same cardinality.\retTwo
   \myComment
   I did my work on paper and now it's late and I don't want to write more tonight.\retTwo\retTwo

   \dispDate{9/11/2024}

   \hOne

   Since today's my day off, I'm gonna work through Munkres' textbook \textit{Topology} some more.\retTwo

   \blab{Theorem 8.4 (Principle of recursive definition):} Let $A$ be a set and let $a_0$ be an element of $A$. Suppose $\rho$ is a function assigning an element of $A$ to each function $f$ mapping a nonempty section of the positive integers onto $A$. Then there exists a unique function $h: \mathbb{Z}_+ \longrightarrow A$ such that:

   {\begin{center}
      \begin{tabular}{l c r}
         $(*)$ & \phantom{aaaa} & 
         \begin{tabular}{l r}
            $h(1) = a_0$ & \\
            $h(i) = \rho(h|_{\{1, \ldots, i-1\}})$ & $\text{for } i > 1\text{.}$
         \end{tabular}
      \end{tabular}\retTwo
   \end{center}}

   \newpage
   \begin{myIndent}\hTwo
      Proof outline:

     \begin{myIndent}\hThree
       Part 1: Given any $n \in \mathbb{Z}_+$, there exists a function $f: \{1, \ldots, n\} \longrightarrow A$ that\\ satisfies $(*)$.
 
       \begin{myIndent}\myComment
          This is obvious from induction.\\ [9pt]
       \end{myIndent}

       Part 2: Suppose that $f: \{1, \ldots, n\} \longrightarrow A$ and $g: \{1, \ldots, m\} \longrightarrow A$ both satisfy $(*)$ for all $i$ in their respective domains. Then $f(i) = g(i)$ for all $i$ in both domains.

      \begin{myIndent}
         Proof:\\
         Suppose not. Let $i$ be the smallest integer for which $f(i) \neq g(i)$.\retTwo
         
         We know $i \neq 1$ because $f(1) = a_0 = g(1)$. But then note that\\ $f|_{\{1, \ldots, i - 1\}} = g|_{\{1, \ldots, i- 1\}}$. Hence:
         
         {\centering $f(i) = \rho(f|_{\{1, \ldots, i - 1\}}) = \rho(g|_{\{1, \ldots, i - 1\}}) = g(i)$.\retTwo\par}

         This contradicts that $i$ is the smallest integer for which $f(i) \neq g(i)$.\\ [9pt]
      \end{myIndent}

      Part 3: Let $f_n: \{1, \ldots, n\} \longrightarrow A$ be the unique function satisfying  $(*)$\\ (uniqueness was proven in part 2). Then we define:

      {\centering $h = \bigcup\limits_{i = 1}^\infty f_n$ \retTwo\par}

      \begin{myIndent}\myComment
         Because of part 2, we can fairly easily show that for each $i \in \mathbb{Z}_+$, there is exactly one element in $h$ with $i$ as it's first coordinate. Hence, the set $h$ defines a functions from $\mathbb{Z}_+$ to $A$.\retTwo

         Also, hopefully it's clear that $h$ satisfies $(*)$.\retTwo
      \end{myIndent}
     \end{myIndent}
   \end{myIndent}

   \mySepTwo

   \blab{Axiom of choice}: Given a collection $\mathcal{A}$ of disjoint nonempty sets, there exists a set $C$ consisting of exactly one element from each element of $\mathcal{A}$.

   
   \begin{myIndent}\myComment
      A few notes:
      \begin{enumerate}
         \item If we restrict $\mathcal{A}$ to being a finite collection, then there is nothing controversial about this axiom. It only becomes controversial when $\mathcal{A}$ is allowed to be infinite.
         \item There are multiple instances in baby Rudin where we made an infinite number of\\ arbitrary choices. Looking at a lot of those proofs closer, I think many of them could avoid using the axiom of choice by specifying that we had to pick rational numbers in a set. However, being able to pick elements without worrying about a preexisting choice function is way easier.\retTwo
         
         My take away from this is that not only does it make proofs cleaner to not worry about using constructed choice functions, but it's also perfectly acceptable now-a-days to use this axiom.\newpage
         
         Plus, some really commonly used theorems require the axiom of choice to prove them. For example, the union of countably many countable sets being countable. This makes it really easy to accidentally use the axiom of choice in a proof.\retTwo
      \end{enumerate}
   \end{myIndent}

   \blab{Lemma 9.2: (Existence of a choice function)} Given a collection $\mathcal{B}$ of nonempty sets (not necessarily disjoint), there exists a function \[c: \mathcal{B} \longrightarrow \bigcup\limits_{B \in \mathcal{B}}B\] such that $c(B)$ is an element of $B$ for each $B \in \mathcal{B}$.\retTwo

   
   \begin{myIndent}\hTwo
      Proof:\\
      Given any set $B \in \mathcal{B}$, we define $B^\prime = \{(B, b) \mid b \in B\}$. Because $B \neq \emptyset$, we know that $B^\prime \neq \emptyset$ as well. Furthermore, given $B_1, B_2 \in \mathcal{B}$ if $B_1 \neq B_2$, then we have that the first element of all the pairs in $B_1^\prime$ are different from that of $B_2^\prime$. So $B_1^\prime$ and $B_2^\prime$ are disjoint.\retTwo

      Now form the collection $\mathcal{C} = \{B^\prime \mid B \in \mathcal{B}\}$. From before, we know that $\mathcal{C}$ is\\ a collection of disjoint sets. So by the axiom of choice, there exists a set $c$\\ consisting of exactly one element from each element of $\mathcal{C}$.\retTwo

      This set $c$ is a subset of $\mathcal{B} \times \bigcup\limits_{B \in \mathcal{B}}B$ which satisfies our definition of a choice function.\\ [-12pt]
      \begin{myTindent}\begin{myTindent}\myComment
         Hopefully it's obvious enough why $c$ satisfies\\ those properties.\retTwo\retTwo
      \end{myTindent}\end{myTindent}
   \end{myIndent}

   \mySepTwo

   A set $A$ with an order relation $<$ is said to be \udefine{well-ordered} if every nonempty subset of $A$ has a smallest element.\retTwo

   
   \begin{myIndent}\myComment
      
      {\fontsize{13}{15}\selectfont%
      \blab{Tangent: inductiveness of $\mathbb{Z}_+$ is equivalent to the well-orderedness of $\mathbb{Z}_+$}
      }
      
      \begin{myIndent}
         This proof is taken from https://math.libretexts.org/ on their page for the\\ well-ordering principle.\retTwo

         ($\Longrightarrow$)\\
         Suppose $S$ is a nonempty subset of $\mathbb{Z}_+$ with no least element. Then let $R$ be the set of lower bounds of $S$. Since $1$ is the least element of $\mathbb{Z}_+$, we know that $1 \in R$.\retTwo
         
         Now given any $k \geq 1$, if $k \in R$, we know that $\{1, \ldots, k\}$ must be a subset of $R$. Also note that $R \cap S = \emptyset$ because if that wasn't true, we'd know that $S$ has a least element. Therefore, $\{1, \ldots, k\} \cap S = \emptyset$. But then that shows that $k + 1 \notin S$ since otherwise $k + 1$ would be the least element of $S$. Furthermore, since no element of $\{1, \ldots, k\}$ is in $S$, we automatically have that $k + 1 \in R$.\retTwo

         By induction, this means that $R = \mathbb{Z}_+$. Hence, we get a contradiction as $S$ must be empty.\newpage

         ($\Longleftarrow$)\\
         Let $S$ be a subset of $\mathbb{Z}_+$ such that $1 \in S$ and $k \in S \Longrightarrow k + 1 \in S$. Then suppose that $S \neq \mathbb{Z}_+$. In that case, we know that $S^\comp \neq \emptyset$, and since $\mathbb{Z}_+$ is well-ordered, we know there is a least element $\alpha$ of $S^\comp$.\retTwo

         Because $1 \in S$, we know that $\alpha \geq 2$. But then consider that $1 \leq \alpha - 1 < \alpha$. Therefore, $\alpha - 1 \in S$, thus implying that $\alpha \in S$. This contradicts that $\alpha \in S^\comp$.\retTwo
      \end{myIndent}

      {\fontsize{13}{15}\selectfont%
      From what I've heard, when defining the positive integers, one usally takes one of the two above properties as an axiom and then proves the other as a theorem. In Munkres' book, he starts with induction and proves well-orderedness.\retTwo\retTwo
      }
   \end{myIndent}

   Facts:
   \begin{itemize}
      \item If $A$ with the order relation $<$ is well-ordered, then any subset of $A$ is well-\\ordered as well with $<$ restricted to that subset.
      \item If $A$ has the order relation $<_1$ and $B$ has the order relation $<_2$ and both are well-ordered, then $A \times B$ is well-ordered with the dictionary order.
      \item Given any countable set $A$, we know there exists a bijection $f$ from $A$ to $\mathbb{Z}_+$. Hence, given $a, b \in A$, we can say that $a < b \Longleftrightarrow f(a) < f(b)$. Then, $A$ is well-ordered by $<$ with the least element of any subset $S$ of $A$ being the element $\alpha \in A$ such that $f(\alpha)$ is the least element in $f(S)$.
      \item If a set $A$ is well-ordered, then we can make a choice function $c: \mathcal{P}(A) \longrightarrow A$ using that well-ordering.
      
      \begin{myIndent}
         Specifically, given any $B \subseteq A$, assign $c(B) = \beta$ where $\beta$\\ is the least element of $B$.

         
         \begin{myIndent}\myComment
            This is why we can pick elements of $\mathbb{Q}$ without worrying about the axiom of choice.\retTwo\retTwo
         \end{myIndent}
      \end{myIndent}
   \end{itemize}

   An important theorem (which I will hopefully prove soon) is:
   
   \begin{myIndent}
      \blab{The Well Ordering Theorem:} If $A$ is a set, there exists an order relation on $A$ that is well-ordering.
      
      \begin{myIndent}\myComment
         Note: this theorem requires the axiom of choice to prove.\retTwo
      \end{myIndent}
   \end{myIndent}

   \exOne\blab{Exercise 10.5:} Show that the well-ordering theorem implies the (infinite) axiom of choice.
   \begin{myIndent}\exTwo
      Let $\mathcal{A}$ be a collection of disjoint sets. By the well-ordering theorem, we can pick an order relation on $\bigcup\limits_{A \in \mathcal{A}}\hspace{-0.4em}A$ that is well-ordering.\newpage
      
      \begin{myIndent}\myComment
         Note that the previous sentence is carefully worded to only make use of the finite axiom of choice. Specifically, the order relation we are picking is an element of some subset of $\bigcup\limits_{A \in \mathcal{A}}A \times \bigcup\limits_{A \in \mathcal{A}}A$.\retTwo

         If we had instead picked a well-ordering for each $A \in \mathcal{A}$, then that would require the axiom of choice as we would be making potentially infinitely many arbitrary choices of order relations.\retTwo
      \end{myIndent}

      Now let $C = \{a \in \hspace{-0.2em}\bigcup\limits_{A \in \mathcal{A}}\hspace{-0.2em}A \mid \exists A \in \mathcal{A} \suchthat a \in A \text{ and } \forall b \in A,\myHS a \leq b \}$.\retTwo Then $C$ fulfils the properties of the set that the axiom of choice would guarentee exists.\retTwo
   \end{myIndent}

   \dispDate{9/14/2024}

   \blab{Exercise 10.1:} Show that every well-ordered set has the least-upper-bound\\ property.

   \begin{myIndent}\exTwo
      Let the set $A$ with the order relation $<$ be well-ordered. Then consider any nonempty $B \subseteq A$ and suppose there exists $\alpha \in A$ such that $b < \alpha$ for all $b \in B$.\retTwo

      Let $U = \{a \in A \mid \forall b \in B, \myHS b \leq a\}$. Since $\alpha \in U$, we know that $U \neq \emptyset$. So, because $A$ is well-ordered, we know that $U$ has a least element $\beta$. This $\beta$ is by definition the least upper bound of $B$. So $\sup{B} = \beta$.\retTwo\retTwo
   \end{myIndent}

   \hOne

   Let $X$ be a well-ordered set. Given $\alpha \in X$, let $S_\alpha$ denote the set $\{x \in X \mid x < \alpha\}$. We call $S_\alpha$ the \udefine{section} of $X$ by $\alpha$.\retTwo

   \blab{Lemma 10.2:} There exists a well-ordered set $A$ having a largest element $\Omega$ such that $S_\Omega$ is uncountable but every other section of $A$ is countable.

   \begin{myIndent}\hTwo
      Proof:\\
      Starting off, let $B$ be an uncountable well-ordered set. Then let $C$ be the well-\\ordered set $\{1, 2\} \times B$ with the dictionary order. Clearly, given any $b \in B$, we have that $S_{(2, b)}$ is uncountable. So the set of $c \in C$ such that $S_c$ is uncountable is not empty.\retTwo

      Let $\Omega$ be the least element of $C$ such that $S_\Omega$ is uncountable. Then define\\ $A = S_\Omega \cup \{\Omega\}$. This is called a \udefine{minimal uncountable well-ordered set}.
      \retTwo
      
      \begin{myIndent}\myComment
         The reason we are considering $\{1, 2\} \times B$ instead of just $B$ is that if we were just considering $B$, then we wouldn't be able to guarentee that there exists $b \in B$ such that $S_b$ is uncountable.\newpage

         User MJD on https://math.stackexchange.com wrote some good intuition for why\\ this is.
         \begin{myIndent}
            While the set $\mathbb{Z}_+$ is countably infinite, all sections $S_x$ of $\mathbb{Z}_+$ are finite.\\ However, when considering $\{1, 2\} \times \mathbb{Z}_+$ with the dictionary order, we\\ have that $S_{(2, 1)}$ is countably infinite. Furthermore, all sections of $S_{(2, 1)}$\\ are finite. Thus, $S_{(2, 1)}$ would be a minimal \textit{countable} well-ordered set.\retTwo
         \end{myIndent}
      \end{myIndent}
   \end{myIndent}

   We call a set described by lemma 10.2 $\overline{S}_\Omega = S_\Omega \cup \{\Omega\}$.\retTwo

   \blab{Theorem 10.3:} If $A$ is a countable subset of $S_\Omega$, then $A$ has an upper bound in $S_\Omega$.
   
   \begin{myIndent}\hTwo
      Proof:\\
      Let $A$ be a countable subset of $S_\Omega$. For all $a \in A$, we know that $S_a$ is countable. Therefore, $B = \bigcup\limits_{a \in A}S_a$ is also countable, meaning that $S_\Omega - B \neq \emptyset$.\retTwo

      If we pick $x \in S_\Omega - B$, we must have that $x$ is an upper bound to $A$ because if $x < a$ for some $a \in A$, we would have that $x \in S_a \subseteq B$.\retTwo
      
      
      \begin{myIndent}\myComment
         If you combine this with exercise 10.1, we know that $A$ has a least upper bound.\retTwo
      \end{myIndent}
   \end{myIndent}

   \exOne
   \blab{Exercise 10.6:} Let $S_\Omega$ be a minimal uncountable well-ordered set.
   
   \begin{itemize}
      \item[(a)] Show that $S_\Omega$ has no largest element.
      \begin{myIndent}\exTwo
         Suppose $\alpha \in S_\Omega$ is the largest element of $S_\Omega$. In that case, we'd have that\\ $S_\alpha = S_\Omega - \{\alpha\}$. However, by theorem 10.3, we know that $S_\alpha$ is countable. This implies that $S_\Omega = S_\alpha \cup \{\alpha\}$ must also be countable, which is a contradiction.
         \retTwo
      \end{myIndent}
      \item[(b)] Show that for every $\alpha \in S_\Omega$, the subset $\{x \in S_\Omega \mid \alpha < x\}$ is uncountable.
      \begin{myIndent}\exTwo
         Let $\alpha \in S_\Omega$. By the law of trichotomy, we know that:
   
         {\centering $S_\Omega = \{x \in S_\Omega \mid x < \alpha\} \cup \{\alpha\} \cup \{x \in S_\Omega \mid \alpha < x\}$.\retTwo\par}
         
         Now suppose $\{x \in S_\Omega \mid \alpha < x\}$ is countable. Then as both $\{x \in S_\Omega \mid x < \alpha\}$ and $\{\alpha\}$ are countable, we have a contradiction as the three's union must also be countable. But we know $S_\Omega$ isn't.\retTwo
      \end{myIndent}
      % \item[(c)] Let $X_0$ be the subset of $S_\Omega$ consisting of all elements $x$ such that $x$ has no immediate predecessor. Show that $X_0$ is uncountable.  
   \end{itemize}

   \hOne
   \mySepTwo

   Some definitions I've been lacking:
   \begin{enumerate}
      \item Let $A$ be a set and suppose $x, y, z$ are any three different elements of $A$.
      
      {\centering\fontsize{11}{13}\selectfont%
         \begin{tabular}{|l|l|}
            \udefine{Simple [Default] Order Relation}: ($<$) & \udefine{Strict Partial Order Relation}: ($\prec$) \\ [4pt] \hline &\\ [-9pt]
   
            \begin{tabular}{l}
               Nonreflexitivity: $x \not< x$ \\
               Transitivity: $x < y$ and $y < z \Rightarrow x < z$ \\
               Comparability: $x < y$ or $y < x$ is true
            \end{tabular} &
            \begin{tabular}{l}
               Nonreflexitivity: $x \not\prec x$ \\
               Transitivity: $x \prec y$ and $y \prec z \Rightarrow x \prec z$\\\phantom{a}
            \end{tabular}
         \end{tabular}
      \par}
      \newpage

      
      \begin{myIndent}\myComment
         Basically, a partial order relation is allowed to not give an order for some pairings of elements. If someone just says a set is ordered, they mean the set is simply ordered.\retTwo
      \end{myIndent}

      \item Let $A$ and $B$ be sets ordered by $<_A$ and $<_B$ respectively. We say that $A$ and\\ $B$ have the same \udefine{order type} if there exists an order-preserving bijection\\ $f: A \longrightarrow B$, meaning that $\forall a_1, a_2 \in A,\myHS a_1 <_A a_2 \Longrightarrow f(a_1) <_B f(a_2)$.
      
      \begin{myIndent}\myComment
         It is trivial to show that if $f$ is an order-preserving bijection, then $f^{-1}$ is also an order-\\preserving bijection.\retTwo
      \end{myIndent}

      \item If $A$ is an ordered set and $a$ and $b$ are two different elements, then consider\\ the set $S = \{x \in A \mid a < x < b\}$. If $S = \emptyset$ we say that $b$ is the \udefine{successor} of\\ $a$ and $a$ is the \udefine{predecessor} of $b$.\retTwo\retTwo
   \end{enumerate}

   \exOne
   \blab{Exercise 10.2}:  
   \begin{itemize}
       \item[(a)] Show that in a well-ordered set, every element except the largest (if one exists) has an immediate successor
       
      \begin{myIndent}\exTwo
         Let $A$ be a well-ordered set and let $\alpha$ be any element in $A$ such that there exists $\beta \in A$ for which $\alpha < \beta$. Then consider the set $S = \{x \in A \mid \alpha < x < \beta\}$. If $S = \emptyset$, then we know $\alpha$ has $\beta$ as its successor. Meanwhile, if $S \neq \emptyset$, then since $A$ is well-ordered, we know that $A$ has a least element $\gamma$. Thus, the set $\{x \in A \mid \alpha < x < \gamma\} = \emptyset$ and we know that $\gamma$ is the successor of $\alpha$.\retTwo
      \end{myIndent}

       \item[(b)] Find a set in which every element has an immediate successor that is not well-ordered. 
       \begin{myIndent}\exTwo
         Consider the set $\mathbb{Z}$ of all integers using the standard ordering. Then for any $n \in \mathbb{Z}$, we know that its successor is $n + 1$. At the same time though, the set of all negative integers has no least element. So $\mathbb{Z}$ is not well-ordered by $<$.\retTwo\retTwo
      \end{myIndent}
   \end{itemize}

   \blab{Exercise 10.6:}
   \begin{itemize}
      \item[(c)] Let $X_0$ be the subset of $S_\Omega$ consisting of all elements $x$ such that $x$ has no\\ immediate predecessor. Show that $X_0$ is uncountable.
      
      \begin{myIndent}\exTwo
         Suppose $X_0$ is bounded above by some $\alpha \in S_\Omega$. Thus, there is a predecessor $x \in S_\Omega$ for any $y$ in the set $T = \{z \in S_\Omega \mid z > \alpha\}$.\newpage

         Now define a function $f: \mathbb{Z}_+ \longrightarrow T$ such that $f(1) =$ the least element of $T$ and $f(n) =$ the successor of $f(n - 1)$ for all $n > 1$. We know this function is well-defined because $S_\Omega$ has no largest element according to exercise 10.6.a. So, all elements of $S_\Omega$ and thus $T$ have a successor by exercises 10.2.a, meaning our formula for $f(n)$ is always defined no matter what $f(n-1)$ is. Hence, the principle of recursive definition guarentees a unique $f$ exists.\retTwo

         Now it's easy to show that $f$ is injective. For suppose that given some $x, n \in \mathbb{Z}_+$ we had that $f(x) = f(x + n)$. Then that would mean that:

         {\centering $f(x) < f(x + 1) < \cdots < f(x + n - 1) < f(x + n) = f(x) $\retTwo\par}

         Hence we have a contradiction as $f(x) < f(x)$.\retTwo

         Next, we show that $f$ is surjective. Suppose the set $R = T - f(\mathbb{Z}_+) \neq \emptyset$. Then since $S_\Omega$ and hence $T$ is well-ordered, we know that $R$ has a least element $\beta$. But note that $\beta$ has a predecessor $\gamma$ which isn't in $R$. More specifically, since we know that the least element of $T$ is in $f(\mathbb{Z}_+)$, we know that $\gamma$ is at least the least of element of $T$. So $\gamma \in T$.\retTwo

         Thus we conclude that $\gamma \in T - (T - f(\mathbb{Z}_+)) = f(\mathbb{Z}_+)$, meaning there exists $N$ such that $f(N) = \gamma$. But this means that $f(N + 1) = \beta$, which contradicts that $\beta$ is the least element of $R$.\retTwo

         With that, we've now shown that $f: \mathbb{Z}_+ \longrightarrow T$ is a bijection, meaning that $T$ is countable. However, this contradicts exercise 10.6.b. which asserts that $T$ is uncountable.\retTwo

         Therefore, we conclude that $X_0$ cannot be bounded above. And by theorem 10.3, that means that $X_0$ can't be a countable subset of $S_\Omega$.\retTwo
      \end{myIndent}
   \end{itemize}
   
   \blab{Exercise 10.4:}
   \begin{itemize}
      \item[(a)] Let $\mathbb{Z}_-$ be the set of negative integers in the usual order. Show that a simply ordered set $A$ fails to be well-ordered if and only if it contains a subset having the same order type as $\mathbb{Z}_-$.
      
      
      \begin{myIndent}\exTwo
         ($\Longleftarrow$)\\
         If for some $B \subseteq A$, we have that $f: \mathbb{Z}_- \longrightarrow B$ is an order preserving bijection, then we must have that $B$ has no least element. Hence, not all subsets of $A$ have a least element, meaning that $A$ is not well-ordered.\retTwo

         ($\Longrightarrow$)\\
         If $A$ is not well ordered, then we know there is a set $B \subseteq A$ with no least element. Now using the axiom of choice, choose any $\beta_1 \in B$. Then for all\\ $n > 1$, choose $\beta_n \in B_{\beta_{n-1}}$. In other words, choose $\beta_n \in B$ such that\\ $\beta_n < \beta_{n-1}$.\newpage
         
         Finally, define $f: \mathbb{Z}_- \longrightarrow \{\beta_n \mid n \in \mathbb{Z}_+\}$ by the rule: $f(n) = \beta_{-n}$. This $f$ is an order preserving bijection. Thus, the set $\{\beta_n \mid n \in \mathbb{Z}_+\} \subseteq A$ has the same order type as $\mathbb{Z}_-$.\retTwo
      \end{myIndent}

      \item[(b)] Show that if $A$ is simply ordered and every countable subset of $A$ is well-ordered, then $A$ is well-ordered.
      
      \begin{myIndent}\exTwo
         It's easy to show the contrapositive of this statement.\retTwo

         If $A$ is not well-ordered, then by part a. we know there exists a set $B \subseteq A$ and a function $f: \mathbb{Z}_- \longrightarrow B$ that is an order-preserving bijection. Clearly, $B$ has no least element. Also, the function $g(n) = f(-n)$ gives a bijection from $\mathbb{Z}_+$ to $B$, meaning that $B$ is countable. Hence, we have shown that $B$ is a countable subset of $A$ that is not well-ordered.\retTwo\retTwo
      \end{myIndent}
   \end{itemize}

   \hOne
   Let $J$ be a well-ordered set. A subset $J_0$ of $J$ is said to be \udefine{inductive} if for every $\alpha \in J$, we have that $(S_\alpha \subseteq J_0) \Longrightarrow \alpha \in J_0$.\retTwo

   \exOne\blab{Exercise 10.7: (The principle of transfinite induction)} If $J$ is a well-ordered set and $J_0$ is an inductive subset of $J$, then $J_0 = J$.
   
   \begin{myIndent}\exTwo
      Proof:\\
      Suppose $J_0 \neq J$. That would mean the set $J - J_0$ is nonempty. So let $\alpha$ be the least element of $J - J_0$. We know that $S_\alpha$ must be disjoint to $J - J_0$, meaning that $S_\alpha \in J_0$. But then by the inductiveness of $J_0$, we must have that $\alpha \in J_0$. This contradicts that $\alpha$ is the least element of $J - J_0$.\retTwo\retTwo
   \end{myIndent}

   \blab{Exercise 10.10: (Theorem)} Let $J$ and $C$ be well-ordered sets; assume that there is no surjective function mapping a section of $J$ onto $C$. Then there exists a unique function $h: J \longrightarrow C$ satisfying for each $x \in J$ the equation:\\ [-14pt]
   
   {\centering $(*)$\phantom{aaaaaaaaaaaaaa}$h(x) = \text{smallest element of } C - h(S_x)$.\retTwo\par}

   \begin{myIndent}\exTwo
      Proof:
      \begin{itemize}
         \item[(a)] If $h$ and $k$ map sections of $J$ or all of $J$ into $C$ and satisfy $(*)$ for all $x$ in their domains, then $h(x) = k(x)$ for all $x$ in both domains.
         
         \begin{myIndent}\exPP
            Proof:\\
            Suppose not. Let $y$ be the smallest element of the domains of $h$ and $k$ for which $h(y) \neq k(y)$. Then note that $\forall z \in S_y$, we must have that $h(z) = k(z)$. Thus, we get a contradiction since:

            {\centering $h(y) = \text{smallest}(C - h(S_y)) = \text{smallest}(C - k(S_y)) = k(y)$.\newpage\par}
         \end{myIndent}

         \item[(b)] If there exists a function $h: S_\alpha \longrightarrow C$ satisfying $(*)$, then there exists a function $k: S_\alpha \cup \{\alpha\} \longrightarrow C$ satisfying $(*)$.
         \begin{myIndent}\exPP
            Proof:\\
            Since there is no surjective function mapping a section of $J$ onto $C$, we know that $C - h(S_\alpha) \neq \emptyset$. Hence, we can define $k(x) = h(x)$ for $x < \alpha$ and $k(\alpha) = \text{smallest}(C - h(S_\alpha))$.\retTwo
         \end{myIndent}

         \item[(c)] If $K \subseteq J$ and for all $\alpha \in K$ there exists $h_\alpha: S_\alpha \longrightarrow C$ satisfying $(*)$, then there exists a function $k: \bigcup\limits_{\alpha \in K}S_\alpha \longrightarrow C$ satisfying $(*)$.\\ [-16pt]
         \begin{myIndent}\exPP
            Proof:\\
            Define $k = \bigcup\limits_{\alpha \in K} h_\alpha$.\retTwo

            We know $k$ is a valid function definition because part (a) guarentees that for all $\alpha_1, \alpha_2 \in K$ greater than $x$, we have that $h_{\alpha_1}(x) = h_{\alpha_2}(x)$. Plus,  given any $x \in \bigcup\limits_{\alpha \in K}S_\alpha$, we know that there is $\alpha \in K$ such that $\forall y \in S_x,\myHS k(y) = h_\alpha(y)$.\\ [-16pt]
            \begin{myDindent}
               \phantom{aa.a}This shows that $k$ satisfies $(*)$ at any $x$ due to the relevant $h_\alpha$\\\phantom{a.aa}satisfying $(*)$.\retTwo
            \end{myDindent}
         \end{myIndent}

         \item[(d)] For all $\beta \in J$, there exists a function $h_\beta: S_\beta \longrightarrow C$ satisfying $(*)$.
         \begin{myIndent}\exPP
            Proof:\\
            Let $J_0$ be the set of all $\beta \in J$ for which there exists a function\\ $h_\beta: S_\beta \longrightarrow C$ satisfying $(*)$. Our goal is to show that $J_0$ is\\ inductive. That way, we can conclude by transfinite induction\\ (exercise 10.7) that $J_0 = J$.\retTwo

            Pick any $\beta\in J$ and suppose $S_\beta \in J_0$.
            \begin{myIndent}
               Case 1: $\beta$ has an immediate predecessor $\alpha$.
               \begin{myIndent}
                  Then $S_\beta = S_\alpha \cup \{\alpha\}$. So, knowing that $h_\alpha$ satisfying $(*)$ exists, we can use part (b) to define $h_\beta$ satisfying $(*)$.\retTwo
               \end{myIndent}

               Case 2: $\beta$ has no immediate predecessor.
               \begin{myIndent}
                  Then $S_\beta = \hspace{-0.2em}\bigcup\limits_{\alpha \in S_\beta}\hspace{-0.2em}S_\alpha$.\\ [0pt] And since we assumed that there exists $h_\alpha: S_\alpha \longrightarrow C$ satisfying $(*)$ for all $\alpha \in S_\beta$, we thus know by part (c) that there exists a function from $\hspace{-0.2em}\bigcup\limits_{\alpha \in S_\beta}\hspace{-0.2em}S_\alpha = S_\beta$ to $C$ satisfying $(*)$.\\
               \end{myIndent}
            \end{myIndent}

            Thus in both cases, we have shown that $S_\beta \in J_0$ implies that $h_\beta: S_\beta \longrightarrow C$ satisfying $(*)$ exists. Or in other words, $S_\beta \in J_0 \Longrightarrow \beta \in J_0$.\retTwo
         \end{myIndent}

         \item[(e)] Finally, we now finish proving this theorem.
         \begin{myIndent}\exPP
            Case 1: $J$ has a max element $\beta$.
            \begin{myIndent}
               Then since we know there exists $h_\beta: S_\beta \longrightarrow C$ satisfying $(*)$, we\\ can apply part (b) to get a function $h$ from $J = S_\beta \cup \{\beta\}$ to $C$\\ satisfying $(*)$.\newpage
            \end{myIndent}

            Case 2: $J$ has no max element.
            \begin{myIndent}
               Then $J = \bigcup\limits_{\beta \in J}S_\beta$.\\ And since there exists $h_\beta: S_\beta \longrightarrow C$ satisfying $(*)$ for all $\beta \in J$, we\\ can thus apply part (c) to get a function $h$ from $J = \bigcup\limits_{\beta \in J}S_\beta$ to $C$\\ [-9pt] satisfying $(*)$.\retTwo\retTwo
            \end{myIndent}
         \end{myIndent}
      \end{itemize}
   \end{myIndent}

   \hOne
   \mySepTwo

   \dispDate{9/17/2024}

   \blab{Theorem (The Hausdorff maximum principle):} Let $A$ be a set and let $\prec$ be a strict partial order on $A$. Then there exists a maximal simply ordered subset $B$ of $A$.
   
   \begin{myTindent}\hThree
      In other words, there exists a subset $B$ of $A$ such that $B$ is simply ordered by $\prec$ and no subset of $A$ that properly contains $B$ is simply ordered by $\prec$.\retTwo
   \end{myTindent}
   
   \begin{myIndent}\hTwo
      Proof:\\
      To start out, let $J$ be a set well-ordered by $<$ such that the elements of $A$\\ are indexed in a bijective fashion by the elements of $J$. In other words,\\ $A = \{a_\alpha \in A \mid \alpha \in J\}$.
      \begin{myIndent}\myComment
         Assuming the well-ordering theorem, we know that $J$ exists. Specifically let $J$ refer to the same set as $A$ but equip $J$ with the well-ordering $<$ that we know exists instead of the partial ordering $\prec$ which we equipped $A$.\retTwo
      \end{myIndent}

      Now our goal is to construct a function $h: J \longrightarrow \{0, 1\}$ such that $h(\alpha) = 1$ if $a_\alpha$\\ is in our maximal simply ordered subset of $A$ and $h(\alpha) = 0$ otherwise. To do this, we rely on the \textbf{general principle of recursive definition}.\retTwo

      \begin{myIndent}
         \begin{myClosureOne}{5.1}
            \\ [-20pt]\textbf{Theorem: (General principle of recursive definition):}
            
            \begin{myIndent}
               Let $J$ be a well-ordered set and $C$ be any set. Given a\newline function $\rho: \mathcal{F} \longrightarrow C$ where $\mathcal{F}$ is the set of all functions\newline mapping sections of $J$ into $C$, we have that there exists a\newline unique functon $h: J \longrightarrow C$ satisfying that $h(\alpha) = \rho(h|_{S_\alpha})$\newline for all $\alpha \in J$.
               \begin{myIndent}\myComment
                  The proof for this is supplementary exercise 1. of this\newline chapter. But I'm not going to do it because it's mostly\newline identical to exercise 10.10.
               \end{myIndent}
            \end{myIndent}
         \end{myClosureOne}\retTwo
      \end{myIndent}

      Given any $\alpha \in J$ and $f: S_\alpha \longrightarrow \{0, 1\}$, define $\rho(\alpha) = 1$ if $a_\alpha \in A$ is comparable to all $a_\beta \in A$ such that $\beta \in f^{-1}(1)$ (the preimage of $1$).
      \begin{myTindent}\myComment
         Note that $a_\alpha$ is comparable to $a_\beta$ if either $a_\alpha \prec a_\beta$ or $a_\beta \prec a_\alpha$.\newpage
      \end{myTindent}

      Then by the general principle of recursive definition, we know a unique function $h: J \longrightarrow \{0, 1\}$ exists such that for all $\alpha \in J$, we have that $h(\alpha) = 1$ only when $a_\alpha$ is comparable to all $a_\beta \in A$ such that $\beta \in S_\alpha$ and $h(\beta) = 1$.\retTwo

      Let $B = \{a_\alpha \in A \mid \alpha \in J \text{ and } h(\alpha) = 1\}$. Then given any $a_\alpha, a_\beta \in B$ such that $\alpha < \beta$, we know that either $a_\alpha \prec a_\beta$ or $a_\beta \prec a_\alpha$. Hence, $B$ is simply ordered by $\prec$. At the same time, if $a_\gamma \notin B$, then we know $h(\gamma) = 0$, meaning there exists $a_\alpha \in B$ such that $\alpha < \gamma$ and  $a_\gamma$ is not comparable to $a_\alpha$. This shows that any set properly containing $B$ is not simply ordered by $\prec$.

      
      \begin{myIndent}\myComment
         Note that the maximal simply ordered subset $B$ is not unique. In fact, choosing a different well-ordering of $J$ is likely to give a completely different maximal simply ordered subset.\retTwo
         
         Also, $B$ is not empty because any set with one element is simply ordered by $\prec$.\retTwo\retTwo
      \end{myIndent}
   \end{myIndent}

   Let $A$ be a set and let $\prec$ be a strict partial order on $A$. If $B$ is a subset of $A$, we say an \udefine{upper bound} on $B$ is an element $c$ of $A$ such that for every $b \in B$, either $b = c$ or $b \prec c$. A \udefine{maximal element} of $A$ is an element $m$ of $A$ such that for no element $a$ of $A$ does the relation $m \prec a$ hold.\retTwo

   \blab{Zorn's Lemma:} Let $A$ be a set that is strictly partially ordered. If every simply ordered subset of $A$ has an upper bound in $A$, then $A$ has a maximal element.

   \begin{myIndent}\hTwo
      Proof:\\
      By the Hausdorff maximum principle, there exists a maximal simply ordered subset $B$ of $A$. Let $c$ be an element of $A$ that is an upperbound to $B$. We claim that $c$ is a maximal element of $A$. For suppose there exists $d \in A$ such that $c \prec d$. We know $d \notin B$ since that would imply $d \prec c$. But by the transitivity of $\prec$, we know that $b \preceq c \prec d \Longrightarrow b \prec d$ for all $b \in B$. Hence, $B \cup \{d\}$ is simply ordered by $\prec$. This contradicts that $B$ is a maximal simply ordered subset of $A$.\retTwo\retTwo
   \end{myIndent}

   \exOne\blab{Exercise 11.1:} If $a$ and $b$ are real numbers, define $a \prec b$ if $b - a$ is positive and rational. 
   \begin{itemize}\exTwo
      \item It's easy to show that $\prec$ is a strict partial order. After all, for all $a \in \mathbb{R}$, we have that $a - a$ is not positive. Also, if $a \prec b$ and $b \prec c$, then we know that $b - a = p$ and $c - b = q$ where $p, q \in \mathbb{Q}_+$. But then $c - a = c - b + b - a = p + q \in \mathbb{Q}_+$. So $a \prec c$.\retTwo
      \item Clearly, given any $x \in \mathbb{R}$, the maximal simply ordered set containing $x$ is the set\\ $\{x + p \mid p \in \mathbb{Q}\}$.\newpage
   \end{itemize}

   \pracOne\mySepTwo

   \blab{Tangent:} I never got around to writing this down last quarter. So here's a proof that\\ assuming the axiom of choice, non-Lebesgue measurable sets exist.
   
   \begin{myIndent}\pracTwo
      Let $\mathcal{B}$ be the collection of sets of the form $S_x = [0, 1] \cap \{x + p \mid p \in \mathbb{Q}\}$ where $x$ is any real number. Obviously, all the sets in $\mathcal{B}$ are nonempty. We also claim that all the sets in $\mathcal{B}$ are disjoint. For suppose $S_x, S_y \in \mathcal{B}$ and $S_x \cap S_y \neq \emptyset$. Then fix $c \in S_x \cap S_y$ and consider any $a \in S_x$ and $b \in S_y$.\retTwo

      We know $c - x = p_1$, $a - x = p_2$, $c - y = q_1$, and $b - y = q_2$ where $p_1, p_2, q_1, q_2 \in \mathbb{Q}$. Thus, we have that $a - y = (a - x) + (x - c) + (c - y) = p_2 - p_1 + q_1 \in \mathbb{Q}$. Similarly, we have that $b - x = (b - y) + (y - c) + (c - x) = q_2 - q_1 + p_1 \in \mathbb{Q}$. This tells us that $a \in S_y$ and $b \in S_x$. And since this works for all $a \in S_x$ and $b \in S_y$, we thus must have that $S_x = S_y$.\retTwo

      Now using the axiom of choice, let $V$ be a set containing one element from each set in $\mathcal{B}$.\retTwo

      To show that $V$ is nonmeasurable, we'll reach a contradiction by supposing $V$ is\\ measurable. Let $q_1, q_2, \ldots$ be an enumeration of all the rational numbers in the set $[-1, 1]$. Then having defined $V + q_n = \{v + q_n \mid v \in V\}$, consider the set: $\hspace{-0.3em}\bigcup\limits_{n \in \mathbb{Z}_+}\hspace{-0.3em}(V + q_n)$.\\

      Obviously, since $V \subseteq [0,1]$, we know that $\hspace{-0.3em}\bigcup\limits_{n \in \mathbb{Z}_+}\hspace{-0.3em}(V + q_n) \subseteq [-1, 2]$.\\

      Also, consider any $x \in [0, 1]$ and let $v$ be the element of $V$ which was chosen from the set $S_x \in \mathcal{B}$. Then $v - x = p$ where $p$ is some rational number in $[-1, 1]$. So, we also know that $[0, 1] \subseteq \hspace{-0.3em}\bigcup\limits_{n \in \mathbb{Z}_+}\hspace{-0.3em}(V + q_n)$. This means that $1 \leq \mu(\hspace{-0.3em}\bigcup\limits_{n \in \mathbb{Z}_+}\hspace{-0.3em}(V + q_n)) \leq 3$.\\

      But now note that for any $n, m \in \mathbb{Z}_+$, we have that $n \neq m \Longrightarrow V + q_n \cap V + q_m = \emptyset$. To prove this, assume $V + q_n \cap V + q_m \neq \emptyset$. Thus, there would exist $v, u \in V$ such that $v + q_n = u + q_m$. In turn, we'd have that $v - u = q_m - q_n \in \mathbb{Q}$, which means that\\ $v \in S_u$. However, this contradicts that $V$ has only one element of $S_u$.\\ [-4pt]

      Now since $\mu$ is countably additive, we have that $\mu(\hspace{-0.3em}\bigcup\limits_{n \in \mathbb{Z}_+}\hspace{-0.3em}(V + q_n)) = \sum\limits_{n = 1}^\infty \mu(V + q_n)$.\\
      
      Finally, note that $\mu(V) = \mu(V + q_n)$ for all $n$. Thus $\sum\limits_{n = 1}^\infty \mu(V + q_n) = \sum\limits_{n = 1}^\infty \mu(V)$ is either\\ [-6pt] $0$ or $\infty$.\retTwo
      
      But this contradicts our earlier finding that the measure was between $1$ and $3$. So, we conclude that $V \notin \mathcal{M}(\mu)$. $\blacksquare$
   \end{myIndent}

   \mySepTwo 

   \exOne\blab{Exercise 11.2:} 
   \begin{itemize}
      \item[(a)] Let $\prec$ be a strict partial order on the set $A$. Define a (non-strict partial) relation $\preceq$ on $A$ by letting $a \preceq b$ if either $a \prec b$ or $a = b$. Show that this relation has the following properties which are called the \textit{partial order axioms}:\newpage
      
      \begin{itemize}\exTwo
         \item[(i)] $a \preceq a$ for all $a \in A$
         \begin{myIndent}
            This is true because $a = a$ for all $x \in A$.
         \end{myIndent}

         \item[(ii)] $a \preceq b \text{ and } b \preceq a \Longrightarrow a = b$.
         \begin{myIndent}
            Given any $a, b\in A$ such that $a \preceq b \text{ and } b \preceq a$, if $a \neq b$, then we'd have that $a \prec b$ and $b \prec a$. This gives a contradiction since $a \prec b \prec a \Longrightarrow a \prec a$ which is not allowed.
         \end{myIndent}

         \item[(iii)] $a \preceq b \text{ and } b \preceq c \Longrightarrow a \preceq c$
         \begin{myIndent}
            Proving this is a matter of considering six rather trivial cases.\retTwo
         \end{myIndent}
      \end{itemize}

      \exOne\item[(b)] Let $P$ be a relation on $A$ satisfying the three axioms above. Define a relation $S$ on $A$ by letting $a\mathop{S}b$ if $a\mathop{P}b$ and $a \neq b$. Show that $S$ is a strict partial order on $A$.
      
      \begin{myIndent}\exTwo
         Obviously, $a \hspace{-0.1em}\not{\hspace{-0.24em}\mathop{S}}\hspace{0.2em} a$ for all $a \in A$ since $a = a$ for all $a \in A$. Meanwhile, suppose\\ $a \mathop{S} b$ and $b \mathop{S} c$. Then we know that $a \mathop{P} b$ and $b \mathop{P} c$, meaning that $a \mathop{P} c$. So we just need to show that $a \neq c$ and then we will have proven that $a \mathop{S} c$.

         Suppose $a = c$. Then we know that $c \mathop{P} a$ and $a \mathop{P} b$, meaning that $c \mathop{P} b$. But then since $b \mathop{P} c$, we know that $b = c$. This contradicts that $b \mathop{S} c$.\retTwo
      \end{myIndent}
   \end{itemize}

   {\hOne In the next exercises we will explore some equivalent theorems to the Hausdorff maximum principle and Zorn's lemma.}
	
   \blab{Exercise 11.5:} Show that Zorn's lemma implies the following:
   
   \begin{myIndent}
      \blab{Kuratowski's Lemma:} Let $\mathcal{A}$ be a collection of sets. Suppose that for every subcollection $\mathcal{B}$ of $\mathcal{A}$ that is simply ordered by proper inclusion, the union of the elements of $\mathcal{B}$ belongs to $\mathcal{A}$. Then $\mathcal{A}$ has an element that is properly contained in no other element of $\mathcal{A}$.\retTwo
      
      \begin{myIndent}\exTwo\color{RedViolet}
         To be clear, given any $A, B \in \mathcal{A}$, we defined above that $A \prec B$ if $A \subset B$. Importantly, our assumption about $\mathcal{A}$ means that every subcollection $\mathcal{B}$ of $\mathcal{A}$ that is simply ordered by $\prec$ has an upper bound in $\mathcal{A}$: $\bigcup\limits_{B \in \mathcal{B}}\hspace{-0.2em}B$.\retTwo
         
         Thus by Zorn's lemma, we know that $\mathcal{A}$ has a maximal element $C$. And since there is no element $D \in \mathcal{A}$ such that $C \prec D$, we know that $C$ is properly contained by no sets in $\mathcal{A}$.\retTwo
      \end{myIndent}
   \end{myIndent}

   \blab{Exercise 11.6:} A collection $\mathcal{A}$ of subsets of a set $X$ is said to be of \textit{finite type}\\ provided that a subset $B$ of $X$ belongs to $\mathcal{A}$ if and only if every finite subset\\ of $B$ belongs to $\mathcal{A}$. Show that the Kuratowski lemma implies the following:

   
   \begin{myIndent}
      \blab{Tukey's Lemma:} Let $\mathcal{A}$ be a collection of sets. If $\mathcal{A}$ is of finite type, then $\mathcal{A}$ has an element that is properly contained in no other element of $\mathcal{A}$.\newpage
      
      \begin{myIndent}\exTwo\color{RedViolet}
         To start off I want to clarify that $\mathcal{A}$ being of finite types means both that:
         \begin{enumerate}
            \item[1.] For each $A \in \mathcal{A}$, every finite subset of $A$ belongs to $\mathcal{A}$.
            \item[2.] If every finite subset of a given set $A$ belongs to $\mathcal{A}$, then $A$ belongs to $\mathcal{A}$.\\
         \end{enumerate}

         Now let $\mathcal{B}$ be any subcollection of $\mathcal{A}$ that is simply ordered by proper\\ inclusion. Next, consider the set $S = \bigcup\limits_{B \in \mathcal{B}}B$. We want to show that any\\ [-8pt] finite subset of $S$ is in $\mathcal{A}$.\retTwo

         To do this, let $n \in \mathbb{Z}_+$ and consider any subset $\{b_1, b_2, \ldots, b_n\}$ of $S$ with $n$ elements. Note that for each $1 \leq i \leq n$, there exists $B_i \in \mathcal{B}$ such that $b_i \in B_i$. Then since $\{B_1, B_2, \ldots B_n\}$ is a simply ordered finite set, we know that it has a maximum element $B_m$ such that $B_i \subseteq B_m$ for all $i$. Hence, we have that $\{b_1, b_2, \ldots b_n\}$ is contained by some $B_m$ in $\{B_1, B_2, \ldots, B_n\} \subseteq \mathcal{B}$. Because $\mathcal{A}$ is of finite type, this tells us that $\{b_1, b_2, \ldots b_n\} \in \mathcal{A}$.\retTwo

         Since we showed above that any finite subset of $S$ is in $\mathcal{A}$, we can thus conclude because $\mathcal{A}$ is of finite type that $S \in \mathcal{A}$. And so, we have now proven the hypothesis of Kuratowski's lemma, meaning that $\mathcal{A}$ must have a set that is properly contained in other element of $\mathcal{A}$.\retTwo
      \end{myIndent}
   \end{myIndent}

   \blab{Exercise 11.7:} Show that the Tukey lemma implies the Hausdorff maximum\\ principle.

   \begin{myIndent}\exTwo\color{RedViolet}
      Let $A$ be a set with the strict partial order $\prec$. Then let $\mathcal{A}$ be the collection of all subsets of $A$ that are simply ordered by $\prec$. We shall show below that $\mathcal{A}$ is of finite type.
      
      \begin{enumerate}
         \item Suppose $B \in \mathcal{A}$. Then given any subset $C$ of $B$ (finite or not), we know that $C$ is also simply ordered by $\prec$. So $C \in \mathcal{A}$.
         \item Let $B \subseteq A$ and suppose every finite subset of $B$ is in $\mathcal{A}$. Then given any two different elements $b_1, b_2 \in B$, we know that $\{b_1, b_2\} \in \mathcal{A}$, meaning that either $b_1 \prec b_2$ or $b_2 \prec b_1$. In other words, $B$ is simply ordered by $\prec$, meaning that $B \in \mathcal{A}$.\retTwo
      \end{enumerate}

      Because $\mathcal{A}$ is of finite type, we know that $\mathcal{A}$ has an element that is properly\\ contained in no other element of $\mathcal{A}$. Or in other words, there exists a subset of $A$ which is simply ordered by $\prec$ and not properly contained in any other subset of $A$ that is simply ordered by $\prec$.\newpage
   \end{myIndent}

   \hOne
   \dispDate{9/19/2024}
   % I realize I have another week before I take my first abstract algebra class. However, the next problem has to do with vector spaces.\newpage
   % 
   % If $A$ is a subset of the vector space $V$, we say a vector belongs to the \textit{span} of $A$ if it equals a finite linear combination of elements of $A$.
   % 
   %    
   % \begin{myIndent}\myComment
   %    In other words, $w \in \mathrm{span}(A)$ if there exists a finite subset $\{v_1, \ldots, v_n\}$ of $A$ and scalars: $c_1, \ldots, c_n$ such that $w = c_1v_1 + \ldots c_nv_n$.\retTwo
   % 
   %    I didn't realize before now but we exclude linear combinations of infinite sets of vectors in $A$. Apparently,
   % \end{myIndent}
   % 
   % \exOne
   % \blab{Exercise 11.8:}

\begin{tabular}{p{2.2in} p{4in}}
   % In the past 14 pages we've\newline learned a lot about the\newline axiom of choice. All the\newline {\color{blue}blue arrows} in the\newline diagram to the right\newline represent proofs we've\newline already done. Meanwhile,\newline the {\color{red}red arrows} represent\newline proofs that Munkres left\newline to the supplementary\newline exercises of section 1 of\newline his book. We're gonna do\newline those proofs now.
   &
   % https://q.uiver.app/#q=WzAsNyxbMCwxLCJcXHRleHR7IEF4aW9tIG9mIENob2ljZSB9Il0sWzAsMiwiXFx0ZXh0e1dlbGwgT3JkZXJpbmcgVGhlb3JlbX0iXSxbMSwwLCJcXHRleHR7Tm9uLW1lYXN1cmFibGUgU2V0c30iXSxbMCwzLCJcXHRleHR7TWF4aW11bSBQcmluY2lwbGV9Il0sWzAsNCwiXFx0ZXh0e1pvcm4ncyBMZW1tYX0iXSxbMSw0LCJcXHRleHR7S3VyYXRvd3NraSdzIExlbW1hfSJdLFsxLDMsIlxcdGV4dHtUdWtleSdzIExlbW1hfSJdLFswLDIsIiIsMix7ImxldmVsIjoyLCJjb2xvdXIiOlsyNDAsNjAsNjBdfV0sWzAsMSwiIiwwLHsib2Zmc2V0IjotNSwibGV2ZWwiOjIsImNvbG91ciI6WzAsNjAsNjBdfV0sWzEsMCwiIiwwLHsib2Zmc2V0IjotNSwibGV2ZWwiOjIsImNvbG91ciI6WzI0MCw2MCw2MF19XSxbMSwzLCIiLDAseyJvZmZzZXQiOi01LCJsZXZlbCI6MiwiY29sb3VyIjpbMjQwLDYwLDYwXX1dLFszLDEsIiIsMCx7Im9mZnNldCI6LTUsImxldmVsIjoyLCJjb2xvdXIiOlswLDYwLDYwXX1dLFszLDQsIiIsMCx7ImxldmVsIjoyLCJjb2xvdXIiOlsyNDAsNjAsNjBdfV0sWzQsNSwiIiwwLHsibGV2ZWwiOjIsImNvbG91ciI6WzI0MCw2MCw2MF19XSxbNSw2LCIiLDAseyJsZXZlbCI6MiwiY29sb3VyIjpbMjQwLDYwLDYwXX1dLFs2LDMsIiIsMCx7ImxldmVsIjoyLCJjb2xvdXIiOlsyNDAsNjAsNjBdfV1d
   \begin{tikzcd}[sep=scriptsize]
      & {\text{Non-measurable Sets}} \\
      {\text{ Axiom of Choice }} \\
      {\text{Well Ordering Theorem}} \\
      {\text{Maximum Principle}} & {\text{Tukey's Lemma}} \\
      {\text{Zorn's Lemma}} & {\text{Kuratowski's Lemma}}
      \arrow[color={rgb,255:red,92;green,92;blue,214}, Rightarrow, from=2-1, to=1-2]
      \arrow[shift left=5, color={rgb,255:red,214;green,92;blue,92}, Rightarrow, from=2-1, to=3-1]
      \arrow[shift left=5, color={rgb,255:red,92;green,92;blue,214}, Rightarrow, from=3-1, to=2-1]
      \arrow[shift left=5, color={rgb,255:red,92;green,92;blue,214}, Rightarrow, from=3-1, to=4-1]
      \arrow[shift left=5, color={rgb,255:red,214;green,92;blue,92}, Rightarrow, from=4-1, to=3-1]
      \arrow[color={rgb,255:red,92;green,92;blue,214}, Rightarrow, from=4-1, to=5-1]
      \arrow[color={rgb,255:red,92;green,92;blue,214}, Rightarrow, from=5-1, to=5-2]
      \arrow[color={rgb,255:red,92;green,92;blue,214}, Rightarrow, from=5-2, to=4-2]
      \arrow[color={rgb,255:red,92;green,92;blue,214}, Rightarrow, from=4-2, to=4-1]
   \end{tikzcd}
\end{tabular}
\\ [-2.6in]
\begin{tabular}{p{2.2in} p{4in}}
   In the past 14 pages, we've\newline learned a lot about the\newline axiom of choice. All the\newline {\color{blue}blue arrows} in the\newline diagram to the right\newline represent proofs we've\newline already done. Meanwhile,\newline the {\color{red}red arrows} represent\newline proofs that Munkres left\newline to the supplementary\newline exercises of section 1 of\newline his book. We're gonna do\newline those proofs now.
   &
\end{tabular}\retTwo

\exOne
\blab{Exercise 1: (General principle of recursive definition)}
\begin{myIndent}\exTwo\color{RedViolet}
   We already addressed this before. I'm skipping proving this because the proof is mostly identical to exercise 10.10. In fact, exercise 10.10 is just this exercise but with a specific $\rho: \mathcal{F} \longrightarrow C$.\retTwo
\end{myIndent} 

\blab{Exercise 2:} 
\begin{itemize}
   \item[(a)] Let $J$ and $E$ be well-ordered sets and let $h: J \longrightarrow E$. Show that the following two statement are equivalent:
   \begin{enumerate}
      \item[(i)] $h$ is order preserving and its image is $E$ or a section of $E$.
      \item[(ii)] $h(\alpha) = \text{smallest}(E - h(S_\alpha))$ for all $\alpha \in J$. 
      
      \begin{myIndent}\exTwo\color{RedViolet}
         (i) $\Longrightarrow$ (ii):\\
         Given any $\alpha \in J$, we know that $h(\alpha)$ must be an upper bound to $h(S_\alpha)$. Now suppose $\exists \beta \in S_{h(\alpha)}$ such that $\beta \notin h(S_\alpha)$. Because of our assumption about the image of $h$, we know that $\beta \in h(J)$, meaning there exists $\gamma \in J$ such that $h(\gamma) = \beta$. But because $h$ is order-preserving, we must have that $\beta < f(\alpha) \Longrightarrow \gamma < \alpha$. This contradicts that $\beta \notin h(S_\alpha)$.\retTwo

         With that, we've now shown that $h(S_\alpha) = S_{h(\alpha)}$. In turn, this shows that $h(\alpha)$ is the smallest element in $E - h(S_\alpha)$.\retTwo

         (ii) $\Longrightarrow$ (i):\\
         It's easy to show $h$ is order preserving. Let $\alpha, \beta \in J$ such that $\alpha < \beta$.\\ Then $h(S_\alpha) \subset h(S_\beta)$, meaning that $E - h(S_\beta) \subset E - h(S_\alpha)$. And\\ since the least element of $E - h(S_\alpha)$ is not in $E - h(S_\beta)$, that means that\\ $h(\alpha) = \text{smallest}(E - h(S_\alpha)) < \text{smallest}(E - h(S_\beta)) = h(\beta)$.\newpage

         As for showing the other property of $h$, let $J_0 = \{\alpha \in J \mid h(S_\alpha) = S_{h(\alpha)}\}$. Now suppose that for some $\alpha \in J$, we have that $S_\alpha \subseteq J_0$. Then we can show that $\alpha \in J_0$.
         
         \begin{myIndent}\exPP
            Case 1: $\alpha$ has an immediate predecessor $\beta$.
            
            \begin{myIndent}
               Then $S_\alpha = S_\beta \cup \{\beta\}$, meaning that:
   
               {\centering $h(S_\alpha) = h(S_\beta) \cup \{h(\beta)\} = S_{h(\beta)} \cup \{h(\beta)\}$.\retTwo\par}
               
               Since $h(\alpha)$ is the least element of $E$ not in $h(S_\alpha)$. We can thus say that $S_{h(\beta)} \cup \{h(\beta)\} = S_{h(\alpha)}$.\retTwo
            \end{myIndent}

            Case 2: $\alpha$ has no immediate predecessor.

            \begin{myIndent}
               Then we have that $h(S_\alpha) = h(\hspace{-0.2em}\bigcup\limits_{\beta \in S_\alpha}\hspace{-0.4em}S_\beta) = \hspace{-0.2em}\bigcup\limits_{\beta \in S_\alpha}\hspace{-0.4em}h(S_\beta) = \hspace{-0.2em}\bigcup\limits_{\beta \in S_\alpha}\hspace{-0.4em}S_{h(\beta)}$.\retTwo

               Hence, $h(S_\alpha)$ is a section of $E$, and since $h(\alpha)$ is the least element not in that section, we can conclude that $h(S_\alpha) = S_{h(\alpha)}$.\retTwo
            \end{myIndent}
         \end{myIndent}

         By transfinite induction, we thus know that $J_0 = J$. So finally, we consider two cases.\retTwo

         \begin{myIndent}\exPP
            Case 1: $J$ has a max element $\alpha$.
            
            \begin{myIndent}
               Then $h(J) = h(S_\alpha) \cup \{h(\alpha)\} = S_{h(\alpha)} \cup \{h(\alpha)\}$. And since $h(\alpha)$ is the least element not in $S_{h(\alpha)}$, we thus know that $h(J)$ is either a section of or the whole of $E$.
            \end{myIndent}

            Case 2: $J$ has no max element.

            \begin{myIndent}
               Then $h(J) = h(\hspace{-0.1em}\bigcup\limits_{\alpha \in J}\hspace{-0.3em}S_\alpha) = \hspace{-0.1em}\bigcup\limits_{\alpha \in J}\hspace{-0.3em}h(S_\alpha) = \hspace{-0.1em}\bigcup\limits_{\alpha \in J}\hspace{-0.3em}S_{h(\alpha)}$.\retTwo
               
               So, $h(J)$ is either a section of or the whole of $E$.\retTwo
            \end{myIndent}
         \end{myIndent}
      \end{myIndent}
   \end{enumerate}

   \item[(b)] If $E$ is a well-ordered set, show that no section of $E$ has the same order type as $E$, nor do any two different sections of $E$ have the same order type.
   
   
   \begin{myIndent}\exTwo\color{RedViolet}
      Let $J$ be any well-ordered set. By combining part (a) of this exercise with\\ exercise 10.10 (which is a special case of the general principle of recursive\\ definition), we know that there is at most one order preserving map from $J$\\ to $E$ whose image is either $E$ or a section of $E$. Hence, $J$ can only have the\\ same order type as one of either the entirety of $E$ or one section of $E$.\retTwo

      Based on that fact, we can get an easy contradiction if we assume that the claim of part (b) is false.\newpage
   \end{myIndent}
\end{itemize}

\hOne
\dispDate{9/21/2024}

Unfortunately I tested positive for Covid on the two days ago. So I've been really delirious. However, right now I'm in an airport in the process of moving back out to California (great idea). And since my flight just got delayed, I feel like I might as well kill time and try to do some math.\retTwo

\exOne\blab{Exercise 3:} Let $J$ and $E$ be well-ordered sets, and suppose there is an order-preserving map $k: J \longrightarrow E$. Using exercises 1 and 2, show that $J$ has the order type of one of either $E$ or one section of $E$.\retTwo


\begin{myIndent}\exTwoP
   Pick any $e_0 \in E$. Then define $h: J \longrightarrow E$ by the rule: 
   
   {\centering$h(\alpha) = \left\{
   \begin{matrix}
      \text{smallest}(E - h(S_\alpha)) & \text{ if } h(S_\alpha) \neq E \\
      e_0 & \text{ otherwise }
   \end{matrix}\right.$\retTwo\par} 

   Note that the second case of our definition of $h$ is just included to ensure that $h$ is well-defined before we begin the proof in earnest. I mention that because our goal now is to show that the second case will never apply.\retTwo

   Let $J_0 = \{\alpha \in J \mid h(\alpha)\leq k(\alpha)\}$. Then suppose that for some $\alpha \in J$, we have that $S_\alpha \subseteq J_0$. Because $k$ is order preserving, we know that $k(\alpha) > k(\beta) \geq h(\beta)$ for all $\beta \in S_\alpha$. Hence, $k(\alpha) \notin h(S_\alpha)$, meaning that $h(S_\alpha) \neq E$. So, we conclude that $h(\alpha) = \text{smallest}(E - h(S_\alpha))$. And since $k(\alpha) \in E - h(S_\alpha)$, we thus know that $h(\alpha) \leq k(\alpha)$\retTwo
   
   Therefore, $\alpha \in J_0$. By transfinite induction, this proves that $J = J_0$. The reason this is relevant is that we can now say that $k(\alpha)$ is never in $h(S_\alpha)$ , meaning that $E - h(S_\alpha) \neq \emptyset$. So $h(\alpha)$, will never be determined by the second case of our definition above.\retTwo

   By exercise 2, we know that $h: J \longrightarrow E$ is the unique order-preserving map whose image is either $E$ or a section of $E$. Thus, $J$ has the same order type as exactly one of either the entirety of $E$ or one section of $E$.\retTwo
\end{myIndent}

\blab{Exercise 4:} Use exercises 1-3 to prove the following:

\begin{itemize}
   \item[(a)] If $A$ and $B$ are well-ordered sets, then exactly one of the following three\\ conditions holds: $A$ and $B$ have the same order type, $A$ has the order type of\\ a section of $B$, or $B$ has the order type of a section of $A$.
   
   
   \begin{myIndent}\exTwoP
      To start off, it's relatively easy to show that at most one of the above three cases is true. After all, $A$ having the same order type as $B$ as well as a section of $B$ contradicts exercise 2. Similarly $B$ having the same order type as $A$ as well as a section of $A$ contradicts exercise 2.\newpage

      Meanwhile, to find a contradiction if $A$ has the order type of $S_\beta$ and $B$ has the order type of $S_\alpha$ where $\alpha \in A$ and $\beta \in B$, let $h: A \longrightarrow S_\alpha$ be the function defined by the rule $h(a) = g(f(a))$ where $f$ is the order-preserving bijection from $A$ to $S_\beta$ and $g$ is the order-preserving bijection from $B$ to $S_\alpha$.\retTwo

      Then given any $a, b \in A$, we know that:
      
      {\centering $a < b \Rightarrow f(a) < f(b) \Rightarrow h(a) = g(f(a)) < g(f(b)) = h(b)$.\retTwo\par}

      Hence, $h$ is an order preserving map from $A$ to $S_\alpha$. This gives us a contradiction since exercise 3 would then imply that $A$ has the same order type as either $S_\alpha$ or a section of $S_\alpha$ (which would still be a section of $A$).\retTwo

      Now, what's left to show is that at least one of the three above cases must be true. Unfortunately, the hinted route for showing this uses an exercise I didn't do. And right now I really don't want to do that exercise. So I'm just going to write out the thing I was supposed to have proven earlier.\retTwo

      {\centering\color{VioletRed}
      \begin{myClosureOne}{5}
         \\ [-20pt]\textbf{Exercise 10.8.a:}
         
         \begin{myIndent}
            Let $A_1$ and $A_2$ be disjoint sets well-ordered by $<_1$ and\newline $<_2$ respectively. Then define an order relation on $A_1 \cup A_2$\newline by letting $a < b$ either if $a, b \in A_1$ and $a <_1 b$, or if\newline $a, b \in A_2$ and $a <_2 b$, or if $a \in A_1$ and $b \in A_2$. This is a\newline well-ordering of $A_1 \cup A_2$.
         \end{myIndent}
      \end{myClosureOne}\retTwo\par}

      Let $A^\prime = \{A\} \times A$ and let $B^\prime = \{B\} \times B$. That way, so long as $A \neq B$, we know that $A^\prime$ and $B^\prime$ are disjoint. (The case where $A = B$ is trivial.)\retTwo
      
      It's hopefully obvious that the well-orderings of $A$ and $B$ can be used to well-\\order $A^\prime$ and $B^\prime$. For $A^\prime$, define $(A, a_1) <_{A^\prime} (A, a_2)$ if $a_1 <_A a_2$. Similarly, define the analogous ordering for $B^\prime$. Clearly, $A$ and $A^\prime$ have the same order type, as do $B$ and $B^\prime$. Also, given any $\alpha \in A$ and $\beta \in B$,\myHS $S_\alpha$ and $S_{(A, \alpha)}$ have the same order type, as do $S_\beta$ and $S_{(B, \beta)}$\retTwo

      Next, define a well-ordering on $A^\prime \cup B^\prime$ by letting $a^\prime < b^\prime$ if either $a^\prime, b^\prime \in A^\prime$ and $a^\prime <_{A^\prime} b^\prime$, or if $a^\prime, b^\prime \in B^\prime$ and $a^\prime <_{B^\prime} b^\prime$, or if $a^\prime \in A^\prime$ and $b^\prime \in B^\prime$.\retTwo

      Note that the inclusion function from $B^\prime$ to $A^\prime \cup B^\prime$ is an order-preserving\\ map. Thus, by exercise 3, we know that $B^\prime$ has the order type of one of either $A^\prime \cup B^\prime$ or one section of $A^\prime \cup B^\prime$.

      \begin{myIndent}
         Case 1: $B^\prime$ has the order type of a section $S_\alpha$ of $A^\prime \cup B^\prime$.
         
         \begin{myIndent}\exPP
            If $\alpha \in A^\prime$, then $B^\prime$ has the order type of a section of $A^\prime$, meaning $B$ has the order type of a section of $A$.\newpage

            If $\alpha$ is the first element of $B$, then $B^\prime$ has the same order type as $A^\prime$, meaning $B$ has the same order type as $A$.\retTwo

            If $\alpha \in B^\prime$, then there exists an order preserving bijection from $B^\prime$ to $A^\prime \cup \{b \in B^\prime \mid b <_{B^\prime} \alpha\}$. So let $f$ be the inverse of that bijection but with it's domain restricted to just $A^\prime$. Since $f$ is also an order-preserving map, we know by exercise 3 that $A^\prime$ has the order type of either $B^\prime$ or a section of $B^\prime$. This would mean that $A$ has the order type of either $B$ or a section of $B$.\retTwo
         \end{myIndent}

         Case 2: $B^\prime$ has the order type of $A^\prime \cup B^\prime$.
         \begin{myIndent}\exPP
            Let $f$ be the inverse of the order preserving bijection from $B^\prime$ to $A^\prime$, except with it's inverse restricted to just $A^\prime$. Since $f$ is also an order-preserving map, we know by exercise 3 that $A^\prime$ has the order type of either $B^\prime$ or a section of $B^\prime$. This would mean that $A$ has the order type of either $B$ or a section of $B$.\retTwo
         \end{myIndent}

         With that, we've now shown that at least one of the three cases posed by the exercise will always be true.\retTwo
      \end{myIndent}
   \end{myIndent}

   \item[(b)] Suppose that $A$ and $B$ are well-ordered sets that are uncountable such that every section of $A$ and of $B$ is countable. Show that $A$ and $B$ have the same order type.
   
   \begin{myIndent}\exTwoP
      If $A$ did not have the same order type as $B$, then by part (a) of this exercise we would know that either $A$ has the order type of a section of $B$ or $B$ has the order type of a section of $A$. However, that would suggest the existence of a bijection between a countable set and an uncountable set, which by definition is not possible.\retTwo
   \end{myIndent}
\end{itemize}

\dispDate{9/23/2024}

\blab{Exercise 5:} Let $X$ be any set and let $\mathcal{A}$ be the collection of all pairs $(A, <)$ where $A$ is a subset of $X$ and $<$ is a well-ordering of $A$. Define:

{\center $(A, <) \prec (A^\prime, <^\prime)$ \retTwo\par}

if $(A, <)$ equals a section of $(A^\prime, <^\prime)$.

\begin{myIndent}\myComment
   In other words, $A = S_\alpha = \{a \in A^\prime \mid a <^\prime \alpha \}$ where $\alpha \in A^\prime$, and $<$ is the order relation $<^\prime$ restricted to $A$.
\end{myIndent}


\begin{itemize}
   \item[(a)] Show that $\prec$ is a strict partial order on $\mathcal{A}$.
   
   \begin{myIndent}\exTwoP
      Clearly no $A$ is a section of itself. So $(A, <) \not\prec (A, <)$.\retTwo Also if $(A, <_A) \prec (B, <_B) \prec (C, <_C)$, then we know that $A$ is a section of a section of $C$ (which is still a section). Plus, $<_A$ is just $<_C$ restricted to $<_A$. Hence, $(A, <) \prec (C, <_C)$.\newpage
   \end{myIndent}

   \item[(b)] Let $\mathcal{B}$ be a subcollection of $\mathcal{A}$ that is simply ordered by $\prec$. Define $B^\prime$ to be the union of the sets $B$ for all $(B, <) \in \mathcal{B}$, and define $<^\prime$ to be the union of the relations $<$ for all $(B, <) \in \mathcal{B}$. Show that $(B^\prime, <^\prime)$ is a well-ordered set.
   
   \begin{myIndent}
      \exTwoP
      To start, let's quickly double check that $<^\prime$ is a valid order relation on $B^\prime$.
      \begin{itemize}
         \item[(i)] Given any $b \in B^\prime$, if $b \in B$ for any $(B, <) \in \mathcal{B}$, then we know that\\ $(b, b) \notin {<}$. So $(b, b) \notin {<^\prime}$.\\ [-9pt]
         \item[(ii)] Suppose $a, b \in B^\prime$ such that $(a, b) \notin {<^\prime}$. Then for all $(B, <) \in \mathcal{B}$ such that $a, b \in B$, we know that $(a, b) \notin {<}$, meaning that $(b, a) \in {<}$. So $(b, a) \in {<^\prime}$.\\ [-9pt]
         \item[(iii)] Given $a, b, c \in B^\prime$, suppose $a <^\prime b <^\prime c$. Then there exists $(B_1, <_1)$ and\\ $(B_2, <_2)$ in $\mathcal{B}$ such that $(a, b) \in {<_1}$ and $(b, c) \in {<_2}$. Now by how we defined $\mathcal{B}$, we know that either ${<_1} \subset {<_2}$ or ${<_2} \subset {<_1}$. Thus, we know $(a, b), (b, c) \in \{<_i\}$ for some $i \in \{1, 2\}$. Hence, $(a, c) \in {<_i}$, meaning that $(a, c) \in {<^\prime}$.\retTwo
      \end{itemize}

      Next, we show that $B^\prime$ is well-ordered by $<^\prime$.\retTwo
      
      Let $S \subseteq B^\prime$ be nonempty and pick any element $\beta$ in $S$. Then we know there exists $(B_1, <_1) \in \mathcal{B}$ such that $\beta \in B_1$. Also, $B_1$ is well-ordered by $<$. So let $\alpha$ be the least element (using $<_1$) of $B_1 \cap S$.\retTwo

      We claim that $\alpha$ is the least element (using $<^\prime$) of $S$. To prove this, suppose there exists $c \in S$ such that $c <^\prime a$. Then we know $(c, \alpha) \in {<_2}$ for some\\ $(B_2, <_2) \in \mathcal{B}$. Importantly, $(B_1, <_1) \neq (B_2, <_2)$ since otherwise we'd have chosen $\alpha$ differently. So one must be a section of the other.
      
      \begin{itemize}
         \item[\bullet] If $(B_2, <_2)$ is a section of $(B_1, <_1)$, then we know that ${<_2} \subset {<_1}$ and\\ $c \in B_1 \cap S$. But this contradicts how we chose $\alpha$.\\ [-9pt]
         
         \item[\bullet] If $(B_1, <_1)$ is a section of $(B_2, <_2$), then we know there exists $\gamma \in B_2$ such that $B_1 = S_\gamma \subseteq B_2$. If $c <_2 \gamma$, then we know that $c \in B_1$ and thus $B_1 \cap S$. This contradicts how we chose $\alpha$. So we must have that $\gamma <_2 c$. But then this also gives us a contradiction as $\alpha <_2 \gamma <_2 c \Longrightarrow \alpha <_2 c$, meaning that $\alpha <^\prime c$.\retTwo
      \end{itemize}
   \end{myIndent}

   \item[(c)] [Not in the book...] Given any $\mathcal{B}$ from part (b) of this problem and defining\\ $(B^\prime, <^\prime)$ as before, we have that $(B, <) \preceq (B^\prime, <^\prime)$ for all $(B, <) \in \mathcal{B}$.
   \begin{myIndent}
      \exTwoP
      Consider any $(B_1, <_1) \in \mathcal{B}$. If $B_1 \neq B^\prime$, then we know there exists $\alpha \in B^\prime - B_1$, thus meaning there exists $(B_2, <_2) \in \mathcal{B}$ such that $\alpha \in B_2$. Since $B_2 \not\prec B_1$, we know that $B_1 \prec B_2$, meaning that $B_1 = S_\beta \subseteq B_2$ for some $\beta \in B_2$.\newpage

      Now we know that $\{b \in B^\prime \mid b <^\prime \beta\} \subseteq \{b \in B_2 \mid b <_2 \beta\}$. For suppose\\ there exists $a$ in the former set but not the latter set. Then there must exist\\ $(B_3, <_3)  \in \mathcal{B}$ such that $(a, b) \in {<_3}$.\retTwo

      If $a \in B_2$, then we'd have that $(b, a) \in <_2$. But that would imply that $(a, b)$ and $(b, a)$ are in $<^\prime$ which we know isn't possible. So we know that $B_3 \not\subseteq B_2$.\retTwo

      Since $\mathcal{B}$ is simply ordered by $\prec$ and we can't have that $B_3 \prec B_2$, we know that $B_2 \prec B_3$. So $B_2 = S_\gamma$ where $\gamma \in S_3$. Now $a <_3 \gamma$ would contradict that\\ $a \notin B_2$. So we must have that $\gamma <_3 a$. However, we also must have that\\ $b <_3 \gamma$, which contradicts that $a <_3 b$.\retTwo

      Hence, we've shown that $(B_1, <_1) \neq (B^\prime, <^\prime)$ implies that $(B_1, <_1) \prec (B^\prime, <^\prime)$.\retTwo
   \end{myIndent}
\end{itemize}

\blab{Exercise 6:} Use exercise 5 to prove that the maximum principle implies the well-\\ordering theorem.

\begin{myIndent}\exTwoP
   Let $X$ be any set and construct $\mathcal{A}$ and $\prec$ as before in exercise 5. By the maximal principle, we know there exists $\mathcal{B} \subseteq \mathcal{A}$ such that $\mathcal{B}$ is simply ordered by $\prec$ and no proper superset of $\mathcal{B}$ is simply ordered by $\prec$.\retTwo

   Next, construct $B^\prime$ and $<^\prime$ as in exercise 5.b. We claim that $B^\prime = X$. To see this, suppose there exists $c \in X - B^\prime$. Then $B^\prime \cup \{c\}$ is well-ordered by the order\\ relation: ${<^\prime}\cup\{(b, c) \mid b \in B^\prime\}$. Hence $(B^\prime \cup \{c\}, {<^\prime}\cup\{(b, c) \mid b \in B^\prime\}) \in \mathcal{A}$.\retTwo

   At the same time, note that $(B^\prime, <^\prime) \prec (B^\prime \cup \{c\}, {<^\prime}\cup\{(b, c) \mid b \in B^\prime\})$. And since we have that $(B, <) \preceq (B^\prime, <^\prime)$ for all $(B, <) \in \mathcal{B}$, we thus know that for any $(B, <) \in \mathcal{B}$:

   {\centering $(B, <) \prec (B^\prime \cup \{c\}, {<^\prime}\cup\{(b, c) \mid b \in B^\prime\})$.\retTwo\par}

   This tells us both that $(B^\prime \cup \{c\}, {<^\prime}\cup\{(b, c) \mid b \in B^\prime\}) \notin \mathcal{B}$ and that\\ $(B^\prime \cup \{c\}, {<^\prime}\cup\{(b, c) \mid b \in B^\prime\})$ is comparable with all elements of $\mathcal{B}$. But\\ that contradicts that $\mathcal{B}$ is a maximal simply ordered subset of $\mathcal{A}$.\retTwo

   So we must have that $B^\prime = X$. And thus by exercise 5.b, we know that a well-ordering of $X$ exists.\newpage
\end{myIndent}

\dispDate{9/24/2024}

\blab{Exercise 7:} Use exercises 1-5 to prove that the choice axiom implies the well-ordering theorem.\retTwo

Let $X$ be a set and $c$ be a fixed choice function for the nonempty subsets of $X$. If $T$ is a subset of $X$ and $<$ is a relation on $T$, we say that $(T, <)$ is a \udefine{tower} in $X$ if $<$ is a well-ordering of $T$ and if for each $x \in T$, $x = c(X - S_x(T))$ where $S_x(T)$ is the section of $T$ by $x$.
\begin{myTindent}\myComment
   Well, shit. I wish I was given that notation for specifying which set I was\\ taking a section of before I did exercise 2. $h(S_x(J)) = S_{h(x)}(E)$ is a lot\\ clearer notation than just $h(S_x) = S_{h(x)}$\retTwo
\end{myTindent}

\begin{itemize}
   \item[(a)] Let $(T_1, <_1)$ and $(T_2, <_2)$ be two towers in $X$. Show that either these two\\ ordered sets are the same or one equals a section of the other.
   
   \begin{myIndent}\exTwoP
      By applying exercise 4 and switching indices if necessary, we know that $T_1$ has the order type of one of either $T_2$ or one section of $T_2$. In other words, there exists an order preserving map $h: T_1 \longrightarrow T_2$ such that $h(T_1)$ equals $T_2$ or a section of $T_2$.\retTwo

      Now we assert that given any $x \in T_1$,\myHS $h(x) = x$. To prove this, first note\\ that because of transfinite induction, we can assume that $h(x) = x$ for all $x$ in $S_x(T_1)$. This means that we can assume $h(S_x(T_1)) = S_x(T_1)$. Also, as part of doing exercise 2, we proved that $h$ must satisfy that $h(S_x(T_1)) = S_{h(x)}(T_2)$.\\ Hence, $S_x(T_1) = S_{h(x)}(T_2)$. This let's us conclude that:

      {\centering $x = c(X - S_x(T_1)) = c(X - S_{h(x)}(T_2)) = h(x)$.\retTwo\par}

      With that we now know that $h(T_1) = T_1$. So $T_1$ equals either $T_2$ or a section of $T_2$.\retTwo
   \end{myIndent}

   \item[(b)] If $(T, <)$ is a tower in $X$ and $T \neq X$, then there is a tower in $X$ of which $(T, <)$ is a section.
   
   \begin{myIndent}\exTwoP
      Since $T \neq X$, let $y = c(X - T)$. Then define $T^\prime = T \cup \{y\}$ and\\ ${<^\prime} = {<} \cup \{(x, y) \mid x \in T\}$. Clearly, $(T^\prime, <^\prime)$ is a tower which contains\\ $(T, <)$ as a section.\retTwo

      
      \begin{myIndent}\exPP
         Clearly $T^\prime$ is well-ordered by $<^\prime$.\retTwo
         
         Also, if $x \in T^\prime - \{y\}$, then we have that $c(X - S_x(T^\prime)) = c(X - S_x(T)) = x$.\\ Plus, we know that $c(X - S_y(T^\prime)) = c(X - T) = y$.\newpage
      \end{myIndent}
   \end{myIndent}

   \item[(c)] Let $\{(T_k, <_k) \mid k \in K\}$ be the collection of all towers in $X$. Then define:
   
   {\centering $ T = \bigcup\limits_{k \in K}T_k$\hspace{0.2em} and \hspace{0.2em}${<} = \bigcup\limits_{k \in K}\hspace{-0.2em}{<_k}$. \retTwo\par}

   Show that $(T, <)$ is a tower in $X$. Conclude that $T = X$.

   \begin{myIndent}\exTwoP
      If we define $\mathcal{A}$ and $\prec$ from $X$ as we did in exercise 5, we can see from part (a) of this problem that $\{(T_k, <_k) \mid k \in K\}$ is a subset of $\mathcal{A}$ that is simply ordered by $\prec$. Thus, from part (b) of exercise 5, we know that $T$ is well-ordered by $<$.\retTwo

      To prove that $T$ is a tower, consider any $y \in T$. Then we know there exists $k \in K$ such that $y \in T_k$. Furthermore, we know that $y = c(X - S_y(T_k))$. By, part (c) of exercise 5, we know that $T_k$ is either a section of $T$ or all of $T$. Hence, $S_y(T) = S_y(T_k)$. And thus we have that $y = c(X - S_y(T))$.\retTwo

      Now that we have shown $(T, <)$ is a tower in $X$, we get an easy contradiction if $T \neq X$. This is because $T$ must contain all towers, but $T$ not equalling $X$ would imply the existence of a tower not contained by $T$ due to part (b) of this exercise.\retTwo

      And since $T = X$, we thus have that $<$ is a well-ordering of $X$. $\blacksquare$\retTwo
   \end{myIndent}
\end{itemize}

\hOne
I'm gonna skip doing exercise 8 of the supplementary exercise. Basically it shows that you can construct a well-ordered set with higher cardinality than an arbitrary well-ordered set, all without using the axiom of choice. Also, while that does mean we can construct a minimal uncountable well-ordered set without using the axiom of choice, theorem 10.3 requires the axiom of choice to prove. So almost nothing we discovered about a minimal uncountable well-ordered set can be proven without the axiom of choice.\retTwo

\mySepTwo

\dispDate{9/25/2024}

I'm gonna try to cram as much topology as I can today before class starts tomorrow. After all, I suspect and fear that a bunch of this will be necessary at some point in 240. As before, I'm shamelessly ripping off James Munkres' book.\retTwo

A \udefine{Topology} on a set $X$ is a collection $\mathcal{T}$ of subsets of $X$ having the properties:
\begin{enumerate}
   \item $\emptyset$ and $X$ are in $\mathcal{T}$.
   \item The union of the elements of any subcollection of $\mathcal{T}$ is in $\mathcal{T}$.
   \item The intersection of the elements of any finite subcollection of $\mathcal{T}$ is in $\mathcal{T}$.\newpage
\end{enumerate}

Technically, a topological space is an ordered pair $(X, \mathcal{T})$ consisting of a set $X$ and a topology $\mathcal{T}$ on $X$. But when no confusion will arise, we usually omit mentioning $\mathcal{T}$ and just call $X$ a topological space.\retTwo

Given a topological space $(X, \mathcal{T})$, we say that a subset $U$ of $X$ is an \udefine{open set} if $U \in \mathcal{T}$.\retTwo

Suppose $\mathcal{T}$ and $\mathcal{T}^\prime$ are topologies on $X$ Such that $\mathcal{T} \subseteq \mathcal{T}^\prime$. Then we say $\mathcal{T}^\prime$ is \udefine{finer} or \udefine{larger} than $\mathcal{T}$ Also, we say $\mathcal{T}$ is \udefine{coarser} or \udefine{smaller} than $\mathcal{T}^\prime$. And we say both are \udefine{comparable} with each other.

\begin{myIndent}\hThree
   If $\mathcal{T}$ is properly contained by $\mathcal{T}^\prime$, then we add the word \textit{strictly} before those adjectives.\retTwo
\end{myIndent}

\mySepTwo

If $X$ is a set, a \udefine{basis} for a topology on $X$ is a collection $\mathcal{B}$ of subsets of $X$ (called \udefine{basis elements}) such that:
\begin{enumerate}
   \item For each $x \in X$, there is at least one basis element $B$ containing $x$.
   \item If $x$ belongs to the intersection of two basis elements $B_1$ and $B_2$, then there is a basis element $B_3$ containing $x$ such that $B_3 \subseteq B_1 \cap B_2$.\retTwo
\end{enumerate}

If $\mathcal{B}$ satisfies these two conditions, then we define the \udefine{topology $\mathcal{T}$ generated by $\mathcal{B}$} as follows:
\begin{myIndent}
   $U \subseteq X$ is open if for each $x \in U$, there is a basis element $B \in \mathcal{B}$ such that $x \in B$ and $B \subseteq U$.\retTwo
\end{myIndent}

\blab{Proof that the $\mathcal{T}$ generated by $\mathcal{B}$ is a topology:}
\begin{myIndent}\hTwo
   We fairly trivially have that $\emptyset$ and $X$ are included in $\mathcal{T}$.\retTwo

   Let $\{U_\alpha\}_{\alpha \in J}$ be an indexed family of elements of $\mathcal{T}$ and define $U = \bigcup\limits_{\alpha \in J}U_\alpha$.\\ [-8pt] Given any $x \in U$, we know there exists $\alpha \in J$ such that $x \in U_\alpha$.\\ And since $U_\alpha$ is open, there exists $B \in \mathcal{B}$ such that $x \in B$ and\\ $B \subseteq U_\alpha \subseteq U$. So, we conclude that $U$ is also open.\retTwo

   Finally, we shall prove by induction that given $U_1, \ldots U_n \in \mathcal{T}$, we have that\\ $U_1 \cap \ldots \cap U_n \in \mathcal{T}$.

   
   \begin{myIndent}\hThree
      Firstly, consider any $U_1, U_2 \in \mathcal{T}$. Then, given any $x \in U_1 \cap U_2$, choose\\ basis elements $B_1, B_2 \in \mathcal{B}$ such that $x \in B_1 \subseteq U_1$ and $x \in B_2 \subseteq U_2$. \\Since $x \in B_1 \cap B_2$, we know there is a basis element $B_3 \in \mathcal{B}$ such that\\ $x \in B_3 \subseteq B_1 \cap B_2$. Then $x \in B_3 \subseteq U$.\retTwo

      With that, we've now shown that the intersection of any two elements of $\mathcal{T}$\\ is also in $\mathcal{T}$. So, we can proceed by induction.\retTwo

      Suppose for $i < n$ that $(U_1 \cap \ldots \cap U_i) \in \mathcal{T}$. Then we know that\\ $(U_1 \cap \ldots \cap U_i) \cap U_{i+1} \in \mathcal{T}$.\newpage
   \end{myIndent}
\end{myIndent}

\blab{Lemma 13.1:} Let $X$ be a set and $\mathcal{B}$ be a basis for a topology $\mathcal{T}$ on $X$. Then $\mathcal{T}$ equals the collection of all unions of elements of $\mathcal{B}$.

\begin{myIndent}\hTwo
   Proof:\\
   Let $\mathcal{T}^\prime$ be the collection of all unions of elements of $\mathcal{B}$.\retTwo

   Since every $B \in \mathcal{B}$ is an element of $\mathcal{T}$, we trivially have that $\mathcal{T}^\prime \subseteq \mathcal{T}$. Meanwhile, given any $U \in \mathcal{T}$, choose for each $x \in U$ an element $B_x$ of $\mathcal{B}$ such that $x \in B_x \subseteq U$. Then $U = \bigcup\limits_{x \in U}B_x$, meaning $U \in \mathcal{T}^\prime$.\\ [-12pt]
   \begin{myTindent}\begin{myTindent}
   \begin{myIndent}
      \color{Red}\fontsize{12}{14}\selectfont
         (Axiom of Choice usage alert!!)\retTwo\retTwo
   \end{myIndent}
   \end{myTindent}\end{myTindent}
\end{myIndent}

\blab{Lemma 13.2}: Let $X$ be a topological space. Suppose that $\mathcal{C}$ is a collection of open sets of $X$ such that for each open set $U$ of $X$ and each $x \in U$, there is an element $C \in \mathcal{C}$ such that $x \in C \subseteq U$. Then $\mathcal{C}$ is a basis for the topology of $X$.

\begin{myIndent}\hTwo
   Proof:\\
   Firstly, we need to show that $\mathcal{C}$ is a basis.
   \begin{myIndent}\hThree
      Since $X$ is an open set, we know by hypothesis that for all $x \in X$, there is $C \in \mathcal{C}$ such that $x \in C$. As for the second condition of a basis, suppose $x \in C_1 \cap C_2$\\ where $C_1, C_2 \in \mathcal{C}$. Since $C_1$ and $C_2$ are open, we know that $C_1 \cap C_2$ is open. So\\ by hypothesis, there is $C_3 \in \mathcal{C}$ such that $x \in C_3 \subseteq (C_1 \cap C_2)$.\retTwo
   \end{myIndent}

   Secondly, we need to show that $\mathcal{C}$ is a basis for the topology of $X$.
   \begin{myIndent}\hThree
      Let $\mathcal{T}$ be the collection of open sets of $X$, and let $\mathcal{T}^\prime$ be the topology generated\\ by $\mathcal{C}$. Firstly, if $U \in \mathcal{T}$ and $x \in U$, there is by hypothesis $C \in \mathcal{C}$ such that $x \in C$ and $C \subseteq U$. So $U \subseteq \mathcal{T}^\prime$. Meanwhile, if $W \in \mathcal{T}^\prime$, then $W$ equals a union of elements of $\mathcal{C}$ by lemma 13.1. Since each element of $\mathcal{C}$ is in $\mathcal{T}$, we know $W$ is the union of elements of $\mathcal{T}$, meaning $W \in \mathcal{T}$. So, we've shown that $\mathcal{T} \subseteq \mathcal{T}^\prime \subseteq \mathcal{T}$.\retTwo\retTwo
   \end{myIndent}
\end{myIndent}

\blab{Lemma 13.3:} Let $\mathcal{B}$ and $\mathcal{B}^\prime$ be bases for the topologies $\mathcal{T}$ and $\mathcal{T}^\prime$ respectively on $X$. Then $\mathcal{T}^\prime$ is finer than $\mathcal{T}$ if and only if for each $x \in X$ and each basis element $B \in \mathcal{B}$ containing $x$, there is a basis element $B^\prime \in \mathcal{B}^\prime$ such that $x \in B^\prime \subseteq B$.

\begin{myIndent}\hTwo
   Proof:\\
   ($\Longrightarrow$) Let $x \in X$ and $B \in \mathcal{B}$ such that $x \in B$. Since $B \in \mathcal{T}$ and we are assuming $\mathcal{T} \subseteq \mathcal{T}^\prime$, we know that $B \in \mathcal{T}^\prime$. Then since $\mathcal{B}^\prime$ generated $\mathcal{T}^\prime$, we know there is $B^\prime \in \mathcal{B}^\prime$ such that $x \in B^\prime \subseteq B$.\retTwo

   ($\Longleftarrow$)\\
   Given an element $U$ of $\mathcal{T}$, we need to show that $U \in \mathcal{T}^\prime$. To do this, consider any $x \in U$. Since $\mathcal{B}$ generates $\mathcal{T}$, there is an element $B \in \mathcal{B}$ such that $x \in B \subseteq U$. Now by hypothesis, there exists $B^\prime \in \mathcal{B}^\prime$ such that $x \in B^\prime \subseteq B$. So $x \in B^\prime \subseteq U$. Hence, $U \in \mathcal{T}^\prime$.\newpage
\end{myIndent}

If $\mathcal{B}$ is the collection of all open intervals $(a, b) = \{x \in \mathbb{R} \mid a < x < b\}$ in the real line, then we call the topology generated by $\mathcal{B}$ the \udefine{standard topology} on the real line.\\ [-16pt]

\begin{myDindent}\hTwo
   We assume $\mathbb{R}$ has this topology unless stated otherwise.\retTwo
\end{myDindent}

If $\mathcal{B}^\prime$ is the collection of all intervals $[a, b)$ of the real line, we call the topology\\ generated by $\mathcal{B}^\prime$ the \udefine{lower limit topology}.\\ [-15pt]
\begin{myDindent}\hTwo
   When $\mathbb{R}$ has this topology, we denote it $\mathbb{R}_l$.\retTwo
\end{myDindent}

Letting $K = \{\frac{1}{n} \mid n \in \mathbb{Z}_+\}$, if $\mathcal{B}^\pprime$ is the collection of all intervals $(a, b)$ of the real line along with all sets of the form $(a, b) - K$, then we call the topology generated by $\mathcal{B}^\pprime$ the \udefine{$K$-topology} on the real line.\\ [-15pt]
\begin{myDindent}\hTwo
   When $\mathbb{R}$ has this topology, we denote it $\mathbb{R}_K$.\retTwo\retTwo
\end{myDindent}

\blab{Lemma 13.4:} The topologies of $\mathbb{R}_l$ and $\mathbb{R}_K$ are strictly finer than the standard\\ topology on $\mathbb{R}$. But, they aren't comparable with one another.

\begin{myIndent}\hTwo
   Proof:\\
   Let $\mathcal{T}, \mathcal{T}^\prime, \mathcal{T}^\pprime$ be the topologies of $\mathbb{R}, \mathbb{R}_l, \mathbb{R}_K$ respectively.\retTwo

   Given any $(a, b) \in \mathcal{B}$ and $x \in (a, b)$, we know that $[x, b) \in \mathcal{B}^\prime$ and that\\ $x \in [x, b) \subseteq (a, b)$. So by lemma 13.3, $\mathcal{T} \subseteq \mathcal{T}^\prime$. On the other hand, for any\\ $[x, b) \in \mathcal{B}^\prime$, there is no set $(a, b) \in \mathbb{B}$ such that $x \in (a, b) \subseteq [x, b)$. So\\ $\mathcal{T}^\prime \not\subseteq \mathcal{T}$. Hence, $\mathcal{T}^\prime$ is strictly finer than $\mathcal{T}$.\retTwo

   Also, given any $(a, b) \in \mathcal{B}$, we also know that $(a, b) \in \mathcal{B}^\pprime$. So $\mathcal{T} \subseteq \mathcal{T}^\pprime$. On\\ the other hand, given $(-1, 1) - K \in \mathcal{B}^\pprime$, we know there is no interval\\ $(a, b) \in \mathcal{B}$ such that $0 \in (a, b) \subseteq (-1, 1) - K$. So by lemma 13.3, we know\\ that $\mathcal{T}^\pprime \not\subseteq \mathcal{T}$. Hence, $\mathcal{T}^\pprime$ is strictly finer than $\mathcal{T}$.\retTwo

   Finally, we show $\mathcal{T}^\prime$ and $\mathcal{T}^\pprime$ aren't comparable. Firstly, given the set $(-1, 1) - K$ in $\mathcal{B}^\pprime$, there is no set $[a, b) \in \mathcal{B}^\prime$ such that $0 \in [a, b) \subseteq (-1, 1) - K$. After all, for any $b > 0$, we can use the archimedean property to find $\frac{1}{n} < b$. Secondly, given the set $[0, 1) \in \mathcal{B}^\prime$, no set of the form $(a, b)$ can satisfy that $0 \in (a, b) \subseteq [0, 1)$. Similarly, no set of the form $(a, b) - K$ can satisfy that $0 \in (a - b) - K \subseteq [0, 1)$. So neither $\mathcal{T}^\prime \subseteq \mathcal{T}^\pprime$ nor $\mathcal{T}^\pprime \subseteq \mathcal{T}^\prime$.\retTwo\retTwo
\end{myIndent}

A \udefine{subbasis} $\mathcal{S}$ for a topology on $X$ is a collection of subsets of $X$ whose union equals $X$. The \udefine{topology generated by the subbasis $\mathcal{S}$} is defined to be the collection $\mathcal{T}$ of all unions of finite intersections of elements of $\mathcal{S}$.\retTwo

\blab{Proof that the $\mathcal{T}$ generated by $\mathcal{S}$ is a topology:}
\begin{myIndent}\hTwo
   It suffices to show that the collection $\mathcal{B}$ of all finite intersections of elements of $\mathcal{S}$ is a basis. The first condition of a basis is trivially true for $\mathcal{B}$ since the union of the elements of $\mathcal{S}$ is all of $X$ and $\mathcal{S} \subseteq \mathcal{B}$.\newpage
   
   As for the second condition of a basis, given any $(S_1 \cap \ldots \cap S_n), (S_1^\prime \cap \ldots \cap S_m^\prime) \in \mathcal{B}$, we know that $(S_1 \cap \ldots \cap S_n) \cap (S_1^\prime \cap \ldots \cap S_m^\prime)$ is a finite intersection of elements of $\mathcal{S}$ and thus an element in $\mathcal{B}$. Thus, the condition easily follows.\retTwo\retTwo
\end{myIndent}

\dispDate{9/26/2024}

Well, it looks like I'll be able to survive 240A with the topology information I've learned so far. However, it doesn't look like I'll be able to survive 240B with what I know right now. So, I've got to study more of this. But if needed for 188 this quarter, I'll take a break to study algebra.\retTwo

\exOne
\blab{Exercise 13.3} Show that $\mathcal{T} = \{U \subseteq X \mid X - U \text{ is countable or all of } X\}$ is a\\ topology on $X$.

\begin{myIndent}\exTwoP
   Clearly $\emptyset, X \in \mathcal{T}$ since $|X - X| = 0$ and $X - \emptyset = X$.\retTwo

   Suppose $\{U_\alpha\}_{\alpha \in A}$ is a collection of sets in $\mathcal{T}$. Then $X - \bigcup\limits_{\alpha \in A}U_\alpha = \bigcap\limits_{\alpha \in A} (X - U_\alpha)$\\ [-6pt] is countable since it's a subset of a countable set.\retTwo

   Hence, $\bigcup\limits_{\alpha \in A}U_\alpha \in \mathcal{T}$.\retTwo

   Finally, consider any $\{U_1, \ldots, U_n\}$ in $\mathcal{T}$. Then $X - \bigcap\limits_{k=1}^n U_k = \bigcup\limits_{k = 1}^n (X - U_k)$\\ [-7pt] is countable since it's a union of finitely many\\ countable sets.\retTwo
   
   Hence, $\bigcap\limits_{k = 1}^n U_k \in \mathcal{T}$.\retTwo
\end{myIndent}

\blab{Exercise 13.4:}
\begin{itemize}
   \item[(a)] If $\{\mathcal{T}_\alpha\}_{\alpha \in A}$ is a family of topologies on $X$, show that $\bigcap\mathcal{T}_\alpha$ is a topology on $X$.\\ Is $\bigcup\mathcal{T}_\alpha$ a topology on $X$?
   
   \begin{myIndent}\exTwoP
      Let $\mathcal{T} = \bigcap\limits_{\alpha \in A}\mathcal{T}_\alpha$.\\

      Since $\emptyset$ and $X$ belong to all $\mathcal{T}_\alpha$, we know that $\emptyset, X \in \mathcal{T}$.\retTwo

      Next, suppose $\{U_\beta\}_{\beta \in B}$ is a collection of sets in $\mathcal{T}$. Since $\{U_\beta\}_{\beta \in B} \subseteq \mathcal{T}_\alpha$ for all $\alpha$, we know that $\bigcup\limits_{\beta \in B}U_\beta \in \mathcal{T}_\alpha$ for all $\alpha$. Hence, $\bigcup\limits_{\beta \in B}U_\beta \in \bigcap\limits_{\alpha \in A}\mathcal{T}_\alpha = \mathcal{T}$.\retTwo

      The same argument as used for arbitrary unions also shows that any finite\\ intersection of sets in $\mathcal{T}$ is also in $\mathcal{T}$.\newpage

      We've now shown that $\mathcal{T}$ is a topology. As for the other question asked, no we don't necessarily have that $\bigcup\limits_{\alpha \in A}\mathcal{T}_\alpha$ is a topology.\retTwo

      To see this, consider the set $X = \{a, b, c\}$ with the topologies\\ $\mathcal{T}_1 = \{\emptyset, \{a\}, \{a, b\}, \{a, b, c\}\}$ and $\mathcal{T}_2 = \{\emptyset, \{c\}, \{b, c\}, \{a, b, c\}\}$. Then\\ $\mathcal{T}_1 \cup \mathcal{T}_2$ is not a topology because $\{a\}, \{c\} \in \mathcal{T}_1 \cup \mathcal{T}_2$ but $\{a, c\} \notin \mathcal{T}_1 \cup \mathcal{T}_2$.\retTwo
   \end{myIndent}

   \item[(b)] Let $\{\mathcal{T}_\alpha\}_{\alpha \in A}$ be a family of topologies on $X$. Show that there is a unique smallest topology on $X$ containing all the collections $\mathcal{T}_\alpha$, and a unique largest topology contained in all $\mathcal{T}_\alpha$.
   
   \begin{myIndent}\exTwoP
      Firstly, let $\{\mathcal{T}_\beta^\prime\}_{\beta \in B}$ be the collection of all topologies on $X$ which contain $\bigcup\limits_{\alpha \in A}\hspace{-0.2em}\mathcal{T}_\alpha$.\retTwo

      We know that $\{\mathcal{T}_\beta^\prime\}_{\beta \in B}$ is not empty because it must at least have $\mathcal{P}(X)$ as an element. Hence, we can apply part (a) of the problem to know that $\bigcap\limits_{\beta \in B}\mathcal{T}_\beta^\prime$ is a\\ [-8pt] topology on $X$.\retTwo

      Importantly, by virtue of being an intersection, that topology is smaller than all other topologies containing $\bigcup\limits_{\alpha \in A}\hspace{-0.2em}\mathcal{T}_\alpha$. At the same time, we know it contains\\ [-8pt] $\bigcup\limits_{\alpha \in A}\hspace{-0.2em}\mathcal{T}_\alpha$.\retTwo
      
      So it is the unique smallest topology on $X$ containing all the collections $\mathcal{T}_\alpha$.\retTwo

      The second part of this question is trivial from part (a). If a topology $\mathcal{T}^\pprime$ is contained in all $\mathcal{T}_\alpha$, then we know that $\mathcal{T}^\pprime \subseteq \bigcap\limits_{\alpha \in A}\mathcal{T}_\alpha$. Clearly, the largest topology\\ [-8pt] satisfying this is $\bigcap\limits_{\alpha \in A}\mathcal{T}_\alpha$.\retTwo
   \end{myIndent}
\end{itemize}

\blab{Exercise 13.5:} Show that if $\mathcal{A}$ is a basis for a topology on $X$, then the topology\\ generated by $\mathcal{A}$ equals the intersection of all topologies on $X$ that contain $\mathcal{A}$.

\begin{myIndent}\exTwoP
   Let $\mathcal{T}$ be the topology generated by $\mathcal{A}$, and suppose $\mathcal{T}^\prime$ is any topology containing $\mathcal{A}$. Then consider any $U \in \mathcal{T}$. By Lemma 13.1, we know that $U = \bigcup\limits_{\beta \in B}A_\beta$ where\\ [-11pt] $\{A_\beta\}$ is some collection of sets in $\mathcal{A}$. Hence, $U$ is a union of\\ sets in $\mathcal{T}^\prime$, meaning $U \in \mathcal{T}^\prime$.\retTwo

   Since $\mathcal{T} \subseteq \mathcal{T}^\prime$ for all $\mathcal{T}^\prime$ containing $\mathcal{A}$, we thus know that $\mathcal{T}$ is the unique\\ smallest topology containing $\mathcal{A}$. At the same time, by exercise 13.4.a, we know that\\ the intersection $\mathcal{T}^\pprime$ of all topologies containing $\mathcal{A}$ is a topology. By virtue of being\\ an intersection, we know it is smaller than all topologies containing $\mathcal{A}$, and that\\ it contains $\mathcal{A}$. So, $\mathcal{T} \subseteq \mathcal{T}^\pprime \subseteq \mathcal{T} \Longrightarrow \mathcal{T} = \mathcal{T}^\pprime$.\newpage
\end{myIndent}

Prove the same if $\mathcal{A}$ is a subbasis.

\begin{myIndent}\exTwoP
   Let $\mathcal{T}$ be the topology generated by $\mathcal{A}$ and suppose $\mathcal{T}^\prime$ is any topology containing $\mathcal{A}$. Then consider any $U \in \mathcal{T}$. We know that $U = \bigcup\limits_{\beta \in B}U_\beta$ where $\{U_\beta\}_{\beta \in B}$ is a\\ [-11pt] collection of finite intersections of sets in $\mathcal{A}$.\retTwo

   Because each $U_\beta$ must be in $\mathcal{T}^\prime$, we thus know that $U \in \mathcal{T}^\prime$. So $\mathcal{T} \subseteq \mathcal{T}^\prime$.\retTwo

   The rest of the proof goes exactly the same as before.
   \begin{myIndent}\myComment
      This fact can be used as a shortcut for finding the unique smallest topology\\ containing all topologies in a collection.\retTwo
   \end{myIndent}
\end{myIndent}

\blab{Exercise 13.1:} Let $X$ be a topological space and let $A$ be a subset of $X$. Suppose that for each $x \in A$, there is an open set $U$ such that $x \in U \subseteq A$. Then $A$ is open in $X$.

\begin{myIndent}\exTwoP
   For all $x \in A$, pick an open set $U_x$ such that $x \in U_x \subseteq A$. Then $A = \bigcup\limits_{x \in A}U_x$ is a\\ [-10pt] union of open sets.
   \begin{myTindent}\myComment\color{Red}
      (A.O.C. usage!!)\retTwo
   \end{myTindent}
\end{myIndent}

\hOne
\mySepTwo

\dispDate{9/27/2024}

If $X$ is simply-ordered, the standard topology for $X$ (called the \udefine{order topology}) is defined as follows:

\begin{myIndent}
   Given any $a, b \in X$ with $a < b$, we define the sets $(a, b)$, $[a, b)$, $(a, b]$ and $[a, b]$ as you would expect. These are the open, closed, and half-open intervals on $X$.\retTwo

   Now let $\mathcal{B}$ be the collection of all sets of the form:
   \begin{itemize}
      \item Open intervals $(a, b)$ in $X$.
      \item Intervals of the form $[a_0, b)$ where $a_0$ is the smallest element of $X$ (if one exists).
      \item Intervals of the form $(a, b_0]$ where $b_0$ is the largest element of $X$ (if one exists).\retTwo
   \end{itemize}

   The collection $\mathcal{B}$ is a basis for a topology on $X$ which is called the order\\ topology.
   
   \begin{myIndent}\hThree
      It's fairly trivial to show that this is a basis. It's just that for the second condition of a basis, there are a bunch of cases that need to be mentioned.\retTwo
   \end{myIndent}
\end{myIndent}

Another way we can define the order topology is through rays. Given any $a \in X$, we define the sets $(a, +\infty)$, $(-\infty, a)$, $[a, +\infty)$ and $(-\infty, a]$ as you would expect.\newpage

Let $\mathcal{S}$ be the set of open rays: $(a, +\infty)$ and $(-\infty, a)$. This is a subbasis for the order topology on $X$.

\begin{myIndent}
   To see this, firstly note that all open rays are open sets in the order topology of $X$. So, every set in the topology generated by $\mathcal{S}$ will be an open set in our original order topology. Hence if $\mathcal{T}$ is the order topology on $X$ and $\mathcal{T}^\prime$ is the topology generated by $S$, we know that $\mathcal{T}^\prime \subseteq \mathcal{T}$.\retTwo

   At the same time, every interval in the previously defined basis of $\mathcal{T}$ is the\\ intersection of two (or one if the interval contains the greatest or least\\ element of $X$) rays. Hence, $\mathcal{T} \subseteq \mathcal{T}^\prime$.\retTwo
\end{myIndent}

\mySepTwo

\dispDate{9/29/2024}

Today I'm going to be studying from Michael Artin's textbook \textit{Algebra, second\\ edition}. My reasoning is that I need to learn more group theory in order to be ready for 188.\retTwo

A \udefine{law of compositon} on a set $S$ is map from $S \times S$ to $S$. Given the ordered pair $(a, b) \in S \times S$, we denote the element the pair is mapped to as either $ab$, $a \times b$, $a \circ b$, $a + b$, or etc. 

\begin{myIndent}
   Typically, $+$ is used if the composition is commutative. Meanwhile, the\\ multiplicative symbols don't imply commutativity.\retTwo
\end{myIndent}

\blab{Proposition 2.1.4:} Let an associative law of composition be given on $S$. Then we can uniquely define for all $n \in \mathbb{N}$ a product of $n$ elements $a_1, \ldots, a_n$ of $S$, temporarily denoted by $[a_1\cdots a_n]$, with the following properties:
\begin{itemize}
   \item[(i)] The product $[a_1]$ of one element is $a_1$.
   \item[(ii)] The product $[a_1a_2] $ of two elements is given by the law of composition.
   \item[(iii)] For any integer $i$ in the range $1 \leq i < n$, $[a_1\cdots a_n] = [a_1\cdots a_i][a_{i+1}\cdots a_n]$. 
   
   \begin{myIndent}\hTwo
      Proof:\\
      We proceed by induction on $n$.\retTwo

      Let us define $[a_1\cdots a_n] = [a_1\cdots a_{n-1}][a_n]$ and suppose that the analogous definition of $[a_1\cdots a_r]$ satisfies our properties for all $1 < r < n$. Then for any $1 \leq i < n - 1$, we have that:

      {\centering 
      \begin{tabular}{l l}
         $[a_1\cdots a_n] = [a_1\cdots a_{n-1}][a_n]$ & (by definition) \\
         $\phantom{[a_1\cdots a_n] }= ([a_1\cdots a_i][a_{i+1}\cdots a_{n-1}])[a_n]$ & (by inductive hypothesis) \\
         $\phantom{[a_1\cdots a_n] }= [a_1\cdots a_i]([a_{i+1}\cdots a_{n-1}][a_n])$ & (by associativity)\\
         $\phantom{[a_1\cdots a_n] }= [a_1\cdots a_i][a_{i+1}\cdots a_{n-1} a_n]$
      \end{tabular} \newpage\par}
   \end{myIndent}
\end{itemize}

Based on the previous proposition, it's safe to just denote the product of $a_1, \ldots, a_n$ as $a_1\cdots a_n$.\retTwo

An identity for a law of composition is an element $e$ of $S$ satisfying that:

{\centering $ea = a$ and $ae = a$ for all $a \in S$. \retTwo\par}

We denote the identity of a law of composition as $0$ or $1$ (depending on whether we are using multiplicative or additive notation). We can only have one identity element.

\begin{myIndent}\hTwo
   Proof:\\
   Suppose $e$ and $e^\prime$ are both identity elements. Then $e = ee^\prime = e^\prime$.\retTwo
\end{myIndent}

An element $a$ of $S$ is \udefine{invertible} if there is another element $b \in S$ such that $ab = 1$ and $ba = 1$. We call $b$ the \udefine{inverse} of $a$ and denote $b$ as $-a$ or $a^{-1}$ depending on whether additive or multiplicative notation is being used.\retTwo

\exOne

\blab{Exercise 1.2:}
\begin{itemize}
   \item If an element $a$ has both a left inverse $l$ and a right inverse $r$, then $l = r$, $a$ is invertible, and $r$ is its inverse.
   
   \begin{myIndent}\exTwoP
      Suppose $l a = 1$ and $a r = 1$. Then we have that:

      {\centering $r = 1r = (la)r = l(ar) = l1 = l$\par}
   \end{myIndent}

   \item If $a$ is invertible, its inverse is unique.
   \begin{myIndent}\exTwoP
      Suppose $b$ and $b^\prime$ are both inverses of $a$. Then:

      {\centering $b = 1b = (b^\prime a)b = b^\prime(ab) = b^\prime1 = b^\prime$\par}
   \end{myIndent}

   \item If $a$ and $b$ are invertible, then $ab$ is invertible with $(ab)^{-1} = b^{-1}a^{-1}$.
   
   \begin{myIndent}\exTwoP
      Proof:

      {\centering$abb^{-1}a^{-1} = a1a^{-1} = aa^{-1} = 1$ and $b^{-1}a^{-1}ab = b^{-1}1b = b^{-1}b = 1$\retTwo\par}
   \end{myIndent}
\end{itemize}

\hOne

\mySepTwo

A \udefine{group} is a set $G$ together with a law of composition such that:
\begin{enumerate}
   \item The law of composition is associative.
   \item $G$ has an identity element.
   \item Every element of $G$ has an inverse.
\end{enumerate}

An \udefine{abelian group} is a group whose law of composition is commutative.\retTwo

The \udefine{order} of a group $G$ is the number of elements it contains. We denote the order $|G|$. If $|G|$ is finite, we say $G$ is a \udefine{finite group}. Otherwise, we say $G$ is an \udefine{infinite group}.\newpage

The \udefine{$n\times n$ general linear group} is the group of all invertible $n\times n$ matrices. It's denoted $GL_n$. If we want to specify whether we are working with real or complex matrices, we write $GL_n(\mathbb{R})$ or $GL_n(\mathbb{C})$.\retTwo

The \udefine{symmetric group} of a set is the set of permutations of the set (with the law of composition being function composition). We denote $S_n$ the group of permutations of the indices $1, 2, \ldots n$.

\exOne\mySepTwo

$S_3$ has order $6$ (this is an easy fact from combinatorics). Now let $x = (1\gap 2\gap 3)$ and $y = (1\gap 2)(3)$ (this is using cyclic notation). Then we can express all the elements of $S_3$ as products of $x$ and $y$.

{\centering\begin{tabular}{l l l}
   $1 = (1)(2)(3)$ & $x = (1\gap 2\gap 3) $ & $x^2 = (1\gap 3\gap 2)$ \\
   $y = (1\gap 2)(3)$ & $xy = (1\gap 3)(2)$ & $x^2y = (1)(2\gap 3)$
\end{tabular}\par\retTwo}

Also note that $x^3 = 1$, $y^2 = 1$, and $yx = x^2 y$.\retTwo

\blab{Exercise 2.1:} Make a multiplication table for the symmetric group $S_3$.

\begin{myIndent}\exTwoP
   Just to clarify, to read this table, take the row element to be on the left side of the product and the column element to be on the right side of the product.

   {\center\begin{tabular}{c||c|c|c|c|c|c}
      $\times$ & $1$ & $x$ & $x^2$ & $y$ & $xy$ & $x^2y$ \\ \hline &&&&&&\\ [-13pt]\hline &&&&&&\\ [-13pt]
      $1$ & $1$ & $ x$ & $x^2 $ & $y $ & $xy $ & $x^2y $ \\\hline &&&&&&\\ [-13pt]
      $x$ & $x$ & $ x^2$ & $1 $ & $xy $ & $x^2y $ & $y $ \\ \hline &&&&&&\\ [-13pt]
      $x^2$ & $x^2$ & $1 $ & $x $ & $x^2y $ & $y $ & $xy $ \\ \hline &&&&&&\\ [-13pt]
      $y$ & $y$ & $x^2y $ & $xy $ & $1 $ & $ x^2$ & $x$ \\ \hline &&&&&&\\ [-13pt]
      $xy$ & $xy$ & $y$ & $ x^2y$ & $x $ & $1$ & $x^2$ \\ \hline &&&&&&\\ [-13pt]
      $x^2y$ & $x^2y$ & $xy$ & $y$ & $ x^2$ & $x$ & $1$
   \end{tabular}\retTwo\par}
\end{myIndent}

\hOne

\mySepTwo

\dispDate{12/14/2024}

My goal for while I'm flying home and can't work on grading is to go prove the\\ following extra exercise of homework 2 in my math 188 class.\retTwo

Let $V = \{F(x) \in \mathbb{C}[[x]] : F(0) = 0\}$ and $W = \{F(x) \in \mathbb{C}[[x]] : G(0) = 1\}$.

\begin{enumerate}
	\item[(a)] Given $F(x) \in V$, show that $\bm{E}(F(x)) = \sum_{n \geq 0} \frac{F(x)^n}{n!}$ is the \textit{unique} formal power\\ [2pt] series $G(x) \in W$ such that $\Df G(x) = \Df F(x)\cdot G(x)$. This defines a function\\ [2pt] $\bm{E}: V \longrightarrow W$. 
	\begin{myIndent}
		Note that we use the convention $F(x)^0 = 1$ even if $F(x) = 0$.\newpage

		\exTwo
		Firstly, note that we have $G(x) \coloneq \bm{E}(F(x)) = \exp(F(x))$ where\\ $\exp(x) \coloneq \sum_{n \geq 0}\frac{x^n}{n!}$. Also, you can check $\exp(x)$ is it's own derivative.\\ Thus by chain rule:

		{\centering $G(x) = \Df F(x) \cdot \Df (\exp)(F(x)) = \Df F(x) \cdot \exp(F(x)) = \Df F(x) \cdot G(x)$ \retTwo\par}

		Next, suppose $H(x)$ is another formal power series in $W$ satisfying that\\ $\Df H(x) = \Df F(x) \cdot H(x)$. Note that since $H(0) \neq 0 \neq G(0)$, we can write that\\ $\frac{\Df H(x)}{H(x)} = \Df F(x) = \frac{\Df G(x)}{G(x)}$. Therefore, we get that:

		{\centering 
		\begin{tabular}{l l}
			\redPen{($*$)} & $\Df H(x) \cdot G(x) = \Df G(x) = H(x)$
		\end{tabular} \retTwo\par}

		Let $H(x) = \sum_{n \geq 0}h_nx^n$ and $G(x) = \sum_{n \geq 0}g_nx^n$. Since we assumed that\\ $H(x), G(x) \in W$, we know that $h_0 = g_0 = 1$. Then, proceeding by induction (assuming that $h_i = g_i$ for all $0 \leq i \leq n$), when we take the $n$th. coefficient of \redPen{($*$)} we get:

		{\centering\exP\begin{tabular}{l}
			$(n+1)h_{n+1} + \sum\limits_{i=0}^{n-1}(i+1)h_{i+1}g_{n-i}$\\
			$\phantom{aaaaaaaaaa} = \sum\limits_{i=0}^n (i+1)h_{i+1}g_{n-i} = \sum\limits_{i = 0}^n (i+1)g_{i+1}h_{n-i}$\\
			$\phantom{\phantom{aaaaaaaaaa} = \sum\limits_{i=0}^n (i+1)h_{i+1}g_{n-i}} = (n+1)g_{n+1} + \sum\limits_{i=0}^{n-1}(i+1)g_{i+1}h_{n-i}$
		\end{tabular}\retTwo\par}

		But by induction we have $\sum\limits_{i = 0}^{n-1}(i+1)g_{i+1}h_{n-i} = \sum\limits_{i=1}^{n-1}(i+1)h_{i+1}g_{n-i}$.\retTwo

		So subtracting out the sum from $i = 0$ to $n - 1$ and then dividing by $n + 1$ which is crucially nonzero, we then have that $h_{n+1} = g_{n+1}$.\retTwo
	\end{myIndent}

	\item[(b)] Given $G(x) \in W$, show that there is a \textit{unique} formal power series $F(x) \in V$ such that $\Df F(x) = \frac{\Df G(x)}{G(x)}$. This let's us define the function $\bm{L}: W \longrightarrow V$ by $\bm{L}(G(x)) = F(x)$.
	
	\begin{myIndent}\exTwo
		Since $G(0) = 1$, we know that $G(x)$ is invertible. So there is a unique formal power series $A(x) = \sum_{n \geq 0}a_n x^n$ such that $A(x) = \frac{\Df G(x)}{G(x)}$.\retTwo

		Then if $F(x) = \sum_{n \geq 0}f_nx^n$ satisfies that $\Df F(x) = \frac{\Df G(x)}{G(x)}$, then we can solve that $f_n = \frac{a_{n-1}}{n}$ for all $n \geq 1$. This shows that $f_n$ is uniquely determined for all $n \geq 1$. Also, since we are forcing $F(0) = 0$, we know that $f_0 = 0$. So $F(x)$ is a unique power series.\retTwo

		From this it's also hopefully clear to see how one can solve for $F(x)$ in order to show that $F(x)$ exists.\newpage
	\end{myIndent}

	\item[(c)] Show that $\bm{E}$ and $\bm{L}$ are inverses of each other.
	
	\begin{myIndent}\exTwo
		Firstly, we'll show $\bm{L}(\bm{E}(F(x))) = F(x)$ for all $F(x) \in V$.
		\begin{myIndent}\exTwoP
			Let $F(x) \in V$, $G(x) = \bm{E}(F(x))$, and $H(x) = \bm{L}(G(x))$. Then we have that:

			{\centering $\Df H(x) = \frac{\Df G(x)}{G(x)} = \frac{\Df F(x) \cdot G(x)}{G(x)} = \Df F(x)$ \retTwo\par}

			This proves that $[x^n]H(x) = [x^n]F(x)$ for all $n \geq 1$. And since both $H(x), F(x) \in V$, we know that $H(0) = F(0).$ so $H(x) = F(x)$.\retTwo
		\end{myIndent}

		Secondly, we'll show $\bm{E}(\bm{L}(F(x))) = F(x)$ for all $F(x) \in W$.
		\begin{myIndent}\exTwoP
			Let $F(x) \in V$, $G(x) = \bm{L}(F(x))$, and $H(x) = \bm{E}(G(x))$. Then we have that:

			{\centering $\Df H(x) = \Df G(x) \cdot H(x) = \frac{\Df F(x)}{F(x)} \cdot H(x)$ \retTwo\par}

			Thus we know that $\Df H(x) \cdot F(x) = \Df F(x) \cdot H(x)$. Since both $H(x)$ and $F(x)$ are in $W$, we can employ identical logic as that of part (a) to show that $H(x) = F(x)$.\retTwo
		\end{myIndent}
	\end{myIndent}

	\item[(d)] Show that $\bm{E}(F_1(x) + F_2(x)) = \bm{E}(F_1(x))\bm{E}(F_2(x))$ for all $F_1(x), F_2(x) \in V$.
	
	
	\begin{myIndent}\exTwo
		Note that:

		{\centering
		\begin{tabular}{l}
			$\bm{E}(F_1(x) + F_2(x)) = \lim\limits_{m \rightarrow \infty} \sum\limits_{n=0}^m \frac{(F_1(x) + F_2(x))^n}{n!}$\\ [12pt]

			$\phantom{\bm{E}(F_1(x) + F_2(x))} = \lim\limits_{m \rightarrow \infty}\sum\limits_{n=0}^m \frac{1}{n!}\sum\limits_{i=0}^n \frac{n!}{i!(n-i)!}F_1(x)^i F_2(x)^{n-i}$\\ [12pt]

			$\phantom{\bm{E}(F_1(x) + F_2(x))} = \lim\limits_{m \rightarrow \infty}\sum\limits_{n=0}^m \sum\limits_{i=0}^n \frac{F_1(x)^i}{i!} \cdot \frac{F_2(x)^{n-i}}{(n-i)!}$\\ [12pt]

			$\phantom{\bm{E}(F_1(x) + F_2(x))} = \lim\limits_{m \rightarrow \infty}\left((\sum\limits_{n=0}^m \frac{F_1(x)^n}{n!})(\sum\limits_{n=0}^m \frac{F_2(x)^n}{n!}) + R_m(x)\right)$\\ [12pt]
		\end{tabular} \retTwo\par}

		In the above manipulations $R_m(x)$ is the formal power series negating all the terms which are in $(\sum\limits_{n=0}^m \frac{F_1(x)^n}{n!})(\sum\limits_{n=0}^m \frac{F_2(x)^n}{n!})$ but aren't in $\sum\limits_{n=0}^m \sum\limits_{i=0}^n \frac{F_1(x)^i}{i!} \cdot \frac{F_2(x)^{n-i}}{(n-i)!}$.\retTwo

		In other words, $R_m(x)$ contains all the terms of the form $\frac{1}{i!j!}F_1(x)^iF_2(x)^j$\\ where $i + j > m$. Importantly, because $F_1(0) = 0 = F_2(0)$, we know that\\ $\mdeg R(x) > m$. So, $R_m(x) \rightarrow 0$ as $m \rightarrow \infty$.\retTwo
		
		In turn:

		{\centering\exP
		\begin{tabular}{l}
			$\lim\limits_{m \rightarrow \infty}\left((\sum\limits_{n=0}^m \frac{F_1(x)^n}{n!})(\sum\limits_{n=0}^m \frac{F_2(x)^n}{n!}) + R_m(x)\right)$\\ [12pt]
			
			$\phantom{aaaa} = \lim\limits_{m \rightarrow \infty}\sum\limits_{n=0}^m \frac{F_1(x)^n}{n!} \cdot \lim\limits_{m \rightarrow \infty}\sum\limits_{n=0}^m \frac{F_2(x)^n}{n!} + \lim\limits_{m \rightarrow \infty}R_m(x) = \bm{E}(F_1(x))\bm{E}(F_2(x))$
		\end{tabular}\newpage\par}
	\end{myIndent}

	\item[(e)] Show that $\bm{L}(F_1(x)F_2(x)) = \bm{L}(F_1(x)) + \bm{L}(F_2(x))$ for all $F_1(x), F_2(x) \in W$.
	
	\begin{myIndent}\exTwo
		By part (d), we know that:
		\begin{itemize}
			\item $\bm{E}(\bm{L}(F_1(x)) + \bm{L}(F_2(x))) = \bm{E}(\bm{L}(F_1(x)))\bm{E}(\bm{L}(F_2(x)))$\retTwo
		\end{itemize}
		
		Meanwhile, by part (c) we know that:
		\begin{itemize}
			\item $\bm{E}(\bm{L}(F_1(x)F_2(x))) = F_1(x)F_2(x)$
			\item $\bm{E}(\bm{L}(F_1(x)))\bm{E}(\bm{L}(F_2(x))) = F_1(x)F_2(x)$\retTwo
		\end{itemize}

		Hence, we've shown that $\bm{E}$ maps the left- and right-hand sides of the above claimed equation to the same formal power series. But from part (c) we know that $\bm{E}$ is an injective map. So we must have that the two sides of the equation are in fact equal.
	\end{myIndent}

	\item[(f)] If $m$ is a positive integer and $G(x) \in W$, show that $\bm{E}(\frac{\bm{L}(G(x))}{m})$ is an $m$th{.} root of $G(x)$.
	
	
	\begin{myIndent}\exTwo
		Let $F(x) = \bm{E}(\frac{\bm{L}(G(x))}{m})$. This implies $m \cdot \bm{L}(F(x)) = \bm{L}(G(x))$. Then by part\\ [3pt] (e), we know that $\bm{L}(F(x)^m) = \bm{L}(G(x))$. And finally, plugging both sides into\\ [3pt] $\bm{E}$ we get $F(x)^m = G(x)$.\retTwo

		Also, $F(0) = 1$. So $F(x)$ is the unique $m$th{.} root of $G(x)$ with $1$ as its constant coefficient.\retTwo
	\end{myIndent}

	Because of part (f), we can now extend our definition of the $z$th{.}\hspace{-0.1em} power of a\\ formal power series $G(x) \in W$ to any complex number $z$. Specifically, for any\\ $G(x) \in W$ we define $G(x)^z \coloneq \bm{E}(z\cdot \bm{L}(G(x)))$. Then, we've shown in part (f) that this definition agrees with our more restricted definition from math 188 on at least the positive rational numbers.\retTwo

	Two important identities (where $G(x) \in W$ and $z, w \in \mathbb{C}$):
	\begin{itemize}		
		\item $G(x)^zG(x)^w = G(x)^{z + w}$?
		
		\begin{myIndent}
			\exTwo Proof:

			{\centering
			\begin{tabular}{l}
				$G(x)^zG(x)^w = \bm{E}(z\cdot \bm{L}(G(x)))\bm{E}(w \cdot \bm{L}(G(x)))$\\
				$\phantom{G(x)^zG(x)^w} = \bm{E}(z\cdot \bm{L}(G(x)) + w \cdot \bm{L}(G(x))) = \bm{E}((z + w)\cdot \bm{L}(G(x)))$\\ 
				$\phantom{G(x)^zG(x)^w = \bm{E}(z\cdot \bm{L}(G(x)) + w \cdot \bm{L}(G(x)))} = G(x)^{z + w}$
			\end{tabular}\retTwo\par}
		\end{myIndent}

		\item $\left(G(x)^z\right)^{w} = G(x)^{zw}$
		
		\begin{myIndent}
			\exTwo Proof:

			{\centering
			\begin{tabular}{l}
				$\left(G(x)^z\right)^{w} = \left(\bm{E}(z \cdot \bm{L}(G(x)))\right)^{w}$\\
				
				$\phantom{\left(G(x)^z\right)^{w}} = \bm{E}(w\cdot \bm{L}(\bm{E}(z \cdot \bm{L}(G(x))))) = \bm{E}(w \cdot (z \cdot \bm{L}(G(x))))$\\ 

				$\phantom{\left(G(x)^z\right)^{w} = \bm{E}(w\cdot \bm{L}(\bm{E}(z \cdot \bm{L}(G(x)))))} = \bm{E}(wz \cdot \bm{L}(G(x))) = G(x)^{zw}$
			\end{tabular}\retTwo\par}
		\end{myIndent}
	\end{itemize}

	Using those identities, here are some corollaries:\newpage

	\begin{itemize}		
		\item $G(x)^{-a}$ gives the multiplicative inverse of $G(x)^a$ 
		\begin{myDindent}\myComment
			(this proves that this definition of exponentiation agrees with our more restricted definition from math 188 on all rational numbers).
		\end{myDindent}
		
		\begin{myIndent}\exTwo
			Proof:\\
			Firstly note that:
			
			{\centering $G(x)^{-a}G(x)^a = G(x)^{-a + a} = G(x)^0 = \bm{E}(0 \cdot \bm{L}(G(x))) = \bm{E}(0)$\retTwo\par}

			Seconly, note that $\bm{E}(0) = \sum_{n \geq 0} \frac{0^n}{n!} = \frac{0^0}{0!} = 1$ (as a reminder we are using the convention that $0^0 = 1$). Therefore, $G(x)^{-a}G(x)^a = 1$, meaning $G(x)^{-a}$ is the multiplicative inverse of $G(x)^a$.\retTwo
		\end{myIndent}

		\item $V$ is a subgroup of $\mathbb{C}[[x]]$ under addition, $W$ is a group under multiplication, and $\bm{E}$ is a group isomorphism between $(V, +)$ and $(W, \cdot)$.
		
		\begin{myIndent}\exTwo
			Proof:\\
			One can easily see without our prior reasoning that $(V, +)$ is subgroup of $\mathbb{C}[[x]]$ under addition (with $0$ as it's identity). \retTwo

			Meanwhile, by our previous corollary we can see that all $G(x) \in W$ have a multiplicative inverse in $W$. Specifically since $G(x) = G(x)^{1}$, we know by the previous corollary that $G(x)$ has the inverse $G(x)^{-1}$ inside $W$.\\ Combining that with the fact that $1 \in W$ and $G(x)H(x) \in W$ when\\ $G(x), H(x) \in W$, we know now that $W$ is a group under multiplcation.\retTwo

			Finally, we know $\bm{E}$ is a group isomorphism because of part (c) of this\\ exercise as well as the fact that $\bm{E}(0) = 1$.\retTwo
		\end{myIndent}

		\item Power Rule: Given $G(x) \in W$, if $H(x) = G(x)^z$, then
		
		{\centering $\Df H(x) = z\Df G(x) G(x)^{z-1}$.\\ [-16pt]\par}
		
		\begin{myIndent}\exTwo
			Proof:

			{\centering
			\begin{tabular}{l}
				$\Df H(x) = \Df (\bm{E}(z \cdot \bm{L}(G(x))))(x)$\\ [4pt]

				$\phantom{\Df H(x)} = \Df(z \cdot \bm{L}(G(x)))(x)\cdot \bm{E}(z \cdot \bm{L}(G(x)))$\\ [4pt]

				$\phantom{\Df H(x)} = z\cdot\Df(\bm{L}(G(x)))(x)\cdot G(x)^z = z\frac{\Df G(x)}{G(x)}G(x)^z$\\ [4pt]

				$\phantom{\Df H(x) = z\cdot\Df(\bm{L}(G(x)))(x)\cdot G(x)^z} = z\Df G(x) G(x)^{-1} G(x)^z$\\ [4pt]

				$\phantom{\Df H(x) = z\cdot\Df(\bm{L}(G(x)))(x)\cdot G(x)^z} = z\Df G(x)G(x)^{z-1}$
			\end{tabular}\retTwo\par}
		\end{myIndent}

		\item Binomial Theorem: Given $z \in \mathbb{C}$, we have that:
		
		{\centering $(1 + x)^{z} = \sum\limits_{n \geq 0}\binom{z}{n}x^n$ where $\binom{z}{n} = \frac{z(z-1)\cdots(z-n+1)}{n!}$ when $n > 0$ and $1$\\ [-12pt] \phantom{aaaaaaaaaaaaaaaaaaaaaaaaaaaaaaaaaaaaaaaaaaaaaaaa}when $n = 0$.\retTwo\par}

		\begin{myIndent}\exTwo
			Proof:\\
			Note that $[x^n](1 + x)^z = \frac{1}{n!}\Df^n((1+x)^z)(0)$. Also, by induction using the power rule we can say for $n > 0$ that:\newpage

			{\centering\begin{tabular}{l}
				$\Df^n((1+x)^z)(x) = z\Df^{n-1}((1+x)^z)(x)$\\
				
				$\phantom{\Df^n((1+x)^z)(x)} = z(z-1)\Df^{n-2}((1+x)^z)$\\
				
				$\phantom{\Df^n((1+x)^z)(x)} = \cdots = z(z-1)\cdots(z-n+1)(1+x)$
			\end{tabular}\retTwo\par}

			Therefore $\Df^n((1+x)^z)(0) = z(z-1)\cdots(z-n+1)(1+0)$ and we thus have that for $n > 0$.

			{\centering$[x^n](1 + x)^z = \frac{1}{n!}z(z-1)\cdots(z-n+1) = \binom{z}{n}$.\retTwo\par}

			Meanwhile, if $n = 0$, then $[x^0](1 + x)^z = 1 = \binom{z}{n}$ (because $(1 + x)^z \in W$).
		\end{myIndent}
	\end{itemize}

	Before going on to parts (g) and (h), here are two more identities (where\\ $F(x) \in V$, $G(x) \in W$, and $z \in \mathbb{C}$):
	\begin{itemize}
		\item $\left(\bm{E}(F(x))\right)^z = \bm{E}(z \cdot \bm{L}(\bm{E}(F(x)))) = \bm{E}(zF(x))$
		\item $\bm{L}(G(x)^z) = \bm{L}(\bm{E}(z \cdot \bm{L}(G(x)))) = z\bm{L}(G(x))$\retTwo
	\end{itemize}

	\item[(g)] Show that if $\sum_{i \geq 0}F_i(x)$ converges to $F(x)$, then $\prod_{i \geq 0}\bm{E}(F_i(x))$ converges to $\bm{E}(F(x))$.
	
	\begin{myIndent}\exTwo
		We start by proving the following lemma: If $A(x) \in \mathbb{C}[[x]]$ and $(B_i(x))_{i \in \mathbb{N}}$ is a\\ sequence in $\mathbb{C}[[x]]$ converging to $B(x)$ as $i \rightarrow \infty$ and satisfying that $B_i(0) = 0$\\ for all $i$, then $A(B_i(x)) \rightarrow A(B(x))$ as $i \rightarrow \infty$.

		\begin{myIndent}\exPP
			Proof:\\
			For notation, we'll denote:

			{\centering $A(x) = \sum_{n \geq 0}a_nx^n$,\phantom{..} $B_i(x) = \sum_{n \geq 0}b_n^{(i)}x^n$, and $B(x) = \sum_{n \geq 0}b_nx^n$\retTwo\par}

			To start, note that for all integers $m \geq 0$, we have that $B_i(x)^m \rightarrow B(x)^m$. Also, since $B_i(0) = 0$ for all $i$, we know that $\mdeg B_i(x)^m \geq m$ for all\\ integers $i$ and $m$, and also that $\mdeg B(x)^m \geq m$ for all integers $m$. Thus, fixing $n \geq 0$ we can say that:

			{\centering{\fontsize{11}{13}\selectfont $ [x^n]A(B_i(x)) = [x^n]\sum\limits_{m = 0}^n a_mB_i(x)^m$} and {\fontsize{11}{13}\selectfont $[x^n]A(B(x)) = [x^n]\sum\limits_{m = 0}^n a_mB(x)^m$ }\retTwo\par}

			Next, let $I_m$ be large enough that $[x^n]B_i(x)^m = [x^n]B(x)^m$ for all $i \geq I_m$. Then set $I = \max(I_0, I_1, \ldots, I_m)$ and note that for all $i \geq I$, we have:
			
			{\centering\fontsize{11}{13}\selectfont
			\begin{tabular}{l}
				$[x^n]\sum\limits_{m = 0}^n a_mB_i(x)^m = \sum\limits_{m = 0}^n a_m[x^n](B_i(x)^m)$\\ [11pt]
				$\phantom{[x^n]\sum\limits_{m = 0}^n a_mB_i(x)^m} = \sum\limits_{m = 0}^n a_m[x^n](B(x)^m) = [x^n]\sum\limits_{m = 0}^n a_mB(x)^m$ 
			\end{tabular}\retTwo\par}

			So for all $i \geq I$, we have that $[x^n]A(B_i(x)) = [x^n]A(B(x))$. This proves $A(B_i(x)) \rightarrow A(B(x))$.
			
			\begin{myDindent}\myComment
				I should have proved this in my math 188 notes when I was\\ showing that $((A + B) \circ C)(x) = (A \circ C)(x) + (B \circ C)(x)$\\ and $(AB \circ C)(x) = (A \circ C)(x)(B \circ C)(x)$. But in my defense the professor didn't mention any of these three facts in his notes.\newpage
			\end{myDindent}
		\end{myIndent}

		As a reminder: $\bm{E}(F(x)) = \exp(F(x))$ where $\exp = \sum_{n \geq 0}\frac{1}{n!}x^n$. Also, by our\\ previous lemma, we know that:\\ [-20pt]

		{\centering $\exp(F(x)) = \exp(\lim\limits_{n \rightarrow \infty}\sum\limits_{i = 0}^n F_i(x)) = \lim\limits_{n \rightarrow \infty}\exp(\sum\limits_{i = 0}^n F_i(x))$ \\ [4pt]\par}

		But then $\exp(\sum\limits_{i = 0}^n F_i(x)) = \bm{E}(\sum\limits_{i = 0}^n F_i(x)) = \prod\limits_{i=0}^n \bm{E}(F_i(x))$.\\
		
		So, we have shown that $\bm{E}(F(x)) = \lim\limits_{n \rightarrow \infty}\prod\limits_{i=0}^n \bm{E}(F_i(x)) = \prod\limits_{i\geq 0} \bm{E}(F_i(x))$\\ [-2pt]
	\end{myIndent}

	Side note: The lemma we proved in this part also tells us that if $(B_i(x))_{i \in \mathbb{N}}$ is a sequence in $V$ converging to $B(x)$, then $\bm{E}(B_i(x)) \rightarrow \bm{E}(B(x))$ as $i \rightarrow \infty$. In other words, $\bm{E}$ is a continuous map.
	
	\begin{myDindent}\myComment
		(If $\rho(A(x), B(x)) = \frac{1}{\mdeg (A - B)(x)}$, then $(\mathbb{C}[[x]], \rho)$ is a metric space in\\ which formal power series convergence is equivalent to convergence in\\ this metric space.)
	\end{myDindent}

	\item[(h)] Show that if $\prod_{i \geq 0}G_i(x)$ converges to $G(x)$, then $\sum_{i \geq 0}\bm{L}(G_i(x))$ converges to $\bm{L}(G(x))$.
	
	\begin{myIndent}\exTwo
		Unfortunately, unlike with $\bm{E}$ we do not (currently) have a formal power\\ series $A(x)$ for which we can generally say $\bm{L}(B(x)) = A(B(x))$. Thus, we\\ can't move the limit from inside $\bm{L}$ to outside $\bm{L}$ as easily as we did in part (g)\\ for $\bm{E}$.\\ [-12pt] 

		However, consider that $\lim\limits_{n \rightarrow \infty}\bm{E}(\sum\limits_{i = 0}^n \bm{L}(G_i(x)))$ exists. Specifically:
		
		{\centering\exP\begin{tabular}{l}
			$\lim\limits_{n \rightarrow \infty}\bm{E}(\sum\limits_{i = 0}^n \bm{L}(G_i(x))) = \lim\limits_{n \rightarrow \infty}\bm{E}(\sum\limits_{i = 0}^n \bm{L}(G_i(x)))$\\ [11pt]
			
			$\phantom{\lim\limits_{n \rightarrow \infty}\bm{E}(\sum\limits_{i = 0}^n \bm{L}(G_i(x)))} = \lim\limits_{n \rightarrow \infty} \prod\limits_{i=0}^n\bm{E}(\bm{L}(G_i(x))) = \lim\limits_{n \rightarrow \infty} \prod\limits_{i=0}^n G_i(x) = G(x)$
		\end{tabular}\retTwo\par}

		Thus, if we can show $\sum_{i\geq 0} \bm{L}(G_i(x))$ converges, then we can use the lemma from part (g) to see that:\\ [-14pt]

		{\centering $\bm{E}(\lim\limits_{n \rightarrow \infty}\sum\limits_{i = 0}^n \bm{L}(G_i(x))) = \lim\limits_{n \rightarrow \infty}\bm{E}(\sum\limits_{i = 0}^n \bm{L}(G_i(x))) = G(x)$,\\ [6pt]\par}

		and then by applying $\bm{L}$ to the left and right sides of this equation, we will get our desired result.\retTwo
		
		\begin{myIndent}\exTwoP
			We now show that $\sum_{i \geq 0}\bm{L}(G_i(x))$ converges. Firstly, note that after\\ [2pt] fixing $n \geq 1$, we have that:

			{\centering$[x^n](\bm{L}(G_i(x)))(x) = \frac{1}{n}[x^{n-1}]\frac{\Df G_i(x)}{G_i(x)}$.\newpage\par}
			
			
			Secondly, note that $\prod_{i \geq 0}G_i(x)$ converging to $G(x)$ and $G_i(0) = 1$\\ [2pt] for all $i$ implies that $\mdeg (G_i(x) - 1) \rightarrow \infty$ as $i \rightarrow \infty$ and\\ [2pt] thus $\mdeg \Df G_i(x) = \mdeg (G_i(x) - 1) - 1 \rightarrow \infty$ as $i \rightarrow \infty$.\retTwo
			
			
			Hence, there exists $I_n \geq 0$ such that $i \geq I_n$ implies that $[x^0]\Df G_i(x),$\\ $[x^1]\Df G_i(x), \ldots, [x^{n-1}]\Df G_i(x)$ are all $0$. In turn, we have for all $i \geq I_n$ that $[x^{n-1}](\Df G_i(x) \cdot \frac{1}{G_i(x)}) = 0$ and $[x^n](\bm{L}(G_i(x)))(x) = \frac{1}{n}\cdot 0 = 0$.\retTwo

			Since we also have by the definition of $\bm{L}$ that $(\bm{L}(G_i(x)))(0) = 0$ for all\\ $i$, we can thus conclude that: $\lim\limits_{i \rightarrow \infty} \mdeg(\bm{L}(G_i(x))) = 0$.\\ [0pt]

			This proves that $\sum_{i \geq 0}\bm{L}(G_i(x))$ converges.\retTwo
		\end{myIndent}
	\end{myIndent}

	To finish off, here is an identity: If $z, \alpha \in \mathbb{C}$ and $G(x) = (1-\alpha x)^z$, then\\ $\bm{L}(G(x)) = \sum\limits_{n \geq 1}\frac{-z}{n}\alpha^nx^n$.

	\begin{myIndent}\exTwo
		Proof:\\
		We already showed that $\bm{L}((1 - \alpha x)^z) = z\cdot \bm{L}(1 - \alpha x)$. Also:

		{\centering $\Df(\bm{L}(1-\alpha x))(x) = \frac{\Df(1 - \alpha x)}{1 - \alpha x} = -\alpha \sum\limits_{n \geq 0} \alpha^n x^n = \sum\limits_{n \geq 0}-\alpha^{n+1}x^n$ \\ [9pt]\par}

		Thus, we have that $z \cdot \bm{L}(1 - \alpha x) = z \cdot \sum\limits_{n \geq 1}\frac{-1}{n}\alpha^{(n+1)-1}x^n = \sum\limits_{n \geq 1}\frac{-z}{n}\alpha^n x^n$.\retTwo
	\end{myIndent}

	Some thoughts on this:
	\begin{itemize}
		\item $\bm{L}(\frac{1}{1-x}) = \sum_{n \geq 1}\frac{+1}{n}(1)^n x^n = \sum_{n \geq 1}\frac{1}{n}x^n$ which is importantly the same\\ as what we found in class.
		
		\item Given that $G(x)$ is a polynomial with constant coefficient $1$, we can combine this identity with part (e) to calculate $\bm{L}(G(x))$.
		\begin{myIndent}\exTwo
			Specifically, let $G(x) = (1 - \frac{1}{\gamma_1}x)\cdots(1-\frac{1}{\gamma_k}x)$ where $\gamma_1, \ldots, \gamma_k$ are the roots of $G(x)$. Since $G(0) = 1$ by assumption, we know that $\gamma_j \neq 0$ for all $j$. Then we have that:

			{\centering
			\begin{tabular}{l}
				$\bm{L}(G(x)) = \bm{L}((1 - \frac{1}{\gamma_1}x)\cdots(1-\frac{1}{\gamma_k}x))$\\ [6pt]

				$\phantom{\bm{L}(G(x))} = \bm{L}(1 - \frac{1}{\gamma_1}x) + \ldots + \bm{L}(1 - \frac{1}{\gamma_k}x)$\\ [-2pt]

				$\phantom{\bm{L}(G(x))} = \sum\limits_{n \geq 1}\frac{-1}{n}\gamma_1^{-n}x^n + \ldots + \sum\limits_{n \geq 1}\frac{-1}{n}\gamma_k^{-n}x^n = \sum\limits_{n \geq 1}\frac{-1}{n}\left(\vphantom{\int}\right.\hspace{-0.2em}\sum\limits_{j=1}^k \gamma_j^{-n}\hspace{-0.2em}\left.\vphantom{\int}\right)x^n$
			\end{tabular}\retTwo\par}
		\end{myIndent}
	\end{itemize}

	And now I'm out of ideas of what else to do with this homework problem.
\end{enumerate}
\newpage

\dispDate{12/19/2024}

For the next while I want to work through some of the exercises in Folland's \textit{Real Analysis} about the Cantor set and Cantor function. Assume for this section that $\mathbb{R}$\\ is equipped with the standard metric $\rho$ and that we are using the complete Lebesgue measure space $(\mathbb{R}, \mathcal{L}, m)$. \retTwo

If $I$ is a bounded interval and $\alpha \in (0, 1)$, then call the open interval with the same\\ midpoint as $I$ and length equal to $\alpha$ times the length of $I$ the "open middle $\alpha$th"\\ of $I$. If $(\alpha_j)_{j \in \mathbb{N}}$ is a sequence of numbers in $(0, 1)$, then we can define a decreasing\\ sequence $(K_j)_{j \in \mathbb{N}}$ of closed sets by setting $K_0 = [0, 1]$ and obtaining $K_j$ by\\ removing the open middle $\alpha_j$th from the intervals that make up $K_{j-1}$. Then\\ $K = \bigcap\limits_{j \in \mathbb{N}} K_j$ is called a \udefine{generalized Cantor set}.\\ [-16pt]

\begin{myTindent}\hThree
	The ordinary \udefine{Cantor set} $C$ is obtained by setting all $\alpha_j$ equal to $\sfrac{1}{3}$.\retTwo
\end{myTindent}

\exOne
\blab{Exercise 2.27:} Let $K = \bigcap K_j$ be a generalized Cantor set created using the\\ sequence $(\alpha_j)_{j \in \mathbb{N}}$ in $(0, 1)$. Prove that $K$ is compact, perfect (i.e. closed and\\ has no isolated points), nowhere dense (i.e. not dense in any nonempty open\\ set), and totally disconnected (i.e. the only connected subsets of $K$ are single\\ points).

\begin{myIndent}\exTwoP
	\begin{itemize}
		\item $K$ is closed because it is an intersection of closed sets. Also, $K$ is a bounded set in $\mathbb{R}$ because it is a subset of $[0, 1]$. Thus, $K$ is compact.\retTwo
		
		\item Let $x \in K$. Then for any $\varepsilon > 0$, pick $J \in \mathbb{N}$ with $2^{-J} < \varepsilon$. Note that all intervals of $K_j$ have a length at most $2^{-j}$. After all, when going from $K_{j - 1}$ to $K_j$, we split all the intervals of $K_{j-1}$ in half and then remove an additional amount of length determined by $\alpha_j$. So, let $I$ be the interval of $K_J$ containing $x$. Then both endpoints of $I$ are in $K$ and also in $B(\varepsilon, x)$. And, at least one of those endpoints is not $x$. So $x$ is a limit point of $K$. Since $K$ is also closed, we have that $K$ is perfect.\retTwo
		
		\item Let $x, y \in K$ and without loss of generality assume $x < y$. Then we know there must exist some integer $J \in \mathbb{N}$ such that $x$ and $y$ are in different intervals of $K_J$. Afterall, as previously mentioned, points in the same interval of $K_j$ are within $2^{-j}$ distance of each other. So if no such $J$ exists, then $\rho(x, y) < 2^{-j}$ for all $j$, meaning $x = y$.\retTwo
		
		We can specifically choose $J$ to be the least integer such that $x$ and $y$ are in two different intervals of $K_J$. Then both $x$ and $y$ are in the same interval $I$ of $K_{J-1}$, but the midpoint of that interval $z$ is not in $K_J \subseteq K$ and $x < z < y$. By a theorem in 140A, this proves that $K$ is totally disconnected since for all $x, y \in K$, $[x, y] \not\subseteq K$ unless $x = y$.\newpage

		\item Since $K$ is perfect, we know that $K$ is only dense on subsets of $K$. However, since all open sets in $\mathbb{R}$ are countable unions of open intervals and $K$ contains no nonempty open intervals since $K$ is totally disconnected, we know that $K$ has no nonempty open subsets.\retTwo
	\end{itemize}
\end{myIndent}

\hOne
Trying to explicitely write out the bijection between $[0, 1]$ and a generalized Cantor set $K = \bigcap K_j$ would be really time consuming and awkward. So I'm going to be more handwavey

\begin{myIndent}\hTwo
	We can define an injection from $\{0, 1\}^\omega$ to $K$ as follows:
	
	\begin{myIndent}\hThree
		Given $\bm{x} \in \{0, 1\}^\omega$, we can define a convergent subsequence in $K$. Specifically, set $a_0 = 0$. Then recursively for $j > 0$, we know $a_{j-1}$ falls into some interval $I$ of $K_{j-1}$. Futhermore, we know that $I$ gets split into two disjoint intervals $I_0$ and $I_1$ when going from $K_{j-1}$ to $K_j$ (take $I_0$ to be the lower interval). Then, let $a_j$ be the left bound on $I_n$ where $n$ is the value at the $j$th index of $\bm{x}$.\retTwo

		Since the endpoints of the intervals in each $K_j$ are all in the final intersection,\\ we know that $(a_j)_{j \in \mathbb{N}}$ is a sequence contained in $K$. Also, $\rho(a_j, a_{j+1}) < 2^{-j}$ for all $j \geq 0$. From that you can easily work out that $(a_j)_{j \in \mathbb{N}}$ is Cauchy. Thus, since $K$ is closed, we know that $(a_j)_{j \in \mathbb{N}}$ converges to some number $y \in K$.\retTwo

		The mapping $\bm{x} \mapsto y$ is injective because $\bm{x}$ uniquely determines which interval\\ of $K_j$ that $y$ is in for all $j$ (specifically the same interval as $a_j$ for each $j$). If $\bm{x}^\prime$ is\\ another sequence of $0$s and $1$s mapped to $y^\prime$, and $\bm{x}$ and $\bm{x}^\prime$ differ at position $J$,\\ then $y$ and $y^\prime$ will be in two different intervals of $K_J$. Since those intervals are\\ disjoint, we know that $y \neq y^\prime$.\retTwo
	\end{myIndent}

	It is possible to show that our above injection is also surjective. However, it's quicker to just say $\mathfrak{c} = \card(\{0, 1\}^{\omega}) \leq \card(K) \leq \card(\mathbb{R}) = \mathfrak{c}$. Thus generalized Cantor sets have the cardinality of the continuum.\retTwo
\end{myIndent}

Finally, note that given a generalized Cantor set $K = \bigcap K_j$, because $m(K_1) < 1$ and $(K_j)_{j \in \mathbb{N}}$ is a decreasing sequence of sets, we know that:

{\centering $m(K) = \lim\limits_{j \rightarrow \infty}m(K_j) = \lim\limits_{j \rightarrow \infty} 2^j\prod\limits_{i=1}^{j}\frac{(1 - \alpha_i)}{2} = \prod\limits_{j=1}^\infty (1-a_j)$ \retTwo\par}

\exOne\hypertarget{Folland Exercise 1.32}{\blab{Exercise 1.32}:}
\begin{enumerate}
	\item[(a)] Suppose $(a_j)_{j \in \mathbb{N}}$ is a sequence in $(0, 1)$. $\prod_{j=1}^\infty(1 - a_j) > 0$ if and only if\\ $\sum_{j=1}^\infty a_j < \infty$.
	\begin{myIndent}\exTwoP
		To start, note that for all $x \in [0, 1)$, we have that $0 \leq x \leq -\log(1-x)$. After all, $x+\log(1-x)$ equals $0$ at $x = 0$. Also, its derivative $1 - \frac{1}{1-x}$ is negative on $[0, 1)$. This tells us that $x + \log(1-x)$ is strictly decreasing as $x$ increases, meaning that the difference of $x$ and $-\log(1-x)$ is less than $0$ for all $x > 0$.\newpage

		This lets us conclude that if $\sum_{j=1}^\infty -\log(1-a_j)$ converges, then by comparison test we must also have that $\sum_{j=1}^\infty a_j$ converges.\retTwo
		
		Meanwhile, for all $x \in [0, \sfrac{1}{2})$ we have that $0 \leq -\log(1-x) \leq 2x$. To see this, note that $2x +\log(1-x)$ also equals $0$ at $x = 0$. But it's derivative: $2 - \frac{1}{1-x}$, is positive for $x < \sfrac{1}{2}$. This tells us that $2x + \log(1-x)$ is strictly increasing for $x \in (0, \sfrac{1}{2})$. So, the difference of $2x$ and $-\log(1-x)$ is greater than $0$ for all $x \in (0, \sfrac{1}{2})$.\retTwo

		Importantly, if $\sum_{j=1}^\infty a_j$ converges, then we know that all $a_j$ after a certain index $J$ will be in the interval $(0, \sfrac{1}{2})$. Then, since the sum of the $-\log(1 - a_j)$ for $j \leq J$ will be finite and since we can use comparison test on the remaining terms, we know that $\sum_{j=1}^\infty -\log(1-a_j)$ also converges.\retTwo

		In other words, we've shown that:
		
		{\centering $\sum\limits_{j=1}^\infty a_j < \infty$ if and only if $\sum\limits_{j=1}^\infty -\log(1 - a_j) < \infty$.\retTwo\par}

		Next, note that $\sum_{j=1}^\infty -\log(1 - a_j)$ converges if and only $\sum_{j=1}^\infty \log(1 - a_j)$ converges.\retTwo

		Finally, consider that $\prod_{j=1}^\infty (1-a_j) > 0$ if and only if $\sum_{j=1}^\infty \log(1 - a_j) > -\infty$.

		\begin{myIndent}\exPP
			If $\prod_{j=1}^\infty (1-a_j) = \alpha > 0$, then we know that $\log(\alpha)$ is a finite negative value. And because $\log$ is a continuous function, we know:

			{\centering\fontsize{11}{13}\selectfont
			\begin{tabular}{l}
				$\log(\alpha) = \log(\lim\limits_{N \rightarrow \infty}\prod_{j=1}^N (1 - a_j))$\\ [6pt]
				$\phantom{\log(\alpha)} = \lim\limits_{N \rightarrow \infty}\log(\prod_{j=1}^N(1-a_j)) = \lim\limits_{N \rightarrow \infty}\sum\limits_{j=1}^N \log(1 -a_j) = \sum\limits_{j=1}^\infty\log(1 - a_j)$
			\end{tabular} \retTwo\par}

			Meanwhile, if $\prod_{j=1}^\infty (1 - a_j) = \lim\limits_{N \rightarrow \infty} \prod_{j=1}^N(1 - a_j) = 0$, then we know that:

			{\centering $\sum\limits_{j=1}^\infty \log(1 - a_j) = \lim\limits_{N\rightarrow \infty}\sum\limits_{j=1}^N \log(1 - a_j) = \lim\limits_{N \rightarrow \infty} \log(\prod\limits_{j=1}^N (1-a_j)) = -\infty$ \retTwo\par}

			
			\begin{myDindent}\myComment
				Note that we always have that $\prod_{j=1}^\infty (1 -a_j) \in [0, 1)$.\retTwo
			\end{myDindent}
		\end{myIndent}
	\end{myIndent}

	\item[(b)] Given $\beta \in (0, 1)$, there exists a sequence $(a_j)_{j \in \mathbb{N}}$ in $(0, 1)$ such that\\ $\prod_{j = 1}^\infty(1 - a_j) = \beta$.
	
	\begin{myIndent}\exTwoP
		Let $c_j = \frac{1}{2^j}(1 - \beta) + \beta$ for all $j \in \mathbb{N}$. That way $(c_j)_{j \in \mathbb{N}}$ is a strictly decreasing sequence in $(0, 1)$ converging to $\beta$. Then, we want to make $\prod_{i=1}^j(1 - a_j) = c_j$ for all $j$. To do this, set $a_j = 1 - \frac{c_j}{c_{j-1}}$ for all $j$. Because $0 < c_j < c_{j-1}$, we know that $\frac{c_j}{c_{j-1}} \in (0, 1)$. And thus, $a_j \in (0, 1)$ for all $j$ as well and $\prod_{j=1}^\infty(1 - a_j) = \beta$.\newpage
	\end{myIndent}
\end{enumerate}

\hOne
Letting $C = \bigcap C_j$ be the standard Cantor set (i.e. where all $\alpha_j = \sfrac{1}{3}$), we now define the Cantor function:

\begin{myIndent}\hTwo
	Note that if $x \in C$, then there exists a unique sequence $(a_j)_{j \in \mathbb{N}}$ with $x = \sum_{j=1}^\infty a_j\frac{1}{3^j}$ and all $a_j$ equal to either $0$ or $2$. (This is because each choice of $a_j$ as either $0$ or $2$ corresponds to which subinterval of $C_j$ that $x$ is in.) Let $f(x) = \sum_{j=1}^\infty b_j2^{-j}$ where $b_j = \frac{a_j}{2}$. Note that $f(x)$ is the binary expansion of a number in $[0, 1]$.\retTwo
\end{myIndent}

Observe that for all $y \in [0, 1]$ there exists $x \in C$ with $f(x) = y$. Also, for $x_1, x_2 \in C$ with $x_1 < x_2$, we have that $f(x_1) \leq f(x_2)$ with equality if and only if $x_1$ and $x_2$ are the end points of a removed interval (thus making $x_1$ and $x_2$ correspond to the binary expansions $0.b_1b_2\ldots 0\overline{1}$ and $0.b_1b_2\ldots 1\overline{0}$). This allows us to continuously extend $f$ to all $[0, 1]$ by making $f$ constant on all the intervals between points of $C$.\retTwo

\exOne\blab{Exercise 2.9:} Let $f: [0, 1] \longrightarrow [0, 1]$ be the Cantor function, and let\\ $g(x) = f(x) + x$.

\begin{itemize}
	\item[(a)] $g$ is a bijection from $[0, 1]$ to $[0, 2]$, and $h = g^{-1}$ is continuous from $[0, 2]$ to $[0, 1]$.
	
	\begin{myIndent}\exTwoP
		To start, note that $g$ is easily checked to be a strictly increasing function. This proves both that $g$ is injective and that $g$ has a range of $[0, 2]$ since $g(0) = 0$ and $g(1) = 2$. Also, note that $g$ is continuous since both $f(x)$ and $x$ are continuous. Thus, by applying I.V.T, we can say that $g$ is surjective from $[0, 1]$ to $[0, 2]$. This proves that $g$ is a bijection.\retTwo

		The proof that $h = g^{-1}$ is continuous works for any strictly increasing\\ continuous function.
		\begin{myIndent}\exPPP
			Given $y \in [0, 2]$ there exists $x \in [0, 1]$ such that $g(x) = y$.\retTwo

			Now let $\varepsilon > 0$ and set $\alpha = \max(0, x - \varepsilon)$ and $\beta = \min(1, x + \varepsilon)$. Then\\ for all $y^\prime \in (g(\alpha), g(\beta))$, we have that $h(y^\prime) \in B(\varepsilon, x)$. So we can set\\ $\delta = \min(|g(\alpha) - y|, |g(\beta) - y|)$. This fulfills the definition of continuity.
			\retTwo
		\end{myIndent}
	\end{myIndent}

   \item[(b)] $m(g(C)) = 1$ where $C = \bigcap C_j$ is the Cantor set.
   
	\begin{myIndent}\exTwoP
		Because $g^{-1}$ is continuous, we know that $g(C)$ and $g(C_j)$ are measurable\\ for all $j$ since $C$ and each $C_j$ are Borel sets. Also $(g(C_j))_{j \in \mathbb{N}}$ is a decreasing\\ sequence of sets with $m(g(C_1)) \leq 2$ and $\bigcap_{j \in \mathbb{N}} g(C_j) = g(C)$. Thus\\ $m(g(C)) = \lim\limits_{j \rightarrow \infty} m(g(C_j))$.\retTwo

		Next note that $C_j$ has $2^j$ many intervals, each with width $3^{-j}$. Also, if\\ $[\alpha, \alpha + 3^{-j}]$ is one of those intervals, then:
		
		{\centering 
		\begin{tabular}{l}
			$g(\alpha + 3^{-j}) - g(\alpha) = f(\alpha + 3^{-j}) + \alpha + 3^{-j} - f(\alpha) - 3^{-j}$\\ [6pt]
			$\phantom{g(\alpha + 3^{-j}) - g(\alpha)} = \sum_{i=1}^j(b_i2^{-i}) + 2^{-j} + \alpha + 3^{-j} - \sum_{i=1}^j(b_i2^{-i}) - \alpha$\\ [6pt]
			$\phantom{g(\alpha + 3^{-j}) - g(\alpha)} = 2^{-j} + 3^{-j}$
		\end{tabular}\newpage\par}

		Thus $m(g(C_j)) = 2^{j}(2^{-j} + 3^{-j}) = 1 + (\frac{2}{3})^j$. Taking $j \to \infty$ we get the desired result.\retTwo
	\end{myIndent}
\end{itemize}

\hOne To do the next parts of that exercise, we first need to do a different exercise.\retTwo

\exOne\blab{Exercise 1.29:}

\begin{itemize}
	\item[(a)] Suppose $E \subseteq V$ where $V$ is a Vitali set (see the tangent on page 22) and $E \in \mathcal{L}$. Prove that $m(E) = 0$.
	
	\begin{myIndent}\exTwoP
		For all $r \in \mathbb{Q} \cap [-1, 1]$, define $E_r = \{v + r : v \in E\}$. By translation invariance, we know that $E_r$ is measurable with $m(E_r) = m(E)$ for all $r$. Also each $E_r$ is disjoint and $\bigcup_{r \in \mathbb{Q} \cap [-1,1]}E_r \subseteq [-1, 2]$. It follows that $\bigcup_{r \in \mathbb{Q} \cap [-1, 1]} E_r$ is measurable and:

		{\centering $3 \geq m(\bigcup_{r \in \mathbb{Q} \cap [-1, 1]} E_r) = \sum_{r \in \mathbb{Q} \cap [-1, 1]} m(E_r) = \sum_{r \in \mathbb{Q} \cap [-1, 1]} m(E)$\retTwo\par}

		The only way this is possible is if $m(E) = 0$.\retTwo
	\end{myIndent}

	\item[(b)] If $m(E) > 0$, then there exists a nonmeasurable set $N \subseteq E$. 
	
	\begin{myIndent}\exTwoP
		\begin{myTindent}\myComment
			Sidenote: the converse of this statement is trivially true because $(\mathbb{R}, \mathcal{L}, m)$ is complete.\retTwo
		\end{myTindent}

		To start, it suffices to show this for $E \subseteq [0, 1]$. After all, we can use the\\ translation invariance of the Lebesgue measure to move $N$ from $[0, 1]$ to where ever $E$ is (we can't have that $N + r$ is measurable because that would imply $(N + r) - r = N$ is measurable).\retTwo

		Now, let $V$ be a Vitali set and $V_r = \{v + r : v \in V\}$ for all $r \in [-1, 1] \cap \mathbb{Q}$. If $E \cap V_r$ is not measurable for some $r$, then we are done. So, suppose $E \cap V_r$ is measurable for all $r$.  Then note $\bigcup_{r \in [-1, 1] \cap \mathbb{Q}}(E \cap V_r) = E \cap \bigcup_{r \in [-1, 1] \cap \mathbb{Q}}(V_r)$. Since $[0, 1]$ is a subset of $\bigcup_{r \in [-1, 1] \cap \mathbb{Q}}(V_r)$ and $E \subseteq [0, 1]$, we thus know that $E \cap \bigcup_{r \in [-1, 1] \cap \mathbb{Q}}(V_r) = E$. Additionally, since each $E \cap V_r$ is disjoint, we know that:

		{\centering $m(E) = \sum_{r \in [-1, 1] \cap \mathbb{Q}}m(E \cap V_r)$ \retTwo\par}

		Now hopefully it's clear how part (a) of this exercise extends to each\\ nonmeasurable set $V_r$. Thus, since we assumed each $E \cap V_r$ is a measurable set, we know that $m(E \cap V_r) = 0$. It follows that $m(E) = 0$, a contradiction of our problem.\retTwo
	\end{myIndent}
\end{itemize}

\hOne Now we return to exercise 2.9.\newpage

\exOne

\begin{itemize}
	\item[(c)] $g(C)$ contains a Lebesgue nonmeasurable set $A$ by exercise 1.29. Let\\ $B = g^{-1}(A)$. Then $B$ is Lebesgue measurable but not Borel.
	
	\begin{myIndent}\exTwoP
		Since $C$ is a measurable null set in the complete measure space $(\mathbb{R}, \mathcal{L}, m)$, we have that all subsets of $C$ including $B$ must be measurable. So $B \in \mathcal{L}$.
		
		
		\begin{myIndent}\myComment
			Side note: since $\card(C) = \card(\mathbb{R})$, we know that: 
			
			{\centering $\card(\mathcal{P}(\mathbb{R})) = \card(\mathcal{P}(C)) \leq \card(\mathcal{L}) \leq \card(\mathcal{P}(\mathbb{R}))$.\retTwo\par} 
		\end{myIndent}

		However, because $g^{-1}$ is continuous, we know that $g^{-1}$ is a Borel measurable function. Hence, if $B$ was borel, then we would have to have that $g(B) = A$ is also Borel, thus contradicting that $A$ is not measurable. So, we know $B$ is measurable but not Borel.\retTwo
	\end{myIndent}

	\item[(d)] There exists a Lebesgue measurable function $F$ and continuous function $G$ on $\mathbb{R}$ such that $F \circ G$ is not Lebesgue measurable.
	
	\begin{myIndent}\exTwoP
		Define $G$ by continuously extending $g^{-1}(x)$ to all $\mathbb{R}$ (One way to do this would be to set $G(x) = x$ when $x < 0$ and $G(x) = 1$ when $x > 2$). Then set $F = \chi_B$ where $B$ is the set found in part (c). Now $(F \circ G)^{-1}(\{1\}) = A$ is not Lebesgue measurable. So $F \circ G$ is not a Lebesgue measurable function.
		\begin{myIndent}\myComment
			The significance of this result is that we've proven that $G$ is continuous but not Lebesgue measurable.\retTwo
		\end{myIndent}
	\end{myIndent}
\end{itemize}

\hOne
One more interesting observation Folland makes is that the collection of Borel sets $\mathcal{B}_\mathbb{R}$ only has the cardinality of the continuum, meaning that most measurable sets are not Borel.


\begin{myIndent}\hTwo
	To prove this, firstly note that by exercise 1.3 in my LaTeX math 240A notes (page 11), we know that $\card(\mathcal{B}_{\mathbb{R}}) \geq \mathfrak{c}$.\retTwo

	Also, consider the following lemmas:
	\begin{enumerate}
		\item \blab{Proposition 0.14:}
			
		\begin{itemize}
			\item[(a)] If $\card(X) \leq \mathfrak{c}$ and $\card(Y) \leq \mathfrak{c}$, then $\card(X \times Y) \leq \mathfrak{c}$.
			\begin{myIndent}
				Proof:\\
				It suffices to take $X = Y = \mathcal{P}(\mathbb{N})$ since then both $X$ and $Y$ have the largest cardinality we are allowing. Next, define $\psi, \phi: \mathbb{N} \rightarrow \mathbb{N}$ by $\psi(n) = 2n$ and $\phi(n) = 2n - 1$. Then $f: \mathcal{P}(\mathbb{N}) \times \mathcal{P}(\mathbb{N}) \rightarrow \mathcal{P}(\mathbb{N})$ given by $f(A, B) = \psi(A) \cup \phi(B)$ is a bijection.
				\retTwo
			\end{myIndent}

			\item[(b)] If $\card(A) \leq \mathfrak{c}$ and $\card(\mathcal{E}_\alpha) \leq \mathfrak{c}$ for all $\alpha \in A$, then $\card(\bigcup_{\alpha \in A}\mathcal{E}_\alpha) \leq \mathfrak{c}$.
			\begin{myIndent}
				Proof:\\
				For each $\alpha \in A$ there is a surjection $f_\alpha: \mathbb{R} \longrightarrow \mathcal{E}_\alpha$. So define the function $f: \mathbb{R} \times A \longrightarrow \bigcup_{\alpha \in A}\mathcal{E}_\alpha$ by $f(x, \alpha) = f_\alpha(x)$. This is a surjection. So, we know that $\card(\bigcup_{\alpha \in A}\mathcal{E}_\alpha) \leq \card(\mathbb{R} \times A)$ and the latter set by part (a) has no greater than the cardinality of the continuum.\newpage
			\end{myIndent}
		\end{itemize}

		\item If $\card(\mathcal{E}) \leq \mathfrak{c}$, then $\card(\mathcal{E}^{\omega}) \leq \mathfrak{c}$.
		
		\begin{myIndent}\hThree
			To see this, we can assume $\mathcal{E} = \{0, 1\}^\omega$ since then $\card(\mathcal{E}) = \mathfrak{c}$. Then note that we can use a diagonalization argument to create a bijection between\\ between $\mathcal{E}$ and $\mathcal{E}^{\omega}$. Writing it out would be a pain so do it yourself.\retTwo
		\end{myIndent}
	\end{enumerate}
	
	Now recall from Folland's proposition 1.23 (the bonus proposition written on page 38 of my LaTeX notes for math 240A) the following construction of $\mathcal{B}_{\mathbb{R}}$.

	\begin{myIndent}
		Let $S_\Omega$ be a minimal uncountable set (by constructing $S_\Omega$ from $\mathbb{R}$ using the construction I copied from Munkres on page 14 of this pdf, we can guarentee that $S_\Omega \subseteq \mathbb{R}$ and thus $\card(S_\Omega) \leq \mathfrak{c}$).\retTwo

		Next, using $0$ to refer to the least element of $S_\Omega$, let $\mathcal{E}_0$ be the set of all rays of the form $[a, \infty)$ where $a \in \mathbb{R}$. Then for all other $\alpha \in S_\Omega$:
		\begin{itemize}
			\item If $\alpha$ has a direct predecessor $\beta$, then let $\mathcal{E}_\alpha$ be the collection of all\\ countable unions of  and complements of sets from $\mathcal{E}_\beta$.
			\item If $\alpha$ does not have a direct predecessor, then set $\mathcal{E}_\alpha = \bigcup_{\beta \in S_{\alpha}}\mathcal{E}_\alpha$.
		\end{itemize}

		Finally, $\mathcal{B}_\mathbb{R} = \bigcup_{\alpha \in S_{\Omega}}\mathcal{E}_\alpha$.\retTwo
	\end{myIndent}

	We obviously have that $\card(\mathcal{E}_0) = \mathfrak{c}$. Then using transfinite induction along with our two previously mentioned lemmas, we can conclude that $\card(\mathcal{E}_\alpha) \leq \mathfrak{c}$ for all $\alpha \in S_{\Omega}$. So by part (b) of proposition 0.14, we conclude that:
	
	{\centering $\card(\mathcal{B}_\mathbb{R}) = \card(\bigcup_{\alpha \in S_{\Omega}}\mathcal{E}_\alpha) \leq \mathfrak{c}$.\retTwo\par}

	Since $\mathfrak{c} \leq \card(\mathcal{B}_\mathbb{R}) \leq \mathfrak{c}$, we know that $\card(\mathcal{B}_\mathbb{R}) = \mathfrak{c}$.\retTwo
\end{myIndent}

\mySepTwo

\dispDate{7/5/2025}

I'm going to be taking more analysis notes from Folland. I'm starting with the\\ section: \ul{The Dual of $C_0(X)$}. Here, $X$ refers to an LCH space.

\begin{myIndent}\hTwo
	To start out, we shall identify all positive bounded linear functionals on $C_0(X)$. Note that if $I$ is such a functional on $C_0(X)$, then we know it is also a positive bounded linear functional on the subspace $C_c(X)$. Meanwhile going in reverse, we have that if $I(f) = \int f \df \mu$ is a positive linear function on $C_c(X)$ that is bounded, then we can uniquely extend it to a positive bounded linear functional on $C_0(X)$ by defining $I(f) = \lim_{n\to \infty}I(f_n)$ for any $f \in C_0(X)$ where $\{f_n\}_{n \in \mathbb{N}}$ is any sequence in $C_c(X)$ converging to $f$ uniformly. So, given any Radon measure $\mu$, we need to determine when $I(f) = \int f \df \mu$ is bounded.\retTwo
	

	Since $X$ is open and $\mu$ is Radon, by the Riesz Representation theorem:
	
	{\centering $\mu(X) = \sup\{I(f) : f \in C_c(X),\hspace{0.2em} \supp(f) \subseteq X,\hspace{0.2em} 0 \leq f \leq 1\}$. \retTwo\par}

	The second condition is redundant and $I(f) = \int f \df \mu$. So we can rewrite this as $\mu(X) = \sup\{\int f \df \mu : f \in C_c(X), \hspace{0.2em} 0 \leq f \leq 1\}$. We now claim $I$ is bounded iff $\mu(X) < \infty$, and that when $I$ is bounded, $\|I\|_{\opnorm} = \mu(X)$.
	
	\begin{myIndent}\exTwo
		$(\Longrightarrow)$\\
		Suppose $\mu(X) = \infty$. Then for any $N > 0$, there is a function $f \in C_c(X)$ with $0 \leq f \leq 1$ such that $\int f \df \mu \geq N$. And since $\|f\|_u \leq 1$, we know that if $|I(f)| \leq C\|f\|_u$, then $N \leq |\int f \df \mu| = |I(f)| \leq C$. This proves no finite $C$ works for all $f \in C_c(X)$, and thus $I$ is unbounded.\retTwo

		$(\Longleftarrow)$\\
		Suppose $\mu(X) < \infty$ and then consider any $f \in C_c(X)$ with $\|f\|_u = 1$. Note that $|I(f)| = |\int f \df \mu| \leq \int |f|\df \mu$. Then since $0 \leq |f| \leq 1$, we know that $\int |f|\df \mu \leq \mu(X)$. So $\|I\|_\opnorm$ exists and is at most $\mu(X)$.\retTwo
		
		To prove that $\mu(X) = \|I\|_\opnorm$, let $\varepsilon > 0$ and pick $f \in C_c(X)$ with $0 \leq f \leq 1$ such that $\int f \df \mu > \mu(X) - \varepsilon$. Thus we have that $\|I\|_\opnorm \|f\|_u > \mu(X) - \varepsilon$. Then since $\|f\|_u \leq 1$, we have that $\|I\|_\opnorm > \mu(X) - \varepsilon$. Taking $\varepsilon \to 0$ finishes the proof.\retTwo
	\end{myIndent}

	So, the positive bounded linear functionals on $C_0(X)$ are precisely given by\\ integration against finite Radon measures (and this correspondence is one-to-one by the Riesz Representation theorem). Next, we identify the other linear\\ functionals on $C_0(X)$.\retTwo

	\exOne\uprop{Lemma 7.15:}  If $I \in C_0(X, \mathbb{R})^*$, there exists positive functionals $I^{\pm} \in C_0(X, \mathbb{R})^*$ such that $I = I^+ - I^-$.

	\begin{myIndent}\exTwoP
		Proof:\\
		If $f \in C_0(X, [0, \infty))$, define:
		
		{\centering$I^+(f) = \sup\{I(g) : g \in C_0(X, \mathbb{R}),\hspace{0.2em} 0 \leq g \leq f\}$.\retTwo\par}

		If $c \geq 0$, then clearly $I^+(cf) = cI^+(f)$. Meanwhile, let $f_1, f_2 \in C_0(X, [0, \infty))$. To show that $I^+(f_1 + f_2) = I^+(f_1) + I^+(f_2)$, first suppose $0 \leq g_1 \leq f_1$ and $0 \leq g_2 \leq f_2$. Then $0 \leq g_1 + g_2 \leq f_1 + f_2$. So, $I^+(f_1 + f_2) \geq I(g_1 + g_2) = I(g_1) + I(g_2)$. By taking $I(g_1) \to I^+(f_1)$ and $I(g_2) \to I^+(f_2)$, we then get that $I^+(f_1 + f_2) \geq I^+(f_1) + I^+(f_2)$.\retTwo
		
		On the other hand, if $0 \leq g \leq f_1 + f_2$, let $g_1 = \min(g, f_1)$ and $g_2 = g - g_1$.\\ Thus $0 \leq g_1 \leq f_1$, $0 \leq g_2 \leq f_2$, and $g_1, g_2$ are continuous. This guarentees $g_1, g_2 \in C_0(X, [0, \infty))$. So, $I(g) = I(g_1) + I(g_2) \leq I^+(f_1) + I^+(f_2)$. And, taking $I(g) \to I^+(f_1 + f_2)$ gets us $I^+(f_1 + f_2) \leq I^+(f_1) + I^+(f_2)$.\retTwo

		Now, we extend $I^+$ to a positive linear functional in $C_0(X, \mathbb{R})^*$. Given\\ $f \in C_0(X, \mathbb{R})$, let $f^+$ and $f^-$ be the positive and negative parts of $f$. Then, $f^+, f^- \in C_0(X, [0, \infty))$. So, define $I^+(f) = I^+(f^+) - I^+(f^-)$. This is linear because if $c \in \mathbb{R}$, then ignoring the trivial edge case where $c = 0$:\newpage
		
		{\centering\begin{tabular}{l}
		$I^+(cf) = \sgn(c)\left(I^+(|c|f^+) - I^+(|c|f^{-})\right)$\\
		$\phantom{I^+(cf)} = \sgn(c)|c|\left(I^+(f^+) - I^+(f^-)\right) = cI^+(f)$.
		\end{tabular}\retTwo\par}

		Also, suppose $f = g + h$ where $g, h \in C_0(X, \mathbb{R})$. Then $f^+ + g^- + h^- = f^- + g^+ + h^+$ where all the functions in that expression are in $C_0(X, [0, \infty))$. So, we know from our earlier work that:

		{\centering\fontsize{11}{13}\selectfont \begin{tabular}{l}
			$I^+(f^+) + I^+(g^-) + I^+(h^-) = I^+(f^+ + g^- + h^-)$\\
			$\phantom{I^+(f^+) + I^+(g^-) + I^+(h^-) } = I^+(f^- + g^+ + h^+) = I^+(f^-) + I^+(g^+) + I^+(h^+)$
		\end{tabular} \retTwo\par}

		Or in other words:

		{\centering\fontsize{11}{13}\selectfont $I^+(f) = I^+(f^+) - I^+(f^-) = I^+(g^+) - I^+(g^-) + I^+(h^+) - I^+(h^-) = I^+(g) + I^+(h)$\retTwo\par}

		To show that $I^+$ is bounded, first note that if $f \in C_0(X, [0, \infty))$, then since $|I(g)| \leq \|I\|\|g\|_u \leq \|I\|\|f\|_u$ for all $0 \leq g \leq f$ and $I(0) = 0$, we have $0 \leq I^+(f) \leq \|I\|\|f\|_u$. (Note, this also proves $I^+$ is positive). Meanwhile, if $f \in C_0(X, \mathbb{R})$, then $I^+(f) = I^+(f^+) - I^+(f^-)$ where both terms in that difference are positive. Hence, we can say that:

		{\centering $|I^+(f)| \leq \max(I^+(f^+), I^-(f^-)) \leq \|I\|\max(\|f^+\|_u, \|f^-\|_u) = \|I\|\|f\|_u$ \retTwo\par}

		Thus, we've finished constructing $I^+$. So now define $I^-(f) = I^+(f) - I(f)$. Then we know $I^- \in C_0(X, \mathbb{R})^*$ because $C_0(X, \mathbb{R})^*$ is a real vector space. Also, $I^-$ is positive because if $f \geq 0$, then you can see from our definition of $I^+(f)$ on $C_0(X, [0, \infty))$ that $I^+(f) \geq I(f)$. Hence, $I^-(f) = I^+(f) - I(f)$ is also nonnegative. $\blacksquare$\retTwo
	\end{myIndent}

	\hOne Now any $I \in C_0(X)^*$ is uniquely determined by its restriction $J$ to $C_0(X, \mathbb{R})$.

	\begin{myIndent}\why

		{\centering $I(f) = I(\rea{f} + i\ima{f}) = I(\rea{f}) + iI(\ima{f}) = J(\rea{f}) + iJ(\ima{f})$.\retTwo\par}
	\end{myIndent}

	Next, there are two real linear functionals $J_1, J_2 \in C_0(X, \mathbb{R})^*$ such that\\ $J = J_1 + iJ_2$. Specifically, set $J_1(f) = \rea{J(f)}$ and $J_2 = \ima{J(f)}$. Then clearly $J_1$ and $J_2$ are real linear functionals and they are bounded with\\ $\|J_i\| \leq \|I\|$.\retTwo


	Using our lemma, we can decompose $J_1$ and $J_2$ into differences of positive bounded linear real functionals. I.e., we write $J = J_1^+ - J_1^- + i(J_2^+ - j_2^-)$.\retTwo

	Finally, define $I_1^+, I_1^-, I_2^+, I_2^-$ such that $I_1^+(f) = J_1^+(\rea{f}) + iJ_1^+(\ima{f})$ and the others have analogous definitions. Then all of our $I_i^\pm$ are well-defined complex linear functionals on $C_0(X)$ that are bounded since:
	
	{\centering $|I_i^\pm(f)| \leq \|J_i^\pm\|(\|\rea{f}\|_u + \|\ima{f}\|_u) \leq 2\|J_i^\pm\|\|f\|_u$. \retTwo\par}
	
	Also, all the $I_i^\pm$ are positive since if $f$ is nonnegative, then $I_i^\pm(f) = J_i^\pm(f)$. This means that there are finite Radon measures $\mu_1, \mu_2, \mu_3, \mu_4$ such that\\ $I_1^+(f) = \int f \df \mu_1$, $I_1^-(f) = \int f \df \mu_2$, $I_2^+(f) = \int f \df \mu_3$. and $I_2^-(f) = \int f \df \mu_4$.\newpage

	Additionally:

	{\centering \begin{tabular}{l}
		$I(f) = J(\rea{f}) + iJ(\ima{f})$\\ [4pt]
		$\phantom{I(f)} = J_1(\rea{f}) + iJ_2(\rea{f}) + iJ_1(\ima{f}) + i^2J_2(\ima{f})$\\ [4pt]
		$\phantom{I(f)} = J_1^+(\rea{f}) - J_1^-(\rea{f}) + iJ_2^+(\rea{f}) - iJ_2^-(\rea{f})$\\
		$\phantom{I(f) +} + iJ_1^+(\ima{f}) - iJ_1^-(\ima{f}) + i^2J_2^+(\ima{f}) - i^2J_2^-(\ima{f})$\\ [8pt]
		$\phantom{I(f)} = I_1^+(f) - I_2^-(f) + iI_2^+(f) - iI_2^-(f)$\\ [4pt]
	\end{tabular} \retTwo\par}

	So for any $I \in C_0(X)^*$, there are finite Radon measures $\mu_1, \mu_2, \mu_3, \mu_4$ such that $I(f) = \int f \df \mu$ where $\mu = \mu_1 - \mu_2 + i\mu_3 - i\mu_4$.\retTwo
\end{myIndent}

\dispDate{7/6/2025}

I'm continuing on in Folland where I left off.

\begin{myIndent}\hTwo
	A \udefine{signed Radon measure} is a signed Borel measure such that it's positive and\\ negative variations are Radon.\retTwo

	A \udefine{complex Radon measure} is a complex Borel measure such that it's real and\\ imaginary variations are signed Radon measures.
	\begin{myIndent}
		\pracTwo Side note: Complex Borel measures are always finite on compact sets. Thus if $X$ is an LCH space in which every open set is $\sigma$-compact, we know by theorem 7.8 that all complex Borel measures are Radon. In particular, if $X$ is a second countable LCH space, then all complex Borel measures are Radon.\retTwo
	\end{myIndent}

	We denote the space of complex Radon measures on $(X, \mathcal{B}_X)$ as $M(X)$ and for $\mu \in M(X)$ we define $\|\mu\| = |\mu|(X)$ where $|\mu|$ is the total variation of $\mu$.\retTwo

	\exOne \hypertarget{Folland prop 7.16}{\ul{Proposition 7.16:}} If $\mu$ is a complex Borel measure, then $\mu$ is Radon iff $|\mu|$ is Radon. Moreover, $M(X)$ is a vector space and $\mu \mapsto \|\mu\|$ is a norm on that space.

	\begin{myIndent}\exTwoP
		Proof:\\
		By proposition 7.5 (which says that Radon measures are inner regular on all their $\sigma$-finite sets), we know that a finite positive measure $|\mu|$ is Radon iff for any Borel set $E$ and $\varepsilon > 0$ there is an open set $U$ and a compact set $K$ with $K \subseteq E \subseteq U$ and $\mu(U - K) < \varepsilon$. From this we show the first assertion as follows. If $\mu = \mu_1 - \mu_2 + i\mu_3 - i\mu_4$ where all the $\mu_j$ are finite positive measures, and $|\mu|(U - K) < \varepsilon$, then $\mu_j(U - K) < \varepsilon$ for all $j$.
		
		\begin{myIndent}
			\why (Also I'm going into more detail cause I am having trouble remembering how to work with the total variation of a complex measure.) Let $\nu$ be some positive measure with $\mu \ll \nu$. Then $\mu_j \ll \nu$ for each $j$, so for each $j$ there are functions $f_j$ with $\df \mu_j = f_j \df \nu$. Also, $\df \mu = (f_1 - f_2 + i(f_3 - f_4))\df \nu$ and $\df |\mu| = |f_1 - f_2 + i(f_3 - f_4)|\df \nu$.\newpage

			Now since all the $f_j$ are real-valued, we have:
			
			{\centering $|f_1 - f_2 + i(f_3 - f_4)| \geq \max(|f_1 - f_2|, |f_3 - f_4|)$.\retTwo\par}

			Next, since $\mu_1 \perp \mu_2$ and $\mu_3 \perp \mu_4$ and all the measures are positive, we know that $\min(f_1, f_2) = 0$ and $\min(f_3, f_4) = 0$ $\nu$-a.e. Hence,

			{\centering $\max(|f_1 - f_2|, |f_3 - f_4|) \geq \max(f_1, f_2, f_3, f_4)$ a.e.\retTwo\par}

			And so, we get $|\mu|(E) \geq \max(\mu_1(E), \mu_2(E), \mu_3(E), \mu_4(E))$ for all\\ $E \in \mathcal{B}_X$.\retTwo
		\end{myIndent}
		
		Meanwhile if we can pick $U_j, K_j$ for all $j$ such that $\mu_j(U_j - K_j) < \sfrac{\varepsilon}{4}$, then set $U = \bigcap U_j$ and $K = \bigcup K_j$. Now, $|\mu|(U - K) < 4 \cdot \sfrac{\varepsilon}{4} = \varepsilon$.

		\begin{myIndent}
			\why By proposition 3.14,
			
			{\centering\begin{tabular}{l}
				$|\mu| = |\mu_1 - \mu_2 + i\mu_3 - i\mu_4| \leq |\mu_1| + |-\mu_2| + |i\mu_3| + |-i\mu_4|$\\
				$\phantom{|\mu| = |\mu_1 - \mu_2 + i\mu_3 - i\mu_4|} = \mu_1 + \mu_2 + \mu_3 + \mu_4$.
			\end{tabular}\retTwo\par}

			Then since $\mu_j(U) \leq \mu_j(U_j)$ and $\mu_j(K) \geq \mu(K_j)$ for all $j$, the claim holds.\retTwo
		\end{myIndent}

		Similar reasoning to that right above can show that $M(X)$ is closed under addition, and that $\|\mu_1 + \mu_2\| \leq \|\mu_1\| + \|\mu_2\|$. Also if $\df \mu = f \df \nu$ for some positive measure $\nu$, then $c \df \mu = c f \df \nu$. So $|c \df \mu| = |c| \df |\mu|$, and from that it is clear that $|\mu|$ being Radon implies $|c \df \mu|$ is Radon. So, $M(X)$ is closed under scalar multiplication. Note this also shows that $\|c\mu\| = |c|\|\mu\|$ for all $c \in \mathbb{C}$ and $\mu \in M(X)$.\retTwo

		Finally, suppose $\mu \in M(X)$ with $\mu \neq 0$. Then there is some set $E \in \mathcal{B}_X$ such that $\mu(E) \neq 0$. Next $0 < |\mu(E)| \leq |\mu|(E)$ (see proposition 3.13). And since $|\mu|$ is Radon, we know that:
		
		{\centering $0 < |\mu|(E) = \inf\{|\mu|(U) : E \subseteq U \text{ where } U \text{ is open}\} \leq |\mu|(X) = \|\mu\|$.\retTwo\par}

		This proves, $M(X)$ is a normed vector space when equipped with $\|\cdot\|$. \blacksquare\retTwo
	\end{myIndent}
\end{myIndent}

\dispDate{7/7/2025}

Before getting to the next theorem, I'd like to return to when I showed that for any $I \in C_0(X)^*$, there are finite Radon measures $\mu_1, \mu_2, \mu_3, \mu_4$ such that $I(f) = \int f \df \mu$ where $\mu = \mu_1 - \mu_2 + i\mu_3 - i\mu_4$.\retTwo

A thing that Folland neglected to show is that while it's clear that $\mu_1 - \mu_2$ and $\mu_3 - \mu_4$ are the real and imaginary variations of $\mu$ respectively, it's not necessarily clear that $\mu_1$ and $\mu_2$ are the positive and negative variations respectively of $(\mu_1 - \mu_2)$ and likewise for $\mu_3$ and $\mu_4$ with respect to $(\mu_3 - \mu_4)$. So, I want to show that today since this will be relevant to the next propositions that Folland covers.\retTwo

\begin{myIndent}\pracOne
	\ul{Lemma (Riesz Representation theorem on $C_0(X, \mathbb{R})^*$):} There is a one-to-one\\ correspondance between positive linear functionals in $C_0(X, \mathbb{R})^*$ and finite Radon measures on $(X, \mathcal{B}_X)$.

	\begin{myIndent}\pracTwo
		Proof:\\
		Recall that there is a bijection between $C_0(X, \mathbb{R})^*$ and $C_0(X)^*$. Namely given any $I \in C_0(X, \mathbb{R})^*$, define $J \in C_0(X)^*$ by setting $J(f) = I(\rea{f}) + iI(\ima{f})$ for all $f$. And to go the other way, just restrict the domain of $J$.\retTwo

		Now in that bijection, it's clear that $I$ is positive iff $J$ is positive. Also, it's clear that $I(f) = \int f \df \mu$ for all $f \in C_0(X, \mathbb{R})$ iff $J(f) = \int f \df \mu$ for all $f \in C_0(X)$. So, we can apply the Riesz Representation theorem we already proved to say that there is a bijective correspondence between finite Radon measures on $X$ and\\ $\{I \in C_0(X, \mathbb{R})^* : I \text{ is positive}\}$.\retTwo
	\end{myIndent}
	
	Now suppose $I \in C_0(X, \mathbb{R})^*$ and let $I = I^+ - I^-$ where $I^\pm \in C_0(X, \mathbb{R})^*$ are as we constructed in Lemma 7.15. As we just demonstrated, there are finite Radon measures $\mu_1$ and $\mu_2$ such that $I^+(f) = \int f \df \mu_1$ and $I^-(f) = \int f \df \mu_2$. In turn, setting $\mu = \mu_1 - \mu_2$ we have that $I(f) = \int f \df \mu$.\retTwo

	\pracOne\ul{Exercise 7.16:} The positive and negative variations of $\mu$ are the Radon measures $\mu_1$ and $\mu_2$ respectively.

	\begin{myIndent}\pracTwo
		Let $\mu^+$ and $\mu^-$ be the positive and negative variations of $\mu$, and let $E \in \mathcal{B}_X$ be a set such that $\mu^+(E) = 0$ and $\mu^-(E^\comp) = 0$.\retTwo
		
		Fixing $f \in C_0(X, [0, \infty))$, note that:

		{\centering\begin{tabular}{l}
			$I^+(f) = \sup\{I(g) : g \in C_0(X, \mathbb{R}), 0 \leq g \leq f\}$\\ [6pt]
			$\phantom{I^+(f)} = \sup\{\int g \df \mu^+ - \int g \df \mu^- : g \in C_0(X, \mathbb{R}), 0 \leq g \leq f\}$ \\ [6pt]
			$\phantom{I^+(f)} \leq \sup\{\int g \df \mu^+ : g \in C_0(X, \mathbb{R}), 0 \leq g \leq f\} = \int f \df \mu^+ $
		\end{tabular}\retTwo\par}

		On the other hand, since $\mu_1, \mu_2$ are finite Radon measures and I showed yesterday that $M(X)$ is a vector space, I know that $\mu = \mu_1 - \mu_2$ is also a finite Radon measure. Also from yesterday, I know that that is equivalent to saying that $|\mu| = \mu^+ + \mu^-$ is Radon. Plus, $\mu$ being finite implies $|\mu|$ is finite. Hence, $f$ vanishes outside of a set with finite measure (that set being all of $X$). So, for any $\varepsilon > 0$ we can apply Lusin's theorem to get a function $\phi \in C_c(X)$ with $\phi = f\chi_{E^\comp}$ except on a set $F \in \mathcal{B}_X$ with $|\mu|(F) < \varepsilon$.\retTwo
		
		If we then set $\psi = \min(\rea{\phi}^+, f)$, we still have that $\psi = f\chi_{E^\comp}$ except on $F$. But then also $\psi \in C_C(X, \mathbb{R}) \subseteq C_0(X, \mathbb{R})$ with $0 \leq \psi \leq f$. So:

		{\centering\begin{tabular}{l}
			$I^+(f) \geq \int \psi \df \mu = \int \psi \df \mu^+ - \int \psi \df \mu^-$\\ [6pt]
			$\phantom{I^+(f) \geq \int \psi \df \mu} = \int_F \psi \df \mu^+ + \int_{F^\comp} f\chi_{E^\comp}\df \mu^+ - \int_F \psi \df \mu^- - \int_{F^\comp} f\chi_{E^\comp}\df \mu^-$\\ [6pt]
			$\phantom{I^+(f) \geq \int \psi \df \mu } \geq 0 + \int_{F^\comp} f\chi_{E^\comp}\df \mu^+ - \int_F \psi \df \mu^- - \int_{F^\comp} f\chi_{E^\comp}\df \mu^-$\\ [6pt]
			$\phantom{I^+(f) \geq \int \psi \df \mu} = \int_{F^\comp} f\df \mu^+ - \int_F \psi \df \mu^- - 0$\\ [6pt]
			$\phantom{I^+(f) \geq \int \psi \df \mu} = \int f \df \mu^+ - \int_{F} f\df \mu^+ - \int_F \psi \df \mu^- - 0$\\ [6pt]
			$\phantom{I^+(f) \geq \int \psi \df \mu} \geq \int f \df\mu^+ - 2\|f\|_u\mu(F) > \int f \df \mu^+ - 2\varepsilon\|f\|_u$\\ [6pt]
		\end{tabular}\newpage\par}

		Since $f$ was fixed, by taking $\varepsilon \to 0$ we have thus proven that $I^+(f) = \int f \df \mu^+$ for all $f \in C_0(X, [0, \infty))$. Then by considering positive and negative parts and making use of the linearity of both sides, we can easily see $I^+(f) = \int f \df \mu^+$ for all $f \in C_0(x, \mathbb{R})$. This proves that $\mu^+$ is the unique Radon measure associated with $I^+$. Hence, $\mu^+ = \mu_1$.\retTwo

		Also, since $I^-(f) = I^+(f) - I(f) = \int f \df \mu^+ - (\int f \df \mu^+ - \int f \df \mu^-)$, we have $I^-(f) = \int f \df \mu^-$ for all $f \in C_0(X, \mathbb{R})$. So $\mu^- = \mu_2$. $\blacksquare$\retTwo
	\end{myIndent}

	Now as seen in the first lemma I showed today, if we extend $I^\pm \in C_0(X, \mathbb{R})^*$ to be a linear functional in $C_0(X)^*$, it doesn't change the measure $\mu$ at all. So, I'm done.\retTwo
\end{myIndent}

\dispDate{7/8/2025}

Firstly, I'm going to finish describing $C_0(X)^*$.

\begin{myIndent}\exOne
	\ul{Proposition 7.17 (The Riesz Representation Theorem):} Let $X$ be an LCH space, and for $\mu \in M(X)$ and $f \in C_0(X)$, let $I_\mu(f) = \int f \df \mu$. Then the map $\mu \mapsto I_\mu$ is an isometric isomorphism from $M(X)$ to $C_0(X)^*$.

	\begin{myIndent}\exTwoP
		Proof:\\
		We already have shown that every $I \in C_0(X)^*$ is of the form $I_\mu$ for some\\ $\mu \in M(X)$. On the other hand, if $\mu \in M(X)$, then we already know that $I_\mu$ is a linear function. Also, by proposition 3.13:
		
		{\centering $|\int f \df \mu| \leq \int |f|\df |\mu| \leq \|f\|_u\|\mu\|$.\retTwo\par}
		
		So, $I_\mu$ is bounded with $\|I_\mu\| \leq \|\mu\|$.\retTwo

		All we have left to do is show $\|\mu\| \leq \|I_\mu\|$. So let $h = \frac{\df \mu}{\df |\mu|}$. Then since $|h| = 1$ by proposition 3.13 and $|\mu|$ is a finite Radon measure, we know\\ [2pt] by Lusin's theorem that for any $\varepsilon > 0$  there exists $f \in C_c(X)$ such that\\ [2pt] $\|f\|_u \leq \|h\|_u$ and $f = \overline{h}$ except on a set $E$ with $|\mu|(E) < \sfrac{\varepsilon}{2}$. (Note that\\ [2pt] since $|\overline{h}| = 1$ almost everywhere, we have that $\|f\|_u = 1$.)\retTwo
		
		Now:

		{\centering \begin{tabular}{l}
			$\|\mu\| = \int 1\df |\mu| = \int |h|^2 \df |\mu|$\\ [4pt]
			$\phantom{\|\mu\| = \int 1\df |\mu|} = \int \overline{h} \df \mu \leq |\int f \df \mu| + |\int (f - \overline{h})\df \mu|$\\ [4pt]
			$\phantom{\|\mu\| = \int 1\df |\mu| = \int \overline{h} \df \mu}\leq |I_\mu(f)| + \int |f - \overline{h}|\df |\mu| \leq \|I_\mu\| + 2|\mu|(E)$\\ [4pt]
			$\phantom{\|\mu\| = \int 1\df |\mu| = \int \overline{h} \df \mu \leq |I_\mu(f)| + \int |f - \overline{h}|\df |\mu|} < \|I_\mu\| + \varepsilon$\\ [4pt]
		\end{tabular}\retTwo\par}

		Thus $\|\mu\| \leq \|I_\mu\|$ and we are done. \blacksquare\retTwo
	\end{myIndent}

	\ul{Corollary 7.18:} If $X$ is a compact Hausdorff space, then $C(X)^*$ is isometrically isomorphic to $M(X)$.\newpage
\end{myIndent}

Next, I plan on taking a break from Folland chapter 7 in order to do some of the section 8.3 exercises in Folland that I never started or finished during the past Spring quarter.\retTwo

\Hstatement\blab{Exercise 8.14 (Wirtinger's Inequality)} If $f \in C^1([a, b])$ and $f(a) = f(b) = 0$, then:

$$\int_a^b |f(x)|^2\df x \leq \left(\frac{b-a}{\pi}\right)^2 \int_a^b |f^\prime(x)|^2\df x$$

Hint: By a change of variables it suffices to assume $a = 0$ and $b = \frac{1}{2}$. Extend $f$ To $[-\frac{1}{2}, \frac{1}{2}]$ by setting $f(-x) = -f(x)$, and then extend $f$ to be periodic on $\mathbb{R}$. Check that $f$, thus extended, is in $C^1(\mathbb{T})$ and apply the Parseval identity.\retTwo

\begin{myIndent}\HexOne
	Given our $f$, we can define $g(x) \coloneq f(a + 2x(b - a))$. Then $g \in C^1([0, \frac{1}{2}])$ with\\ $g(0) = g(\frac{1}{2}) = 0$ and $f(x) = g(\frac{x-a}{2(b-a)})$. Now suppose we prove the inequality for $g$.\\ I.e., we show $\int_0^{1/2} |g(x)|^2\df x \leq (\frac{1}{2\pi})^2 \int_0^{1/2} |g^\prime(x)|^2\df x$. Then:

	\begin{itemize}
		\item $\int_0^{1/2}|g(x)|^2 \df x = \int_0^{1/2}|f(a+2x(b-a))|^2 \df x = 2(b-a)\int_a^b |f(y)|^2\df y$,
		\item $(\frac{1}{2\pi})^2\int_0^{1/2}|g^\prime(x)|^2\df x = (\frac{1}{2\pi})^2\int_0^{1/2}|2(b-a)f^\prime(a + 2x(b-a))|^2\df x$\\
		$\phantom{(\frac{1}{2\pi})^2\int_0^{1/2}|g^\prime(x)|^2\df x} = (\frac{1}{2\pi})^2 (2(b-a))^3\int_a^{b}|f^\prime(y)|^2\df y = 2(b-a)(\frac{b-a}{\pi})^2\int_a^{b}|f^\prime(y)|^2\df y$.\retTwo
	\end{itemize}

	By canceling out the $2(b-a)$ term (which is positive since $b > a$), we see the result still holds for $f$ if it held for $g$.\retTwo

	But now we need to actually prove the result for $g$. To do this, extend out $g$ to all of $\mathbb{R}$ by first setting $g(-x) \coloneq -g(x)$ for $x \in [0, \frac{1}{2}]$, and then extending $g$ to be periodic on $\mathbb{R}$. Note that this is well defined specifically because $g(0) = g(\frac{1}{2}) = 0$.\retTwo
	
	To see that $g$ is in $C^1(\mathbb{T})$, note that since $g(x) = -g(-x)$ for $x \in (-\frac{1}{2}, 0)$ we have that $g^\prime(x) = g^\prime(-x)$ on $(-\frac{1}{2}, 0)$. Thus $g$ is continuously differentiable on $(-\frac{1}{2}, 0)$ since we already know $g$ is continuously differentiable on $(0, \frac{1}{2})$.\retTwo
	
	As for at $x = 0$, note that:
	
	{\centering $\lim\limits_{h \to 0^-}\frac{g(h) - g(0)}{h} = \lim\limits_{h \to 0^-}\frac{g(h)}{h} = \lim\limits_{h \to 0^-}\frac{-g(-h)}{h} = \lim\limits_{h \to 0^+}\frac{-g(h)}{-h} = \lim\limits_{h \to 0^+}\frac{g(h)}{h} = \lim\limits_{h \to 0^+}\frac{g(h) - g(0)}{h}$\retTwo\par}

	Thus $g^\prime(0)$ still exists on the extended domain. Also, since $\lim_{x \to 0^+}g^\prime(x) = g^\prime(0)$, we know that $\lim_{x \to 0^-}g^\prime(x) = \lim_{x \to 0^-}g^\prime(-x) = \lim_{x \to 0^+}g^\prime(x) = g^\prime(0)$. So, $g^\prime$ is continuous at $t = 0$. Similar reasoning also works at $x = \frac{1}{2}$, although the looping structure of $\mathbb{T}$ makes the expressions slightly messier.\retTwo

	Now since $\mathbb{T}$ is compact, we know that $C(\mathbb{T}) \subseteq L^p$ for all $p$ (and in particular, for $p = 2$). Thus both $g$ and $g^\prime$ are in $L^2$. Applying Parseval's identity to $g$ we get that:

	{\centering $\int_{-1/2}^{1/2} |g(x)|^2dx = \|g\|_{L^2(\mathbb{T})}^2 = \sum\limits_{k \in \mathbb{Z}}|\widehat{g}(k)|^2 = \sum\limits_{k \in \mathbb{Z}}|\int_{-1/2}^{1/2} g(x)e^{-2\pi i kx}\df x|^2$ \retTwo\par}

	If we do integration by parts, then since $g(\frac{1}{2}) = g(-\frac{1}{2}) = 0$, we get for all $k \neq 0$ that:

	{\centering $\widehat{g}(k) = \int_{-1/2}^{1/2} g(x)e^{-2\pi i kx}\df x = \frac{1}{2\pi i k}\int_{-1/2}^{1/2} g^\prime(x)e^{-2\pi i kx}\df x = \frac{1}{2\pi i k}\widehat{g^\prime}(k)$ \newpage\par}

	Meanwhile, because of the way we extended $g$, we know $g$ is an odd function. Thus,\\ $\widehat{g}(0) = \int_{-1/2}^{1/2} g(x)\df x = 0$ and we've thus shown that:
	
	{\centering$\int_{-1/2}^{1/2} |g(x)|^2dx \leq \sum\limits_{\begin{smallmatrix}
	k \in \mathbb{Z}\\ k \neq 0
\end{smallmatrix}} \left|\frac{1}{2\pi i k}\widehat{g^\prime}(k)\right|^2 = \frac{1}{4\pi^2}\sum\limits_{\begin{smallmatrix}
	k \in \mathbb{Z}\\ k \neq 0
\end{smallmatrix}} \frac{1}{k^2}\left|\widehat{g^\prime}(k)\right|^2$\retTwo\par}

	Now $\sum\limits_{\begin{smallmatrix}
	k \in \mathbb{Z}\\ k \neq 0
	\end{smallmatrix}} \frac{1}{k^2}\left|\widehat{g^\prime}(k)\right|^2 \leq \sum\limits_{k \in \mathbb{Z}} 1\cdot \left|\widehat{g^\prime}(k)\right|^2$ and the latter is equal to $\|g^\prime\|_{L^2(\mathbb{T})}^2 = \int_{-1/2}^{1/2} |g^\prime(x)|^2 \df x$\\ [-12pt]\phantom{aaaaaaaaaaaaaaaaaaaaaaaaaaaaaaaaaaaaaaaaaaaaaaaaaaaaaaa} by Parseval's identity.\retTwo

	Hence, we've proven that $\int_{-1/2}^{1/2} |g(x)|^2dx \leq \left(\frac{1}{2\pi}\right)^2\int_{-1/2}^{1/2} |g^\prime(x)|^2 \df x$.\retTwo

	Finally, since $|g(x)|^2$ and $|g^\prime(x)|^2$ are both even on account of $g$ being odd, we know that $\int_{-1/2}^{1/2} |g(x)|^2dx = 2\int_{0}^{1/2} |g(x)|^2dx$ and $\int_{-1/2}^{1/2} |g^\prime(x)|^2 \df x = 2\int_{0}^{1/2} |g^\prime(x)|^2 \df$. After\\ canceling out the factor of $2$, we've thus proven our desired inequality. $\blacksquare$\retTwo
\end{myIndent}

\blab{Exercise 8.16:} Let $f_k = \chi_{[-1, 1]} * \chi_{[-k, k]}$. (Also assume $k \in \mathbb{N}$ with $k > 0$).
\begin{enumerate}
	\item[(a)] Compute $f_k(x)$ explicitely and show that $\|f\|_u = 2$.
	\begin{myIndent}\HexOne
		You can fairly easily see that for any $x \in \mathbb{R}$, $f_k(x) = \int_{-k}^k \chi_{[-1, 1]}(x - y)\df y$. Evaluating that gives the formula:

		{\centering $f_k(x) = \left\{\begin{matrix}
			2 & \text{if } |x| \leq k - 1 \phantom{<--k + 1} \\
			k - x + 1 & \text{if } k - 1 \leq x \leq k + 1\phantom{aaw.}\\
			x + 1 + k & \text{if } -k - 1 \leq x \leq -k + 1\\
			0 &  \text{if } |x| \geq k + 1\phantom{<--k + 1} \\
		\end{matrix}\right.$ \retTwo\par}

		From that it is hopefully clear that $\|f\|_u = 2$. After all, $f_k(0) = 2$. Also,\\ $f_k(x) = \int_{-k}^k \chi_{[-1, 1]}(x - y)\df y \leq \int \chi_{[-1, 1]}(x - y)\df y = 2$.\retTwo
	\end{myIndent}

	\item[(b)] Show $f_k^\vee(x) = (\pi x)^{-2}\sin(2\pi x)\sin(2 \pi k x)$, and $\|f_K^\vee\|_1 \to \infty$ as $k \to \infty$.
	\begin{myIndent}\HexOne
		Recall from the homework that $\chi_{[-a , a]}^\wedge = \chi_{[-a , a]}^\vee = 2a\frac{\sin(2a\pi x)}{2\pi a x} = \frac{\sin(2\pi a x)}{\pi x}$.\\ [-2pt]
		
		Also, for any $f, g \in L^1$, by identical reasoning as we used to show $\widehat{f * g} = \widehat{f}\widehat{g}$, we know that $(f * g)^\vee = f^\vee g^\vee$. Therefore:\\ [-14pt]

		{\centering \begin{tabular}{l}
	$f_k^\vee(x) = \chi_{[-1, 1]}^\vee(x) \chi_{[-k, k]}^\vee(x) = \left(\frac{\sin(2\pi x)}{\pi x}\right)\left(\frac{\sin(2\pi k x)}{\pi x}\right)$\\ [6pt]
	$\phantom{f_k^\vee(x) = \chi_{[-1, 1]}^\vee(x) \chi_{[-k, k]}^\vee(x)} = (\pi x)^{-2}\sin(2\pi x)\sin(2 \pi k x)$.
\end{tabular} \retTwo\par}

		Next, let $y = 2\pi k x$. Then:
		
		{\centering \begin{tabular}{l}
			$\int |f_k^\vee(x)|\df x = \int_{-\infty}^\infty |(\pi x)^{-2}\sin(2\pi x)\sin(2\pi k x)|\df x$\\ [6pt]
			$\phantom{\int |f_k^\vee(x)|\df x} = \frac{1}{2\pi k}\int_{-\infty}^\infty |\frac{4k^2}{y^2}\sin(\frac{y}{k})\sin(y)|\df y = \frac{2}{\pi}\int_{-\infty}^\infty |\frac{\sin(y/k)}{y/k} \cdot \frac{\sin(y)}{y}|\df y$\\ [6pt]
			$\phantom{\int |f_k^\vee(x)|\df x = \frac{1}{2\pi k}\int_{-\infty}^\infty |\frac{4k^2}{y^2}\sin(\frac{y}{k})\sin(y)|\df y} = \frac{4}{\pi}\int_{0}^\infty |\frac{\sin(y/k)}{y/k} \cdot \frac{\sin(y)}{y}|\df y$.
		\end{tabular}\retTwo\par}

		(the last equality holds because the integrand is even)\newpage

		Now, because $\frac{\sin(x)}{x} \to 1$ as $x \to 0$, we know that $|\frac{\sin(y/k)}{y/k} \cdot \frac{\sin(y)}{y}|$ converges pointwise\\ [-3pt] to $|\frac{\sin(y)}{y}|$ as $k \to \infty$. Also, observe that $|\frac{\sin(x)}{x}| \leq 1$ for all $x > 0$.

		\begin{myIndent}\exTwoP
			Proof:\\
			Let $g(x) = \frac{\sin(x)}{x}$. Then clearly $|g(x)| \leq \frac{1}{x} \leq 1$ when $x \geq 1$.\retTwo
			
			Meanwhile, if $x < 1$, note that $g^\prime(x) = \frac{x\cos(x) - \sin(x)}{x^2}$. Since $x^2 > 0$, it suffices to show that the numerator: $h(x) = x\cos(x) - \sin(x)$, is negative when $x < 1$ in order to prove that $g^\prime(x)$ is not positive when $x < 1$, Luckily, note that $h(0) = 0$ and $h^\prime(x) = -x\sin(x)$. Since $\sin(x) \geq 0$ for $x \leq \pi \approx 3.14$, we thus know that $h^\prime(x) \leq 0$ for all $x \in [0, 1]$. In turn, we know that $h(x) \leq h(0) = 0$ for all $x \in [0, 1]$. So, we've proven that $g^\prime(x)$ is not positive on $(0, 1]$.\retTwo

			This proves that $g(x)$ is monotonically decreasing on $(0, 1)$. And since $\lim_{x \to 0} g(x) = 1$, this proves that $g(x) \leq 1$ for all $x \in (0, 1]$. Also, since $\sin(x) > 0$ when $0 < x < \pi \approx 3.14$, we know that $g(x) > 0$ for all\\ $x \in (0, 1]$. So, $|g(x)| \leq 1$ for all $x > 0$.\retTwo
		\end{myIndent}

		If we fix a constant $b > 0$, we have that: $|\frac{\sin(y/k)}{y/k} \cdot \frac{\sin(y)}{y}\cdot \chi_{[0, b]}(y)| \leq 1 \cdot 1 \cdot \chi_{[0, b]}(y)$ for all $k \in \mathbb{N}$. Hence by the dominated convergence theorem:

		{\centering $\liminf\limits_{k \to \infty}\frac{4}{\pi}\int_0^\infty |\frac{\sin(y/k)}{y/k} \cdot \frac{\sin(y)}{y}| \df y \geq \lim\limits_{k \to \infty}\frac{4}{\pi}\int_0^b |\frac{\sin(y/k)}{y/k} \cdot \frac{\sin(y)}{y}| \df y = \frac{4}{\pi}\int_0^b |\frac{\sin(y)}{y}| \df y$  \retTwo\par}
		
		But now note that $\int_0^\infty |\frac{\sin(y)}{y}| \df y = \infty$.\\ [-20pt]
		\begin{myIndent}\HexTwoP
			{\centering \begin{tabular}{l}
				$\int_0^\infty |\frac{\sin(y)}{y}| \df y \geq \sum\limits_{n = 0}^\infty\int_{n\pi +\frac{\pi}{6}}^{n\pi + \frac{5\pi}{6}}|\frac{\sin(y)}{y}| \df y = \sum\limits_{n = 0}^\infty\int_{n\pi +\frac{\pi}{6}}^{n\pi + \frac{5\pi}{6}} \frac{1}{2y} \df y$\\ [14pt]

				$\phantom{\int_0^\infty |\frac{\sin(y)}{y}| \df y \geq \sum\limits_{n = 0}^\infty\int_{n\pi +\frac{\pi}{6}}^{n\pi + \frac{5\pi}{6}}|\frac{\sin(y)}{y}| \df y} \geq \frac{1}{2}\sum\limits_{n=0}^\infty \int_{n\pi +\frac{\pi}{6}}^{n\pi + \frac{5\pi}{6}}\frac{1}{n\pi + \frac{5\pi}{6}}\df y$\\ [14pt]

				$\phantom{\int_0^\infty |\frac{\sin(y)}{y}| \df y \geq \sum\limits_{n = 0}^\infty\int_{n\pi +\frac{\pi}{6}}^{n\pi + \frac{5\pi}{6}}|\frac{\sin(y)}{y}| \df y} = \frac{\pi}{3}\sum\limits_{n=0}^\infty \frac{1}{n\pi + \frac{5\pi}{6}} \geq \frac{\pi}{3}\sum\limits_{n=1}^\infty \frac{1}{n\pi} = \frac{1}{3}\sum\limits_{n=1}^\infty \frac{1}{n} = \infty$\\ [14pt]
			\end{tabular}\retTwo\par}
		\end{myIndent}

		Thus, we can make $\int_0^b |\frac{\sin(y)}{y}|\df y$ arbitrarily big by making $b$ big enough. Hence, we've\\ [-2pt] proven that $\lim_{k \to \infty} \|f_k^\vee\|_1 = \lim_{k \to \infty} \frac{4}{\pi}\int_0^b |\frac{\sin(y)}{y}| \df y = \infty$.

		\begin{myIndent}\pracTwo\fontsize{11}{13}
			Side note: while $\int_0^\infty \frac{\sin(y)}{y} \df y$ is not defined as a Lebesgue integral, it is defined as an improper Riemann integral and we can calculate that integral as follows.\retTwo

			Let $s > 0$. Then note that $\frac{\sin(y)}{y}$ and $e^{-sy}\chi_{[0, \infty)}$ are both in $L^2$. After all,\\ $|\frac{\sin(y)}{y}|^2 \leq \chi_{[-1, 1]}(y) + \frac{1}{y^2}\chi_{[-1, 1]^\comp}(y)$ and the right side is in $L^2$. Meanwhile,\\ [2pt] $\|e^{-sy}\|_2 = \frac{1}{2s}$. Thus by the Plancharel theorem, we know:
			
			{\centering $\int_0^\infty \frac{\sin(y)}{y}e^{-sy}\df y = \int_{-\infty}^\infty \mathcal{F}(\frac{\sin(y)}{y})\overline{\mathcal{F}(e^{-sy}\chi_{[0, \infty)}(y))} \df y$\newpage\par}

			Now since $\chi_{[-a , a]}^\vee = \frac{\sin(2\pi a x)}{\pi x}$ for any $a \geq 0$, we can see that: $\mathcal{F}(\frac{\sin(y)}{y}) = \pi \chi_{[-\frac{1}{2\pi}, \frac{1}{2\pi}]}(\xi)$. Meanwhile:
			
			{\centering $\mathcal{F}(e^{-sy}\chi_{[0, \infty)}) = \int_0^\infty e^{-(s + 2\pi i \xi )y }\df y = \frac{-1}{s + 2\pi i \xi}\left(0 - 1\right) = \frac{1}{s + 2\pi i \xi}$\retTwo\par}

			Hence, we've shown that:

			{\centering $\int_0^\infty \frac{\sin(y)}{y}e^{-sy}\df y = \pi \int_{-\frac{1}{2\pi}}^{\frac{1}{2\pi}} \overline{\left(\frac{1}{s + 2\pi i \xi}\right)} \df \xi = \pi \int_{-\frac{1}{2\pi}}^{\frac{1}{2\pi}} \frac{s + 2\pi i \xi}{s^2 + 4\pi^2 \xi^2} \df \xi = \pi \int_{-\frac{1}{2\pi}}^{\frac{1}{2\pi}} \frac{s}{s^2 + 4\pi^2 \xi^2} \df \xi$ \retTwo\par}

			(Note, the last equality follows because we know that the imaginary part of the integral has to cancel since $\int_0^\infty \frac{\sin(y)}{y}e^{-sy}\df y$ is purely real-valued.)\retTwo

			Now:
			
			{\centering \begin{tabular}{l}
				$\pi \int_{-\frac{1}{2\pi}}^{\frac{1}{2\pi}} \frac{s}{s^2 + 4\pi^2 \xi^2} \df \xi = \frac{s\pi}{4\pi^2}\int_{-\frac{1}{2\pi}}^{\frac{1}{2\pi}} \frac{1}{(\frac{s}{2\pi})^2 + \xi^2} \df \xi$\\ [8pt]

				$\phantom{\pi \int_{-\frac{1}{2\pi}}^{\frac{1}{2\pi}} \frac{s}{s^2 + 4\pi^2 \xi^2} \df \xi} = \frac{s}{4\pi}\left(\frac{2\pi}{s}\right)\left[\arctan(\frac{2\pi}{s}\xi)\right]^{\xi = \frac{1}{2\pi}}_{\xi = -\frac{1}{2\pi}} = \frac{1}{2}\left(\arctan(\frac{1}{s}) - \arctan(-\frac{1}{s})\right) = \arctan(\frac{1}{s})$
			\end{tabular}\retTwo\par}

			Thus, we've proven that $\int_0^\infty \frac{\sin(y)}{y}e^{-sy}\df y = \arctan(\frac{1}{s})$ for all $s > 0$.\retTwo
			
			Taking the limit as $s \to 0$, we get that $\int_0^\infty \frac{\sin(y)}{y}e^{-sy}\df y \to \frac{\pi}{2}$. That said, some care is needed since $\int_0^\infty \frac{\sin(y)}{y} \df y$ and $\lim_{s \to 0} \int_0^\infty\frac{\sin(y)}{y}e^{-sy}\df y$ are defined differently. In fact, we still have not showed that the former which is equal to $\lim_{b \to \infty}\int_0^b \frac{\sin(y)}{y} \df y$\\ [2pt] exists. So, let's do that now.\retTwo

			Note that for any $b \in (0, \infty)$, there are unique $n \in \mathbb{Z}_{\geq 0}$ and $\alpha \in [0, \pi)$ such that $b = n\pi + \alpha$. Then for all $s \geq 0$, we have that: 
			
			{\centering $\int_0^{b} \frac{\sin(y)}{y}e^{-sy}\df y = \sum\limits_{j=0}^{n-1} (-1)^j\int_{j\pi}^{(j+1)\pi} |\frac{\sin(y)}{y} e^{-sy}|\df y + \int_{n\pi}^{n\pi + \alpha} |\frac{\sin(y)}{y} e^{-sy}|\df y$ \retTwo\par}

			Now, the leftover term will approach $0$ as $b \to \infty$ since it is at most $\frac{\alpha}{n\pi}$ when $b \geq 1$ and $n \to \infty$ as $b \to \infty$. Hence, letting $c_n = \int_{j\pi}^{(j+1)\pi} |\frac{\sin(y)}{y} e^{-sy}|\df y$, we know that:\\ [-2pt] $\lim_{b \to \infty} \int_0^b \frac{\sin(y)}{y} e^{-sy} \df y = \sum_{n=0}^\infty (-1)^n c_n$. It's easily verified using the alternating series test that the series converges. This proves that our improper Riemann integral exists for all $s \geq 0$ (including $s = 0$).\retTwo

			Importantly, this series also converges uniformly over all $s \in [0, \infty)$. To see why, observe that since $(c_n)_{n \in \mathbb{N}}$ is a strictly decreasing sequence, for any $N \geq 0$ we have that: $c_N \geq \left|\sum_{n = N}^\infty (-1)^n a_n \right|$. This can be proven via induction fairly easily. Next, making $s$ larger makes all the $c_n$ strictly smaller. So, by picking $N$ large enough so that $c_N < \varepsilon$ when $s = 0$, we can guarentee that $c_N < \varepsilon$ for all $s$. It then follows that the error from the limit point: $\left|\sum_{n = N}^\infty (-1)^n c_n\right|$, is also less than $\varepsilon$ for all $s$.\retTwo
			
			With that, we know there is some $b > 0$ such that:
			
			{\centering $\left|\int_0^\infty \frac{\sin(y)}{y} e^{-sy} \df y - \int_0^b \frac{\sin(y)}{y} e^{-sy} \df y\right| < \sfrac{\varepsilon}{4}$ for all $s \geq 0$.\newpage\par}

			Also, by dominated convergence theorem (its 4am and I don't want to type out verifications for all the conditions), we know that $\int_0^b \frac{\sin(y)}{y} e^{-sy} \df y \to \int_0^b \frac{\sin(y)}{y} \df y$ as $s \to 0$. So, there is some $s > 0$ such that:

			{\centering $\left|\int_0^b \frac{\sin(y)}{y} \df y - \int_0^b \frac{\sin(y)}{y} e^{-sy} \df y\right| < \sfrac{\varepsilon}{4}$.\retTwo\par}

			Also, by making $s$ potentially smaller, we can also guarentee that:

			{\centering $\left|\frac{\pi}{2} - \int_0^\infty \frac{\sin(y)}{y} e^{-sy} \df y\right| < \sfrac{\varepsilon}{4}$.\retTwo\par}

			And chaining those together, we get that:

			{\centering\fontsize{10}{12}\selectfont \begin{tabular}{l}
				$\left|\int_0^\infty \frac{\sin(y)}{y} \df y - \frac{\pi}{2}\right| \leq \left|\int_0^\infty \frac{\sin(y)}{y} \df y - \int_0^b \frac{\sin(y)}{y} \df y\right| + \left|\int_0^b \frac{\sin(y)}{y} \df y - \int_0^b \frac{\sin(y)}{y}e^{-sy} \df y\right|$\\
				$\phantom{\left|\int_0^\infty \frac{\sin(y)}{y} \df y - \frac{\pi}{2}\right| aaaa} + \left|\int_0^b \frac{\sin(y)}{y}e^{-sy} \df y - \int_0^\infty \frac{\sin(y)}{y}e^{-sy} \df y\right| + \left|\int_0^\infty \frac{\sin(y)}{y}e^{-sy} - \frac{\pi}{2}\right|$\\ [8pt]
				$\phantom{\left|\int_0^\infty \frac{\sin(y)}{y} \df y - \frac{\pi}{2}\right|} < \sfrac{\varepsilon}{4} + \sfrac{\varepsilon}{4}+ \sfrac{\varepsilon}{4}+ \sfrac{\varepsilon}{4} = \varepsilon$
			\end{tabular} \retTwo\par}

			Taking $\varepsilon \to 0$, we've finally shown that $\lim\limits_{b \to \infty}\int_0^b \frac{\sin(y)}{y} \df y = \frac{\pi}{2}$.
		\end{myIndent}
	\end{myIndent}

	\item[(c)] Prove that $\mathcal{F}(L^1)$ is a proper subset of $C_0$.
	
	\begin{myIndent}\HexOne
		To start off with, recall that if $f \in L^1$ and $\widehat{f} = 0$, then $f = 0$ a.e. As a corollary to this, we have that if $f, g \in L^1$ and $\widehat{f} = \widehat{g}$, then $f = g$ a.e. This is because $(f - g)^\wedge = 0$\\ [2pt] implies that $f - g = 0$ a.e. So, we know that $\mathcal{F}$ is an injective map from $L^1$ to $C_0$. If\\ [2pt] $\mathcal{F}$ was also surjective, then we would know that $\mathcal{F}$ is a bijection, and that therefore a\\ [2pt] function $\mathcal{F}^{-1}: C_0 \to L^1$ exists. Also, by the open map theorem, we would know that\\ [2pt] $\mathcal{F}^{-1}$ is bounded.\retTwo

		However, in part (b) we found that $\|f_k\|_u = 2$ for all $k \in \mathbb{N}$ but $\|f^\vee_k\|_1 \to \infty$ as\\ [2pt] $k \to \infty$. Importantly, we can see from our work earlier that $f^\wedge_k = f^\vee_k$ and\\ $\|f^\vee_k\|_1 < \infty$ for all $k$. After all, $f^\vee_k(y)$ is bounded by $1$ when $|y| \leq 1$ and by $\frac{k}{y^2}$\\ when $|y| \geq 1$. So by the Fourier inversion theorem, we know that $(f_k^\vee)^\wedge = f_k$ (with\\ [2pt] equality holding everywhere since both sides are continuous). And so, $\mathcal{F}^{-1}(f_k) = f_k^\vee$.\retTwo

		This proves that $\mathcal{F}^{-1}$ is not bounded since $\|\mathcal{F}^{-1}(f_k)\|_1$ can be made arbitrarily large even while $\|f_k\|_u = 2$ for all $k$.\retTwo
	\end{myIndent}
\end{enumerate}

\hOne

\dispDate{7/10/2025}

Today I'm gonna do more problems from chapter 8 of Folland.\retTwo

\Hstatement
Recall that for $f \in L^p(\mathbb{R})$, if there exists $h \in L^p(\mathbb{R})$ such that $\lim_{y \to 0}\|y^{-1}(\tau_{-y}f - f) - h\|_p = 0$, we call $h$ the \udefine{(strong) $L^p$ derivative} of $f$. If $f \in L^p(\mathbb{R}^n)$, $L^p$ partial derivatives of $f$ are defined similarly. (Also, the notation $\tau_{y}f(x)$ refers to $f(x - y)$.)\retTwo

\blab{Exercise 8.8:} Suppose that $p$ and $q$ are conjugate exponents, $f \in L^p$, $g \in L^q$, and the $L^p$\\ derivative $\partial_j f$ exists. Then $\partial_j(f * g)$ exists (in the ordinary sense) and equals $(\partial_j f) * g$.

\begin{myIndent}\HexOne 
	Note that:
	
	{\centering\begin{tabular}{l}
		$\lim\limits_{t \to 0}\frac{(f * g)(x + te_j) - (f * g)(x)}{t} = ((\partial_j f) * g)(x)$ iff $\lim\limits_{t \to 0} \left|\frac{(f * g)(x + te_j) - (f * g)(x)}{t} - ((\partial_j f) * g)(x)\right| = 0$.
	\end{tabular}\newpage\par}

	Now for all $t \neq 0$:

	{\centering \begin{tabular}{l}
		$0 \leq \left|\frac{(f * g)(x + te_j) - (f * g)(x)}{t} - ((\partial_j f) * g)(x)\right|$\\ [10pt]
		$\phantom{0} = \left|t^{-1}\int \left(f(x + te_j - y) - f(x - y)\right)g(y)\df y - \int \partial_j f(x - y)g(y)\df y \right|$\\ [10pt]
		$\phantom{0} = \left|\int t^{-1}(\tau_{-te_j}f - f)(x- y)g(y)\df y - \int \partial_j f(x - y)g(y)\df y \right|$\\ [10pt]
		$\phantom{0} \leq \int |t^{-1}(\tau_{-te_j}f - f)(x- y) - \partial_j f(x - y)||g(y)|\df y$\\ [10pt]
		$\phantom{0} \leq \|t^{-1}(\tau_{-te_j}f - f) - \partial_j f\|_p\|g\|_q$\\ [10pt]
	\end{tabular}\retTwo\par}

	Since $\|g\|_q$ is fixed and $\|t^{-1}(\tau_{-te_j}f - f) - \partial_j f\|_p \to 0$ as $t \to 0$, we've thus shown that $\left|\frac{(f * g)(x + te_j) - (f * g)(x)}{t} - ((\partial_j f) * g)(x)\right| \to 0$ as $t \to 0$.\retTwo
\end{myIndent}

\blab{Exercise 8.9:} Let $1 \leq p < \infty$. If $f \in L^p(\mathbb{R})$, the $L^p$ derivative of $f$ (call it $h$; see Exercise 8) exists iff $f$ is absolutely continuous on every bounded interval (perhaps after modification on a null set) and its pointwise derivative $f^\prime$ is in $L^p$, in which case $h = f^\prime$ a.e.

\begin{myIndent}\HexOne
	($\Longrightarrow$)\\
	Suppose $f$ has an $L^p$ derivative $h$. Then setting $\varphi(x) = (1 - |x|)\chi_{[-1, 1]}$, note that\\ $\int \varphi(x) \df x = 1$ and $0 \leq \varphi \leq 1 \leq \frac{4}{(1 + |x|)^2}$. Thus, $\varphi$ satisfies the hypothesis of theorem 8.15 (see page 31 of my paper notes) and so we know that:

	\begin{itemize}
		\item $(f \ast \varphi_{1/n})(x) = \int f(x - y) \cdot n\varphi(ny) \df y \to f(x)$ as $n \to \infty$ for all $x \in L_f$,
		\item $(h \ast \varphi_{1/n})(x) = \int h(x - y) \cdot n\varphi(ny) \df y \to h(x)$ as $n \to \infty$ for all $x \in L_h$
	\end{itemize}

	(where $L_f$ and $L_h$ are the Lebesgue sets of $f$ and $h$ respectively).
	\begin{myIndent}\pracTwo\fontsize{11}{13}\selectfont
		Side note: If $g \in L^1_{\mathsf{loc}}$, then we know that $(L_g)^\comp$ has measure zero. Now while it's obvious that $L^1, L^\infty \subseteq L^1_{\mathsf{loc}}$, I'm currently realizing I've never justified to myself why $L^p \subseteq L^1_{\mathsf{loc}}$ for all $1 < p < \infty$.\retTwo

		If $E$ is a measurable set with finite measure, then the measure restricted to $E$ does not have any sets of arbitrarily large measure. Thus for any $p, q \in (0, \infty)$ with $p < q$, we have that $L^q(E) \subseteq L^p(E)$. Specifically, this means that for any $1 < p < \infty$, $L^p(E) \subseteq L^1(E)$. It follows that $L^p \subseteq L^1_{\mathsf{loc}}$ for all $1 < p < \infty$.\retTwo
	\end{myIndent}

	Also, if $q$ is the conjugate exponent of $p$, we know that $\phi_{1/n} \in L^q$. Therefore, by the previous exercise we know that $(f \ast \phi_{1/n})^\prime = h \ast \varphi_{1/n}$. Additionally:

	{\centering \begin{tabular}{l}
		$|h \ast \varphi_{1/n}(x)| \leq \int |h(x - y)\varphi_{1/n}(y)| \df y \leq \|h\|_p \left(\int_{-1/n}^{1/n} |n(1 - |nx|)|^q \df x \right)^{1/q}$\\ [5pt]
		$\phantom{|h \ast \varphi_{1/n}(x)| \leq \int |h(x - y)\phi_{1/n}(y)| \df y} \leq n\|h\|_p \left(\int_{-1/n}^{1/n} 1 \df x \right)^{1/q} = n\left(\frac{2}{n}\right)^{\sfrac{1}{q}}\|h\|_p \leq 2^{\sfrac{1}{q}}\|h\|_p$
	\end{tabular} \retTwo\par}

	This tells us that for all $n \in \mathbb{N}$, $f \ast \varphi_{1/n}$ has a bounded derivative. It then follows by the mean value theorem that $f \ast \varphi_{1/n}$ is absolutely continuous. So for any $a, x \in \mathbb{R}$ with\\ $a < x$, we have that:

	{\centering $(f * \varphi_{1/n})(x) - (f * \varphi_{1/n})(a) = \int_a^x (f * \varphi_{1/n})^\prime(y) \df y = \int_a^x (h * \varphi_{1/n})(y) \df y$ \newpage\par}

	And, since $h * \varphi_{1/n} \to h$ pointwise a.e. and $2^{\sfrac{1}{q}} \|h\|_p \chi_{[a, x]} \in L^1$, we know by dominated convergence theorem that:
	
	{\centering $\lim\limits_{n \to \infty}\left((f * \varphi_{1/n})(x) - (f * \varphi_{1/n})(a)\right) = \lim\limits_{n \to \infty} \int_a^x (h * \varphi_{1/n})(y) \df y = \int_a^x h(y)\df y$ \retTwo\par}

	Now, we're finally ready to show the right hand side of our implication. Suppose $a, b \in L_f$ are fixed with $a < b$. Then for any $x \in L_f \cap [a, b]$, we have that:

	{\centering $f(x) - f(a) = \lim\limits_{n \to \infty}\left((f * \varphi_{1/n})(x)\right) - \lim\limits_{n \to \infty}\left((f * \varphi_{1/n})(a)\right) = \int_a^x h(y)\df y$ \retTwo\par}

	By redefining $f$ on the null space $(L_f)^\comp \cap [a, b]$, we can thus guarentee that\\ $f(x) - f(a) = \int_a^x h(y)\df y$ for all $x \in [a, b]$. In turn, by the fundamental theorem of calculus we know $f$ is absolutely continuous on $[a, b]$ and that $h = f^\prime$ a.e. on $[a, b]$.\retTwo

	If $I \subseteq \mathbb{R}$ is any arbitrary bounded interval, then we can still apply the former reasoning by finding $a, b \in L_f$ such that $I \subseteq [a, b]$. Then $f$ being absolutely continuous on $[a, b]$ implies that $f$ is absolutely continuous $I$. Also, since $\mathbb{R}$ can be completed covered by these intervals, we know that $f^\prime = h$ a.e. The only snag we still have to sort out is to show that our redefinitions of $f(x)$ for $x \in (L_f)^\comp$ are well defined (i.e. not dependent on our choice of $a, b \in L^f$.)

	\begin{myIndent}\HexTwoP
		Suppose $a_1, a_2 \in L_f$ and without loss of generality assume $a_1 < a_2 < x$. Then: 
		
		{\centering $\left(\int_{a_1}^x h(y)\df y + f(a_1)\right) - \left(\int_{a_2}^x h(y)\df y + f(a_2)\right) = \int_{a_1}^{a_2} h(y)\df y - \left(f(a_2) - f(a_1)\right)$.\retTwo\par}

		Since $a_1, a_2 \in L^f$, we know that $\int_{a_1}^{a_2} h(y)\df y = f(a_2) - f(a_1)$. So our above expression equals $0$ and we've shown that:

		{\centering $f(x) = \int_{a_1}^x h(y)\df y + f(a_1) = \int_{a_2}^x h(y)\df y + f(a_2)$ is well defined. \retTwo\par}
	\end{myIndent}

	($\Longleftarrow$)\\
	Note that if $y > 0$, then our assumptions about $f$ tell us that:

	{\centering \begin{tabular}{l}
		$\frac{f(x + y) - f(x)}{y} - f^\prime(x) = \frac{1}{y}\int_x^{x + y}f^\prime(t)\df t - f^\prime(x) = \frac{1}{y}\int_x^{x+y}f^\prime(t) - f^\prime(x)\df t$\\ [6pt]
		$\phantom{\frac{f(x + y) - f(x)}{y} - f^\prime(x) = \frac{1}{y}\int_x^{x + y}f^\prime(t)\df t - f^\prime(x)} = \frac{1}{y}\int_0^y f^\prime(x + t) - f^\prime(x)\df t$
	\end{tabular} \retTwo\par}

	Similarly, if $y < 0$, then we know:

	{\centering \begin{tabular}{l}
		$\frac{f(x + y) - f(x)}{y} - f^\prime(x) = \frac{-1}{y}\int_{x+y}^{x}f^\prime(t)\df t - f^\prime(x) = \frac{-1}{y}\int_{x+y}^xf^\prime(t) - f^\prime(x)\df t$\\ [6pt]
		$\phantom{\frac{f(x + y) - f(x)}{y} - f^\prime(x) = \frac{-1}{y}\int_{x+y}^xf^\prime(t)\df t - f^\prime(x)} = \frac{-1}{y}\int_y^0 f^\prime(x + t) - f^\prime(x)\df t$
	\end{tabular} \retTwo\par}

	In either case, we can see that:

	{\centering $\left| \frac{f(x + y) - f(x)}{y} - f^\prime(x)\right| \leq \int_{-|y|}^{|y|} \frac{1}{|y|}|\tau_{-t}f^\prime(x) - f^\prime(x)|\df t$ \retTwo\par}

	Thus by Minkowski's inequality for integrals:

	{\centering \begin{tabular}{l}
		$\left\| \frac{f(x + y) - f(x)}{y} - f^\prime(x) \right\|_p \leq \left\| \int_{-|y|}^{|y|} \frac{1}{|y|}|\tau_{-t}f^\prime(x) - f^\prime(x)|\df t \right\|_p \leq \frac{1}{|y|}\int_{-|y|}^{|y|} \|\tau_{-t} f^\prime(x) - f^\prime(x)\|_p \df t$
	\end{tabular}\retTwo\par}

	And since translation is continuous with respect to the $L^p$ norm for $1 \leq p < \infty$, we know that $\|\tau_{-t} f^\prime(x) - f^\prime(x)\|_p \to 0$ as $t \to 0$. Hence given $\varepsilon > 0$, we have for $|y|$ small enough that:

	{\centering $\frac{1}{|y|}\int_{-|y|}^{|y|} \|\tau_{-t} f^\prime(x) - f^\prime(x)\|_p \df t < \frac{1}{|y|} \int_{-|y|}^{|y|} \varepsilon \df t = \frac{2|y|\varepsilon}{|y|} = 2\varepsilon$ \newpage\par}

	By taking $\varepsilon \to 0$, this proves that $\left\| \frac{f(x + y) - f(x)}{y} - f^\prime(x) \right\|_p \to 0$ as $y \to 0$. Hence $f^\prime$ is an $L^p$\\ [-6pt] derivative of $f$. $\blacksquare$\retTwo

	\begin{myIndent}\pracTwo\fontsize{11}{13}\selectfont
		So what's the significance of this result?
		\begin{itemize}
			\item A function on $\mathbb{R}$ having an $L^p$ derivative is a strictly stronger assumption than the function just being differentiable almost everywhere.
			\item Any two $L^p$ derivatives of a function are equal a.e.\hspace{-0.1em} to the ordinary derivative of the\\ function. Thus there's at most one $L^p$ derivative of any function in $L^p(\mathbb{R})$.
			\item Any function $L^p(\mathbb{R})$ that is differentiable a.e. and whose derivative is bounded and also in $L^p$ has an $L^p$ derivative.\retTwo
		\end{itemize}
	\end{myIndent}
\end{myIndent}

\dispDate{7/11/2025}

\blab{Exercise 8.18:} Suppose $f \in L^2(\mathbb{R})$.
\begin{enumerate}
\item[(a)] The $L^2$ derivative $f^\prime$ exists iff $\xi \widehat{f} \in L^2$, in which case $\widehat{f^\prime}(\xi) = 2\pi i \xi \widehat{f}(\xi)$.

\begin{myIndent}\HexOne
	($\Longrightarrow$)\\
	Once again set $\varphi(x) = (1 - |x|)\chi_{[-1, 1]}$. Then by theorem 8.14(a) (see page 29 of my\\ [2pt] paper notes): $f * \varphi_{1/n} \to f$ in $L^2$ as $n \to \infty$. In turn, since the Fourier transform is continuous on $L^2$, we know that $(f * \varphi_{1/n})^\wedge \to \widehat{f}$ as $n \to \infty$.\retTwo

	Next, note that $f * \varphi_{1/n} \in C^1$.
	\begin{myIndent}\pracTwo\fontsize{11}{13}\selectfont
		Why: Recall from exercise 8.8 that $(f * \varphi_{1/n})^\prime = f^\prime * \varphi_{1/n}$. Also, since $f^\prime \in L_{\mathsf{loc}}^1$ and $\varphi_{1/n} \in C^0$ has compact support, we know from exercise 8.7 (which was a homework problem in Math 240C), that $f^\prime * \varphi_{1/n} \in C^0$.\retTwo
	\end{myIndent}

	Also, since $f$, $f^\prime$ and $\varphi_{1/n}$ are all in $L^2$, we know by proposition 8.8 (see page 25 of my paper notes) that $f * \varphi_{1/n} \in C_0$, and we know by Young's inequality (see page 26 of my paper notes) that $f * \varphi_{1/n}, f^\prime *\varphi_{1/n} \in L^1$. All together, this lets us conclude via integration by parts that:
	
	{\centering $(f * \varphi_{1/n})^\wedge = \frac{1}{2\pi i \xi}((f * \varphi_{1/n})^\prime)^\wedge = \frac{1}{2\pi i \xi}(f^\prime * \varphi_{1/n})^\wedge$.\retTwo\par}

	Finally, since $f^\prime * \varphi_{1/n} \to f^\prime$ in $L^2$ as $n \to \infty$ and the Fourier transform is continuous on $L^2$, we know that:

	{\centering $\widehat{f}(\xi) = \lim\limits_{n \to \infty}(f * \varphi_{1/n})^\wedge(\xi) = \frac{1}{2\pi i \xi} \lim\limits_{n \to \infty}(f^\prime * \varphi_{1/n})^\wedge(\xi) = \frac{1}{2\pi i \xi} \widehat{f^\prime}(\xi)$ a.e.\retTwo\par}

	Since $\frac{1}{2\pi i}\widehat{f^\prime}$ is in $L^2$, this thus proves that $\xi \widehat{f} \in L^2$. Also, by rearranging out expression we get that $\widehat{f^\prime} = 2\pi i \xi \widehat{f}(\xi)$.\retTwo

	($\Longleftarrow$)\\
	Define $h(\xi) =  2\pi i \xi \widehat{f}(\xi)$. Then by assumption we know that $h \in L^2$. So, there exists\\ [-2pt] a function $H \in L^2$ such that $\widehat{H} = h$. And since the Fourier transform is a continuous isometric linear operator on $L^2$, we know that for all $y \neq 0$:

	{\centering \begin{tabular}{l}
		$\|\frac{1}{y}(\tau_{-y}f - f) - H\|_2 = \|\mathcal{F}(\frac{1}{y}(\tau_{-y}f - f) - H)\|_2$\\ [9pt]
		$\phantom{\|\frac{1}{y}(\tau_{-y}f - f) - H\|_2} = \|\frac{1}{y}(\mathcal{F}(\tau_{-y}f) - \mathcal{F}(f)) - h\|_2$
	\end{tabular} \newpage\par}

	Now we claim that $\mathcal{F}(\tau_{-y}f) = e^{2\pi i \xi y}\mathcal{F}(f)$ for all $f \in L^2$.

	\begin{myIndent}\HexTwoP
		Proof:\\
		Let $(f_n)_{n \in \mathbb{N}}$ be a sequence of Schwartz functions converging to $f \in L^2$. Then since $\int g = \int \tau_{-y}g$ for all functions $g$, we can easily see that $\tau_{-y}f_n \to \tau_{-y}f$ in $L^2$. Therefore, $\mathcal{F}(\tau_{-y}f) = \lim\limits_{n \to \infty} \mathcal{F}(\tau_{-y}f_n)$.\retTwo

		Next, since $f_n \in L^1$ for all $n$, we know that $\mathcal{F}(\tau_{-y}f_n)(\xi) = e^{2\pi i \xi y}\widehat{f_n}(\xi)$. Then finally, since the Fourier transform is continuous on $L^2$, we have that $\widehat{f_n} \to \widehat{f}$ in $L^2$ as $n \to \infty$. By passing to a subsequence, we can assume $\widehat{f_n} \to \widehat{f}$ pointwise a.e. And so, $\mathcal{F}(\tau_{-y}f) = \lim\limits_{n \to \infty} e^{2\pi i \xi y}\widehat{f_n}(\xi) = e^{2\pi i \xi y}\widehat{f}(\xi)$ a.e.\retTwo
	\end{myIndent}

	Thus, we know that:

	{\centering \begin{tabular}{l}
		$\left|\frac{1}{y}(\mathcal{F}(\tau_{-y}f) - \mathcal{F}(f)) - h\right|^2 = \left|\left(\frac{1}{y}e^{2\pi i \xi y} - \frac{1}{y} - 2\pi i \xi\right)\widehat{f}(\xi)\right|^2$\\ [12pt]
		$\phantom{\left|\frac{1}{y}(\mathcal{F}(\tau_{-y}f) - \mathcal{F}(f)) - h\right|^2} = \left|\left((\frac{\cos(2\pi \xi y)}{y} - \frac{1}{y}) + i(\frac{\sin(2\pi \xi y)}{y} - 2\pi \xi)\right)\widehat{f}(\xi)\right|^2$\\ [12pt]
		$\phantom{\left|\frac{1}{y}(\mathcal{F}(\tau_{-y}f) - \mathcal{F}(f)) - h\right|^2} = \left|\left((\frac{\cos(2\pi \xi y) - 1}{\xi y}) + i(\frac{\sin(2\pi \xi y)}{y \xi} - 2\pi)\right)\xi\widehat{f}(\xi)\right|^2$\\ [12pt]
	\end{tabular} \retTwo\par}

	Now, note that $\lim_{y \to 0}\frac{\cos(2\pi \xi y) - 1}{\xi y} = 2\pi \lim_{t \to 0}\frac{\cos(t) - 1}{t} = 2\pi \cdot 0 = 0$ for all $\xi \neq 0$. Similarly, we have that $\lim_{y \to 0}\frac{\sin(2\pi \xi y)}{y \xi} = 2\pi \lim_{t \to 0} \frac{\sin(t)}{t} = 2\pi \cdot 1$ for all $\xi \neq 0$. This\\ [1pt] proves that $|\frac{1}{y}(\mathcal{F}(\tau_{-y}f) - \mathcal{F}(f)) - h|^2 \to 0$ pointwise a.e. as $y \to 0$.\retTwo

	Meanwhile, note that $\left|\frac{\sin(2\pi x)}{x}\right|$ and $\left|\frac{\cos(2 \pi x) - 1}{x}\right|$ are both less than or equal to $2 \pi$ on their domains. Therefore, we can get that for all $\xi \neq 0$ and $y \neq 0$, we have that:
	
	{\centering $|(\frac{\cos(2\pi \xi y) - 1}{\xi y}) + i(\frac{\sin(2\pi \xi y)}{y \xi} - 2\pi)| \leq |(\frac{\cos(2\pi \xi y) - 1}{\xi y})| + |(\frac{\sin(2\pi \xi y)}{y \xi} - 2\pi)| \leq 6\pi$\retTwo\par}

	\begin{myIndent}
		\HexPPP(I'm not sure how to prove $|\frac{\cos(2 \pi x) - 1}{x}| \leq 2\pi$ without pulling out numerical\\ methods. But you'll see that it is true if you graph it.)\retTwo
	\end{myIndent}

	Thus using $36\pi^2|\xi \widehat{f}(\xi)|^2$ as our upper bound function (which is in $L^1$ since $\xi \widehat{f} \in L^2$), we can conclude via the dominated convergence theorem that:

	{\centering $\lim\limits_{y \to 0}\|\frac{1}{y}(\mathcal{F}(\tau_{-y}f) - \mathcal{F}(f)) - h\|_2^2 = \lim\limits_{y \to 0} \int \left|\frac{1}{y}(\mathcal{F}(\tau_{-y}f) - \mathcal{F}(f)) - h\right|^2 = 0$ \retTwo\par}

	So, $f$ has $H = h^\vee$ as it's $L^2$ derivative.\retTwo
\end{myIndent}
\end{enumerate}

\dispDate{7/12/2025}

\hOne

Ok. I think that in order to prove part (b) of exercise 8.18, I need to make a pit stop in the exercises of section 3.5 of Folland. This is because Folland's hinted solution\newpage  route is to use integration by parts. However, right now I've only shown that you can do integration by parts if the two functions in your integrand are continuously\\ differentiable. Yet that's not guarenteeable in exercise 8.18(b). So, my current\\ objective is to weaken my requirements for doing integration by parts.\retTwo

\Hstatement\blab{Exercise 3.35:} If $F$ and $G$ are absolutely continuous on $[a, b]$, then so is $FG$ and:

{\centering $\int_a^b (FG^\prime + GF^\prime)(x)\df x = F(b)G(b) - F(a)G(a)$ \retTwo\par}

\begin{myIndent}\HexOne
	Proof:\\
	By extreme value theorem, there exists $M \geq 0$ such that $\max(|F(x)|, |G(x)|) \leq M$ for all $x \in [a, b]$. Now for any $\varepsilon > 0$, let $\delta > 0$ be such that for all finite collections of disjoint intervals $(a_1, b_1), \ldots, (a_n, b_n) \subseteq [a, b]$ with $\sum_{i=1}^n |b_i - a_i| < \delta$, we have:\\ [-9pt]

	{\centering $\sum_{i=1}^n |F(b_i) - F(a_i)| < \frac{\varepsilon}{2M}$ and $\sum_{i=1}^n |G(b_i) - G(a_i)| < \frac{\varepsilon}{2M}$\retTwo\par}

	Then we have:
	
	{\centering\fontsize{11}{12}\selectfont\begin{tabular}{l}
		$\sum\limits_{i=1}^n |F(b_i)G(b_i) - F(a_i)G(a_i)| \leq \sum\limits_{i=1}^n |F(b_i)G(b_i) - F(b_i)G(a_i)| + \sum\limits_{i=1}^n |F(b_i)G(a_i) - F(a_i)G(a_i)|$\\ [14pt]
		$\phantom{\sum\limits_{i=1}^n |F(b_i)G(b_i) - F(a_i)G(a_i)|} = \sum\limits_{i=1}^n |F(b_i)||G(b_i) - G(a_i)| + \sum\limits_{i=1}^n |G(a_i)||F(b_i) - F(a_i)|$\\ [14pt]
		$\phantom{\sum\limits_{i=1}^n |F(b_i)G(b_i) - F(a_i)G(a_i)|} \leq M\sum\limits_{i=1}^n |G(b_i) - G(a_i)| + M\sum\limits_{i=1}^n |F(b_i) - F(a_i)|$\\ [14pt]
		$\phantom{\sum\limits_{i=1}^n |F(b_i)G(b_i) - F(a_i)G(a_i)|} < M\frac{\varepsilon}{2M} + M\frac{\varepsilon}{2M} = \varepsilon$\\ [14pt]
	\end{tabular}\retTwo\par}

	Hence, $FG$ is also absolutely continuous on $[a, b]$. It follows by the fundamental theorem of calculus for Lebesgue integrals that:

	{\centering $F(b)G(b) - G(b)G(a) = \int_a^b (FG)^\prime(x)\df x = \int_a^b (FG^\prime + GF^\prime)(x)\df x$. $\blacksquare$ \retTwo\par}
\end{myIndent}

\hOne The following is also tangentially relevant to exercise 8.18(b) in addition to being interesting in its own right. A function $f : \mathbb{R} \to \mathbb{C}$ is \udefine{singular} if $f$ is continuous\\ everywhere, $f^\prime$ exists a.e. with $f^\prime = 0$ a.e., and $f$ is not a constant function.\retTwo

Recalling page 53 of this journal, it's easy to see that the Cantor function is singular (if you continuously extend it to be constant outside $[0, 1]$). However, it's also constant on a small enough neighborhood around almost every point. Hence, it doesn't defy our intution too overly much. In the next two exercises however, we'll construct a strictly increasing singular function.\retTwo

\Hstatement\blab{Exercise 3.39:} If $(F_k)_{k \in \mathbb{N}}$ is a sequence of nonnegative increasing functions on $[a, b]$ such that $F(x) = \sum_{k=1}^\infty F_k(x) < \infty$ for all $x \in [a, b]$, then $F^\prime(x) = \sum_{k=1}^\infty F_k^\prime(x)$ for a.e. $x \in [a, b]$.

\begin{myIndent}\HexOne
	For all $k$, define:
	
	{\centering $G_k(x) = \left\{\begin{matrix}
	F_k(a+) & \text{if } x \leq a\\
	F_k(x+) & \text{if } x \in [a, b)\\
	F_k(b) & \text{if } x \geq b
	\end{matrix}\right.$ \hspace{1em} and \hspace{1em} $G(x) = \left\{\begin{matrix}
	F(a+) & \text{if } x \leq a\\
	F(x+) & \text{if } x \in [a, b)\\
	F(b) & \text{if } x \geq b
	\end{matrix}\right.$ \newpage\par}

	Then by theorem 3.23, we know that $G_k^\prime = F_k^\prime$; $G^\prime = F^\prime$ a.e. on $[a, b]$ Also, each $G_k$ is a nonnegative monotone increasing function with $G(x) = \sum_{k=1}^\infty G_k(x) < \infty$ for all $x \in \mathbb{R}$.
	\begin{myIndent}\HexTwoP
		Why: For any $x \in [a, b)$, we can apply the dominated convergence theorem to $l^1(\mathbb{N})$ using the upper bound $F(b) = \sum_{k=1}^\infty F_k(b)$ in order to get that:

		{\centering\fontsize{11}{13}\selectfont $G(x) = F(x+) = \lim\limits_{t \to 0^+} \sum\limits_{k=1}^\infty F_k(x + t) = \sum\limits_{k=1}^\infty \lim\limits_{t \to 0^+} F_k(x + t) = \sum\limits_{k=1}^\infty F_k(x+) = \sum\limits_{k=1}^\infty G_k(x)$\retTwo\par}
	\end{myIndent}

	Taking things one step further, define $H_k(x) = G_k(x) - G_k(a)$ and $H(x) = G(x) - G(a)$. Since adding by a constant doesn't change the derivative of a function at all, we still know that $H_k^\prime = F_k^\prime$; $H^\prime = F^\prime$ a.e. on $[a, b]$ Also, since $G_k$ and $G$ are monotone increasing, we\\ [1pt] know that $G_k(x) \geq G_k(a)$ and $G(x) \geq G(a)$ for all $x$. Hence, all of our $H_k$ and $H$ are still nonnegative monotone increasing functions on $\mathbb{R}$. And clearly $H(x) = \sum_{k=1}^\infty H_k(x)$ for all $x \in \mathbb{R}$.\retTwo

	The significance of this is that if we now prove that $H^\prime(x) = \sum_{k = 1}^\infty H_k^\prime(x)$ for a.e.\\ [1pt] $x \in [a, b]$, then we will have also shown that $F^\prime(x) = \sum_{k=1}^\infty F_k^\prime(x)$ for a.e. $x \in [a, b]$.\\ [0pt] But importantly, all our $H_k$ and $H$ are in $\NBV$. After all, they are in $\BV$  because they\\ [0pt] are bounded and monotone increasing. Also, they are all right continuous and\\ [0pt] $H_k(-\infty) = 0 = H(-\infty)$. It then follows that there are unique finite Borel measures\\ [0pt]  $\mu_{H_k}$ and$\mu_H$ such that $\mu_{H_k}((-\infty, x]) = H_k(x)$ and $\mu_{H}((-\infty, x]) = H(x)$.\retTwo

	Now let $\df \mu_{H_k} = \df \lambda_k + f_k \df m$ and $\df \mu_H = \df \lambda + f \df m$ be the Radon-Nikodym\\ representations of $\mu_{H_k}$ and $\mu_H$ with respect to the Lebesgue measure. Then since $\mu_H$ and $\mu_{H_k}$ are both finite measures in the separable locally compact Hausdorff space $\mathbb{R}$, we know by theorem 7.8 that $\mu_{H_k}$ and $\mu_H$ are regular. Also, if we let $E_r(x) = (x, x + r]$ for all $r > 0$ and $x \in \mathbb{R}$, then we know that $E_r$ shrinks nicely to $x$ for all $x$. Therefore, by the generalized Lebesgue differentiation theorem, we have:

	{\centering $H^\prime(x) = \lim\limits_{h \to 0^+}\frac{H(x + h) - H(x)}{h} =
		\lim\limits_{h \to 0^+}\frac{\mu_H((x, x+h])}{h} = \lim\limits_{h \to 0^+}\frac{\mu_H(E_h(x))}{m(E_h(x))} = f(x)$ for $m$-a.e. $x$.\par}

	\phantom{aaaaaaaaaaaaaaaaaaaaaaaaaaa}{\HexPPP(And similarly we have that $H_k^\prime(x) = f_k(x)$ for $m$-a.e. $x$.)}\retTwo

	Next note that for any $(a, b) \subseteq \mathbb{R}$, we have that:
	
	{\centering \begin{tabular}{l}
		$\mu_H((a, b)) = \lim\limits_{\beta \to b-}\mu_H((a, \beta]) = \lim\limits_{\beta \to b-}\left(H(\beta) - H(a)\right) = \lim\limits_{\beta \to b-}\sum\limits_{k=1}^\infty (H_k(\beta) - H_k(a))$.
	\end{tabular} \retTwo\par}

	Once again, $H_k(\beta) - H_k(a) \geq 0$ for all $k$ and our series is bounded from above by\\ $\sum_{k=1}^\infty (H_k(b) - H_k(a)) = \mu_{H_k}(a, b] < \infty$. So, by applying the dominated convergence theorem we get that:

	{\centering \begin{tabular}{l}
		$\sum\limits_{k=1}^\infty (H_k(\beta) - H_k(a)) = \sum\limits_{k=1}^\infty \lim\limits_{\beta \to b-} (H_k(\beta) - H_k(a)) = \sum\limits_{k=1}^\infty \lim\limits_{\beta \to b-} \mu_{H_k}((a, \beta]) = \sum\limits_{k=1}^\infty \mu_{H_k}((a, b))$.
	\end{tabular} \retTwo\par}

	This in turn proves that $\mu_H = \sum_{k=1}^\infty \mu_{H_k}$ on all open sets by the countable additivity of measures, and all measurable sets in general by outer regularity. Hence, we know that $\df \lambda + H^\prime \df m = \sum_{k=1}^\infty (\df \lambda_k + H_k^\prime \df m) = \sum_{k=1}^\infty \df \lambda_k + \sum_{k=1}^\infty H_k^\prime \df m$.\retTwo

	Lastly, note that $\sum_{k=1}^\infty \lambda_k \perp m$ and $\sum_{k=1}^\infty H_k^\prime \df m = \left(\sum_{k=1}^\infty H_k^\prime \right) \df m \ll m$. By the Radon-\\ [2pt]Nikodym theorem, we thus know that $ \lambda = \sum_{k=1}^\infty \lambda_k$ and $H^\prime \df m =  \left(\sum_{k=1}^\infty H_k^\prime \right) \df m$ with\\ [2pt] $H^\prime = \sum_{k=1}^\infty H_k^\prime$ $m$-a.e. $\blacksquare$\newpage
\end{myIndent}

\blab{Exercise 3.40:} Let $F$ denote the Cantor function on $[0, 1]$ and set $F(x) = 0$ for $x > 1$ and $F(x) = 1$ for $x > 1$. Let $\{[a_n, b_n]\}_{n \in \mathbb{N}}$ be an enumeration of the closed subintervals of $[0, 1]$ with distinct rational endpoints, and let $F_n(x) = F(\frac{x - a_n}{b_n - a_n})$. Then $ G = \sum_{n=1}^\infty 2^{-n}F_n$ is continuous\\ [-2pt] and strictly increasing on $[0, 1]$, and $G^\prime = 0$ a.e.

\begin{myIndent}\HexOne
	Since $\frac{x - a_n}{b_n - a_n}$ is continuous and increasing, we know that each $F_n$ is still continuous and\\ [-2pt] monotone increasing. Also, we clearly have that if $x \geq b_n$, then $F_n(x) = 1$. Meanwhile,\\ [1pt] if $x \leq a_n$, then $F_n(x) = 0$. Thus, it's easy to see that:
	
	\begin{itemize}
		\item $G(x) = 0$ for $x \leq 0$ and $G(x) = 1$ for $x \geq 1$
		\item $G$ is monotone increasing
		\item $\sum_{n=1}^\infty 2^{-n}F_n$ converges uniformly to $G$, thus making $G$ continuous.
		\item By an easy application of exercise 3.39, $G^\prime = 0$ a.e. since $F^\prime$ being zero almost\\ everywhere implies $(2^{-n}F_n)^\prime = 0$ a.e. for each $n$.
		
		\begin{myIndent}
			\HexPPP Reminder, for any $x$ not in the Cantor set, we know either $x \notin [0, 1]$ or there is an open interval containing $x$ that was removed to form the Cantor set. In either scenario, we have that $f$ is constant on a neighborhood of $x$. So, $f^\prime(x) = 0$.
		\end{myIndent}
	\end{itemize}

	Finally, to show that $G$ is strictly increasing on $[0, 1]$, note that for any $x, y \in [0, 1]$ with $x < y$, we know there is a closed subinterval $[a_n, b_n]$ with $x < a_n < b_n < y$. In turn, we know that $F_n(x) = 0$ while $F_n(y) = 1$. Then since $F_n(y) \geq F_n(x)$ for all other $n$, we know that $G(x)$ is strictly less than $G(y)$.\retTwo

	\begin{myIndent}\pracTwo\fontsize{11}{13}\selectfont
		Note: We can also fairly easily see now that $\sum_{n \in \mathbb{Z}}(n + G(x - n))\chi_{[n, n+1]}$ is strictly increasing and continuous everywhere with a derivative equal to zero almost everywhere.\retTwo
	\end{myIndent}
\end{myIndent}

\hOne This poses a challenge because in exercise 8.18(b), we're going to need to be able say that a function having a derivative of zero almost everywhere implies that the function is constant. So here is one more lemma.\retTwo

\exOne \ul{Lemma:} If $f: \mathbb{R} \to \mathbb{C}$ is absolutely continuous on $[a, b]$ and $f^\prime = 0$ a.e. on $[a, b]$, then $f$ is constant on $[a, b]$.
\begin{myIndent}
	\why By the fundamental theorem of calculus for Lebesgue integrals, we know that if\\ $x \in [a, b]$, then $f(x) - f(a) = \int_a^x f^\prime(t)\df t = \int_a^x 0 \df t = 0$. So, $f(x) = f(a)$.\retTwo
\end{myIndent}

\dispDate{7/13/2025}

\Hstatement\blab{Exercise 8.18 (continued):}
\begin{enumerate}
	\item[(b)] If the $L^2$ derivative $f^\prime$ exists, then: $\left[\int |f(x)|^2 \df x\right]^2 \leq 4 \int |xf(x)|^2 \df x \int |f^\prime(x)|^2 \df x$.

\begin{myIndent}\HexOne
	To start out, we need to make sure this inequality is well defined. Note that since $f, f^\prime \in L^2$, we know that $\int |f(x)|^2 \df x < \infty$ and $\int |f^\prime(x)|^2 \df x < \infty$. So, to guarentee that this inequality is well defined, we just need to show that if $\int |f^\prime(x)|^2 \df x = 0$, then we will never have that $\int |xf(x)|^2 \df x = \infty$ (thus making the right-hand side $4(\infty \cdot 0))$. Luckily, by exercise 8.9, we know that $f$ having an $L^2$ derivative means that $f$ is absolutely continuous on every bounded interval. So by the lemma I ended yesterday with, we know that if $\int |f^\prime(x)|^2 \df x = 0$, then $f$ must be constant on every bounded interval since the ordinary derivative of $f$ is zero almost everywhere. This proves that $f = c$ where $c$ is some constant. But since $f \in L^2$, we must have that $\int_{-\infty}^\infty |c|^2 \df x < \infty$. The only way this is possible is if $c = 0$. So, $f = 0$ a.e. and we've thus shown that $\int |xf(x)|^2 \df x = 0$ as well.\newpage

	Next, for any $a < b$ note that $|f|^2$ is absolutely continuous on $[a, b]$. This is because as\\ mentioned before, $f$ is absolutely continuous on $[a, b]$. Then in turn, it is easy to see\\ [-1pt] that $\overline{f}$ is absolutely continuous on $[a, b]$. So, by exercise 3.35, we know that $f \overline{f} = |f|^2$\\ [2pt] is absolutely continuous on $[a, b]$. Also $g(x) = x$ is absolutely continuous on $[a, b]$. Thus by exercise 3.35, we know that:\\ [-15pt]

	{\centering $\int_a^b (1|f(x)|^2 + x\frac{\df}{\df x}|f(x)|^2) = b|f(b)|^2 - a|f(a)|^2$ \retTwo\par}

	Or in other words: $\int_a^b |f(x)|^2 \df x = b|f(b)|^2 - a|f(a)|^2 - \int_a^b x\frac{\df}{\df x}|f(x)|^2 \df x$.\retTwo
	
	Also, note:\\ [-15pt]
	
	{\centering $\frac{\df}{\df x}|f(x)|^2 = \frac{\df}{\df x}(f(x)\overline{f(x)}) = f^\prime(x)\overline{f(x)} + f(x)\overline{f^\prime(x)} = 2\rea{f^\prime(x)\overline{f(x)}}$.\retTwo\par}

	Hence, for any $a < b$, we have:\\ [-15pt]
	
	{\centering $\int_a^b |f(x)|^2\df x = b|f(b)|^2 - a|f(a)|^2 - 2\rea{\int_a^b x f^\prime(x)\overline{f(x)}\df x}$\retTwo\par}


	Now since the inequality we want to prove is trivial if $\int |xf(x)|^2 \df x = \infty$, we can\\ [-1pt] safely assume $\int |xf(x)|^2 \df x < \infty$. This is important because it guarentees that for\\ [-1pt] any $n \in \mathbb{N}$ and $\varepsilon > 0$, we can pick $a_n < -n$ and $b_n > n$ such that $|a_nf(a_n)|^2 < \sfrac{1}{n}$\\ [2pt] and $|b_n f(b_n)|^2 < \sfrac{1}{n}$. In turn, this lets us say that $a_n|f(a_n)|^2 \in (-\sfrac{1}{n}, 0]$ and\\ [1pt] $b_n|f(b_n)|^2 \in [0, \sfrac{1}{n})$ since $a_n < -1$ and $b_n > 1$.\retTwo

	Now by an application of dominated convergence theorem, we know that:\\ [-15pt]

	{\centering \begin{tabular}{l}
		$\int |f(x)|^2 \df x = \lim\limits_{n \to \infty} \int_{a_n}^{b_n} |f(x)|^2 \df x$\\ [4pt]
		$\phantom{\int |f(x)|^2 \df x} = \lim\limits_{n \to \infty} \left(b_n|f(b_n)|^2 - a_n|f(a_n)|^2 - 2 \rea{\int_{a_n}^{b_n} x f^\prime(x)\overline{f(x)}\df x}\right)$\\ [8pt]
		$\phantom{\int |f(x)|^2 \df x} = -2 \cdot \lim\limits_{n \to \infty} \rea{\int_{a_n}^{b_n} xf^\prime(x)\overline{f(x)}\df x}$\\
	\end{tabular} \retTwo\par}

	Then by the Cauchy-Schwartz inequality (using the fact that $f^\prime\chi_{[a_n, b_n]}, xf \in L^2)$, we can say that:\\ [-19pt]

	{\centering \begin{tabular}{l}
		$-2 \cdot \lim\limits_{n \to \infty} \rea{\int_{a_n}^{b_n} xf^\prime(x)\overline{f(x)}\df x} \leq 2 \lim\limits_{n \to \infty} \left|\int_{a_n}^{b_n} xf^\prime(x)\overline{f(x)} \df x\right|$\\ [8pt]
		$\phantom{-2 \cdot \lim\limits_{n \to \infty} \rea{\int_{a_n}^{b_n} xf^\prime(x)\overline{f(x)}\df x}} \leq 2 \lim\limits_{n \to \infty} \left(\int_{a_n}^{b_n} |f^\prime(x)|^2 \df x\right)^{\sfrac{1}{2}}\hspace{-0.4em}\left(\int |xf(x)|^2 \df x\right)^{\sfrac{1}{2}}$\\ [8pt]
	\end{tabular} \retTwo\par}

	By a final application of dominated convergence theorem using an upper bound of $|f^\prime(x)|^2$, we get that:\\ [-19pt]
	
	{\centering $\lim\limits_{n \to \infty} \left(\int_{a_n}^{b_n} |f^\prime(x)|^2 \df x\right)^{\sfrac{1}{2}} = \left(\int |f^\prime(x)|^2 \df x\right)^{\sfrac{1}{2}}$\retTwo\par}

	So, $\int |f(x)|^2\df x \leq \left(\int |f^\prime(x)|^2 \df x\right)^{\sfrac{1}{2}}\left(\int |xf(x)|^2 \df x\right)^{\sfrac{1}{2}}$. Squaring both sides gives the desired inequality.\retTwo
\end{myIndent}

\item[(c)] (\blab{Heisenberg's Inequality}) For any $b, \beta \in \mathbb{R}$,

{\centering $\int (x-b)^2|f(x)|^2\df x\int (\xi - \beta)^2|\widehat{f}(\xi)|^2\df \xi \geq \frac{\|f\|_2^4}{16\pi^2}$\par}

\begin{myIndent}\HexOne
	To start off, $f = 0$ a.e. if and only if $\widehat{f} = 0$ a.e. It follows that we will never have a $0 \cdot \infty$ situation on the left-hand side, and thus our inequality is well-defined. Also, if\newpage either of the two left hand integrals are infinite, then the inequality is trivial. So, we may assume both integrals are finite.\retTwo

	Next note that if we consider $g(x) = f(x + b)$, then I already showed on page 71 that $\widehat{g}(\xi) = e^{2\pi i \xi b}\widehat{f}(\xi)$. In turn, we know that:\\ [-13pt]

	\begin{itemize}
		\item $\int x^2|g(x)|^2\df x = \int x^2|f(x + b)|^2 \df x = \int (x - b)^2|f(x)|^2\df x$,
		\item $\int (\xi - \beta)^2|\widehat{g}(\xi)|^2 \df \xi = \int (\xi - \beta)^2|e^{2\pi i \xi b}\widehat{f}(\xi)|^2 \df \xi = \int (\xi - \beta)^2|e^{2\pi i \xi b}\widehat{f}(\xi)|^2 \df \xi$,\\ [-9pt]
		\item $\|g\|_2 = \|f\|_2$.\retTwo
	\end{itemize}

	So, by proving our inequality for $g$ when $b = 0$, we've also proven it for $f$ when $b$ is anything.\retTwo

	Going a step further, set $h(x) = e^{-2\pi i \beta x}g(x)$. Then $h^\vee(\xi) = g^\vee(\xi - \beta) = \widehat{g}(\beta - \xi)$. So, we know that $\widehat{h}(\xi) = \widehat{g}(\beta + \xi)$ since $\widehat{g}(\xi) = g^\vee(-\xi)$. In turn:
	\begin{itemize}
		\item $\int x^2|h(x)|^2\df x = \int x^2|e^{-2\pi i \beta x}g(x)|^2 \df x = \int x^2|g(x)|^2\df x$,
		\item $\int \xi^2|\widehat{h}(\xi)|^2 \df \xi = \int \xi^2|\widehat{g}(\xi + \beta)|^2 \df \xi = \int (\xi - \beta)^2|\widehat{g}(\xi)|^2 \df \xi$,\\ [-9pt]
		\item $\|h\|_2 = \|g\|_2$.\retTwo
	\end{itemize}

	So, by proving our inequality for $h$ when $b = 0$ and $\beta = 0$, we've also proven it for $f$ when $b$ and $\beta$ are anything. Luckily, proving that for $h$ is easy due to what we've already proven in parts (a) and (b) of this exercise.\retTwo

	Since $\int \xi^2|\widehat{h}(\xi)|^2\df \xi < \infty$, we know from part (a) that $h$ has an $L^2$ derivative $h^\prime$\\ which satisfies that:\\ [-18pt]
	
	{\centering $\frac{1}{2\pi i \xi}\widehat{h^\prime}(\xi) = \widehat{h}(\xi)$.\retTwo\par}

	In turn, we can rewrite $\int \xi^2|\widehat{h}(\xi)|^2\df \xi = \frac{1}{4\pi^2}\int |\widehat{h^\prime}(\xi)|^2\df \xi$ and the latter is just\\ $\frac{1}{4\pi^2}\int |h^\prime(\xi)|^2\df \xi$ by the Plancharel theorem. Finally, by applying part (b) we get that:

	{\centering $\frac{1}{4\pi^2} \int x^2|h(x)|^2\df x \int |h^\prime(\xi)|^2\df \xi \geq \frac{\|h\|_2^4}{4} \cdot \frac{1}{4\pi^2} = \frac{\|h\|_2^4}{16\pi^2}$. $\blacksquare$\retTwo\par}

	\begin{myIndent}\pracTwo\fontsize{11}{13}
		This inequality is the cause of the quantum uncertainty principle. To see why, first note that in quantum mechanics, a property of a particle at a given point in time is modeled as a probability density function whose density at a point $x$ is\\ [-2pt] $|f(x)|^2$ where $f$ is some function in $L^2$ (importantly this means $\|f\|_2 = 1$ always in this context).\retTwo
		
		In turn, $\int (x - b)^2|f(x)|^2\df x$ is the formula for the variance of that probability\\ distribution around $b$. So, that integral evaluates to something small precisely when the probability distribution of the property of the particle has a small\\ standard deviation and $b$ is close to the mean of the distribution.\newpage

		Next, note that in quantum mechanics, pairs of properties are related to each other by a Fourier transformation. Hence, $|\widehat{f}(\xi)|^2\df x$ is the probability density function of another property of the particle.\retTwo
		
		Similarly to before, {$\int (x - \beta)^2|\widehat{f}(x)|^2\df x$} is the formula for the variance of that probability distribution around $\beta$, and that will be small precisely when the\\ probability distribution of the property has a small standard deviation and $\beta$ is close to the mean of the distribution.\retTwo

		Now finally, $\int (x - b)^2|f(x)|^2\df x\int (x - \beta)^2|\widehat{f}(x)|^2\df x \geq \frac{1}{16\pi^2}$ for all $b, \beta \in \mathbb{R}$ means that it's impossible for both probability distributions to simultaneously have a standard deviation less than $\frac{1}{2\sqrt{\pi}}$, and decreasing one of the standard deviations beyond that value necessarily requires increasing the other. This is the quantum uncertainty principle.\retTwo
	\end{myIndent}
\end{myIndent}
\end{enumerate}

\blab{Exercise 8.19:} If $f \in L^2(\mathbb{R}^n)$ and the set $S = \{x : f(x) \neq 0\}$ has finite measure, then for any measurable $E \subseteq \mathbb{R}^n$, $\int_E |\widehat{f}|^2 \leq \|f\|_2^2 m(S)m(E)$.

\begin{myIndent}\HexOne
	By Minkowski's inequality for integrals, we have:

	{\centering\fontsize{11}{12} \begin{tabular}{l}
		$\int_E |\widehat{f}|^2 = \int \chi_{E}(\xi) | \int f(x)e^{-2\pi i \xi \cdot x}\df x|^2 \df \xi$\\ [6pt]
		$\phantom{\int_E |\widehat{f}|^2} = \int | \int f(x)e^{-2\pi i \xi \cdot x} \sqrt{\chi_{E}(\xi)}\df x|^2 \df \xi$\\ [6pt]
		$\phantom{\int_E |\widehat{f}|^2} \leq \left[\int (\int |f(x)e^{-2\pi i \xi \cdot x} \sqrt{\chi_{E}(\xi)}|^2\df \xi)^{\sfrac{1}{2}} \df x\right]^2 = \left[\int |f(x)| (\int_E\df \xi)^{\sfrac{1}{2}} \df x\right]^2 = m(E)\left(\int |f(x)|\df x\right)^2$
	\end{tabular}\retTwo\par}

	Next, by Hölder's inequality we have:

	{\centering\fontsize{11}{12} \begin{tabular}{l}
		$\int |f(x)|\df x = \int |\chi_S(x)f(x)|\df x \leq \|\chi_S\|_2\|f\|_2 = \sqrt{m(S)}\|f\|_2$.
	\end{tabular}\retTwo\par}

	Thus $\int_E |\widehat{f}|^2 \leq m(E)(\sqrt{m(S)}\|f\|_2)^2 = m(E)m(S)\|f\|_2^2$. $\blacksquare$\retTwo

	\begin{myIndent}\pracTwo\fontsize{11}{13}
		This inequality is another cause/statement of the quantum uncertainty principle. This is because to optimize the precision of our measurement of the property associated to the wave $|\widehat{f}|^2$, we'd want to maximize $\int_E |\widehat{f}|^2$ while simultaneously minimizing $m(E)$. But, this inequality says that doing that requires increasing $m(S)$. I.e., it requires us to know less about the property associated to the wave $|f|^2$.\retTwo
	\end{myIndent}
\end{myIndent}

\dispDate{7/15/2025}\hOne

Welp, I'm currently sick. Anyways, now that I've scanned my paper notes from\\ winter quarter, I'm thinking I want to finally learn vector calculus properly since I never really learned it in math 20E. Also, since I'm crashing the physics 4 sequence, I'm going to eventually need to finally learn Stokes' theorem and Divergence\\ theorem when they cover E\&M.\retTwo

For now, my plan is to sort of follow along with Munkres' \ul{Analysis on Manifolds}, starting at chapter 5. That said, I want to work with Lebesgue integrals. So, I might go on some tangents and or come up with different proofs for things. Also, I might skip something if I'm bored.\newpage

Conventions:
\begin{itemize}
	\item $\{e_1, \ldots, e_n\}$ will refer to the standard basis on $\mathbb{R}^n$.\\ [-16pt]
	\item If $f \in C^r(U)$ where $U \subseteq \mathbb{R}^n$ is open and $r \geq 1$, then $\Df f$ will refer to the\\ derivative matrix of $f$ with respect to the standard bases. I.e, $\Df f$ is the matrix of partial derivatives of $f$.\retTwo
\end{itemize}

\exOne\ul{Lemma:} Let $W$ be a $k$-dimensional linear subspace of $\mathbb{R}^n$. Then there is an\\ orthogonal (i.e. unitary) linear transformation $h: \mathbb{R}^n \to \mathbb{R}^n$ that carries $W$ onto the subspace $\mathbb{R}^k \times 0^{n-k}$ of $\mathbb{R}^n$.

\begin{myIndent}\exTwoP
	Proof:\\
	Let $\{v_1, \ldots, v_n\}$ be an orthonormal basis for $W$ such that $\{v_1, \ldots, v_k\}$ is an\\ orthonormal basis for $W$. Then if $g$ is the linear map whose matrix with respect\\ to the standard basis has columns, $v_1, \ldots, v_n$, we know $g$ is orthogonal and\\ $g(e_i) = v_i$ for all $i$. Now just set $h = g^{-1}$.\retTwo
\end{myIndent}

\ul{Theorem:} Let $k, n \in \mathbb{N}$ with $0 < k \leq n$. There is a unique function $V$ that assigns to each $k$-tuple: $(x_1, \ldots, x_k)$, of elements in $\mathbb{R}^n$ a nonnegative number such that:\\ [-22pt]

\begin{enumerate}
	\item If $h: \mathbb{R}^n \to \mathbb{R}^n$ is an orthogonal transformation, then:

	{\centering $V(h(x_1), \ldots, h(x_k)) = V(x_1, \ldots, x_k)$\par}

	\item If $y_1, \ldots, y_k$ belong to the subspace $\mathbb{R}^k \times 0^{n-k}$ with $y_i = \left[\begin{smallmatrix}
		z_i \\ 0
	\end{smallmatrix}\right]$ where $z_i \in \mathbb{R}^k$, then $V(y_1, \ldots, y_k) = |\det(z_1, \ldots, z_k)|$.
\end{enumerate}

Specifically, we have, $V(x_1, \ldots, x_k) = \left(\det(X^\trans X)\right)^{\sfrac{1}{2}}$ where $X$ is the $n \times k$ matrix $X = [x_1, \ldots, x_k]$. 

\begin{myTindent}\begin{myDindent}\exPPP
	(Note: we will typically abreviate $V(x_1, \ldots, x_k)$ as $V(X)$\dots)\retTwo
\end{myDindent}\end{myTindent}

\begin{myIndent}\exTwoP
	Proof:\\
	Given $X = [x_1, \ldots, x_k]$, define $F(X) = \det(X^{\trans}X)$. Then note:
	\begin{itemize}
		\item If $h: \mathbb{R}^n \to \mathbb{R}^n$ is an orthogonal linear map, then $h(x) = Ax$ where $A$ is an\\ orthogonal matrix. Now $[h(x_1), \ldots, h(x_k)] = [Ax_1, \ldots, Ax_k] = AX$. Thus:
		
		{\centering $F(h(X)) = f(AX) = \det{(AX)^\trans AX} = \det{X^\trans A^\trans A X} = \det{X^\trans X} = F(X)$\retTwo\par}

		\item If $Z$ is a $k\times k$ matrix and $Y$ is the $n \times k$ matrix $\left[\begin{smallmatrix}
		Z \\ 0 \end{smallmatrix}\right]$, then:

		{\centering $F(Y) = \det\left(\left[\begin{matrix}
		Z^\trans & 0 \end{matrix}\right] \left[\begin{matrix}
		Z \\ 0 \end{matrix}\right]\right) = \det\left(Z^\trans Z\right) = \left(\det(Z)\right)^2$ \retTwo\par}
	\end{itemize}

	With that, all we need to do is show that $F$ is nonnegative so that we can take the square root of $F$. Luckily, if $\{x_1, \ldots, x_k\}$ are any $k$-tuple of vectors in $\mathbb{R}^n$, then we know there is a $k$-dimensional linear subspace of $\mathbb{R}^n$ containing $x_1, \ldots, x_k$. By our prior lemma, there exists an orthogonal linear map $h$ taking $W$ to $\mathbb{R}^k \to 0^{n-k}$. Next by our first bullet, we know that $F(h(X)) = F(X)$. And finally, by our second bullet we know $F(h(X)) = (\det(h(X)))^2 \geq 0$.\newpage

	Now just define $V(X) = \sqrt{F(X)}$. This proves the existence part of the theorem.\retTwo

	To prove uniqueness, suppose $U$ also satisfies our two axioms. Then for any\\ $\{x_1, \ldots, x_k\} \subseteq \mathbb{R}^n$, let $h(x) = Ax$ be an orthogonal linear map taking $\{x_1, \ldots, x_k\}$ into $\mathbb{R}^k \times 0^{n-k}$. Thus $U(X) = U(h(X)) = |\det(AX)| = V(h(X)) = V(X)$

	\begin{myIndent}
		\pracTwo Side note: if $\{x_1, \ldots, x_k\}$ are not linearly independent, then $V(X) = 0$.\retTwo

		Also, hopefully it's clear that the significance of $V$ is that we now have a way of defining the $k$-dimensional volume of a $k$-dimensional parallelepiped in $\mathbb{R}^n$.\retTwo
	\end{myIndent}
\end{myIndent}

\hOne With that, we're ready to start integrating on manifolds. We'll start with the simple case of a manifold parametrized by a single function.\retTwo

Let $k \leq n$. Let $A$ be open in $\mathbb{R}^k$ and let $\alpha : A \to \mathbb{R}^n$ be an injective $C^r$ map (where $r \geq 1$). The set $Y = \alpha(A)$ together with the map $\alpha$ constitute a \udefine{parametrized-\\manifold} of dimension $k$. We denote this parametrized manifold $Y_\alpha$. For a topology, we equip $Y_\alpha$ with the subspace topology of $Y$ with respect to $\mathbb{R}^n$. That way $\alpha$ is still a continuous map.\retTwo

Next, we define a Borel measure on $Y_\alpha$. Given any set $E \in \mathcal{B}_{Y_\alpha}$, we define:

{\centering $V(E) \coloneq \int_{\alpha^{-1}(E)} V(\Df \alpha)$\par}

\begin{myTindent}\exPPP
	(unfortunately the measure is typically called $V$ even though we already named another function that.)\retTwo
\end{myTindent}

\begin{myIndent}
	Note, $V(\Df \alpha)$ is Borel measurable because the matrix determinant is a\\ continuous function with respect to all the matrix entries and all the entries of $\Df \alpha$ are continuous since $\alpha \in C^1$. It follows that our integral is well-defined.\retTwo

	Since $V(\emptyset) = 0$ and $V$ is clearly countably additive, we know $V$ is a measure. Also, the naturalness of this measure is hopefully clear. After all, you can\\ imagine that we are approximating the $k$-dimensional volume using a bunch of tiny parallelepipeds.\retTwo
\end{myIndent}

Then, given any measurable function $f: Y \to \mathbb{C}$ on our manifold, we can integrate via the formula: $\int_{Y_\alpha} f \df V = \int_A (f \circ \alpha)V(\Df \alpha) \df m$.

\begin{myIndent}
	\why This formula clearly holds for simple functions. Then you can extend that to\\ nonnegative functions via the monotone convergence theorem and then to real and\\ complex functions in the standard way.\retTwo
\end{myIndent}

\exOne\ul{Theorem:} Let $g: A \to B$ be a diffeomorphism of open sets in $\mathbb{R}^k$ and let $\beta: B \to \mathbb{R}^n$ be an injective $C^r$ map (with $r \geq 1$). If we define $\alpha = \beta \circ g$, then $\alpha: A \to \mathbb{R}^n$ is also an injective $C_r$ map and $\alpha(A) = \beta(B) = Y$. Then, a function $f: Y \to \mathbb{C}$ is integrable over $Y_\beta$ if and only if it is integrable over $Y_\alpha$, in which case:

{\centering $\int_{Y_\alpha} f\df V_\alpha = \int_{Y_\beta} f\df V_\beta$ \retTwo\par}

\begin{myIndent}\exTwoP 
	Proof:\\
	The measurability of $f$ is independent of our parametrization since the topology of $Y_\alpha$ and $Y_\beta$ was not defined using $\alpha$ or $\beta$. Next note that by change of variables:

	{\centering\begin{tabular}{l}
		$\int_{Y_\beta} f \df V_\beta = \int_B (f \circ \beta)V(\Df \beta)\df m = \int_A (f \circ \beta \circ g)(V(\Df \beta)\circ g)|\det(\Df g)|\df m$\\ [6pt]
		$\phantom{\int_{Y_\beta} f \df V_\beta = \int_B (f \circ \beta)V(\Df \beta)\df m} = \int_A (f \circ \alpha)(V(\Df \beta)\circ g)|\det(\Df g)|\df m$
	\end{tabular} \retTwo\par}

	Thus, we just need to show that $(V(\Df \beta)\circ g)|\det(\Df g)| = V(\Df \alpha)$. To do that, note by chain rule that $\Df \alpha = ((\Df \beta) \circ g) \Df g$. Therefore:
	
	{\centering \begin{tabular}{l}
		$(V(\Df \alpha))^2 = \det((((\Df \beta) \circ g) \Df g)^\trans(((\Df \beta) \circ g) \Df g))$\\ [6pt]
		$\phantom{(V(\Df \alpha))^2} = (\det(\Df g))^2\det(((\Df \beta) \circ g)^\trans((\Df \beta) \circ g)) = (\det(\Df g))^2(V((\Df \beta) \circ g))^2$ $\blacksquare$
	\end{tabular} \retTwo\par}
\end{myIndent}

\pracOne\ul{Exercise 22.1:} Let $A$ be open in $\mathbb{R}^k$, $\alpha: A \to \mathbb{R}^n$ be a $C^1$ map, and $Y = \alpha(A)$. Suppose $h: \mathbb{R}^n \to \mathbb{R}^n$ is an isometry and let $Z = h(Y)$ and $\beta = h \circ \alpha$. Then $Y_\alpha$ and $Z_\beta$ have the\\ [-3pt] same volume. 

\begin{myIndent}\pracTwo
	Since $h$ is an isometry, we know $h$ has the form $h(x) = Qx + b$ where $Q$ is an orthogonal matrix and $b$ is a constant vector. Thus by chain rule, we have that $\Df \beta = Q \Df \alpha$. So:

	{\centering $\int_{Z_\beta} \df V = \int_A V(\Df \beta) = \int_A V(Q\Df \alpha) = \int_A V(\Df \alpha) = \int_{Y_\alpha} \df V$ \retTwo\par}
\end{myIndent}

\hOne\mySepTwo

\dispDate{7/16/2025}

Now our previous approach to manifolds is lacking in some respects. For one, it'd be nice if our manifolds were able to include their boundaries. However, requiring the domains of our parametrizations to be strictly open makes that difficult. Also, it'd be nice if we could talk about manifolds that can't be parametrized by a single function (such as the unit sphere $S^2$ in $\mathbb{R}^3$).\retTwo

To address the first issue, we extend our notion differentiability. Let $S$ be any subset of $\mathbb{R}^k$ and let $f: S \to \mathbb{R}^n$. We say \udefine{$f \in C^r(S)$} iff there exists an open set $U \supset S$ and a function $g: U \to \mathbb{R}^n$ in $C^r(U)$ such that $g|_S = f$.

\begin{myIndent}\hTwo
	Note that if $f_1, f_2 \in C^r(S)$, then we have that $f_1 + f_2 \in C^r(S)$ and $f_1f_2 \in C^r(S)$. After all, supposing we can extend $f_1$ and $f_2$ to the functions $g_1$ and $g_2$ defined respectively on the open sets $U$ and $V$ containing $S$, then $U \cap V$ is still open and contains $S$, and $g_1 + g_2, g_1g_2 \in C^r(U \cap V)$.\retTwo

	Also, if $f_1 \in C^r(S)$ and $f_2 \in C^r(T)$ where $f_1(S) \subseteq T$, then $f_2 \circ f_1 \in C^r(S)$. This is because if $g_1$ extends $f_1$ to an open set $U \supseteq S$ and $g_2$ extends $f_2$ to an open set $V \supseteq T$, then $g_2 \circ g_1$ is a $C^r$ extension of $f_2 \circ f_1$ to the open set $g_1^{-1}(V) \supseteq S$.\retTwo
\end{myIndent}

Next, we show that being in $C^r(S)$ is a local property. However, to do that we first prove a result about partitions of unity.\newpage

Recall from Math 240B that a \udefine{partition of unity} on a set $E \subseteq X$ (where $X$ is an LCH space such as $\mathbb{R}^n$) is a collection of functions $\{h_\alpha\}_{\alpha \in A} \in C(X, [0, 1])$ such that:
\begin{itemize}
	\item Each $x \in X$ has a neighborhood on which only finitely many $h_\alpha$ are nonzero,
	\begin{myIndent}\begin{myTindent}\pracTwo
		Although Munkres only requires this to hold for all $x \in E$.
	\end{myTindent}\end{myIndent}
	\item $\sum_{\alpha \in A}h_\alpha(x) = 1$ for all $x \in E$.\retTwo
\end{itemize}

Also, we say $\{h_\alpha\}_{\alpha \in A}$ is \udefine{subordinate to the open cover $\mathcal{U}$} of $E$ if for each $\alpha$ there exists $U \in \mathcal{U}$ such that $\supp(h_\alpha) \subseteq U$.\retTwo

A really cool result Munkres spends all of chapter 16 proving is that for any collection $\mathcal{A}$ of open sets in $\mathbb{R}^n$ whose union is $A$, there is a partition of unity $\{h_m\}_{m \in \mathbb{N}}$ of $A$ consisting of $C_c^\infty$ functions and such that $\{h_m\}_{m \in \mathbb{N}}$ is subordinate to $\mathcal{A}$.\retTwo

\begin{myIndent}\exOne
	\ul{Lemma 16.2:} Let $\mathcal{A}$ be a collection of open sets in $\mathbb{R}^n$ whose union is $A$.\\ Then there exists a countable collection $\{Q_i\}_{i \in \mathbb{N}}$ of rectangles (i.e. Cartesian\\ products of closed intervals) contained in $A$ such that:
	\begin{enumerate}
		\item The sets $\{Q_i^{\circ}\}_{i \in \mathbb{N}}$ cover $A$.
		\item Each $Q_i$ is contained in an element of $\mathcal{A}$.
		\item Each point of $A$ has a neighborhood that intersects only finitely many of the sets $Q_i$.
	\end{enumerate} 

	\begin{myIndent}\exTwoP
		Proof:\\
		Step 1: Dividing $A$ into a nicely structured sequence of compact sets:

		\begin{myIndent}\exPPP
			Because $\mathbb{R}^n$ is $\sigma$-compact, we know $A$ is a countable union of compact sets. Then, by taking larger and larger finite unions of those compact sets we get an increasing sequence of compact sets $\{K_{i}\}_{i \in \mathbb{N}}$ whose union is $A$.\retTwo

			Next, to get a nicer sequence, we do more finagling. Let $D_1 = K_1$. Then for $i \geq 1$, define $D_{i+1}$ inductively as follows:
			\begin{myDindent}
				We know there exists a precompact open set $V$ such that\\ $(D_i \cup K_i) \subseteq V \subseteq \overline{V} \subseteq A$. Thus, set $D_{i+1} = \overline{V}$.\retTwo
			\end{myDindent}

			Thus, $\{D_i\}_{i \in \mathbb{N}}$ is a sequence of compact sets whose union is $A$ and which satisfies that $D_i \subseteq D_{i+1}^\circ$ for all $i$. Also, for convenience of notation let\\ $D_i = \emptyset$ if $i \leq 0$.\retTwo

			Finally, set $B_i = D_i - D_{i-1}^\circ$ for all $i$. Then note that $\{B_i\}_{i \in \mathbb{N}}$ is a sequence of compact sets whose union is $A$ and which satisfies that $B_i$ is disjoint from $D_{i-2}$ for all $i \geq 2$. Consequently, this means that any $B_i$ only intercepts $B_{i-1}$ and $B_{i+1}$ in our sequence. Also, $U_i \coloneq D_{i+1}^\circ - D_{i-2}$ is an open neighborhood of $B_i$ which intercepts only $B_{i-1}$, $B_i$, and $B_{i+1}$.\newpage
		\end{myIndent}

		Step 2: Making our covering of $A$:

		\begin{myIndent}\exPPP
			After fixing $i$, note that for any $x \in B_i$  we know that there is a set $E \in \mathcal{A}$ such that $x \in E$. So, we can pick a rectangle $Q_x$ such that $x \in Q_x^\circ$ and\\ $Q_x \subseteq E \cap U_i$. Doing this for all $x \in B_i$, we get a collection $\{Q_x\}_{x \in B_i}$ of sets whose interiors give an open covering of $B_i$. So, because $B_i$ is compact, there is a finite collection $\mathcal{C}_i \coloneq \{Q_{x_1}, \ldots, C_{Q_{n_i}}\}$ of rectangles whose\\ interiors cover all of $B_i$ and such that each rectangle is a subset of some element of $\mathcal{A}$ intercepted with $U_i$.\retTwo

			Repeat this process for all $i$ and let $\mathcal{C} = \bigcup_{i \in \mathbb{N}} \mathcal{C}_i$. Then $\mathcal{C}$ is a countable\\ collection of rectangles such that each $Q \in \mathcal{C}$ is contained in an element of $\mathcal{A}$. Also, for any $x \in A$, we know there is some $j$ with:
			
			{\centering $x \in \left(B_{j-1} \cup B_{j} \cup B_{j+1}\right) - \hspace{-1.8em}\bigcup\limits_{\begin{smallmatrix}
			i \in \mathbb{N}\\ i \notin \{j-1, j, j+1\}
			\end{smallmatrix}}\hspace{-1.8em} B_i$\retTwo\par}

			In turn, $x$ is in the interior of one the rectangles in $\mathcal{C}_{j-1} \cup \mathcal{C}_{j} \cup \mathcal{C}_{j+1}$.\retTwo

			Finally, supposing $x \in B_k$, then we know the open neighborhood $U_k \subseteq A$ of $x$ only intercepts $U_{k-1}$ and $U_{k+1}$, which in turn only intercept $B_{k-2}$ through $B_{k+2}$. So, $U_k$ is an open neighborhood of $x$ which intercepts at most the rectangles from $\mathcal{C}_{j-2}$ through $\mathcal{C}_{j+2}$. $\blacksquare$\retTwo
		\end{myIndent}
	\end{myIndent}

	\ul{Theorem 16.3:} Let $\mathcal{A}$ be a collection of open sets in $\mathbb{R}^n$ whose union is $A$. Then there is a partition of unity $\{h_m\}_{m \in \mathbb{N}}$ subordinate to $\mathcal{A}$ such that each $h_m$ is in $C_c^\infty$.

	\begin{myIndent}\exTwoP
		Proof:\\
		Construct $\{Q_m\}_{m \in \mathbb{N}}$ like in the previous lemma. Then note that for each $Q_m$, there is a $C_c^\infty$ function $g_m$ such that $g_m(x) > 0$ if $x \in Q_m^{\circ}$ and $g_m(x) = 0$\\ otherwise. 

		\begin{myIndent}\exPPP
			Specifically: If you define $f(x) = e^{1/x}$ when $x > 0$ and $f(x) = 0$ when $x \leq 0$, then $f \in C^\infty(\mathbb{R})$. In turn, $g \coloneq f(x)f(1-x) \in C^\infty(\mathbb{R})$ with $g(x) > 0$ when $x \in (0, 1)$ and $g(x) = 0$ otherwise.\retTwo
			
			Finally, if $Q_m = [a_1, b_1] \times \ldots \times [a_n, b_n]$, then define:

			{\center $g_m(x_1, \ldots, x_n) = g(\frac{x_1 - a_1}{b_1 - a_1})\cdots g(\frac{x_n - a_n}{b_n - a_n})$ \retTwo\par}

			Thus $g_m \in C^\infty(\mathbb{R}^n)$ with $g_m(x) > 0$ when $x \in Q_m^\circ$ and $g_m(x) = 0$ otherwise.\retTwo
		\end{myIndent}

		Having done that, we've now guarenteed that $\{g_m\}_{m \in \mathbb{N}}$ has all the properties we want except that we don't necessarily have that $\sum_{m \in \mathbb{N}} g_m(x) = 1$ for all $x \in A$. To fix that, we normalize our functions.\retTwo
		
		Let $\lambda(x) \coloneqq \sum_{m=1}^\infty g_m(x)$. Then for any $x \in A$, we know there is an open neighborhood $N_x$ of $x$ that intercepts only finitely many $\supp(g_m)$. It follows then that $\lambda(x) < \infty$ for all $x\in A$ and that $\lambda$ is infinitely differentiable for all $x \in A$. Meanwhile, since any $x \in A$ is in the interior of at least one $Q_m$, we know that $\lambda(x) > 0$ for all $x \in A$.\newpage

		Now for each $m$ define $h_m(x) = g_m(x) / \lambda(x)$ when $x \in A$ and $h_m(x) = 0$ when $x \notin A$. Then it's clear that $\sum_{m \in \mathbb{N}} h_m(x) = 1$ when $x \in A$. Also, we still have that each $h_m \in C^\infty$. To see that, first note that $h_m$ is infinitely differentiable via quotient rule on $A$. Also, since $Q_m$ is compact, $A^\comp$ is closed, and both are disjoint, we know there is some minimum distance $\delta$ between the two sets. So for any $x \in A^\comp$, we know that $h_m$ is just the zero function while on a ball of radius $\delta/2$ around $x$. So, all of the derivatives of $h_m$ exist and equal zero at $x$ for any $x \in A^\comp$. Finally, since $\supp(h_m) = \supp(g_m)$, we have that $h_m$ satisfies our other requirements.\retTwo
	\end{myIndent}
\end{myIndent}

Now returning to our goal of extending the concept of differentiability, we have the following result:\retTwo

\exOne\ul{Lemma 23.1:} Let $S$ be a subset of $\mathbb{R}^k$ and let $f: S \to \mathbb{R}^n$. If for each $x \in S$ there is a neighborhood $U_x$ of $x$ and a $C^r$ function $g_x: U_x \to \mathbb{R}^n$ which agrees with $f$ on $U_x \cap S$, then $f \in C^r(S)$.

\begin{myIndent}\exTwoP
	Proof:\\
	For each $x \in S$, pick a set $U_x$ and a $C^r$ function $g_x : U_x \to \mathbb{R}^n$ as allowed by the hypothesis of the lemma. Then set $\mathcal{A} = \{U_x : x \in S\}$ and call the union of that collection of sets $A$. Via the prior result, there is a partition of unity $\{\phi_m\}_{m \in \mathbb{N}}$ on $A$ consisting of $C^\infty_c$ functions and which is subordinate to $\mathcal{A}$. In turn, for any $m \in \mathbb{N}$ there exists $x_m$ with $\supp(\phi_m) \subseteq U_{x_m}$. It then follows that $h_m \coloneqq \phi_m g_{x_m} \in C^r(U_m)$ and that $h_m$ vanishes outside a compact subset of $U_m$, meaning we can extend $h_m$ to being in $C^r(\mathbb{R}^k)$ by setting $h_m = 0$ outside $U_{x_m}$.\retTwo

	Finally, define $g = \sum_{m=1}^\infty h_m$ on $A$. Then for any $x \in A$, we know that $x$ has a neighborhood on which $g$ is only a sum of finitely many $h_m$. So, $g \in C^r(A)$. Also, if $x \in S$, then for any $m$ with $\phi_m(x) \neq 0$, $h_m(x) = \phi_m(x)g_m(x) = \phi_m(x)f(x)$. Therefore, for any $x \in A \cap S$:

	{\centering $g(x) = \sum\limits_{m \in \mathbb{N}} h_m(x) = f(x)\sum\limits_{m \in \mathbb{N}} \phi_m(x) = f(x)$. $\blacksquare$ \retTwo\par}
\end{myIndent}

\dispDate{7/18/2025}

\hOne Let $H^k = \{(x_1, \ldots, x_k) \in \mathbb{R}^k : x_k \geq 0\}$ denote the \udefine{"upper-Half-space"}. Also, let $H^k_+ = \{(x_1, \ldots, x_k) \in \mathbb{R}^k : x_k > 0\}$ denote the \udefine{"open upper-Half-space"} in $\mathbb{R}^k$.\retTwo

\exOne\ul{Lemma 23.2:} Let $U$ be open in $H^k$ but not in $\mathbb{R}^k$, and let $\alpha: U \to \mathbb{R}^n$ be in $C^r(U)$. That way, there exists a $C^r$ extension $\beta: U^\prime \to \mathbb{R}^n$ of $\alpha$ defined on an open set $U^\prime$ of $\mathbb{R}^k$. Then for $x \in U$, the derivative $\Df \beta(x)$ depends only on the function $\alpha$ and is independent of the extension $\beta$. Hence it follows we may denote this derivative by $\Df \alpha(x)$ without ambiguity.

\begin{myIndent}
	\why We know that $\frac{\partial}{\partial x_k} \beta$ is fully determined by the right-hand limit:

	{\centering $\lim\limits_{h \to 0^+}\frac{\beta(x + he_k) - \beta(x)}{h} = \lim\limits_{h \to 0^+}\frac{\alpha(x + he_k) - \alpha(x)}{h}$ \newpage\par}

	It follows that all first order partial derivatives of $\beta$ on $U$ are uniquely determined by $\alpha$. Then proceeding by induction and noting that $\partial^\gamma \beta$ extends $\partial^\gamma \alpha$ from $U$ to $U^\prime$ for all multi-indices $\gamma$ of degree less than $r$, we can apply the same reasoning to conclude that all partial derivatives of $\beta$ with order less than $r$ on $U$ are uniquely determined by $\alpha$.\retTwo
\end{myIndent}

\hOne With that, we now have the ability to define parametrized manifolds with a\\ boundary by making the domain of our parametrization open with respect to $H^k$\\ instead of $\mathbb{R}^k$. As for our other issue of wanting to talk about manifolds that can't be parametrized by a single function, we shall deal with that now.\retTwo

Let $k > 0$. A \udefine{$k$-manifold} in $\mathbb{R}^n$ of class $C^r$ is a subset $M$ of $\mathbb{R}^n$ having the following property: For each $p \in M$ there is an open set $V$ of $M$ containing $p$, a set $U$ that is open in either $\mathbb{R}^k$ or $H^k$, and a continuous bijective map $\alpha: U \to V$ such that:\\ [-26pt]

\begin{enumerate}
	\item $\alpha \in C^r(U)$\\ [-22pt]
	\item $\alpha^{-1}: V \to U$ is continuous\\ [-22pt]
	\item $\Df \alpha(x)$ has rank $k$ for each $x \in U$.\retTwo
\end{enumerate}

The map $\alpha$ is called a \udefine{coordinate patch} on $M$ about $p$.\retTwo

Also, we call a discrete collection of points in $\mathbb{R}^n$ to be a \udefine{$0$-manifold}.\retTwo

\exOne\ul{Lemma 23.3:} Let $M$ be a manifold in $\mathbb{R}^n$ and $\alpha: U \to V$ be a coordinate patch on $M$. If $U_0$ is a subset of $U$ that is open in $U$, then the restriction of $\alpha$ to $U_0$ is also a coordinate patch on $M$.

\begin{myIndent}\exTwoP
	Proof:\\
	Since $\alpha^{-1}$ is continuous and $U_0$ is open in $U$, we know that $V_0 \coloneqq \alpha(U_0)$ is also open in $V$ and thus also $M$. Hence, $\alpha|_{U_0}$ is a coordinate patch on $M$ because it carries $U_0$ onto $V_0$ in a bijective fashion, and it's a $C^r$ map with a continuous inverse and $\Df (\alpha|_{U_0})$ having rank $k$ just because it's the restriction of $\alpha$ which has all of those things.\retTwo
\end{myIndent}

\pracOne\ul{Exercise 23.3(b):} Why is $\alpha : [0, 1) \to \mathbb{R}^2$ defined by $\alpha(t) = (\cos(2\pi t), \sin(2\pi t))$ not a\\ coordinate patch for the unit circle $S^1$?
\begin{myIndent}\pracTwo
	In this example, $\alpha^{-1}$ is not continuous at $\alpha(0) = (1, 0)$. One way to see this is that the limit of $\alpha^{-1}$ going one way around the circle towards $(1, 0)$ will be $1$, whereas the limit going the other way around the circle will be $0$.\retTwo

	It should be noted though that $S^1$ is still a $1$-manifold. It's just that we need to use multiple overlapping coordinate patches that don't individually go all the way around the circle in order to cover it.\retTwo
\end{myIndent}

\hOne In order to prove the next theorem, I actually need to generalize the version of the inverse function theorem that I learned in 140C so that if $f$ is a bijective $C^r$ map with a nonsingular derivative, then I know that $g = f^{-1}$ is also $C^r$ rather than just merely $C^1$. But to do that, I need to finally learn Cramer's rule.\newpage

I'm also realizing right about now that in my original notes where I defined $\det$ (my MITx notes which I just got back from my parents, yay!), while my construction still generalizes to matrices defined on arbitrary scalar fields perfectly well, it does have a slight problem of using the parity of permutations in it's definition. However, in Math 100A we defined the parity of a permutation by taking the determinant of it's matrix representation. So, I might as well deal with that cyclic definition now\dots\retTwo

\begin{myIndent}\pracOne
	Given a permutation $\sigma: \{1, \ldots, n\} \to \{1, \ldots, n\}$, we can define the function:

	{\centering$\sgn(\sigma) = \prod\limits_{i < j}\sgn(\sigma(j) - \sigma(i))$\retTwo\par}

	Now I claim that $\sgn$ is a group homomorphism from $S_n$ to the multiplicative group $\{-1, 1\} \subseteq \mathbb{R}^n$ To see this, suppose $\sigma \in S_n$ and that $\tau_{k_1,k_2} \in S_n$ is the transposition swapping $k_1$ and $k_2$. Then it's clear that:
	
	{\centering $\sgn(\tau_{k_1, k_2} \circ \sigma) = -\sgn(\sigma) = \sgn(\tau_{k_1,k_2})\sgn(\sigma)$.\retTwo\par}

	In turn, if $\sigma^\prime \in S_n$ is another arbitrary permutation, then by expressing\\ $\sigma^\prime = \tau_1 \circ \tau_2 \circ \ldots \circ \tau_N$ where all the $\tau_i$ are transpositions, we have that:

	{\center\begin{tabular}{l}
		$\sgn(\sigma^\prime \circ \sigma) = \sgn(\tau_1 \circ \tau_2 \circ \ldots \circ \tau_{N-1}\circ\tau_N \circ \sigma)$\\ [4pt]
		$\phantom{\sgn(\sigma^\prime \circ \sigma)} = \sgn(\tau_1)\sgn(\tau_2 \circ \ldots \circ \tau_{N-1} \circ \tau_N \circ \sigma)$\\

		$\phantom{\sgn(\sigma^\prime \circ \sigma)aaaaaaaaaaaaaaaaa} \vdots$\\

		$\phantom{\sgn(\sigma^\prime \circ \sigma)} = \sgn(\tau_1)\sgn(\tau_2)\ldots\sgn(\tau_{N-1})\sgn(\tau_N)\sgn(\sigma)$\\ [4pt]
		$\phantom{\sgn(\sigma^\prime \circ \sigma)} = \sgn(\tau_1)\sgn(\tau_2)\ldots\sgn(\tau_{N-1} \circ \tau_N)\sgn(\sigma)$\\

		$\phantom{\sgn(\sigma^\prime \circ \sigma)aaaaaaaaaaaaaaaaa} \vdots$\\

		$\phantom{\sgn(\sigma^\prime \circ \sigma) } = \sgn(\tau_1)\sgn(\tau_2 \circ \cdots \tau_{N-1} \circ \tau_N)\sgn(\sigma)$\\ [4pt]
		$\phantom{\sgn(\sigma^\prime \circ \sigma) } = \sgn(\tau_1 \circ \tau_2 \circ \cdots \tau_{N-1} \circ \tau_N)\sgn(\sigma) = \sgn(\sigma^\prime)\sgn(\sigma)$\\ [4pt]
	\end{tabular}\retTwo\par}

	Also, it's easily checked that $\sgn(\myId) = 1$. Thus $\sgn$ is a group homomorphism.\\ And, since every transposition has a negative sign, we get the following nice\\ interpretation of the sign of a permutation. Specifically: $\sgn(\sigma) = 1$ if $\sigma$ can only be constructed using an even number of transpositions starting from the identity, and $\sgn(\sigma) = -1$ if $\sigma$ can only be constructed using an odd number of transpositions starting from the identity.\retTwo
\end{myIndent}

Next, here are \udefine{Cramer's rules:}\retTwo

\exOne\ul{Theorem 2.13:} Let $A = \begin{bmatrix}
	a_1  & \cdots & a_n
\end{bmatrix}$ be an $n\times n$ matrix. Also let:

{\centering \phantom{aaaaaaaaaaaaaaa}$x = \begin{bmatrix}
	x_1 \\ \vdots \\ x_n
\end{bmatrix}$ and $c = \begin{bmatrix}
	c_1 \\ \vdots \\ c_n
\end{bmatrix}$.\retTwo\par}

Then if $Ax = c$, we have:

{\centering $\det(A) x_i = \det\left(\begin{bmatrix}a_1 & \cdots & a_{i-1}& c & a_{i+1} & \cdots & a_n\end{bmatrix}\right)$\newpage\par}

\begin{myIndent}\exTwoP
	Proof:\\
	Define $C = \begin{bmatrix}e_1 & \cdots & e_{i-1}& x & e_{i+1} & \cdots & e_n\end{bmatrix}$.\retTwo
	
	Then, we have that $AC = \begin{bmatrix}a_1 & \cdots & a_{i-1}& c & a_{i+1} & \cdots & a_n\end{bmatrix}$. Thus, we know:

	{\centering $\det(A)\det(C) = \det\left(\begin{bmatrix}a_1 & \cdots & a_{i-1}& c & a_{i+1} & \cdots & a_n\end{bmatrix}\right)$. \retTwo\par}

	Also note that $\det(C) = x_{i}(-1)^{i + i}\det(I_{n-1}) = x_i(-1)^{2i} = x_i$. The desired\\ conclusion then follows.\retTwo
\end{myIndent}

\ul{Theorem 2.14:} Let $A$ be an $n \times n$ matrix of rank $n$ and let $B = A^{-1} = [b_{i,j}]$. Then\\ letting $A_{j,i}$ denote the matrix which results from removing the $j$th row and $i$th\\ column of $A$, we have that:

{\centering $b_{i,j} = \frac{(-1)^{j+i}\det(A_{j,i})}{\det(A)}$ \retTwo\par}

\begin{myIndent}\exTwoP
	Proof:\\ [-4pt]
	After fixing $j$, set $x = \left[\begin{smallmatrix}x_1 \\ \vdots \\ x_n\end{smallmatrix}\right]$ equal to the $j$th column of $B$.\retTwo

	Now since $AB = I_{n}$, we know $Ax = e_j$. Therefore, by our last theorem we know:

	{\centering $\det(A)x_i = \det\left(\begin{bmatrix}a_1 & \cdots & a_{i-1}& e_j & a_{i+1} & \cdots & a_n\end{bmatrix}\right)$ \retTwo\par}

	The latter determinant equals $(-1)^{j+i}\det(A_{j,i})$. Therefore:

	{\center $x_i = b_{i,j} = \frac{(-1)^{j+i}\det(A_{j,i})}{\det(A)}$. $\blacksquare$ \retTwo\par}
\end{myIndent}

\hOne Now, here's the generalization of the inverse function theorem, that if $f \in C^r$ and satisfies all the other hypotheses of the inverse function theorem, then we can guarentee that our inverse function $g = f^{-1}$ is also $C^r$.

\begin{myIndent}\exTwoP
	Proof:\\
	Jumping to where we ended our proof of the inverse function theorem in math 140C, we had shown that $\Df g(y) = (\Df f(g(y)))^{-1}$ for all $y$ in some open subset of the image of $f$. If $g(y) = (g_1(y), \ldots, g_n(y))$, then by our prior theorem we know that:

	{\centering $\frac{\partial}{\partial y_j}g_i(y) = \frac{(-1)^{j + i}\det([\Df f(g(y))]_{j,i})}{\det(\Df f(g(y)))}$ \retTwo\par}

	Now we already know that $g$ is $C^1$. So, when proceeding by induction for $r > 1$, it suffices to assume $g$ is also $C^{r-1}$. But then note that since $\det(\Df f(g(y))) \neq 0$ for all $y$, and since all the partial derivatives of $f$ are $C^{r-1}$, our above expression shows that $\frac{\partial}{\partial y_j}g_i(y)$ is also $C^{r-1}$. And since this works for all partial derivatives of $g$, we've\\ [-3pt] proven that $g$ is $C^{(r-1)+1}$. $\blacksquare$\retTwo
\end{myIndent}

And finally to finish off for tonight\ldots\newpage

\exOne\ul{Theorem 24.1} Let $M$ be a $k$-manifold in $\mathbb{R}^n$ of class $C^r$. Also let $a_0: U_0 \to V_0$ and $a_1: U_1 \to V_1$ be coordinate patches on $M$ with $W = V_0 \cap V_1$ nonempty, and let $W_i = \alpha_i^{-1}(W)$. Then the map $(\alpha_1^{-1} \circ \alpha_0): W_0 \to W_1$ is a $C^r$ map with a nonsingular derivative.

\begin{myIndent}
	(Side note: We often call $\alpha_1^{-1} \circ \alpha_0$ the \udefine{transition function} between the\\ coordinate patches $\alpha_0$ and $\alpha_1$.)\retTwo

	\exTwoP Proof:\\
	It suffices to show that if $\alpha: U \to V$ is a coordinate patch on $M$, then $\alpha^{-1} : V \to \mathbb{R}^k$ is in $C^r(V)$.
	\begin{myIndent}
		\why If $\alpha_0$ and $\alpha_1^{-1}$ are both of class $C^r$, then so is their composite $\alpha_1^{-1} \circ \alpha_0$. By similar reasoning, we also know that $\alpha_0^{-1} \circ \alpha_1: W_1 \to W_0$ is in $C^r(W_0)$. And since $(\alpha_1^{-1} \circ \alpha_0)$ and $\alpha_0^{-1} \circ \alpha_1$ are inverses of each other, we know by chain rule that for any $x \in W_0$ and $y = (\alpha_1^{-1} \circ \alpha_0)(x)$:

		{\centering $\Df(\alpha_0^{-1} \circ \alpha_1)(y)\Df(\alpha_1^{-1} \circ \alpha_0)(x) = \mathbf{1}$ \retTwo\par}

		The only way this is possible is if $\det(\Df(\alpha_1^{-1} \circ \alpha_0)) \neq 0$ for all $x \in W_0$.\retTwo
	\end{myIndent}

	Next, to prove that $\alpha^{-1}$ is of class $C^r$, it suffices to show that it is locally of class $C^r$. So let $p_0$ be a point of $V$ and set $x_0 = \alpha^{-1}(p_0)$.\retTwo

	First consider the case $U$ is open in $H^k$ but not in $\mathbb{R}^k$. Then, we can extend $\alpha$ to a $C^r$ map $\beta$ on an open set $U^\prime$ of $\mathbb{R}^k$. Now $\Df \alpha(x_0)$ has rank $k$. So after some suitable permutation of our standard basis vectors, we can assume the first $k$ rows of the matrix $\Df \alpha(x_0)$ are linearly independent. If we then define $\pi: \mathbb{R}^n \to \mathbb{R}^k$ to be the projection from $\mathbb{R}^n$ onto those first $k$ basis vector coordinates, we have that the map $g = \pi \circ \beta$ maps $U^\prime$ into $\mathbb{R}^k$ and $\Df g(x_0)$ is non-singular. So by the inverse function theorem, we know $g$ is a $C^r$ diffeomorphism on an open set $W$ of $\mathbb{R}^k$ about $x_0$ (meaning $g$ is $C^r$, $g$ has an inverse $g^{-1}$, and $g^{-1}$ is $C^r$).\retTwo

	Now we claim $h = g^{-1} \circ \pi$ (which is a $C^r$ map) extends $\alpha^{-1}$ to a neighborhood $A$ of $p_0$. Firstly note $U_0 \coloneq W \cap U$ is open in $U$. Hence, $\alpha^{-1}$ being continuous implies that $V_0 \coloneq \alpha(U_0)$ is open in $V$. This means there is an open set $A \subseteq \mathbb{R}^n$ such that $A \cap V = V_0$. By intercepting $A$ with $\pi^{-1}(g(W))$ ($= \beta(W)$), we can force $A$ to be contained in the domain of $h$.\retTwo

	Now $h: A \to \mathbb{R}^k$ is of class $C^r$, and if $p \in A \cap V = V_0$, then when letting\\ $x = \alpha^{-1}(p)$ we have:

	{\centering $h(p) = h(\alpha(x)) = g^{-1}(\pi(\alpha(x))) = g^{-1}(g(x)) = x = \alpha^{-1}(p)$.\retTwo\par}

	As for the case where $U$ is open in $\mathbb{R}^k$, then just set $U^\prime = U$ and $\beta = \alpha$ and the prior reasoning still works. $\blacksquare$

	\begin{myDindent}\pracTwo
		Side note: As a corollary, we now know that if two coordinate patches parametrize the same manifold, then the domains of those two coordinate patches are diffeomorphic. So hopefully that adds to the significance of theorem I wrote at the bottom of page 79. \newpage 
	\end{myDindent}
\end{myIndent}

\dispDate{7/19/2025}

\hOne Today I shall formalize what I mean by the "boundary" of a manifold and maybe also do some cool exercises. I'll try to get as much done as possible since I'm going to be busy at San Diego pride tomorrow.\retTwo

Let $M$ be a $k$-manifold in $\mathbb{R}^n$ and let $p \in M$. If there is a coordinate patch\\ $\alpha: U \to V$ on $M$ about $p$ such that $U$ is open in $\mathbb{R}^k$, we say $p$ is an \udefine{interior point}\\ of $M$. Otherwise, we say $p$ is a \udefine{boundary point} of $M$.\retTwo

We denote the set of boundary points of $M$ as $\partial M$ and call it the \udefine{boundary} of $M$. Meanwhile, we call $M - \partial M$ the \udefine{interior} of $M$. Note that these definitions are distinct from the topological definitions of boundaries and interiors.
\begin{myIndent}
	\pracTwo Untill I get bored of this and go back to Folland or someone else, I'll be using $\partial M$ to refer to the manifold definition of boundary as opposed to a different definition.\retTwo
\end{myIndent}

\exOne\ul{Lemma 24.2:} Let $M$ be a $k$-manifold in $\mathbb{R}^n$ and $\alpha:U \to V$ be a coordinate patch about the point $p$ of $M$.\\ [-24pt]
\begin{enumerate}
	\item[(a)] If $U$ is open in $\mathbb{R}^k$, then $p$ is an interior point of $M$.\\ [-22pt]
	\item[(b)] If $U$ is open in $H^k$ and $p = \alpha(x_0)$ where $x_0 \in H^k_+$, then $p$ is an interior point\\ of $M$.\\ [-22pt]
	\item[(c)] If $U$ is open in $H^k$ and $p = \alpha(x_0)$ where $x_0 \in \mathbb{R}^{k-1} \times 0$, then $p$ is a boundary point of $M$.\\ [-22pt]
\end{enumerate}

\begin{myIndent}\exTwoP
	Proof:\\
	Parts (a) and (b) are trivial. As for part (c), let $\alpha_0: U_0 \to V_0$ be the coordinate patch in the hypothesis of the lemma and suppose (for the sake of contradiction) that there is another coordinate patch $\alpha_1: U_1 \to V_1$ about $p$ with $U_1$ open in $\mathbb{R}^k$.\retTwo

	Since $V_0$ and $V_1$ are open in $M$, the set $W = V_0 \cap V_1$ is also open in $M$. Let\\ $W_i = \alpha_i^{-1}(W)$ for $i = 0, 1$. Then $W_0$ is open in $H^k$ and contains $x_0$ (which\\ consequently means $W_0$ isn't open in $\mathbb{R}^k$). Also, $W_1$ is open in $\mathbb{R}^k$. But now note that by our prior theorem, $\alpha_0^{-1} \circ \alpha_1$ is a $C^r$ map from $W_1$ to $W_0 \subseteq \mathbb{R}^k$ with a nonsingular derivative matrix. So, by specifically part (A) of the inverse function theorem (as covered in math 140C), we know $W_1$ maps to an open set in $\mathbb{R}^k$. Yet $\alpha_0^{-1} \circ \alpha_1(W_1) = W_0$ is not open in $\mathbb{R}^k$. Hence, a contradiction. $\blacksquare$
	\begin{myIndent}\exPPP
		Side note: Holy fuck I did not realize before now that in math 140C we proved that $C^1$ functions to $\mathbb{R}^k$ with a nonsingular derivative matrix are open maps.\retTwo
	\end{myIndent}
\end{myIndent}

\hOne Note, we trivially have that $H^k$ is a $k$-manifold of class $C^\infty$ in $\mathbb{R}^k$ (just define the coordinate patch to be the identity map on $H^k$.) Then, $\partial H^k = \mathbb{R}^{k-1} \times 0$ by the prior lemma.\newpage

\exOne\ul{Theorem 24.3:} Let $M$ be a $k$-manifold in $\mathbb{R}^n$ of class $C^r$. If $\partial M$ is nonempty, then $\partial M$ is a $k-1$ manifld without boundary in $\mathbb{R}^n$ of class $C^r$.

\begin{myIndent}\exTwoP
	Proof:\\
	Let $p \in \partial M$, and then let $\alpha: U \to V$ be a coordinate patch on $M$ about $p$. Then $U$ is open in $H^k$ and $p = \alpha(x_0)$ for some $x_0 \in \partial H^k$. By the prior lemma, $\alpha(x) \in \partial M$ for all $x \in U \cap \partial H^k$ and $\alpha(x) \notin \partial M$ for all $x \in U - \partial H^k$. Thus, we know that the restriction of $\alpha$ to $U \cap \partial H^k$ is a bijective map onto the open set $V_0 \coloneq V \cap \partial M$ of $\partial M$.\retTwo

	Now let $U_0$ be the open set in $\mathbb{R}^{k-1}$ such that $U_0 \times 0 = U \cap \partial H^k$. Then for any $x \in U_0$, define $\alpha_0(x) = \alpha(x, 0)$. Thus, $\alpha_0: U_0 \to V_0$ is a coordinate patch on $\partial M$ about $p$.
	\begin{itemize}
		\item It is $C^r$ because so is $\alpha$.
		\item $\Df \alpha_0(x)$ has rank $k - 1$ for all $x$ since $\Df \alpha_0$ just consists of the first $k - 1$ columns of $\Df \alpha (x, 0)$.
		\item Finally, $\alpha_0^{-1}$ is continuous because it equals the composition of $\alpha^{-1}$ restricted to the set $V_0$ followed by the projection of $\mathbb{R}^k$ onto it's first $(k-1)$-coordinates (and both of those functions are continuous).\retTwo
	\end{itemize} 

	This proves $\partial M$ is a manifold. Also, this shows that $p$ is an interior point of $\partial M$. So, $\partial M$ has no boundary.\retTwo
\end{myIndent}

\ul{Theorem 24.4:} Let $\mathcal{O}$ be an open set in $\mathbb{R}^n$, and let $f: \mathcal{O} \to \mathbb{R}$ be a $C^r$ map. Also let $M$ be the set of points for which $f(x) = 0$ and $N$ be the set of points for which $f(x) \geq 0$. If $M \neq \emptyset$ and $\Df f(x)$ has rank $1$ for all $x$ in $M$, then $N$ is an $n$-manifold in $\mathbb{R}^n$ and $M = \partial N$.

\begin{myIndent}\exTwoP
	\begin{myDindent}\exPPP 
		Consequently, a level set of $f$ is a manifold so long as $f$ has no critical points in that level set.\retTwo
	\end{myDindent}

	Proof:\\
	Firstly, suppose $p \in N$ with $f(p) > 0$. Then if  $\alpha$ is the identity map on the set $U \coloneq f^{-1}((0, \infty))$, we have that $\alpha$ is a $C^\infty$ bijective map from the open set $U$ in $\mathbb{R}^n$ to itself such that $\alpha$ has a continuous inverse and a full rank derivative matrix. So, $\alpha$ is a coordinate patch on $N$.\retTwo

	Meanwhile, suppose $f(p) = 0$. Then since $\Df f(p) \neq 0$, at least one partial\\ derivative $\frac{\partial}{\partial x_i} f(p)$ is nonzero. By a sufficient permutation of our standard basis\\ vectors, we can assume $i = n$. So, define $F: \mathcal{O} \to \mathbb{R}^n$ by the equation\\ $F(x) = (x_1, \ldots, x_{n-1}, f(x))$. Thus, $F$ is a $C^r$ map with a nonsingular derivative matrix at $p$ since:

	{\centering $\Df F = \begin{bmatrix}
	I_{n-1} & 0 \\ * & \frac{\partial}{\partial x_n} f \end{bmatrix}$ \newpage\par}

	By the inverse function theorem, we know $F$ is a $C^r$ diffeomorphism from an open neighborhood $V$ of $p$ in $\mathbb{R}^n$ to an open set $U$ of $\mathbb{R}^n$. Furthermore, $F$ carries the open set $V \cap N$ of $N$ onto the open set $U \cap H^n$ of $H^n$. Therefore, $F^{-1}|_{(U \cap H^n)}$ works as our coordinate patch on $N$ about $p$.\retTwo
	
	Finally note that $F(p) \in \partial H^n$. This shows that $M = \partial N$. $\blacksquare$\retTwo
\end{myIndent}

\ul{Corollary 24.5:} The $n$-ball $B^n(a) \coloneqq \{x : \|x\|_2 \leq a\}$ is a $C^\infty$ $n$-manifold whose\\ boundary is $S^{n-1} \coloneqq \{x : \|x\|_2 = a\}$.

\begin{myIndent}\exTwoP
	Proof:\\
	Consider the function $f(x) = a^2 - (\|x\|_2)^2$.\retTwo
\end{myIndent}

\hOne The next exercise gives us an important tool for constructing manifolds (which makes it kinda shocking that Munkres leaves it as an exercise).


\pracOne\ul{Exercise 24.2} Let $f: \mathbb{R}^{n + k} \to \mathbb{R}^n$ be of class $C^r$. Let $M$ be the set of all $x$ such that\\ $f(x) = 0$. Assume that $M$ is nonempty and $\Df f(x)$ has rank $n$ for all $x \in M$. Then $M$ is a $k$-manifold in $\mathbb{R}^{n+k}$ without boundary. Furthermore, if $N$ is the set of all $x$ such that $f_1(x) = \cdots = f_{n-1}(x) = 0$ and $f_n(x) \geq 0$, and the matrix $\partial(f_1, \ldots, f_{n-1})/\partial x$ has rank $n-1$ at each point of $N$, then $N$ is a $k+1$ manifold and $M = \partial N$.\\ [-4pt] 

\begin{myIndent}\pracTwo 
	\ul{Lemma:} Let $m \leq n$ and suppose $f: \mathbb{R}^{n + k} \to \mathbb{R}^m$ is a $C^r$ map such that the matrix $\Df f$ has rank $m$ at the point $p$. Then there are open sets $U, V \subseteq \mathbb{R}^{n + k}$ with $p \in V$, as well as a $C^r$ diffeomorphism $G: U \to V$ with rank $n + k$ satisfying that $f \circ G(x) = \pi_m(x)$ where $\pi_m$ is a projection from $\mathbb{R}^{n + k}$ to $m$ of it's coordinates (and by applying a suitable permutation of our basis vectors, we can assume that those coordinates are the first $m$ coordinates).

	\begin{myIndent}
		Proof:\\
		Since $\Df f$ has rank $m$ at $p$, we know that the derivative matrix has $m$ linearly\\ independent columns at $p$, and by a permutation of our standard bases, we can\\ assume those $m$ columns are the first $m$ columns. Therefore, it makes sense to adopt the notation of writing $x = (x^{(1)}, x^{(2)})$ in $\mathbb{R}^{n+k}$ where $x^{(1)} \in \mathbb{R}^{m}$ and $x^{(2)} \in \mathbb{R}^{n+k-m}$. Also, it makes sense to define the projection $\pi_m(x) = x^{(1)}$.\retTwo

		Now define the function $F: \mathbb{R}^{n+k} \to \mathbb{R}^{n+k}$ by:
		
		{\centering $F(x) = (f_1(x), \ldots, f_{m}(x), x^{(2)})$.\retTwo\par}

		Then $F$ is a $C^r$ map with the derivative matrix:

		{\centering $\Df F= \begin{bmatrix}
		\partial(f_1, \ldots, f_{m})/\partial x^{(1)} & \partial(f_1, \ldots, f_{m})/\partial x^{(2)} \\
		0 & I_{n+k-m}\end{bmatrix}$ \retTwo\par}

		Since $\Df F$ is full rank at $p$, by the inverse function theorem there are open sets\\ $U, V \subseteq \mathbb{R}^{n+k}$ with $p \in V$ such that $F$ is a $C^r$ diffeomorphism from $V$ to $U$ with full rank. Taking $G \coloneqq F^{-1}$, we get:

		{\centering $f \circ G(x) = (\pi_m \circ F) \circ G(x) = \pi_m \circ (F \circ F^{-1})(x) = \pi_m(x) = x^{(1)}$ \newpage\par}
	\end{myIndent}

	Given $f = (f_1, \ldots, f_{n-1}, f_n)$, we'll denote $\tilde{f} = (f_1, \ldots, f_{n-1})$. Now, the rest of the exercise just involves applying the prior lemma three times.

	\begin{enumerate}
		\item[1.] Suppose $p \in N - M$ (meaning $f_n(p) > 0$ and $\tilde{f}(p) = 0$). Then using our prior lemma, there are open sets $U, V \subseteq \mathbb{R}^{n+k}$ with $p \in V$ as well as a $C^r$ diffeomorphism $\tilde{G}: U \to V$ with full rank such that $\tilde{f} \circ \tilde{G}(x) = \pi_{n-1}(x)$. Using the continuity of $f_n$, we can assume $f_n(x) > 0$ for all $x \in V$ by making $V$ sufficiently small.\retTwo
		
		Now note that $V \cap N$ is an open subset of $N$ containing $p$ such that:
	
		{\centering\begin{tabular}{l}
		$V \cap N = \tilde{G}(U) \cap \tilde{f}^{-1}(\{0\}) \cap \{x : f_n(x) > 0\}$\\ [3pt]
		$\phantom{V \cap N} = \tilde{G}(U) \cap (\pi_{n-1} \circ \tilde{G}^{-1})^{-1}(\{0\}) \cap \{x : f_n(x) > 0\}$\\ [3pt]
		$\phantom{V \cap N} = \tilde{G}(U) \cap \tilde{G}(\pi_{n-1}^{-1}(\{0\})) \cap \{x : f_n(x) > 0\}$\\ [3pt]
		$\phantom{V \cap N} = \tilde{G}\left(U \cap (0^{n-1} \times \mathbb{R}^{k+1})\right) \cap \{x : f_n(x) > 0\}$\\ [3pt]
		$\phantom{V \cap N} = \tilde{G}\left(U \cap (0^{n-1} \times \mathbb{R}^{k+1}) \cap \tilde{G}^{-1}(\{x : f_n(x) > 0\})\right)$\\ [3pt]
		\end{tabular}\retTwo\par}

		Then since $U \cap \tilde{G}^{-1}(\{x : f_n(x) > 0\})$ is open in $\mathbb{R}^{n+k}$, we can deduce that there is\\ [-1pt] an open set $A \subseteq \mathbb{R}^{k+1}$ with $0^{n-1} \times A = U \cap (0^{n-1} \times \mathbb{R}^{k+1}) \cap \tilde{G}^{-1}(\{x : f_n(x) > 0\})$.\\ [-2pt] And by defining $\alpha(x_1, \ldots, x_{k+1}) = \tilde{G}(0^{n-1}, x_1, \ldots, x_{k+1})$, we have that $\alpha$ is a\\ [0pt] bijective $C^r$ map from the open set $A$ of $\mathbb{R}^{k+1}$ to $V \cap N$ with rank $k+1$ and a continuous inverse. Hence, $\alpha$ is\\ [0pt] a coordinate patch on $N$ about $p$.\retTwo

		\item[2.] Suppose $p \in M$ and only assume the part of the exercise statement that comes before the word "furthermore". Then let $U, V \subseteq \mathbb{R}^{n+k}$ be open sets with $p \in V$, and let $G: U \to V$ be a $C^r$ diffeomorphism with full rank satisfying that $f \circ G(x) = \pi_n(x)$. Then $V \cap M$ is an open subset of $M$ containing $p$ such that:
		
		{\centering\begin{tabular}{l}
		$V \cap M = G(U) \cap f^{-1}(\{0\})$\\ [3pt]
		$\phantom{V \cap M} = G(U) \cap (\pi_n \circ G^{-1})^{-1}(\{0\}) = G(U) \cap G(\pi_n^{-1}(\{0\}))$\\ [3pt]
		$\phantom{V \cap M = G(U) \cap (\pi_n \circ G^{-1})^{-1}(\{0\})} = G(U \cap (0^n \times \mathbb{R}^{k}))$\\ [3pt]
		\end{tabular}\retTwo\par}

		Now we know there is some open set $A \subseteq \mathbb{R}^k$ such that $0^n \times A = U \cap (0^n \times \mathbb{R}^k)$.\\ Therefore, by defining $\alpha(x_1, \ldots, x_{k}) = \tilde{G}(0^{n}, x_1, \ldots, x_{k})$, we have that $\alpha$ is a\\ bijective $C^r$ map from the open set $A$ of $\mathbb{R}^{k+1}$ to $V \cap M$ with rank $k$ and a continuous inverse. Hence, $\alpha$ is a coordinate patch on $M$ about $p$.\retTwo

		\item[3.] Finally, suppose $p \in M$ and this time assume the entire hypothesis of the exercise.\\ Also let $G: U \to V$ be as in the prior part. Now:
		
		{\centering\begin{tabular}{l}
		$V \cap N = G(U) \cap f^{-1}(\{0\}) = G(U) \cap G(\pi_n^{-1}(0^{n-1} \times [0, \infty)))$\\ [3pt]
		$\phantom{V \cap N = G(U) \cap f^{-1}(\{0\})} = G(U \cap (0^{n-1} \times [0, \infty) \times \mathbb{R}^{k}))$\\ [3pt]
		\end{tabular}\retTwo\par}

		Now if $\tau$ is the function permuting the first and $(k + 1)$th basis vectors, then we know there is some open set $A \subseteq H^{k+1}$ such that $0^{n-1} \times \tau(A) = U \cap (0^{n-1} \times [0, \infty) \times \mathbb{R}^{k})$. So, define $\alpha(x_1, \ldots, x_{k+1}) = G(0^{n-1}, x_{k+1}, x_1, \ldots, x_k)$. Then $\alpha$ is a bijective $C^r$ map from the open set $A$ of $H^{k+1}$ to $V \cap N$ with rank $k+1$ and a continuous inverse. Hence, $\alpha$ is a coordinate patch on $N$ about $p$.\newpage

		Also, if $x \in U$ satisfies that $\alpha(x) = p$, then since $f \circ G = \pi_n$, we know that $f(G(x)) = f(p) = 0$. So $x \in \partial H^{k+1}$. This proves $p$ is on the boundary of $N$. $\blacksquare$\retTwo

		\begin{myIndent}\myComment
			Note from 8/28/2025: How the fuck did I just realize this is just implicit function theorem.\retTwo
		\end{myIndent}
	\end{enumerate}
\end{myIndent}

\hOne\dispDate{7/28/2025}

Here's a fun application of the prior exercise.\retTwo

\pracOne

\ul{Exercise 24.6:} Let $\mathcal{O}(n)$ denote the set of all orthogonal $n$ by $n$ matrices, considered as a subspace of $\mathbb{R}^N$ where $N = n^2$. Show that $\mathcal{O}(n)$ is a compact $\binom{n}{2}$-manifold of class $C^\infty$ in\\ [-1pt] $\mathbb{R}^N$ without boundary.
\begin{myIndent}\pracTwo
	Firstly, define $F: \mathbb{R}^N \to \mathbb{R}^N$ by $F(\mMat{A}) = \mMat{A}\mMat{A}^T - I_n$. Also define $f: \mathbb{R}^N \to \mathbb{R}^{\frac{1}{2}n(n+1)}$ such that $f(\mMat{A})$ is the vector containing just the elements of $F(\mMat{A})$ on or above the main diagonal. Then it is fairly clear that $F(\mMat{A}) = 0$ if and only if $f(\mMat{A}) = 0$ if and only if $\mMat{A} \in \mathcal{O}(n)$.\retTwo

	Next note that if $\mMat{A} = [x_{i,j}]$, then the $(i,j)$th. column of the derivative matrix $\Df F$ is:
	
	{\centering \begin{tabular}{l}
		$\frac{\partial}{\partial x_{i,j}}F(\mMat{A}) = \left(\frac{\partial}{\partial x_{i,j}}\mMat{A}\right)\mMat{A}^T + \mMat{A}\left(\frac{\partial}{\partial x_{i,j}}\mMat{A}\right)^T$\\ [12pt]

		$\phantom{\frac{\partial}{\partial x_{i,j}}F(A)} = \begin{bmatrix} &
		\mMat{0}^{(i-1)\times n} & \\ &&\\ x_{1,j} & \cdots & x_{n,j}\\ & & \\ &
		\mMat{0}^{(n-i)\times n} &
		\end{bmatrix} + \begin{bmatrix} &
		\mMat{0}^{(i-1)\times n} & \\ &&\\ x_{1,j} & \cdots & x_{n,j}\\ & & \\ &
		\mMat{0}^{(n-i)\times n} &
		\end{bmatrix}^T$\\ [35pt]

		$\phantom{\frac{\partial}{\partial x_{i,j}}F(A)} = \begin{bmatrix} 
			& & x_{1,j} & & \\
			& & \vdots &  & \\
			x_{1,j} & \cdots & 2x_{i,j} & \cdots & x_{n,j}\\
			& & \vdots & & \\
			& & x_{n,j} & & \\

		\end{bmatrix}$\\ [12pt]
	\end{tabular}\retTwo\par}

	Meanwhile, the $(i,j)$th. column of $\Df f$ is just the vector with the elements of $\Df F$ on or above the diagonal. From this, it is clear that $f$ and $F$ are $C^\infty$ maps.\retTwo

	Taking a different view, if $k \leq \ell$, then the $(k, \ell)$th. row of $\Df f$ is:

	{\centering $\left(\left[ \begin{matrix}
		\\ x_{\ell, 1} & \cdots & x_{\ell,n} \\ \phantom{}
	\end{matrix}\middle]\begin{matrix}
		\\ (k\text{th. row}) \\\phantom{}
	\end{matrix}\right.\right)
	+ \left(\left[ \begin{matrix}
		\\ x_{k, 1} & \cdots & x_{k,n} \\ \phantom{}
	\end{matrix}\middle]\begin{matrix}
		\\ (\ell\text{th. row}) \\\phantom{}
	\end{matrix}\right.\right)$ \retTwo\par}
	
	Now let $v_{k,\ell}$ denote the $(k,\ell)$th. row of $\Df f$; let $r_i$ denote the $i$th. row of $\mMat{A}$; and let $\delta_{i,j}(x, y)$ equal $1$ if $(x, y) = (i, j)$ and equal $0$ otherwise. Then supposing $k_1 \leq \ell_1$ and $k_2 \leq \ell_2$, we have that:

	{\centering $v_{k_1,\ell_1} \cdot v_{k_2,\ell_2} = \delta_{k_1, k_2}(r_{\ell_1} \cdot r_{\ell_2}) + \delta_{k_1, \ell_2}(r_{\ell_1} \cdot r_{k_2}) + \delta_{\ell_1, k_2}(r_{k_1} \cdot r_{\ell_2}) + \delta_{\ell_1, \ell_2}(r_{k_1} \cdot r_{k_2})$ \newpage\par}

	If $\mMat{A}$ is orthogonal, then this simplifies to:

	{\centering\begin{tabular}{l}
	$v_{k_1,\ell_1} \cdot v_{k_2,\ell_2} = \delta_{k_1, k_2}\delta_{\ell_1, \ell_2} + \delta_{k_1, \ell_2}\delta_{\ell_1, k_2} + \delta_{\ell_1, k_2}\delta_{k_1, \ell_2} + \delta_{\ell_1, \ell_2}\delta_{k_1, k_2}$\\ [6pt]
	$\phantom{v_{k_1,\ell_1} \cdot v_{k_2,\ell_2}} = 2\delta_{k_1, k_2}\delta_{\ell_1, \ell_2} + 2\delta_{k_1, \ell_2}\delta_{\ell_1, k_2}$\\ [6pt]
	\end{tabular} \retTwo\par}

	There are two cases where this dot product will be nonzero. The first case is if $k_1 = k_2$ and $\ell_1 = \ell_2$. Meanwhile, the second case is if $k_1 = \ell_2$ and $\ell_1 = k_2$. However, since we are requiring that $k_1 \leq \ell_1$ and $k_2 \leq \ell_2$, the second case actually implies the first case.\retTwo

	This proves that the rows of $\Df f$ actually form an orthogonal set of vectors in $\mathbb{R}^N$. Hence, $f$ has full rank on $\mathcal{O}(n)$.\retTwo

	It now follows from the previous exercise that $\mathcal{O}(n)$ is a manifold without boundary in $\mathbb{R}^N$. It's dimension will be $n^2 - \frac{(n+1)n}{2} = \frac{n^2}{2} - \frac{n}{2} = \binom{n}{2}$. Meanwhile, to see that the manifold is compact, note firstly that it is bounded by the set $\{x \in \mathbb{R}^N : \|x\|_\infty \leq 1\}$. Also, the points in the manifold are given by the set $f^{-1}(\{0\})$, and that set is closed since $f$ continuous and $\{0\}$. $\blacksquare$\retTwo
\end{myIndent}

\dispDate{7/29/2025}

\hOne I'm gonna finish the current chapter of Munkres and then switch to a different book to learn about differential forms. For today, my agenda is to define integration of scalar-valued functions on general manifolds in $\mathbb{R}^n$.\retTwo

Let $M$ be a $k$-manifold of class $C^r$ in $\mathbb{R}^n$ (with $r \geq 1$ and $k \leq n$). Then recall that we already defined integration on $M$ if $M$ is parametrized by a single coordinate patch from an open set of $\mathbb{R}^k$. Hopefully, it's also clear to see that our previous definition works if our coordinate patch is from an open set of $H^k$. Also, by theorem 24.1 plus the theorem at the bottom of page 79 of this journal, we now know that our definition of the integral is independent of the parametrization we use.

\begin{myIndent}\pracOne
	Note from 7/31/2025: Actually I haven't yet showed that the parametrization is independent if the domain of that parametrization is open in $H^k$ but not $\mathbb{R}^k$.\retTwo
\end{myIndent}

In general though, $M$ probably can't be parametrized by just one coordinate patch. So, we instead bodge our definition using a partition of unity.

\begin{myIndent}
	\exTwo\ul{Lemma:} There is a countable set $\{\alpha_n: U_n \to V_n\}_{n \in \mathbb{N}}$ of coordinate patches on $M$ such that $\bigcup_{n \in \mathbb{N}} V_n = M$.

	\begin{myIndent}\exTwoP
		Proof:
		\begin{itemize} 
			\item[1.] Note that topologically speaking, $M$ is a second countable LCH space. 

			\begin{myIndent}\exPPP
				The fact that $M$ is second countable and Hausdorff is just a consequence of the fact that $\mathbb{R}^n$ is both of those things and $M$ is equipped with the subspace topology.\newpage

				To show that $M$ is locally compact, note that if $p \in M$, then there is a homeomorphism $\alpha$ from some open set $U$ in $\mathbb{R}^k$ or $H^k$ to an open set $V \subseteq M$ containing $p$. Then given the $x \in U$ satisfying that $\alpha(x) = p$, there is a compact set $K \subseteq U$ with $x \in K$.
				\begin{myIndent}\pracTwo
					If $U$ is open in $\mathbb{R}^k$, then it's obvious that $K$ exists. Meanwhile, if $U$ is open in $H^k$, then consider picking a $U^\prime$ which is open in $\mathbb{R}^k$ and satisfies that $U^\prime \cap H^k = U$. Then we know there is a compact set $K^\prime$ such that $x \in K^\prime \subseteq U^\prime$. And since $H^k$ is closed, we can set $K = K^\prime \cap H^k$ and know $K$ is compact.\retTwo
				\end{myIndent}

				Next, by Urysohn's lemma, there is a precompact open set $V$ such that $K \subseteq V \subseteq \overline{V} \subseteq U$. Hence, we have that $\alpha(\overline{V})$ is a compact subset of $M$ containing $p$. Also $\alpha(V)$ is an open subset of $\alpha(\overline{V})$ which contains $p$. Hence, $\alpha(\overline{V})$ is a compact neighborhood of $p$.\retTwo
			\end{myIndent}

		\item[2.] In turn, we know that $M$ is $\sigma$-compact. From that hopefully it is obvious how we can get a countable covering of coordinate patches over $M$. $\blacksquare$\retTwo
	\end{itemize}
	\end{myIndent}

	\ul{Lemma:} Let $\{\alpha_a: U_a \to V_a\}_{a \in A}$ be a collection of coordinate patches on $M$ such that $\bigcup_{a \in A} V_a = M$. Then there exists a countable collection of $C^\infty$ functions\\ $\{\phi_i: \mathbb{R}^n \to \mathbb{R}\}_{i \in \mathbb{N}}$ satisfying that:
	\begin{itemize}
		\item $\phi_i(x) \geq 0$ for all $x \in \mathbb{R}^n$ and $i \in \mathbb{N}$.
		\item Each $p \in M$ has a neighborhood in $M$ on which only finitely many $\phi_i$ are\\ nonzero.
		\item $\sum_{i \in \mathbb{N}} \phi_i(p) = 1$ for all $p \in M$.
		\item For each $i \in \mathbb{N}$, there is some $a \in A$ such that $\supp(\phi_i) \cap M \subseteq V_{a}$.
	\end{itemize}

	\begin{myIndent}\exTwoP
		\begin{myIndent}\exPPP
			In the future I will just refer to $\{\phi_i\}_{i \in \mathbb{N}}$ as being a partition of unity on $M$\\ subordinate to our collection of coordinate patches.\retTwo
		\end{myIndent}

		Proof:\\
		For each coordinate patch $\alpha_a$, choose an open set $V^\prime_a \subseteq \mathbb{R}^n$ such that\\ $V^\prime_a \cap M = V_a$. Then by theorem 16.3, we get our desired partition which is subordinate to $\{V^\prime_a\}_{a \in A}$.\retTwo
	\end{myIndent}

	\hTwo\ul{Definition:} Let $(\phi_i)_{i \in \mathbb{N}}$ be a partition of unity on $M$ subordinate to the collection of coordinate patches $\{\alpha_i: U_i \to V_i\}_{i \in \mathbb{N}}$ which cover $M$. Without loss of generality, suppose $\supp(\phi_i) \cap M \subseteq V_i$. Then, we define a Borel measure on $M$ by:
	
	{\centering $V(E) \coloneqq \sum\limits_{i=1}^\infty \int_{E \cap V_i} \phi_i \df V_{\alpha_i} = \sum\limits_{i=1}^\infty \int_{\alpha_i^{-1}(E \cap V_i)} (\phi_i \circ \alpha_i) V(\Df \alpha_i) \df m$ \retTwo\par}

	From here it's pretty obvious that $V(\emptyset) = 0$ and that $V$ is countably additive. Also, similarly to before we can then deduce that:

	{\centering $\int_M f \df V = \sum\limits_{i=1}^\infty \int_{V_i} f\phi_i \df V_{\alpha_i} = \sum\limits_{i=1}^\infty \int_{U_i} (f\phi_i \circ \alpha_i) V(\Df \alpha_i) \df m$ \newpage\par}

	Now our first challenge is to show that this definition is independent of our choice of coordinate patches and partition of unity.

	\begin{myIndent}\pracTwo
		Let $(\psi_i)_{i \in \mathbb{N}}$ be another partition of unity on $M$ subordinate to another collection of coordinate patches $\{\beta_i: U^\prime_i \to V^\prime_i\}_{i \in \mathbb{N}}$ which cover $M$ (and like before suppose\\ $\supp(\psi_i) \cap M \subseteq V^\prime_i$).\retTwo

		Now importantly, by our prior results about integration on manifolds parametrized by single coordinate patches, we know that $\int_{V_i \cap V_j^\prime } f \df V_{\alpha_i} = \int_{V_i \cap V_j^\prime } f \df V_{\beta_j}$ for all\\ [-2pt] integrable $f$ on $V_i \cap V_j^\prime$. Therefore, if $E \in \mathcal{B}_M$, we have that:

		{\centering\begin{tabular}{l}
			$\sum\limits_{i=1}^\infty \int_{V_i} \phi_i\chi_E \df V_{\alpha_i} = \sum\limits_{i=1}^\infty \int_{V_i} \phi_i\chi_E \left(\sum\limits_{j=1}^\infty \psi_j \chi_{V^\prime_j}\right) \df V_{\alpha_i}$\\ [12pt]

			$\phantom{\sum\limits_{i=1}^\infty \int_{V_i} \phi_i\chi_E \df V_{\alpha_i}} = \sum\limits_{i=1}^\infty \sum\limits_{j=1}^\infty \int_{V_i \cap V_j^\prime } \phi_i\psi_j\chi_E \df V_{\alpha_i} \phantom{= \sum\limits_{j=1}^\infty \int_{V_j^\prime} \psi_j\chi_E \df V_{\beta_j}}$\\ [12pt]
		
			$\phantom{\sum\limits_{i=1}^\infty \int_{V_i} \phi_i\chi_E \df V_{\alpha_i}} = \sum\limits_{i=1}^\infty \sum\limits_{j=1}^\infty \int_{V_i \cap V_j^\prime } \phi_i\psi_j\chi_E \df V_{\beta_j}$\\ [12pt]

			$\phantom{\sum\limits_{i=1}^\infty \int_{V_i} \phi_i\chi_E \df V_{\alpha_i}} = \sum\limits_{j=1}^\infty \sum\limits_{i=1}^\infty \int_{V_i \cap V_j^\prime } \phi_i\psi_j\chi_E \df V_{\beta_j}$\\ [12pt]

			$\phantom{\sum\limits_{i=1}^\infty \int_{V_i} \phi_i\chi_E \df V_{\alpha_i}} = \sum\limits_{j=1}^\infty \int_{V_j^\prime} \psi_j\chi_E \left(\sum\limits_{i=1}^\infty \phi_i \chi_{V_i}\right) \df V_{\beta_j} = \sum\limits_{j=1}^\infty \int_{V_j^\prime} \psi_j\chi_E \df V_{\beta_j}$
		\end{tabular}\retTwo\par}

		\begin{myTindent}
			Side note: we can swap the order of the sums in the second to last line by applying Fubini-Tonelli's theorem.\retTwo
		\end{myTindent}

		This shows that our definition of the measure of $E$ is independent of our patches and partition of unity.\retTwo

		One other thing to note is that if $M$ can be parametrized by a single coordinate patch, then this result shows that our definition for parametrized manifolds is\\ compatible with this one. After all, just take $\phi_i = \delta_{1,i}$ (where $\delta$ is the Kronecker delta function), and let $\alpha_i$ be the same coordinate patch parametrizing all of our manifold for all $i$.\retTwo
	\end{myIndent}
\end{myIndent}

\mySepTwo

\dispDate{7/31/2025}

Ok. So in the previous journal entry, I was sorta following along with Munkres while only slightly modifying his thoerems. However, after some thought, I actually want to take a completely different approach to defining a measure on manifolds which is more in line with math 240A. So, I'm going to completely depart from Munkres and do a bunch of stuff entirely on my own (i.e. not based on any people's books).\newpage

Let $M$ be a $k$-manifold of class $C^r$ in $\mathbb{R}^n$ (with $r \geq 1$ and $k \leq n$). Then recall from the last journal entry that $M$ is an LCH second countable topological space. Hence, it follows that $M$ is $\sigma$-compact, and from there it follows easily that given any arbitrary collection $\{\alpha_a: U_a \to V_a\}_{a \in A}$ of coordinate patches covering $E \subseteq M$, there is a countable subcover over $E$.\retTwo

Also, recall that if $\alpha: U \to V$ is a coordinate patch on $M$, then we can use the formula $\int_{\alpha^{-1}(E)} V(\Df \alpha)\df m$ to calculate the "surface-volume" of any Borel set\\ $E \subseteq V$.
\begin{myIndent}\pracOne
	Note that if $\alpha_1: U_1 \to V_1$ and $\alpha_2: U_2 \to V_2$ are both coordinate patches on $M$,\\ $E \subseteq V_1 \cap V_2$ is Borel, and $U_1, U_2$ are open in $\mathbb{R}^k$, then as we already noted, by\\ theorem 24.1 plus the theorem at the bottom of page 79 of this journal:
	
	{\centering $\int_{\alpha_1^{-1}(E)} V(\Df \alpha_1)\df m = \int_{\alpha_2^{-1}(E)} V(\Df \alpha_2)\df m$.\retTwo\par}
	
	That said, before continuing on I want to show that this equivalence still holds if $U_1$ or $U_2$ is open in $H^k$ but not $\mathbb{R}^k$.

	\begin{myIndent}\pracTwo
		To start off, by restricting $\alpha_1$ and $\alpha_2$ to their preimages of $V_1 \cap V_2$, we can without loss of generality assume that $V_1 = V_2$. This is important because it makes it so that $\alpha_2^{-1} \circ \alpha_1(U_1) = U_2$ and $\alpha_1^{-1} \circ \alpha_2(U_2) = U_1$.\retTwo

		Now, its impossible for $U_1$ to be open in $\mathbb{R}^k$ but not $U_2$, or vice versa. After all, if $U_1$ is open in $\mathbb{R}^k$, we must have that the transition function $\alpha_2^{-1} \circ \alpha_1: U_1 \to U_2$ maps $U_1$ to an open set in $\mathbb{R}^k$. Analogous reasoning using the transition function $\alpha_1^{-1} \circ \alpha_2$ works if $U_2$ is open in $\mathbb{R}^k$.\retTwo

		Now suppose both $U_1$ and $U_2$ are open only in $H^k$. Then, we can easily see that $W_i \coloneqq U_i - \partial H^k$ is the interior of $U_i$ in $\mathbb{R}^k$ and that $m(U_i - W_i) = 0$ for both $i$. Also, since the transition functions map open sets of $\mathbb{R}^k$ to open sets of $\mathbb{R}^k$, we have that $\alpha_2^{-1} \circ \alpha_1(W_1) \subseteq W_2$ and $\alpha_1^{-1} \circ \alpha_2(W_2) \subseteq W_1$. This is enough to say that the transition functions restricted to $W_1$ and $W_2$ are a diffeomorphism.\retTwo

		Now finally, by applying the lemma at the bottom of page 79, we have for all\\ functions $f$ which are integrable over $V$:

		{\centering\begin{tabular}{l}
			$\int_{U_1} (f \circ \alpha_1) V(\Df \alpha_1)\df m = 0 + \int_{W_1} (f \circ \alpha_1) V(\Df \alpha_1)\df m$\\
			$\phantom{\int_{U_1} (f \circ \alpha_1) V(\Df \alpha_1)\df m} = 0 + \int_{W_2} (f \circ \alpha_2) V(\Df \alpha_2)\df m = \int_{U_2} (f \circ \alpha_2) V(\Df \alpha_2)\df m$\\
		\end{tabular}\retTwo\par}

		Set $f = \chi_E$ and we are done.\retTwo
	\end{myIndent}
\end{myIndent}

We take the following steps to define a measure on $M$:
\begin{enumerate}\hTwo
	\item[(1)] Defining an algebra or even a ring of sets is too much to ask for right now. But, we can at least define a collection of sets $\mathcal{A}$ satisfying that if $A \in \mathcal{A}$ and $E \subseteq A$ is a Borel subset of $M$, then $E \in \mathcal{A}$. Specifically, let $\mathcal{A}$ be the collection of Borel sets $A \subseteq M$ for which there exists a coordinate patch $\alpha: U \to V$ with $A \subseteq V$.\newpage
	
	\item[(2)] Next, we define a "premeasure" $\mu_0$ on $\mathcal{A}$. Specifically, for each $A \in \mathcal{A}$, define\\ $\mu_0(A) = \int_{\alpha^{-1}(A)} V(\Df \alpha) \df m$ where $\alpha: U \to V$ is some coordinate patch with $A \subseteq V$.\retTwo
	
	Importantly, even though $\mu_0$ isn't a proper premeasure according to Folland's\\ definition since $\mathcal{A}$ isn't actually an algebra, it is still the case that $\mu_0$ and $\mathcal{A}$ are\\ structured enough that the following is easily seen to hold:\\ [-24pt]
	\begin{itemize}
		\item $\mu_0(\emptyset) = 0$
		\item If $A, B \in \mathcal{A}$ satisfy $A \subseteq B$. then $\mu_0(A) \leq \mu_0(B)$.
		\item If $(A_j)_{j \in \mathbb{N}}$ is a sequence of disjoint sets in $\mathcal{A}$ with $\bigcup_{j \in \mathbb{N}}A_j \in \mathcal{A}$, then:
		
		{\centering $\mu_0(\bigcup\limits_{j \in \mathbb{N}}A_j) = \sum\limits_{j=1}^\infty \mu_0(A_j)$.  \retTwo\par}
	\end{itemize}

	\item[(3)] Now, we use $\mu_0$ to define an outer measure on $\mathcal{A}$. For any $E \subseteq M$, define:
	
	{\centering $\mu^*(E) = \inf\left\{\sum\limits_{i=1}^\infty \mu_0(A_i) : A_i \in \mathcal{A} \text{ for all } i \text{ and } E \subseteq \bigcup\limits_{i \in \mathbb{N}}A_i\right\}$ \retTwo\par}

	Since any subset of $M$ can be covered by countably many coordinate patches, we know that $\mu^*(E)$ is well defined. The rest of the proof that $\mu^*$ is an outer measure is then identical to proposition 1.10 of Folland (which is at the top of page 21 of my Latex math 240a notes).\retTwo

	\exTwo\item[(4)] \ul{Lemma:} $\mu^*|_\mathcal{A} = \mu_0$ and every set in $\mathcal{A}$ is $\mu^*$ measurable.

	\begin{myIndent}\exTwoP
		Proof:\\
		To prove the first claim, suppose $E \in \mathcal{A}$ and $(A_m)_{m \in \mathbb{N}}$ is a sequence of sets in $\mathcal{A}$ covering $E$. It's trivial that $\mu^*(E) \leq \mu_0(E)$. Meanwhile let $B_1 = E \cap A_1$ and $B_m = E \cap A_m - \bigcup_{j=1}^{m-1} A_j$. Then the $B_m$ are each disjoint Borel subsets of $E \in \mathcal{A}$ whose union is all of $E$. Hence all the $B_m$ are in $\mathcal{A}$ and we have:

		{\centering $\mu_0(E) = \mu_0(\bigcup\limits_{m \in \mathbb{N}} B_m) = \sum\limits_{m\in\mathbb{N}}\mu_0(B_m) \leq \sum\limits_{m\in\mathbb{N}}\mu_0(A_m)$ \retTwo\par}

		This shows that $\mu_0(E) \leq \mu^*(E)$, thus proving the first claim.\retTwo

		To show the second claim, suppose $A \in \mathcal{A}$, $E \subseteq X$, and $\varepsilon > 0$. Then there exists a sequence $(B_j)_{j \in \mathbb{N}}$ of sets in $\mathcal{A}$ such that $E \subseteq \bigcup_{j=1}^\infty B_j$ and $\sum_{j=1}^\infty \mu_0(B_j) \leq \mu^*(E) + \varepsilon$. Importantly, $(B_j \cap A)_{j \in \mathbb{N}}$ and $(B_j - A)_{j \in \mathbb{N}}$ are both sequences of sets in $\mathcal{A}$ covering $E \cap A$ and $E - A$ respectively. Also, $\mu_0(B_j) = \mu_0(B_j \cap A) + \mu_0(B_j - A)$ since $B_j \cap A$ and $B_j - A$ are disjoint sets in $\mathcal{A}$ whose union $B_j$ is also in $\mathcal{A}$. Therefore, we have that:

		{\centering\begin{tabular}{l}
			$\mu^*(E) + \varepsilon \geq \sum\limits_{j=1}^\infty \mu_0(B_j)$\\ [12pt]
			$\phantom{\mu^*(E) + \varepsilon} = \sum\limits_{j=1}^\infty \mu_0(B_j \cap A) + \sum\limits_{j=1}^\infty \mu_0(B_j - A) \geq \mu^*(E \cap A) + \mu^*(E - A)$
		\end{tabular} \newpage\par}
		
		Taking $\varepsilon \to 0$, we have that $\mu^*(E) \geq \mu^*(E \cap A) + \mu^*(E - A)$ for all $E \subseteq M$. Thus $A$ is $\mu^*$ measurable. $\blacksquare$\retTwo
	\end{myIndent}

	\hTwo\item[(5)] We now know by Carathéodory's theorem that if $\mathcal{N}$ is the $\sigma$-algebra of $\mu^*$-measurable sets, then $\mu \coloneq \mu^*|_\mathcal{N}$ is a complete measure on $\mathcal{N}$. Furthermore, $\mathcal{A} \subseteq \mathcal{N}$ with\\ $\mu|_\mathcal{A} = \mu_0$.\retTwo

	\exTwo\item[(6)] \ul{Lemma:} If $\mathcal{B}_M$ is the collection of Borel sets on our manifold $M$, then $\mathcal{B}_M \subseteq \mathcal{N}$.\\ Hence, we can restrict $\mu$ to be a Borel measure if we wanted.
	
	\begin{myIndent}\exTwoP
		Proof:\\
		If $E \in \mathcal{B}_m$, then let $(A_j)_{j \in \mathbb{N}}$ be a countable covering of $E$ consisting of sets in $\mathcal{A}$. Then $E \cap A_j \in \mathcal{A} \subseteq \mathcal{N}$ for all $j$ and $\bigcup_{j \in \mathbb{N}} (E \cap A_j) = E$. So $E \in \mathcal{N}$.\retTwo
	\end{myIndent}

	\hTwo\item[(7)] The next thing we want to do now is show that $\mu$ is $\sigma$-finite. One reason we want to do this is so that we can apply theorems such as Fubini-Tonelli and Radon-Nikodym. Another reason is so that (as I'll show in the next step) $\mu$ is guarenteed to be the unique measure on $(M, \mathcal{N})$ which preserves our definition of the measure of a\\ manifold parametrized by a single coordinate patch. Hence, this construction agrees with what I was doing back when I was loosely following Munkres.

	\begin{myIndent}\pracTwo
		To start, we will show that every point $p \in M$ has a neighborhood $A \in \mathcal{N}$ with finite measure. 
		\begin{myIndent}
			Let $\alpha: U \to V$ be a coordinate patch on $M$ about $p$. Then, given $x \in U$ satisfying that $\alpha(x) = p$, let $K \subseteq U$ be a compact neighborhood of $x$. Since $\alpha$ is a homeomorphism, it is clear that $A \coloneq \alpha(K)$ is a compact set containing $p$ in its interior. Hence, $A \in \mathcal{B}_M \subseteq \mathcal{N}$, $A$ is a neighborhood of $p$ and:

			{\centering$\mu(A) = \int_{K} V(\Df \alpha)\df m$\retTwo}

			Now $V(\Df \alpha)$ is continuous. So by the extreme value theorem, there exists some $C \geq 0$ such that:

			{\centering$\int_{K} V(\Df \alpha)\df m \leq \int_K C \df m = Cm(K)$\retTwo}

			And since $m(K) < \infty$, we've shown that $\mu(A) < \infty$.		
		\end{myIndent}

		Now since $M$ is $\sigma$-compact, if we pick a set $A_p$ for each $p \in M$ using the reasoning above, then there is a countable subcovering of the $A_p$ over $M$. This proves that $M$ is $\sigma$-finite. $\blacksquare$\retTwo
	\end{myIndent}

	\exTwo\item[(8)] \ul{Lemma:} If $\nu: \mathcal{N} \to [0, \infty]$ is another measure on $(M, \mathcal{N})$ satisfying that $\nu|_\mathcal{A} = \mu_0$, then $\nu = \mu$.

	\begin{myIndent}\exTwoP
		The proof of this is almost identical to that of theorem 1.14 in Folland (bottom of page 24 on my Latex math 240a notes).\newpage

		The one difference is that if $(A_j)_{j \in \mathbb{N}} \subseteq \mathcal{A}$ and $A = \bigcup_{j \in \mathbb{N}} A_j$, then $\nu(A) = \mu(A)$ because $A_m - \bigcup_{j=1}^{m-1} A_j \in \mathcal{A}$ for all $m$ and hence:

		{\centering $\nu(A) = \sum\limits_{m=1}^\infty \nu(A_m - \bigcup_{j=1}^{m-1} A_j) = \sum\limits_{m=1}^\infty \mu(A_m - \bigcup_{j=1}^{m-1} A_j) = \mu(A)$. $\blacksquare$ \retTwo\par}
	\end{myIndent}
\end{enumerate}

Why did I take this pivot? The reason is that now all of Munkres theorems from chapter 25 of his book are obvious including the final theorem about how one would in practice calculate an integral on $M$. Also, I got to show that this construction is universal in a sense.\retTwo

\hOne\mySepTwo

I'm now going to switch over to following Guillemin's \ul{Differential Forms} since I was recommended this book by another tutor. Once again, I'm not going to perfectly follow the book. But I will be using the book as a loose guide.\retTwo

\blab{Tensors:}\\
Let $V$ be an $n$-dimensional vector space over a field $F$. Let $V^k$ be the set of all $k$-tuples of elements of $V$. Then a function $T: V^K \to F$ is said to be \udefine{linear in its $i$th. variable} if for all $u, v_1, \ldots, v_k \in V$ and $a, b \in F$, we have that:

{\centering\begin{tabular}{l}
	$T(v_1, \ldots, v_{i-1}, av_i + bu, v_{i+1}, \ldots, v_k)$\\ [2pt]
	$\phantom{aaaaaaaaaa} = aT(v_1, \ldots, v_{i-1}, v_i, v_{i+1}, \ldots, v_k) + bT(v_1, \ldots, v_{i-1}, u, v_{i+1}, \ldots, v_k)$.
\end{tabular}\retTwo\par}

If $T$ is linear in all it's variables, we say $T$ is \udefine{$k$-linear}, and that $T$ is a \udefine{$k$-tensor}.\retTwo

Given $k \geq 1$, we shall denote $\mathcal{L}^k(V)$ to be the set of all $k$-tensors from $V$. Also, we shall denote $\mathcal{L}^0(V) \coloneqq F$. Note that we can add together and scale $k$-linear maps to get more $k$-linear maps. Thus, $\mathcal{L}^k(V)$ is a vector space.
\begin{myIndent}\hTwo
	Side note: $\mathcal{L}^1(V)$ is also called the \udefine{dual space} of $V$ and denoted $V^*$.\retTwo\retTwo\retTwo

	\pracOne Sorry but my apartment mate sent me down a rabbithole that led me to writing a proof on the following pages. I'll come back to tensors right after I'm done writing the following\dots
\end{myIndent}

\newpage

\dispDate{8/5/2025}

So for context, my apartment mate was reading through a paper by Joel Spencer (who is famous for using probabilistic methods to study combinatorics and graph theory). This paper, which is titled \ul{Balancing Unit Vectors}, gives a pretty neat\\ existence proof using probability that given any finite collection $\{u_1, \ldots, u_m\}$ of\\ vectors in $\mathbb{R}^n$ with $\|u_i\|_2 \leq 1$ for all $i$, there exists coefficients $\varepsilon_1, \ldots, \varepsilon_m \in \{-1, +1\}$ such that: $\|\varepsilon_1u_1 + \ldots + \varepsilon_mu_m\|_2 \leq \sqrt{n}$. However, my apartment mate was having trouble with the following part of the paper and thus asked for my help with it:

{\centering\includegraphics[scale=1.3]{Joel_Spencer_balancing_vectors.png}\retTwo\par}

Our issue was figuring out what linear algebra argument Spencer was fucking using. After thinking about it for three days while I was grading tests, I finally came up with a proof:\retTwo

\Hstatement\blab{Theorem:} Let $u_1, \ldots, u_m$ be vectors in $\mathbb{R}^n$. Then there are constants $a_1, \ldots, a_m$ satisfying\\ $a_1u_1 + \ldots + a_mu_m = 0$ such that all $|a_i| \leq 1$ and $a_i = \pm 1$ for all but at most $n$ $i$'s.

\begin{myIndent}\HexOne
	Proof:\\
	We'll proceed by an inductive argument. For our base case, let $u_{m+1} \coloneq 0$. That way\\ $\sum_{i = 1}^m 0 u_i = u_{m+1}$. Next, suppose that for $n + 1 \leq k \leq m$, we've shown that there\\ are constants $b_1, \ldots, b_{k} \in [-1, 1]$ and $\varepsilon_{k+1}, \ldots, \varepsilon_{m+1} \in \{-1, +1\}$ satisfying that:

	{\centering $b_1u_1 + \ldots + b_ku_k = \varepsilon_{k+1}u_{k+1} + \ldots + \varepsilon_{m+1}u_{m+1}$. \retTwo\par}

	If we consider the matrix $U \coloneqq \begin{bmatrix}u_1 & \cdots & u_k\end{bmatrix}$, then letting $b = (b_1, \ldots, b_k)$ we have that any $a = (a_1, \ldots, a_k) \in b + \ker(U)$ will satisfy that:
	
	{\centering $a_1u_1 + \ldots + a_{k}u_{k} = \varepsilon_{k+1}u_{k+1} + \ldots + \varepsilon_{m+1}u_{m+1}$.\retTwo\par}
	
	Since $k > n$, we know that $\ker(U)$ is nontrivial and hence unbounded. At the same time, $\|b\|_{\infty} \leq 1$. Hence, by the connectedness of $b + \ker(U)$ and the continuity of the $\infty$-norm, we know there is some $a \in b + \ker(U)$ with $\|a\|_\infty = 1$. After reordering our $u_i$, this is the same as saying that there exists $a_1, \ldots, a_{k-1} \in [-1, 1]$ and $\varepsilon_k = \pm 1$ such that:

	{\centering $a_1u_1 + \ldots + a_{k-1}u_{k-1} + \varepsilon_{k}u_k = \varepsilon_{k+1}u_{k+1} + \ldots + \varepsilon_{m+1}u_{m+1}$.\retTwo\par}
	
	Subtract both sides by $\varepsilon_{k}u_k$ to complete the induction step.\retTwo

	After induction, we will eventaully get constants $a_1, \ldots, a_n \in [-1, 1]$ and $\varepsilon_{n+1}, \ldots, \varepsilon_{m+1}$ equal to $\pm 1$ such that:

	{\centering $a_1u_1 + \ldots + a_nu_n = \varepsilon_{n+1}u_{n+1} + \ldots + \varepsilon_mu_m + \varepsilon_{m+1}u_{m+1}$. \retTwo\par}

	Move everything over to one side of the equation and forget about the $u_{m+1}$ and we have proven what we wanted.\newpage
\end{myIndent}

\dispDate{8/6/2025}\hOne

Now I'm going to go back to studying tensors.\retTwo

A \udefine{multi-index of length $k$} is a $k$-tuple $I = (i_1, \ldots, i_k)$ of integers. We say $I$ is a multi-index of $n$ if each $i$ is between $1$ and $n$. Now let $u_1, \ldots, u_n$ is a basis of $V$. For $T \in \mathcal{L}^k(V)$, write $T_I \coloneq T(u_{i_1}, \ldots, u_{i_k})$ for every multi-index $I$ of $n$ of length $k$.\retTwo

\exOne\ul{Proposition 1.3.7:} The $T_I$ uniquely determine $T$.

\begin{myIndent}\exTwoP
	Proof:\\
	When $k = 1$, $T$ is just a linear map and we've already proven this for linear maps. For $k > 1$, we proceed by induction. For each $i$, define $T_i \in L^{k-1}(V)$ by:
	
	{\centering $(v_1, \ldots, v_{k-1}) \mapsto T(v_1, \ldots, v_{k-1}, u_i)$.\retTwo\par} Then for $v = c_1u_1 + \ldots + c_nu_n$, we have:

	{\centering $T(v_1, \ldots, v_{k-1}, v) = \sum\limits_{i=1}^n c_iT_i(v_1, \ldots, v_{k-1})$\retTwo\par}

	Also, by induction each $T_i$ is uniquely determined by the coefficients $T_I$ where $I$ is a multi-index of $n$ of length $k$ with a final index equal to $i$.\retTwo

	\begin{myIndent}\exPPP
		Side note: We can see that if $C = (c_{i,j}) \in \mathcal{M}_{n \times k}(F)$ is the matrix satisfying that $v_{j} = \sum_{i=1}^n c_{i,j} u_i$ for each $j$, then:

		{\centering $T(v_1, \ldots, v_k) = \hspace{-1em}\sum\limits_{I = (i_1, \ldots, i_k)}\hspace{-0.5em} \left(\hspace{0.1em}\prod\limits_{j=1}^k c_{i_j, j}\right)T_I$ \retTwo\par}
	\end{myIndent}
\end{myIndent}

\hOne%
Given two tensors: $T_1 \in \mathcal{L}^k(V)$ and $T_2 \in \mathcal{L}^\ell(V)$, we define the \udefine{tensor product} of $T_1$ and $T_2$ as:

{\centering $(T_1 \otimes T_2)(v_1, \ldots, v_k, v_{k+1}, \ldots, v_{k + \ell}) = T_1(v_1, \ldots, v_k)T_2(v_{k + 1}, \ldots, v_{k + \ell})$\retTwo\par}

Note that:\hTwo\\ [-20pt]
\begin{itemize}
	\item $\otimes$ is associative.
	\item If $T_1$ or $T_2$ is a $0$-tensor, then $\otimes$ is just scalar multiplication.
	\item If $\lambda \in F$, $T_1 \in \mathcal{L}^k(V)$, and $T_2 \in \mathcal{L}^\ell(V)$, then $\lambda(T_1 \otimes T_2) = (\lambda T_1) \otimes T_2 = T_1 \otimes (\lambda T_2)$.
	\item If $T_1, T_2 \in \mathcal{L}^k(V)$ and $T_3 \in \mathcal{L}^\ell(V)$, then $(T_1 + T_2) \otimes T_3 = (T_1 \otimes T_3) + (T_2 \otimes T_3)$. Also $T_3 \otimes (T_1 + T_2)  = (T_3 \otimes T_1 ) + (T_3 \otimes T_2)$\retTwo
\end{itemize}

\hOne Suppose $T \in \mathcal{L}^k(V)$ and there exists $\ell_1, \ldots, \ell_k \in V^* = \mathcal{L}^1(V)$ such that $T = \ell_1 \otimes \cdots \otimes \ell_k$. Then we say $T$ is \udefine{decomposable}.\retTwo

If $u_1, \ldots, u_n$ is a basis for $V$, then we can define the \udefine{dual basis}: $u_1^*, \ldots, u_n^*$ for $V^*$ such that if $v = \sum_{i=1}^n c_iu_i$, then $u_j^*(v) = c_j$.

\begin{myIndent}\exTwoP
	To prove that this is a basis, first note that if $f \in V^*$ and $\lambda_i = f(u_i)$, then we can easily see that for any $v = \sum_{i=1}^n c_iu_i$, then:\newpage
	
	{\centering $f(v) = \sum_{i=1}^n c_if(u_i) = \sum_{i=1}^n \lambda_ic_i = \sum_{i=1}^n \lambda_iu_i^*(v)$.\retTwo\par}

	It follow that $u_1^*, \ldots, u_n^*$ span all of $V^*$. Also, consider if $f = \sum_{i=1}^n \lambda_i u_i^* = 0$ where all the $\lambda_i \in F$. Then we know that $0 = f(u_j) = \lambda_j$ for all $j$. This shows that\\ $u_1^*, \ldots, u_n^*$ are linearly  independent.\retTwo
\end{myIndent}

\pracOne\mySepTwo
Tangent: if $V, W$ are vector fields over $F$ and $A: V \to W$ is a linear map, then we define the \udefine{transpose} $A^\dag: W^* \to V^*$ of $A$ by $f \mapsto A^\dag(f) = f \circ A$.\retTwo

\exOne\ul{Claim 1.2.15:} Suppose $e_1, \ldots, e_m$ is a basis of $V$ and $u_1, \ldots, u_n$ is a basis of $W$. Then if $A = (a_{i,j})$ is the the matrix of $A$ with respect to the given bases, we have that the matrix of $A^\dag$ with respect to the dual bases of $V$ and $W$ is given by $(a_{j,i})$.

\begin{myIndent}\exTwoP
	Proof:\\
	Suppose $(c_{j,i})$ is the matrix representation of $A^\dag$. Then:

	{\centering $A^\dag(u_i^*)(e_j) = u_i^*(A(e_j)) = u_i^*(\sum_{k=1}^n a_{k,j}u_k) = a_{i,j}$ \retTwo\par}

	Simultaneously:

	{\centering $A^\dag(u_i^*)(e_j) = \sum_{k=1}^m c_{k, i} e_k^*(e_j) = c_{j, i}$ \retTwo\par}
\end{myIndent}
\pracOne\mySepTwo

\hOne Let $u_1, \ldots, u_n$ be a basis of $V$ and $u_1^*, \ldots, u_n^*$ be the corresponding dual basis of $V^*$. Then for every multi-index $I = (i_1, \ldots, i_k)$ of $n$ of length $k$, define:

{\centering $u_I^* = u_{i_1}^* \otimes \cdots \otimes u_{i_k}^*$.\retTwo\par}

Note that if $J = (j_1, \ldots, j_k)$ is another multi-index of $n$ of length $k$, then\\ $u_I^*(u_{j_1}, \ldots, u_{j_k}) = \delta_{I,J}$ where $\delta$ is the Kronecker delta function.\retTwo

\exOne\ul{Theorem 1.3.13:} The $k$-tensors $u_I^*$ form a basis for $\mathcal{L}^k(V)$.
\begin{myIndent}\exTwoP 
	Proof:\\
	Suppose $T \in \mathcal{L}^k(V)$. Then if $T^\prime \coloneq \sum_{I} T_Iu_I^*$, we have that $T^\prime_J = T_J$ for all multi-indices $J$ of $n$ of length $k$. Therefore, since $T^\prime$ and $T$ are uniquely determined by the same $T_I$, this proves that $T = T^\prime$. So, $T$ is in the span of the $u_I^*$. This proves that the $u_I^*$ span all of $\mathcal{L}^k(V)$.\retTwo

	Next suppose $T^\prime = \sum_{I} C_Iu_I^* = 0$ where each $C_I \in F$. Then if $J$ is a multi-index of $n$ of length $k$,
	
	{\centering$0 = T^\prime(u_{j_1}, \ldots, u_{j_k}) = \sum\limits_{I} C_Iu_I^*(u_{j_1}, \ldots, u_{j_k}) = C_J$\retTwo\par}

	This shows that the $u_I^*$ are linearly independent. \blacksquare\retTwo
\end{myIndent}

\ul{Corollary 1.3.15:} If $V$ is an $n$-dimensional vector space, then $\mathcal{L}^k(V)$ is an\\ $n^k$-dimensional vector space.\newpage

\hOne If $V, W$ are vector fields over $F$, $A: V \to W$ is a linear map, and $T \in \mathcal{L}^k(W)$, we define $A^\dag T(v_1, \ldots, v_k) \coloneq T(Av_1, \ldots, Av_k)$.\retTwo

We call $A^\dag T$ the \udefine{pullback} of $T$ by the map $A$. Note that this is just a generalization of taking the transpose of $A$.\retTwo

\exOne\ul{Proposition 1.3.18:} If we denote $A^\dag : \mathcal{L}^k(W) \to \mathcal{L}^k(V)$ to be the map $T \mapsto A^\dag T$, then $A^\dag$ is a linear map.

\begin{myIndent}\exTwoP This should be pretty obvious\dots\retTwo
\end{myIndent}

Furthermore, $A^\dag(T_1 \otimes T_2) = (A^\dag T_1) \otimes (A^\dag T_2)$.

\begin{myIndent}\exTwoP
	Proof:\\
	Suppose $T_1 \in \mathcal{L}^k(V)$ and $T_2 \in \mathcal{L}^\ell(V)$. Then:

	{\centering\begin{tabular}{l}
		$A^\dag(T_1 \otimes T_2)(v_1, \ldots, v_k, v_{k+1}, \ldots, v_{k+\ell})$\\
		$\phantom{aaaaaaaaaaaaaa} = (T_1 \otimes T_2)(Av_1, \ldots, Av_k, Av_{k+1}, \ldots, Av_{k+\ell})$\\
		$\phantom{aaaaaaaaaaaaaa} = T_1(Av_1, \ldots, Av_k)T_2(Av_{k+1}, \ldots, Av_{k+\ell})$\\
		$\phantom{aaaaaaaaaaaaaa} = A^\dag T_1(v_1, \ldots, v_k)A^\dag T_2(v_{k+1}, \ldots, v_{k+\ell})$\\
		$\phantom{aaaaaaaaaaaaaa} = (A^\dag T_1 \otimes A^\dag T_2)(v_1, \ldots, v_k, v_{k+1}, \ldots, v_{k+\ell})$. $\blacksquare$
	\end{tabular}\retTwo\par}
\end{myIndent}

As a corollary to the above fact, we know that pullbacks map decomposable tensors to decomposable tensors.\retTwo

Finally, suppose $B: U \to V$ is another linear map. Then $(AB)^\dag T = B^\dag(A^\dag T)$ for all $T \in \mathcal{L}^k(W)$. In other words, $(AB)^\dag = B^\dag A^\dag$.

\begin{myIndent}\exTwoP
	Proof:

	{\centering\begin{tabular}{l}
		$B^\dag(A^\dag T)(v_1, \ldots, v_k) = A^\dag T(Bv_1, \ldots, Bv_k)$\\
		$\phantom{B^\dag(A^\dag T)(v_1, \ldots, v_k)} = T(ABv_1, \ldots, ABv_k) = (AB)^\dag T(v_1, \ldots, v_k)$. $\blacksquare$
	\end{tabular}\retTwo\par}
\end{myIndent}

\hOne\blab{Alternating $k$-Tensors:}\\
Let $V$ be an $n$-dimensional vector space over a field $F$ with characteristic $\neq 2$ and $S_k$ be the symmetric group over $\{1, \ldots, k\}$. For $\sigma \in S_k$ and $T \in \mathcal{L}^k(V)$, we define $T^\sigma \in \mathcal{L}^k(V)$ by:

{\centering $T^\sigma(v_1, \ldots, v_k) = T(v_{\sigma^{-1}(1)}, \ldots, v_{\sigma^{-1}(k)})$ \retTwo\par}

\exOne\ul{Proposition 1.4.14:}
\begin{itemize}
	\item[(a)] If $T = \ell_1 \otimes \cdots \otimes \ell_k$ where each $\ell_i \in V^*$, then $T^\sigma = \ell_{\sigma(1)} \otimes \cdots \otimes \ell_{\sigma(k)}$.
	
	\begin{myIndent}\exTwoP
		Proof:

		{\centering\begin{tabular}{l}
			$T^\sigma(v_1, \ldots, v_k) = T(v_{\sigma^{-1}(1)}, \ldots, v_{\sigma^{-1}(k)}) = \ell_1(v_{\sigma^{-1}(1)}) \cdots \ell_k(v_{\sigma^{-1}(k)})$\\
			$\phantom{T^\sigma(v_1, \ldots, v_k) = T(v_{\sigma^{-1}(1)}, \ldots, v_{\sigma^{-1}(k)})} = \ell_{\sigma(1)}(v_1) \cdots \ell_{\sigma(k)}(v_k)$\\
			$\phantom{T^\sigma(v_1, \ldots, v_k) = T(v_{\sigma^{-1}(1)}, \ldots, v_{\sigma^{-1}(k)})} = (\ell_{\sigma(1)} \otimes \cdots \otimes \ell_{\sigma(k)})(v_1, \ldots, v_k)$
		\end{tabular}\retTwo\par}
	\end{myIndent}

	\item[(b)] If $\sigma \in S_k$, the function $T \mapsto T^\sigma$ is a linear map from $\mathcal{L}^k(V)$ to $\mathcal{L}^k(V)$.
	\begin{myIndent}\exTwoP
		This should be obvious. Also note that this map is invertible via the function $T \mapsto T^{(\sigma^{-1})}$.\newpage
	\end{myIndent}

	\item[(c)] If $\sigma, \tau \in S_k$, then $(T^\sigma)^\tau = T^{\sigma \tau}$.
	
	\begin{myIndent}\exTwoP
		Proof:\\
		Let $u_i \coloneq v_{\tau^{-1}(i)}$ for all $i$. Then:

		{\centering\begin{tabular}{l}
			$(T^\sigma)^\tau(v_1, \ldots, v_k) = T^\sigma(v_{\tau^{-1}(1)}, \ldots, v_{\tau^{-1}(k)})$\\ [2pt]

			$\phantom{(T^\sigma)^\tau(v_1, \ldots, v_k)} = T^\sigma(u_1, \ldots, u_k) = T(u_{\sigma^{-1}(1)}, \ldots, u_{\sigma^{-1}(k)})$\\ [2pt]

			$\phantom{(T^\sigma)^\tau(v_1, \ldots, v_k) = T^\sigma(u_1, \ldots, u_k)} = T(v_{\tau^{-1}(\sigma^{-1}(1))}, \ldots, v_{\tau^{-1}(\sigma^{-1}(k))})$\\ [2pt]

			$\phantom{(T^\sigma)^\tau(v_1, \ldots, v_k) = T^\sigma(u_1, \ldots, u_k)} = T(v_{(\sigma\tau)^{-1}(1)}, \ldots, v_{(\sigma\tau)^{-1}(k)})$\\ [2pt]

			$\phantom{(T^\sigma)^\tau(v_1, \ldots, v_k) = T^\sigma(u_1, \ldots, u_k)} = T^{\sigma\tau}(v_1, \ldots, v_k)$
		\end{tabular}\retTwo\par}
	\end{myIndent}
\end{itemize}

\hOne Let $V$ be a vector space and $k \geq 1$ be an integer. Then $T \in \mathcal{L}^k(V)$ is \udefine{alternating} if $T^\sigma = \sgn(\sigma)T$ for all $\sigma \in S_k$. We denote $\mathcal{A}^k(V)$ as the set of all alternating\\ $k$-tensors on $V$.

\begin{myIndent}\hTwo
	Note:
	\begin{itemize}
		\item If $c_1, c_2 \in F$ and $T_1, T_2 \in \mathcal{A}^k(V)$, then since $T \mapsto T^\sigma$ is a linear map, we have for all $\sigma \in S_k$ that:
		
		{\centering\begin{tabular}{l}
			$(c_1T_1 + c_2T_2)^{\sigma} = c_1T_1^{\sigma} + c_2T_2^\sigma$\\
			$\phantom{(c_1T_1 + c_2T_2)^{\sigma}} = c_1\sgn(\sigma)T_1 + c_2\sgn(\sigma)T_2 = \sgn(\sigma)(c_1T_1 + c_2T_2)$
		\end{tabular}\retTwo\par}

		This proves that $\mathcal{A}^k(V)$ is a subspace of $\mathcal{L}^k(V)$.\retTwo

		\item We shall define $\mathcal{A}^0(V) \coloneqq \mathcal{L}^0(V) = F$.\retTwo
	\end{itemize}
\end{myIndent}

Given an integer $k > 0$, and a tensor $T \in \mathcal{L}^k(V)$, let:

{\centering $\Alt(T) \coloneqq \sum\limits_{\tau \in S_k} \sgn(\tau)T^\tau$.\retTwo\par}

Then the \udefine{alternation operation} has the following properties:\\
\exOne\ul{Proposition 1.4.17:}
\begin{itemize}
	\item[(a)] Given any $T \in \mathcal{L}^k(V)$ and $\sigma \in S_k$ (where $k > 0$), we have that\\ $(\Alt(T))^{\sigma} = \sgn(\sigma)\Alt(T)$. I.e., $\Alt(T)$ is an alternating tensor.
	
	\begin{myIndent}\exTwoP
		Proof:\\
		By proposition 1.4.14 plus the fact that $(\sgn(\sigma))^2 = 1$, we have that:

		{\centering\begin{tabular}{l}
			$(\Alt(T))^\sigma = (\sum\limits_{\tau \in S_k}\sgn(\tau)T^{\tau})^{\sigma}$\\ [16pt]
		
			$\phantom{(\Alt(T))^\sigma} = 1 \cdot \hspace{-0.4em}\sum\limits_{\tau \in S_k}\sgn(\tau)T^{\tau\sigma} = (\sgn(\sigma))^2 \sum\limits_{\tau \in S_k}\sgn(\tau)T^{\tau\sigma}$\\ [16pt]

			$\phantom{(\Alt(T))^\sigma = 1 \cdot \hspace{-0.4em}\sum\limits_{\tau \in S_k}\sgn(\tau)T^{\tau\sigma}} = \sgn(\sigma) \sum\limits_{\tau \in S_k}\sgn(\tau\sigma)T^{\tau\sigma} = \sgn(\sigma) \sum\limits_{\tau^\prime \in S_k}\sgn(\tau^\prime)T^{\tau^\prime}$\\ [16pt]

			$\phantom{(\Alt(T))^\sigma = 1 \cdot \hspace{-0.4em}\sum\limits_{\tau \in S_k}\sgn(\tau)T^{\tau\sigma} = \sgn(\sigma) \sum\limits_{\tau \in S_k}\sgn(\tau\sigma)T^{\tau\sigma}} = \sgn(\sigma) \Alt(T)$\\ [16pt]
		\end{tabular} \newpage\par}
	\end{myIndent}

	\item[(b)] If $T \in \mathcal{A}^k(V)$, the $\Alt(T) = k!T$.
	
	\begin{myIndent}\exTwoP
		Proof:\\
		Since $T^\tau = \sgn(\tau)T$ for all $\tau \in S_k$, we know:

		{\centering\begin{tabular}{l}
			$\Alt(T) = \sum\limits_{\tau \in S_k} \sgn(\tau)T^\tau = \sum\limits_{\tau \in S_k} (\sgn(\tau))^2T = \sum\limits_{\tau \in S_k} (1)T = |S_k|T = k!T$
		\end{tabular} \retTwo\par}
	\end{myIndent}

	\item[(c)] $\Alt(T^\sigma) = (\Alt(T))^\sigma$.
	
	\begin{myIndent}\exTwoP
		Proof:\\
		By similar reasoning to in part (a), we have that:

		{\centering\begin{tabular}{l}
			$\Alt(T^\sigma) = 1 \cdot \hspace{-0.4em}\sum\limits_{\tau \in S_k} \sgn(\tau) T^{\sigma\tau} = (\sgn(\sigma))^2 \sum\limits_{\tau \in S_k} \sgn(\tau) T^{\sigma\tau}$\\ [12pt]

			$\phantom{\Alt(T^\sigma) = 1 \cdot \hspace{-0.4em}\sum\limits_{\tau \in S_k} \sgn(\tau) T^{\sigma\tau}} = \sgn(\sigma) \sum\limits_{\tau \in S_k} \sgn(\sigma\tau) T^{\sigma\tau}$\\ [12pt]

			$\phantom{\Alt(T^\sigma) = 1 \cdot \hspace{-0.4em}\sum\limits_{\tau \in S_k} \sgn(\tau) T^{\sigma\tau}} = \sgn(\sigma) \sum\limits_{\tau^\prime \in S_k} \sgn(\tau^\prime) T^{\tau^\prime} = \sgn(\sigma)\Alt(T)$
		\end{tabular} \retTwo\par}

		And since $\sgn(\sigma)\Alt(T) = (\Alt(T))^\sigma$ by part (a), we know $\Alt(T^\sigma) = (\Alt(T))^\sigma$.\retTwo
	\end{myIndent}

	\item[(d)] The map $\Alt: \mathcal{L}^k(V) \to \mathcal{L}^k(V)$ defined by $T \mapsto \Alt(T)$ is a linear map. (Also it's onto if $F$ has characteristic $0$ or $> k$.)
	
	\begin{myIndent}\exTwoP
		Proof:\\
		$\Alt$ is a linear map because it is a linear combination of a bunch of linear maps. The onto property follows from part (b).\retTwo
	\end{myIndent}
\end{itemize}

\hOne\dispDate{8/7/2025}

If $I = (i_1, \ldots, i_k)$ is a multi-index of $n$ of length $k$, then we write:
\begin{itemize}
	\item $I$ is \udefine{repeating} if $i_s = i_r$ for some $s \neq r$.
	\item $I$ is \udefine{increasing} if $i_1 < i_2 < \ldots < i_k$.
	\item Given $\sigma \in S_k$, we define $I^\sigma = (i_{\sigma(1)}, \ldots, i_{\sigma(k)})$.\retTwo
\end{itemize}

Note that if $I$ is not repeating, then there is a unique permutation $\sigma \in S_k$ such that $I^\sigma$ is increasing.\retTwo

Let $u_1, \ldots, u_n$ be a basis of the vector space $V$ over a field $F$ of characteristic $\neq 2$, and let $u_1^*, \ldots, u_n^*$ be the corresponding dual basis. Now given the multi-index\\ $I = (i_1, \ldots, i_k)$, set $u_I^* = u_{i_1}^* \otimes \cdots \otimes u_{i_k}^*$. Next define $\Psi_I = \Alt(u_I^*)$.\retTwo

\exOne\ul{Proposition 1.4.20:} Let $I = (i_1, \ldots, i_k)$ and $J = (j_1, \ldots, j_k)$ be multi-indices.
\begin{itemize}
	\item[(a)] $\Psi_{I^\sigma} = \sgn(\sigma)\Psi_I$.
	\begin{myIndent}\exTwoP
		To start off, by the last proposition:
		
		{\centering$\sgn(\sigma)\Psi_I = \sgn(\sigma)\Alt(u_I^*) = (\Alt(u_I^*))^\sigma = \Alt((u_I^*)^\sigma)$.\newpage\par}

		Next, set $\ell_j = u^*_{i_j}$ for $1 \leq j \leq k$. Then by proposition 1.4.14, we have:
		
		{\centering$(u_I^*)^\sigma = (\ell_1 \otimes \cdots \otimes \ell_k)^\sigma = \ell_{\sigma(1)} \otimes \cdots \otimes \ell_{\sigma(k)} = u_{i_{\sigma(1)}}^* \otimes \cdots \otimes u_{i_{\sigma(k)}}^* = u^*_{I^\sigma}$.\retTwo\par}

		Thus $\sgn(\sigma)\Psi_I = \Alt((u_I^*)^\sigma) = \Alt(u^*_{I^\sigma}) = \Psi_{I^\sigma}$.\retTwo
	\end{myIndent}

	\item[(b)] If $I$ is repeating, $\Psi_{I} = 0$.
	
	\begin{myIndent}\exTwoP
		If $I$ is repeating, then there exists $r \neq s$ such that $i_r = i_s$. Then in turn, if $\tau_{r,s} \in S_k$ is the transposition of $r$ and $s$, then $\sgn(\tau_{r,s}) = -1$ and $I^{\tau_{r,s}} = I$. Then by part (a), we have:

		{\centering$\Psi_I = \Psi_{I^{\tau_{r,s}}} = \sgn(\tau_{r,s})\Psi_{I} = -\Psi_{I}$\retTwo\par}
		
		This is only possible if $\Psi_I(v_1, \ldots, v_k) = 0$ for all $v_1, \ldots, v_k \in V$. Hence $\Psi_I$ is the zero map.\retTwo
	\end{myIndent}


	\item[(c)] If $I$ and $J$ are strictly increasing, then $\Psi_I(u_{j_1}, \ldots, u_{j_k}) = \delta_{I,J}$ where $\delta$ is the Kronecker delta function.
	
	\begin{myIndent}\exTwoP
		To start off, note that:

		{\centering\begin{tabular}{l}
			$\Psi_I(u_{j_1}, \ldots, u_{k_k}) = \sum\limits_{\tau \in S_k}\sgn(\tau)(u_I^*)^{\tau}(u_{j_1}, \ldots, u_{j_k})$\\ [18pt]

			$\phantom{\Psi_I(u_{j_1}, \ldots, u_{k_k})} = \sum\limits_{\tau \in S_k}\sgn(\tau)(u_{i_1}^* \otimes \cdots \otimes u_{i_k}^*)^{\tau}(u_{j_1}, \ldots, u_{j_k})$\\ [18pt]

			$\phantom{\Psi_I(u_{j_1}, \ldots, u_{k_k})} = \sum\limits_{\tau \in S_k}\sgn(\tau)(u_{i_{\tau(1)}}^* \otimes \cdots \otimes u_{i_{\tau(k)}}^*)(u_{j_1}, \ldots, u_{j_k})$\\ [18pt]

			$\phantom{\Psi_I(u_{j_1}, \ldots, u_{k_k})} = \sum\limits_{\tau \in S_k}\sgn(\tau)u_{i_{\tau(1)}}^*(u_{j_1})\cdots u_{i_{\tau(k)}}^*(u_{j_k})$\\ [18pt]
		\end{tabular}\retTwo\par}

		Now it's clear that $u_{i_{\tau(1)}}^*(u_{j_1})\cdots u_{i_{\tau(k)}}^*(u_{j_k}) = \delta_{I^\tau, J}$. Also, since both $J$ and $I$ are strictly increasing and also since there is only one permutation such that $I^\sigma$ is strictly increasing for any nonrepeating $I$, we know that $\delta_{I^\tau, J} = 1$ iff $\tau = \myId$ and $I = J$. And in that case $\sgn(\tau) = 1$. Hence: 
		
		{\centering $\sum\limits_{\tau \in S_k}\sgn(\tau)u_{i_{\tau(1)}}^*(u_{j_1})\cdots u_{i_{\tau(k)}}^*(u_{j_k}) = \delta_{I,J}$.\retTwo\par}
	\end{myIndent}
\end{itemize}

\ul{Proposition 1.4.24:} Suppose $F$ has characteristic $0$ or greater than $k$. Then\\ $\{\Psi_J : J \text{ is increasing}\}$ is a basis for $\mathcal{A}^k(V)$.

\begin{myIndent}\exTwoP
	Proof:\\
	Suppose $T \in \mathcal{A}^k(V)$. By theorem 1.3.13, we know there exists $a_I \in F$ such that $T = \sum_{I} a_I u_I^*$. However, we also know that $k!T = \Alt(T)$. Since $\Alt$ is a linear map, we thus know that:

	{\centering $T = \frac{1}{k!}\Alt(T) = \sum\limits_I \frac{a_I}{k!} \Alt(u_I^*) = \sum\limits_I \frac{a_I}{k!} \Psi_I$\newpage\par}

	If $I$ is repeating, then $\frac{a_I}{k!} \Psi_I^*$  cancels. Otherwise, there is some $\sigma \in S_k$ and some increasing multi-index $J$ such that:
	
	{\centering $\frac{a_I}{k!} \Psi_I = \frac{a_I}{k!} \Psi_{J^\sigma} = \frac{a_I\sgn(\sigma)}{k!} \Psi_{J}$.\retTwo\par}

	By collecting terms, we get that $T = \hspace{-1em}\sum\limits_{J \text{ increasing}}\hspace{-1em}c_J\Phi_J$ where each $c_J \in F$.\retTwo

	This shows that the $\Psi_J$ span all of $\mathcal{A}^k(V)$. Next we show that they form a basis.\\ Suppose $T = \hspace{-1em}\sum\limits_{J \text{ increasing}}\hspace{-1em}c_J\Phi_J = 0$.\retTwo

	Then by part (c) of the last proposition, we know that if $I = (i_1, \ldots, i_k)$ is an\\ increasing multi-index, then:

	{\centering $0 = T(u_{i_1}, \ldots, u_{i_k}) = C_I$ \retTwo\par}

	So, all the $C_I$ are equal to $0$. $\blacksquare$\retTwo
\end{myIndent}

\ul{Corollary:} If $F$ has characteristic $0$ or greater than $k$, then $\mathcal{A}^k(V)$ has dimension $\binom{n}{k}$.\retTwo

\ul{Corollary 2:} If $F$ has characteristic $0$ or greater than $k \geq n$, then any alternating $n$-tensor on $V$ is a scalar multiple of a determinant function. Also, there are no\\ nontrivial alternating $m$-tensors where $n < m \leq k$.\retTwo

\pracOne\ul{Exercise 1.4.ix:} Suppose $A: V \to W$ is a linear map. Then if $T \in \mathcal{A}^k(W)$, we have that $A^\dag T \in \mathcal{A}^k(V)$. Hence, the pullback operation maps alternating tensors to alternating tensors.

\begin{myIndent}\pracTwo
	Proof:\\
	Suppose $\sigma \in S_k$. Then for any $v_1, \ldots, v_k \in V$, we have that:

	{\centering\begin{tabular}{l}
		$(A^\dag T)^\sigma(v_1, \ldots, v_k) = A^\dag T(v_{\sigma^{-1}(1)}, \ldots, v_{\sigma^{-1}(k)})$\\ [6pt]

		$\phantom{(A^\dag T)^\sigma(v_1, \ldots, v_k)} = T(Av_{\sigma^{-1}(1)}, \ldots, Av_{\sigma^{-1}(k)})$\\ [6pt]

		$\phantom{(A^\dag T)^\sigma(v_1, \ldots, v_k)} = T^\sigma(Av_1, \ldots, Av_k)$\\ [6pt]

		$\phantom{(A^\dag T)^\sigma(v_1, \ldots, v_k)} = \sgn(\sigma)T(Av_1, \ldots, Av_k) = \sgn(\sigma)A^\dag T(v_1, \ldots, v_k)$
	\end{tabular}\retTwo\par}

	Hence, $(A^\dag T)^\sigma = \sgn(\sigma)A^\dag T$. This proves that $A^\dag T$ is alternating. $\blacksquare$\retTwo
\end{myIndent}

\ul{Exercise 1.4.x:} Additionally to the last exercise, we have that if $T \in \mathcal{L}^k(V)$, then\\ $A^\dag(\Alt(T)) = \Alt(A^\dag T)$.

\begin{myIndent}\pracTwo
	Proof:\\
	If $v_1, \ldots, v_k \in V$, then:

	{\centering\begin{tabular}{l}
		$\Alt(A^\dag T)(v_1, \ldots, v_k) = \sum\limits_{\tau \in S_k}\sgn(\tau)(A^\dag T )^\tau(v_1, \ldots, v_k)$\\ [16pt]

		$\phantom{\Alt(A^\dag T)(v_1, \ldots, v_k)} = \sum\limits_{\tau \in S_k}\sgn(\tau)A^\dag T(v_{\tau^{-1}(1)}, \ldots, v_{\tau^{-1}(k)})$\\ [16pt]

		$\phantom{\Alt(A^\dag T)(v_1, \ldots, v_k) } = \sum\limits_{\tau \in S_k}\sgn(\tau)T(Av_{\tau^{-1}(1)}, \ldots, Av_{\tau^{-1}(k)})$\\ [-24pt]

		$\phantom{\Alt(A^\dag T)(v_1, \ldots, v_k) = \Alt(T)(Av_1, \ldots, Av_k) = A^\dag(\Alt (T))(v_1, \ldots, v_k)}$
	\end{tabular}\newpage
	
	\begin{tabular}{l}
		$\phantom{\Alt(A^\dag T)(v_1, \ldots, v_k)} = \sum\limits_{\tau \in S_k}\sgn(\tau)T^\tau(Av_1, \ldots, Av_k)$\\ [16pt]

		$\phantom{\Alt(A^\dag T)(v_1, \ldots, v_k) } = \Alt(T)(Av_1, \ldots, Av_k) = A^\dag(\Alt (T))(v_1, \ldots, v_k)$
	\end{tabular}\retTwo\par}

	This shows that $\Alt(A^\dag T) =  A^\dag(\Alt (T))$. $\blacksquare$\retTwo
\end{myIndent}

\hOne\blab{The space $\Lambda^k(V^*)$:}\\
If $k > 1$, a decomposable $k$-tensor $\ell_1 \otimes \cdots \otimes \ell_k$ with each $\ell_i \in V^*$ is called \udefine{redundant} if $\ell_i = \ell_{i+1}$ for some index $i$. We let $\mathcal{I}^k(V)$ be the span of all redundant $k$-tensors.

\begin{myIndent}\hTwo 
	If $k = 1$, we define $\mathcal{I}^1(V) \coloneq \{0\} \subseteq \mathcal{L}^1(V)$.\\ Also if $k = 0$, we define $\mathcal{I}^0(V) \coloneq \{0\} \subseteq F$.\retTwo 
\end{myIndent}

\exOne\ul{Proposition 1.5.2:} Suppose $F$ has characteristic $\neq 2$. If $T \in \mathcal{I}^k(V)$, then $\Alt(T) = 0$. In other words, $\mathcal{I}^k(V) \subseteq \ker(\Alt)$.

\begin{myIndent}\exTwoP
	Proof:\\
	If $T \in \mathcal{I}^k(V)$, then we know there are redundant decomposable $k$-tensors\\ $T_1, \ldots, T_m$ as well as scalars $c_1, \ldots, c_m \in F$ such that $T = \sum_{j=1}^m c_jT_j$. Then since\\ $\Alt(T) = \sum_{j=1}^m c_j\Alt(T_j)$, all we need to do now is show that $\Alt(T_j) = 0$ for\\ every $j$.\retTwo

	Since $T_j$ is a redundant decomposable $k$-tensor, we know that $T_j = \ell_1 \otimes \cdots \otimes \ell_k$ where $\ell_i = \ell_{i+1}$ for some $1 \leq i < k$. In turn, if $\tau_{i, i+1} \in S_k$ is the transposition of $i$ and $i+1$, we have that $(T_j)^{\tau_{i,i+1}} = T_j$ and $\sgn(\tau_{i, i+1}) = -1$. Hence:

	{\centering$\Alt(T_j) = \Alt((T_j)^{\tau_{i, i+1}}) = \sgn(\tau_{i,i+1})\Alt(T_j) = -\Alt(T_j)$\retTwo\par}

	This implies that $\Alt(T_j) = 0$. $\blacksquare$\retTwo
\end{myIndent}

\ul{Proposition 1.5.3:} If $T \in \mathcal{I}^r(V)$ and $T^\prime \in \mathcal{L}^s(V)$, then $T \otimes T^\prime$ and $T^\prime \otimes T$ are in $\mathcal{I}^{r+s}(V)$.

\begin{myIndent}\exTwoP
	Proof:\\
	The argument for $T^\prime \otimes T$ being in $\mathcal{I}^{r+s}(V)$ is mostly identical to the argument for $T \otimes T^\prime$  being in $\mathcal{I}^{r+s}(V)$. So, I'll focus only on proving the latter.\retTwo

	To start off, like before we know that there are redundant decomposable $r$-tensors $T_1, \ldots, T_m$ as well as scalars $c_1, \ldots, c_m \in F$ such that $T = \sum_{j=1}^m c_jT_j$. Hence, it suffices to show that $T_j \otimes T^\prime \in \mathcal{I}^{r+s}(V)$ for all $1 \leq j \leq m$ since:

	{\center$T \otimes T^\prime = (\sum_{j=1}^m c_jT_j) \otimes T^\prime = \sum_{j=1}^m c_j (T_j \otimes T^\prime)$\retTwo\par}

	Fortunately, by writing $T^\prime = \sum_{I} d_Iu_I^*$, we can see that:

	{\centering $T_j \otimes T^\prime = T_j \otimes (\sum\limits_{I} d_I u_I^*) = \sum\limits_{I} d_I(T_j \otimes u_I^*)$\retTwo\par}

	Now since both $u_I^*$ and $T_j$ are decomposable and $T_j$ is redundant, we can easily see that $T_j \otimes u_I^*$ is decomposable and redundant. It follows that $T_j \otimes T^\prime \in \mathcal{I}^{r+s}(V)$. $\blacksquare$\retTwo
\end{myIndent}

\ul{Proposition 1.5.4:} Suppose $F$ has characteristic $\neq 2$. If $T \in \mathcal{L}^k(V)$ and $\sigma \in S_k$, then $T^\sigma = \sgn(\sigma)T + S$ where $S \in \mathcal{I}^k(V)$.\newpage

\begin{myIndent}\exTwoP
	Proof:\\
	Hopefully you're getting use to this trick. It suffices to assume $T$ is decomposable. After all, after writing $T = \sum_{I} c_Iu_I^*$, if we can show for all multi-indexes $I$ that $(u_{I}^*)^\sigma = \sgn(\sigma)u_{I}^* + S_I$ where $S_I \in \mathcal{I}^k(V)$, then we can set $S = \sum_{I} c_I S_I \in \mathcal{I}^k(V)$ and have that:

	{\centering $T^\sigma = \sum\limits_{I} c_I (u_I^*)^\sigma = \sgn(\sigma)\sum\limits_{I} c_I u_I^* + \sum\limits_{I} S_I = \sgn(\sigma)T + S$\retTwo\par}
	
	So suppose $T = \ell_1 \otimes \cdots \otimes \ell_k$. Then given $\sigma \in S^k$, we can write $\sigma = \tau_1 \ldots \tau_m$ as the product of $m$ many transpositions of adjacent pairs of numbers in $\{1, \ldots, k\}$.
	\begin{myIndent}\exPPP
		(by adjacent I mean a pair $\{j, j+1\}$ where $1 \leq j < k$\dots)\retTwo
	\end{myIndent}

	We shall induct on $m$. First assume $m = 1$. Thus $\sigma = \tau_{j,j+1}$ for some $1 \leq j < k$ and hence $\sgn(\sigma) = -1$. Also:

	{\centering\begin{tabular}{l}
		$T^\sigma -\sgn(\sigma)T = T^\sigma + T$\\ [3pt]

		$\phantom{T^\sigma -\sgn(\sigma)T} = (\ell_1 \otimes \cdots \otimes \ell_{j-1} \otimes \ell_{j+1} \otimes \ell_{j} \otimes \ell_{j+2} \otimes \cdots \otimes \ell_{k}) + (\ell_1 \otimes \cdots \otimes \ell_k)$\\ [3pt]

		$\phantom{T^\sigma -\sgn(\sigma)T} = (\ell_1 \otimes \cdots \otimes \ell_{j-1}) \otimes \left((\ell_{j+1} \otimes \ell_j) + (\ell_j \otimes \ell_{j+1})\right) \otimes (\ell_{j+2} \otimes \cdots \otimes \ell_{k})$
	\end{tabular}\retTwo\par}

	Now note that:

	{\centering\begin{tabular}{l}
		$(\ell_{j} + \ell_{j+1}) \otimes (\ell_{j} + \ell_{j+1}) = (\ell_j \otimes \ell_j) + (\ell_j \otimes \ell_{j+1}) + (\ell_{j+1} \otimes \ell_j) + (\ell_{j+1} \otimes \ell_{j+1})$.
	\end{tabular}\retTwo\par}

	Therefore:

	{\centering\begin{tabular}{l}
		$T^\sigma - \sgn(\sigma)T = (\ell_1 \otimes \cdots \otimes \ell_{j-1}) \otimes (\ell_j + \ell_{j+1}) \otimes (\ell_j + \ell_{j+1}) \otimes (\ell_{j+1} \otimes \cdots \otimes \ell_{k})$\\
		$\phantom{T^\sigma - \sgn(\sigma)T = aaaaaaaaaaa} - (\ell_1 \otimes \cdots \otimes \ell_{j-1}) \otimes \ell_j \otimes \ell_j \otimes (\ell_{j+2} \otimes \cdots \otimes \ell_{k})$\\
		$\phantom{T^\sigma - \sgn(\sigma)T = aaaaaaaaaaa} - (\ell_1 \otimes \cdots \otimes \ell_{j-1}) \otimes \ell_{j+1} \otimes \ell_{j+1} \otimes (\ell_{j+2} \otimes \cdots \otimes \ell_{k})$\\
	\end{tabular}\retTwo\par}

	Hence $T^\sigma - \sgn(\sigma)T \in \mathcal{I}^k(V)$ and we are done with this case.\retTwo

	Now suppose $m > 1$. Then $\sigma = \tau_{j,j+1} \sigma^\prime$ where $\sigma^\prime$ is the product of $m-1$\\ transpositions. By induction, we know that there exists $S_1 \in \mathcal{I}^k(V)$ such that:

	{\centering $T^\sigma = (T^{\tau_{j,j+1}})^{\sigma^\prime} = \sgn(\sigma^\prime)T^{\tau_{j,j+1}} + S_1$ \retTwo\par}

	Also by our base case, there is $S_2 \in \mathcal{I}^k(V)$ such that $T^{\tau_{j,j+1}} = \sgn(\tau_{j,j+1})T + S_2$. Then setting $S = \sgn(\tau_{j,j+1})S_2 + S_1$, we have that $S \in \mathcal{I}^K(V)$ and:

	{\centering $T^\sigma = \sgn(\sigma^\prime)(\sgn(\tau_{j,j+1})T + S_2) + S_1 = \sgn(\sigma)T + S$. $\blacksquare$\retTwo\par}
\end{myIndent}

\ul{Corollary 1.5.6:} Suppose $F$ has characteristic $\neq 2$. If $T \in \mathcal{L}^k(V)$, then\\ $\Alt(T) = k!T + S$ where $S \in \mathcal{I}^k(V)$.

\begin{myIndent}\exTwoP
	Proof:\\
	Given any $\tau \in S_k$, let $S_\tau \in \mathcal{I}^k(V)$ be such that $T^\tau = \sgn(\tau) T + S_\tau$. Then\\ $S \coloneq \sum_{\tau \in S_k} \sgn(\tau)S_\tau \in \mathcal{I}^k(V)$ and:

	{\center $\Alt(T) = \sum\limits_{\tau \in S_k}\sgn(\tau)T^\tau = \sum\limits_{\tau \in S_k} (\sgn(\tau))^2 T + \sum\limits_{\tau \in S_k} \sgn(\tau)S_{\tau} = k!T + S$. $\blacksquare$ \retTwo\par}
\end{myIndent}

\ul{Corollary 1.5.8:} Let $k \geq 1$. Then let $V$ be a vector space over a field $F$ of\\ characteristic $0$ or $> \max(k, 2)$. Then:

{\centering $\mathcal{I}^k(V) = \ker(\Alt: \mathcal{L}^k(V) \to \mathcal{A}^k(V))$ \retTwo\par}

\begin{myIndent}\exTwoP
	Proof:\\
	We already know from proposition 1.5.2 that $\mathcal{I}^k(V) \subseteq \ker(\Alt)$. To prove the\\ reverse relation, suppose $T \in \mathcal{L}^k(V)$ satisfies that $\Alt(T) = 0$. Then based on the previous corollary, we know there exists $S \in \mathcal{I}^k(V)$ such that $-\frac{1}{k!}S = T - \Alt(T)$. Hence $T \in \mathcal{I}^k(V)$. $\blacksquare$\retTwo
\end{myIndent}

\ul{Theorem 1.5.9:} Suppose $F$ is a field of characteristic $0$ or $> \max(k, 2)$. Then any element $T \in \mathcal{L}^k(V)$ can be written uniquely as a sum $T_1 + T_2$ where $T_1 \in \mathcal{A}^k(V)$ and $T_2 \in \mathcal{I}^k(V)$. I.e, $\mathcal{L}^k(V) = \mathcal{A}^k(V) \oplus \mathcal{I}^k(V)$.

\begin{myIndent}\exTwoP
	Proof:\\
	Let $W \in \mathcal{I}^k(V)$ satisfy that $\Alt(T) = k!T + W$. Then set $T_1 = \frac{1}{k!}\Alt(T)$ and\\ $T_2 = -\frac{1}{k!}W$. Then clearly $T = T_1 + T_2$ with $T_1 \in \mathcal{A}^k(V)$ and $T_2 \in \mathcal{I}^k(V)$.\retTwo

	Next, to prove uniqueness suppose $T^\prime_1 + T^\prime = T$ with $T^\prime_1 \in \mathcal{A}^k(V)$ and $T^\prime_2 \in \mathcal{I}^k(V)$. Then $T_1 - T^\prime_1 \in \mathcal{A}^k(V)$, $T_2 - T^\prime_2 \in \mathcal{I}^k(V)$, and $(T_1 - T^\prime_1) + (T_2 - T_2^\prime) = 0$. So:

	{\center $0 = \Alt(0) = \Alt((T_1 - T^\prime_1) + (T_2 - T_2^\prime)) = k!(T_1 - T^\prime_1)$ \retTwo\par}

	Hence $T_1 = T^\prime_1$ and it easily follows $T_2 = T^\prime_2$. $\blacksquare$\retTwo
\end{myIndent}

\hOne Let $k \geq 0$. Let $V$ be a finite dimensional vector space over a field $F$ of characteristic $0$ or $> \max(k, 2)$. Then we define:

{\centering $\Lambda^k(V^*) \coloneq \mathcal{L}^k(V) / \mathcal{I}^k(V)$ \retTwo\par}

By the first isomorphism theorem along with the previous theorem, we have that $\Lambda^k(V^*) \cong \mathcal{A}^k(V)$.\retTwo

\dispDate{8/8/2025}

Here is a tangent  about symmetric tensors. For this section, suppose $V$ is an\\ $n$-dimensional vector space over a field $F$ with characteristic $\neq 2$.\retTwo

A tensor $T \in \mathcal{L}^k(V)$ is \udefine{symmetric} if $T^\sigma = T$ for all $\sigma \in S_k$. We denote the space of\\ symmetric tensors $\mathcal{S}^k(V)$.
\begin{myIndent}\hTwo
	You can show by the same reasoning as with $\mathcal{A}^k(V)$ that $\mathcal{S}^k(V)$ is a vector subspace.\retTwo
\end{myIndent}

\pracOne\ul{Exercise 1.5.iii:} Suppose $F$ has characteristic $0$ or $> k$. Then if $T$ is a symmetric $k$-tensor and $k \geq 2$, we have that $T \in \mathcal{I}^k(V)$.

\begin{myIndent}\pracTwo
	Proof:\\
	Let $\sigma \in S_k$ be an odd permutation. Then by proposition 1.4.17:
	
	{\centering $\Alt(T) = \Alt(T^\sigma) = \sgn(\sigma)\Alt(T) = -\Alt(T)$.\newpage\par}

	The only way this is possible is if $\Alt(T) = 0$. Hence $T \in \ker(\Alt)$, and by theorem 1.5.8 that means that $T \in \mathcal{I}^k(V)$. $\blacksquare$\retTwo
\end{myIndent}

\hOne We define a \udefine{symmetrization} operator as follows. Given $T \in \mathcal{L}^k(V)$, define:

{\centering $\Sym(T) \coloneqq \sum\limits_{\sigma \in S_k} T^\sigma$ \retTwo\par}

Then like in proposition 1.4.17, we can show that given any $T \in \mathcal{L}^k(V)$ and $\sigma \in S_k$:
\begin{itemize}\hTwo
	\item[(a)] $(\Sym(T))^\sigma = \Sym(T)$ (i.e. $\Sym(T) \in \mathcal{S}^k(V)$\dots)
	\item[(b)] If $T \in \mathcal{S}^k(V)$, then $\Sym(T) = k!T$
	\item[(c)] $\Sym(T^\sigma) = \Sym(T)$
	\item[(d)] $\Sym: \mathcal{L}^k(V) \to \mathcal{A}^k(V)$ is an linear map (which is surjective so long as $k! \neq 0$ in the field $F$\dots).\retTwo 
\end{itemize}

Supposing $k! \neq 0$ in $F$, then by following a process very similar to what we did with the alternation operation, we can construct a basis for the symmetric tensors:

{\centering$\{\Phi_I^*: I \text{ is a non-decreasing multi-index}\}$.\retTwo\par}

\begin{myDindent}\pracTwo
	Note: By non-decreasing the textbook means that $I = (i_1, \ldots, i_k)$ satisfies that\\ $i_1 \leq i_2 \leq \cdots \leq i_k$.\retTwo
\end{myDindent}

I'm bored and won't do that construction here. But the important point is that this means $\mathcal{S}^k(V)$ has the same number of dimensions as there are non-decreasing multi-indexes of $n$ of length $k$. And since there are $\binom{n+k-1}{k}$ ways of picking $k$\\ elements of the set $\{1, \ldots, n\}$ when you allow yourself to pick the same element multiple times, this means that $\dim(\mathcal{S}^k(V)) = \binom{n+k-1}{k}$.

\begin{myIndent}\pracOne
	Side note: If $k! \neq 0$ in $F$, then we have already shown that $\dim(\mathcal{I}^k(V)) = n^k - \binom{n}{k}$. Since $\mathcal{S}^k(V) \subseteq \mathcal{I}^k(V)$, this shows that $\mathcal{S}^2(V) = \mathcal{I}^2(V)$. That said, we don't in general have that $\dim(\mathcal{I}^k(V)) = \dim(\mathcal{S}^k(V))$ when $k > 2$.\retTwo
\end{myIndent}

Next, here's some other miscellaneous results.\retTwo

\pracOne\ul{Exercise 1.5.vii:} Suppose $F$ has characteristic $0$ or $> \max(k, 2)$. Then if $T \in \mathcal{I}^k(V)$, we have that $T^\sigma \in \mathcal{I}^k(V)$ for all $\sigma \in S_k$.

\begin{myIndent}\pracTwo
	Proof:\\
	Since $T \in \mathcal{I}^k(V)$, we know that: $\Alt(T^\sigma) = \sgn(\sigma)\Alt(T) = 0$. Therefore,\\ $T^\sigma \in \ker(\Alt)$, and by corollary 1.5.8 we know that $\ker(\Alt) = \mathcal{I}^k(V)$. $\blacksquare$\retTwo
\end{myIndent}

\pracOne\ul{Corollary / Exercise 1.5.v:} Let $k \geq 2$ and suppose $F$ has characteristic $0$ or $> k$. Then if\\ $T \in \mathcal{L}^{k-2}(V)$ and $\ell \in V^*$, we have that $\ell \otimes T \otimes \ell \in \mathcal{I}^k(V)$.

\begin{myIndent}\pracTwo
	Proof:\\
	There is a permutation $\sigma \in S_k$ satisfying that $(\ell \otimes T \otimes \ell)^\sigma = (\ell \otimes \ell) \otimes T$. Then by proposition 1.5.3, we have that $(\ell \otimes \ell) \otimes T \in \mathcal{I}^k(V)$. And finally, by applying the last exercise we have that $\ell \otimes T \otimes \ell = ((\ell \otimes \ell) \otimes T)^{\sigma^{-1}} \in \mathcal{I}^k(V)$. $\blacksquare$\newpage
\end{myIndent}

\pracOne\ul{Corollary / Exercise 1.5.vi:} Let $k \geq 2$ and suppose $F$ has characteristic $0$ or $> k$. Then if\\ $T \in \mathcal{L}^{k-2}(V)$ and $\ell_1, \ell_2 \in V^*$, we have that $(\ell_1 \otimes T \otimes \ell_2) + (\ell_2 \otimes T \otimes \ell_1) \in \mathcal{I}^k(V)$.

\begin{myIndent}\pracTwo
	Proof:\\
	Apply the last exercise plus the fact that:

	{\centering\begin{tabular}{l}
	$(\ell_1 \otimes T \otimes \ell_2) + (\ell_2 \otimes T \otimes \ell_1) = ((\ell_1 + \ell_2) \otimes T \otimes (\ell_1 + \ell_2))$\\
	$\phantom{(\ell_1 \otimes T \otimes \ell_2) + (\ell_2 \otimes T \otimes \ell_1)aaaa} - (\ell_1 \otimes T \otimes \ell_1) - (\ell_2 \otimes T \otimes \ell_2)$. $\blacksquare$
	\end{tabular}\retTwo\par}
\end{myIndent}

\hOne\dispDate{8/9/2025}

\blab{The Wedge Product:}\\
In this section, we'll suppose $V$ is an $n$-dimensional vector space over a field $F$\\ of characteristic $0$. Also, we shall for each $k$ define the map $\pi: \mathcal{L}^k(V) \to \Lambda^k(V^*)$\\ such that $\pi(T) = T + \mathcal{I}^k(V)$ for all $T \in \mathcal{L}^k(V)$.\retTwo

Suppose for each $i \in \{1, 2\}$ we have $\omega_i \in \Lambda^{k_i}(V^*)$. Then if for each $i$ we are given $T_i \in \mathcal{L}^{k_i}(V)$ satisfying that $\pi(T_i) = \omega_i$, we define:

{\centering $\omega_1 \wedge \omega_2 \coloneqq \pi(T_1 \otimes T_2)$.\retTwo\par}

\exOne\ul{Claim 1.6.3:} The wedge product is well defined. 

\begin{myIndent}\exTwoP
	Proof:\\
	Suppose for each $i \in \{1,2\}$ that we also have $T_i^\prime \in \mathcal{L}^{k_i}(V)$ satisfying that\\ $\pi(T_i^\prime) = \pi(T_i) = \omega_i$. Then for each $i$ there exists $W_i \in \mathcal{I}^k(V)$ such that\\ $T_i^\prime = T_i + W_i$. Hence:

	{\centering$\pi(T_1^\prime \otimes T_2^\prime) = \pi((T_1 \otimes T_2) + (T_1 \otimes W_2) + (W_1 \otimes T_2) + (W_1 \otimes W_2))$\retTwo\par}

	Then by applying proposition 1.5.3, we know that:
	
	{\centering$(T_1 \otimes W_2) + (W_1 \otimes T_2) + (W_1 \otimes W_2) \in\mathcal{I}^k(V)$\retTwo\par}

	Hence, $\pi(T_1^\prime \otimes T_2^\prime) = \pi(T_1 \otimes T_2)$. $\blacksquare$
	\retTwo
\end{myIndent}

\hOne More generally, if for each $i \in \{1, \ldots, m\}$ we have $\omega_i \in \Lambda^{k_i}(V^*)$ and $T_i \in \mathcal{L}^{k_i}(V)$ satsifying that $\pi(T_i) = \omega_i$, then we define:

{\centering $\omega_1 \wedge \cdots \wedge \omega_m \coloneq \pi(T_1 \otimes \cdots \otimes T_m)$\retTwo\par}

\begin{myIndent}\exTwoP
	This is well defined for basically the same reasoning as before, although to avoid some overly long expressions, it suffices to replace only one tensor at a time.\retTwo
\end{myIndent}

\exOne\ul{Claim:} Given any $m \geq 3$, we have that:

{\centering$\omega_1 \wedge (\omega_2 \wedge \cdots \wedge \omega_{m}) = \omega_1 \wedge \cdots \wedge \omega_m = (\omega_1 \wedge \cdots \wedge \omega_{m-1}) \wedge \omega_m$\retTwo\par}

\begin{myIndent}\exTwoP
	Proof:\\
	If for each $i \in \{1, \ldots, m\}$ we have some $T_i \in \mathcal{L}^{k_i}(V)$ satisfying that $\pi(T_i) = \omega_i$, then $\pi(T_2 \otimes \cdots \otimes T_{m}) = \omega_2 \wedge \cdots \wedge \omega_{m}$ and $\pi(T_1 \otimes \cdots \otimes T_{m-1}) = \omega_1 \wedge \cdots \wedge \omega_{m-1}$. In turn:

	{\centering \begin{tabular}{l}
		$\omega_1 \wedge \cdots \wedge \omega_m = \pi(T_1 \otimes \cdots \otimes T_m)$\\

		$\phantom{\omega_1 \wedge \cdots \wedge \omega_m} = \pi(T_1 \otimes (T_2 \otimes \cdots \otimes T_m)) = \omega_1 \wedge (\omega_2 \wedge \cdots \wedge \omega_{m})$\\

		$\phantom{\omega_1 \wedge \cdots \wedge \omega_m} = \pi((T_1 \otimes \cdots \otimes T_{m-1}) \otimes T_m) = (\omega_1 \wedge \cdots \wedge  \omega_{m-1}) \wedge \omega_m$
	\end{tabular}\newpage\par}
\end{myIndent}

\ul{Corollary:} The wedge product is associative and we get the same result no matter how we use parentheses to group together the $\omega_i$.

\begin{myIndent}\exTwoP
	Proof:\\
	Suppose we have $\omega_1, \ldots, \omega_m$. If $m = 3$, then we're already done by the last claim. Meanwhile, for $m > 3$ is suffices due to the strong inductive hypothesis on $m$ to show that for any $1 \leq s < r \leq m$ with not both $s = 1$ and $r = m$:

	{\centering $\omega_1 \wedge \cdots \wedge \omega_m = \omega_1 \wedge \cdots \wedge \omega_{s-1} \wedge (\omega_s \wedge \cdots \wedge \omega_r) \wedge \omega_{r+1} \wedge \cdots \wedge \omega_m$ \retTwo\par}

	Luckily, when focusing on the case that $r \neq m$, note that by the previous claim as well as the strong inductive hypothesis:

	{\centering\begin{tabular}{l}\HexTwoP
		$\omega_1 \wedge \cdots \wedge \omega_{s-1} \wedge (\omega_s \wedge \cdots \wedge \omega_r) \wedge \omega_{r+1} \wedge \cdots \wedge \omega_{m-1} \wedge \omega_m$\\

		$\phantom{aaaaaaa} = (\omega_1 \wedge \cdots \wedge \omega_{s-1} \wedge (\omega_s \wedge \cdots \wedge \omega_r) \wedge \omega_{r+1} \wedge \cdots \wedge \omega_{m-1}) \wedge \omega_m$\\

		$\phantom{aaaaaaa} = (\omega_1 \wedge \cdots \wedge \omega_{m-1}) \wedge \omega_m = \omega_1 \wedge \cdots \wedge \omega_m$\\
	\end{tabular} \retTwo\par}

	The other case is analogous.\retTwo
\end{myIndent}

\hOne Here are some other properties of the wedge product which I'm too bored to\\ properly write proofs for:\\ [-22pt]
\begin{itemize}
	\item If $\lambda \in F$, then $\lambda (\omega_1 \wedge \omega_2) = (\lambda \omega_1) \wedge \omega_2 = \omega_1 \wedge (\lambda \omega_2)$.\\ [-18pt]
	\item $(\omega_1 + \omega_2) \wedge \omega_3 = (\omega_1 \wedge \omega_3) + (\omega_2 \wedge \omega_3)$\\ [-18pt]
	\item $\omega_1 \wedge (\omega_2 + \omega_3) = (\omega_1 \wedge \omega_2) + (\omega_1 \wedge \omega_3)$
\end{itemize}

\begin{myDindent}\pracOne\fontsize{12}{13}\selectfont
	Side note: if we were instead writing the definition of the wedge product in terms of alternating tensors, we'd be defining:
	
	{\centering$T_1 \wedge \cdots \wedge T_m = \frac{1}{(k_1 + \ldots + k_m)!}\Alt(T_1 \otimes \cdots \otimes T_m)$.\retTwo\par}

	Hopefully its obvious why this definition is inferior.\retTwo
\end{myDindent}

Note that since $\mathcal{I}^1(V) = \{0\}$, we can just identify $\Lambda^1(V^*) = V^*$. Then given $\ell_1, \ldots, \ell_k \in V^* = \Lambda^1(V^*)$, we say that $\omega = \pi(\ell_1 \otimes \cdots \otimes \ell_k) = \ell_1 \wedge \cdots \wedge \ell_k$ is a \udefine{decomposable} element of $\Lambda^k(V^*)$.\retTwo

\exOne\ul{Claim:} For any $\sigma \in S_k$, we have: $\ell_{\sigma(1)} \wedge \cdots \wedge \ell_{\sigma(k)} = \sgn(\sigma)\ell_1 \wedge \cdots \wedge \ell_k$.

\begin{myIndent}\exTwoP
	Proof:\\
	\ul{Lemma:} If $T \in \mathcal{L}^k(V)$ and $\sigma \in S_k$, then $\pi(T^\sigma) = \sgn(\sigma)\pi(T)$.
	\begin{myIndent}\exPPP
		This is just a consequence of proposition 1.5.4.\retTwo
	\end{myIndent}

	As a result of that lemma:
	
	{\centering\begin{tabular}{l}
		$\ell_{\sigma(1)} \wedge \cdots \wedge \ell_{\sigma(k)} = \pi(\ell_{\sigma(1)} \otimes \cdots \otimes \ell_{\sigma(k)})$\\
		$\phantom{\ell_{\sigma(1)} \wedge \cdots \wedge \ell_{\sigma(k)}} = \pi((\ell_{1} \otimes \cdots \otimes \ell_{k})^\sigma)$\\
		$\phantom{\ell_{\sigma(1)} \wedge \cdots \wedge \ell_{\sigma(k)}} = \sgn(\sigma)\pi(\ell_1 \otimes \cdots \otimes \ell_k)$\\
		$\phantom{\ell_{\sigma(1)} \wedge \cdots \wedge \ell_{\sigma(k)}} = \sgn(\sigma)\ell_1 \wedge \cdots \wedge \ell_k$
	\end{tabular}\retTwo\par}
\end{myIndent}

\hOne As a corollary, given any $\ell_1, \ell_2 \in V^*$ we have that $\ell_1 \wedge \ell_2 = -\ell_2 \wedge \ell_1$. Also, given $\ell_1, \ell_2, \ell_3 \in V^*$, we have:

{\centering\begin{tabular}{l}
	$\ell_1 \wedge \ell_2 \wedge \ell_3 = -\ell_2 \wedge \ell_1 \wedge \ell_3 = \ell_2 \wedge \ell_3 \wedge \ell_1$\\
	$\phantom{\ell_1 \wedge \ell_2 \wedge \ell_3} = -\ell_1 \wedge \ell_3 \wedge \ell_2 = \ell_3 \wedge \ell_1 \wedge \ell_2$
\end{tabular}\newpage\par}

Let $u_1, \ldots, u_n$ be a basis for $V$ and let $u_1^*, \ldots, u_n^*$ be the corresponding dual basis. Then the collection of $u_{i_1}^* \wedge \cdots \wedge u_{i_k}^*$ such that $I = (i_1, \ldots, i_k)$ is an increasing multi-index forms a basis for $\Lambda^k(V^*)$. 

\begin{myIndent}\exTwoP
	Proof:\\
	Recall that when defining $u_I^* = u_{i_1}^* \otimes \cdots \otimes u_{i_k}$ for a multi-index $I = (i_1, \ldots, i_k)$,\\ we then have that the $\Psi_I \coloneq \Alt(u_I^*)$ where $I$ is increasing form a basis of $\mathcal{A}^k(V)$. It follows that each $\pi(\Psi_I)$ where $I$ is increasing is a basis vector of $\Lambda^k(V^*)$. But note that:

	{\centering\exThreeP\begin{tabular}{l}
		$\pi(\Psi_I) = \pi\left(\sum\limits_{\tau \in S_k}\sgn(\tau)(u_{I}^*)\tau\right) = \sum\limits_{\tau \in S_k}\sgn(\tau)\pi((u_{I}^*)^\tau) = \sum\limits_{\tau \in S_k}(\sgn(\tau))^2\pi(u_{I}^*) = k!\pi(u_I^*)$
	\end{tabular}\retTwo\par}

	So, the $\pi(u_I^*) = u_{i_1} \wedge \cdots \wedge u_{i_k}$ also form a basis for $\Lambda^k(V^*)$.\retTwo
\end{myIndent}

This now let's us prove the following general result:
\begin{myIndent}\exTwo
	\ul{Theorem 1.6.10:} If $\omega_1 \in \Lambda^r(V^*)$ and $\omega_2 \in \Lambda^s(V^*)$, then $\omega_1 \wedge \omega_2 = (-1)^{rs}\omega_2 \wedge \omega_1$.

	\begin{myIndent}\exThreeP
		Proof:\\
		Express $\omega_1 = \sum\limits_I c_I u_{i_1}^* \wedge \cdots \wedge u_{i_{r}}^*$ and $\omega_2 = \sum\limits_J d_J u_{j_1}^* \wedge \cdots \wedge u_{j_{s}}^*$.\retTwo

		Then we have that:\\ [-8pt]
		
		{\centering\begin{tabular}{l}
			$\omega_1 \wedge \omega_2 = \sum\limits_{I,J} c_Id_J (u_{i_1}^* \wedge \cdots \wedge u_{i_r}^* \wedge u_{j_1}^* \wedge \cdots \wedge u_{j_s}^*)$\\ [16pt]

			$\phantom{\omega_1 \wedge \omega_2} = \sum\limits_{I,J} c_Id_J(-1)^{rs} (u_{j_1}^* \wedge \cdots \wedge u_{j_s}^* \wedge u_{i_1}^* \wedge \cdots \wedge u_{i_r}^*)$\\ [16pt]

			$\phantom{\omega_1 \wedge \omega_2} = (-1)^{rs}\sum\limits_{I,J} d_Jc_I (u_{j_1}^* \wedge \cdots \wedge u_{j_s}^* \wedge u_{i_1}^* \wedge \cdots \wedge u_{i_r}^*) = (-1)^{rs}\omega_2 \wedge \omega_1$.
		\end{tabular}\retTwo\par}
	\end{myIndent}	
\end{myIndent}

One more note I'd like to make is that we can identify $\Lambda^0(V^*)$ and $F$. Then if\\ $\omega \in \Lambda^k(V^*)$ and $\lambda \in F$, we have that $\lambda \wedge \omega = \lambda\omega = \omega \wedge \lambda$.\retTwo

\dispDate{8/10/2025}

Before moving onto the next section of the book, I'm going to do a few of the\\ exercises.\retTwo

\pracOne\ul{Exercise 1.6.iii:} Given $\omega \in \Lambda^r(V^*)$, we define $\omega^1 \coloneqq \omega$ and $\omega^k \coloneqq \omega \wedge \omega^{k-1} \in \Lambda^{rk}(V^*)$ for all $k > 1$. In other words, $\omega^k$ is the $k$-fold wedge product of $\omega$ with itself.
\begin{itemize}
	\item[(A)] If $r$ is odd, then $\omega^k = 0$ for all $k > 1$.
	\begin{myIndent}\pracTwo
		Proof:\\
		By an easy application of theorem 1.6.10, we have that:
		
		{\centering $\omega^k = \omega \wedge \omega^{k-1} = (-1)^{r\cdot r^{k-1}}\omega^{k-1} \wedge \omega = (-1)^{r^k}\omega^k$ \retTwo\par}

		But $r^k$ is odd if $r$ is odd. Then in turn, $\omega^k = -\omega^k$. The only way this is possible is if $\omega^k = 0$.\newpage
	\end{myIndent}

	\item[(B)] If $\omega$ is decomposable, then $\omega^k = 0$ for all $k > 1$.
	
	\begin{myIndent}\pracTwo
		Proof:\\
		For the ease of notation we'll $\omega^0 = 1 \in F$. Now if $\omega = \ell_1 \wedge \cdots \wedge \ell_r$, then by just swapping two occurences of $\ell_1$, we have that:
		
		{\centering\begin{tabular}{l}
			$\omega^k = \ell_1 \wedge \cdots \wedge \ell_r \wedge \ell_1 \wedge \cdots \wedge \ell_r \wedge \omega^{k-2}$\\
			$\phantom{\omega^k} = (-1)\ell_1 \wedge \cdots \wedge \ell_r \wedge \ell_1 \wedge \cdots \wedge \ell_r \wedge \omega^{k-2} = -\omega^k$
		\end{tabular}\retTwo\par}

		This implies $\omega^k = 0$.\retTwo
	\end{myIndent}
\end{itemize}

\ul{Exercise 1.6.iv:} If $\omega, \mu \in \Lambda^r(V^*)$, then:

{\centering $(\omega + \mu)^k = \sum\limits_{i=0}^k \binom{k}{i} \omega^i \wedge \mu^{k-i}$.\retTwo\par}

\begin{myIndent}\pracTwo
	This is obvious so I'm skipping this problem. I just wanted to write out the result.\retTwo
\end{myIndent}

\hOne\mySepTwo

\blab{The interior Product:}\\
All the assumptions about $V$ and $F$ made yesterday still apply and you should keep assuming them until I tell you to stop (cause I don't want to keep writing this shtick).\retTwo

Given $T \in \mathcal{L}^k(V)$ where $k > 1$ and $v \in V$, we define the $(k-1)$-tensor:

{\centering $\iota_v T(v_1, \ldots, v_{k-1}) \coloneqq \sum\limits_{r = 1}^k (-1)^{r-1}T(v_1, \ldots, v_{r-1}, v, v_r, \ldots, v_{k-1})$\retTwo\par}

Also if $\lambda \in \mathcal{L}^0(V) = F$, we define $\iota_v \lambda = 0$ for all $v \in V$.\retTwo

Note that if $v = c_1v_1 + c_2v_2$ and $T = d_1T_1 + d_2T_2$, then:

{\centering$\iota_v T = c_1\iota_{v_1}T + c_2\iota_{v_2}T$ and $\iota_v T = d_1\iota_v T_1 + d_2\iota_v T_2$.\retTwo\par}

Also, if $T = \ell_1 \otimes \cdots \otimes \ell_k$ where each $\ell_i \in V^*$, then when writing $\hat{\ell_r}$ to mean that we are deleting $\ell_r$ from that term of the expression, we have that:

{\centering $\iota_v T = \sum\limits_{r=1}^k (-1)^{r-1}\ell_r(v) \ell_1 \otimes \cdots \otimes \hat{\ell_r} \otimes \cdots \otimes \ell_k$ \retTwo\par}

Slightly less obviously, if $T_1 \in \mathcal{L}^p(V)$ and $T_2 \in \mathcal{L}^q(V)$, we have that:

{\centering $\iota_{v}(T_1 \otimes T_2) = (\iota_v T_1)\otimes T_2 + (-1)^{p}T_1 \otimes (\iota_v T_2)$.\retTwo\par}

\exOne\ul{Lemma 1.7.8:} Let $V$ be a vector space and $T \in \mathcal{L}^k(V)$ where $k \geq 1$. Then for all $v \in V$, $\iota_v(\iota_v(T)) = 0$.

\begin{myIndent}\exTwoP
	Proof:\\
	By linearity it suffices to prove this statement for decomposable $T$. Also, this statement is trivial when $k = 1$. So, we can proceed by induction, assuming that the theorem holds for $T \in \mathcal{L}^{r}(V)$ where $r < k$. Then after expressing $T = T^\prime \otimes \ell$ where $T^\prime \in \mathcal{L}^{k-1}(V)$ and $\ell \in V^*$, we have that:\newpage

	{\centering \begin{tabular}{l}
		$\iota_v T = \iota_v(T^\prime \otimes \ell) = (\iota_v T^\prime) \otimes \ell + (-1)^{k-1}T^\prime \otimes (\iota_v T^\prime)$\\ [2pt]
		$\phantom{\iota_v T = \iota_v(T^\prime \otimes \ell)} = (\iota_v T^\prime) \otimes \ell + (-1)^{k-1}\ell(v)T^\prime$ 
	\end{tabular}\retTwo\par}

	By induction $\iota_v(\iota_v T^\prime) = 0$. Combining that with the above reasoning shows:

	{\centering \begin{tabular}{l}
		$\iota_v (\iota_v T) = \iota_v((\iota_v T^\prime) \otimes \ell + (-1)^{k-1}\ell(v)T^\prime)$\\ [2pt]
		$\phantom{\iota_v (\iota_v T)} = \iota_v((\iota_v T^\prime) \otimes \ell) + (-1)^{k-1}\ell(v)\iota_v(T^\prime)$\\ [2pt]
		$\phantom{\iota_v (\iota_v T)} = \left(\iota_v(\iota_v T^\prime) \otimes \ell + (-1)^{k-2}(\iota_v T^\prime) \otimes (\iota_v \ell)\right) + (-1)^{k-1}\ell(v)\iota_v(T^\prime)$\\ [2pt]
		$\phantom{\iota_v (\iota_v T)} = 0 + (-1)^{k-2}\ell(v)(\iota_v T^\prime) + (-1)^{k-1}\ell(v)\iota_v(T^\prime) = 0$. $\blacksquare$
	\end{tabular}\retTwo\par}
\end{myIndent}

\ul{Corollary:} If $v_1, v_2 \in V$ and $T \in \mathcal{L}^k(V)$, then $\iota_{v_1}(\iota_{v_2}T) = -\iota_{v_2}(\iota_{v_1}T)$.

\begin{myIndent}\exTwoP
	Proof:\\
	We know from the prior lemma that:

	{\centering $\iota_{v_1 + v_2}(\iota_{v_1 + v_2} T) = 0$ \retTwo\par}

	Therefore:

	{\centering $0 + \iota_{v_1}(\iota_{v_2} T) = \iota_{v_1}(\iota_{v_1 + v_2} T) = - \iota_{v_2}(\iota_{v_1 + v_2} T) = -\iota_{v_2}(\iota_{v_1} T) - 0$ \retTwo\par}
\end{myIndent}

\ul{Lemma 1.7.11:} If $T \in \mathcal{L}^k(V)$ is redundant, then so is $\iota_v T$.
\begin{myIndent}\exTwoP
	Proof:\\
	Write $T = T_1 \otimes \ell \otimes \ell \otimes T_2$ where $\ell \in V^*$, $T_1 \in \mathcal{L}^p(V)$, and $T_2 \in \mathcal{L}^q(V)$. Then:

	{\center$\iota_v T = \iota_v(T_1) \otimes \ell \otimes \ell \otimes T_2 + (-1)^p T_1 \otimes \iota_v(\ell \otimes \ell) \otimes T_2 + (-1)^{p+2} T_1 \otimes \ell \otimes \ell \otimes \iota_v(T_2)$ \retTwo\par}

	Now the first and third terms are obvious redundant. Meanwhile, the second term cancels because $\iota_v(\ell \otimes \ell) = \ell(v)\ell - \ell(v)\ell = 0$. $\blacksquare$\retTwo
\end{myIndent}

\ul{Corollary:} If $T \in \mathcal{I}^k(V)$, then $\iota_v T \in \mathcal{I}^{k-1}(V)$.\retTwo

\hOne Now we define the \udefine{interior product operator} $\iota_v$ on $\Lambda^k(V^*)$. If $\pi$ is the projection\\ of $\mathcal{L}^k(V)$ onto $\Lambda^k(V^*)$ and $\omega = \pi(T) \in \Lambda^{k}(V^*)$, then we define:

{\centering $\iota_v \omega \coloneqq \pi(\iota_v T) \in \Lambda^{k-1}(V^*)$.\retTwo\par}

This is well defined since by the previous corollary, if both $T$ and $T^\prime$ satisfy\\ that $\pi(T) = \pi(T^\prime) = \omega$, then there is some tensor $S \in \mathcal{I}^{k-1}(V)$ such that\\ $\iota_v T = \iota_v T^\prime + S$.\retTwo

It is easily shown then that if $v_1, v_2, v \in V$\hspace{-0.2em},\hspace{0.2em} $\omega, \omega_1 \in \Lambda^p(V^*)$, and $\omega_2 \in \Lambda^q(V^*)$, then:
\begin{itemize}
	\item $\iota_{v_1 + v_2}\omega = \iota_{v_1} \omega + \iota_{v_2} \omega$;
	\item $\iota_{v}(\lambda_1\omega_1 + \lambda_2\omega_2) = \lambda_1\iota_v \omega_1 + \lambda_2\iota_v \omega_2$ (where $\lambda_1, \lambda_2 \in F$);
	\item $\iota_v(\omega_1 \wedge \omega_2) = (\iota_v \omega_1) \wedge \omega_2 + (-1)^p \omega_1 \wedge (\iota_v \omega_2)$.\\
\end{itemize}

Also, if you squint you can see that $\iota_v(\iota_v \omega) = \pi(\iota_v(\iota_v T))$ where $T$ satisfies that $\pi(T) = \omega$. Hence, we have that $\iota_v(\iota_v \omega) = 0$, and from there we can show that $\iota_{v_1}(\iota_{v_2}\omega) = -\iota_{v_2}(\iota_{v_1}\omega)$ just like before.\newpage

\dispDate{8/13/2025}

I'm going to take a break from Guillemin's book and instead try to learn some\\ algebraic topology. To do this I'm going to start following Munkres' \ul{Topology}.\retTwo

If $f_1, f_2: X \to Y$ are continuous maps, we say $f_1$ is \udefine{homotopic} to $f_2$ if there is a  continuous map $F: X \times [0, 1] \to Y$ such that $F(x, 0) = f_1(x)$ and $F(x, 1) = f_2(x)$. $F$ is called a \udefine{homotopy} between $f_1$ and $f_2$. And if $f_1$ and $f_2$ are homotopic, we write $f_1 \simeq f_2$. If $f_1 \simeq f_2$ and $f_2$ is a constant map, then we say $f_1$ is \udefine{nulhomotopic}.\\

\begin{myIndent}\hTwo
	An important special case is when $f_1$ and $f_2$ are paths (i.e. continuous maps from $[0, 1]$ to a topological space $X$). In this case, it can be helpful to make the following stricter distinction. We say $f_1$ and $f_2$ are \udefine{path homotopic} if they have the same initial point $x_0$ and final point $x_1$, and there is a homotopy $F$ between the two paths such that $F(0, t) = x_0$ and $F(1, t) = x_1$ for all $t$. Also, we call $F$ a \udefine{path homotopy} and say $f_1 \simeq_p f_2$. \retTwo
\end{myIndent}

\exOne\ul{Lemma 51.1:} $\simeq$ and $\simeq_p$ are equivalence relations.

\begin{myIndent}\exTwoP
	Proof:\\
	It's clear that any $f$ is homotopic to itself. Also, if $F(x, t)$ is a homotopy showing that $f_1 \sim f_2$, then $G(x, t) = F(x, 1-t)$ is a homotopy showing that $f_2 \sim f_1$.\retTwo
	
	Finally, suppose $f_1 \simeq f_2$ and $f_2 \simeq f_3$. Then there exists two homotopy's $F^{(1)}$\\ between $f_1$ and $f_2$ and $F^{(2)}$ between $f_2$ and $f_3$. So, define:

	{\centering $G(x, t) = \left\{\begin{matrix}
		F^{(1)}(x, 2t) & \text{for } t \in [0, \sfrac{1}{2}] \\
		F^{(2)}(x, 2t - 1) & \text{for } t \in [\sfrac{1}{2}, 1]
	\end{matrix}\right.$ \retTwo\par}

	Then $G$ is a homotopy between $f_1$ and $f_3$, meaning $f_1 \simeq f_3$.
	\begin{myIndent}\exPPP
		We know $G$ is continuous by the pasting lemma.\retTwo
	\end{myIndent}

	The added stuff needed to $\simeq_p$ is an equivalence relation is obvious.\retTwo
\end{myIndent}

\hOne If $f$ is a path in $X$ from $x_0$ to $x_1$ and $g$ is a path in $X$ from $x_1$ to $x_2$, we define the product $f * g$ to be the path $h$ given by the equation:

{\centering $h(s) = \left\{\begin{matrix}
		f(2s) & \text{for } s \in [0, \sfrac{1}{2}] \\
		g(2s - 1) & \text{for } s \in [\sfrac{1}{2}, 1]
\end{matrix}\right.$ \retTwo\par}

\begin{myIndent}\hTwo
	By the pasting lemma, $h$ is a well-defined path in $X$ from $x_0$ to $x_2$.\retTwo
\end{myIndent}

If $f: [0, 1] \to X$ is a path, let $[f]$ denote the path homotopy class of $f$. Then\\ the product operation induces a well-defined operation on path-homotopy classes. Specifically, given a class $[f]$ from $x_0$ to $x_1$ and a class $[g]$ from $x_1$ to $x_2$, define\\ $[f]*[g] = [f*g]$.\newpage

\begin{myIndent}\exTwoP
	To verify that this is well defined, suppose $f \simeq_p f^\prime$ and $g \simeq_p g^\prime$. Then if $F$ is a homotopy from $f$ to $f^\prime$ and $G$ is a homotopy from $g$ to $g^\prime$, we can define a homotopy $H$ from $f * g$ to $f^\prime * g^\prime$ by the formula:

	{\centering $H(s, t) = \left\{\begin{matrix}
		F(2s, t) & \text{for } s \in [0, \sfrac{1}{2}] \\
		G(2s - 1, t) & \text{for } s \in [\sfrac{1}{2}, 1]
	\end{matrix}\right.$ \retTwo\par}

	$H$ is well-defined and continuous by pasting lemma.\retTwo
\end{myIndent}

Recall that a groupoid is a category in which every morphism is an isomorphism (look at my old Allufi notes to see what a category is\dots). Using the product operation of path-homotopy classes, we can define a groupoid as follows:

\begin{myIndent}\hTwo
	Consider the space $X$ as a collection of objects, and for any $x_0, x_1 \in X$, let\\ $\Hom_X(x_0, x_1)$ be the collection of path homotopy classes from $x_0$ to $x_1$. For the law of composition, say that if $[f] \in \Hom_X(x_0, x_1)$ and $[g] \in \Hom_X(x_1, x_2)$, then $[g][f] = [f * g] \in \Hom_X(x_0, x_2)$.\retTwo

	We claim:
	\begin{itemize}
	\item Every point has an identity morphism (namely the homotopy class of the\\ constant map).
	\item For any $[f] \in \Hom(x_0, x_1)$, you can reverse the path $f$ (i.e. define\\ $\bar{f}(s) \coloneqq f(1-s)$) in order to get an inverse morphism in $\Hom(x_1, x_0)$.
	\item Finally, if $[f] \in \Hom(x_0, x_1)$, $[g] \in \Hom(x_1, x_2)$, and $[h] \in \Hom(x_2, x_3)$, then:
	
	{\centering $[f]*([g] * [h]) = ([f]*[g])*[h]$.\retTwo\par}
	\end{itemize}

	\begin{myIndent}\exTwoP
		Proof:\\
		We start with two lemmas:
		\begin{enumerate}
			\item[1.] If $k: X \to Y$ is a continuous map and $F$ is a path homotopy in $X$ between the paths $f$ and $f^\prime$, then $k \circ F$ is a path homotopy in $Y$ between the paths $k \circ f$ and $k \circ f^\prime$.
			\item[2.] If $k: X \to Y$ is a continuous map and $f$ and $g$ are paths in $X$ with\\ $f(1) = g(0)$, then $k \circ (f * g) = (k \circ f) * (k \circ g)$.\retTwo
		\end{enumerate}

		To prove the first bullet point, let $e_0: [0, 1] \to [0, 1]$ be the constant function equal to $0$ and $i: [0,1] \to [0, 1]$ be the identity map. Then when considering both of those as paths in $[0, 1]$, we can fairly easily find a path homotopy $G$ from $e_0 * i$ to $i$.

		\begin{myIndent}\exPPP
			One path homotopy that works is to define
			
			{\centering$F(s, t) = t(e_0 * i)(s) + (1-t)i(s)$.\retTwo\par}
		\end{myIndent}

		Now suppose $e_{x_0}: [0, 1] \to X$ is constant at $x_0$ and $f: [0, 1] \to X$ is a path from $x_0$ to $x_1$. Then $e_{x_0} = f \circ e_0$, $f = f \circ i$, and by our two lemmas, $f \circ G$ is a path homotopy from $f = f \circ i$ to $f \circ (e_0 * i) = (f \circ e_0) * (f \circ i) = e_{x_0} * f$.\\ Similar reasoning shows that if $e_{x_1}: [0, 1] \to X$ is constant at $x_1$, then\\ $f \simeq_p f * e_{x_1}$. This proves bullet point 1.\newpage

		To prove the second bullet point, let $\bar{i} = i(1-s)$. Then we can find a\\ homotopy $G$ from $e_0$ to $i * \bar{i}$ {\exPPP(one that works is $G(s, t) = t((i * \bar{i})(s))$)}.\retTwo
		
		Then for any path $f$ from $x_0$ to $x_1$, we can easily see that $e_{x_0} = f \circ e_0$,\\ $f = f \circ i$, and $\bar{f} = f \circ \bar{i}$. Hence by our two lemmas, $f \circ G$ is a homotopy between $e_{x_0} = f \circ (e_0)$ and $f * \bar{f} = (f \circ i) * (f \circ \bar{i}) = f \circ (i * \bar{i})$. Also, once again similar reasoning shows that $e_{x_1} \simeq_p \bar{f} * f$. This proves bullet point 2.\retTwo

		I'm bored. So tldr: to prove the third bullet point just note that we can apply a continuous reparametrization $k(s)$ to $((f * g) * h)(s)$ to get $(f * (g * h))(s)$. Hence, we can define a homotopy:
		
		{\centering $G(s, t) \coloneqq ((f * g) * h)((1-t)s + tk(s))$. $\blacksquare$\retTwo\par}
	\end{myIndent}
\end{myIndent}

Now given a point $x_0 \in X$, define $\pi_1(X, x_0) \coloneq \End(x_0)$. This is the \udefine{fundamental\\ group} of $X$ relative to $x_0$, and it is in fact a group with respect to our product\\ operation since $X$ was a groupoid. I'm going to state the next proposition as\\ abstractly as I can cause why the hell not.\retTwo

\pracOne\ul{Proposition:} Let $\mcateg{C}$ be a groupoid and let $A, B \in \Obj(\mcateg{C})$. If there exists $g \in \Hom(A, B)$, then $\End(A) \cong \End(B)$.

\begin{myIndent}\pracTwo 
	Proof:\\
	If $f \in \End(A)$, then define $\phi(f) = gfg^{-1}$. Then it's clear that $\phi$ is a group\\ homomorphism from $\End(A)$ to $\End(B)$. To show that $\phi$ is injective, suppose $\phi(f) = e_B$ where $e_B$ is the identity morphism on $B$. Then $f = g^{-1}e_B g = g^{-1}g = e_A$ where $e_A$ is the identity morphism on $A$. Next, to show that $\phi$ is surjective, suppose $h \in \End(B)$. Then $f \coloneqq g^{-1}hg$ satisfies that $\phi(f) = h$.\retTwo
\end{myIndent}

\ul{Corollary:} If $X$ is path connected, then $\pi_1(X, x_0) \cong \pi_1(X, x_1)$ for all $x_0, x_1 \in X$.\retTwo

\hOne We say a space $X$ is \udefine{simply connected} if $X$ is path connected and for some $x_0 \in X$, $\pi_1(X, x_0) = \{1\}$.\retTwo

\exOne\ul{Lemma 52.3:} Let $X$ be a path-connected topological space. Then $X$ is simply\\ connected iff every pair of paths $f$ and $f^\prime$ with the same initial and final point are path homotopic.

\begin{myIndent}\exTwoP
	$(\Longrightarrow)$\\
	Suppose $f$ and $f^\prime$ are paths from $x_0$ to $x_1$. Then if we set $\bar{f}$ and $\bar{f^\prime}$ to be the reversed paths, we know that $[f^\prime * \bar{f}], [f * \bar{f^\prime}] \in \pi_1(X, x_0)$. But now since $\pi_1(X, x_0)$ is trivial, we know that $[f^\prime * \bar{f}] = 1 = [f * \bar{f^\prime}]$. So, there is a path homotopy $F$ between $f^\prime * \bar{f}$ and $f * \bar{f}$. Now just define $G(s, t) = F(\frac{1}{2}s, t)$ and we have shown that $f$ and $f^\prime$ are path homotopic.\retTwo

	$(\Longleftarrow)$\\
	Suppose $f \in \pi_1(X, x_0)$ for some $x_0 \in X$. Also suppose $g$ is a path in $X$ from $x_0$ to $x_1$ where $x_1 \neq x_0$. (Note, this lemma is trivial if $X$ has only one point. So, we can without loss of generality assume $X$ has more than one point.)\newpage

	Then since both $g$ and $f * g$ are paths from $x_0$ to $x_1$, we know there is a homotopy $F$ between the two paths. So, $[g] = [f * g] = [g] * [f]$. If we apply on the left side the class $[\bar{g}]$, then this means that $1 = [f]$. Hence, $\pi_1(X, x_0)$ is trivial. $\blacksquare$\retTwo

	\begin{myIndent}\exPPP
		A consequence of this lemma is that all convex subsets of $\mathbb{R}^n$ are simply connected.
	\end{myIndent}
\end{myIndent}

\hOne \mySepTwo

Recall the two lemmas stated on page 118. Those lemmas will let us define an\\ important thing.\retTwo

Suppose $h: X \to Y$ is a continuous map such that $h(x_0) = y_0$. We will denote this by writing $h: (X, x_0) \to (Y, y_0)$. Then we define the \udefine{homomorphism induced by $h$}, $h_*: \pi_1(X, x_0) \to \pi_2(X, y_0)$, by the map:

{\centering $h_*([f]) \coloneqq [h \circ f]$ \retTwo\par} 

This is well defined since if $F$ is a homotopy between $f$ and $f^\prime$, then $h \circ F$ is a homotopy between $h \circ f$ and $h \circ f^\prime$. Also, this is indeed a group homomorphism since:

{\centering $h_*([f] * [g]) = [h \circ (f * g)] = [h \circ f] * [h \circ g] = h_*([f]) * h_*([g])$ \retTwo\par}

\exOne\ul{Theorem 52.4:} If $h: (X, x_0) \to (Y, y_0)$ and $k: (Y, y_0) \to (Z, z_0)$ are continuous\\ maps, then $(k \circ h)_* = k_* \circ h_*$. Also if $i: (X, x_0) \to (X, x_0)$ is the identity map on $X$, then $i_*$ is the identity map on $\pi_1(X, x_0)$.

\begin{myIndent}\exTwoP
	Hopefully this is self-explanatory.\retTwo
\end{myIndent}

\ul{Corollary 52.5:} If $h: (X, x_0) \to (Y, y_0)$ is a homeomorphism from $X$ to $Y$, then $h_*$ is an isomorphism from $\pi_1(X, x_0)$ to $\pi_1(Y, y_0)$.

\begin{myIndent}\exTwoP
	Proof:\\
	Let $k = h^{-1}$. Then from the last theorem we can easily see that $k_*$ and $h_*$ are\\ inverses of each other. Hence, $h_*$ is invertible and thus a group  isomorphism. $\blacksquare$\retTwo

	\begin{myIndent}\exPPP
		Consequently, we know that the fundamental group of a path connected space is\\ a topological property. So, if two path connected spaces do not have the same\\ fundamental group, they can't be homeomorphic.\retTwo
	\end{myIndent}
\end{myIndent}

\dispDate{8/15/2025}

\pracOne If $\alpha$ is a path from $x_0$ to $x_1$ in $X$, we shall denote $\bar{\alpha}$ to be the reversed path from $x_1$ to $x_0$. Also we shall denote $\hat{\alpha}: \pi_1(X, x_0) \to \pi_1(X, x_1)$ to be the isomorphism $[f] \mapsto [\bar{\alpha} * f * \alpha]$.\retTwo

\ul{Exercise 52.3:} Let $x_0 \to x_1$ be points of the path-connected space $X$. Show that $\pi_1(X, x_0)$ is abelian iff for every pair $\alpha$ and $\beta$ of paths from $x_0$ to $x_1$, we have $\hat{\alpha} = \hat{\beta}$. 

\begin{myIndent}\pracTwo
	$(\Longrightarrow)$\\
	Suppose $\pi_1(X, x_0)$ is abelian and we have two paths $\alpha, \beta$ from $x_0$ to $x_1$. Then for any $[f] \in \pi_1(X, x_0)$, we have that $[f] * [\alpha * \bar{\beta}] = [\alpha * \bar{b}] * [f]$. Therefore:
	
	{\centering$\hat{\alpha}(f) = [\bar{\alpha} * f * \alpha] = [\bar{\beta} * f * \beta] = \hat{\beta}(f)$.\newpage\par}

	$(\Longleftarrow)$\\
	Suppose $[f] \in \pi_1(X, x_0)$ and $\alpha, \beta$ are paths from $x_0$ to $x_1$. Then since\\ $\hat{\alpha}([f * \alpha * \bar{\beta}]) = \hat{\beta}([f * \alpha * \bar{\beta}])$, we have that:

	{\centering$[\bar{\alpha} * f * \alpha * \bar{\beta} * \alpha] = [\bar{\beta} * f * \alpha * \bar{\beta} * \beta] = [\bar{\beta} * f * \alpha]$\retTwo\par}

	Hence $[f] * [\alpha * \bar{\beta}] = [\alpha * \bar{\beta}] * [f]$, and this proves that the group operation of $\pi_1(X, x_0)$ is commutative so long as one of the arguments passes through $x_1$.\retTwo

	Now to prove general commutativity, suppose $[f], [g] \in \pi_1(X, x_0)$ and $\alpha$ is a path from $x_0$ to $x_1$. Then from before we know that:

	{\centering $[f] * [g] = [f] * [g * \alpha * \bar{\alpha}] = [g * \alpha * \bar{\alpha}] * [f] = [g] * [f]$. \retTwo\par}
\end{myIndent}

\ul{Exercise 52.4:} Let $A \subseteq X$, suppose $r: X \to A$ is a continuous map such that $r(a) = a$ for each $a \in A$. (The map $r$ is called a \udefine{retraction} of $X$ onto $A$ and we call $A$ a \udefine{retract} of $X$.) If $a_0 \in A$, then show that $r_*: \pi_1(X, a_0) \to \pi_1(A, a_0)$ is surjective.

\begin{myIndent}\pracTwo
	Suppose $[f]_A \in \pi_1(A, a_0)$. Then $f$ is also a loop in $X$, meaning it has a class\\ $[f]_X \in \pi_1(X, a_0)$. And $r_*([f]_X) = [r \circ f]_A = [f]_A$. Hence $r_*$ is surjective.\retTwo
\end{myIndent}

\ul{Exercise 52.5:} Let $A$ be a subspace of a simply connected space $X$, and let\\ $h: (A, a_0) \to (Y, y_0)$ be a continuous map. If $h$ is extendable to a continuous map of $X$ into $Y$, then $h_*$ is the trivial homomorphism (i.e. the homomorphism mapping everything to the identity element).

\begin{myIndent}\pracTwo
	Let $h^\prime$ be a continuous extension of $h$ to all of $X$. Now importantly, for any loop\\ $[f]_A \in \pi_1(A, a_0)$, we know $f$ also defines a class $[f]_X \in \pi_1(X, a_0)$. Also importantly:
	
	{\centering$h^\prime_*([f]_X) = [h^\prime \circ f]_Y = [h \circ f]_Y = h_*([f]_A)$.\retTwo\par}

	However, since $X$ is simply connected, we know $[f]_X = 1$ and thus $h^\prime_*([f]_X) = 1$. So, we've shown that $h_*([f]_A) = 1$ for all $[f]_A \in \pi_1(A, a_0)$.

	\begin{myTindent}\myComment
		Note: Munkres specifically sets $X = \mathbb{R}^n$ in his statement of the exercise.\retTwo
	\end{myTindent}
\end{myIndent}

\ul{Exercise 52.6:} Suppose $h: X \to Y$ is a continuous map, $\alpha$ is a path from $x_0$ to $x_1$ in $X$, and $\beta \coloneqq h \circ \alpha$. Then $h_* \circ \hat{\alpha} = \hat{\beta} \circ h_*$.

\begin{myIndent}\pracTwo
	Suppose $[f] \in \pi_1(X, x_0)$. Then:

	{\centering\begin{tabular}{l}
	$h_*(\hat{\alpha}([f])) = [h \circ (\bar{\alpha} * f * \alpha)] = [(h \circ \bar{\alpha}) * (h \circ f) * (h \circ \alpha)]$\\
	$\phantom{h_*(\hat{\alpha}([f])) = [h \circ (\bar{\alpha} * f * \alpha)]} = [\bar{\beta} * h_*([f]) * \beta] = \hat{\beta}(h_*([f]))$.
	\end{tabular}\retTwo\par}
\end{myIndent}

\hOne\mySepTwo

Let $p: E \to B$ be a continuous surjective map. Then an open set $U \subseteq B$ is said to be \udefine{evenly covered} by $p$ if $p^{-1}(U)$ is a union of disjoint open sets $V_\alpha \subseteq E$ satisfying that $p|_{V_\alpha}$ is a homeomorphism from $V_\alpha$ to $U$. The collection $\{V_\alpha\}_{\alpha \in A}$ will be called a partition of $p^{-1}(U)$ into \udefine{slices}.

\begin{myIndent}\hTwo
	Note that if $W \subseteq U$ is also open, then $W$ is also openly covered by $p$ with there being a partition of slices $\{V_\alpha \cap p^{-1}(W)\}_{\alpha \in A}$.
	\retTwo
\end{myIndent}

If every point $b \in B$ has an open neighborhood $U \subseteq B$ that is evenly covered by $p$, we call $p$ a \udefine{covering map} and $E$ a \udefine{covering space} of $B$.\newpage

\exOne\ul{Claim:} If $p: E \to B$ is a covering map of $B$, then $p$ is an open map.

\begin{myIndent}\exTwoP
	Proof:\\
	Suppose $A \subseteq E$ is open. Then for any $y \in f(A)$, there exists an open neighborhood $U$ of $y$ that is evenly covered by $p$. In turn there exists $x \in A$ with $p(x) = y$ and an open neighborhood $V_\alpha$ of $x$ such that $p|_{V_\alpha}$ is a homeomorphism from $V_\alpha$ to $U_y$. And hence, $p(A \cap V_\alpha)$ is an open neighborhood of $y$ in $U \subseteq B$ that is also a subset of $p(A)$. It follows that $p(A)$ is an open subset of $B$. $\blacksquare$\retTwo
\end{myIndent}

\exOne\ul{Corollary:} If $p: E \to B$ is a covering map of $B$, then $p$ is a \udefine{local homeomorphism}, meaning each point $e \in E$ has an open neighborhood that is mapped\\ homeomorphically by $p$ onto an open subset of $B$.

\begin{myIndent}\exTwoP
	Proof:\\
	If $x \in E$, then the reasoning from the prior proof lets us pick an open set $A \cap V_\alpha$ containing $x$ such that, $p|_{A \cap V_\alpha}$ is a homeomorphism to an open set $p(A \cap V_\alpha)$ in $B$.\retTwo
\end{myIndent}

\exOne\ul{Theorem 53.1:} The map $p: \mathbb{R} \to S^1$ given by $p(x) = (\cos(2\pi x), \sin(2\pi x))$ is a covering map of $S^1$.

\begin{myIndent}\exTwoP
	Hopefully this is obvious. We can cover $S^1$ with four open sets gotten by intersecting $S^1$ with the top, bottom, right, and left open halfs respectively of the coordinate plane. Then it's easy to find a partition of each preimage into slices.\retTwo

	\begin{myIndent}\exPPP
		As a side note, if $H^1 = \{x \in \mathbb{R} : x \geq 0\}$, then $p|_{H^1}$ is surjective and a local homeomorphism. That said, it is not a covering map since the point $(1, 0) \in S^1$ has no open neighborhood $U$ that is evenly covered by $p|_{H^1}$.

		\begin{myIndent}
			(The specific issue we run into is that for any open set $U$ we pick, $p|_{H^1}$ will not be a surjective map from the slice of the preimage containing $0$ to $U$\dots)\retTwo
		\end{myIndent}
	\end{myIndent}
\end{myIndent}

\ul{Theorem 53.2:} Let $p: E \to B$ be a covering map. If $B_0$ is a subspace of $B$ and $E_0 \coloneqq p^{-1}(B_0)$, then the map $p_0 : E_0 \to B_0$ obtained by restricting $p$ is a covering map.
\begin{myIndent}\exTwoP
	Proof:\\
	Given $y \in B_0$, let $U \subseteq B$ be an open neighborhood of $y$ that is evenly covered\\ and let $\{V_\alpha\}_{\alpha \in A}$ be a partition of $p^{-1}(U)$ into slices. Then $U \cap B_0$ is an open\\ neighborhood of $y$ in $B_0$ that is evenly covered by $p_0$ via the partition\\ $\{V_\alpha \cap E_0\}_{\alpha \in A}$ of $p^{-1}(U \cap B_0)$ into slices. $\blacksquare$\retTwo
\end{myIndent}

\ul{Theorem 53.3:} If $p: E \to B$ are covering maps and $p^\prime : E^\prime \to B^\prime$ are covering maps, then the map $p \times p^\prime: E \times E^\prime \to B \times B^\prime$ defined by $(e, e^\prime) \mapsto (p(e), p^\prime(e^\prime))$ is a covering map.
\begin{myIndent}\exTwoP
	Proof:\\
	Given $b \in B$ and $b^\prime \in B^\prime$, let $U$ and $U^\prime$ be open neighborhoods of $b$ and $b^\prime$ that are evenly covered by $p$ and $p^\prime$ respectively. Next let $\{V_\alpha\}_{\alpha \in A}$ and $\{V^\prime_\gamma\}_{\gamma \in C}$ be partitions of $p^{-1}(U)$ and $(p^\prime)^{-1}(U^\prime)$ respectively into slices. Then:

	{\centering $(p \times p^\prime)^{-1}(U \times U^\prime) = \bigcup\limits_{\alpha \in A}(\bigcup\limits_{\gamma \in C} (V_\alpha \times V^\prime_\gamma))$.\newpage\par}

	Also $U \times U^\prime$ is open in $B \times B^\prime$; $V_\alpha \times V^\prime_\gamma$ is open in $E \times E^\prime$ for all $\alpha$ and $\gamma$; the $V_\alpha \times V^\prime_\gamma$ are all disjoint; and each $V_\alpha \times V^\prime_\gamma$ is mapped homeomorphically onto $U \times U^\prime$. $\blacksquare$\retTwo
\end{myIndent}

\pracOne\ul{Exercise 53.2:} Let $p: E \to B$ be continuous and surjective, and suppose $U \subseteq B$ is an\\ open set that is evenly covered by $p$. If $U$ is connected, then the partition of $p^{-1}(U)$ into\\ slices is unique.
\begin{myIndent}\pracTwo
	Proof:\\
	For the sake of contradiction, suppose $\{V_\alpha\}_{\alpha \in A}$ and $\{V^\prime_\beta\}_{\beta \in B}$ are two different partitions of $p^{-1}(U)$ into slices. Then we know there exists $V_{\alpha_0}, V^\prime_{\beta_0}$ in those two partitions satisfying that $V_{\alpha_0} \cap V^\prime_{\beta_0} \neq \emptyset$ and $V_{\alpha_0} \neq V^\prime_{\beta_0}$. Without loss of generality suppose $V_{\alpha_0} - V^\prime_{\beta_0} \neq \emptyset$. Then since $p$ is a homeomorphism from $V_{\alpha_0}$ to $U$, we know $p(V_{\alpha_0} \cap V^\prime_{\beta_0})$ is open in $U$. Also, since $V^\prime_{\beta_0} = p^{-1}(U) - (\bigcup_{\beta \neq \beta_0} V^\prime_\beta)$, we know $V^\prime_{\beta_0}$ is closed in $p^{-1}(U)$. Hence $V_{\alpha_0} - V^\prime_{\beta_0}$ is open in $V_{\alpha_0}$ and so $p(V_{\alpha_0} - V^\prime_{\beta_0})$ is also open in $U$.\retTwo
	
	But now $p(V_{\alpha_0} - V^\prime_{\beta_0})$ and $p(V_{\alpha_0} \cap V^\prime_{\beta_0})$ are two disjoint nonempty open subsets of $U$. This contradicts that $U$ is connected. $\blacksquare$\retTwo
\end{myIndent}

\ul{Exercise 53.3:} Let $p: E \to B$ be a covering map and let $B$ be connected. If $p^{-1}(\{b_0\})$ has $k$ elements for some $b_0 \in B$, then $p^{-1}(\{b\})$ has $k$ elements for all $b \in B$. In such a case $E$ is called a \udefine{$k$-fold covering} of $B$.
\begin{myIndent}\pracTwo
	Proof:\\
	Let $B_k \coloneqq \{b \in B : |p^{-1}(\{b\})| = k\}$. Now it's very clear that if $b \in B$ and $U \subseteq B$ is an open neighborhood of $b$ that is evenly covered by $p$, then $U \subseteq B_k$. Hence, $B_k$ is open. Also note that if $b \in (B_k)^\comp$ and $U \subseteq B$ is an open neighborhood of $b$ that is evenly covered by $p$, then $U \subseteq (B_k)^\comp$. Hence $(B_k)^\comp$ is open. Since $B$ is connected, this means that $B_k$ or $(B_{k})^\comp$ must be empty. and since $b_0 \in B_k$, we know the empty set isn't $B_k$. $\blacksquare$\retTwo
\end{myIndent}

\hOne Let $p: E \to B$ be a map. If $f: X \to B$ is a continuous map, a \udefine{lifting} of $f$ is a\\ map $\tilde{f}: X \to E$ such that $p \circ \tilde{f} = f$. Or in other words, the following diagram\\ commutes:

% https://q.uiver.app/#q=WzAsMyxbMCwyLCJYIl0sWzIsMCwiRSJdLFsyLDIsIkIiXSxbMCwxLCJcXHRpbGRle2Z9Il0sWzEsMiwicCJdLFswLDIsImYiLDJdXQ==
\[\begin{tikzcd}[column sep=large,row sep=scriptsize]
	&& E \\
	\\
	X && B
	\arrow["p", from=1-3, to=3-3]
	\arrow["{\tilde{f}}", from=3-1, to=1-3]
	\arrow["f"', from=3-1, to=3-3]
\end{tikzcd}\]

\exOne\ul{Lemma 54.1:} Let $p: E \to B$ be a covering map and let $e_0 \in p^{-1}(\{b_0\})$. Then if\\ $f: [0, 1] \to B$ is a path beginning at $b_0$, there is a unique lifting of $f$ to a path $\tilde{f} \in E$\\ beginning at $e_0$.

\begin{myIndent}\exTwoP
	Proof:\\
	Let $\{U_\alpha\}_{\alpha \in A}$ be an open covering of $B$ consisting of open sets that are evenly\\ covered by $p$. Then in turn $\{f^{-1}(U_\alpha)\}_{\alpha \in A}$ is an open covering of $[0, 1]$. So, we can invoke the following useful lemma.\newpage

	\ul{Lebesgue Number Lemma:} If $(X, d)$ is a compact metric space and an open cover $\mathcal{U}$ of $X$ is given, then the cover admits some \udefine{Lebesgue number} $\delta > 0$. That is, for some $\delta > 0$ we have that: $\diam(E) < \delta \Longrightarrow \exists U \in \mathcal{U} \suchthat E \subseteq U$.
	\begin{myIndent}\exPPP
		Proof:\\
		Let $\{U_1, \ldots, U_n\} \subseteq \mathcal{U}$ be a finite subcover of $X$. If any $U_i = X$, then trivially any $\delta > 0$ works. So assume $U_i \neq X$ for all $i$. Then since $F_i \coloneqq X - U_i$ is nonempty for all $i$, we know that $d(x, F_i) = \inf\{d(x, y) : y \in F_i\}$ is well defined for all $x \in X$ and $i \in \{1, \ldots, n\}$. Furthermore, it is easy to see that $d(x, F_i)$ is a continuous function of $x$. And since $F_i$ is closed, we have that $d(x, F_i) = 0$ iff $x \in F_i$.\retTwo

		Now define $f: X \to \mathbb{R}$ by $f(x) = \sum\limits_{i = 1}^n d(x, F_i)$.\retTwo

		Since $f$ is a continuous map from a compact set, we know by the extreme value\\ theorem that $f$ attains a minimum $\alpha \in \mathbb{R}$. Also, since the $F_i$ cover $X$, we know that $f(x) \neq 0$ for any $x \in X$. Hence, $\alpha > 0$.\retTwo

		Now set $\delta = \sfrac{\alpha}{n}$. Then for any $E \subseteq X$ with $\diam(E) < \delta$, if we pick $x_0 \in E$ we have that $E \subseteq B_{\delta}(x_0)$. Importantly, since $f(x_0) > \alpha = n\delta$, we must have that $d(x, F_i) > \delta$ for some $i$. It follows then that $E \subseteq U_i$. This proves that $\delta$ works as a Lebesgue number. $\blacksquare$\retTwo
	\end{myIndent}

	By the above lemma, we know that there exists $0 = s_0 < s_1 < \ldots < s_n = 1$\\ satisfying for each $0 \leq i < n$ that $[e_i, e_{i+1}] \subseteq f^{-1}(U_{\alpha_i})$ for some $\alpha_i \in A$. Or in other words, $f([e_i, e_{i+1}]) \subseteq U_{\alpha_i}$ for some $\alpha_i \in A$. After setting $\tilde{f}(0) = e_0$, we may proceed by induction as follows in order to show a suitable $\tilde{f}$ exists.

	\begin{myIndent}
		Assume that $\tilde{f}$ was already defined for all $s \in [0, s_i]$ where $i < n$. Since\\ $U_{\alpha_i}$ is evenly covered by $p$, we know there is a unique open set $V \subseteq E$\\ such that $\tilde{f}(s_i) \in V$ and $p|_V$ is a homeomorphism onto $U_{\alpha_i}$. Then since\\ $f([s_i, s_{i+1}]) \subseteq U_{\alpha_i}$, we know it is well defined to set $\tilde{f}(s) = (p|_V)^{-1}(f(s))$\\ for all $s \in [s_i, s_{i+1}]$. Also, it is clear then that $\tilde{f}|_{[s_i, s_{i+1}]}$ is continuous since it\\ is the composition of two continuous functions. In turn, we can see by the\\ pasting lemma that $\tilde{f}$ will be continuous on the domain $[0, s_{i+1}]$ \retTwo
	\end{myIndent}
	
	To finish off, we now need to show that $\tilde{f}$ is unique. So suppose $g: [0, 1] \to E$ also satisfies that $p \circ g = E$ and $g(0) = e_0$. Then we trivially have that $g(s_0) = \tilde{f}(s_0)$. So, we may proceed by induction as follows in order to show $\tilde{f} = g$.
	\begin{myIndent}
		Assume we've already shown that $\tilde{f}(s) = g(s)$ for all $s \in [0, s_i]$ where\\ $0 \leq i < n$. Now letting $V$ be as in the prior reasoning, we know that $g(s)$\\ must be in $V$ for all $s \in [s_i, s_{i+1}]$. After all, if $\{V_\gamma\}_{\gamma \in C}$ is a partition of $p^{-1}(U_{\alpha_i})$\\ into slices, we must have that:
		
		{\centering$g([s_i, s_{i+1}]) \subseteq p^{-1}(U_{\alpha_i}) = \bigcup_{\gamma \in C} V_\gamma$.\retTwo\par}

		But since all the $V_\gamma$ are disjoint, nonempty, and open; and $g([s_i, s_{i+1}])$ is\\ connected, it must be the case that $g([s_i, s_{i+1}])$ only intercepts one $V_\gamma$.\\ Specifically, that one $V_\gamma$ is $V$ since $g(s_i) \in V$.\newpage

		But now we must have for each $s \in [s_i, s_{i+1}]$ that $g(s)$ satisfies that\\ $p(g(s)) = f(s)$. Yet the only point in $V$ which satisfies that is $\tilde{f}(s)$. Hence, $\tilde{f}(s) = g(s)$ for all $[0, s_{i+1}]$. $\blacksquare$\retTwo
	\end{myIndent}
\end{myIndent}

\ul{Lemma 54.2:} Let $p: E \to B$ be a covering map and let $e_0 \in p^{-1}(\{b_0\})$. Then if\\ $F: [0, 1]^2 \to B$ is a continuous map satisfying that $F(0, 0) = b_0$, there is a unique lifting of $F$ to a continuous function $\tilde{F}: [0, 1]^2 \to E$ such that $\tilde{F}(0, 0) = e_0$.

\begin{myIndent}\exTwoP
	Proof:\\
	We'll start by showing uniqueness. Suppose $\tilde{F}$ and $G$ both are continuous functions from $[0, 1]^2$ to $E$ satisfying our lemma. By an easy application of the last lemma, if $(x_0, y_0) \in [0, 1]^2$, then we must have $\tilde{f}(s) \coloneqq \tilde{F}(sx_0, sy_0)$ and $g(s) \coloneqq G(sx_0, sy_0)$ are equal for all $s \in [0, 1]$ since both are the unique lifting of the path\\ $f(s) \coloneqq F(sx_0, sy_0)$ to a path in $E$ starting at $e_0$. But then we've shown that\\ $\tilde{F}(x_0, y_0) = \tilde{f}(1) = g(1) = G(x_0, y_0)$. And since $(x_0, y_0)$ was arbitrary, we have shown that $\tilde{F}$ is unique if it exists.\retTwo

	Next, we need to show that a sufficient $\tilde{F}$ exists in the first place. So start by setting $\tilde{F}(0, 0) = e_0$. Now for any fixed $(x_0, y_0) \in [0,1]^2$, if we define $f(s) = F(sx_0, sy_0)$, then we know by the prior lemma that there is a continuous map $\tilde{f}: [0, 1] \to E$ such that $p \circ \tilde{f}(s) = f(s)$ for all $s \in [0, 1]$. Hence, we may define $\tilde{F}(x_0, y_0) = \tilde{f}(1)$. After doing this for all choices of $(x_0, y_0)$, we will have constructed a function\\ $\tilde{F}: [0, 1]^2 \to E$ such that $p \circ \tilde{F} = F$.\retTwo

	What's still not clear is that $\tilde{F}$ is continuous. To show this, we will need the\\ observation that if $f(s) = F(sx_0, sy_0)$ and $\tilde{f}$ is the unique lifting of $f$ to path in $E$ such that $p \circ \tilde{f} = f$ and $\tilde{f}(0) = e_0$, then by an easy application of the last lemma we can show that $\tilde{F}(sx_0, sy_0) = \tilde{f}(s)$. Hence, for any $(x_0, y_0) \in [0,1]^2$ we have that $\tilde{F}$ varies continuously as one moves along the straight line between $(0, 0)$ to $(x_0, y_0)$.\retTwo

	Now use the Lebesgue number lemma to pick $0 = s_0 < s_1 < \ldots < s_n = 1$ such that for all $0 \leq i, j < n$, $F([s_i, s_{i+1}] \times [s_j, s_{j+1}])$ is contained in some open set $U$ that is evenly covered by $p$. Then make the inductive hypotheses that there exists $0 \leq i, j < n$ for which we've already shown that $\tilde{F}$ is continuous on:
	
	{\centering$([0, s_i] \times [0, 1]) \cup ([s_i, s_{i+1}] \times [0, s_j])$.\retTwo\par}

	\begin{myTindent}\exPPP
		Note that if $i = 0$ or $j = 0$, then the fact that $\tilde{F}$ is continuous on the line connecting $(0, 0)$ to $(0, 1)$ and on the line connecting $(0, 0)$ to $(1, 0)$ proves this base case.\retTwo
	\end{myTindent}

	Now let $U$ be an open set that contains $F([s_i, s_{i+1}] \times [s_j, s_{j+1}])$ and is evenly covered by $p$. Then let $V \subseteq E$ be an open set disjoint from the rest of the preimage of $U$ such that $p|_V$ is a homeomorphism from $V$ to $U$ and $\tilde{F}(s_i, s_j) \in V$. Since we already know by induction that $F$ varies continuously along the straight lines from $(s_i, s_{j})$ to $(s_i, s_{j+1})$, and from $(s_i, s_{j})$ to $(s_{i+1}, s_j)$, we know that $F(x, s_j)$ and $F(s_i, y)$ are in $V$ for all $x \in [s_i, s_{i+1}]$ and $y \in [s_j, s_{j+1}]$. This is important because we know that for any $(x, y) \in [s_i, s_{i+1}] \times [s_j, s_{j+1}]$, the straight line from $(0, 0)$ to $(x, y)$ must cross one of those two borders of the rectangle. And since $F$ varies continuously along the straight line from $(0, 0)$ to $(x, y)$, this proves that $F(x, y) \in V$ for all $(x, y) \in [s_i, s_{i+1}] \times [s_j, s_{j+1}]$.\retTwo

	In turn, we have for all $(x, y) \in [s_i, s_{i+1}] \times [s_j, s_{j+1}]$ that $\tilde{F} = (p|_V)^{-1}(F(x, y))$. Thus $\tilde{F}$ is continuous on $[s_i, s_{i+1}] \times [s_j, s_{j+1}]$ since it is the composition of two\\ continuous functions. Also by pasting lemma, we can thus say that $\tilde{F}$ is continuous on $([0, s_i] \times [0, 1]) \cup ([s_i, s_{i+1}] \times [0, s_{j+1}])$. $\blacksquare$\retTwo
\end{myIndent}

\ul{Corollary:} If $F$ is a path homotopy then the lifting in the prior lemma: $\tilde{F}$, is a path\\ homotopy.

\begin{myIndent}\exTwoP
	Proof:\\
	If $F(0, t) = b_0$ for all $t \in [0, 1]$, then we know that $\tilde{F}(\{0\} \times [0, 1]) \subseteq p^{-1}(b_0)$. But the latter set will have the discrete topology as a subspace of $E$. Since $\tilde{F}(\{0\} \times [0, 1])$ is connected, this must mean that $\tilde{F}$ is constant on $0 \times [0, 1]$. Similar reasoning also shows that $\tilde{F}(\{1\} \times [0, 1])$ has one element. $\blacksquare$\retTwo

	\begin{myIndent}\exPPP
		Note: $p^{-1}(b_1)$ as a subspace of $E$ has the discrete topology because $p$ being a\\ covering map means that each of the elements of $p^{-1}(b_1)$ are contained in distinct disjoint open sets. Also, a subset of a space with the discrete topology is connected iff it has one element.\retTwo
	\end{myIndent}
\end{myIndent}


\ul{Theorem 54.3:} Let $p: E \to B$ be a covering map, and let $e_0 \in p^{-1}(b_0)$. Next let $f$ and $g$ be paths in $B$ from $b_0$ to $b_1$ and let $\tilde{f}$ and $\tilde{g}$ be their respective liftings to a path in $E$ starting at $e_0$. If $f$ and $g$ are path homotopic, then $\tilde{f}$ and $\tilde{g}$ end at the same point and are path homotopic.

\begin{myIndent}\exTwoP
	Proof:\\
	Let $F: [0, 1]^2 \to B$ be a homotopy between $f$ and $g$. Then $F(0, 0) = b_0$,\\ meaning there is a unique continuous lifting $\tilde{F}: [0, 1]^2 \to E$ of $F$ to $E$\\ satisfying that $\tilde{F}(0, 0) = e_0$. By our prior corollary, we know that $\tilde{F}$ will be a\\ homotopy, meaning that $\tilde{f}(\{0\} \times [0, 1]) = \{e_0\}$ and $\tilde{f}(\{1\} \times [0, 1]) = \{e_1\}$ where $e_1 \in p^{-1}(b_1)$. Also, due to the uniqueness we proved in lemma 54.1, it's easy to see that $\tilde{F}(s, 0) = \tilde{f}(s)$ and $\tilde{F}(s, 1) = \tilde{g}(s)$ for all $s$. $\blacksquare$\retTwo
\end{myIndent}

\hOne Let $p: E \to B$ be a covering map and let $b_0 \in B$. Choose an $e_0 \in E$ such that $p(b_0) = b_0$. Then given an element $[f] \in \pi_1(B, b_0)$, define $\phi([f]) \coloneqq \tilde{f}(1)$ where $\tilde{f}$ is the unique lifting of $f$ to path in $E$ starting at $e_0$.\retTwo

By the last theorem, $\phi: \pi_1(B, b_0) \to p^{-1}(b_0)$ is a well-defined map which we call the \udefine{lifting correspondence} derived from $p$. Note, $\phi$ is dependent on our choice of $e_0$.\newpage

\exOne\ul{Theorem 54.4:} Let $p: E \to B$ be a covering map and for some $b_0 \in B$, choose\\ some $e_0 \in p^{-1}(b_0)$. If $E$ is path connected, then the lifting correspondence\\ $\phi: \pi_1(B, b_0) \to p^{-1}(b_0)$ is surjective. Furthermore, if $E$ is simply connected,\\ then $\phi$ is bijective.

\begin{myIndent}\exTwoP
	Proof:\\
	If $E$ is path connected, then given any $e_1 \in p^{-1}(b_0)$ there exists a path $\tilde{f}$ from $e_0$ to $e_1$. In turn, $f \coloneqq p \circ \tilde{f}$ is a continuous loop from $b_0$ to itself in $B$ such that $\phi([f]) = \tilde{f}(1) = e_1$.\retTwo

	Next suppose $E$ is simply connected. Then suppose $[f], [g] \in \pi_1(B, b_0)$ satisfy that $\phi([f]) = \phi([g])$. By letting $\tilde{f}$ and $\tilde{g}$ be the liftings of $f$ and $g$ respectively, we know that $\tilde{f}$ and $\tilde{g}$ are both paths from $e_0$ to some $e_1$. Therefore, by lemma 52.3 plus the fact that $E$ is simply connected, we know that $\tilde{f}$ and $\tilde{g}$ are path homotopic via a homotopy $\tilde{F}$. And since $p$ is a continuous map, we have that $F \coloneqq p \circ \tilde{F}$ is a path homotopy of $f = p \circ \tilde{f}$ and $g = p \circ \tilde{g}$. This proves that $[f] = [g]$. $\blacksquare$\retTwo
\end{myIndent}

\ul{Theorem 54.5:} The fundamental group of $S^1 \coloneqq \{x \in \mathbb{R}^2 : \|x\|_2 = 1\}$ is isomorphic to $\mathbb{Z}$.

\begin{myIndent}\exTwoP
	Proof:\\
	Let $p : \mathbb{R} \to S^1$ be the covering map $p(t) = (\cos(2\pi t), \sin(2\pi t))$. Also pick $e_0 = 0$ and let $b_0 = p(e_0) = (1, 0)$. Then $p^1(\{b_0\}) = \mathbb{Z}$. And since $\mathbb{R}$ is simply connected, we have that the lifting correspondence $\phi: \pi_1(S^1, b_0) \to \mathbb{Z}$ is bijective.\retTwo

	To show that $\pi_1(S^1, b_0) \cong \mathbb{Z}$, we shall now show that $\phi$ is a group homomorphism.
	\begin{myIndent}
		Suppose $[f], [g] \in \pi_1(S^1, b_0)$. Then let $\tilde{f}$ and $\tilde{g}$ be the respective liftings of $f$ and $g$ to $\mathbb{R}$ to paths starting at $0$. Also let $n \coloneqq \phi([f])$ and $m \coloneqq \phi([g])$. If we define $\tilde{h}(s) = \tilde{g}(s) + n$, then because $p(x + n) = p(x)$ for all $x \in \mathbb{R}$, we know $\tilde{h}$ is another lifting of $g$. However $\tilde{h}$ starts at $n$ instead of $0$. It follows that $\tilde{f} * \tilde{h}$ is a well-defined product of paths, and it is lifting of $f * g$ to a path in $\mathbb{R}$ starting at $0$. Hence:
		
		{\centering$\phi([f] * [g]) = \phi([f * g]) = (\tilde{f} * \tilde{h})(1) = n+m = \phi([f]) + \phi([g])$. $\blacksquare$\retTwo\par}
		
		\exPPP The intution for this result is that the fundamental group of $S^1$ categorizes all loops in $S^1$ by the net number of complete revolutions of the path around $S^1$.\retTwo
	\end{myIndent}
\end{myIndent}

\ul{Theorem 54.6:} Let $p: E \to B$ be a covering map and let $p(e_0) = b_0$.
\begin{itemize}
	\item[(a)] The induced homomorphism $p_*: \pi_1(E, e_0) \to \pi_1(B, b_0)$ is injective.
	
	\begin{myIndent}\exTwoP
		Proof:\\
		Suppose $[\tilde{h}] \in \pi_1(E, e_0)$ such that $p_*([\tilde{h}])$ is the identity element of $\pi_1(B, b_0)$. Then there is a homotopy $F$ between $p \circ \tilde{h}$ and the constant loop based at $b_0$. If $\tilde{F}$ is the lifting of $F$ to $E$ satisfying that $\tilde{F}(0, 0) = e_0$, then $\tilde{F}$ will be a homotopy between $\tilde{h}$ and the constant loop based at $e_0$.\newpage
	\end{myIndent}

	\item[(b)] Let $H = p_*(\pi_1(E, e_0))$. Then if $\pi_1(B, b_0)/H$ is collection of right cosets of $H$, then the lifting correspondance of $\phi$ induces an injective map:
	
	{\centering $\Phi: \pi_1(B, b_0)/H \to p^{-1}(b_0)$ \retTwo\par}

	Furthermore, $\Phi$ is bijective, if $E$ is path connected.

	\begin{myIndent}\exTwoP
		Proof:\\
		Let $f$ and $g$ be loops in $B$ based at $b_0$, and let $\tilde{f}$ and $\tilde{g}$ be the liftings of those loops in $E$ to paths starting at $e_0$. Since $H$ is a subgroup of $\pi_1(B, b_0)$, we know that $[g] \in H * [f]$ iff $H * [g] = H * [f]$. Thus, to show that $\Phi$ is well-defined, we just need to show that $[g] \in H * [f] \Longrightarrow \phi([g]) = \phi([f])$. Meanwhile, to show that $\Phi$ is injective, we just need to show that $[g] \in H * [f] \Longleftarrow \phi([g]) = \phi([f])$.
		\begin{myIndent}\exThreeP
			$(\Longrightarrow)$\\
			If $[f] \in H * [g]$, then there exists $[h] \in H$ with $[f] = [h * g]$. And due to how we defined $H$, the lifting $\tilde{h}$ of $h$ to $E$ starting $e_0$ will be a loop.
			
			\begin{myIndent}\exPPP
				Why: We know there exists a loop $\tilde{h^\prime}$ in $E$ based at $e_0$ such that\\ $[h] = [p \circ \tilde{h^\prime}]$. But now by theorem 54.3, we have that the liftings $\tilde{h}$ and $\tilde{h^\prime}$ of $h$ and $p \circ \tilde{h^\prime}$ respectively are path homotopic. So, $\tilde{h}$ is also a loop based at $e_0$.\retTwo
			\end{myIndent}

			It follows that the product $\tilde{h} * \tilde{g}$ is well-defined and is a lifting of $h * g$. That plus the fact that $[f] = [h * g]$ means that by theorem 54.3, $\tilde{f}$ and $\tilde{h} * \tilde{g}$ have\\ [2pt] the same end point. So:

			{\centering $\phi([f]) = \tilde{f}(1) = \tilde{h} * \tilde{g}(1) = \tilde{g}(1) = \phi([g])$ \retTwo\par}

			$(\Longleftarrow)$\\
			Next suppose $\phi([f]) = \phi([g])$. Then $\tilde{f}$ and $\tilde{g}$ end at the same point of $E$. So, we can define the loop $\tilde{h}$ in $E$ based at $e_0$ as the product of $\tilde{f}$ and the reverse of $\tilde{g}$. Importantly, $[\tilde{h} * \tilde{g}] = [\tilde{f}]$. So, if $\tilde{F}$ is a path homotopy between $\tilde{h} * \tilde{g}$ and $\tilde{f}$, then $F \coloneqq p \circ \tilde{F}$ is a path homotopy between $f = p \circ \tilde{f}$ and $p \circ (\tilde{h} * \tilde{g}) = (p \circ \tilde{h}) * g$. Also, $p \circ \tilde{h} \in H$. So $[f] \in H * [g]$.\retTwo
		\end{myIndent}

		If $E$ is path connected, then we know $\phi$ is surjective. In turn $\Phi$ will also be\\ surjective.\retTwo
	\end{myIndent}
	
	\item[(c)] If $f$ is a loop in $B$ based at $b_0$, then $[f] \in H$ iff $f$ lifts to a loop in $E$ based at $e_0$.
	
	\begin{myIndent}\exTwoP
		Proof:\\
		Since $\Phi$ is injective and the constant loop about $b_0$ is mapped to $e_0$ by $\phi$, we know that $\phi([f]) = e_0$ iff $[f] \in H$. But $\phi([f]) = e_0$ precisely iff $f$ lifts to a loop in $E$ based at $e_0$. $\blacksquare$\newpage
	\end{myIndent}
\end{itemize}

\hOne\dispDate{8/17/2025}

\exOne\ul{Lemma 55.1:} If $A$ is a retract of $X$, then the homomorphism of fundamental groups induced by the inclusion map $j: A \to X$ is injective.

\begin{myIndent}\exTwoP
	Proof:\\
	If $r: X \to A$ is a retraction, then $r \circ j$ is just the identiy map on $A$. It follows that $(r \circ j)_* = r_* \circ j_*$ is the identiy map on $\pi_1(A, a_0)$ for any $a_0 \in A$. This is only possible if $r_*$ is surjective (which we admittedly already proved on page 121) and $j_*$ is injective. $\blacksquare$\retTwo
\end{myIndent}

\hOne We'll denote $D^2 \coloneqq \{x \in \mathbb{R}^2 : \|x\|_2 \leq 1\}$.\retTwo

\exOne\ul{Theorem 55.2:} There is no retraction from $D^2$ to $S^1$.

\begin{myIndent}\exTwoP
	Proof:\\
	If such a retraction did exist, then we would know by the previous lemma that there exists an injective group homomorphism from the fundamental group of $S^1$ to the fundamental group of $D^2$. However, the fundamental group of $S^1$ has strictly greater cardinality than that of $D^2$. So, no such injection exists. $\blacksquare$\retTwo
\end{myIndent}

\ul{Lemma 55.3:} Suppose $h: S^1 \to X$ is a continuous map. Then the following are equivalent.\\ [-25pt]
\begin{itemize}
	\item[(1)] $h$ is nulhomotopic (meaning there is a homotopy from $h$ to a constant function on $X$).\\ [-20pt]
	\item[(2)] $h$ extends to a continuous map $k: D^2 \to X$.\\ [-20pt]
	\item[(3)] $h_*$ is the trivial homomorphism of fundamental groups.
\end{itemize}

\begin{myIndent}\exTwoP
	$(1 \Longrightarrow 2)$\\
	Let $H : S^1 \times [0, 1] \to X$ be a homotopy from $h$ to some constant map. Then if we define $\pi: S^1 \times [0, 1] \to D^2$ by $\pi(x, t) = x(1 - t)$, we have that $\pi$ defines a quotient map. 
	\begin{myIndent}\exPPP
		It's obvious that $\pi$ is continuous and surjective. Also, it's clear that $\pi$ is an open function since any closed set of $S^1 \times [0, 1]$ is compact and thus maps to another compact set of $D^2$ which must be closed.\retTwo
	\end{myIndent}

	Now since $H$ is constant over the preimage with respect to $\pi$ of any given singleton, we know that $H$ induces a continuous map $k: D^2 \to X$ satisfying that $H = k \circ \pi$. Also, it's clear that for $x \in S^1$, $k(x) = H(x, 0) = h(x)$.\retTwo

	$(2 \Longrightarrow 3)$\\
	Let $k : D^2 \to X$ be an extension of $h$ and let $j: S^1 \to D^2$ be the inclusion map. Then $h = k \circ j$, meaning $h_* = k_* \circ j_*$. But now since the fundamental group of $D^2$ is trivial, we know that $h_*$ must be the trivial homomorphism.\newpage

	$(3 \Longrightarrow 1)$\\
	Let $p: \mathbb{R} \to S^1$ be the standard covering map (the one defined in theorem 53.1). Then $p_0 \coloneqq p|_{[0, 1]}$ is a generator for $\pi_1(S^1, 0)$. If we let $x_0 = h(b_0)$, then $f = h \circ p_0$ is a loop in $X$. And since $h_*$ is trivial, we know there exists a path homotopy $F$ from $h \circ p_0$ to the constant map $x_0$.\retTwo

	Now the map $(p_0 \times i): [0,1]^2 \to S^1 \times [0, 1]$ defined by $(p_0 \times i)(x, t) = (p_0(x), t)$ is a quotient map. 
	\begin{myIndent}\exPPP
		It's continuous and surjective. Also, once again it maps any closed set to another closed set since all closed subsets of $[0, 1]^2$ are compact and $S^1 \times [0, 1]$ is Hausdorff.\retTwo
	\end{myIndent}

	Since $F$ is constant on $(p_0 \times i)^{-1}(y)$ for any $y \in S^1 \times [0, 1]$, we have that $F$ induces a continuous map $H: S^1 \times [0, 1] \to X$ satisfying that $H \circ (p_0 \times i) = F$. Also, $x_0 = F(s, 1) = H(p_0(s), 1)$ for all $s \in [0, 1]$. This shows that $H$ is our desired homotopy. $\blacksquare$\retTwo
\end{myIndent}

\ul{Corollary 55.4:} The inclusion map $j: S^1 \to \mathbb{R} - \{0\}$ is not nulhomotopic. Also the identity map $i: S^1 \to S^1$ is not nulhomotopic.

\begin{myIndent}\exTwoP
	Proof:\\
	There is a retraction $k: \mathbb{R} - \{0\} \to S^1$ given by $k(x) = x / \|x\|_2$. It follows by lemma 55.1 that $j_*$ is injective. And since the fundamental group of $S^1$ is not trivial, that means that $j_*$ is not the trivial homomorphism. Hence, $j$ isn't nulhomotopic by the last lemma.\retTwo

	Similarly, $i_*$ is the identity homomorphism and thus isn't trivial.So $i$ isn't nulhomotopic by the last lemma. $\blacksquare$\retTwo
\end{myIndent}

\hOne\mySepTwo

\dispDate{8/18/2025}

Today I want to start learning about general manifolds. So, I will switch over to following John Lee's book \ul{Introduction to Smooth Manifolds}. Soon I might switch back either to Munkres or Guillemin.\retTwo

Suppose $M$ is a topological space. Then we say $M$ is a \udefine{topological manifold of\\ dimension $n$} (a.k.a a \udefine{topological $n$-manifold}) if:

\begin{itemize}
	\item $M$ is second countable and Hausdorff.
	\item $M$ is \udefine{locally Euclidean of dimension $n$}, meaning that for any $p \in M$, there exists an open neighborhood $U \subseteq M$ of $p$ and a homeorphism $\varphi$ from $U$ to an open set $\hat{U}$ of $H^n$ or $\mathbb{R}^n$.\retTwo
\end{itemize}

If $n = 0$, we say $M$ is a topological $0$-manifold if $M$ is countable and equipped with the discrete topology.\newpage

While it's clear that each pair $(U, \varphi)$ define \udefine{local coordinates} on $M$, one weird thing notation-wise is that those coordinates are commonly written as:

{\centering $\varphi(p) = (x^1(p), \ldots, x^n(p))$ (as opposed to using subscripts).\retTwo\par}

This notation also extends to trivial Euclidean manifolds where (for example) John Lee denotes $x \in \mathbb{R}$ as $x = (x^1, \ldots, x^n)$. I don't know why this is apparently common notation in this field of math. (Also, to be clear my differential geometry professor before I dropped the course also used that notation. So, I'm not making it up that the notation is common.\dots)\retTwo

\begin{myIndent}\pracOne
	One manifold I haven't seen before is the \udefine{$n$-dimensional real projective space},\\ denoted $\mathbb{RP}^n$. It is defined as the set of $1$-dimensional linear subspaces of $\mathbb{R}^{n+1}$ equipped with the quotient topology determined by the map $\pi: \mathbb{R}^{n+1} \to \mathbb{RP}^n$ where $x \mapsto \mSpan\{x\}$.
	\begin{myIndent}\pracTwo
		Note: Given $x \in \mathbb{R}^{n+1}$, denote $[x] \coloneqq \pi(x)$.\retTwo
	\end{myIndent}

	To prove that $\mathbb{RP}^n$ is in fact a manifold, for each $i \in \{1, \ldots, n+1\}$ let\\ $\tilde{U_i} = \{x \in \mathbb{R}^{n+1}: x^i \neq 0\}$. Then let $U = \pi(\tilde{U_i})$. Now $\tilde{U_i}$ is a \udefine{saturated set} with respect to $\pi$ (meaning $\pi^{-1}(\pi(U_i)) = U_i$) for each $i$. Hence $\pi(U_i)$ is open $\mathbb{RP}^n$ for each $i$.\retTwo

	Next, define a map $\varphi_i: U_i \to \mathbb{R}^n$ by:
	
	{\centering$\varphi([(x^1, \ldots, x^{n+1})]) \coloneqq (\frac{x^1}{x^i}, \ldots, \frac{x^{i-1}}{x^i}, \frac{x^{i+1}}{x^i}, \ldots,  \frac{x^{n+1}}{x^i})$.\retTwo\par}

	If we scale $x \in \mathbb{R}^{n+1}$ by a nonzero constant, $\varphi_i([x])$ doesn't change. It follows that $\varphi_i$ is a well-defined map. Also, $\varphi_i$ is continuous since it's clear that $\varphi_i \circ \pi$ is continuous and $\pi$ is a quotient map. Finally, to show that $\varphi_i$ has a continuous inverse, note that:

	{\centering $\varphi_i^{-1}(u^1, \ldots, u^n) = [(u^1, \ldots, u^{i-1}, 1, u^{i+1}, \ldots, u^{n+1})]$.\retTwo\par}

	Since $\varphi_i^{-1}$ is the composition of a continuous function from $\mathbb{R}^n$ to $\mathbb{R}^{n+1}$ and $\pi$ which is also continuous, we have that $\varphi_i^{-1}$ is continuous. So $\varphi_i$ is a homeomorphism.\retTwo

	Since the $U_i$ cover $\mathbb{RP}^n$ we've thus shown that $\mathbb{RP}^n$ is locally Euclidean of dimension $n$. The proof that $\mathbb{RP}^n$ is second countable and Hausdorff is left as an exercise by Lee.
	\begin{myIndent}\pracTwo
		We know $\mathbb{RP}^n$ is second countable since it is the union of a finite number of open second countable sets.\retTwo

		To show that $\mathbb{RP}^n$ is Hausdorff, suppose that $[x_1], [x_2] \in \mathbb{RP}^n$ with $[x_1] \neq [x_2]$, and without loss of generality suppose $x_1$ and $x_2$ are unit vectors. Then $x_1$ and $x_2$ are not collinear. So, using the Hausdorffness of $S^n$, there is an open neighborhood $U_1 \subseteq S^n$ of $x_1$ such that $x_2, -x_2 \notin U_1$, and similarly there is an open neighborhood $U_2 \subseteq S^n$ of $x_2$ such that $x_1, -x_1 \notin U_2$.\newpage

		Now define $V_i = \{cu : c \in \mathbb{R} - \{0\}, u \in U_i\}$ for each $i$. Then each $V_i$ is easily checked to be a saturated open set with respect to $\pi$. Also, we have that $[x_2] \notin \pi(V_1)$ and $[x_1] \notin \pi(V_2)$. So $\pi(V_1)$ and $\pi(V_2)$ are disjoint open sets in $\mathbb{RP}^n$ separating $[x_1]$ and $[x_2]$.\retTwo

		One more note is that if we restrict $\pi$ to just $S^n$, then we get a continuous surjective map from a compact set to $\mathbb{RP}^n$. This says that $\mathbb{RP}^n$ is compact. $\blacksquare$\retTwo
	\end{myIndent}
\end{myIndent}

\exOne\ul{Lemma 1.10:} Every topological manifold  $M$
without boundary has a countable basis of precompact \udefine{coordinate balls} (i.e. sets that are homeomorphic to an open ball in $\mathbb{R}^n$) and \udefine{coordinate half-balls} (i.e. sets that are homeomorphic to $B_r(x) \cap H^n$ where $B_r(x)$ is a ball centered at a point in $\partial H^n$).

\begin{myIndent}\exTwoP
	Proof:\\
	Firstly, since every point in $M$ has a coordinate patch (Lee uses the word "chart") about it, there exists an open covering $\{U_\alpha\}_{\alpha \in A}$ of $M$ consisting of domains of\\ coordinate patches $\varphi_\alpha: U_\alpha \to \hat{U_\alpha} \subseteq \mathbb{R}$.\retTwo

	\ul{Lemma:} If $X$ is a second countable space and $\{U_\alpha\}_{\alpha \in A}$ is an open cover of $X$, then there exists a countable subcover $\{U_{\alpha_n}\}_{n \in \mathbb{N}}$ of $X$.
	\begin{myIndent}\exPPP
		Proof:\\
		Let $\mathcal{B} = \{B_n\}_{n \in \mathbb{N}}$ be a countable basis for the topology of $X$. Then for each $n$ we know there exists $U_{\alpha_n}$ with $B_n \subseteq U_{\alpha_n}$. Then in turn, since $\mathcal{B}$ is an open cover of $X$, we have that $\{U_{\alpha_n}\}_{n \in \mathbb{N}}$ also covers $X$.\retTwo
	\end{myIndent}

	It follows from that lemma there is a countable collection of coordinate charts\\ $\varphi_{\alpha_n} : U_{\alpha_n} \to \hat{U}_{\alpha_n}$ covering $M$. Then, it's obvious how each $\varphi_{\alpha_n}$ defines a countable basis for $U_{\alpha_n}$ consisting of precompact coordinate balls and half balls, and if we take the union of all those bases for each $n$, we get a countable basis for all of $M$. $\blacksquare$\retTwo
\end{myIndent}

\ul{Corollary:} Every topological manifold $M$ is \udefine{locally path connected} (meaning there exists a basis of $M$ consisting of path connected sets). 
\begin{myIndent}\exPPP
	This is cause every coordinate ball is path connected on account of being homemorphic to a path connected set.\retTwo
\end{myIndent}

\hOne\mySepTwo

Now due to my patchy background, I'm only sorta familiar with how topological connectedness works. So, here is my attempt at reviewing / teaching some stuff to myself:

\begin{myIndent}\hTwo
	Here's what I already know / have proven in math 240B (Folland exercise 4.10):
	\begin{itemize}
		\item A topological space $X$ is \udefine{connected} if there doesn't exist two nonempty open sets $U$ and $V$ in $X$ such that $U \cap V = \emptyset$ and $U \cup V = X$.
		\item Equivalently, $X$ is connected if the only two clopen sets are $\emptyset$ and $X$.\newpage
		\item If $\{E_{\alpha}\}_{\alpha \in A}$ is a collection of connected sets with nonempty intersection, then $E \coloneqq \bigcup_{\alpha \in A} E_\alpha$ is connected.
		\item If $E \subseteq X$ is connected, then so is $\overline{E}$.
		\item For every $x \in X$ there is a maximal connected set $E \subseteq X$ containing $x$. This\\ set is called a \udefine{component} of $X$. Additionally, we know that $E$ is closed.
		\item A topological space $X$ is \udefine{path connected} if for any $x, y \in X$, there is a\\ continuous map $f : [0, 1] \to X$ satisfying that $f(0) = x$ and $f(1) = y$.
		\item Any convex set in a topological vector space is path connected.\retTwo
	\end{itemize}

	Here's some stuff I've used but not ever gotten around to proving before.
	\begin{itemize}
		\item If $X$ is path connected, then $X$ is connected.
		\begin{myIndent}\pracTwo
			Proof:\\
			If $X$ has only $1$ element, then $X$ is trivially both path-connected and connected.\retTwo

			For the sake of contradiction, suppose $X$ is path connected and that there exists nonempty open sets $U, V \subseteq X$ such that $U \cap V = \emptyset$ and $U \cup V = X$. Then pick $x \in U - V$ and $y \in V - U$. Since $X$ is path connected, we know there is a path $f: [0, 1] \to X$ from $x$ to $y$.\retTwo

			If we define $U^\prime = f^{-1}(U)$ and $V^\prime = f^{-1}(V)$, then the continuity of $f$\\ guarentees that $U^\prime$ and $V^\prime$ are open subsets of $[0, 1]$. Also, since $U$ and $V$ are disjoint and contain the entire range of $f$, we know that $U^\prime$ and $V^\prime$ are disjoint and their union is all of $[0, 1]$. And since $0 \in U^\prime$ and $1 \in V^\prime$, neither sets are empty. All of this would points towards $[0, 1]$ being a disconnected set.\retTwo

			However, $[0, 1]$ is connected.
			\begin{myIndent}\pracTwo
				Let $U$ and $V$ be two open sets partitioning $[0, 1]$, and without loss of generality suppose $0 \in U$. Then set $\alpha \coloneqq \sup \{s > 0 : [0, s) \subseteq U\}$. If $\alpha < 1$, then it'd be clear that $\alpha \notin U$. But it'd also be clear that $\alpha$ is not an interior point of $V = [0, 1] - U$, which would contradict that $V$ is open. So, we know that $\alpha = 1$. But now this requires that $[0, 1] - U$ equals either $\{1\}$ or $\emptyset$, and only the latter is open in $[0, 1]$. So there does not exist two open nonempty sets which form a partition of $[0, 1]$.\retTwo
			\end{myIndent}

			Hence, we have a contradiction and conclude that it is impossible for $X$ to be path connected but not connected. $\blacksquare$\retTwo
		\end{myIndent}

		\item If $f: X \to Y$ is continuous and $X$ is connected, then $f(X)$ is connected.\\ Similarly, if $X$ is path connected, then $f(X)$ is path connected.
		\begin{myIndent}\pracTwo
			Proof:\\
			If $X$ is connected, let $U$ and $V$ be open sets in the subspace topology which\\ partition $f(X)$. Then $f^{-1}(U)$ and $f^{-1}(V)$ are disjoint open subsets of $X$ whose union is $X$. Since $X$ is connected, it follows that either $f^{-1}(U)$ or $f^{-1}(V)$ is empty. And since $f(f^{-1}(U)) = U$ and $f(f^{-1}(V)) = V$ on\\ account of the fact that both $U, V \subseteq f(X)$, we know either $U$ or $V$ is empty. So, $f(X)$ is connected.\newpage

			If $X$ is path connected, consider any $y_1, y_2 \in f(X)$. Then $\exists x_1, x_2 \in X$\\ with $f(x_i) = y_i$ for $i = 1,2$, as well as a path $g: [0, 1] \to X$ going from $x_1$ to $x_2$. It follows that $f \circ g$ is a path going from $y_1$ to $y_2$ in $f(X)$. $\blacksquare$\retTwo
		\end{myIndent}

		\item If $\{E_\alpha\}_{\alpha \in A}$ is a collection of path connected sets with nonempty intersection, then $E \coloneqq \bigcup_{\alpha \in A}E_\alpha$ is connected.
		\begin{myIndent}\pracTwo
			Proof:\\
			Consider any $x, y \in E$, and let $z \in \bigcap_{\alpha \in A}E_\alpha$. Then there is path contained in one of the $E_\alpha$ going from $x$ to $z$, and there is another path in another $E_\alpha$\\ going from $z$ to $y$. Combining those paths gives us a path from $x$ to $y$. $\blacksquare$\retTwo
		\end{myIndent}

		\item $E$ being path-connected does not necessarily mean that $\overline{E}$ is. 
		\begin{myIndent}\pracTwo
			Proof:\\
			Let $S = \{(t, \sin(1/t)) \in \mathbb{R}^2 : t > 0\}$. Then $S$ is clearly path connected. Also, we can see that $\overline{S} = S \cup (\{0\} \times [-1, 1])$. But, there is no continuous path going from any point in $\{0\} \times [-1, 1]$ to any point in $S$. Hence $\overline{S}$ is not connected. $\blacksquare$\retTwo
		\end{myIndent}
		
		\item For every $x \in X$ there is a maximal path connected set $E \subseteq X$ containing $x$. This set is called a \udefine{path component} of $X$.
		\begin{myIndent}\pracTwo
			Proof:\\
			Let $\mathcal{E} = \{E_{\alpha}\}_{\alpha \in A}$ be the collection of every path connected set in $X$ containing $x$. Note that $\mathcal{E}$ is not empty since $\{x\} \in \mathcal{E}$. Then $E = \bigcup_{\alpha \in A} E_\alpha$ is path connected. Also, clearly it is a maximal path connected set containing $x$. $\blacksquare$\retTwo
		\end{myIndent}

		\item Let $X$ be a topological space, and given $x \in X$ let $E_x$ be the maximal connected component of $X$ containing $x$. Then $\{E_x\}_{x \in X}$ forms a partition of $X$, meaning that if $E_x \cap E_y \neq \emptyset$, then $E_x = E_y$. 
		\begin{myIndent}\pracTwo
			Proof:\\
			Suppose $x, y \in X$ satisfy that $E_x \cap E_y \neq \emptyset$. Then $E_x \cup E_y$ is a connected subset containing both $x$ and $y$. So, since $E_x$ and $E_y$ are maximal, we have that $E_x \cup E_y \subseteq E_x, E_y$. It follows that $E_x = E_x \cup E_y = E_y$. $\blacksquare$\retTwo
		\end{myIndent}

		\item The previous bullet point also holds if we replace "maximal connected\\ component" with "maximal path connected component".\retTwo

		\item Two sets $A$ and $B$ are \udefine{separated} if $A \cap \overline{B} = \emptyset$ and $\overline{A} \cap B = \emptyset$. A topological space $X$ is connected if and only if no two nonempty sets whose union is all of $X$ are disconnected. (This is the definition in math 140a\dots)
		\begin{myIndent}\pracTwo
			$(\Longrightarrow)$\\
			Let $A$ and $B$ be two nonempty separated sets satisfying that $A \cup B = X$.\\ Then since $\overline{A} \cup B = X$ and $\overline{A} \cap B = \emptyset$, it's clear that $B = (\overline{A})^\comp$. Similarly,\\ it's clear that $A = (\overline{B})^\comp$. So, both $A$ and $B$ are open. But now $A$ and $B$ are\\ disjoint nonempty open sets partitioning $X$. This means that $X$ is not\\ connected.\newpage

			$(\Longleftarrow)$\\
			Let $U$ and $V$ be two disjoint nonempty open sets whose union is $X$. Then $V = U^\comp$ and $U = V^\comp$. This means $U$ and $V$ are closed and so $U = \overline{U}$, $V = \overline{V}$, and clearly both $U$ and $V$ are separated. $\blacksquare$\retTwo
		\end{myIndent}

		\item If $X$ is a topological space and $E \subseteq X$ is clopen, then $E$ is a union of connected components.
		\begin{myIndent}\pracTwo
			Proof:\\
			It suffices to show that if $x \in E$ and $A_x$ is the connected component of $X$ containing $x$, then $A_x \subseteq E$. Fortunately, since $A_x$ is closed and $E$ is clopen, we know that $A_x \cap E$ and $A_x - E$ are both closed subsets of $X$. In turn, we know that  $A_x \cap E$ and $A_x - E$ are disjoint open sets in the relative topology of $A_x$ whose union is all of $A_x$. Since $A_x$ is connected, it follows that either $A_x \cap E = \emptyset$ or $A_x - E = \emptyset$. But the former case is not true since $x \in A_x \cap E$. So, we know that $A_x - E = \emptyset$. This shows that $A_x \subseteq E$. $\blacksquare$\retTwo
		\end{myIndent}
	\end{itemize}

	Here's some nicher proofs:
	\begin{itemize}
		\item A space $X$ is \udefine{locally connected} if $X$ has a basis consisting of sets which are connected. Similarly, $X$ is \udefine{locally path connected} if $X$ has a basis consisting of sets which are path connected.\retTwo
		
		\item If $X$ is locally connected, then every connected component of $X$ is open.
		\begin{myIndent}\pracTwo
			Proof:\\
			Let $E$ be a connected component of $X$. Then for all $y \in E$, there exists a\\ connected open set $U_y$ containing $y$. And since $E$ is the maximal connected set containing $y$, we have that $U_y \subseteq E$. Hence, $E = \bigcup_{y \in E}U_y$ is open. $\blacksquare$\retTwo
		\end{myIndent}

		\item By identical reasoning to the last bullet point, if $X$ is locally path connected, then every path component of $X$ is open.

		\item Suppose $X$ is locally path connected. Then $X$ is connected if and only if $X$ is path connected.
		\begin{myIndent}\pracTwo
			$(\Longrightarrow)$\\
			It suffices to show that $X$ has only one path component. Luckily, if $X$ had more than one, then since every path component is open, we'd be able to take the union of all but one in order to get two disjoint nonempty open sets whose union is all of $X$. But this contradicts that $X$ is connected.\retTwo

			$(\Longleftarrow)$\\
			We proved this direction several bullet points ago. $\blacksquare$\retTwo
		\end{myIndent}

		\item If $\{X_\alpha\}_{\alpha \in A}$ is a family of connected topological spaces, then the product space $X = \prod_{\alpha \in A} X_\alpha$ is also connected.
		\begin{myIndent}\pracTwo
			Proof:\\
			Let $<$ be a well-ordering of $A$, and for any $\beta \in A$ let $S_\beta = \{\alpha \in A: \alpha < \beta\}$\\ [-2pt] and $\overline{S_\beta} = S_\beta \cup \{\beta\}$. We shall proceed via transfinite induction.\newpage

			Let $\langle x_\alpha \rangle_{\alpha \in A} \in \prod_{\alpha \in A} X_\alpha$. Then suppose that $\beta \in A$ satisfies that\\ $(\prod\limits_{\alpha \in \overline{S_\gamma}} X_\alpha) \times (\hspace{-0.4em}\prod\limits_{\alpha \in (\overline{S_\gamma})^\comp}\hspace{-0.4em} \{x_\alpha\})$ is connected for every $\gamma \in S_\beta$.\retTwo

			We claim that $(\prod\limits_{\alpha \in \overline{S_\beta}} X_\alpha) \times (\hspace{-0.4em}\prod\limits_{\alpha \in (\overline{S_\beta})^\comp}\hspace{-0.4em} \{x_\alpha\})$ is connected.\retTwo

			To prove this, first note that if $\beta \in A$, then $X_\beta \times \prod_{\alpha \in A - \{\beta\}} \{x_\alpha\}$ is connected. 
			
			\begin{myIndent}
				Proof:\\
				Let $U$ and $V$ be disjoint open sets which partition $X_\beta \times \prod_{\alpha \in A - \{\beta\}} \{x_\alpha\}$, and let $\pi_\beta$ be the projection of $X$ onto $X_\beta$. Then $\pi_\beta(U)$ and $\pi_\beta(V)$ are disjoint open sets which partition $X_\beta$. It follows that one of those two sets is empty, and the only way that is possible is if either $U$ or $V$ is empty.\retTwo
			\end{myIndent}

			If $\overline{S_\beta} = \{\beta\}$, then this already proves our claim. Otherwise, notice that:

			{\centering $(\prod\limits_{\alpha \in \overline{S_\beta}} X_\alpha) \times (\hspace{-0.4em}\prod\limits_{\alpha \in (\overline{S_\beta})^\comp}\hspace{-0.4em} \{x_\alpha\}) = \bigcup\limits_{\gamma \in S_\beta}(X_\beta \times (\prod\limits_{\alpha \in \overline{S_\gamma}} X_\alpha) \times (\hspace{-1.3em}\prod\limits_{\alpha \in (\overline{S_\gamma})^\comp - \{\beta\}}\hspace{-1.3em} \{x_\alpha\}))$ \retTwo\par}

			Now each of the sets in that union contain $\langle x_\alpha \rangle$. Also, we claim that each of them are connected. After all, note that:
			
			{\centering $X_\beta \times (\prod\limits_{\alpha \in \overline{S_\gamma}} X_\alpha) \times (\hspace{-1.3em}\prod\limits_{\alpha \in (\overline{S_\gamma})^\comp - \{\beta\}}\hspace{-1.3em} \{x_\alpha\}) = \bigcup\limits_{y \in X_\beta} (\{y\} \times (\prod\limits_{\alpha \in \overline{S_\gamma}} X_\alpha) \times (\hspace{-1.3em}\prod\limits_{\alpha \in (\overline{S_\gamma})^\comp - \{\beta\}}\hspace{-1.3em} \{x_\alpha\}))$ \retTwo\par}

			We already know that $\{x_\beta\} \times (\prod\limits_{\alpha \in \overline{S_\gamma}} X_\alpha) \times (\hspace{-1.3em}\prod\limits_{\alpha \in (\overline{S_\gamma})^\comp - \{\beta\}}\hspace{-1.3em} \{x_\alpha\})$ is connected.\retTwo

			Also, for any $y \in X_\beta$, we can find a continuous map $f$ from $X$ to itself which sets the $\beta$th coordinate of a point to $y$ and otherwise acts as the identity for all other coordinates. (We know this map is continuous because $\pi_\alpha \circ f$ is trivially continuous for all $\alpha \in A$\dots) This in turn shows that:

			{\centering $\{y\} \times (\prod\limits_{\alpha \in \overline{S_\gamma}} X_\alpha) \times (\hspace{-1.3em}\prod\limits_{\alpha \in (\overline{S_\gamma})^\comp - \{\beta\}}\hspace{-1.3em} \{x_\alpha\})$ is connected for all $y$. \retTwo\par}

			And since $X_\beta \times \prod_{\alpha \in A - \{\beta\}} \{x_\alpha\}$ is a connected set intersecting the set in the previous paragraph for all $y$ and is a subset of the union:
			
			{\center $\bigcup\limits_{y \in X_\beta}( \{y\} \times (\prod\limits_{\alpha \in \overline{S_\gamma}} X_\alpha) \times (\hspace{-1.3em}\prod\limits_{\alpha \in (\overline{S_\gamma})^\comp - \{\beta\}}\hspace{-1.3em} \{x_\alpha\}))$,\par}

			we've thus shown that the entire union is connected. This finishes proving our claim at the top of this page.\retTwo
			
			By transfinite induction, we can now conclude that:
			
			{\centering $(\prod\limits_{\alpha \in \overline{S_\beta}} X_\alpha) \times (\hspace{-0.4em}\prod\limits_{\alpha \in (\overline{S_\beta})^\comp}\hspace{-0.4em} \{x_\alpha\})$ is connected for all $\beta \in A$.\retTwo\par}

			Finally, to finish our proof we can just note that:

			{\centering $X = \bigcup\limits_{\beta \in A}( (\prod\limits_{\alpha \in \overline{S_\beta}} X_\alpha) \times (\hspace{-0.4em}\prod\limits_{\alpha \in (\overline{S_\beta})^\comp}\hspace{-0.4em} \{x_\alpha\}))$ \newpage\par}

			Also, every set in the above union is connected and contains $\langle x_\alpha \rangle_{\alpha \in A}$. Hence $X$ is connected.\retTwo
		\end{myIndent}

		\item If $\{X_\alpha\}_{\alpha \in A}$ is a family of path connected topological spaces, then the product space $X = \prod_{\alpha \in A} X_\alpha$ is also path connected.
		
		\begin{myIndent}\pracTwo
			Proof:\\
			Let $\langle x_\alpha\rangle_{\alpha \in A}$ and $\langle y_\alpha \rangle_{\alpha \in A}$ be elements of $X$. Then we know for each $\alpha \in A$ that there is a path $f_\alpha: [0, 1] \to X_\alpha$ such that $f_\alpha(0) = x_\alpha$ and $f_\alpha(1) = y_\alpha$. If we now define $f: [0, 1] \to X$ by $f(t) = \langle f_\alpha(t) \rangle_{\alpha \in A}$, we will have that $f$ is a continuous path from $\langle x_\alpha\rangle_{\alpha \in A}$ to $\langle y_\alpha \rangle_{\alpha \in A}$.
			\begin{myIndent}
				(It is continuous because $\pi_{\alpha} \circ f = f_\alpha$ is continuous for all $\alpha \in A$.) $\blacksquare$\retTwo
			\end{myIndent}
		\end{myIndent}

		\item Every quotient space of a connected space is connected. Also, every quotient space of a path connected space is path connected.
		
		\begin{myIndent}\pracTwo
			Proof:\\
			Let $X^*$ be a partition of $X$ and let $f: X \to X^*$ be the function mapping every element to the set in $X^*$ containing it. If we equip $X^*$ with the quotient\\ topology with resepct to $f$, then $f$ will be continuous. Hence since $X$ is connected, so will $f(X) = X^*$.\retTwo
			
			Similar reasoning works when $X$ is path connected. $\blacksquare$\retTwo
		\end{myIndent}

		\item If $X$ is a locally connected space, then every open set in $X$ is locally connected. Similarly, if $X$ is a locally path connected space, then so is every open set in $X$.
		\begin{myIndent}\pracTwo
			This should be obvious.\retTwo
		\end{myIndent}

		\item If $(X, \mathcal{T})$ is a topological space and $\mathcal{T}^\prime$ is a coarser topology on $\mathcal{T}$, then $(X, \mathcal{T})$ being connected implies that $(X, \mathcal{T}^\prime)$ is connected. Similarly, $(X, \mathcal{T})$ being path connected implies that $(X, \mathcal{T}^\prime)$ is path connected. Another way of thinking about this is that adding sets to a topology only makes your space more\\ disconnected.

		\begin{myIndent}\pracTwo
			Proof:\\
			If $(X, \mathcal{T}^\prime)$ weren't connected, then the disjoint sets in $\mathcal{T}^\prime$  partitioning $X$ would also be an open partition in $(X, \mathcal{T})$.\retTwo
			
			Next, if $f : [0, 1] \to X$ is continuous with respect to $\mathcal{T}$, then we know it is also continuous with respect to $\mathcal{T}^\prime$. This shows that a path with respect to $\mathcal{T}$ is also a path with respect to $\mathcal{T}^\prime$. $\blacksquare$\retTwo
		\end{myIndent}
	\end{itemize}
\end{myIndent}

\mySepTwo

\hOne\dispDate{8/22/2025}

To start off today, I'm going to do an exercise from Folland.\newpage

\Hstatement\blab{Exercise 4.57:} A collection $\mathcal{U}$ of open sets in $X$ is called \udefine{locally finite} if each $x \in X$ has a\\ neighborhood that intersects only finitely many members of $\mathcal{U}$. If $\mathcal{U}$ and $\mathcal{V}$ are open covers of $X$, $\mathcal{V}$ is a \udefine{refinement} of $\mathcal{U}$ if for each $V \in \mathcal{V}$ there exists $U \in \mathcal{U}$ with $V \subseteq U$. $X$ is called \udefine{paracompact} if every open cover of $X$ has a locally finite refinement.

\begin{myIndent}\myComment\fontsize{11}{12}\selectfont
	Clearly any compact set is automatically paracompact since a finite subcover of an open cover will automatically be a locally finite refinement of that cover. However, beware that a refinement of a cover doesn't need to be a subset of the original cover, and it is possible for a cover to be locally finite without being finite.\retTwo

	The following exercise generalizes the theorems written on pages 81 and 82 of this journal.
\end{myIndent}
\begin{itemize}
	\item[(a)] If $X$ is a $\sigma$-compact LCH space, then $X$ is paracompact. In fact, every open cover $\mathcal{U}$ has locally finite refinements $\{V_\alpha\}_{\alpha \in A}$ and $\{W_\alpha\}_{\alpha \in A}$ such that $\overline{V_\alpha}$ is compact and $\overline{W}_\alpha \subseteq V_\alpha$ for all $\alpha \in A$.
	\begin{myIndent}\HexOne
		Let $(U_n)_{n \in \mathbb{N}}$ be an increasing sequence of precompact open sets such that $\overline{U_n} \subseteq U_{n+1}$ and $X = \bigcup_{n \in \mathbb{N}} U_n$. (Since $X$ is a $\sigma$-compact LCH space, we proved in math 240b that such a sequence must exist\dots) Also, for ease of notation take $U_n = \emptyset$ whenever $n \leq 0$\retTwo

		Now, the collection $\{E \cap (U_{n+2} - \overline{U_{n-1}}): E \in \mathcal{U}\}$ is an open cover of $\overline{U_{n+1}} - U_n$.\\ [-2pt] And since $\overline{U_{n+1}} - U_n$ is compact, we can choose a finite subcover $\mathcal{V}_n$ from that\\ collection. Doing this for all $n$ and then setting $\mathcal{V} = \bigcup_{n \in \mathbb{N}} \mathcal{V}_n$, we have that $\mathcal{V}$ is a\\ [-2pt] locally finite refinement of $\mathcal{U}$. After all, each $x$ has a neighborhood $U_{n+2} - \overline{U_{n-1}}$ which\\  only the finitely many open sets in $\mathcal{V}_n, \mathcal{V}_{n+1}, \mathcal{V}_{n+2}$, $\mathcal{V}_{n-1}$ and $\mathcal{V}_{n-2}$ can intercept. Also,\\ each set in $\mathcal{V}$ is contained in some set of $\mathcal{U}$. And thirdly, we claim that if $V_\alpha \in \mathcal{V}$, then\\ [-1pt] $V_\alpha$ is precompact. This is because $V_\alpha$ will be a closed subset of $\overline{U_{n+2}}$ for some $n$ and\\ the latter is compact.\retTwo

		Having constructed our first refinement $\mathcal{V} = \{V_\alpha\}_{\alpha \in A}$, we're now ready to construct our second. Fix $n$ and note that each $x \in X$ has a compact neighborhood $N_x \subseteq V_\alpha$ for some $V_\alpha \in \mathcal{V}_n$. The $N_x^\circ$ form an open cover of $\overline{U_{n+1}} - U_n$. Hence, there is a finite collection $\{x_1, \ldots, x_m\}$ of points in $\overline{U_{n+1}} - U_n$ such that $\overline{U_{n+1}} - U_n \subseteq \bigcup_{j=1}^m N_{x_j}^\circ$. So, for each $\alpha \in A$ with $V_\alpha \in \mathcal{V}_n$ let $W_\alpha$ be the union of all the $N_{x_j}^\circ$ such that $N_{x_j} \subseteq V_\alpha$, and let $\mathcal{W}_n$ be the collection of $W_\alpha$ defined in this sentence. It's clear that $\mathcal{W}_n$ is also an open cover of $\overline{U_{n+1}} - U_n$, and that $\overline{W_\alpha} \subseteq V_\alpha$ for all $V_\alpha \in \mathcal{V}_n$. Doing this for all $n$, we then have that $\mathcal{W} = \bigcup_{n \in \mathbb{N}}\mathcal{W}_n$ is our second desired refinement.\retTwo
	\end{myIndent}
	
	\item[(b)] If $X$ is a $\sigma$-compact LCH space, for any open cover $\mathcal{U}$ of $X$ there is a partition of unity on $X$ subordinate to $\mathcal{U}$ and consisting of compactly supported functions.
	\begin{myIndent}\HexOne
		Let $\{V_n\}_{n \in \mathbb{N}}$ and $\{W_n\}_{n \in \mathbb{N}}$ be refinements of $\mathcal{U}$  constructed as in part (a). Note that our refinements in the last part were countable. So, there is no issue just taking the indexing set $A$ to be $\mathbb{N}$.\retTwo

		For each $n$ there exists by Urysohn's lemma a function $f_n \in C_c(X, [0, 1])$ such that\\ [-1pt] $f_n(x) = 1$ for all $x \in \overline{W_n}$ and $f_n(x) = 0$ for all $x \in V_n^\comp$. Setting $f = \sum_{n=1}^\infty f_n$, we\\ [1pt] then have that $f$ is continuous and finite everywhere in $X$ on account of the fact that every $x \in X$ has a neighborhood where $f$ is only a sum of finitely many continuous\\ [1pt] functions on that neighborhood. Also, since every $x \in X$ is contained in $W_n$, we know that $f(x) \geq 1$ for all $x$. Hence, if we define $g_n = f_n / f$ for each $n$, we have that $g_n$ is well-defined and still in $C_c(X, [0, 1])$.\newpage

		We claim $(g_n)_{n \in \mathbb{N}}$ is a partition of unity subordinate to $\mathcal{U}$. After all,\\ $\sum_{n \in \mathbb{N}} g_n = \frac{1}{f}\sum_{n \in \mathbb{N}} f_n = 1$. Also, each $g_n$ satisfies that $\supp(g_n) \subseteq V_n$ and $V_n \subseteq U$ for some $U \in \mathcal{U}$. And finally, it is still the case that every $x \in X$ has a neighborhood on which only finitely many of the $g_n$ are nonzero.

		\begin{myIndent}\HexPPP
			As a side note, we can extend this result to proving theorem 16.3 on page 82 of this journal by just noting that $\mathbb{R}^n$ is a $\sigma$-compact LCH space and that the Urysohn lemma on $\mathbb{R}^n$ specifies that we can choose each of our $f_n$ to be in $C_c^\infty(X, [0, 1])$.
		\end{myIndent}
	\end{myIndent}
\end{itemize}

\hOne Now I shall go over some more of John Lee's book.\retTwo

Since manifolds are locally path connected, we know that a manifold is connected if and only if it is path connected. Also, it is clear that the path components of a manifold are identical to it's components. One more proposition is as follows:\retTwo

\exOne\ul{Proposition 1.11.d:} If $M$ is a topological manifold, then $M$ has countably many\\ components, each of which are open subsets of $M$ and a topological manifold\\ by themselves.

\begin{myIndent}\exTwoP
	Proof:\\
	Since $M$ is locally path connected, we know every single component is an open set. It follows that the components form an open cover $\mathcal{U}$ of $M$. And since $M$ is second countable, this means that there is a countable subcover of $\mathcal{U}$. Yet, because all the sets of $\mathcal{U}$ are disjoint, the only way this is possible is if $\mathcal{U}$ was countable to begin with.\retTwo

	Also, it's clear that every component equipped with the subspace topology will still be second countable and Hausdorff. And to show that each component is locally Euclidean of dimension $n$, just restrict the domain and codomain of the coordinate patches covering that component. $\blacksquare$\retTwo
\end{myIndent}

\hOne Another consequence of every manifold having a countable basis of precompact coordinate balls and half balls is that manifolds are locally compact and $\sigma$-compact. And by the exercise I did earlier today, that means that every manifold is\\ paracompact.

\begin{myIndent}\hTwo
	In fact, by slightly modifying our construction of $\mathcal{W}$ in the exercise I did before (namely by picking each $N_x$ to be the closure of a precompact coordinate ball and then letting $\mathcal{W}_n$ consist of the $N_{x_j}^\circ$ without bothering to take any unions), we can construct a locally finite refinement consisting of coordinate balls and half balls for any open cover of $M$.\retTwo
\end{myIndent}

\pracOne Here is a proposition about locally finite collections of sets.\retTwo

\ul{Exercise 1.14:} Suppose $\mathcalli{X}$ is a locally finite collection of subsets of a topological space $M$.
\begin{itemize}
	\item[(a)] The collection $\{\overline{X} : X \in \mathcalli{X}\}$ is also locally finite.\newpage
	
	\begin{myIndent}\pracTwo
		Proof:\\
		Let $x \in M$ and let $U \subseteq M$ be an open set containing $x$ that intersects only the\\ elements of a finite subset $\mathcalli{Y}$ of $\mathcalli{X}$. Now we want to show that if $X \in \mathcalli{X}$ satisfies\\ $\overline{X} \cap U \neq \emptyset$, then $X \in \mathcalli{Y}$. Luckily, if $\overline{X} \cap U \neq \emptyset$ but $X \cap U = \emptyset$ so that $X \notin \mathcalli{Y}$, then it must be that any $x \in \overline{X} \cap U$ is an accumulation point of $X$. However, that is immediately contradicted by the fact that $U$ is a neighborhood of $x$ which does not intercept $X$ anywhere. $\blacksquare$\retTwo	
	\end{myIndent}
	
	\item[(b)] $\overline{\bigcup_{X \in \mathcalli{X}} X} = \bigcup_{X \in \mathcalli{X}} \overline{X}$.
	
	\begin{myIndent}\pracTwo
		Proof:\\
		It is always true that $\bigcup_{X \in \mathcalli{X}} \overline{X} \subseteq \overline{\bigcup_{X \in \mathcalli{X}} X}$.\retTwo

		Meanwhile, to show the other inclusion, suppose $x$ is an accumulation point of\\ $\bigcup_{X \in \mathcalli{X}} X$ but not in any individual $\overline{X}$. Then it must be the case that every\\ neighborhood of $x$ intercepts some set in $\mathcalli{X}$, yet it must also be the case that for every $X \in \mathcalli{X}$ there is a neighborhood of $x$ with doesn't intercept $X$. The only way this is possible is if every neighborhood intercepts infinitely many $X \in \mathcalli{X}$.
		\begin{myIndent}
			Let $N$ be any neighborhood of $x$. Then we can construct an infinite sequence of distinct sets $(X_n)_{n \in \mathbb{N}}$ in $\mathcalli{X}$ intercepting $N$ as follows.\retTwo

			For the ease of notation set $U_0 = M$. Now at step $n$, choose any $X_n \in \mathcalli{X}$ that intersects $N \cap \bigcap_{k=0}^{n-1} U_k$. Note that we can do this since a finite intersection of neighborhoods of $x$ is still a neighborhood of $x$. But now we can choose some neighborhood $U_{n}$ of $x$ such that $X_n \cap U_n = \emptyset$. And now repeat this reasoning.\retTwo

			It's clear that all the chosen $X_n$ intercept $\bigcap_{k=0}^{n-1} U_k \subseteq N$. Additionally, all the chosen $X_n$ are distinct since $X_n \cap \bigcap_{k=0}^{N} U_k \subseteq X_n \cap U_n = \emptyset$ for all $N \geq n$.\retTwo
		\end{myIndent}

		But that contradicts that $\mathcalli{X}$ is locally finite. $\blacksquare$\retTwo
	\end{myIndent}
\end{itemize}

\exOne\ul{Proposition 1.16:} The fundamental group of a topological manifold $M$ is countable.
\begin{myIndent}\exTwoP
	Proof:\\
	Let $\mathcalli{B}$ be a countable collection of coordinate balls and half balls covering $M$. For any $B, B^\prime \in \mathcalli{B}$, the intersection $B \cap B^\prime$ has at most countably many components each of which is path connected.
	\begin{myIndent}\exPPP
		Why? $B \cap B^\prime$ is second countable and locally path connected on account of being an open subset of $M$. In turn, by the same reasoning as in proposition 1.11.d we know that $B \cap B^\prime$ has only countably many components.\retTwo
	\end{myIndent}

	Let $\mathcalli{X}$ be a countable set containing a point from each component of $B \cap B^\prime$ for every $B, B^\prime \in \mathcalli{B}$ (including $B = B^\prime$). Also, for each $B \in \mathcalli{B}$ and $x, x^\prime \in \mathcalli{X}$ satisfying that $x, x^\prime \in B$, let $h^B_{x, x^\prime}$ be some path from $x$ to $x^\prime$ in $B$. Since $\mathcalli{X}$ intercepts every component of $M$, it suffices when calculating the fundamental group to take our base point $p$ to be in $\mathcalli{X}$. Then, we define a \textit{special loop} to be a loop based at $p$ that is a finite product of paths $h^B_{x,x^\prime}$.\newpage

	Now there are only countably many special loops. Therefore, in order to prove that $\pi_1(M, p)$ is countable, it suffices to show that if $f: [0,1] \to M$ is a loop based at $p$, then $f$ is homotopic to some special loop.\retTwo

	Fortunately, the collection of the components of the sets $f^{-1}(B)$ with $B \in \mathcalli{B}$ is an open cover of $[0, 1]$. So, there exists $0 = a_0 < a_1 < \cdots < a_k = 1$ such that for each $i \geq 1$, $[a_{i-1}, a_i] \subseteq f^{-1}(B)$ for some $B \in \mathcalli{B}$. Now for each $i$, let $f_i$ be the restriction of $f$ to interval $[a_{i-1}, a_i]$ and then reparametrized so that its domain is $[0, 1]$, and also let $B_i \in \mathcalli{B}$ be a coordinate ball containing the image of $f_i$. For each $1 \leq i < k$, $f(a_i) \in B_i \cap B_{i + 1}$. Also there is some $x_i \in \mathcalli{X}$ that lies in the same component of $B_i \cap B_{i+1}$ as $f(a_i)$. So, let $g_i$ be a path in $B_i \cap B_{i+1}$ from $x_i$ to $f(a_i)$. 
	\begin{myIndent}\exPPP
		Note, we'll also write $x_0 = x_k = p$ and $g_0 = g_k$ are the constant paths based at $p$.\\ Also, like in Munkres, we'll denote $\bar{g}_i$ to be the reverse path.\retTwo
	\end{myIndent}

	Now if we denote $\tilde{f}_i \coloneqq g_{i-1} * f_i * \bar{g}_i$, we have that:

	{\center\begin{tabular}{l}
	$f \simeq_p f_1 * \cdots * f_k$\\ [6pt]
	$\phantom{f} \simeq_p g_0 * f_1 * \bar{g}_1 * g_1 * f_2 * \bar{g}_2 * g_2 * \cdots * \bar{g}_{k-1} * g_{k-1} * f_k * \bar{g}_k$\\ [6pt]
	$\phantom{f} \simeq_p \tilde{f}_1 * \cdots * \tilde{f}_k$.
	\end{tabular}\retTwo\par}

	But now since each $B_i$ is simply connected, we know that $\tilde{f}_i$ is path homotopic to $h^{B_i}_{x_{i-1}, x_i}$. In turn, we have that $f$ is path homotopic to a special loop. $\blacksquare$\retTwo
\end{myIndent}

\hOne\mySepTwo

A question I've had for a while is how smoothness can even be defined for manifolds which aren't subsets of $\mathbb{R}^n$. After all, differentiability as a concept relies on the topological structure of the reals. It turns out that the answer to this problem is to reframe some prior theorems as defining axioms which must be met.\retTwo

\begin{myIndent}\hTwo
	Let $M$ be a topological $n$-manifold. Two charts $(U, \varphi)$ and $(V, \psi)$ on $M$ are said to be \udefine{$C^r$ compatible} if either $U \cap V = \emptyset$ or the transition map $\psi \circ \varphi^{-1}$ from $\varphi(U \cap V)$ to $\psi(U \cap V)$ is a $C^r$ diffeomorphism.
	\begin{myIndent}\myComment
		Note that the transition map is a function from an open open subset of either $H^n$ or $\mathbb{R}^n$ to another open subset of either $H^n$ or $\mathbb{R}^n$. So, it makes sense to talk about the transition map as being differentiable.\retTwo
	\end{myIndent}

	An \udefine{atlas} $\mathcal{A}$ for $M$ is a collection of charts covering $M$. $\mathcal{A}$ is called a \udefine{$C^r$ atlas} if any two charts in $\mathcal{A}$ are $C^r$ compatible.\retTwo
\end{myIndent}

Very roughly speaking, if $1 \leq s, r$, we want to define that a manifold $M$ is $C^r$\\ smooth if $M$ has a $C^r$ atlas $\mathcal{A}$. Additionally, we want to then say that a function\\ $f: M \to \mathbb{R}$ is $C^s$ differentiable if and only if $f \circ \varphi^{-1}$ is $C^s$ differentiable for all\\ charts $(U, \varphi) \in \mathcal{A}$.\newpage

A snag we need to work out though is that a space can have many $C^r$ atlases. Also, some of those atlases may have charts which aren't $C^r$ compatible with the charts in the other atlases. In a sense, that would mean they define different "smooth structures". At the same time, it could be the case that two atlases contain charts which are all $C^r$ compatible with the charts in the other atlas. In that case, the two atlases could be said to define the same "smooth structure". We'll get around this issue as follows:

\begin{myIndent}\hTwo
	We define a $C^r$ atlas $\mathcal{A}$ on a topological manifold $M$ to be \udefine{maximal} or \udefine{complete} if it is not contained in a larger $C^r$ atlas. Or in other words, if a chart $(U, \varphi)$ is $C^r$ compatible with every chart in $\mathcal{A}$, then $(U, \varphi) \in \mathcal{A}$.\retTwo

	If $M$ is a topological manifold, then a \udefine{$C^r$ structure on $M$} is a maximal $C^r$ atlas.\retTwo

	A \udefine{$C^r$ manifold} is a pair $(M, \mathcal{A})$ where $M$ is a topological manifold and $\mathcal{A}$ is a $C^r$ structure on $M$.\retTwo
\end{myIndent}

\exOne\ul{Proposition 1.17:} Let $M$ be a topological manifold.
\begin{itemize}
	\item[(a)] Every $C^r$ atlas $\mathcal{A}$ for $M$ is contained in a unique maximal $C^r$ atlas called the \udefine{$C^r$ structure determined by $\mathcal{A}$}.
	
	\begin{myIndent}\exTwoP
		Proof:\\
		Let $\mathcal{A}$ be a $C^r$ atlas for $M$ and define $\overline{\mathcal{A}}$ as the set of all charts which are $C^r$ compatible with every chart in $\mathcal{A}$. If $\overline{\mathcal{A}}$ were a $C^r$ atlas, it would be obvious that it is maximal. After all, if a chart was $C^r$ compatible with every chart in $\overline{\mathcal{A}}$, then it would also be compatible with every chart in $\mathcal{A}$ on account of the fact that $\mathcal{A} \subseteq \overline{\mathcal{A}}$. But that would mean that that chart is also in $\overline{\mathcal{A}}$.\retTwo

		Hence, we proceed by trying to prove that $\overline{\mathcal{A}}$ is a $C^r$ atlas on $M$. Or in other words, we want to show that for any $(U, \varphi), (V, \psi) \in \overline{\mathcal{A}}$, with $U \cap V \neq \emptyset$, the map $\psi \circ \varphi^{-1}: \varphi(U \cap V) \to \psi(U \cap V)$ is smooth.
		\begin{myIndent}\exThreeP
			Choose $x = \varphi(p) \in \varphi(U \cap V)$. Then there is some chart $(W, \theta) \in \mathcal{A}$\\ with $p \in W$. And since every chart in $\overline{\mathcal{A}}$ is $C^r$ compatible with $(W, \theta)$,\\ we know that $\theta \circ \varphi^{-1}$ and $\psi \circ \theta^{-1}$ are both $C^r$ maps. It follows that\\ $\psi \circ \varphi^{-1} = (\psi \circ \theta^{-1}) \circ (\theta \circ \varphi^{-1})$ is a $C^r$ map on the neighborhood\\ $\varphi(U \cap V \cap W)$ of $x$.\retTwo

			This proves that $\phi \circ \varphi^{-1}$ is locally $C^r$. Hence it is $C^r$ in general.\retTwo
		\end{myIndent}

		With that, we've prove that there exists a maximal $C^r$ atlas $\overline{\mathcal{A}}$ containing $\mathcal{A}$. To finish off we show uniqueness. Suppose $\mathcal{B}$ is another maximal $C^r$ atlas containg $\mathcal{A}$. Then every chart in $\mathcal{B}$ must be $C^r$ compatible with every char in $\mathcal{A}$. But that then implies that $\mathcal{B} \subseteq \overline{\mathcal{A}}$. And since $\mathcal{B}$ is maximal, we have that $\mathcal{B} = \overline{\mathcal{A}}$. $\blacksquare$\newpage
	\end{myIndent}
	
	\item[(b)] Two $C^r$ atlases determine the same $C^r$ structure iff their union is is a $C^r$ atlas.
	
	\begin{myIndent}\exTwoP
		Proof:\\
		Let $\mathcal{A}$ and $\mathcal{B}$ be $C^r$ atlases and let $\overline{\mathcal{A}}$ and $\overline{\mathcal{B}}$ denote the smooth structures\\ determined by $\mathcal{A}$ and $\mathcal{B}$ respectively.\retTwo

		$(\Longrightarrow)$\\
		If $\overline{\mathcal{A}} = \overline{\mathcal{B}}$, then we know $\mathcal{B} \subseteq \overline{\mathcal{A}}$. So, every chart of $\mathcal{B}$ is $C^r$ compatible with every chart of $\mathcal{A}$. It follows that any two charts in $\mathcal{A} \cup \mathcal{B}$ are smoothly compatible.\retTwo

		$(\Longleftarrow)$\\
		If $\mathcal{A} \cup \mathcal{B}$ is a $C^r$ atlas, then we know that $\mathcal{A} \cup \mathcal{B} \subseteq \overline{\mathcal{A}}$ and that $\mathcal{A} \cup \mathcal{B} \subseteq \overline{\mathcal{B}}$. It follows that both $\overline{\mathcal{A}}$ and $\overline{\mathcal{B}}$ are the unique smooth structure determined by $\mathcal{A} \cup \mathcal{B}$. So, $\mathcal{A} = \mathcal{B}$. $\blacksquare$\retTwo
	\end{myIndent}
\end{itemize}

\hOne Note that in the prior reasoning, we took $r \in \mathbb{Z}_{>0} \cup \{\infty\}$. If $r = \infty$, we call a $C^r$ manifold a \udefine{smooth manifold}.\retTwo

\pracOne Before doing the following exercise, I want to establish a useful result.\retTwo 

\ul{Proposition:} Let $M$ be a topological manifold. Also let $\mathcal{A}$ be a $C^r$ atlas for $M$ and $\overline{\mathcal{A}}$ be the $C^r$ structure determined by $\mathcal{A}$.
\begin{itemize}
	\item[(a)] If $(U, \varphi)$ is a chart in $\mathcal{A}$, then given any $C^r$ diffeomorphism $h$ acting on $\varphi(U)$, we have that the chart $(U, h \circ \varphi) \in \overline{\mathcal{A}}$.
	
	\begin{myIndent}\pracTwo
		Proof:\\
		By the prior proposition it suffices to show that given another chart $(V, \psi)$ in $\mathcal{A}$ such\\ that $V \cap U \neq \emptyset$, $\psi \circ (h \circ \varphi)^{-1}$ and $(h \circ \varphi) \circ \psi^{-1}$ are $C^r$ maps. Luckily, since $h$, $h^{-1}$, $\varphi \circ \psi^{-1}$, and $\psi \circ \varphi^{-1}$ are all $C^r$ maps, this is obvious. $\blacksquare$\retTwo
	\end{myIndent}

	\item[(b)] If $(U, \varphi)$ is a chart in $\mathcal{A}$, then given any open set $V \subseteq U$, $(V, \varphi|_V)$ is a chart in $\overline{\mathcal{A}}$.
	
	\begin{myIndent}\pracTwo
		Hopefully this is obvious.
	\end{myIndent}
\end{itemize}

The significance of the above lemma is that we can cut up and smoothly reparametrize a coordinate chart and we'll still get a chart which is in the $C^r$ structure we've equipped our manifold with.\retTwo

\ul{Problem 1-6:} Let $M$ be a nonempty topological manifold of dimension $n \geq 1$. If $M$ has a $C^r$ (or smooth) structure, then show that it has uncountably many distinct ones.\newpage
\begin{myIndent}\pracTwo
	Proof:\\
	We start by proving the following lemma\dots

	\mySepThree

	\ul{Lemma:} If $s > 0$, then $F_s(x) \coloneqq \|x\|_2^{s-1} x$ defines a homeomorphism from $B^n$ to itself\\ and from $H^n \cap B^n$ to itself (where $B^n$ is the open unit ball in $\mathbb{R}^n$). Also, $F_s$ is a $C^\infty$\\ diffeomorphism on $\mathbb{R}^n - \{0\}$, and $F_s$ is also a $C^k$ diffeomorphism (for any $k$) on $\mathbb{R}^n$ if\\ and only if $s = 1$.
	\begin{myIndent}
		Proof:\\
		It's clear that $F_s$ is a continuous function on $\mathbb{R}^n$. Also if $x \neq 0$ satisfies that\\ $\|x\|_2^{s-1} x = y$, then $x = \|x\|_2^{1-s}y$ and $\|y\|_2 = \|x\|_2^s$. In turn $\|y\|_2^{(1-s)/s} = \|x\|_2^{1-s}$\\ [2pt] and we have thus derived the following inverse function for $F_s$:

		{\center$F_s^{-1}(x) = \left\{\begin{matrix}
			0 & \text{ if } x = 0\\
			\|x\|_2^{(1-s)/s} x & \text{ if } x \neq 0
		\end{matrix}\right.$ \retTwo\par}

		We claim that $F_s^{-1}$ is continuous. This is clearly true on $\mathbb{R}^n - \{0\}$. Meanwhile, to show that $\|x\|_2^{(1-s)/s} x \to 0$ as $x \to 0$, note that:
		
		{\centering$\left\|\|x\|_2^{(1-s)/s} x\right\|_2 = \|x\|_2^{1 + \frac{1-s}{s}} = \|x\|_2^{1/s}$.\retTwo\par}

		The latter clearly goes to $0$ as $x$ goes to $0$. So $F_s^{-1}$ is continuous at $0$. This proves that $F_s$ is a homeomorphism from $\mathbb{R}^n$ to itself. To show that restricting $F_s$ defines a homeomorphism on $B^n$ or $H^n \cap B^n$ just requires noting that both $F_s$ and $F_s^{-1}$ map $B^n$ into $B^n$ and $H^n$ into $H^n$. Also, since we have formulas for $F_s$ and $F_s^{-1}$, we can now clearly see that $F_s$ and $F_s^{-1}$ are smooth on $\mathbb{R}^n - \{0\}$.\retTwo


		Finally, note that if $s = 1$, then $F_s(x) = x$ is clearly a smooth diffeomorphism. Meanwhile, if $s \neq 1$, then we either have that $s - 1 < 0$ or $(1-s)/s < 0$. In the former case, we know that $F_s$ is not differentiable at $0$. After all, given any $A \in L(\mathbb{R}^n, \mathbb{R}^n)$, we have that:

		{\centering $\frac{\left\| F_s(h) - F_s(0) - Ah \right\|_2}{\|h\|_2} = \frac{\left\| \|h\|^{s-1} h - Ah \right\|_2}{\|h\|_2} = \left\|(\|h\|^{s-1} I - A)\frac{h}{\|h\|_2} \right\|_2$ \retTwo\par}

		It follows then that there is some sequence $(h_n)_{n \in \mathbb{N}}$ in $\mathbb{R}^n$ converging to $0$ such that:

		{\centering $\left\|(\|h_n\|^{s-1} I - A)\frac{h_n}{\|h_n\|_2} \right\|_2 = \| (\|h_n\|^{s-1}I - A)\|_\opnorm$ for all $n$.\retTwo\par}

		\begin{myDindent}\myComment\fontsize{11}{12}\selectfont
			I'll also note that by taking the negative of any neccesary $h_n$, we can force $(h_n)_{n \in \mathbb{N}}$ to be a sequence in $H^n$. I'm not actually sure if this is strictly\\ necessary but who cares.\retTwo
		\end{myDindent}

		But now $\| (\|h_n\|^{s-1}I - A)\|_\opnorm \geq \left| \|h_n\|^{s-1}_\opnorm - \|A\|_\opnorm \right| \to \infty$ as $n \to \infty$ since\\ [-2pt] $\|h_n\|^{s-1} \to \infty$ as $h_n \to 0$. This proves that the derivative of $F_s$ at $0$ doesn't exist\\ [1pt] when $s - 1 < 0$. Analogous reasoning shows that the derivative of $F_s^{-1}$ at $0$ doesn't\\ [1pt] exist when $(1-s)/s < 0$. So, $F_s$ is not a diffeomorphism at $0$ when $s \neq 1$.\retTwo
	\end{myIndent}

	\mySepThree\newpage

	Now let $\mathcal{A}$ be an atlas contained in our $C^r$ structure for $M$. Then choose $p \in M$ and let $(U, \varphi)$ be a chart in $M$ containing $p$. By using the proposition I showed before doing this exercise, we can define another $C^r$ atlas $\mathcal{A}^\prime$ in the same structure on $M$ such that $(U, \varphi)$ is the only chart containing $p$. Specifically, define:

	{\centering $\mathcal{A}^\prime \coloneqq \left\{(V - \{p\}, \psi|_{V - \{p\}}) : (V, \psi) \in \mathcal{A}\right\} \cup \left\{(U, \varphi)\right\}$. \retTwo\par}

	Next, by restricting $\varphi$ to a coordinate ball or half ball $W$ centered at $p$ and then\\ reparametrizing, we can say that there is a chart $(W, \varphi^\prime)$ in our structure on $M$ satisfying that $p \in W$, $\varphi^\prime(p) = 0$, and $\varphi^\prime(W)$ equals either $B^n$ or $H^n \cap B^n$. Hence, we define the $C^r$ atlas:

	{\centering $\mathcal{A}^\pprime \coloneqq \left\{(V - \{p\}, \psi|_{V - \{p\}}) : (V, \psi) \in \mathcal{A}\right\} \cup \left\{(W, \varphi^\prime)\right\}$ \retTwo\par}

	And now we're in a position to use our lemma. Given $s > 0$, let $F_s$ be as in our lemma and define the atlas:
	
	{\centering $\mathcalli{B}_s \coloneqq \left\{(V - \{p\}, \psi|_{V - \{p\}}) : (V, \psi) \in \mathcal{A}\right\} \cup \left\{(W, F_s \circ \varphi^\prime)\right\}$\retTwo\par}

	Note that $\mathcalli{B}_s$ is in fact an atlas for every $s$ since $F_s$ is a homeomorphism, meaning that $(W, F_s \circ \varphi^\prime)$ is a well-defined chart in $M$. Also, we can see that $\mathcalli{B}_s$ is actually a $C^r$ atlas. After all, we know from before that every pair of charts in $\mathcal{B}_s$ not including $(W, F_s \circ \varphi^\prime)$ are $C^r$ compatible. Meanwhile, note that if $(V, \psi) \in \mathcal{A}$ satisfies that $(V - \{p\}) \cap W \neq \emptyset$, then $F_s$ is a diffeomorphism on the set $\varphi^\prime(V \cap W - \{p\})$ on account of $0 = \varphi^\prime(p)$ not being in that set. It follows easily that $(F_s \circ \varphi^\prime) \circ \psi^{-1}$ defined on $\psi(V \cap W - \{p\})$ is a $C^r$ diffeomorphism.\retTwo

	But now note that if $s \neq t$, then $\mathcal{B}_s$ and $\mathcal{B}_t$ do not generate the same $C^r$ structure on $M$. After all, the charts $(W, F_s \circ \varphi^\prime)$ and $(W, F_t \circ \varphi^\prime)$ are not $C^r$ compatible unless $s = t$. This is because when $x \neq 0$,
	
	{\centering\begin{tabular}{l}
		$((F_s \circ \varphi^\prime) \circ (F_t \circ \varphi^\prime)^{-1})(x) = (F_s \circ \varphi^\prime \circ (\varphi^\prime)^{-1} \circ F_t^{-1})(x)$\\ [6pt]

		$\phantom{((F_s \circ \varphi^\prime) \circ (F_t \circ \varphi^\prime)^{-1})(x)} = (F_s \circ F_t^{-1})(x)$\\ [6pt]

		$\phantom{((F_s \circ \varphi^\prime) \circ (F_t \circ \varphi^\prime)^{-1})(x)} = F_s(\|x\|_2^{(1-t)/t}x)$\\ [6pt]

		$\phantom{((F_s \circ \varphi^\prime) \circ (F_t \circ \varphi^\prime)^{-1})(x)} = \| (\|x\|_2^{(1-t)/t})x \|_2^{s-1} \cdot \|x\|_2^{(1-t)/t}x$\\ [6pt]

		$\phantom{((F_s \circ \varphi^\prime) \circ (F_t \circ \varphi^\prime)^{-1})(x)} = \|x\|_2^{\frac{(1-t)s}{t}}\|x\|_2^{s-1}x = \|x\|_2^{\frac{s}{t} - 1}x = F_{s/t}(x)$.\\ [6pt]
	\end{tabular}\retTwo\par}

	Also, you can manually check that the transition map also equals $F_{s/t}(0)$ at $x = 0$. And since $F_{s/t}$ is a diffeomorphism of any class iff $s/t = 1$, we know that the two charts are $C^r$ compatible if and only if $s = t$. $\blacksquare$\retTwo
\end{myIndent}

\hOne In a sense this exercise proves how important it is to keep in mind that we consider a manifold to be smooth with respect to a specific structure. That said, if we're not working with multiple different structures, then it's annoying to explicitely mention the structure over and over. So, we take the approach of calling a chart a \udefine{smooth chart} if it's in our structure.\retTwo

To finish off today, I want to briefly address the boundary of a manifold.\newpage

\begin{myIndent}\hTwo
	Like before if $M$ is a topological manifold, we say a point $p \in M$ is an \udefine{interior point} of $M$ if there exists a chart $(U, \varphi)$ covering $p$ such that $\varphi(U)$ is open in $\mathbb{R}^n$. Meanwhile, if no such chart exists, we say $p$ is a \udefine{boundary point}. Also, we denote the \udefine{interior} of $M$: $\Interior{M}$, to be to be collection of interior points and the \udefine{boundary} of $M$: $\partial M \coloneqq M - \Interior{M}$.\retTwo

	It is easy to see that $\Interior{M}$ is an open subset of $M$ and a manifold by itself without a boundary. Based on that we can also easily see that $\partial M$ is a closed subset of $M$.\retTwo
\end{myIndent}

Oh, I also forgot to mention: if $U \subseteq M$ is open and $M$ is a $C^r$ manifold, then we can view $U \subseteq M$ as a $C^r$ \udefine{submanifold} of $M$. Specifically, it's clear that $U$ is a second countable Hausdorff space in the relative topology. Also, given any smooth chart $(V, \varphi)$ on $M$, we define a smooth chart $(U \cap V, \varphi|_{U \cap V})$ on $U$. This gives us a $C^r$ structure on $U$.\retTwo

\dispDate{8/24/2025}

Let $r, k \in \mathbb{Z}_{>0} \cup \{\infty\}$ and always assume $k \leq r$.\retTwo

Suppose $M$ is a $C^r$ manifold and $f: M \to \mathbb{R}^m$ is a function. We say $f$ is a \udefine{$C^k$\\ function} if for every $p \in M$ there exists a smooth chart $(U, \varphi)$ such that $p \in U$\\ and $f \circ \varphi^{-1}$ is $C^k$ on the open set $\varphi(U)$ of either $H^n$ or $\mathbb{R}^n$. In this case we denote $f \in C^k(M, \mathbb{R}^m)$ (although when $m = 1$ we usually shorthand this as $f \in C^k(M)$).\retTwo

\pracOne\ul{Exercise 2.3:} Let $M$ be a $C^r$ manifold and suppose $f: M \to \mathbb{R}^k$ is a $C^k$ function where $k \leq r$. Show that $f \circ \psi^{-1}: \varphi(U) \to \mathbb{R}^m$ is $C^k$ for every smooth chart $(U, \psi)$ on $M$.

\begin{myIndent}\pracTwo
	Proof:\\
	Given any smooth chart $(U, \psi)$, we can show that $f \circ \psi^{-1}$ is locally $C^k$ as follows. Take any $x = \psi(p) \in \psi(U)$. Now we know there exists another smooth chart $(V, \varphi)$  satisfying that $p \in V$ and that $f \circ \varphi^{-1}$ is $C^k$. Also, $\varphi \circ \psi^{-1}$ is a $C^r$ map on $\psi(U \cap V)$. Thus, $f \circ \psi^{-1} = (f \circ \varphi^{-1}) \circ (\varphi \circ \psi^{-1})$ is a $C^k$ map on the open neighborhood $\psi(U \cap V)$ of $x$ in $H^n$ or $\mathbb{R}^n$. $\blacksquare$\retTwo

	\begin{myIndent}\myComment
		Side note: A recurring theme will be that you need $M$ to be at to be at least $C^k$ in order for $C^k$ functions on $M$ to be well-behaved. After all, if $r < k$ then the composite function above is no longer guarenteed to be $C^k$.\retTwo
	\end{myIndent}
\end{myIndent}

\ul{Corollary / Exercise 2.1:} $C^k(M, \mathbb{R}^m)$ is a real vector space. If $m = 1$, then $C^k(M)$ is a a real commutative algebra.\retTwo

\hOne We want to generalize the previous definition even more to cover maps from\\ manifolds to other manifolds. To do this, it's worth noting that any open set $U$ of either $H^n$ or $\mathbb{R}^n$ can be thought of as a manifold. Furthermore, $U$ is a $C^r$  manifold\\ for any $r$ when equipped with the \udefine{standard structure}, i.e. the one determined\\ by the atlas $\{(U, \myId)\}$.

\begin{myIndent}\hThree
	(Unless specified otherwise, always assume an open subset of $\mathbb{R}^n$ or $H^n$ is equipped with the standard structure\dots)\newpage
\end{myIndent}

\pracOne\ul{Exercise 2.2:} Let $U$ be an open subset of $H^n$ or $\mathbb{R}^n$. Then a function $f: U \to \mathbb{R}^m$ is $C^k$ in the traditional real analysis definition iff it is $C^k$ with respect to our new definition.

\begin{myIndent}\pracTwo
	$(\Longrightarrow)$\\
	Suppose $(V, \varphi)$ is any smooth chart on $U$. Then $\varphi^{-1} = \myId \circ \varphi^{-1}$ is a $C^r$ function. So, $f \circ \varphi^{-1}$ is $C^k$.\retTwo

	$(\Longleftarrow)$\\
	It must be the case that $f \circ \myId^{-1} = f$ is $C^k$ in the traditional real analysis sense since $(U, \myId)$ is a smooth chart. $\blacksquare$\retTwo
\end{myIndent}

\hOne Now note that when viewing $\mathbb{R}^m$ as being a $C^r$ manifold, we can "symmetrize" our definition by noting that a function $f: M \to \mathbb{R}^m$ is $C^k$ if and only if for all $p \in M$ there exists a smooth chart $(U, \varphi)$ on $M$ and another smooth chart $(V, \psi)$ on $\mathbb{R}^m$ with $f(U) \subseteq V$ such that $\psi \circ f \circ \varphi^{-1}$ is $C^k$ from $\varphi(U)$ into $\psi(V)$.
\begin{myIndent}\pracTwo
	$(\Longrightarrow)$\\
	Since $f$ is $C^k$, let $(U, \varphi)$ be a smooth chart such that $f \circ \varphi^{-1}$ is $C^k$. Then let $(V, \psi)$ be any smooth chart on $\mathbb{R}^m$ with $f(U) \subseteq V$. Note that such a chart must exist since we know $(\mathbb{R}^m, \myId)$ works. Now $\psi \circ \myId^{-1} = \psi$ is a $C^r$ map from $V$. So, $\psi \circ f \circ \varphi^{-1}$ is a $C^k$ from $\varphi(U)$ into $\psi(V)$.\retTwo

	$(\Longleftarrow)$\\
	This direction is obvious when you just take $(V, \psi)$ to be the chart $(\mathbb{R}^m, \myId)$. $\blacksquare$\retTwo
\end{myIndent}

This points us to following generalization of differentiability on manifolds. Suppose $M$ and $N$ are both $C^r$ manifolds and let $F: M \to N$ be any map. We say $F$ is a \udefine{$C^k$ map} if for every $p \in M$ there exists a smooth chart $(U, \varphi)$ on $M$ with $p \in U$ and another smooth chart $(V, \psi)$ on $N$ with $F(U) \subseteq V$ satisfying that the composite map $\psi \circ F \circ \varphi^{-1}$ is a $C^k$ map from $\varphi(U)$ into $\psi(V)$.\retTwo

\exOne\ul{Proposition 2.4:} Every $C^k$ map between two $C^r$ manifolds $M$ and $N$ is continuous.
\begin{myIndent}\exTwoP
	Proof:\\
	Suppose $F: M \to N$ is $C^k$. Then given any $p \in M$, we can show that $F$ is continuous on a neighborhood $U$ of $p$. Specifically, let $(U, \varphi)$ and $(V, \psi)$ be smooth charts as in the prior definition. Then $\psi \circ F \circ \varphi^{-1}$ is continuous on the set $\varphi(U)$ on account of it being a differentiable function between two subsets of $\mathbb{R}^n$. Also, since both $\psi$ and $\varphi$ are homeomorphisms, we have that $F = \psi^{-1} \circ (\psi \circ F \circ \varphi^{-1}) \circ \varphi$ is a continuous map from $U \subseteq M$ into $N$.\retTwo

	Since each $p \in M$ has a neighborhood on which $F$ is continuous, it follows that $F$ is a continuous map from $M$ to $N$. $\blacksquare$\retTwo

	\begin{myIndent}\exPPP
		Shit I just realized that I've never actually proven that continuity is local like that. So here's a quick lemma\dots\retTwo

		\ul{Lemma:} Let $f: X \to Y$ be a map and suppose that every $x \in X$ has a\\ neighborhood $N_x$ such that $f|_{N_x}$ is continuous. Then $f$ is continuous.\newpage

		\begin{myIndent}
			Proof:\\
			For any $x \in X$, let $N_x$ be a neighborhood satisfying that $f|_{N_x}$ is continuous. Then given any neighborhood $V$ of $f(x)$ in $Y$, we know $U \coloneqq f^{-1}(V) \cap N_x$ must be a neighborhood of $x$. satisfying that $f(U) \subseteq V$. So, $f$ (not restricted to any subset) is continuous at $x$.\retTwo

			Since $f$ is continuous at all $x \in X$, we know that $f$ is continuous on $X$. $\blacksquare$\retTwo
		\end{myIndent}
	\end{myIndent}
\end{myIndent}

\pracOne As a side note, when we were defining what it means for a map between manifolds,\\ $F: M \to N$, to be differentiable, perhaps it felt overly restricting for us to force the\\ chart $(V, \psi)$ on $N$ to satisfy that $F(U) \subseteq V$ in our definition. However, it turns out that without that requirement, it is no longer the case that $F$ being differentiable implies that $F$ is continuous.\retTwo

\ul{Problem 2-1:} Define $f: \mathbb{R} \to \mathbb{R}$ by $f(x) = \left\{\begin{matrix}
	1 & \text{ if } x \geq 0 \\ 0 & \text{ if } x < 0
\end{matrix}\right.$.\\

Now for every $x \in \mathbb{R}$ there are smooth coordinate charts $(U, \varphi)$ containing $x$ and $(V, \psi)$ containing $f(x)$ such that $\psi \circ f \circ \varphi^{-1}$ is smooth as a map from $\varphi(U \cap f^{-1}(V))$ to $\psi(V)$. However $f$ is clearly not continuous, nor smooth according to according to definition of smoothness of maps between two manifolds.
\begin{myIndent}\pracTwo
	If $x \neq 0$, we can just pick $(U, \varphi) = (\mathbb{R} - \{0\}, \myId)$ and $(V, \psi) = (\mathbb{R}, \myId)$. Then it's clear that $f(x) \in V$ and $\psi \circ f \circ \varphi^{-1} = f$ is smooth as a map from $\varphi(U \cap f^{-1}(V)) = \mathbb{R} - \{0\}$ to $\mathbb{R}$.\retTwo

	Meanwhile, if $x = 0$, then pick $(U, \varphi) = (\mathbb{R}, \myId)$ and $(V, \psi) = ((0, \infty), \myId)$. Then it is still the case that $f(x) \in V$. Also, $\psi \circ f \circ \varphi^{-1} = f$ is just the constant function $1$ on the set $\varphi(U \cap f^{-1}(V)) = [0, \infty)$. Thus since it can be extended to a differentiable function on an open set containing $[0, \infty)$, we can say that $\psi \circ f \circ \varphi^{-1}$ is also a smooth map from $\varphi(U \cap f^{-1}(V))$.\retTwo

	However, $f$ is not differentiable or even continuous in the traditional analysis sense. Thus, $f$ cannot be a smooth map from $\mathbb{R}$ as a manifold into $\mathbb{R}$ by exercise 2.2. Then in turn, we know from our prior efforts in generalizing differentiability that $f$ is not smooth as a map into $\mathbb{R}$ as a manifold either. $\blacksquare$\retTwo
\end{myIndent}

\exOne\ul{Proposition 2.5:} Suppose $M$ and $N$ are $C^r$ manifolds, and $F: M \to N$ is a map. Then the following are equivalent.

\begin{itemize}
	\item[(a)] $F$ is a $C^k$ map.
	\item[(b)] For every $p \in M$ there exist smooth charts $(U, \varphi)$ containing $p$ and $(V, \psi)$\\ containing $F(p)$ such that $U \cap F^{-1}(V)$ is open in $M$ and the map $\psi \circ F \circ \varphi^{-1}$ is $C^k$ from $\varphi(U \cap F^{-1}(V))$ into $\psi(V)$.
	\item[(c)] $F$ is continuous and there exists atlases $\{(U_\alpha, \varphi_\alpha)\}_{\alpha \in A}$ of $M$ and $\{(V_\beta, \psi_\beta)\}_{\beta \in B}$ of $N$ consisting of smooth charts such that for each $\alpha$ and $\beta$:
	
	{\centering$\psi_\beta \circ F \circ \varphi_\alpha^{-1}$ is a $C^k$ map from $\varphi_\alpha(U_\alpha \cap F^{-1}(V_\beta))$ into $\psi_\beta(V_\beta)$.\newpage\par}
\end{itemize}

\begin{myIndent}\exTwoP
	$(a \Longrightarrow c)$\\
	From the last proposition we know that $F$ is continuous. Also, suppose $(U, \varphi)$ and $(V, \psi)$ are \textit{any} smooth charts on $M$ and $N$ respectively such that $U \cap F^{-1}(V) \neq \emptyset$. Then we claim that $\psi \circ F \circ \varphi^{-1}$ is a $C^k$ map from $\varphi(U \cap F^{-1}(V))$ into $\psi(V)$.
	\begin{myIndent}\exPPP
		Proof:\\
		Let $x = \varphi(p)$ be in $\varphi(U \cap F^{-1}(V))$. Then since $F$ is $C^k$, we know there are\\ smooth charts $(W_m, \theta_m)$ in $M$ and $(W_n, \theta_n)$ in $N$ such that $p \in W_m$,\\ $F(W_m) \subseteq W_n$, and $\theta_n \circ F \circ \theta_m^{-1}$ is a $C^k$ map from $\theta_m(W_m)$ into $\theta_n(W_n)$. Also,\\ we have that $\psi \circ \theta_n^{-1}$ is a $C^r$ map from $\theta_n(V \cap W_n)$ to $\psi(V \cap W_n)$. And\\ similarly, we have that $\theta_m \circ \varphi^{-1}$ is a $C^r$ map from $\varphi(U \cap W_m)$ to $\theta_m(U \cap W_m)$.\retTwo

		Now we get to composing the functions established above (and I'll do this slowly since my head is already spinning from all the symbols written above).
		\begin{itemize}
			\item $\hspace{-0.5em}$\begin{tabular}{l}
					$(\psi \circ \theta_n^{-1}) \circ (\theta_n \circ F \circ \theta_m^{-1})$ is a $C^k$ map from $\theta_m(W_m \cap F^{-1}(V \cap W_n))$ into $\psi(V)$.
				\end{tabular}\\ [-6pt]

			\item $\hspace{-0.5em}$\begin{tabular}{l}
					Because $F^{-1}(W_n) \supseteq W_m$, we have that $W_m \cap F^{-1}(V \cap W_n) = W_m \cap F^{-1}(V)$.
				\end{tabular}\\ [-9pt]
			
			\item Thus $(\psi \circ \theta_n^{-1}) \circ (\theta_n \circ F \circ \theta_m^{-1}) \circ (\theta_m \circ \varphi^{-1})$ is a $C^k$ map from\\ $\phantom{aaaaaaaaaaaaaaaaaaaaaaaaa} \varphi(U \cap W_m \cap F^{-1}(V))$ into $\psi(V)$.\\ [-9pt]
			
			\item Also $\psi \circ F \circ \varphi^{-1} = (\psi \circ \theta_n^{-1}) \circ (\theta_n \circ F \circ \theta_m^{-1}) \circ (\theta_m \circ \varphi^{-1})$ and we know $x \in \varphi(U \cap W_m \cap F^{-1}(V))$.\\ [-9pt]
			
			\item Since $F$ is continuous, we know that $F^{-1}(V)$ is open. It follows that\\ $\varphi(U \cap W_m \cap F^{-1}(V))$ is an open neighborhood of $x$ in either $H^n$ or $\mathbb{R}^n$.\\ [-9pt]
			
			\item This shows that any $x \in \varphi(U \cap F^{-1}(V))$ has an open neighborhood in either $H^n$ or $\mathbb{R}^n$ for which $\psi \circ F \circ \varphi^{-1}$ is a $C^k$ map when restricted to that\\ neighborhood. It  follows that $\psi \circ F \circ \varphi^{-1}$ is $C^k$ on $\varphi(U \cap F^{-1}(V))$.\retTwo
		\end{itemize}

		\myComment Side note: I essentially just proved an analog of exercise 2.3 a few pages ago.\\ So that I can cite it later, I'll write it out as follows\ldots\retTwo
		
		\ul{Proposition} If $M$ and $N$ are $C^r$ manifolds and $F: M \to N$ is a $C^k$ map, then\\ $\psi \circ F \circ \varphi^{-1}$ is a $C^k$ map on $\varphi(U \cap F^{-1}(V))$ for all smooth charts $(U, \varphi)$ on $M$\\ and $(V, \psi)$ on $N$.\retTwo
	\end{myIndent}

	Based on the prior reasoning, it suffices to choose any covering of $M$ and $N$ of smooth charts and we are done showing (c).\retTwo

	$(c \Longrightarrow b)$\\
	Let $p \in M$ and let $(U_\alpha, \varphi_\alpha)$ be a smooth chart on $M$ covering $p$. Next let $(V_\beta, \psi_\beta)$ be a smooth chart on $N$ covering $F(p)$. Since $F$ is continuous, we know that\\ $U_\alpha \cap F^{-1}(V_\beta)$ is open. Also, we know by assumption that $\psi_\beta \circ F \circ \varphi_\alpha^{-1}$ is $C^k$ from\\ $\varphi_\alpha(U_\alpha \cap F^{-1}(V_\beta))$ into $\psi_\beta(V_\beta)$. This proves (b).\retTwo

	$(b \Longrightarrow a)$\\
	Given any $p \in M$ let $(U, \varphi)$ and $(V, \psi)$ be as in the hypothesis of (b). Then since\\ $U \cap F^{-1}(V)$ is open and contains $p$, we have that:
	
	{\centering$(U^\prime, \varphi^\prime) \coloneqq (U \cap F^{-1}(V), \varphi|_{U \cap F^{-1}(V)})$ is another chart on $M$ containing $p$.\newpage\par}

	Importantly, $\psi \circ F \circ (\varphi^\prime)^{-1}$ will still be a $C^k$ map from $\varphi^\prime(U^\prime)$ into $\psi(V)$ since\\ $\psi \circ F \circ \varphi^{-1}$ also is that. But, we also have $F(U^\prime) \subseteq V$. This proves (a). $\blacksquare$\retTwo
\end{myIndent}

\exOne\ul{Proposition 2.6:} Let $M$ and $N$ be $C^r$ manifolds, and let $F: M \to N$ be a map.
\begin{itemize}
	\item If every point $p \in M$ has an open neighborhood $U$ such that the restriction $F|_U$ is $C^k$, then $F$ is $C^k$ globaly. 
	\item If $F$ is $C^k$ globaly, then it's restriction to every open subset $U \subseteq M$ is $C^k$.
\end{itemize}

\begin{myIndent}\exTwoP
	Proof:\\
	Hopefully it is obvious that the latter bullet point is just a corollary of the prosition I noted on the last page. Meanwhile, the first bullet point is proved just by noting that any smooth chart in $U$ is also a smooth chart in $M$. $\blacksquare$\retTwo
\end{myIndent}

\ul{Corollary 2.8: (Gluing lemma)} Let $M$ and $N$ be $C^r$ manifolds and let $\{U_\alpha\}_{\alpha \in A}$\\ be an open cover for $M$. Suppose that for each $\alpha \in A$ we are given a $C^k$ map\\ $F_\alpha: U_\alpha \to N$ and suppose that $F_\alpha|_{U_\alpha \cap U_\beta} = F_\beta|_{U_\alpha \cap U_\beta}$ for all $\alpha$ and $\beta$. Then there exists a unique $C^k$ map $F: M \to N$ such that $F|_{U_\alpha} = F_\alpha$ for all $\alpha$.

\begin{myIndent}\exTwoP
	Proof:\\
	Define $F = \bigcup_{\alpha \in A}F_\alpha$. This a well defined function on $M$ since the $U_\alpha$ cover all of $M$ and for any $p \in M$ there is only one $q \in N$ such that $(p, q) \in F$. Also, it is clear that $F$ is the unique map satisying that $F|_{U_\alpha} = F_\alpha$ for all $\alpha \in A$. And since all the $F_\alpha$ are $C^k$, we know that $F$ is locally $C^k$. So by the last proposition, we know that $F$ is a $C^k$ map globally. $\blacksquare$\retTwo
\end{myIndent}

\ul{Proposition 2.10:} Let $M$, $N$, and $P$ be $C^r$ manifolds.
\begin{itemize}
	\item[(a)] Every constant map $c: M \to M$ is $C^r$.
	
	\begin{myIndent}\exTwoP
		Proof:\\
		Suppose $c(p) = q$ for all $p \in M$. Then let $(V, \psi)$ be a smooth chart covering $q$. If $(U, \varphi)$ is any smooth chart on $M$, we know that $c(U) = \{q\} \subseteq V$ and that $\psi \circ c \circ \varphi^{-1} = \psi(q)$ is a constant function on $\varphi(U)$. Therefore $\psi \circ c \circ \varphi^{-1}$ is $C^r$ and we've proven that $c$ is a $C^r$ map. $\blacksquare$
	\end{myIndent}

	\item[(b)] The identity map on $M$ is $C^r$.
	
	\begin{myIndent}\exTwoP
		Proof:\\
		Let $(U, \varphi)$ be any chart on $M$. Then $\myId_M(U) \subseteq U$ and $\varphi \circ \myId_M \circ \varphi^{-1} = \myId_{\mathbb{R}^n}$\\ is $C^r$. This proves that $\myId_M$ is $C^r$.$\blacksquare$.
	\end{myIndent}

	\item[(c)] If $U \subseteq M$ is an open submanifold, then the inclusion map $U \hookrightarrow M$ is $C^r$.
	
	\begin{myIndent}\exTwoP
		Proof:\\
		Just apply proposition 2.6 to the identity map on $M$. $\blacksquare$
	\end{myIndent}

	\item[(d)] If $F: M \to N$ and $G: N \to P$ are $C^k$, then so is $G \circ F: M \to P$.\newpage
	
	\begin{myIndent}\exTwoP
		Proof:\\
		Let $p \in M$. Then by definition there are smooth charts $(V, \theta)$ on $N$ and $(W, \psi)$ on $P$ such that $F(p) \in V$, $G(V) \subseteq W$, and $\psi \circ G \circ \theta^{-1}$ is a $C^k$ map on $\theta(V)$. Also, since $F$ is continuous, we know that $F^{-1}(V)$ is an open set in $M$. Therefore, we can find a smooth chart $(U, \varphi)$ such that $p \in U \subseteq F^{-1}(V)$. In turn, $(G \circ F)(U) \subseteq W$. And since $F$ is $C^k$, we know that $\theta \circ F \circ \varphi^{-1}$ is a $C^k$ map from $\varphi(U)$. Hence:

		{\centering $\psi \circ (G \circ F) \circ \varphi^{-1} = (\psi \circ G \circ \theta^{-1}) \circ (\theta \circ F \circ \varphi^{-1})$ is a $C^k$ map from $\varphi(U)$ \retTwo\par}

		This proves that $G \circ F$ is a $C^k$ map. $\blacksquare$\retTwo
	\end{myIndent}
\end{itemize}

\hOne\mySepTwo

\dispDate{8/25/2025}

Today I'm going to jump back to Guillemin's \ul{Differential Forms}. My reasoning for this is that I want a working fomulation of Stokes theorem before the end of the Summer, and at the rate I'm going through Lee's book, I'm not going to get that formulation from Lee. I also never quite finished the chapter on tensors before. I will continue to return to Lee off and on though.\retTwo

\hTwo\blab{The pullback operation on $\Lambda^k(V^*)$:}\\
Let $V$ be an $n$-dimensional spaces over a field $F$ with characteristic $0$, and let $W$ be an $m$-dimensional vector space over $F$. Also let $A: V \to W$ be a linear map. Recall that for any $T \in \mathcal{L}^k(W)$ we defined $A^\dag T(v_1, \ldots, v_k) = T(Av_1, \ldots, Av_k)$.\retTwo

\exTwo\ul{Lemma 1.8.1:} If $T \in \mathcal{I}^k(W)$, then $A^\dag T \in \mathcal{I}^k(V)$.

\begin{myIndent}\exThreeP
	Proof:\\
	Since $T$ can be expressed as a linear combination of redundant $k$-tensors and $A^{\dag}$ is a linear map from $\mathcal{L}^k(W)$ to $\mathcal{L}^k(V)$, it suffices to assume $T$ is itself a redundant $k$-tensor. So let $T = \ell_1 \otimes \cdots \otimes \ell_k$ where each $\ell_j \in W^*$ and $\ell_i = \ell_{i+1}$ for some $i$. Then by proposition 1.3.18 (on page 103) we have that $A^\dag T = (A^\dag \ell_1) \otimes \cdots \otimes (A^\dag \ell_k)$. It follows that $A^\dag T$ is a redundant $k$-tensor. $\blacksquare$.\retTwo
\end{myIndent}

\hTwo Let $\pi_W$ and $\pi_V$ be the projections of $\mathcal{L}^k(W)$ and $\mathcal{L}^k(V)$ onto $\Lambda^k(W^*)$ and $\Lambda^k(V^*)$\\ respectively.\retTwo

If $\omega \in \Lambda^k(W^*)$ and $T \in \mathcal{L}^k(W)$ satisfies that $\pi_W(T) = \omega$, then we define:

{\centering$A^\dag \omega \coloneqq \pi_V(A^\dag T)$.\retTwo\par}

\begin{myIndent}\pracTwo
	To see that this is well defined, suppose $\omega = \pi(T) = \pi(T^\prime)$. Then $T = T^\prime + S$ for some $S \in \mathcal{I}^k(V)$. So, $A^\dag T  = A^\dag T^\prime  + A^\dag S$. And since $A^\dag S  \in \mathcal{I}^k(V)$ by the last lemma, we have that $\pi(A^\dag T) = \pi(A^\dag T^\prime)$.\newpage
\end{myIndent}

\exTwo\ul{Proposition 1.8.4:} The map $A^\dag : \Lambda^k(W^*) \to \Lambda^k(V^*)$ sending $\omega$ to $A^\dag \omega$ is linear. Moreover:
\begin{itemize}
	\item if $\omega_i \in \Lambda^{k_i}(W^*)$ for $i = 1, 2$, then $A^\dag(\omega_1 \wedge \omega_2) = (A^\dag \omega_1) \wedge (A^\dag \omega_2)$;
	
	\item if $U$ is a vector space and $B: U \to V$ is a linear map, then for $\omega \in \Lambda^k(W^*)$, $B^\dag (A^\dag \omega) = (AB)^\dag \omega$.
\end{itemize}

\begin{myIndent}\exThreeP
	Proof:\\
	Firstly, let $\omega, \omega^\prime \in \Lambda^k(W^*)$. Then if $T \in \pi_W^{-1}(\{\omega\})$ and $T^\prime \in \pi_W^{-1}(\{\omega\})$, we have for any $\lambda, \lambda^\prime \in F$ that $\pi_W(\lambda T + \lambda^\prime T^\prime) = \lambda \omega + \lambda^\prime \omega^\prime$. And now, showing that $A^\dag$ as a map from $\Lambda^k(W^*)$ is linear is as simple as noting that $\pi_V \circ A^\dag$ is linear (where we view $A^\dag$ as a map from $\mathcal{L}^k(W)$).\retTwo

	Next, let $\omega_1, \omega_2$ be as in the proposition statement. Then suppose $T_1$ and $T_2$ both satisfy that $\pi_W(T_1) = \omega_1$ and $\pi_W(T_2) = \omega_2$ (ignore my abuse of notation). Then:
	
	{\centering\begin{tabular}{l}
		$A^\dag(\omega_1 \wedge \omega_2) = \pi_V(A^\dag(T_1 \otimes T_2))$\\
		$\phantom{A^\dag(\omega_1 \wedge \omega_2)} = \pi_V((A^\dag T_1) \otimes (A^\dag T_2))$\\
		$\phantom{A^\dag(\omega_1 \wedge \omega_2)} = \pi_V((A^\dag T_1)) \wedge \pi_V((A^\dag T_2)) = (A^\dag \omega_1) \wedge (A^\dag \omega_2)$.
	\end{tabular}\retTwo\par}

	Finally, let $\omega \in \Lambda^k(W^*)$ and choose $T \in \mathcal{L}^k(W)$ such that $\pi_W(T) = \omega$. Then if you squint you can see that:
	
	{\centering$B^\dag(A^\dag \omega) = \pi_V(B^\dag(A^\dag T)) = \pi_V((AB)^\dag T) = (AB)^\dag \omega$.\retTwo\par}
\end{myIndent}

\hTwo One application of the pullback operation is that it gives us a way of defining determinants completely independently of any chosen basis. Specifically, let $V$ be an $n$-dimensional vector space over a field $F$ with characteristic $0$, and suppose $A: V \to V$ is a linear map.

\begin{myIndent}\myComment
	(If you want to be pedantic, everything that follows should work so long as $F$ has a\\ characteristic that makes it so that $n! \neq 0$ and $-1 \neq 1$\dots)\retTwo
\end{myIndent}

Since $\dim \Lambda^n(V^*) = \binom{n}{n} = 1$ and $A^\dag: \Lambda^n(V^*) \to \Lambda^n(V^*)$ is linear, it must be that\\ [-2pt] the map $A^\dag$ is just multiplication by a constant. We denote this constant $\det(A)$ and call it the \udefine{determinant} of $A$. In other words, we define $\det(A) \in F$ to be the constant such that $A^\dag \omega = \det(A)\omega$ for all $\omega \in \Lambda^n(V^*)$.\retTwo

\exTwo\ul{Proposition 1.8.7:} If $A$ and $B$ are linear mappings of $V$ into $V$, then $\det(AB) = \det(A)\det(B)$.

\begin{myIndent}\exThreeP
	Proof:\\
	Given any $\omega \in \Lambda^n(V^*)$:
	
	{\centering $\det(AB)\omega = (AB)^\dag \omega = B^\dag(A^\dag \omega) = \det(B) A^\dag \omega = \det(B)\det(A)\omega$\retTwo\par}

	It follows that $\det(A)\det(B) = \det(AB)$. $\blacksquare$\retTwo
\end{myIndent}

\ul{Proposition 1.8.8:} Write $\myId_V: V \to V$ for the identity map. Then $\det(\myId_V) = 1$.

\begin{myIndent}\exThreeP
	Proof:\\
	Note that if $\omega \in \Lambda^k(V^*)$ for any $k$ and $T \in \mathcal{L}^k(V)$ satisfies that $\pi_V(T) = \omega$, then $\myId^\dag \omega = \pi_V(\myId^\dag T) = \pi_V(T) = \omega$. This shows that for any $k$, $\myId^\dag$ is the identity map on $\Lambda^k(V^*)$. Hence $\det(\myId) = 1$. $\blacksquare$\newpage
\end{myIndent}

\ul{Corollary:} If $A: V \to V$ is a surjective linear map, then $\det(A) \neq 0$. 

\begin{myIndent}\exThreeP
	Proof:\\
	If $A$ is surjective, then we know by the rank-nullity theorem that $A$ has nullity $0$. So, $A$ is bijective and has an inverse $A^{-1}$. Then by our last two propositions, we know that $\det(A)\det(A^{-1}) = \det(\myId_V) = 1$. Thus, it cannot be that $\det(A) = 0$. $\blacksquare$\retTwo
\end{myIndent}

\ul{Proposition 1.8.9:} If $A: V \to V$ is not surjective, then $\det(A) = 0$.

\begin{myIndent}\exThreeP
	Proof:\\
	Let $W$ be the image of $A$. If $A$ is not surjective, we know that $\dim W < n$ and thus\\ $\Lambda^n(W^*) = \{0\}$. So, let $A = i_W B$ where $i_W$ is the inclusion map $W \hookrightarrow V$ and let $B$\\ be the map $A$ with it's codomain restricted to $W$. Then by proposition 1.8.4 we have that\\ $A^\dag \omega = B^\dag(i_W^\dag \omega)$. But now note $i_W^\dag \omega \in \Lambda^n(W^*) = \{0\}$. This means that $i_W^\dag \omega = 0$ and we trivially have that $B^\dag(i_W^\dag) = 0$. It follows that $\det(A) = 0$. $\blacksquare$\retTwo
\end{myIndent}

\hTwo We still need to show that this definition of the determinant agrees with the usual one. To do this we can first prove something slightly more general.\retTwo

Suppose $A: V \to W$ is a linear map, and let $e_1, \ldots, e_n$ and $f_1, \ldots, f_n$ be bases of $V$ and $W$ respectively. Then let $e_1^*, \ldots, e_n^*$ and $f_1^*, \ldots, f_n^*$ be the corresponding dual bases. If $(a_{i,j})$ is the $n \times n$ matrix representing $A$ with respect to our bases (i.e. $A e_j = \sum_{i=1}^n a_{i,j}f_i$ for all $j$), then:

{\centering $A^\dag f_j^* = \sum_{i=1}^n a_{j,i}e_{i}^*$ for all $j$ (see claim 1.2.15 on page 102\dots).\retTwo\par}

In turn:\\ [-9pt]

{\centering\begin{tabular}{l}
	$A^\dag(f_1^* \wedge \cdots \wedge f_n^*) = (A^\dag f_1^*) \wedge \cdots \wedge (A^\dag f_n^*)$\\ [4pt]
	$\phantom{A^\dag(f_1^* \wedge \cdots \wedge f_n^*)} = (\sum\limits_{i=1}^n a_{1,i}e_i^*) \wedge \cdots \wedge (\sum\limits_{i=1}^n a_{n,i}e_i^*) = \hspace{-1.5em}\sum\limits_{1 \leq k_1, \ldots, k_n \leq n}\hspace{-1.5em} a_{1,k_1}\cdots a_{n,k_n} (e_{k_1}^* \wedge \cdots \wedge e_{k_n}^*)$\\ 
\end{tabular}\retTwo\par}

Next, if the multi-index $I = (k_1, \ldots, k_n)$ is repeating, then $e_{k_1}^* \wedge \cdots \wedge e_{k_n}^* = 0$. (This is a consequence of the fact that the wedge product with respect to $1$-tensors is anti-commutative). It follows that we can cancel out a bunch of terms in the sum and be left with:

{\centering $A^\dag(f_1^* \wedge \cdots \wedge f_n^*) = \sum\limits_{\sigma \in S_n}a_{1, \sigma(1)} \cdots a_{n, \sigma(n)} (e_{\sigma(1)}^* \wedge \cdots \wedge e_{\sigma(n)}^*)$ \retTwo\par}

But now note that:\\ [-9pt]

{\centering\begin{tabular}{l}
	$(e_{\sigma(1)}^* \wedge \cdots \wedge e_{\sigma(n)}^*) = \pi_V(e_{\sigma(1)}^* \otimes \cdots \otimes e_{\sigma(n)}^*)$\\ [6pt]
	$\phantom{(e_{\sigma(1)}^* \wedge \cdots \wedge e_{\sigma(n)}^*)} = \pi_V((e_1^* \otimes \cdots \otimes e_n^*)^\sigma) = \sgn(\sigma) \pi_V(e_1^* \otimes \cdots \otimes e_n^*) = \sgn(\sigma) e_1^* \wedge \cdots \wedge e_n^*$.
\end{tabular}\retTwo\par}

So, we conclude that $A^\dag(f_1^* \wedge \cdots \wedge f_n^*) = \sum\limits_{\sigma \in S_n}\sgn(\sigma)a_{1, \sigma(1)} \cdots a_{n, \sigma(n)} (e_1^* \wedge \cdots \wedge e_n^*)$.\retTwo

Letting $W = V$ and $f_i = e_i$ for all $i$, we in turn have shown that:

{\centering$\det(A) = \sum\limits_{\sigma \in S_n}\sgn(\sigma)a_{1, \sigma(1)} \cdots a_{n, \sigma(n)}$.\newpage\par}

\hOne\dispDate{8/26/2025}

In physics yesterday we started talking about statistical mechanics, and that inspired me to try focusing on real analysis again. So, I will be returning to Folland for a bit. I think I'll start off where I left off at chapter 7.\retTwo

\hTwo Let $X$ be an LCH space and $\mathcalli{B}_X$ be the collection of Borel sets on $X$. At where we left off (on page 62), we had showed that the space $M(X)$ of complex Radon measures on $(X, \mathcalli{B}_X)$ is a normed complex vector space when equipped with the norm $\mu \mapsto \|\mu\| = |\mu|(X)$. Also, we had shown that the map $\mu \mapsto I_\mu$ where $I_\mu(f) = \int f \df \mu$ is an isometric isomorphism from $M(X)$ to $C_0(X)^*$.\retTwo

\Hstatement\blab{Exercise 7.8:} Suppose that $\mu$ is a Radon measure on $X$. If $\phi \in L^1(\mu)$ and $\phi \geq 0$, then $\nu = \phi \df \mu$ is a Radon measure.
\begin{myIndent}\HexOne
	Since $\phi \in L^1(\mu)$, we know that $\nu(E) = \int_E \phi \df \mu$ is finite for all $E \in \mathcalli{B}_X$. Thus $\nu$ is a finite measure. This trivially satisfies the requirement that $\nu(K)$ is finite for all compact $K$.\retTwo

	Now let $\varepsilon > 0$ and note that by corollary 3.6 (see my math 240a notes from Fall quarter), there exists $\delta > 0$ such that $\mu(A) < \delta$ implies that $|\nu(A)| = \nu(A) < \varepsilon$ for all $A \in \mathcalli{B}_X$. This easily let's us show all the desired regularity properties of $\nu$.
	\begin{myIndent}
		If $E \in \mathcalli{B}_X$ then let $U \supseteq E$ be an open set such that $\mu(U - E) < \delta$. Then we know that $\nu(U - E) < \varepsilon$. Taking $\varepsilon \to 0$ shows that $\nu$ is outer regular on $E$.\retTwo

		If $U \subseteq X$ is open, then let $K \subseteq U$ be a compact set such that $\mu(U - K) < \delta$. Then we know that $\nu(U - K) < \varepsilon$. Taking $\varepsilon \to 0$ shows that $\nu$ is inner regular on $U$.\retTwo
	\end{myIndent}
\end{myIndent}

\hypertarget{Folland exercise 7.8 corollary}{\blab{Corollary:}} If $\mu$ is a fixed positive Radon measure and $f \in L^1(\mu)$, then $\nu = f \df \mu$ is a complex Radon measure.

\begin{myIndent}\HexOne
	If we write $f = f_1 - f_2 + i(f_3 - f_4)$ and set $\nu_j = f_j \df \mu$ for all $j$, then it's clear from\\ the last exercise that all the $\nu_j$ are finite (and thus complex) Radon measures. Also,\\ $\nu = \nu_1 - \nu_2 + i(\nu_3 - \nu_4)$. So $\nu$ is also a complex radon measure.\retTwo
\end{myIndent}

\hTwo\mySepTwo

In one sense the following is completely unnecessary. I will never need the full generality\\ of what I'm about to prove (probably). On the other hand, Folland doesn't prove this and instead points his readers to the bibliography. So, for the first time in my life I went and procured a book from the bibliography of a math textbook. Also, this was especially a pain since when I finally got the textbook, it wouldn't convert to a pdf for some reason and my main E-book reader couldn't read it. So I eventually downloaded a DJVu reader that looks like it's from the fucking 2000s in order to finally read the book.\retTwo

Considering the fact that there aren't just pdfs of this book floating around willy nilly on the internet, I should probably give a citation instead of just vaguely describing it. The book I will be briefly following along with is Hewitt and Stromberg's \ul{Real and Abstract Analysis}. I've also made a bibiliography section at the end of the pdf where this book will have the honor of being the first citation.\newpage

\hTwo 
A measure $\mu$ on $(X, \mathcal{M})$ is called \udefine{decomposable} if there is a family $\mathcal{F} \subseteq \mathcal{M}$ with the\\ following properties:\\ [-20pt]
\begin{itemize}
	\item[(i)] $\mu(F) < \infty$ for all $F \in \mathcal{F}$;\\ [-18pt]
	\item[(ii)] The members of $\mathcal{F}$ are disjoint and their union is $X$;\\ [-18pt]
	\item[(iii)] If $\mu(E) < \infty$, then $\mu(E) = \sum_{F \in \mathcal{F}} \mu(E \cap F)$;\\ [-18pt]
	\item[(iv)] If $E \subseteq X$ and $E \cap F \in \mathcal{M}$ for all $F \in \mathcal{F}$, then $E \in \mathcal{M}$.\\ [-12pt]
\end{itemize}

Also, we call $\mathcal{F}$ a \udefine{decomposition} of $(X, \mathcal{M}, \mu)$.
\begin{myIndent}\myComment
	Note that if $\mu$ is $\sigma$-finite, then we clearly have that $\mu$ is decomposable on $(X, \mathcal{M})$.\retTwo
\end{myIndent}

\exTwo\ul{Lemma 19.26:} Let $(X, \mathcal{M})$ be a measurable space and let $\mu$ and $\nu$ be arbitrary measures on $(X, \mathcal{M})$ such that $\mu(X) < \infty$ and $\nu \ll \mu$. Then there exists a set $E \in \mathcal{M}$ such that:
\begin{itemize}
	\item[(i.)] For all $A \in \mathcal{M}$ with $A \subseteq E$, either $\nu(A) = 0$ or $\nu(A) = \infty$. Also, if $\nu(A) = 0$, then so does $\mu(A) = 0$.
	\item[(ii.)] $\nu$ is $\sigma$-finite on $E^\comp$. 
\end{itemize}

\begin{myIndent}\exThreeP
	Proof:\\
	Consider the family:
	
	{\centering $\mathcalli{B} \coloneqq \{B \in \mathcal{M} : \forall C \in \mathcal{A},\hspace{0.3em} C \subseteq B \Longrightarrow \nu(C) = 0 \text{ or } \nu(C) = \infty\}$.\retTwo\par}

	Importantly, we know that $\mathcalli{B} \neq \emptyset$ since $\emptyset \in \mathcalli{B}$, and also that $\mu(B) \leq \mu(X) < \infty$ for all $B \in \mathcalli{B}$. So, it is well defined to set $\alpha \coloneqq \sup_{B \in \mathcal{B}} \mu(B)$. Next note that if $B , B^\prime \in \mathcalli{B}$, then $B \cup B^\prime \in \mathcalli{B}$.
	\begin{myIndent}\exPPP
		Suppose $C \subseteq B \cup B^\prime$ is measurable. Then $\nu(C) = \nu(C \cap B) + \nu(C \cap (B^\prime - B))$. And since $C \cap B$ and $C \cap (B^\prime - B)$ are both measurable subsets of sets in $\mathcalli{B}$, we know that both have $\nu$-measure zero or infinity. It follows that $\nu(C) \in \{0, \infty\}$.\retTwo
	\end{myIndent}
	
	We can thus construct a nondecreasing sequence of sets $(B_n)_{n \in \mathbb{N}}$ in $\mathcalli{B}$ such that\\ $\lim_{n \to \infty} \mu(B_n) = \alpha$. Setting $D = \bigcup_{n \in \mathbb{N}} B_n$, we have that $\mu(D) = \alpha$ and we can also show that $D \in \mathcalli{B}$ using near identical reasoning as in the prior pink text.\retTwo

	Next we show that $\nu$ is semifinite on $D^\comp$. Suppose $F \in \mathcal{M}$ with $F \subseteq D$ and $\nu(F) = \infty$. Then for the sake of contradiction assume that $\nu(G)$ equals $0$ or $\infty$ for every measurable subset $G \subseteq F$. It would follow that $F \cup D \in \mathcalli{B}$. But then since $\nu(F) > 0$ and $\nu \ll \mu$, we'd know that $\mu(F) > 0$. Hence, $F \cup D$ would be a set in $\mathcalli{B}$ with $\mu(F \cup D) > \alpha$ (and that inequality is strict since $\alpha < \infty$). Yet that contradicts how we defined $\alpha$.\retTwo

	We furthermore show that $\nu$ is $\sigma$-finite on $D^\comp$. Let:

	{\centering $\mathcalli{F} \coloneqq \{F \in \mathcal{M} : F \subseteq D^\comp \text{ and } \nu \text{ is \sigma-finite on } F\}$.\retTwo\par}

	Like before, there exists a nondecreasing sequence $(F_n)_{n \in \mathbb{N}}$ in $\mathcalli{F}$ such that\\ $\lim_{n\to \infty}\mu(F_n) = \sup_{F \in \mathcalli{F}}\mu(F) \eqqcolon \beta$. And if we set $F = \bigcup_{n \in \mathbb{N}} F_n$, then we clearly have that $F \in \mathcalli{F}$ and $\mu(F) = \beta$. Our claim is that $\nu(F^\comp \cap D^\comp) = 0$.
	\begin{myIndent}\exPPP
		Suppose not. Since $\nu$ is semifinite on $D^\comp$, we would have that there exists a\\ measurable set $H \subseteq F^\comp \cap D^\comp$ with $0 < \nu(H) < \infty$. But since $\nu \ll \mu$, we'd have that $\mu(H) > 0$. It would then follow that $F \cup H \in \mathcalli{F}$ and satisfies that $\mu(F \cup H) > \beta$. But this contradicts how we defined $\beta$.\newpage
	\end{myIndent}

	It easily follows that $\nu$ is $\sigma$-finite on $F \cup (F^\comp \cap D^\comp) = D^\comp$.\retTwo

	To finish off, let $\mathcalli{G} \coloneqq \{B \in \mathcal{M}: B \subseteq D \text{ and } \nu(B) = 0\}$. Now $\mathcalli{G}$ is nonempty since $\emptyset \in \mathcalli{G}$. So, it is well defined to set $\gamma \coloneqq \sup_{B \in \mathcalli{G}} \mu(B)$. Also, like before we have that if $B, B^\prime \in \mathcalli{G}$ then $B \cup B^\prime \in \mathcalli{G}$. So, we can once again take the union of a nondecreasing\\ sequence of sets to get a set $G \subseteq D$ in $\mathcalli{G}$ with $\mu(G) = \gamma$. And finally, set $E \coloneqq D \cap G^\comp$.\\ [-10pt]

	\begin{itemize}
		\item[(i.)] Since $E \subseteq D$, we know that any measurable $A \subseteq E$ satisfies that $\nu(A) \in \{0, \infty\}$.\\ Also, if $\nu(A) = 0$ then we must have that $\mu(A) = 0$ since otherwise $G \cup A \in \mathcalli{G}$ and $\mu(G \cup A) > \gamma$, which is a contradiction.
		
		\item[(ii.)] $E^\comp = D^\comp \cup G$. And since $\nu$ is $\sigma$-finite on $D^\comp$ and $\nu(G) = 0$, we have that $\nu$ is\\ $\sigma$-finite on $E^\comp$. $\blacksquare$\retTwo
	\end{itemize}
\end{myIndent}

\exTwo\ul{Theorem 19.27: An Extension of the Lebesgue-Radon-Nikodym Theorem:} 
\begin{myIndent}
	Let $(X, \mathcal{M}, \mu)$ be decomposable via the decomposition $\mathcalli{F}$, and let $\nu$ be an arbitrary\\ signed measure such that $\nu \ll \mu$.  Then there exists an extended real $\mathcal{M}$-measurable\\ function $f: X \to \overline{\mathbb{R}}$ such that $\nu(A) = \int_A f \df \mu$ for all $A \in \mathcal{M}$ for which $\mu$ is $\sigma$-finite.\\ Also, we can take $f$ to be finite on any $F$ on which $\nu$ is $\sigma$-finite, and if $\nu$ is a positive measure, we can take $f$ to be nonnegative. Furthermore, if $g$ is any extended real $\mathcal{M}$-measurable function such that $\nu(A) = \int_A g \df \mu$ when $\mu(A) < \infty$, then we\\ know $f\chi_E = g\chi_E$ $\mu$-a.e. for all $E \in \mathcal{M}$ on which $\mu$ is $\sigma$-finite.\retTwo

	\exThreeP%
	Proof:\\
	We shall first consider the simpler case where $\nu$ is a positive measure.\retTwo

	Now restricting $\mu$ and $\nu$ to subspaces of $(X, \mathcal{M})$ won't change that $\nu \ll \mu$. Consequently, by restricting $\mu$ and $\nu$ to any subspace $F \in \mathcalli{F}$ and applying our prior lemma, we can conclude that there are sets $D_F, E_F \in \mathcal{M}$ such that $D_F \cap E_F = \emptyset$; $D_F \cup E_F = F$;\\ $\nu$ is $\sigma$-finite on $D_F$; and all measurable $A \subseteq E_F$ satisfy that $\nu(A) = 0 \Longrightarrow \mu(A) = 0$ and $\nu(A) \in \{0, \infty\}$.
	\begin{myIndent}\exPPP
		Note that if $\nu$ is $\sigma$-finite on all of $F$, we can just take $D_F = F$ and $E_F = \emptyset$.\retTwo
	\end{myIndent}

	And going one step further, if we restrict $\mu$ and $\nu$ to the subspace $D_F$ of $(X, \mathcal{M})$, then we know by the typical Lebesgue-Radon-Nikodym theorem that there is a finite measurable nonnegative function $f_0^{(F)}: X \to [0, \infty)$ such that $\nu(A) = \int_A f_0^{(F)} \df \mu$ for all $A \in \mathcal{M}$ with $A \subseteq D_F$.\retTwo

	Now define $f$ by pasting together the functions:

	{\centering $f|_F(x) \coloneqq \left\{\begin{matrix}
		f_0^{(F)}(x) & \text{ if } x \in D_F \\ \infty & \text{ if } x \in E_F
	\end{matrix}\right.$ \retTwo\par}

	\begin{itemize}
		\item To show that $f$ is measurable, let $(a, \infty]$ be an open ray in $\overline{\mathbb{R}}$. Then:
		
		{\centering$f^{-1}((a, \infty]) \cap F= f|_F^{-1}((a, \infty]) = (f_0^{(F)})^{-1}((a, \infty]) \cup E_F$.\retTwo\par}
		
		And since $f_0^{(F)}$ is measurable on the subspace $D_F$ of $(X, \mathcal{M})$, we've thus proven that $f^{-1}((a, \infty]) \cap F \in \mathcal{M}$ for all $F$. It follows from the definition of a decomposition that $f^{-1}((a, \infty]) \in \mathcal{M}$. And since the open rays form a basis for $\mathcalli{B}_{\overline{\mathbb{R}}}$, we have proven that $f$ is measurable.\newpage

		\item To show that $\nu(A) = \int_A f \df \mu$ when $\mu$ is $\sigma$-finite on $A$, first suppose that $\mu(A) < \infty$. Then we know from axiom (iii) of the definition of a decomoposition that there exists a countable subset $\mathcalli{F}_0 \subseteq \mathcalli{F}$ such that $\mu(A) = \sum_{F \in \mathcalli{F}_0} \mu(A \cap F)$. One consequence of this is that $\int_A f \df \mu = \sum_{F \in \mathcalli{F}_0} \int_{A \cap F} f \df \mu$.\retTwo
		
		Another consequence is that because the $F \in \mathcalli{F}$ partition $X$, we know\\ $\mu(A \cap (\bigcup_{F \in \mathcalli{F}_0^\comp} F)) = 0$. Then in turn, since $\nu \ll \mu$, we also know that\\ $\nu(A \cap (\bigcup_{F \in \mathcalli{F}_0^\comp} F)) = 0$. So, we must have that $\nu(A) = \sum_{F \in \mathcalli{F}_0}\nu(A \cap F)$.\\ And now we claim for each $F$ that $\nu(A \cap F) = \int_{A \cap F} f \df \mu$.

		\begin{myIndent}\exPPP
			Clearly $\nu(A \cap F) = \nu(A \cap D_F) + \nu(A \cap E_F) = \int_{A \cap D_F} f \df \mu + \nu(A \cap E_F)$.\retTwo
			
			Also, because $A \cap E_f \subseteq E_F$, we know that $\nu(A \cap E_F) \in \{0, \infty\}$ with\\ $\mu(A \cap E_F) = 0$ if and only if $\nu(A \cap E_f) = 0$. So if $\mu(A \cap E_F) = 0$,\\ then $\int_{A \cap E_F} f \df \mu = 0 = \nu(A \cap E_F)$. Meanwhile, if $\mu(A \cap E_F) > 0$, then\\ $\int_{A \cap E_F} f \df \mu = \int_{A \cap E_F} (\infty) \df \mu = \infty = \nu(A \cap E_F)$. Either way, we have that:

			{\centering $\nu(A \cap F) = \int_{A \cap D_F} f \df \mu + \int_{A \cap E_F} f \df \mu = \int_A f \df \mu$ \retTwo\par}
		\end{myIndent}

		So, we've shown that $\nu(A) = \sum_{F \in \mathcalli{F}_0} \int_{A \cap F} f \df \mu = \int_A f \df \mu$ when $\mu(A) < \infty$. The case where $\mu(A) = \infty$ and $\mu$ is $\sigma$-finite on $A$ then easily follows.\retTwo

		\item Finally, suppose $g$ is as in the theorem statement. Then for every $F \in \mathcalli{F}$ and every measurable set $A \subseteq D_F$, we have that $\nu(A) = \int_A f \df \mu = \int_A g \df \mu$. The only way this is possible is if $f = g$ $\mu$-a.e. on $D_F$.\retTwo
		
		Meanwhile, we know for any $F \in \mathcalli{F}$ that $A \coloneqq E_F \cap g^{-1}([0, \infty)) \in \mathcal{M}$ because $g$ is measurable. Now suppose $\mu(A) > 0$. Then letting $A_n \coloneqq \{x \in A : g(x) < n\}$ for each $n$, we have that the $A_n$ form an increasing sequence of sets whose union is $A$. So, we know that $\nu(A) = \lim_{n \to \infty}\nu(A_n)$. Also, since $\nu(A) > 0$ we know that $\nu(A_n) > 0$ for some $n$. Then in turn $\nu(A_n) = \infty$ since $A_n \subseteq E_F$. But this contradicts the fact that $\nu(A_n) = \int g \df \mu \leq n \mu(A_n) < \infty$. So, we conclude that $\mu(A) = 0$.\retTwo

		As a result, we've shown for any $F \in \mathcalli{F}$ that $f = g$ $\mu$-a.e. on $D_F \cup E_F = F$.\retTwo
		
		The fact that $g = f$ $\mu$-a.e. on any measurable set such that $\mu(E) < \infty$ is then a simple consequence of the fact that there is a countable subset $\mathcalli{F}_0 \subseteq \mathcalli{F}$ as well a $\mu$-null set $N \subseteq X$ such that $E = N \cup (\bigcup_{F \in \mathcalli{F}_0} (F \cap E))$. And if $\mu$ is $\sigma$-finite on $E$, then we can show that $g = f$ $\mu$-a.e. on $E$ by considering $E$ as a countable union of sets on which we've already showed $g = f$ $\mu$-a.e.
	\end{itemize}

	\mySepThree

	Now we return to the case where $\nu$ is signed. Let $\nu^+$ and $\nu^-$ be the positive and negative variations of $\nu$, and also let $P$ and $N$ be measurable subsets of $X$ such that $\nu^+(N) = 0$ and $\nu^-(P) = 0$. Then by our prior reasoning we know there exists measurable functions $f^+$ and $f^-$ such that $\nu^+(A) = \int_A f^+ \df \mu$ and $\nu^-(A) = \int_A f^- \df \mu$ for all $A$ on which $\mu$\\ is $\sigma$-finite. Also, since either $\nu^+$ or $\nu^-$ is finite, we know either $f^+$ or $f^-$ never outputs $\infty$. So by setting $f = f^+ - f^-$, we get a measurable function such that $\nu(A) = \int_A f \df \mu$ for all $A$ on which $\mu$ is $\sigma$-finite.\newpage

	Now suppose $g$ is as in the theorem statement, and let $g^+$ and $g^-$ be it's positive and\\ negative parts. Given any $F \in \mathcalli{F}$ we must have that $\int_A g \df \mu = \nu(A) = \nu^+(A) \geq 0$ for\\ all measurable $A \subseteq F \cap P$. It follows that $g \geq 0$ $\mu$-a.e. on $F \cap P$, and in turn:
	
	{\centering $\int_A g^+ \df \mu = \nu^+(A)$ when $A \subseteq F \cap P$.\retTwo\par}

	By similar reasoning, we can show that $g \leq 0$ $\mu$-a.e. on $F \cap N$. This is important because it shows that $g^+ = 0$ $\mu$-a.e. on $F \cap N$. So $\int_A g^+ \df \mu = 0 = \nu^+(A)$ for all $A \subseteq F \cap N$. And hence, we can conclude that $\int_A g^+ \df \mu = \nu^+(A)$ for all $A \subseteq F$. It easily follows that $\int_A g^+ \df \mu = \nu^+(A)$ for all $A$ satisfying that $\mu(A) < \infty$. But then by the prior reasoning we did, we know that $g^+ = f^+$ $\mu$-a.e. on any set $E$ which $\mu$ is $\sigma$-finite on.\retTwo

	Analogous reasoning can be used to show that $g^- = f^-$ $\mu$-a.e. on any set $E$ which $\mu$ is $\sigma$-finite on. $\blacksquare$\retTwo
\end{myIndent}

\ul{Corollary:} Let $(X, \mathcal{M}, \mu)$ be a $\sigma$-finite measure space and suppose $\nu$ is any arbitrary signed measure such that $\nu \ll \mu$. Then there exists a measurable function $f: X \to \overline{\mathbb{R}}$ such that\\ $\nu = f \df \mu$. Also, if $g: X \to \overline{\mathbb{R}}$ is another function satisfying that $\nu = g \df \mu$, then $f = g$\\ $\mu$-a.e.\retTwo

\hTwo We can of course also write a version of the prior theorem and corollary for when $\nu$ is a complex measure. Specifically, just apply the prior theorem to the real and imaginary parts of $\nu$ separately. That said, the corollary stops being interesting if $\nu$ is complex.\retTwo

Now, it's unfortunate that this extension of the Lebesgue-Radon-Nikodym theorem loses the ability to decompose $\nu$ into a continuous part and a mutually singular part. That said, an extremely convenient fact which makes the last theorem feel more worthwhile is that it turns out that every positive Radon measure is decomposable (with an asterisk attached).\retTwo

\mySepTwo

To start off, we need an exercise from Folland.

\begin{myIndent}\myComment
	Note that this exercise references a bunch of exercises which I already did in my Math 240a notes. Look there if you want to see stuff like the definition of the saturation of a measure. I even just now went back and clean up a bunch of problems with my answer for exercise 1.16(e)!! (haha why was I so stupid I swear I want to put an evoker to my head\dots)\retTwo 
\end{myIndent}

\Hstatement\blab{Exercise 1.22:} Let $(X, \mathcal{M}, \mu)$ be a measure space and let $\mu^*$ be the outer measure induced by treating $\mu$ as a premeasure on $\mathcal{M}$. Then let $\mathcal{M}^*$ be the $\sigma$-algebra of $\mu^*$-measurable sets and $\overline{\mu} \coloneqq \mu^*|_{\mathcal{M}^*}$.
\begin{itemize}
	\item[(a)] If $\mu$ is $\sigma$-finite, then $(X, \mathcal{M}^*, \overline{\mu})$ is the the completion of $(X, \mathcal{M}, \mu)$.
	
	\begin{myIndent}\HexOne
		If $E \in \mathcal{M}^*$, then since $\mu$ is $\sigma$-finite we know by exercise 1.18 that there exists a set $A$ in $\mathcal{M}_{\sigma \delta} \subseteq \mathcal{M}$ such that $E \subseteq A$ and $\overline{\mu}(A - E) = 0$. Hence $E = A \cup N$ where $A \in \mathcal{M}$ and $\overline{\mu}(N) = 0$. And since $\overline{\mu}$ is the restriction of the outer measure induced by $\mu$, we know there is a null set $N^\prime \in \mathcal{M}$ with $N \subseteq N^\prime$. This proves that $\mathcal{M}^*$ is a subset of the completion of $\mathcal{M}$ (which I will hereafter call $\mathcal{M}^\wedge$).\newpage
		
		Meanwhile, by Carathéodory's theorem we know that $(X, \mathcal{M}^*, \overline{\mu})$ is a complete\\ measure space. So, suppose $E = F \cup N$ where $F \in \mathcal{M}$ and $N \subseteq N^\prime$ with\\ $\mu(N^\prime) = 0$. Then since $N^\prime \in \mathcal{M}^*$ and $\overline{\mu}(N^\prime) = \mu(N^\prime) = 0$, we know $N \in \mathcal{M}^*$.\\ In turn this means that $E = F \cup N \in \mathcal{M}^*$ and we've showed that $\mathcal{M}^\wedge \subseteq \mathcal{M}^*$.\retTwo

		Combining the last two paragraphs, we have that $\mathcal{M}^* = \mathcal{M}^\wedge$. And since there is only one measure that extends $\mu$ to $\mathcal{M}^\wedge$, we automatically have that $\overline{\mu}$ is the completion of $\mu$.\retTwo
	\end{myIndent}
	
	\item[(b)] In general, $\overline{\mu}$ is the saturation of the completion of $\mu$.
	
	\begin{myIndent}\HexOne
		For the sake of notation, I will write that $\mathcal{M}^\wedge$ and $\mu^\wedge$ are the completions of $\mathcal{M}$ and\\ [-2pt] $\mu$ respectively; that $\widetilde{\mathcal{M}}^\wedge$ is the collection of $\mu^\wedge$-locally measurable sets; and that $\widetilde{\mu}^\wedge$\\ [2pt] is the saturation of of $\mu^\wedge$.\\ [-4pt]

		
		Our first goal is to show that $\widetilde{\mathcal{M}}^\wedge = \mathcal{M}^*$. Equivalently, this means we need to show that a  set $E \subseteq X$ is $\mu^*$-measurable if and only if it is locally $\mu^\wedge$-measurable.
		\begin{myIndent}\HexTwoP
			$(\Longrightarrow)$\\
			Suppose $E \subseteq X$ is $\mu^*$-measurable and let $A \subseteq X$ be any set in $\mathcal{M}^\wedge$ with\\ $\mu^\wedge(A) < \infty$. In part (a) we were able to show without using any $\sigma$-finiteness\\ that $\mathcal{M}^\wedge \subseteq \mathcal{M}^*$. Thus, $A$ and $E \cap A$ are $\mu^*$-measurable. Also note that by\\ definition of the completion of a measure, we can pick a set $F \in \mathcal{M}$ with\\ $A \subseteq F$ and $\mu^\wedge(A) = \mu(F)$.\retTwo

			Now we claim by a similar argument as in exercise 1.18 that there exists a set $B \in \mathcal{M}$ such that $E \cap A \subseteq B$ and $\mu^*(B - (A \cap E)) = 0$.
			\begin{myIndent}\HexPPP
				Using exercise 1.18(a), for each $j \in \mathbb{N}$ pick $B_j \in \mathcal{M}$ such that $(A \cap E) \subseteq B_j$\\ and $\mu^*(B_j) \leq \mu^*(E \cap A) + \frac{1}{j}$. Then since $E \cap A$ is $\mu^*$-measurable and\\ $B \cap (E \cap A) = E \cap A$, we have that:
				
				{\centering $\mu^*(E \cap A) + \mu^*(B_j - (E \cap A)) = \mu^*(B_j) \leq \mu^*(E \cap A) + \frac{1}{j}$ \retTwo\par}

				Next note that $\mu^*(E \cap A) \leq \mu^*(A) \leq \mu^*(F) = \mu(F) = \mu^\wedge(A) < \infty$. Thus, we can subtract $\mu^*(E \cap A)$ from both sides of our inequality above to get that:
				
				{\centering $\mu^*(B_j - (E \cap A)) < \frac{1}{j}$ \retTwo\par}

				And now, if we define $B = \bigcap_{j \in \mathbb{N}} B_j$, it's clear that $B \in \mathcal{M}$ and that\\ $\mu^*(B - (E \cap A)) \leq \mu^*(B_j - (E \cap A)) < \frac{1}{j}$ for all $j \in \mathbb{N}$. So, $\mu^*(B - (E \cap A)) = 0$.\retTwo
			\end{myIndent}

			Now since $\mu^*(B - (A \cap E)) = 0$, we know that there exists a set $C \in \mathcal{M}$ with $\mu(C) = 0$ and $B - (A \cap E)\subseteq C$.\\ [-16pt]
			\begin{myIndent}\HexPPP
				Specifically, for each $n \in \mathbb{N}$ there exists a countable covering $\{C_j^{(n)}\}_{j \in \mathbb{N}}$ of\\ [-2pt] $B - (A \cap E)$ such that $\mu(\bigcup_{j \in \mathbb{N}} C_j^{(n)}) \leq \sum_{j = 0}^\infty \mu(C_j^{(n)}) < \sfrac{1}{n}$ In turn,\\ [-2pt] $C \coloneqq \bigcap_{n \in \mathbb{N}} (\bigcup_{j \in \mathbb{N}} C_j^{(n)})$ is a set in $\mathcal{M}$ with $B - (A \cap E) \subseteq C$ and $\mu(C) = 0$.\retTwo
			\end{myIndent}

			Finally, we have that $N \coloneqq B - (A \cap E) \in \mathcal{M}^\wedge$ because $N \subseteq C$\\ and $(X, \mathcal{M}^\wedge, \mu^\wedge)$ is complete. And since $B \in \mathcal{M} \subseteq \mathcal{M}^\wedge$, we have that\\ $(A \cap E) = B - N \in \mathcal{M}^\wedge$. This proves that $E$ is locally $\mu^\wedge$-measurable.\newpage

			$(\Longleftarrow)$\\
			Suppose $E \subseteq X$ is locally $\mu^\wedge$-measurable and then choose any $F \subseteq X$. If $\mu^*(F) = \infty$, then we trivially have that $\mu^*(F \cap E) + \mu^*(F - E) \leq \mu^*(F)$. So, it suffices to assume that $\mu^*(F) < \infty$. But then by exercise 1.18(a) there exists for each $j \in \mathbb{N}$  a set $A_j \in \mathcal{M}$ such that $F \subseteq A_j$ and $\mu(A_j) < \mu^*(F) + \sfrac{1}{j}$.\\ And, by taking the intersection of all the $A_j$ we get a set $A \in \mathcal{M}$ such that\\ $A \subseteq F$ and $\mu(A) = \mu^*(F) < \infty$.\retTwo

			Since $E$ is locally $\mu^\wedge$-measurable, it follows that $A \cap E \in \mathcal{M}^\wedge$. Then since $\mathcal{M}^\wedge \subseteq \mathcal{M}^*$, we know that $\mu^*(F) = \mu^*(F \cap (A \cap E)) + \mu^*(F - (A \cap E))$.\\ And this shows that $E$ is $\mu^*$-measurable since $F \cap (A \cap E) = F \cap E$ and $F - (A \cap E) = F - E$ on account of the fact that $F \subseteq A$.\retTwo
		\end{myIndent}

		With that, we've now shown that $\widetilde{\mathcal{M}}^\wedge = \mathcal{M}^*$. Also, since there is only one extension\\ of $\mu$ to $\mathcal{M}^\wedge$ and both $\overline{\mu}$ and $\widetilde{\mu}^\wedge$ extend $\mu$ to $\mathcal{M}^\wedge$, we must have that $\overline{\mu}(E) = \widetilde{\mu}^\wedge(E)$ whenever $E \in \mathcal{M}^\wedge \subseteq \mathcal{M}^*$ Also, by definition of the saturation of a measure, we have that if $E \in \mathcal{M}^* - \mathcal{M}^\wedge$, then $\widetilde{\mu}^\wedge(E) = \infty$. So, all we need to do left is show that if $E$ is locally $\mu^\wedge$-measurable but not in $\mathcal{M}^\wedge$, then $\overline{\mu}(E) = \mu^*(E) = \infty$.\retTwo

		Suppose $E$ is $\mu^*$-measurable and $\overline{\mu}(E) = \mu^*(E) < \infty$. When we were proving\\ the backwards implication before, we showed that there is a set $A \in \mathcal{M}$ with\\ $A \subseteq E$ and $\mu^*(E) = \mu(A) < \infty$. Next, when we were proving the forwards\\ implication, we showed that there is a set $B \in \mathcal{M}$ with $A \cap E = E \subseteq B$ and\\ $\mu^*(B - (A \cap E)) = \mu^*(B - E) = 0$. Then afterwords, we showed that there exists\\ $C \in \mathcal{M}$ with $\mu(C) = 0$ and $B - E \subseteq C$. Finally, we have that $B - E \in \mathcal{M}^\wedge$\\ and in turn that $E = B - (B - E) \in \mathcal{M}^\wedge$. $\blacksquare$\retTwo

		\begin{myIndent}\myComment
			As a side note, this shows that you can't indefinitely extend a measure space to larger and larger $\sigma$-algebras just by applying the theorem on page 24 of my latex math 240a notes over and over again. After all, the saturation of a complete measure is still complete (by exercise 1.16 in my latex math 240a notes). So, if you complete it again you just get back the same measure space \color{BrickRed}(see page 178 for a proof of this...)\color{RawerSienna}. Also, it's easy to see that saturating a measure does not add any new locally measurable sets since all the added measurables sets have infinite measure. So, taking a saturation twice back to back is the same as taking it once. \retTwo
		\end{myIndent}
	\end{myIndent}
\end{itemize}

\hTwo Now let $X$ be an LCH space and let $\mu$ be a Radon measure on $(X, \mathcalli{B}_X)$. Then set:

{\centering$\mu^*(E) \coloneqq \inf \{\mu(U) : U \supseteq E \text{ with } U \text{ open}\}$ for all $E \in \mathcal{P}(X)$.\retTwo\par}

Recall from the proof of the Riesz Representation thorem in my math 240c notes that $\mu^*$ is a well-defined outer measure for which every Borel set is $\mu^*$-measurable and $\mu^*|_{\mathcalli{B}_X} = \mu$.
Furthermore, if $\mathcal{M}$ is the collection of $\mu^*$-measurable sets and $\overline{\mu} = \mu^*|_{\mathcal{M}}$, we have by the definition of $\mu^*$ that $\overline{\mu}$ is outer regular on all of $\mathcal{M}$. Also, $\overline{\mu}$ is fully determined by the linear functional $I(f) \coloneqq \int f \df \mu$. (since $I$ uniquely determines $\mu(U)$ for each open $U$).\retTwo

Importantly, we can also say that $\mu^*(E) = \inf \{\mu(B) : B \supseteq E \text{ with } B \in \mathcalli{B}_X\}$ for all $E \in \mathcal{P}(X)$. This is because (and I will be abusing notation to write this since otherwise it would be really cumbersome) if $B$ represents any Borel set and $U$ any open set, then:\newpage

{\centering $\mu^*(E) \leq \mu^*(\bigcap\limits_{B \supseteq E} B) \leq \inf\limits_{B \supseteq E} \mu(B) \leq \inf\limits_{U \supseteq E} \mu(U) = \mu^*(E)$.  \retTwo\par}

Next, we claim that:

{\centering\hThree $\inf \{\mu(B) : B \supseteq E \text{ with } B \in \mathcalli{B}_X\} = \inf\{\sum\limits_{n = 1}^\infty \mu(B_n) : \text{ all } B_n \text{ are Borel and } E \subseteq \bigcup_{n \in \mathbb{N}} B_n\}$\retTwo\par}

The fact that the right side is at most the left side is trivial. Meanwhile, to show the other inequality we can just note that if $(B_n)_{n \in \mathbb{N}}$ is a covering of $E$ by Borel sets, then by taking differences we can get a sequence of disjoint Borel sets $(B_n^\prime)_{n \in \mathbb{N}}$ covering $E$ such that $B_n^\prime \subseteq B_n$ for all $n$; $B \coloneqq \bigcup_{n \in \mathbb{N}} B_n$ is a Borel set containing $E$; and:

{\center$\mu(B) = \sum_{n \in \mathbb{N}}\mu(B_n^\prime) \leq \sum_{n \in \mathbb{N}}\mu(B_n)$\retTwo\par}

As a result, $\mu^*$ is equal to the outer measure induced by treating $\mu$ as a premeasure on $(X, \mathcalli{B}_X)$. Combining this with the previous exercise I did, we thus know that $(X, \mathcal{M}, \overline{\mu})$ is precisely the saturation of the completion of $(X, \mathcalli{B}_X, \mu)$.\retTwo

Now returning to Hewitt and Stromberg, here is one more lemma before I reset which variable names I have assigned to what.\retTwo

\exTwo\ul{Lemma 10.31:} For any $A \subseteq X$, the following are equivalent:
\begin{itemize}
	\item[(i)] $A$ is $\mu^*$-measurable;
	\item[(ii)] $\mu^*(U) \geq \mu^*(U \cap A) + \mu^*(U - A)$ for all open $U \subseteq X$ such that $\mu(U) < \infty$;
	\item[(iii)] $A \cap U$ is $\mu^*$-measurable for all open $U \subseteq X$ such that $\mu(U) < \infty$;
	\item[(iv)] $A \cap K$ is $\mu^*$-measurable for all compact $K \subseteq X$. 
\end{itemize}

\begin{myIndent}\exThreeP
	Proof:\\
	It's trivial that (i) implies (iv).\retTwo

	Next suppose (iv) holds and let $U$ be an open set such that $\mu(U) < \infty$. Then since $\mu$ is inner regular on all open sets, we know for each $n \in \mathbb{N}$ that there is a compact set $F_n \subseteq U$ such that $\mu(F_n) > \mu(U) - \sfrac{1}{n}$. Letting $F = \bigcup_{n \in \mathbb{N}} F_n$ we have that $F \subseteq U$; $F$ is\\ $\mu^*$-measurable (since all the $F_n$ are); and $\mu(F) \geq \mu(F_n) > \mu(U) - \sfrac{1}{n}$ for all $n$. It follows that $\mu(F) = \mu(U)$ and $\mu(U - F) = 0$.\retTwo
	
	One consequence of this is that:

	{\centering\begin{tabular}{l}
		$A \cap U = A \cap (F \cup (U - F)) = (A \cap F) \cup (A \cap (U - F))$\\
		$\phantom{A \cap U = A \cap (F \cup (U - F))} = (\bigcup_{n \in \mathbb{N}} (A \cap F_n)) \cup (A \cap (U - F))$.
	\end{tabular}\retTwo\par}

	And since $A \cap (U - F) \subseteq U - F$ with $\mu(U - F) = 0$, we know by the completeness of $(X, \mathcal{M}, \overline{\mu})$  that $A \cap (U - F)$ is in $\mathcal{M}$. Also, since we have that all the $A \cap F_n$ are in $\mathcal{M}$ by (iv), we know that $A \cap U$ is a countable union of sets in $\mathcal{M}$. This proves (iii).\retTwo

	Now suppose (iii) and let $U \subseteq X$ be an open set such that $\mu(U) < \infty$. Then both $U$ and $U \cap A$ are in $\mathcal{M}$ by (iii). And in turn we also have that $U - A = U - (U \cap A) \in \mathcal{M}$. Thus $\overline{\mu}(U) = \overline{\mu}(U \cap A) + \overline{\mu}(U - A)$ and we've shown (ii).\newpage

	Finally suppose (ii) and let $F$ be any subset of $X$. If $\mu^*(F) = \infty$, then we trivially have that $\mu^*(F) \geq \mu^*(A \cap F) + \mu^*(A - F)$. So assume $\mu^*(F) < \infty$. Then for any given $\varepsilon > 0$ we know there is an open set $U$ such that $F \subseteq U$ and $\mu(U) < \mu^*(F) + \varepsilon$. In turn:

	{\center\begin{tabular}{l}
		$\mu^*(F) + \varepsilon > \mu(U) \geq \mu^*(U \cap A) + \mu^*(U - A) \geq \mu^*(F \cap A) + \mu^*(F - A)$
	\end{tabular}\retTwo\par}

	Taking $\varepsilon \to 0$ finishes proving (i). $\blacksquare$\retTwo
\end{myIndent}

\hTwo Oh, one more lemma I'll need is that $\overline{\mu}$ is inner regular on all of its $\sigma$-finite sets. (You\\ can prove this identically to how we proved $\mu$ is inner regular on all of its $\sigma$-finite sets\\ [0pt] \vphantom{a}[see my math 240c notes]).

\mySepTwo

\exTwo\ul{Theorem 19.30:} Let $X$ be an LCH space and let $(X, \mathcal{M}, \mu)$ be the saturation of the\\ completion of a positive Borel Radon measure on $X$. Then there exists a family $\mathcalli{F}_0$ of subsets of $X$ with the following properties:
\begin{itemize}
	\item[(i)] the sets in $\mathcalli{F}_0$ are compact and have finite measure greater than $0$;
	\item[(ii)] the sets in $\mathcalli{F}_0$ are pairwise disjoint; 
	\item[(iii)] if $F \in \mathcalli{F}_0$, $U$ is open, and $U \cap F \neq \emptyset$, then $\mu(U \cap F) > 0$;
	\item[(iv)] if $E \in \mathcal{M}$ and $\mu(E) < \infty$, then $\mu(E \cap F) > 0$ for only countably many $F \in \mathcalli{F}_0$;
	\item[(v)] the set $D \coloneqq X - (\bigcup_{F \in \mathcalli{F}} F)$ is measurable and locally null (which means that\\ $\mu(E \cap A) = 0$ for all $A \in \mathcal{M}$ with $\mu(A) < \infty$ [see my homework from math 240b\\ for more info about this\dots]);
	\item[(vi)] if $Y$ is a subset of $X$ such that $Y \cap F \in \mathcal{M}$ for all $F \in \mathcalli{F}_0$, then $Y \in \mathcal{M}$.
\end{itemize}

\begin{myIndent}\exThreeP
	Proof:\\
	Let $\mathcalli{X}$ be the collection of all families of subsets in $\mathcal{M}$ which satisfy properties (i), (ii), and (iii). Then firstly note that $\emptyset \in \mathcalli{X}$. So, $\mathcalli{X}$ is not empty. Also, it easy to see that if $\mathcalli{X}_0$ is a subset of $\mathcalli{X}$ that is linearly ordered by the subset relation, then $\mathcalli{X}_0$ has an upper bound in $\mathcalli{X}$. Just set $\mathcalli{F} = \bigcup_{\mathcalli{F}^\pprime \in \mathcalli{X}_0} \mathcalli{F}^\pprime$. Then it's obvious that $\mathcalli{F}$ satisfies properties (i) and (iii) since\\ [-2pt] every $F \in \mathcalli{F}$ is in some $\mathcalli{F}^\pprime \in \mathcalli{X}_0 \subseteq \mathcalli{X}$. Also, if $F_1$ and $F_2$ are sets in $\mathcalli{F}$, then we know there exists a collection $\mathcalli{F}^\pprime \in \mathcalli{X}_0$ such that both $F_1, F_2 \in \mathcalli{F}^\pprime$. And thus in turn we know that $F_1 \cap F_2 = \emptyset$, meaning $\mathcalli{F}$ satisfies property (ii).\retTwo

	It follows by Zorn's lemma that $\mathcalli{X}$ contains a maximal family $\mathcalli{F}_0$. Our goal now is to show that $\mathcalli{F}_0$ satisfies properties (iv), (v), and (vi).\retTwo

	Suppose $E \in \mathcal{M}$ and $\mu(E) < \infty$. Then for the sake of finding a contradiction, suppose $E \cap F \neq \emptyset$ for uncountably many $F \in \mathcalli{F}_0$. Then by the outer regularity of $\mu$, we know there exists an open set $U$ such that $U \supseteq E$ and $\mu(U) < \mu(E) + 1 < \infty$. Then in turn, by property (ii) we have that $\mu(U) \geq \sum_{F \in \mathcalli{F}_0} \mu(U \cap F)$ (because $U$ is a superset of any finite union of the $F \cap U$). But now since $U \cap F \supseteq E \cap F \neq \emptyset$ for uncountably many $F \in \mathcalli{F}_0$, we know by property (iii) that $\mu(U \cap F) > 0$ for uncountably many $F \in \mathcalli{F}_0$. But that implies that $\sum_{F \in \mathcalli{F}_0} \mu(U \cap F) = \infty$, which is a contradiction. Hence, we've proven property (iv).\newpage

	Next let $U$ be any open set with $\mu(U) < \infty$ and let $\mathcalli{F}_1$ be the countable subfamily of $\mathcalli{F}_0$\\ [2pt] containing all the $F$ such that $\mu(U \cap F) > 0$. Since all $F \in \mathcalli{F}_1$ are in $\mathcal{M}$, we know that\\ [2pt] $\bigcup_{F \in \mathcalli{F}_1} F \in \mathcal{M}$. Thus $\mu(U) = \mu(U \cap \bigcup_{F \in \mathcalli{F}_1} F) + \mu(U - \bigcup_{F \in \mathcalli{F}_1} F)$. But by (iii) we have\\ [2pt] that $\mathcalli{F}_1 = \{F \in \mathcalli{F}_0 : U \cap F \neq \emptyset\}$. Thus, it's clear that $U \cap \bigcup_{F \in \mathcalli{F}_1} F = U \cap \bigcup_{F \in \mathcalli{F}_0} F$\\ [2pt] and $U - \bigcup_{F \in \mathcalli{F}_1} F = U - \bigcup_{F \in \mathcalli{F}_0} F$. And by the previous lemma, this proves that\\ [2pt] $\bigcup_{F \in \mathcalli{F}_0} F$ and $D \coloneqq X - \bigcup_{F \in \mathcalli{F}_0} F$ are in $\mathcal{M}$.\retTwo

	Now we still need to show that $D$ is locally null. So suppose for the sake of contradiction\\ that there exists $A \in \mathcal{M}$ with $0 < \mu(A \cap D) < \infty$. Then since $\mu$ is inner regular on\\ $A \cap D$, there'd be a compact set $K \subseteq A \cap D$ such that $0 < \mu(K) < \mu(A \cap D) < \infty$.\\ Also, if we consider the collection $\mathcalli{U}$ of open sets $U \subseteq X$ such that $\mu(U \cap K) = 0$, then\\ $K \cap \bigcup_{U \in \mathcalli{U}}U$ is measurable on account of $\bigcup_{U \in \mathcalli{U}}U$ being open. We claim $K \cap \bigcup_{U \in \mathcalli{U}}U$ is\\ a null set. 
	\begin{myIndent}\exPPP
		Otherwise, by inner regularity there would exist a compact set $C \subseteq K \cap \bigcup_{U \in \mathcalli{U}} U$ with $\mu(C) > 0$. And since $\mathcalli{U}$ is an open cover of $C$, we'd have that there is a finite subcover of sets $U \cap H$, all with measure zero, covering $C$. This is a contradiction.\retTwo
	\end{myIndent}

	Setting $H \coloneqq K - \bigcup_{U \in \mathcalli{U}}U$, we'd know that $H$ is compact on account of being a closed subset of $K$. Also, we'd know that $\mu(H) = \mu(K) > 0$. And it's clear that $H$ would be disjoint from all the $F \in \mathcalli{F}_0$. And finally, if $V$ were any open set such that $V \cap H \neq \emptyset$ then we'd know that $V \notin \mathcalli{U}$. But that would mean that $\mu(V \cap K) > 0$. And so:

	{\centering $\mu(V \cap H) = \mu(V \cap K) - \mu(V \cap (K - \bigcup_{U \in \mathcalli{U}}U)) = \mu(V \cap K) - 0 > 0$ \retTwo\par}

	Hence, we've shown that $\mathcalli{F}_0 \cup \{H\}$ is a collection in $\mathcalli{X}$ that is strictly larger than $\mathcalli{F}_0$. But that contradicts that $\mathcalli{F}_0$ is maximal. Thus, we've proven property (v).\retTwo

	Finally, let $Y$ be as in the theorem statement and consider any open set $U \subseteq X$ such that $\mu(U) < \infty$. Then:

	{\centering $U \cap Y = (U \cap Y \cap D) \cup (U \cap Y \cap \hspace{-0.3em}\bigcup\limits_{F \in \mathcalli{F}_0} \hspace{-0.3em} F) = (U \cap Y \cap D) \cup \hspace{-0.3em}\bigcup\limits_{F \in \mathcalli{F}_0} \hspace{-0.3em}(U \cap (Y \cap F)) $\retTwo\par}

	But now we know from before that there is a countable subfamily of $\mathcalli{F}_0$ containing all the $F$ which intersect $U$. Also, since $D$ is locally null, we know that $\mu(U \cap D) = 0$ and in turn $U \cap Y \cap D$ is measurable due to it being a subset of a null set. It follows that $U \cap Y$ is a countable union of measurable sets. So, $U \cap Y$ is measurable. By our prior lemma, we thus have that $Y \in \mathcal{M}$. This proves (vi). $\blacksquare$\retTwo

	\begin{myIndent}\myComment
		Side note: You may note that since $D$ is locally null and all compact sets have finite measure, we must have that any compact set $K \subseteq X$ contained in $D$ has measure zero. By inner regularity, this in turn implies that any open set $U \subseteq X$ contained in $D$ has measure zero.\retTwo
	\end{myIndent}
\end{myIndent}

\ul{Corollary 19.31:} Let $X$ be an LCH space and let $(X, \mathcal{M}, \mu)$ be the saturation of the\\ completion of a positive Borel Radon measure on $X$. Then $(X, \mathcal{M}, \mu)$ is decomposable.

\begin{myIndent}\exThreeP
	Proof:\\
	Let $\mathcalli{F}_0$ and $D$ be as in the last theorem. Then set $\mathcalli{F} \coloneqq \mathcalli{F}_0 \cup \{ \{x\} : x \in D\}$. We claim $\mathcalli{F}$ is a decomposition.\newpage
	\begin{itemize}
		\item[(i)] Since $X$ is Hausdorff, we know that $\{x\}$ is compact for all $x \in D$. Hence, every set in $\mathcalli{F}$ is compact and thus has finite measure.
		\item[(ii)] It's clear that all the elements of $\mathcalli{F}$ form a partition of $X$.
		\item[(iii)] If $\mu(E) < \infty$ then since $D$ is locally null, we have that $\mu(E \cap D) = 0$. Also, by how we chose $\mathcal{F}_0$ we know there are only countably many $F \in \mathcal{F}_0$ with $F \cap E \neq \emptyset$. So:
		
		{\centering $\mu(E) = \sum_{F \in \mathcalli{F}} \mu(E \cap F)$. \retTwo}

		\item[(iv)] This is an immediate consequence of the sixth property that we proved about $\mathcalli{F}_0$. $\blacksquare$\retTwo
	\end{itemize}
\end{myIndent}

\ul{Corollary 19.32: Another Extension of the Lebesgue-Radon-Nikodym Theorem:} 
\begin{myIndent}
	Let $X$ be an LCH space and let $(X, \mathcal{M}, \mu)$ be the saturation of the completion of a positive Borel Radon measure on $X$. Also let $\nu$ be any measure on $(X, \mathcal{M})$ such that $\nu \ll \mu$. Then all the conclusions of theorem 19.27 (on page 156 of my journal) hold.\retTwo
\end{myIndent}

\hTwo\mySepTwo

\blab{Reflection:}\\
It is here that Hewitt and Stromberg mostly stop being useful. So, I will give one last quote from them before expositing some more about this topic on my own (also geeze I just noticed that this quote has a typo. It's supposed to be "(19.32)"; not "(19.33)"\dots):

{\center\includegraphics[scale=0.8]{A quote.png}\retTwo\par}

I'm not gonna lie, while going down research rabbitholes is really fun, I feel like Folland\\ really oversold the significance of this result. When I went into this rabbithole, I was under the impression from Folland that I'd be proving that if $\mu$ is an arbitrary Radon measure and $\nu \ll \mu$, then $\nu$ always has the form $f \df \mu$. But that is not what was proven in theorems 19.27 and 19.32 so I don't know what Folland was yappin about. Considering the lack of specific details that Folland gives, I can't help but suspect that Folland didn't entirely understand what Hewitt and Stromberg actually proved. Although, I don't necessarily blame Folland for that.\retTwo

Hewitt and Stromberg were difficult for me to understand partly because they had a lot of (in my opinion) very questionable conventions. For example, I tried my best to hide it away as much as possible but Hewitt and Stromberg explicitely treated every measure as the restriction of an outer measure to the $\sigma$-algebra of $\mu^*$-measurable sets rather than abstracting that away because I guess they always wanted their $\sigma$-algebras to be as large as possible. In fact, while I was skimming the book I even saw some other sections where they talked about extending their measures to even larger $\sigma$-algebras. Never mind that there are advantages to working on a smaller $\sigma$-algebra (see for instance page 55 of my journal: when working on a larger $\sigma$-algebra we lose that continuity implies measurability\dots).\newpage

I'll also mention that the choice of math font in their book is awful and I couldn't help but wonder if they knew what kerning is. Anyways I guess my review of their book is that it feels dated and very clunky. Although, maybe I only had a mixed experience because I'm not an intended reader. (cough cough)\retTwo

Anyways, while getting a milkshake with my apartment-mates, I thought of a few ideas for how to make the preceding theorems actually useful.\retTwo

\mySepTwo

My first challenge is that I'm primarily working with Borel Radon measures while Hewitt and Stromberg weren't. This leads me to the following result:\retTwo

\exTwo\ul{Proposition:} Suppose $X$ is an LCH space and let $\mu$ and $\nu$ be positive Radon measures on $(X, \mathcalli{B}_X)$ such that $\nu \ll \mu$. Then let $(X, \mathcal{M}_\mu, \overline{\mu})$ be the saturation of the completion of $(X, \mathcalli{B}_x, \mu)$ and let $(X, \mathcal{M}_\nu, \overline{\nu})$ be the saturation of the completion of $(X, \mathcalli{B}_x, \nu)$.
\begin{itemize}
	\item In general $\mathcal{M}_\mu \neq \mathcal{M}_\nu$.
	
	\begin{myIndent}\exThreeP
		Proof:\\
		Let $X = \mathbb{R}$. Then set $\mu$ to be the Lebesgue measure and set $\nu$ equal to the zero\\ measure (i.e. the measure for which every set is null). Note that we trivially have\\ that $\nu \ll \mu$. Also, both are easily seen to be Radon measures.\retTwo
		
		Since $\mu$ and $\nu$ are $\sigma$-finite, we know by the exercise on pages 158 and 159 that $\mathcal{M}_\mu$ and $\mathcal{M}_\nu$ are just the completions of $\mathcal{M}$ with respect to $\mu$ and $\nu$ respectively. Since $\nu(X) = 0$, we have that $\mathcal{M}_\nu = \mathcal{P}(X)$. Meanwhile, because of the existance of Vitali sets, we know that $\mathcal{M}_\mu \neq \mathcal{P}(X)$. $\blacksquare$.\retTwo
	\end{myIndent}

	\item We do always have that $\mathcal{M}_\mu \subseteq \mathcal{M}_\nu$ when $\mu$ and $\nu$ are Radon and $\nu \ll \mu$.
	
	\begin{myIndent}\exThreeP
		Proof: (I got this argument from Hewitt and Stromberg 19.33)\\
		Suppose $A \in \mathcal{M}_\mu$ and let $F \subseteq X$ be any compact set. Since $A$ is locally measurable with respect to the completion of $\mu$, we know that $A \cap F$ is in the completion of $\mathcalli{B}_X$ with respect to $\mu$. So let $A \cap F = E \cup N$ where $E \in \mathcalli{B}_X$ and $N \subseteq N^\prime$ with $N^\prime \in \mathcalli{B}_X$ and $\mu(N^\prime) = 0$. Then since $\mu(N^\prime) = 0$ implies that $\nu(N^\prime) = 0$, we know that $A \cap F$ is in the complection of $\mathcalli{B}_X$ with respect to $\nu$. By the lemma on page 161, this proves that $A \in \mathcal{M}_\nu$. $\blacksquare$\retTwo
	\end{myIndent}

	\item We do always have that $\overline{\nu}|_{\mathcal{M}_\mu} \ll \overline{\mu}$.
	
	\begin{myIndent}\exThreeP
		Proof:\\
		Suppose $A \in \mathcal{M}_\mu$ with $\overline{\mu}(A) = 0$. Since $A$ has finite measure, we know that $A$ is not merely locally measurable but also that $A$ is in the completion of $\mathcalli{B}_X$ with respect to $\mu$. Hence, there exists a set $N \in \mathcalli{B}_X$ such that $A \subseteq N$ and $\mu(N) = 0$. But now since $\nu \ll \mu$, we have that $\nu(N) = 0$. In turn, $\overline{\nu}(A) \leq \nu(N) = 0$. This proves that $\overline{\mu}(A) = 0 \Longrightarrow \overline{\nu}(A) = 0$.  $\blacksquare$\newpage
	\end{myIndent}
\end{itemize}

\hTwo It follows that if $X$ is an LCH space, then given any two positive Radon measures $\mu$ and $\nu$ on $(X, \mathcalli{B}_X)$ with $\nu \ll \mu$ we can extend $\mu$ and $\nu$ to measures $\overline{\mu}$ and $\overline{\nu}$ on a common $\sigma$-algebra $\mathcal{M}$ with precisely the following properties:\\ [-18pt]
\begin{itemize}
	\item $\overline{\mu}$ and $\overline{\nu}$ are both outer regular on all sets and inner regular on all $\sigma$-finite sets;\\ [-18pt]
	\item $\overline{\mu}$ and $\overline{\nu}$ are both finite on all compact sets;\\ [-18pt]
	\item $(X, \mathcal{M}, \overline{\mu})$ is the saturation of the completion of $(X, \mathcalli{B}_X, \mu)$;\\ [-18pt]
	\item $(X, \mathcal{M}, \overline{\mu})$ is decomposable;\\ [-18pt]
	\item $\overline{\nu} \ll \overline{\mu}$.\retTwo
\end{itemize}

Now using the theorems from Hewitt and Stromberg, I want to show that $\overline{\nu} = \overline{f} \df \overline{\mu}$ for some $\overline{\mu}$-measurable function $\overline{f}$. As it turns out, when $\nu$ is $\sigma$-finite (which in turn means $\overline{\nu}$ is $\sigma$-finite), this is always possible:\retTwo

\exTwo\mySepTwo

\ul{Proposition:} Let $(X, \mathcal{M})$ be a measurable LCH space and suppose $\mu$ and $\nu$ are positive measures on $(X, \mathcal{M})$ satisfying that $\nu$ is inner regular on all $\sigma$-finite sets; that $\mu$ is finite on all compact sets; that $\nu \ll \mu$; and that $(X, \mathcal{M}, \mu)$ is decomposable via the decomposition $\mathcalli{F}$. If $\nu$ is $\sigma$-finite, then there exists an $\mathcal{M}$-measurable function $f: X \to [0, \infty)$ which vanishes outside a set on which $\mu$ is $\sigma$-finite such that $\nu(A) = \int_A f \df \mu$ for all $A \in \mathcal{M}$. Furthermore, if $\nu$ is finite, then $f \in L^1(\mu)$. And if $g : X \to [0, \infty)$ is another function satisfying $\nu(A) = \int_A g \df \mu$ for all $A \in \mathcal{M}$, then $f = g$ $\mu$-a.e.

\begin{myIndent}\exThreeP
	Proof:\\
	We shall first suppose that $\nu$ is finite. Then by applying theorem 19.27 on page 156, we know there exists a measurable function $f: X \to [0, \infty)$ satisfying that $\nu(A) = \int_A f \df \mu$ whenever $\mu$ is $\sigma$-finite on $A$. Furthermore, since $\mu(F) < \infty$ for all $F \in \mathcalli{F}$, we thus know that $\nu(F) = \int_F f \df \mu$ for all $F \in \mathcalli{F}$.\retTwo

	Now recall that $\sum_{F \in \mathcalli{F}} \nu(F)$ is by definition equal to the suprememum of all finite sums of the $\nu(F)$. Thus, while we cannot currently guarentee strict equality due to the fact that $\mathcalli{F}$ may be uncountable, we can at least say that $\nu(X) \geq \sum_{F \in \mathcalli{F}}\nu(F)$. And since $\nu(X) < \infty$, this proves that there is a countable subset $\mathcalli{F}_1 \subseteq \mathcalli{F}$ satisfying that $\nu(F) = 0$ for all $F \notin \mathcalli{F}_1$. And clearly $\nu(X) = \nu(\hspace{-0.1em}\bigcup\limits_{F \notin \mathcalli{F}_1}\hspace{-0.1em} F) + \hspace{-0.1em}\sum\limits_{F \in \mathcalli{F}_1} \hspace{-0.1em}\nu(F)$.\retTwo

	Now we claim $\nu(\bigcup_{F \notin \mathcalli{F}_1} F) = 0$. After all, if the set weren't null, then we'd have that\\ $0 < \nu(\bigcup_{F \notin \mathcalli{F}_1} F) \subseteq \nu(X) < \infty$. It would thus follow by the inner regularity of $\nu$ that\\ there is a compact set $K \subseteq \bigcup_{F \notin \mathcalli{F}_1} F$ with $\nu(K) > 0$. But then we'd have that\\ $\mu(K) < \infty$, which means $\nu(K) = \int_K f \df \mu$ and $\mu(K) = \sum_{F \in \mathcalli{F}}\mu(F \cap K)$.\retTwo
	
	Now we can't have that $\mu(K) = 0$ since that contradicts that $\nu \ll \mu$. Hence, since $\mu$ is decomposable there's a nonempty countable family $\mathcalli{F}_2 \subseteq \mathcalli{F}$ such that $\mu(F \cap K) > 0$ for all $F \in \mathcalli{F}_2$ and $\mu(K) = \sum_{F \in \mathcalli{F}_2}\mu(F \cap K)$ (which in turn means $\mu(\bigcup_{F \notin \mathcalli{F}_2} F \cap K) = 0$). Also, since $K$ doesn't intercept any $F \in \mathcalli{F}_1$, we know that $\mathcalli{F}_1 \cap \mathcalli{F}_2 = \emptyset$. But now note that:
	
	{\centering$0 < \nu(K) = \int_K f \df \mu = \sum_{F \in \mathcalli{F}_2} \int_{F \cap K} f \df \mu = \sum_{F \in \mathcalli{F}_2} \nu(F \cap K)$.\newpage\par}

	This implies that there is some $F \in \mathcalli{F}_2$ such that $\nu(F) > \nu(F \cap K) > 0$. But that is a contradiction since we already know that $\nu(F) = 0$ if $F \notin \mathcalli{F}_1$. Hence, we have proven that there is a set $E = \bigcup_{F \in \mathcalli{F}_1} F$ which $\mu$ is $\sigma$-finite on such that $\nu(X) = \nu(E)$. And in turn, we've now proven that $\nu = f\chi_E \df \mu$.
	\begin{myIndent}\exPPP
		And as a side note: $f\chi_E \in L^1(\mu)$ because $\int f \chi_E \df \mu = \nu(X) < \infty$.\retTwo
	\end{myIndent}

	Next, we consider the case where $\nu$ is $\sigma$-finite. Let $(X_n)_{n \in \mathbb{N}}$ be a partition of $X$ consisting of finitely measurable sets. Then by identical reasoning as before, we can find for each $n$ a set $E_n \subseteq X$ on which $\mu$ is $\sigma$-finite and $\nu(X_n \cap A) = \int_A f \chi_{E_n} \df \mu$ for all $A \in \mathcal{M}$. Also, since $E_n \subseteq X_n$ for all $n$, it is clear that they are all disjoint. So, $E \coloneqq \bigcup_{n \in \mathbb{N}} E_n$ satisfies that $\mu$ is $\sigma$-finite on $E$ and $f\chi_E = \sum_{n \in \mathbb{N}} f\chi_{E_n}$. Also, it is clear that $\nu = f\chi_E \df \mu$ since:
	
	{\centering$\nu(A) = \sum_{n \in \mathbb{N}}\nu(X_n \cap A) = \sum_{n \in \mathbb{N}}\int_A f\chi_{E_n} \df \mu = \int_A f \chi_E \df \mu$ for all $A \in \mathcal{M}$.\retTwo\par}
	
	Finally, we show uniqueness. Suppose $g: X \to [0, \infty)$ is another function satisfying that $\nu = g \df \mu$ and let $B = \{x \in X: f(x)\chi_E(x) \neq g(x)\}$. Then it's clear that:
	
	{\centering$B = (B \cap E) \cup (B - E)$.\retTwo\par}
	
	Since $E$ is $\sigma$-finite, we already know from theorem 19.27 that there is a $\mu$-null set $N_1$ with $B \cap E \subseteq N_1$. Also, $B - E = \{x \in X - E : g(x) > 0\}$. So, if $\mu(B - E) > 0$, then we'd have that $\nu(B - E) = \int_{B - E} g \df \mu > 0$. But that contradicts that $\nu(X - E) = 0$. So, we can conclude that $\mu(B - E) = 0$. This proves that $f\chi_E = g$ $\mu$-a.e. $\blacksquare$\retTwo
\end{myIndent}

\mySepTwo

\hTwo To finish making this useful for our purposes we need to find a $\mathcalli{B}_X$-measurable function $f$ such that $\overline{f} = f$ $\overline{\mu}$-a.e. That way, for all $A \in \mathcalli{B}_X$ we have that:

{\centering $\int_A f \df \mu = \int_A f \df \overline{\mu} = \int_A \overline{f} \df \overline{\mu} = \overline{\nu}(A) = \nu(A)$ \retTwo\par}

Fortunately, if we let $(X, \mathcalli{N}, \overline{\mu}|_{\mathcalli{N}})$ be the completion of $(X, \mathcalli{B}_X, \mu)$ (meaning $(X, \mathcal{M}, \overline{\mu})$ is the saturation of $(X, \mathcalli{N}, \overline{\mu}|_{\mathcalli{N}})$), then we know  that $\overline{f}$ is $\mathcalli{N}$-measurable.

\begin{myIndent}\pracTwo
	Proof:\\
	We know $\overline{f}$ vanishes outside a set on which $\overline{\mu}$ is $\sigma$-finite. It follows that if $a < 0$, then $\overline{f}^{-1}((a, \infty)) = X \in \mathcalli{N}$; and if $a \geq 0$, then $\overline{f}^{-1}((a, \infty))$ is $\sigma$-finite. Now we claim that all sets for which $\overline{\mu}$ is $\sigma$-finite are in $\mathcalli{N}$. After all, if $\overline{\mu}$ is $\sigma$-finite on $E$, then we know there is a sequence of sets $(E_n)_{n \in \mathbb{N}}$ whose union is $E$ and which satisfy that $\overline{\mu}(E_n) < \infty$ for all $n$. Then since all the $E_n$ have finite measure, we know that $E_n \in \mathcalli{N}$ for all $n$. So, $E$ is also in $\mathcalli{N}$.\retTwo

	With that, we've shown that $\overline{f}^{-1}((a, \infty))$ is $\mathcalli{N}$-measurable for all $a \in \mathbb{R}$. This proves that $\overline{f}$ is $\mathcalli{N}$-measurable. $\blacksquare$\retTwo
\end{myIndent}

Now, it is a simple application of the proposition on page 46 of my latex math 240a notes to show there exists a $\mathcalli{B}_X$-measurable function $f$ such that $\overline{f} = f$ a.e. with respect to $(X, \mathcalli{N}, \overline{\mu}|_{N_x})$. And since saturating the latter $\sigma$-algebra doesn't add any new null sets, we have that $f = \overline{f}$ $\overline{\mu}$-a.e. Thus, we have found a $\mathcalli{B}_X$-measurable function $f$ such that\\ $\nu = f \df \mu$.\newpage

One more note I want to make is that if $g$ is another $\mathcalli{B}_X$-measurable function such that $\nu = g \df \mu$, then $f = g$ $\mu$-a.e.

\begin{myIndent}\pracTwo
	Proof:\\
	Since $\overline{f}$ vanishes outside a set on which $\overline{\mu}$ is $\sigma$-finite and $\overline{f} = f$ $\overline{\mu}$-a.e., we also know that $f$ vanishes outside a set on which $\overline{\mu}$ is $\sigma$-finite. If we call that set $E$, then we have that $f = f\chi_E$ everywhere. Also note that since $E$ is in $\mathcalli{N}$ on account of $\overline{\mu}$ being $\sigma$-finite on it, we can expand $E$ to a larger set in $\mathcalli{B}_X$ that $\overline{\mu}$ is still $\sigma$-finite on. Hence, we may without loss of generality just say that $E \in \mathcalli{B}_X$. One final note is that if $\overline{\mu}$ is $\sigma$-finite on $E$, then we also know that $\mu$ is $\sigma$-finite on $E$. So, we conclude that $f = f\chi_E$ where $E \in \mathcalli{B}_X$ and $\mu$ is $\sigma$-finite on $E$.\retTwo

	Now let $B \coloneqq \{x \in X: g(x) \neq f(x)\}$. Then we know that $B = (B \cap E) \cup (B - E)$. Our claim is that both of the latter sets are null with respect to $\mu$.
	\begin{myIndent}
		First suppose that $\mu(B - E) > 0$. Then since $B - E = \{x \in X - E : g(x) > 0\}$, we'd have to have that $\nu(B - E) = \int_{B - E} g \df \mu > 0$. But that contradicts that $\nu(B - E) = \int_{B - E} f \df \mu = \int_{B - E} 0 \df \mu = 0$. So, we conclude that $\mu(B - E) = 0$.\retTwo

		Meanwhile, if $B \cap E$ is not a null set, then because $\mu$ is $\sigma$-finite (and thus also\\ semifinite) on $E$, we must be able to pick a set $C \in \mathcalli{B}_X$ such that $0 < \mu(C) < \infty$\\ and $g(x) \neq f(x)$ for all $x \in C$. Then in turn we'd have that:

		{\centering $\int_C |f(x) - g(x)| \df \mu> 0$ \retTwo\par}

		But note that since $\nu$ is $\sigma$-finite, we know there there is a disjoint sequence of\\ sets $\{C_n\}_{n \in \mathbb{N}}$ such that $C = \bigcup_{n \in \mathbb{N}} C_n$ and $\nu(C_n) = \int_{C_n}f \df \mu = \int_{C_n} g \df \mu < \infty$\\ for all $n$. Also, for each $n$ we can define $C_n^+ \coloneqq \{x \in C_n : f(x) > g(x)\}$ and\\ $C_n^- \coloneqq \{x \in C_n : g(x) > f(x)\}$. And thus:
		
		{\center $\int_C |f(x) - g(x)| \df \mu = \sum\limits_{n \in \mathbb{N}} \int_{C_n^+}(f(x) - g(x))\df \mu + \sum\limits_{n \in \mathbb{N}} \int_{C_n^-}(g(x) - f(x))\df \mu$ \retTwo\par}

		But now because $\int_{C_n^\pm} f(x) \df \mu = \int_{C_n^\pm} g(x) \df \mu = \nu(C_n^\pm)$ where $\nu(C_n^\pm)$ is finite, we must have that $\int_C |f(x) - g(x)| \df \mu = 0$ as:
		
		{\center $\int_{C_n^+} (f(x) - g(x)) \df \mu = 0 = \int_{C_n^-} (g(x) - f(x))\df \mu$ for all $n$.\retTwo\par}

		This is a contradiction. So, we've proven that $\mu(B \cap E) = 0$. $\blacksquare$\retTwo
	\end{myIndent}
\end{myIndent}

So to sum all my previous work up, we have the following result:\retTwo

\exTwo\ul{A Third Extension of the Lebesgue-Radon-Nikodym Theorem:}
\begin{myIndent}
	If $X$ is an LCH space and $\mu$ and $\nu$ are positive Radon measures on $(X, \mathcalli{B}_X)$ such that\\ $\nu$ is $\sigma$-finite and $\nu \ll \mu$, then there exists a measurable function $f: X \to [0, \infty)$\\ which vanishes outside of a set where $\mu$ is $\sigma$-finite and which satisfies that $\nu = f \df \mu$.\\ Furthermore, if $g$ is another measurable function satisfying that $\nu = g \df \mu$, then\\ $f = g$ $\mu$-a.e.\retTwo
\end{myIndent}

\hTwo Also, we can clearly extend this theorem to the case where $\nu$ is signed by just applying the theorem to the positive and negative variations separately of $\nu$. And in order to prove uniqueness in this case we can employ a similar strategy as was shown at the top of page 158.\newpage

Going a step further, if $\nu$ is complex, then we can apply the prior reasoning to the real and imaginary variations of $\nu$. Thus, we get the most general result that I will attempt to prove:\retTwo

\exTwo\ul{A Fourth Extension of the Lebesgue-Radon-Nikodym Theorem:}
\begin{myIndent}
	Suppose $X$ is an LCH space and $\mu$ is a Radon measure on $(X, \mathcalli{B}_X)$. If $\nu$ is a complex\\ or $\sigma$-finite signed Radon measure with $\nu \ll \mu$, then there exists a measurable\\ function $f: X \to \mathbb{R}$ (or $\mathbb{C}$) which vanishes outside of a set where $\mu$ is $\sigma$-finite and which satisfies that $\nu = f \df \mu$. Furthermore, if $g$ is another measurable function satisfying that $\nu = g \df \mu$, then $f = g$ $\mu$-a.e.\retTwo
\end{myIndent}

\hTwo I will note that this is still not as general of a statement as Folland was claiming Hewitt and Stromberg were making. Yet it is enough for a specific claim which Folland makes and I will take notes on tomorrow to be true. Anyways, it's been four days since I started going down this rabbithole and I need to grade.

\mySepTwo

\hOne\dispDate{8/30/2025}

\hTwo Let $X$ be an LCH space. Given any fixed positive Radon measure $\mu$, we can isometrically embed $L^1(\mu)$ into $M(X)$ as follows:

\begin{myIndent}\hThree
	Define a map $L^1(\mu) \hookrightarrow M(X)$ such that $f \mapsto f \df \mu$ for all $f \in L^1(\mu)$. By the exercise on page 154, we know that $f \df \mu$ is in fact Radon. Also, we know that the image of this map is precisely the subset of $M(X)$ consisting of all complex measures that are absolutely continuous with respect to $\mu$ (this works even if $\mu$ is not Radon by the rabbithole I went down this past week).\retTwo

	Also, we can easily see that $\| f \df \mu \|_{M(X)} = \|f\|_1$.
	\begin{myIndent}\pracTwo
		I want to be slightly careful about saying this since I'm not assuming $\mu$ is $\sigma$-finite.\retTwo

		Let $\nu = f \df \mu$. Now we know from math 240a that there exists a measurable function\\ [2pt] $g$ with $|g| = 1$ $|\nu|$-a.e. such that $\nu = g \df |\nu|$. Furthermore, we know from the rabbit\\ [2pt] hole before that there exists a nonnegative function $\frac{\df |\nu|}{\df \mu}$ such that $\df |\nu| = \frac{\df |\nu|}{\df \mu} \df \mu$. Hence, we have that $\nu = g \frac{\df |\nu|}{\df \mu} \df \mu$, and this proves that $f = g \frac{\df |\nu|}{\df \mu}$ a.e.. Then in turn we also have that $|f| = \frac{\df |\nu|}{\df \mu}$ and $|\nu| = |f|\df \mu$. This proves that even if $\mu$ is not\\ $\sigma$-finite, we still have that $\nu = f \df \mu \Longrightarrow |\nu| = |f|\df\mu$.\retTwo
	\end{myIndent}
\end{myIndent}

An important application of the above fact is that if $m$ is the Lebesgue measure on $\mathbb{R}^n$, then we can identify $L^1(m)$ as a subspace $M(\mathbb{R}^n)$. This will be used in my journal later.\retTwo

Given a net $\langle \mu_\alpha \rangle_{\alpha \in A}$ and another measure $\mu$ in $M(X)$, we have that $\mu_\alpha \to \mu$ in the \weakst topology on $M(X) = C_0(X)^*$ iff we have that $\int f \df \mu_\alpha \to \int f \df \mu$ for all $f \in C_0(X)$. This topology is important enough to be given a second name: the \udefine{vague topology} on $M(X)$.\newpage

Before covering the next theorem in Folland, I'm gonna do another side quest concerning the Stone-Weierstrass Theorem.\retTwo

\Hstatement\blab{Exercise 4.67:} Let $X$ be a noncompact LCH space. If $\mathcalli{A}$ is a closed subalgebra of $C_0(X, \mathbb{R})$ that separates points, then either $\mathcalli{A} = C_0(X, \mathbb{R})$ or $\mathcalli{A} = \{f \in C_0(X, \mathbb{R}) : f(x_0) = 0\}$ for some $x_0 \in X$.

\begin{myIndent}\HexOne
	Proof:\\
	First suppose there does not exist any $x_0 \in X$ such that $f(x_0) = 0$ for all $f \in \mathcalli{A}$. Then let $Y$ be the Alexandroff (i.e. one-point) compactification of $X$ and $i: C_0(X, \mathbb{R}) \hookrightarrow C(Y, \mathbb{R})$ be the injective map that continuously extends each $f$ to $Y$ by setting $f(\infty) = 0$. Now $i(\mathcalli{A})$ is a subalgebra of $C(Y, \mathbb{R})$ that separates points and for which $x_0 \coloneqq \infty$ satisfies that $f(x_0) = 0$ for all $f \in i(\mathcalli{A})$. So by the Stone Weierstrass theorem we already proved, we know that:\\ [-16pt]

	{\centering $\overline{i(\mathcalli{A})} = \{f \in C(Y, \mathbb{R}) : f(\infty) = 0\}$ \retTwo\par}
	
	But now we claim that $i(\mathcalli{A})$ is closed. For suppose $g \in \overline{i(\mathcalli{A})}$ and  $\{f_n\}_{n \in \mathbb{N}}$ is a sequence in $i(\mathcalli{A})$ converging uniformly to $g$. Then it's clear that $f_n|_X \to g|_X$ uniformly. So $g|_X \in \mathcalli{A}$ since $\mathcalli{A}$ is closed. And in turn $g \in i(\mathcalli{A})$.\retTwo
	
	This shows that $i(\mathcalli{A}) = \overline{i(\mathcalli{A})} = \{f \in C(Y, \mathbb{R}) : f(\infty) = 0\}$. It then follows that\\ $\mathcalli{A} = C_0(X, \mathbb{R})$.\retTwo

	Next suppose there does exist some $x_0 \in X$ such that $f(x_0) = 0$ for all $f \in \mathcalli{A}$. This poses a problem to our previous approach because $i(\mathcalli{A})$ no longer separates points. So to get around this, we first consider the subspace $X^\prime = X - \{x_0\}$. Importantly, $X^\prime$ is still Hausdorff. Also, since $X^\prime$ is an open subset of $X$, we know any $x \in X^\prime$ has a compact neighborhood $N \subseteq X^\prime$. Hence, $X^\prime$ is locally compact.\retTwo

	Next, let $j$ be the map restricting the domain of each $f \in C_0(X, \mathbb{R})$ to $X^\prime$. Then note that\\ [1pt] if $f(x_0) = 0$, then $f|_{X^\prime} \in C_0(X^\prime, \mathbb{R})$. After all, if $\{x \in X : f(x) > \varepsilon\}$ is compact and\\ [1pt] entirely contained in $X^\prime$, then we also know that $\{x \in X^\prime : f|_X(x) > \varepsilon\}$ is compact in\\ [1pt] the subspace topology of $X^\prime$. As a result of this, we know that $j$ is an injective map from\\ [1pt] $\{f \in C_0(X, \mathbb{R}) : f(x_0) = 0\}$ into $C_0(X^\prime, \mathbb{R})$.\retTwo

	Now, it is easily seen that $j(\mathcalli{A})$ is an algebra that separates points and vanishes nowhere. And, it is also seen similarly to earlier that $j(\mathcalli{A})$ is closed. But then by the first case\\ we proved in this exercise, we know that $j(\mathcalli{A}) = C_0(X^\prime, \mathbb{R})$. Hence, it follows that\\ $\mathcalli{A} = \{f \in C_0(X, \mathbb{R}) : f(x_0) = 0\}$. $\blacksquare$\retTwo
\end{myIndent}

\hTwo Now in the next theorem, Folland brought up this exercise because he needs it to prove that $C^1_c(\mathbb{R})$ is dense in $C_0(\mathbb{R})$. However, in my math 240c notes I already proved a stronger statement that $C^\infty_c(\mathbb{R})$ is dense in $C_0(\mathbb{R})$. So was proving this unnecessary? No comment.\retTwo

\exTwo\hypertarget{Folland Proposition 7.19}{\ul{Proposition 7.19:}} Suppose $\mu, \mu_1, \mu_2, \ldots \in M(\mathbb{R})$ and let $F_n(x) = \mu_n((-\infty, x])$ and\\ $F(x) = \mu((-\infty, x])$.\\ [-22pt]
\begin{itemize}
	\item[(a)] If $\sup_{n \in \mathbb{N}}\|\mu_n\| < \infty$ and $F_n(x) \to F(x)$ for every $x$ at which $F$ is continuous, then $\mu_n \to \mu$ vaguely.\newpage
	
	\begin{myIndent}\exThreeP
		Proof:\\
		By Folland theorem 3.29 (check my paper math 240a notes), we know $F$ and each of the $F_n$ are all in \NBV. In turn, this means that $F$ and all the $F_n$ are continuous except at countably many points. So, our assumed conditions actually guarentee that $F_n \to F$ a.e. with respect to the Lebesgue measure.\retTwo

		Now consider any $f \in C^1_c(\mathbb{R})$. By applying theorem 3.36 from math 240a, we can say that $\int f \df \mu = -\int f^\prime F \df x$ and that $\int f \df \mu_n = -\int f^\prime F_n \df x$ for all $n$.
		\begin{myIndent}\exPPP
			Side note: I just realized why theorem 3.36 is a strict generalization of\\ integration by parts as taught in undergrad analysis. If $f$ is a $C^1$ function\\ on $[a, b]$, then we know $f$ is absolutely continuous on $[a, b]$ via the mean\\ value theorem. And it follows then that the measure $f^\prime \df t$ satisfies that\\ $f(x) - f(a) = \int_{(a, x]}f^\prime(t)\df t$ for all $x \in [a, b]$.\retTwo
			
			Next suppose $f$ is $C^1$ everywhere,  $f^\prime \in L^1(\mathbb{R})$, and $f(-\infty) = 0$. Then\\ by taking $a \to -\infty$ and $b \to +\infty$ in the last paragraph, we get that $f^\prime \df t$\\ is a measure satisfying that $f(x) = \int_{(-\infty, x]} f^\prime \df t$.\retTwo

			Thus, so long as $f$ is $C^1$, $f(-\infty) = 0$, and $f^\prime \in L^1$, we know that $f^\prime \df t$\\ is the unique borel measure such that $f(x) = \int_{(-\infty, x]} f^\prime \df t$. And since $f$ is continuous, we can always apply theorem 3.36 to $f$ and any other {\NBV} function.
			\begin{myIndent}\pracTwo
				Now if only I had thought about that while I was actually taking math 240. This is why I failed the qual. AAUGH.\retTwo
			\end{myIndent}
		\end{myIndent}

		But note that $\|F_n\|_u \leq \|\mu_n\|$ for all $n$. So if we set $C = \|f^\prime\|_u \sup_{n \in \mathbb{N}}\|\mu_n\|$, then we can apply dominated convergence theorem using an upper bound of $C\chi_{\supp(f^\prime)}$ to show that $-\int f^\prime F_n \df x \to -\int f^\prime F \df x$ as $n \to \infty$. Hence, $\int f \df \mu_n \to \int f \df \mu$ as $n \to \infty$.\retTwo

		Consequently, we have shown that $\int f \df \mu_n \to \int f \df \mu$ on a dense subset of $C_0(\mathbb{R})$. And so, by proposition 5.17 (see math 240b notes), we know $\int f \df \mu_n \to \int \df \mu$ for all $f \in C_0(\mathbb{R})$. This proves that $\mu_n \to \mu$ vaguely. $\blacksquare$\retTwo
	\end{myIndent}

	\item[(b)] If $\mu_n \to \mu$ vaguely, then $\sup_{n \in \mathbb{N}} \|\mu_n\| < \infty$. If in addition the $\mu_n$ are positive, then $F_n(x) \to F(x)$ at every $x$ at which $F$ is continuous. 

	\begin{myIndent}\exThreeP
		Proof:\\
		By the Riesz Representation theore, we already know that the linear functional\\ $f \mapsto \int f \df \mu_n$ from $C_0(\mathbb{R})$ to $\mathbb{C}$ has the operator norm $\|\mu_n\|$. Also, since $\mu_n \to \mu$\\ vaguely, we know that $\sup_{n \in \mathbb{N}}|\int f \df \mu_n| < \infty$ for all $f \in C_0(\mathbb{R})$. Thus by the uniform\\ boundedness principle, we know that $\sup_{n \in \mathbb{N}}\|\mu_n\| < \infty$.\retTwo

		Next suppose all the $\mu_n$ are positive. Then since $\mu_n \to \mu$ vaguely, we claim that $\mu$ is positive.
		\begin{myIndent}\exPPP
			Note: The following the reasoning works on any general LCH space $X$.\newpage 

			Write $\mu = \mu^{(1)} - \mu^{(2)} + i(\mu^{(3)} - \mu^{(4)})$ where the four latter measures are\\ positive Radon measures. Since $\int f \df \mu_n \to \int f \df \mu$ for all $f \in C_0(X)$ and all the $\mu_n$ are positive, we know that if $f$ is nonnegative and real, then $\int f \df \mu \geq 0$. This is enough to show that $\mu^{(i)} = 0$ unless $i = 1$.\retTwo

			Suppose for the sake of contradiction that $\mu^{(2)} = \alpha > 0$ on some set\\ $A \in \mathcalli{B}_X$. Then by restricting our set $A$ using a Hahn decomposition, we can also say without loss of generality that $\mu^{(1)}(A) = 0$. But now pick any $\varepsilon \in (0, \frac{\alpha}{2})$. By the outer regularity of $\mu^{(1)}$ we know there exists an open set $U \supseteq A$ such that $\mu^{(1)}(U) < \varepsilon$. At the same time, by the inner regularity of $\mu^{(2)}$ we know there exists a compact set $K \subseteq A$ such that $\mu^{(2)}(K) > \alpha - \varepsilon$. And by Urysohn's lemma, we know there is a function $\phi \in C_c(X, [0, 1])$ such that $\phi(K) = \{1\}$ and $\supp(\phi) \subseteq U$. It now follows that:
			
			{\centering\begin{tabular}{l}
				$\rea{\int \phi \df \mu} = \int \phi \df \mu^{(1)} - \int \phi \df \mu^{(2)}$\\
				$\phantom{\rea{\int \phi \df \mu}} \leq \int \chi_U \df \mu^{(1)} - \int \chi_K \df \mu^{(2)} < \varepsilon - (\alpha - \varepsilon) < 0$.
			\end{tabular}\retTwo\par}

			But that contradicts our earlier statement that $\int \phi \df \mu$ is real and nonnegative if $\phi$ is real and nonnegative. So, we conclude no such $A$ exists.\retTwo

			Similar reasoning shows that $\mu^{(3)}$ and $\mu^{(4)}$ are zero.\retTwo
		\end{myIndent}

		Now given any $a$ at which $F$ is continuous, choose any $\varepsilon > 0$ and $N$ such that\\ $-N < a - 2\varepsilon$. Then letting $f \in C_c(\mathbb{R})$ be the function that is $1$ on $[-N, a]$,\\ $0$ on $(-\infty, -N - \varepsilon) \cup (a + \varepsilon, \infty)$, we have that:

		{\centering\begin{tabular}{l}
			$F_n(a) - F_n(-N) = \mu_n((-N, a])$\\ [4pt]
			$\phantom{F_n(a) - F_n(-N)} \leq \int f \df \mu_n \to \int f \df \mu \leq F(a + \varepsilon) - F(-N - \varepsilon)$.
		\end{tabular}\retTwo\par}

		And by taking $N \to \infty$ we thus get that $\limsup_{n \to \infty} F_n(a) \leq F(a + \varepsilon)$.\retTwo

		Meanwhile, letting $g \in C_c(\mathbb{R})$ be the function that is $1$ on $[-N + \varepsilon, a - \varepsilon]$, $0$\\ on $(-\infty, -N] \cup [a, \infty]$, and linear in between, we have that:

		{\centering\begin{tabular}{l}
			$F_n(a) - F_n(-N) = \mu_n((-N, a))$\\ [4pt]
			$\phantom{F_n(a) - F_n(-N)} \geq \int g \df \mu_n \to \int g \df \mu \geq F(a - \varepsilon) - F(-N + \varepsilon)$.
		\end{tabular}\retTwo\par}

		And by taking $N \to \infty$ we thus get that $\liminf_{n \in \infty} F_n(a) \geq F(a - \varepsilon)$.\retTwo

		Now by taking $\varepsilon \to 0$ and using the fact that $F$ is continuous at $a$, we prove that $F_n(a) \to F(a)$ as $n \to \infty$. $\blacksquare$\retTwo
	\end{myIndent}
\end{itemize}

\dispDate{9/1/2025}

\hOne Today I want to clean up my knowledge about product measures by doing some exercises from Folland that I never got around to while I was taking math 240.\newpage

\Hstatement\blab{Exercise 2.11:} If $f : \mathbb{R} \times \mathbb{R}^k \to \mathbb{C}$ satisfies that $f(x, \cdot)$ is measurable on $\mathbb{R}^k$ for all $x \in \mathbb{R}$ and that $f(\cdot, y)$ is continuous on $\mathbb{R}$ for all $y \in \mathbb{R}^k$, then $f$ is Borel measurable on $\mathbb{R} \times \mathbb{R}^k$.

\begin{myIndent}\HexOne
	Proof:\\
	Let $n$ be any positive integer and for all $i \in \mathbb{Z}$, define $a_{i} = \sfrac{i}{n}$. Then define a function $f_n$ on $\mathbb{R} \times \mathbb{R}^k$ by setting for all $y \in \mathbb{R}^k$ and $x \in [a_{i}, a_{i+1}]$:

	\[f_n(x, y) = \frac{f(a_{i+1}, y)(x - a_{i}) - f(a_{i}, y)(x - a_{i+1})}{a_{i+1} - a_{i}}\]

	\phantom{.}\\

	In order to show that each $f_n$ is well defined, note that if $x = a_{i}$ for some $i \in \mathbb{Z}$, then:
	
	{\center$\frac{f(a_{i}, y)(x - a_{i-1}) - f(a_{i-1}, y)(x - a_{i})}{a_{i} - a_{i-1}} = f(a_{i}, y) = \frac{f(a_{i+1}, y)(x - a_{i}) - f(a_{i}, y)(x - a_{i+1})}{a_{i+1} - a_{i}}$\retTwo\par}

	\begin{myTindent}\HexPPP
		You may note that $f_n$ is essentially linearly interpolating because the\\ differents sets $\{a_i\} \times \mathbb{R}^k$ of the domain. And as $n \to \infty$ we are sampling more often.\retTwo
	\end{myTindent}

	We also claim that $f_n$ is measurable. After all, if $E \in \mathcal{M}$, then:

	{\centering $f_n^{-1}(E) = \bigcup_{i \in \mathbb{Z}} (f_n^{-1}(E) \cap ([a_i, a_{i+1}] \times \mathbb{R}^k))$ \retTwo\par}

	But now we know that $f_n$ is measurable on the domain $[a_i, a_{i+1}] \times \mathbb{R}^k$ since it is equal to a sum of products of measurable functions. $g(x, y) \coloneqq f(a_{i+1}, y)$ is measurable as a function from $\mathbb{R} \times \mathbb{R}^k$ because the projection map $(x, y) \mapsto (a_i, y)$ is a continuous map from $\mathbb{R} \times \mathbb{R}^k$ to $\mathbb{R}^k$ and we already assumed that $f(a_i, \cdot)$ is measurable on $\mathbb{R}^k$. Similarly, we can show that $h(x, y) \coloneqq f(a_i, y)$ is measurable. And since $(x - a_i)$ and $(x - a_{i+1})$ are continuous, we also know those parts of the expression for $f_n$ are measurable. And since $[a_i, a_{i+1}] \times \mathbb{R}^k$ is in $\mathcalli{B}_\mathbb{R} \otimes \mathcalli{B}_{\mathbb{R}^k}$ for each $i$, we can now conclude $f_n$ is measurable on its entire domain.\retTwo

	Now in order to show that $f$ is measurable, we claim that $f_n \to f$ pointwise. To prove this, consider any $x \in \mathbb{R}$ and $y \in \mathbb{R}^k$ and let $\varepsilon > 0$. Since $f(\cdot, y)$ is continuous, we know that there exists $\delta > 0$ such that $|f(x^\prime, y) - f(x, y)| < \varepsilon$ for all $x^\prime$ satisfying that $|x^\prime - x| < \delta$. Now suppose $n$ is large enough so that $\sfrac{1}{n} < \sfrac{\delta}{3}$. Then we know there exists $i \in \mathbb{Z}$ such that $x - \delta < a_i \leq x \leq a_{i+1} < x + \delta$. And in turn:

	{\center\begin{tabular}{l}
	$|f_n(x, y) - f(x, y)| = \left|\frac{f(a_{i+1}, y)(x-a_i) - f(a_i, y)(x-a_{i+1})}{a_{i+1} - a_i} - f(x, y)\right|$\\ [14pt]
	$\phantom{|f_n(x, y) - f(x, y)|} = \left|\frac{f(a_{i+1}, y)(x-a_i) - f(a_i, y)(x-a_{i+1})}{a_{i+1} - a_i} - f(x, y)\frac{(x - a_i) - (x - a_{i+1})}{a_{i+1} - a_i}\right|$\\ [14pt]
	$\phantom{|f_n(x, y) - f(x, y)|} = \left|\frac{(f(a_{i+1}, y) - f(x, y))(x-a_i)}{a_{i+1} - a_i} - \frac{(f(a_i, y) - f(x, y))(x-a_{i+1})}{a_{i+1} - a_i}\right|$\\ [14pt]
	$\phantom{|f_n(x, y) - f(x, y)|} = |f(a_{i+1}, y) - f(x, y)||\frac{(x-a_i)}{a_{i+1} - a_i}| + |f(a_i, y) - f(x, y)||\frac{(x-a_{i+1})}{a_{i+1} - a_i}|$\\ [14pt]
	$\phantom{|f_n(x, y) - f(x, y)|} \leq |f(a_{i+1}, y) - f(x, y)|\cdot 1 + |f(a_i, y) - f(x, y)|\cdot 1$\\ [14pt]
	$\phantom{|f_n(x, y) - f(x, y)|} < \varepsilon + \varepsilon = 2\varepsilon$. $\blacksquare$
	\end{tabular}\newpage\par}
\end{myIndent}

As a corollary, if $f: \mathbb{R}^n \to \mathbb{C}$ is continuous in each variable separately, then $f$ is measurable.

\begin{myIndent}\HexOne
	Proof:\\
	Suppose we already proved that if $g: \mathbb{R}^{n-1} \to \mathbb{C}$ is continuous in each variable, then $g$ is Borel measurable. Then consider $f$ as a function from $\mathbb{R} \times \mathbb{R}^{n-1}$ to $\mathbb{C}$. We know by our assumption that $f(\cdot, y)$ is continuous for all $y \in \mathbb{R}^{n-1}$. Also, we know by our inductive hypothesis that $f(x, \cdot)$ is measurable for all $x \in \mathbb{R}$. So by the reasoning in the last exercise, we have that $f$ is Borel measurable on $\mathbb{R} \times \mathbb{R}^{n-1}$.\retTwo
	
	And since $\mathcalli{B}_{\mathbb{R} \times \mathbb{R}^{n-1}} = \mathcalli{B}_{\mathbb{R}^n}$ when identifying $\mathbb{R} \times \mathbb{R}^{n-1}$ with $\mathbb{R}^n$, we've shown that $f$ is measurable. $\blacksquare$
	\begin{myTindent}\myComment\fontsize{11}{13}\selectfont
		I realize I still have not yet properly convinced myself of why\\ $\mathcalli{B}_{\mathbb{R} \times \mathbb{R}^{n-1}} \cong \mathcalli{B}_{\mathbb{R}^n}$. So, I'd like to set that and a few other conerns about products of measures straight tomorrow. But for now I need to study for physics.\retTwo
	\end{myTindent}
\end{myIndent}

\pracOne One more note I'd like to make is that this exercise is significant because we don't generally have that $f: \mathbb{R}^n \to \mathbb{C}$ being continuous with respect to each variable separately implies that $f$ is continuous. So essentially this exercise gives a strictly weaker sufficient condition for $f: \mathbb{R}^n \to \mathbb{C}$ to be Borel measurable than strict continuity. Also, this new condition is way easier to show.\retTwo

\dispDate{9/2/2025}

\hypertarget{Folland Exercise 2.45}{\Hstatement\blab{Exercise 2.45:}} Let $n \geq 3$ and suppose $(X_j, \mathcal{M}_j, \mu_j)$ is a measure space for $j = 1, \ldots, n$, and let us identify $\prod_{j=1}^n X_j$ with $(\prod_{j=1}^{n-1} X_j) \times X_n$  and $X_1 \times (\prod_{j=2}^{n} X_j)$ in the obvious ways.
\begin{itemize}
	\item We have that $\bigotimes_{j=1}^n \mathcal{M}_j = (\bigotimes_{j=1}^{n-1} \mathcal{M}_j) \otimes \mathcal{M}_n$.
	\begin{myIndent}\HexOne 
		Proof:\\ 
		For $i = 1, \ldots, n$ let $\pi_i$ be the projection of $\prod_{j=1}^n X_j$ onto $X_i$. Also let $\pi_{\widehat{n}}$ be the\\ projection of $\prod_{j=1}^n X_j$ onto $\prod_{j=1}^{n-1} X_j$ and for $k=1,\ldots,n-1$ let $\tau_{k}$ be the projection\\ of $\prod_{j=1}^{n-1} X_j$ onto $X_k$.\retTwo

		We know by definition that $\bigotimes_{j=1}^{n} \mathcal{M}_j$ is generated by the collection of sets:

		{\centering $A_1 \coloneqq \{\pi_j^{-1}(E) : j=1,\ldots,n \text{ and } E \in \mathcal{M}_j\}$.\retTwo\par}
		
		Meanwhile, by the proposition at the top of page 13 of my latex math 240a notes,\\ since $\{\tau_k^{-1}(E) : k=1,\ldots,n-1 \text{ and } E \in \mathcal{M}_k\}$ is a base for $\bigotimes_{j=1}^{n-1} \mathcal{M}_j$, we have that $(\bigotimes_{j=1}^{n-1} \mathcal{M}_j) \otimes \mathcal{M}_3$ is generated by the collection of sets:
		
		{\centering $A_2 \coloneqq \{\pi_{\widehat{n}}^{-1}(\tau^{-1}_k(E)) : k=1,\ldots,{n-1} \text{ and } E \in \mathcal{M}_k\} \cup \{\pi_n^{-1}(E) : E \in \mathcal{M}_3\}$.\retTwo\par}

		But now if $E \in \mathcal{M}_1$, we know that:
		
		{\centering$\pi^{-1}_{\widehat{n}}(\tau^{-1}_1(E)) = \pi^{-1}_{\widehat{n}}(E \times X_2 \times \cdots \times X_{n-1}) = E \times X_2 \times \cdots \times X_{n-1} \times X_n$.\retTwo\par}
		
		Repeating this reasoning for all $2 \leq k \leq n-1$ we can in fact see that $A_1 = A_2$. This shows that both of the sigma algebras in the problem are equal.\newpage
	\end{myIndent}

	\item By nearly identical reasoning, we have that $\bigotimes_{j=1}^n \mathcal{M}_j = \mathcal{M}_1 \otimes (\bigotimes_{j=2}^{n} \mathcal{M}_j)$. And note that this is enough to show that the $\bigotimes$ operation is associative and we get the same result no matter how we use parentheses to group together the $\mathcal{M}_j$ (see the top of page 113 of my math journal).\retTwo

	\item If all the $\mu_j$ are $\sigma$-finite, then $\mu_1 \times \cdots \times \mu_n = (\mu_1 \times \cdots \times \mu_{n-1}) \times \mu_n$.
	\begin{myIndent}\HexOne 
		Proof:\\ 
		If all the $\mu_j$ are $\sigma$-finite, then we know that $\mu_1 \times \cdots \times \mu_n$ is the unique measure\\ on $\bigotimes_{j=1}^n \mathcal{M}_j$ satisfying that for any rectangle $E_1 \times \cdots \times E_n$ where $E_k \in \mathcal{M}_k$ for\\ all $k=1,\ldots,n$:

		{\centering $\mu_1 \times \cdots \times \mu_n(E_1 \times \cdots \times E_n) = \mu_1(E_1)\cdots\mu_n(E_n)$ \retTwo\par}

		Meanwhile, we also have that $\mu_1 \times \cdots \times \mu_{n-1}$ is a measure on $\bigotimes_{j=1}^{n-1} \mathcal{M}_j$ satisfying that if $E_k \in \mathcal{M}_k$ for all $k = 1,\ldots,n-1$, then:
		
		{\centering$\mu_1 \times \cdots \times \mu_{n-1}(E_1 \times \cdots \times E_n) = \mu_1(E_1)\cdots\mu_{n-1}(E_{n-1})$\retTwo\par}

		Furthermore, $(\mu_1 \times \cdots \times \mu_{n-1}) \times \mu_n$ is a measure on $\bigotimes_{j=1}^n$ satisfying that if\\ $A \in \bigotimes_{j=1}^{n-1} \mathcal{M}_j$ and $E_n \in \mathcal{M}_n$, then:

		{\centering$((\mu_1 \times \cdots \times \mu_{n-1}) \times \mu_n)(A \times E_n) = \mu_1 \times \cdots \times \mu_{n-1}\mu(A) \cdot \mu_n(E_{n})$\retTwo\par}

		So, if we let $E_k \in \mathcal{M}_k$ for all $k=1,\ldots,n$, then we can clearly see that:
		
		{\centering\begin{tabular}{l}
			$((\mu_1 \times \cdots \times \mu_{n-1}) \times \mu_n)(E_1 \times \cdots \times E_n)$\\ [4pt]
			$\phantom{aaaaaaaaaaaa} = \mu_1 \times \cdots \times \mu_{n-1}(E_1 \times \cdots \times E_{n-1}) \cdot \mu_{n}(E_n)$\\ [4pt]
			$\phantom{aaaaaaaaaaaa} = \mu_1(E_1) \cdots \mu_{n-1}(E_{n-1})\mu_{n}(E_n)$
		\end{tabular}\retTwo\par}

		This shows that both measures in the problem have the same property which uniquely determines $\mu_1 \times \cdots \times \mu_n$. So, both measures must be equal.\retTwo
	\end{myIndent}

	\item By nearly identical reasoning we have that $\mu_1 \times \cdots \times \mu_n = \mu_1 \times (\mu_2 \times \cdots \times \mu_n)$ if all the $\mu_j$ are $\sigma$-finite. Then similarly to two bullet points ago, this is enough to show that the $\times$ operation with respect to $\sigma$-finite measures is associative and we get the same result no matter how we use parentheses to group together the $\mu_j$.
\end{itemize}

\begin{myDindent}\myComment
	I said yesterday that $\mathcalli{B}_{\mathbb{R} \times \mathbb{R}^{n-1}} = \mathcalli{B}_{\mathbb{R}^n}$. To prove this, first note that by the corollary on page 14 of my latex math 240a notes: 
	
	{\centering$\mathcalli{B}_{\mathbb{R} \times \mathbb{R}^{n-1}} = \mathcalli{B}_{\mathbb{R}} \otimes  \mathcalli{B}_{\mathbb{R}^{n-1}} = \mathcalli{B}_\mathbb{R} \times (\bigotimes_{j=1}^{n-1} \mathcalli{B}_\mathbb{R})$ and $\mathcalli{B}_{\mathbb{R}^n} = \bigotimes_{j=1}^n \mathcalli{B}_\mathbb{R}$.\retTwo\par}

	Also, by the last exercise we have that $\mathcalli{B}_\mathbb{R} \times (\bigotimes_{j=1}^{n-1} \mathcalli{B}_\mathbb{R}) = \bigotimes_{j=1}^n \mathcalli{B}_\mathbb{R}$. So hopefully this addresses my worry from yesterday.\retTwo
\end{myDindent}

\hTwo Next order of business: in my paper math notes for math 240a, I copied down a\\ sentence from Folland stating that the completion of $(\mathbb{R}^n, \mathcalli{B}_{\mathbb{R}^n}, m^n)$ and the completion of $(\mathbb{R}^n, \bigotimes_{j=1}^n \mathcal{L}, m^n)$ are equal. I want to show that now by proving something slightly more general.\newpage

\exTwo\ul{Claim:} For each $j=1,\ldots,n$ let $(X_j, \mathcal{M}_j, \mu_j)$ be a $\sigma$-finite measure space and let\\ $(X_j, \overline{\mathcal{M}_j}, \overline{\mu_j})$ be the completion of $(X_j, \mathcal{M}_j, \mu_j)$.
\begin{itemize}
	\item Suppose that $(X, \mathcal{N}, \nu) = (\prod_{j=1}^n X_j, \bigotimes_{j=1}^n \mathcal{M}_j, \prod_{j=1}^n\mu_j)$ and that\\ $(X, \mathcal{N}^\prime, \nu^\prime) = (\prod_{j=1}^n X_j, \bigotimes_{j=1}^n \overline{\mathcal{M}_j}, \prod_{j=1}^n\overline{\mu_j})$. Then $\mathcal{N} \subseteq \mathcal{N}^\prime$ and $\nu^\prime(A) = \nu(A)$\\ [3pt] for all $A \in \mathcal{N}$.
	
	\begin{myIndent}\exThreeP
		Proof:\\
		Since $\mathcal{N}$ is a finite product of $\sigma$-algebras, we know that $\mathcal{N}$ is generated by the\\ collection of sets:

		{\centering $A_1 = \{E_1 \times \cdots \times E_n : E_k \in \mathcal{M}_k \text{ for all } k=1,\ldots,n\}$ \retTwo\par}

		Similarly, we know that $\mathcal{N}^\prime$ is generated by the collection of sets:

		{\centering $A_2 = \{E_1 \times \cdots \times E_n : E_k \in \overline{\mathcal{M}_k} \text{ for all } k=1,\ldots,n\}$ \retTwo\par}

		But clearly $A_1 \subseteq A_2$. So $\mathcal{N} \subseteq \mathcal{N}^\prime$. To prove the other claim, consider the restriction $\nu^\prime|_\mathcal{N}$. Since all the $\mu_j$ are $\sigma$-finite, we know that $\nu$ is the unique measure on $\mathcal{N}$\\ such that $\nu(E_1 \times \cdots \times E_n) = \mu_1(E_1)\cdots \mu_n(E_n)$ for all $E_1 \times \cdots \times E_n \in A^1$. However, we also have that:

		{\centering $\mu_1(E_1)\cdots \mu_n(E_n) = \overline{\mu_1}(E_1)\cdots \overline{\mu_n}(E_n) = \nu^\prime(E_1 \times \cdots \times E_n)$ \retTwo\par}

		So, $\nu = \nu^\prime|_\mathcal{N}$. $\blacksquare$\retTwo
	\end{myIndent}

	\item Let $(X, \overline{\mathcal{N}}, \overline{\nu})$ be the completion of $(X, \mathcal{N}, \nu)$. Then $\mathcal{N}^\prime \subseteq \overline{\mathcal{N}}$ and $\overline{\nu}(A) = \nu^\prime(A)$ for all $A \in \mathcal{N}^\prime$.
	
	\begin{myIndent}\exThreeP
		Proof:\\
		To start off, first note that $\nu^\prime$ is the restriction of the outer measure induced by the\\ function $\pi(E_1 \times \cdots \times E_n) = \overline{\mu_1}(E_1)\cdots \overline{\mu_n}(E_n)$ for all rectangles $E_1 \times \cdots \times E_n$\\ where $E_k \in \overline{\mathcal{M}_k}$ for all $k=1,\ldots,n$.
		\begin{myIndent}\exPPP
			Technically Folland defined the product measure by taking the outer measure\\ of a premeasure $\pi$ defined on the collection $\mathcal{A}$ of all finite disjoint unions of\\ rectangles. However, by just expanding each element of a sequence\\ $\{A_n\}_{n \in \mathbb{N}} \subseteq \mathcal{A}$ into an expression of disjoint rectangles, it's obvious how we\\ can find a sequence $\{B_m\}_{m \in \mathbb{N}}$ consisting only of rectangles such that\\ $\bigcup_{n \in \mathbb{N}}A_n = \bigcup_{m \in \mathbb{N}}B_m$ and $\sum_{n \in \mathbb{N}}\pi(A_n) = \sum_{m \in \mathbb{N}}\pi(B_m)$.\retTwo

			The only reason Folland uses the larger collection $\mathcal{A}$ is that $\mathcal{A}$ is an algebra of sets and so by defining a premeasure on $\mathcal{A}$ we can thus abstract away the outer measure definition and just apply the theorems from chapter 1 of his book. That said, the values that any additive measure takes on $\mathcal{A}$ are clearly determined entirely by the values that the measure takes on the collection of rectangles.\retTwo
		\end{myIndent}

		But now note that if $E_k \in \overline{\mathcal{M}_k}$, then we can pick a set $F_k \in \mathcal{M}_k$ with $E_k \subseteq F_k$ and $\mu_k(F_k) = \overline{\mu_k}(F_k) = \overline{\mu_k}(E_k)$.\newpage
		
		Doing this for all $k$, we have that for any rectangle $E = E_1 \times \cdots \times E_n$ with $E_k \in \overline{\mathcal{M}_k}$ for all $k$, there exists a rectangle $F = F_1 \times \cdots \times F_n$ with $E_k \subseteq F_k$ and $F_k \in \mathcal{M}_k$ for all $k$ and which satisfies that $\nu^\prime(E) = \nu^\prime(F) = \nu(F)$.\retTwo

		As a consequence of all of the above reasoning, if $A \in \mathcal{N}^\prime$ with $\nu^\prime(A) < \infty$, then for each $k \in \mathbb{N}$ we may pick a sequence of rectangles $\{A_m^{(k)}\}_{m \in \mathbb{N}}$ in $\mathcal{N}^\prime$ such that $A \subseteq \bigcup_{m \in \mathbb{N}} A_m^{(k)}$ and $\sum_{m \in \mathbb{N}}\nu^\prime(A_m^{(k)}) < \nu^\prime(A) + \sfrac{1}{k}$. Next, for all $k$ and $n$ we may pick a rectangle $B_m^{(k)} \in \mathcal{N}$ such that $A_m^{(k)} \subseteq B_m^{(k)}$ and $\nu^\prime(B_m^{(k)}) = \nu^\prime(A_m^{(k)})$. And now, if we set $B \coloneqq \bigcap_{k \in \mathbb{N}}\left(\bigcup_{m \in \mathbb{N}} B_m^{(k)}\right)$, we will know that $A \subseteq B$, $B \in \mathcal{N}$, and $\nu(B) = \nu^\prime(B) = \nu^\prime(A)$. Also note that clearly $\nu^\prime(B - A) = 0$.\retTwo

		And this at long last leads us to the following construction for any $A \in \mathcal{N}^\prime$:
		\begin{myIndent}\exPPP
			Let $\{E_i\}_{i \in \mathbb{N}} \subseteq \mathcal{N}$ be a partition of $X$ consisting of sets with finite measure.\\ Then for each $i$ we can pick a set $C_i \in \mathcal{N}$ such that $C_i \supseteq E_i - A$ and\\ $\nu^\prime(C_i - (E_i - A)) = 0$. And in turn, if we let $B_i \coloneqq E_i - C_i$, we will have\\ that $B_i \in \mathcal{N}$, $B_i \subseteq E_i \cap A$, and $\nu^\prime((A \cap E_i) - B_i) = 0$.\retTwo

			Meanwhile, for each $i$ we may also pick $D_i \in \mathcal{N}$ such that $D_i \supseteq A_i \cap E_i$ and $\nu^\prime(D_i - (A_i \cap E_i)) = 0$. Now it's clear that:

			{\centering $\nu(D_i - B_i) = \nu^\prime(D_i - (A \cap E_i)) + \nu^\prime((A \cap E_i) - B_i) = 0$ \retTwo\par}

			Finally, set $B = \bigcup_{i \in \mathbb{N}} B_i$ and $D = \bigcup_{i \in \mathbb{N}}D$. Then $B$ and $D$ are in $\mathcal{N}$,\\ $B \subseteq A \subseteq D$, and $\nu(D - B) = 0$.\retTwo

			Now $A = B \cup (A - B)$ where $B \in \mathcal{N}$ and $A - B \subseteq D - B$ with $D - B$ being a $\nu$-null set in $\mathcal{N}$. From that it's clear that $A \in \overline{\mathcal{N}}$. Also, since we have that $0 \leq \nu^\prime(A - B) \leq \nu^\prime(D - B) = \nu(D - B) = 0$, we know:
			
			{\centering$\overline{\nu}(A) = \nu(B) = \nu^\prime(B) = \nu^\prime(A)$. $\blacksquare$\retTwo\par}
		\end{myIndent}

		\myComment Note from 9/4/2025: Frick I thought of a way cleaner proof for this. I'll write it below.\retTwo

		Suppose $\pi_k$ is the projection of $X$ onto $X_k$ and that $A \in \overline{\mathcal{M}_k}$. Then $A = E \cup F$ where $E \in \mathcal{M}_k$ and $F \subseteq N$ where $N \in \mathcal{M}_k$ and $\mu_k(N) = 0$. It now follows that $\pi_k^{-1}(A) = \pi_k^{-1}(E) \cup \pi_k^{-1}(F)$ where $\pi_k^{-1}(E) \in \mathcal{N}$ and $\pi_k^{-1}(F) \subseteq \pi_k^{-1}(N)$ with $\nu(\pi_k^{-1}(N)) = 0$. Hence, $\pi_k^{-1}(A) \in \overline{\mathcal{N}}$.\retTwo

		Now since the collection $\mathcal{A} = \{\pi_k^{-1}(A) : k=1,\ldots,n \text{ and } A \in \overline{\mathcal{M}_k}\}$ is a base for $\mathcal{N}^\prime$ and $\mathcal{A}$ is a subset of the $\sigma$-algebra $\overline{\mathcal{N}}$, we must have that $\mathcal{N}^\prime \subseteq \overline{\mathcal{N}}$.\retTwo

		Finally, consider that for any $A \in \mathcal{N}^\prime$, if we write $A = E \cup F$ where $E \in \mathcal{N}$ and $F \subseteq N$ with $N \in \mathcal{N}$ and $\nu(N) = 0$, then we know that $\nu^\prime(A - E) = 0$. Hence, $\overline{\nu}(A) = \nu(E)$ and $\nu^\prime(A) = \nu^\prime(E) + \nu^\prime(A - E) = \nu(E) + 0$. So $\overline{\nu}(A) = \nu^\prime(A)$.\newpage
	\end{myIndent}
\end{itemize}

\pracOne\mySepTwo

Now before continuing, we need the following lemmas which I'm shocked I didn't prove during homeworks for math 240a. (To be clear, I thought I had proved this stuff at some point and I've used all of these things in proofs before. But now that I'm looking, I can't actually find a proof written down anywhere in my notes for this stuff.)\retTwo

\ul{Lemma 1:} Suppose $(X, \mathcal{M}, \mu)$ and $(X, \mathcal{N}, \nu)$ are measure spaces such that $\mathcal{M} \subseteq \mathcal{N}$ and $\nu|_\mathcal{M} = \mu$. Then if $(X, \overline{\mathcal{M}}, \overline{\mu})$ and $(X, \overline{\mathcal{N}}, \overline{\nu})$ are the completions of $(X, \mathcal{M}, \mu)$ and $(X, \mathcal{N}, \nu)$ respectively, we have that $\overline{\mathcal{M}} \subseteq \overline{\mathcal{N}}$ and $\overline{\nu}|_{\overline{\mathcal{M}}} = \overline{\mu}$.

\begin{myIndent}\pracTwo
	Proof:\\
	Suppose $A \in \overline{\mathcal{M}}$ and let $E, F, N$ be sets satisfying that $A = E \cup F$; $F \subseteq N$;\\ $E, N \in \mathcal{M}$; and $\mu(N) = 0$. Then we also have that $E, N \in \mathcal{N}$ and that\\ $\nu(N) = \mu(N) = 0$. So, it's clear $A \in \overline{\mathcal{N}}$ and $\overline{\mu}(A) = \mu(E) = \nu(E) = \overline{\nu}(A)$.\\ $\blacksquare$\retTwo
\end{myIndent}

\ul{Lemma 2:} Suppose $(X, \mathcal{M}, \mu)$ is a complete measure space and let $(X, \overline{\mathcal{M}}, \overline{\mu})$ be the\\ completion of $(X, \mathcal{M}, \mu)$. Then $\mathcal{M} = \overline{\mathcal{M}}$ and $\mu = \overline{\mu}$.

\begin{myIndent}\pracTwo
	Proof:\\
	If $A \in \overline{\mathcal{M}}$, then there exists sets $E, F, N$ satisfying that $A = E \cup F$; $F \subseteq N$;\\ $E, N \in \mathcal{M}$; and $\mu(N) = 0$. But now since $\mu$ is complete, we know that $F \in \mathcal{M}$.\\ So, $A = E \cup F$ is also in $\mathcal{M}$, meaning $\overline{\mathcal{M}} \subseteq \mathcal{M}$. And since the other inclusion is\\ obvious, we know that $\overline{\mathcal{M}} = \mathcal{M}$  Also, we thus have that $\overline{\mu} = \overline{\mu}|_\mathcal{M} = \mu$. $\blacksquare$\retTwo
\end{myIndent}

For the sake of nicer notation in the next corollary, let us refer to a given measure space $(X, \mathcal{M}, \mu)$ using a single symbol such as $A$. Also let $c(A)$ denote the completion of $A$. And if $B = (Y, \mathcal{N}, \nu)$ is another measure space, let us write $A \subseteq B$ to mean that $X = Y$, $\mathcal{M} \subseteq \mathcal{N}$, and $\nu|_\mathcal{M} = \mu$.\retTwo

\ul{Theorem:} If $A$ and $B$ are measures spaces such that $A \subseteq B \subseteq c(A)$, then $c(A) = c(B)$.

\begin{myIndent}\pracTwo
	Proof:\\
	We know by lemma 1 that $A \subseteq B \subseteq c(A)$ implies that $c(A) \subseteq c(B) \subseteq c(c(A))$. However, by lemma 2 we know that $c(c(A)) = c(A)$. So, $c(A) \subseteq c(B) \subseteq c(A)$. This is the same as saying that $c(A) = c(B)$. $\blacksquare$\retTwo
\end{myIndent}

\mySepTwo

\exTwo Returning to what we were doing before the prior tangent:
\begin{itemize}
	\item Let $(X, \overline{\mathcal{N}^\prime}, \overline{\nu^\prime})$ be the completion of $(X, \mathcal{N}^\prime, \nu^\prime)$. Then $(X, \overline{\mathcal{N}^\prime}, \overline{\nu^\prime}) = (X, \overline{\mathcal{N}}, \overline{\nu})$.
	
	\begin{myIndent}\exThreeP
		Proof:\\
		We'll continue using the notation in the theorem I showed right above. Let\\ $A = (X, \mathcal{N}, \nu)$ and $B = (X, \mathcal{N}^\prime, \nu^\prime)$. Then note that $c(A) = (X, \overline{\mathcal{N}}, \overline{\nu})$ and\\ $c(B) = (X, \overline{\mathcal{N}^\prime}, \overline{\nu^\prime})$.\newpage

		In the first bullet point of our claim, we showed that $A \subseteq B$. And in the second bullet point of our claim, we showed that $B \subseteq c(A)$. Thus, by applying our theorem right above, we know that $c(B) = c(A)$. $\blacksquare$\retTwo
	\end{myIndent}
\end{itemize}

\myComment As a side note before I clock out today, suppose $A, B, C$ are $\sigma$-finite measure spaces and let us write $AB$ to denote the product measure space of $A$ and $B$. Then we can use the theorems I covered today to say things like:

{\centering$c(c(AB)C) = c(c(c(AB))c(C)) = c(c(AB)c(C)) = c((AB)C) = c(ABC)$\retTwo\par}

You can have more fun playing around with that! The rules of the game are:
\begin{itemize}
	\item $c(A_1\cdots A_n) = c(c(A_1)\cdots c(A_n))$,
	\item multiplication is associative (i.e. $A_1A_2A_3 = (A_1A_2)A_3 = A_1(A_2A_3)$). \retTwo
\end{itemize}


\hOne\dispDate{9/4/2025}

One more thing I want to do before finally moving onto new content is finally prove the reformulation of Fubini's theorem but for complete measures. The theorem statement is in my paper notes but I'll also write it below. Also as a side note: if $f(x, y)$ is a function from $X \times Y$ to $\mathbb{C}$ (or $\overline{\mathbb{R}}$), then Folland denotes $f_x \coloneqq f(x, \cdot)$ and $f^y \coloneqq f(\cdot, y)$. Additionally, if $E \subseteq X \times Y$, then Folland denotes: 

{\centering $E_x \coloneqq \{y \in Y : (x, y) \in E\}$ and $E^y \coloneqq \{x \in X : (x, y) \in E\}$.\retTwo\par}

\exTwo\ul{Theorem 2.39: (The Fubini-Tonelli Theorem for Complete Measures)}
\begin{myIndent}
	Let $(X, \mathcal{M}, \mu)$ and $(Y, \mathcal{N}, \nu)$ be complete $\sigma$-finite measure spaces, and let\\ $(X \times Y, \mathcal{L}, \lambda)$ be the completion of $(X \times Y, \mathcal{M} \otimes \mathcal{N}, \mu \times \nu)$. If $f$ is $\mathcal{L}$-measurable and either (a) $f \geq 0$ or (b) $f \in L^1(\lambda)$, then $f_x$ is $\mathcal{N}$-measurable for a.e. $x$ and $f^y$ is $\mathcal{M}$-measurable for a.e. $y$. Also, in the case that (b) is true, $f_x$ and $f^y$ are integrable for a.e. $x$ and a.e. $y$.\retTwo

	Moreover, $x \mapsto \int f_x \df \nu$ and $y \mapsto \int f^y \df \mu$ are measurable. And in the case that (b) is true they are also integrable.\retTwo

	And finally, in the case of either (a) or (b), we have that:

	$$\int f \df \lambda =  \iint f(x, y) \df \mu(x) \df \nu(y) = \iint f(x, y) \df \nu(y) \df \nu(x)$$
\end{myIndent}

\phantom{.}\\

\Hstatement\blab{Exercise 2.49:} Prove the above theorem.
\begin{myIndent}\HexOne
	\ul{Lemma 1:} If $E \in \mathcal{M} \times \mathcal{N}$ and $\mu \times \nu(E) = 0$, then $\nu(E_x) = \mu(E^y) = 0$ for a.e. $x$ and a.e. $y$.
	\begin{myIndent}\HexTwoP
		Proof:\\ 
		Consider the function $f(x, y) = \chi_E(x, y)$. By the Fubini-Tonelli theorem I already proved in my math 240a notes, we know that:

		{\centering $0 = \mu \times \nu(E) = \int f \df (\mu \times \nu) = \int (\int f_x \df \nu(y))\df \nu(x) = \int( \int f^y \df \mu(x))\df \nu(y)$\retTwo\par}

		So, $\int f_x \df \nu(y) = \nu(E_x) = 0$ for a.e. $x$ and $\int f^y \df \mu(x) = \mu(E^y) = 0$ for a.e. $y$.\newpage
	\end{myIndent}

	\ul{Lemma 2:} If $f$ is $\mathcal{L}$-measurable and $f = 0$ $\lambda$-a.e., then $f_x$ and $f^y$ are integrable for a.e. $x$ and a.e. $y$, and $\int f_x \df \nu = \int f^y \df \mu = 0$ for a.e. $x$ and a.e. $y$.
	\begin{myIndent}\HexTwoP
		Proof:\\ 
		Let $A = \{(x, y) \in X \times Y : f(x, y) \neq 0\}$. Then since $(X \times Y, \mathcal{L}, \lambda)$ is the completion of $(X \times Y, \mathcal{M} \otimes \mathcal{N}, \mu \times \nu)$, we know there exists a set $E \in \mathcal{M} \otimes \mathcal{N}$ such that $\mu \times \nu(E) = 0$ and $A \subseteq E$.\retTwo

		By our last lemma, $\nu(E_x) = 0$ for a.e. $x$ and $\mu(E^y) = 0$ for a.e. $y$. Also, $f_x = \chi_{E_x}$\\ [-1pt] on $(E_x)^\comp$ and $f^y = \chi_{E^y}$ on $(E^y)^\comp$. Hence for a.e. $y$ and a.e. $x$ respectively, we\\ have that $f_x = \chi_{E_x}$ $\nu$-a.e. and $f^y = \chi_{E^y}$ $\mu$-a.e. Because $\mu$ and $\nu$ are complete,\\ we can in turn conclude by exercise 2.10 in my latex math 240a notes that $f_x$ is\\ $\nu$-measurable and $f^y$ is $\mu$-measurable for a.e. $x$ and a.e. $y$ respectively.\retTwo
		
		Also, since $|f_x| = \chi_{E_x}$ on $(E_x)^\comp$ and $|f^y| = \chi_{E^y}$ on $(E^y)^\comp$, and both $|f_x|$ and $|f^y|$\\ are measurable, we know for a.e. $x$ that $\int |f_x| \df \nu(y) = \int \chi_{E_x} \df \nu(y) = \nu(E_x) = 0$.\\ And similarly for a.e. $y$ we know that $\int |f^y| \df \mu(x) = \int \chi_{E^y} \df \mu(x) = \mu(E^y) = 0$.\\ Hence, we know that $f_x$ and $f^y$ are integrable for a.e. $x$ and a.e. $y$, and that\\ $\int f_x \df \nu(y) = 0$ and $\int f^y \df \mu(x) = 0$ for a.e. $x$ and a.e. $y$.\retTwo
	\end{myIndent}

	Now we get to the main part of the proof of our theorem. Let $f$ be any $\mathcal{L}$-measurable function. Then from the proposition on page 46 of my latex math 240a notes, we know that there exists a function $g$ that is $(\mathcal{M} \otimes \mathcal{N})$-measurable such that $f = g$ $\lambda$-a.e. So, consider the identity $f = (f - g) + g$.\retTwo

	 We have that $f - g = 0$ $\lambda$-a.e. So by lemma 2, we know for a.e. $x$ that $(f - g)_x$ is\\ $\nu$-measurable with $\int (f - g)_x \df \nu= 0$. And similarly, we know for a.e. $y$ that $(f - g)^y$\\ is $\mu$-measurable with $\int (f-g)^y \df \mu = 0$.\retTwo

	Next, if $f \geq 0$ then we can without loss of generality take $g \geq 0$. And then by the Fubini-Tonelli theorem for noncompleted product measures, we know that $g_x$ is $\nu$-measurable for a.e. $x$; $g^y$ is $\mu$-measurable for a.e. $y$; $\int g_x \df \nu$ and $\int g^y \df \mu$ are measurable and nonnegative; and $\int g \df (\mu \times \nu) = \iint g_x \df \nu \df \mu = \iint g^y \df \mu \df \nu$.\retTwo

	It then follows that $f_x = (f - g)_x + g_x$ is $\nu$-measurable for a.e. $x$; that\\ $f^y = (f - g)^y + g^y$ is $\mu$-measurable for a.e. $y$; and that $\int f_x \df \nu = \int (f - g)_x \df \nu + \int g_x \df \nu$\\ and $\int f^y \df \mu = \int (f - g)^y \df \mu + \int g \df \mu$ are measurable.\retTwo
	
	Additionally, note that $\int f \df \lambda = \int g \df \lambda = \int g \df (\mu \times \nu)$ since $f = g$ $\lambda$-a.e. And on the other hand we have that:
	\begin{itemize}
		\item $\iint f_x \df \nu \df \mu = \iint (f - g)_x + g_x \df \nu \df \mu = \int (\int (f - g)_x \df \nu) + (\int g_x \df \nu) \df \mu$\\ [6pt]
		$\phantom{\iint f_x \df \nu \df \mu = \iint (f - g)_x + g_x \df \nu \df \mu} = \int 0 + (\int g_x \df \nu) \df \mu$\\ [6pt]
		$\phantom{\iint f_x \df \nu \df \mu = \iint (f - g)_x + g_x \df \nu \df \mu} = \iint g_x \df \nu \df \mu = \int g \df (\mu \times \nu)$\retTwo

		\item $\iint f_y \df \mu \df \nu = \iint (f - g)^y + g^y \df \mu \df \nu = \int (\int (f - g)^y \df \mu) + (\int g^y \df \mu) \df \nu$\\ [6pt]
		$\phantom{\iint f_y \df \mu \df \nu = \iint (f - g)^y + g^y \df \mu \df \nu} = \int 0 + (\int g^y \df \mu) \df \nu$\\ [6pt]
		$\phantom{\iint f_y \df \mu \df \nu = \iint (f - g)^y + g^y \df \mu \df \nu} = \iint g^y \df \mu \df \nu = \int g \df (\mu \times \nu)$\retTwo
	\end{itemize}
	
	This proves the theorem for when $f \geq 0$.\newpage

	The case where $f \in L^1(\lambda)$ is really similar and I'm bored. So I'm going to end the proof here. $\blacksquare$\retTwo
\end{myIndent}

\myComment Something I want to add before moving on is that we can easily extend the Fubini-Tonelli theorem to products of more than two measures by applying the previous things we proved about product measures in the past three days.
\begin{myIndent}
	For example, if $(X_j, \mathcal{M}_j, \mu_j)$ is a $\sigma$-finite measure space for each $j = 1,2,3$, then by considering $\mu_1 \times \mu_2 \times \mu_3 = (\mu_1 \times \mu_2) \times \mu_3$ we can say that:

	{\centering $\int f \df (\mu_1 \times \mu_2 \times \mu_3) = \iint f \df (\mu_1 \times \mu_2) \df \mu_3 = \iint f \df \mu_3 \df (\mu_1 \times \mu_2)$. \retTwo\par}

	Also, going further we have that:
	
	\begin{itemize}
		\item $\int f \df (\mu_1 \times \mu_2) = \iint f \df \mu_1 \df \mu_2 = \iint f \df \mu_2 \df \mu_1$
		\item $\int (\int f \df \mu_3) \df (\mu_1 \times \mu_2) = \iint (\int f \df \mu_3)\df \mu_1 \df\mu _2 = \iint (\int f \df \mu_3) \df \mu_2 \df \mu_1$\retTwo
	\end{itemize}

	Hence we have already shown that:

	{\centering\begin{tabular}{l}
		$\int f \df(\mu_1 \times \mu_2 \times \mu_3) = \iiint f \df\mu_1 \df\mu_2\df\mu_3 = \iiint f \df \mu_2 \df \mu_1 \df \mu_3$\\ [6pt]
		$\phantom{\int f \df(\mu_1 \times \mu_2 \times \mu_3)} = \iiint f \df\mu_3 \df \mu_1 \df \mu_2 = \iiint f \df \mu_3 \df \mu_2 \df \mu_1$ 
	\end{tabular}\retTwo\par}

	Meanwhile, if we identify $\mu_1 \times \mu_2 \times \mu_3$ with $\mu_1 \times (\mu_2 \times \mu_3)$, we can show that:

	{\center\begin{tabular}{l}
		$\int f \df(\mu_1 \times \mu_2 \times \mu_3) = \iiint f \df\mu_1 \df\mu_3\df\mu_2 = \iiint f \df \mu_2 \df \mu_3 \df \mu_1$.
	\end{tabular}\retTwo\par}
\end{myIndent}

\hOne\mySepTwo

At long last I shall start making progress in Folland again.\retTwo

\hTwo\blab{Products of Radon Measures:}\\
Let $X$ and $Y$ be LCH spaces and let $\pi_X$ and $\pi_Y$ denote the projections of $X \times Y$ onto $X$ and $Y$ respectively. I will note the following exercise I did for my math 240b homework:\retTwo

\Hstatement\blab{Exercise 4.59:} The product of finitely many locally compact spaces is locally compact.
\begin{myIndent}\HexOne
	Lemma 1: If $E_\alpha \subseteq X_\alpha$ for all $\alpha \in A$, then the product of the relative topologies of $E_\alpha$ on $\prod_{\alpha \in A}E_\alpha$ (denoted $\mathcal{T}$) is equal to the relative topology of $\prod_{\alpha \in A}E_\alpha$ induced by the product topology on $\prod_{\alpha \in A} X_\alpha$ (denoted $\mathcal{T}^\pprime$).

	\begin{myIndent}\HexTwoP
		($\Longrightarrow$)\\
		Suppose $V \in \mathcal{T}$ and $x \in V$. Then there are sets $V_\alpha$ all open relative to $E_\alpha$ with all but finitely many $V_\alpha$ equal to $E_\alpha$ and $x \in \prod_{\alpha \in A}V_\alpha \subseteq V$. Now if $V_\alpha = E_\alpha$, set $V^\prime_\alpha = X_\alpha$. Otherwise, let $V_\alpha^\prime$ just be any open set in $X_\alpha$ such that $V_\alpha = V_\alpha^\prime \cap E_\alpha$. That way, $U^\prime \coloneq \prod_{\alpha \in A}V_\alpha^\prime$ is open in $\prod_{\alpha \in A}X_\alpha$. And finally, $U \coloneq U^\prime \cap \prod_{\alpha \in A}E_\alpha \in \mathcal{T}^\prime$ with $x \in U \subseteq V$. This proves that $V$ is a union of sets in $\mathcal{T}^\prime$.\retTwo

		($\Longleftarrow$)\\
		Suppose $U \in \mathcal{T}^\prime$ and $x \in U$. Then there exists $U^\prime$ in $\prod_{\alpha \in A}X_\alpha$ which is open and satisfies that $U = U^\prime \cap \prod_{\alpha \in A}E_\alpha$. Next, for all $\alpha \in A$ there are open sets $V_\alpha^\prime$ with all but finitely many $V_\alpha^\prime$ equal to $X_\alpha$ and $x \in \prod_{\alpha \in A}V_\alpha^\prime \subseteq U^\prime$. Finally, for all $\alpha$ set $V_\alpha = V_\alpha^\prime \cap E_\alpha$. Then all $V_\alpha$ are open in $E_\alpha$ relative to $X_\alpha$ and all but finitely many $V_\alpha$ are equal to $E_\alpha$. Also $x \in V \coloneq \prod_{\alpha \in A}V_\alpha$. So $V \in \mathcal{T}$ and $x \in V \subseteq U$. This proves that $U$ is a union of sets in $\mathcal{T}$.\retTwo
	\end{myIndent}

	As a corollary to the above lemma and because of Tychonoff's theorem, if $N_\alpha \subseteq X_\alpha$ is compact for all $\alpha \in A$, then $\prod_{\alpha \in A}N_\alpha$ is compact in $\prod_{\alpha \in A}X_\alpha$.\newpage

	Now we finally use the fact $A$ is finite. Note that because $A$ is finite, if $x \in \prod_{\alpha \in A}X_\alpha$ and we are given the neighborhoods $N_\alpha$ of $\pi_\alpha(x)$ for all $\alpha \in A$, then $\prod_{\alpha \in A}N_\alpha$ is a neighborhood of $x$. Therefore, to get a compact neighborhood of $x \in \prod_{\alpha \in A}X_\alpha$, just take a product of compact neighborhoods of $\pi_\alpha(x)$ for each $\alpha \in A$.\retTwo
\end{myIndent}

\hTwo Also, we know from proposition 4.10 in my math 240b notes that arbitrary products of Hausdorff spaces are Hausdorff. Hence, putting the above exercise and that proposition together, we know that a finite topological product of LCH spaces is an LCH space.\retTwo

\exTwo\ul{Theorem 7.20:}
\begin{itemize}
	\item[(a)] $\mathcalli{B}_X \otimes \mathcalli{B}_Y \subseteq \mathcalli{B}_{X \times Y}$.
	
	\begin{myIndent}\exThreeP
		Proof:\\
		This is just an obvious generalization of the proposition at the bottom of page 13 of my latex math 240a notes.\retTwo
	\end{myIndent}

	\item[(b)] If $X$ and $Y$ are second countable, than $\mathcalli{B}_X \otimes \mathcalli{B}_Y = \mathcalli{B}_{X \times Y}$.
	
	\begin{myIndent}\exThreeP
		Proof:\\
		It suffices to show that $\mathcalli{B}_{X \times Y} \subseteq \mathcalli{B}_X \times \mathcalli{B}_Y$. To do this, let $\mathcal{B}_x$ be a countable basis for $X$ and $\mathcal{B}_y$ be a countable basis for $Y$. Then $\mathcal{B} = \{B_x \times B_y : B_x \in \mathcal{B}_x \text{ and } \mathcal{B}_y\}$ is a countable basis for the product topology on $X \times Y$.
		\begin{myIndent}\exPPP
			It's clear that $\mathcal{B}$ is countable and contains only open sets. Meanwhile, in order to prove that $\mathcal{B}$ is a basis, suppose $(x, y) \in U \subseteq X \times Y$ where $U$ is open. Then we know there is a set $V_1 \times V_2 \subseteq U$ such that $V_1$ is open in $X$; $V_2$ is open in $Y$; and $x \in V_1$ and $y \in V_2$. Next, there are sets $B_1 \in \mathcal{B}_x$ and $B_2 \in \mathcal{B}_y$ such that $x \in B_1 \subseteq V_1$ and $y \in B_2 \subseteq V_2$. Now it's clear that $B_1 \times B_2 \in \mathcal{B}$ and $(x, y) \in B_1 \times B_2 \subseteq V_1 \times V_2 \subseteq U$.\retTwo
		\end{myIndent}

		Now since every open set in $X \times Y$ is a countable union of sets in $\mathcal{B}$, we know that $\mathcal{T} \subseteq \mathcal{M}(\mathcal{B})$ where $\mathcal{T}$ is the topology on $X \times Y$. It follows that $\mathcalli{B}_{X \times Y} \subseteq \mathcal{M}(\mathcal{B})$. But at the same time note that $\mathcal{B}$ is a subset of the entire collection of products of open sets, and we know by the proposition at the top of page 13 of my latex math 240a notes that the latter collection generates $\mathcalli{B}_X \otimes \mathcalli{B}_Y$. So, $\mathcalli{B}_{X \times Y} \subseteq \mathcal{M}(\mathcal{B}) \subseteq \mathcalli{B}_X \times \mathcalli{B}_Y$. $\blacksquare$\retTwo
	\end{myIndent}

	\item[\hypertarget{Folland Theorem 7.20c}{(c)}] If $X$ and $Y$ are second countable and $\mu$ and $\nu$ are Radon measures on $X$ and $Y$, then $\mu \times \nu$ is a Radon measure on $X \times Y$.
	
	\begin{myIndent}\exThreeP
		Proof:\\
		Since we showed in part (b) that $\mu \times \nu$ is a Borel measure on $X \times Y$ and that $X \times Y$ is second countable, we know by Folland theorem 7.8 (see my math 240c notes) that it suffices to show that $\mu \times \nu$ is finite on any compact subset $K$ of $X \times Y$ in order to show that $\mu \times \nu$ is regular and thus Radon. But fortunately since $\pi_X$ and $\pi_Y$ are continuous, we know that $\pi_X(K)$ and $\pi_Y(K)$ are compact subsets in $X$ and $Y$ respectively. And since both $\mu$ and $\nu$ are Radon, we know that $\mu(\pi_X(K)) < \infty$ and $\nu(\pi_Y(K)) < \infty$. Also, $K \subseteq \pi_1(K) \times \pi_2(K)$. So:
		
		{\centering$(\mu \times \nu)(K) \leq (\mu \times \nu)(\pi_1(K) \times \pi_2(K)) = \mu(\pi_1(K))\nu(\pi_2(K)) < \infty$. $\blacksquare$\newpage\par}
	\end{myIndent}
\end{itemize}

\hTwo Given functions $g:X \to \mathbb{C}$ and $h: X \to \mathbb{C}$, we define $g \otimes h(x, y) \coloneqq g(x)h(y)$.\retTwo

\exTwo\ul{Proposition 7.21:} Let $\mathcalli{P}$ be the vector space spanned by the functions $g \otimes h$ with\\ $g \in C_c(X)$ and $h \in C_c(Y)$. Then $\mathcalli{P}$ is dense in $C_c(X \times Y)$ in the uniform norm. More\\ precisely, given $f \in C_c(X \times Y)$, $\varepsilon > 0$, and precompact open sets $U \subseteq X$ and $V \subseteq Y$\\ containing $\pi_X(\supp(f))$ and $\pi_Y(\supp(f))$, there exists $F \in \mathcalli{P}$ such that $\|F - f\|_u < \varepsilon$\\ and $\supp(F) \subseteq U \times V$.

\begin{myIndent}\exThreeP
	Proof:\\
	$\overline{U} \times \overline{V}$ is a compact Hausdorff space. Also, we clearly have that the linear span $\mathcalli{A}$ of $\{g \otimes h : g \in C(\overline{U}) \text{ and } h \in C(\overline{V})\}$ is a subalgebra of $C(\overline{U}\times\overline{V})$ which contains all the constant functions and is closed under complex conjugation. We can also see that $\mathcalli{A}$ separates points as follows:
	\begin{myIndent}\exPPP
		Suppose $(x_1, y_1)$ and $(x_2, y_2)$ be distinct points in $\overline{U} \times \overline{V}$. Then we know that either $x_1 \neq x_2$ or $y_1 \neq y_2$. I'll focus on the case that $x_1 \neq x_2$ since the other case is basically identical. We know that $\overline{U}$ is normal (since all compact Hausdorff spaces are normal). So, by Urysohn's lemma there exists a function $g \in C(\overline{U})$ such that $g(x_1) = 1$ and $g(x_2) = 0$. And now if we just set $h(y) = 1 \in C(\overline{V})$, we have that $g \otimes h \in C(\overline{U} \times \overline{V})$ with $g \otimes h(x_1, y_1) = 1$ and $g \otimes h(x_2, y_2) = 0$.\retTwo
	\end{myIndent}

	Thus by the Stone-Weierstrass theorem, we know that $\mathcalli{A}$ is dense in $C(\overline{U} \times \overline{V})$. In\\ particular, this means that there is an element $G \in \mathcalli{A}$ with $\sup_{(x,y) \in \overline{U} \times \overline{V}}|G - f| < \varepsilon$.\retTwo

	Meanwhile, by Urysohn's lemma we know there exists functions $\phi \in C_c(U, [0, 1])$ and $\psi \in C_c(V, [0, 1])$ such that $\phi = 1$ on $\pi_X(\supp(f))$ and $\psi = 1$ on $\pi_Y(\supp(f))$. Thus if we define $F = (\phi \otimes \psi)G$ on $\overline{U} \times \overline{V}$ and $F = 0$ elsewhere, we have that $F \in \mathcalli{P}$, $\|F - f\|_u < \varepsilon$, and $\supp(F) \subseteq U \times V$. $\blacksquare$\retTwo
\end{myIndent}

\ul{Proposition 7.22:} Every $f \in C_c(X, Y)$ is $(\mathcalli{B}_X \otimes \mathcalli{B}_Y)$-measurable. Moreover, if $\mu$ and $\nu$ are Radon measures on $X$ and $Y$, then $C_c(X \times Y) \subseteq L^1(\mu \times \nu)$ and:\\ [-25pt]

$$\int f \df(\mu \times \nu) = \iint f \df \mu \df \nu = \iint f \df \nu \df \mu \text{ for all } f \in C_c(X \times Y)\text{.}$$

\phantom{.}\\ [-14pt]
\begin{myIndent}\exThreeP 
	Proof:\\
	If $g \in C_c(X)$ and $h \in C_c(Y)$, then $g \otimes h = (g \circ \pi_X)(h \circ \pi_Y)$ is $(\mathcalli{B}_X \otimes \mathcalli{B}_Y)$-measurable. Hence since products, sums, and pointwise limits of measurable function are measurable, we know by the last proposition that every $f \in C_c(X, Y)$ is $(\mathcalli{B}_X, \mathcalli{B}_Y)$-measurable.\retTwo

	Also, every $f \in C_c(X, Y)$ is bounded and supported in a set of finite $(\mu \times \nu)$-measure.
	\begin{myIndent}\exPPP
		Specifically consider the set $K \coloneqq \pi_X(\supp(f)) \times \pi_Y(\supp(f)) \in X \times Y$. Then $\supp(f) \subseteq K$, $K$ is compact, and $\mu \times \nu(K) = \mu(\pi_X(\supp(f)))\nu(\pi_Y(\supp(f)))$. Also, the latter expression is finite since $\pi_X(\supp(f))$ and $\pi_Y(\supp(f))$ are compact and $\mu$ and $\nu$ are Radon.\retTwo
	\end{myIndent}

	This proves that every $f \in C_c(X, Y)$ is also in $L^1(\mu \times \nu)$. Finally, even if $\mu$ and $\nu$ are not $\sigma$-finite, we can still show that Fubini's theorem holds. Specifically, consider letting $U$ and $V$ be precompact open sets in $X$ and $Y$ respectively such that $\pi_X^{-1}(\supp(f)) \subseteq U$ and $\pi_Y^{-1}(\supp(f)) \subseteq V$. Then set $\mu^\prime(E) \coloneqq \mu(E \cap U)$ and $\nu^\prime(E) \coloneqq \nu(E \cap V)$.\newpage

	Now it's clear that:
	
	\begin{itemize}
		\item $\int f \df \mu^\prime = \int f \chi_{U} \df \mu = \int f \df \mu$,
		
		\item $\int (\int f \df \mu^\prime) \df \nu^\prime = \int (\int f \df \mu) \df \nu^\prime = \int (\int f \df \mu) \chi_{V} \df \nu = \int (\int f \df \mu) \df \nu$,\\ [-4pt]
		
		\item $\int f \df \nu^\prime = \int f \chi_{V} \df \nu = \int f \df \nu$,
		
		\item $\int (\int f \df \nu^\prime) \df \mu^\prime = \int (\int f \df \nu) \df \mu^\prime = \int (\int f \df \nu) \chi_{U} \df \mu = \int (\int f \df \nu) \df \mu$.\retTwo
	\end{itemize}

	Also, we can see as follows that $\int f \df(\mu \times \nu) = \int f \df(\mu^\prime \times \nu^\prime)$\dots
	\begin{myIndent}\exPPP
		Note that if $g \in L^1(\mu)$ and $h \in L^1(\nu)$, then we automatically have that $g \in L^1(\mu^\prime)$, $h \in L^1(\nu^\prime)$, $\int g \df \mu^\prime = \int g \chi_U \df \mu$, and $\int h \df \nu^\prime = \int h \chi_V \df \nu$. Additionally, by applying exercise 2.51 from my math 240a homework, we know that:
		
		{\center\begin{tabular}{l}
			$\int (g \chi_U)\otimes(h\chi_V)\df (\mu \times \nu) = (\int g \chi_U \df \mu)(\int h \chi_V \df \nu)$\\ [6pt]
			
			$\phantom{\int (g \chi_U)\otimes(h\chi_V)\df (\mu \times \nu)} = (\int g \df \mu^\prime)(\int h  \df \nu^\prime) = \int (g \otimes h) \df (\mu^\prime \times \nu^\prime)$.
		\end{tabular} \retTwo\par}

		It then follows that if $\mathcalli{P}$ is as in the last proposition, $F \in \mathcalli{P}$, and $\supp(F) \subseteq U \times V$, then we know that $\int F \df (\mu^\prime \times \nu^\prime) = \int F \df (\mu \times \nu)$. However, we also showed in the last proposition that we can find a sequence $(F_n)_{n \in \mathbb{N}}$ of such functions such that $F_n \to f$ uniformly. Since $\mu \times \nu(U \times V) = \mu^\prime \times \nu^\prime(U \times V) < \infty$, we can thus conclude via the dominated convergence theorem (using an upper bound of $f + \chi_{U \times V})$ that:

		{\center$\int f \df (\mu \times \nu) = \lim\limits_{n \to \infty} \int F_n \df (\mu \times \nu) = \lim\limits_{n \to \infty} \int F_n \df (\mu^\prime \times \nu^\prime) = \int f \df (\mu^\prime \times \nu^\prime)$ \retTwo\par}
	\end{myIndent}

	A consequence of this is that since $f \in L^1(\mu \times \nu)$, we now know that $f \in L^1(\mu^\prime \times \nu^\prime)$. In turn, since both $\mu^\prime$ and $\nu^\prime$ are finite measures, we can apply the Fubini-Tonelli theorem to see that $\int f \df (\mu^\prime \times \nu^\prime) = \iint f \df \mu^\prime \df \nu^\prime = \iint f \df \nu^\prime \df \mu^\prime$. And now we're done since we already showed earlier that $\int f \df (\mu^\prime \times \nu^\prime) = \int f \df (\mu \times \nu)$, $\iint f \df \mu^\prime \df \nu^\prime = \iint f \df \mu \df \nu$, and $\iint f \df \nu^\prime \df \mu^\prime = \iint f \df \nu \df \mu$. $\blacksquare$\retTwo
\end{myIndent}

\hTwo The significance of the last proposition is that we now know that if $\mu$ and $\nu$ are positive\\ Radon measures on $X$ and $Y$ respectively, then $I(f) = \int f \df (\mu \times \nu)$ is a well-defined\\ positive linear functional from $C_c(X \times Y)$ to $\mathbb{C}$. Hence, there exists a unique well- \\defined positive Radon measure (which we denote $\mu \radtimes \nu$) on $\mathcalli{B}_{X \times Y}$ such that\\ $\int f \df (\mu \radtimes \nu) = \int f (\mu \times \nu)$ for all $f \in C_c(X \times Y)$. We call $\mu \radtimes \nu$ the \udefine{Radon product}\\ of $\mu$ and $\nu$.\retTwo

I'm going to stop taking notes on this section of Folland for the time being cause I don't think Radon products will be necessary for anything I want to study in Folland for a while. Next time, I will be studying some Fourier analysis from Folland. Also, I'm going to finally add in-pdf hyperlinks to this document.\retTwo

\dispDate{9/7/2025}

Let $M(\mathbb{R}^n)$ be the space of complex Borel Radon measures on $\mathbb{R}^n$. Then given any\\ $\mu, \nu \in M(\mathbb{R}^n)$, we define $\mu \times \nu \in M(\mathbb{R}^n \times \mathbb{R}^n)$ by:

{\centering $\df (\mu \times \nu)(x, y) \coloneqq \frac{\df \mu}{\df |\mu|}(x) \frac{\df \nu}{\df |\nu|}(y) \df(|\mu| \times |\nu|)(x, y)$ \newpage\par}

\begin{myIndent}\hThree
	It's clear that $\mu \times \nu$ is a well-defined Radon measure on $M(\mathbb{R}^n \times \mathbb{R}^n)$ by \inLinkRap{Folland prop 7.16}{proposition 7.16}, \inLinkRap{Folland Theorem 7.20c}{theorem 7.20(c)}, and the \inLinkRap{Folland exercise 7.8 corollary}{corollary to exercise 7.8}. Also, this definition doesn't conflict with our previous definition for product measures because if $\mu$ and $\nu$ are positive measures, then $|\mu| = \mu$ and $|\nu| = \nu$. So $\frac{\df \mu}{\df |\mu|}(x) \frac{\df \nu}{\df |\nu|}(y) = 1$ and $\df(|\mu| \times |\nu|) = \df (\mu \times \nu)$.

	\begin{myIndent}\myComment
		Note: we need this new definition of product measures because we want to be able to take products of complex measures rather than  just positive measures.\retTwo

		Also, this definition can be used to define products for complex measures in general (as opposed to just on $(\mathbb{R}^n, \mathcalli{B}_{\mathbb{R}^n})$).\retTwo
	\end{myIndent}
\end{myIndent}

\Hstatement\mySepTwo

\blab{Exercise 3.12:} For $j = 1,2$, let $\mu_j, \nu_j$ be $\sigma$-finite positive measures on $(X_j, \mathcal{M}_j)$ such that\\ $\nu_j \ll \mu_j$. Then $\nu_1 \times \nu_1 \ll \mu_1 \times \mu_2$ and:

$$\frac{\df (\nu_1 \times \nu_2)}{\df (\mu_1 \times \mu_2)}(x_1, x_2) = \frac{\df \nu_1}{\df \mu_1}(x_1) \frac{\df \nu_2}{\df \mu_2} (x_2)$$

\phantom{.}\\
\begin{myIndent}\HexOne
	Proof:\\
	Note that for all $E \in \mathcal{M}_1 \otimes \mathcal{M}_2$, since $\frac{\df \nu_j}{\df \mu_j} \geq 0$ for $j=1,2$ and $\chi_E \geq 0$, we have by\\ [-2pt] Tonelli's theorem that:

	{\center\begin{tabular}{l}
		$\nu_1 \times \nu_2(E) = \int \chi_E \df (\nu_1 \times \nu_2) = \iint \chi_E(x_1, x_2) \df \nu_1(x_1) \df \nu_2(x_2)$\\ [6pt]
		$\phantom{\nu_1 \times \nu_2(E) = \int \chi_E \df (\nu_1 \times \nu_2)} = \iint \chi_E(x_1, x_2)\frac{\df \nu_1}{\df \mu_1}(x_1) \df \mu_1(x_1) \frac{\df \nu_2}{\df \mu_2}(x_2)\df \mu_2(x_2)$\\ [6pt]
		$\phantom{\nu_1 \times \nu_2(E) = \int \chi_E \df (\nu_1 \times \nu_2)} = \int_E \frac{\df \nu_1}{\df \mu_1}(x_1)\frac{\df \nu_2}{\df \mu_2}(x_2) \df (\mu_1 \times \mu_2)(x_1, x_2)$.
	\end{tabular}\retTwo\par}

	This proves that $\nu_1 \times \nu_2 \ll \mu_1 \times \mu_2$ and that $\frac{\df (\nu_1 \times \nu_2)}{\df (\mu_1 \times \mu_2)}(x_1, x_2) = \frac{\df \nu_1}{\df \mu_1}(x_1) \frac{\df \nu_2}{\df \mu_2} (x_2)$\\ [-2pt] $(\mu_1 \times \mu_2)$-a.e. $\blacksquare$\retTwo
\end{myIndent}

\blab{Extension (not in Folland):} For $j = 1,2$ let $\mu_j$ be a $\sigma$-finite measure on $(X_j, \mathcal{M}_j)$ and $\nu_j$ be a complex measure on $(X_j, \mathcal{M}_j)$ such that $\nu_j \ll \mu_j$. Then $\nu_1 \times \nu_2 \ll \mu_1 \times \mu_2$ and:

$$\frac{\df (\nu_1 \times \nu_2)}{\df (\mu_1 \times \mu_2)}(x_1, x_2) = \frac{\df \nu_1}{\df \mu_1}(x_1) \frac{\df \nu_2}{\df \mu_2} (x_2)$$

\phantom{.}\\
\begin{myIndent}\HexOne
	Proof:\\
	Starting off, we know that $\left|\frac{\df \nu_1}{\df |\nu_1|}(x_1)\frac{\df \nu_2}{\df |\nu_2|}(x_2)\right| = 1$ for $(|\nu_1| \times |\nu_2|)$-a.e. $(x_1, x_2)$.\retTwo

	Thus since $|\nu_1|$ and $|\nu_2|$ are finite measures, we know $\frac{\df \nu_1}{\df |\nu_1|}(x_1)\frac{\df \nu_2}{\df |\nu_2|}(x_2) \in L^1(|\nu_1| \times |\nu_2|)$.\\ [-2pt] Hence, we can apply Fubini's theorem to say that:

	{\center\begin{tabular}{l}
		$\nu_1 \times \nu_2(E) = \int_E \frac{\df \nu_1}{\df |\nu_1|}(x_1)\frac{\df \nu_2}{\df |\nu_2|}(x_2) \df(|\nu_1| \times |\nu_2|)(x_1, x_2)$\\ [12pt]

		$\phantom{\nu_1 \times \nu_2(E)} = \iint \chi_E \frac{\df \nu_1}{\df |\nu_1|}(x_1)\frac{\df \nu_2}{\df |\nu_2|}(x_2) \df |\nu_1|(x_1) \df |\nu_2|(x_2)$\\ [12pt]
	\end{tabular}\retTwo\par}

	Next, since $\nu_j \ll \mu_j$ for $j=1,2$ we know that $|\nu_j| \ll \mu_j$ for all $j = 1,2$. Thus for $j = 1,2$ there exists $\frac{\df \nu_j}{\df |\nu_j|}$. And so we may write:\newpage

	{\center\begin{tabular}{l}
		$\iint \chi_E \frac{\df \nu_1}{\df |\nu_1|}(x_1)\frac{\df \nu_2}{\df |\nu_2|}(x_2) \df |\nu_1|(x_1) \df |\nu_2|(x_2)$\\ [12pt]

		$\phantom{aaaaaaaaaaaa} = \iint \chi_E \frac{\df \nu_1}{\df |\nu_1|}(x_1)\frac{\df |\nu_1|}{\df \mu_1}(x_1) \cdot \frac{\df \nu_2}{\df |\nu_2|}(x_2)\frac{\df |\nu_2|}{\df \mu_2}(x_2) \df \mu_1(x_1) \df \mu_2(x_2)$\\ [12pt]

		$\phantom{aaaaaaaaaaaa} = \iint \chi_E \frac{\df \nu_1}{\df \mu_1}(x_1) \cdot \frac{\df \nu_2}{\df \mu_2}(x_2) \df \mu_1(x_1) \df \mu_2(x_2)$
	\end{tabular}\retTwo\par}

	Finally, $\frac{\df \nu_j}{\df \mu_j}\in L^1(\mu_j)$ for $j=1,2$ since $|\nu_j|$ is a finite measure. Thus by Tonelli's theorem we have that:\\ [-8pt]

	{\centering\begin{tabular}{l}
		$\int | \frac{\df \nu_1}{\df \mu_1}(x_1) \frac{\df \nu_2}{\df \mu_2}(x_2)| \df (\mu_1 \times \mu_2)(x_1, x_2)$\\ [8pt]
		$\phantom{aaaaaaaaaaaaaa} = \iint| \frac{\df \nu_1}{\df \mu_1}(x_1) \frac{\df \nu_2}{\df \mu_2}(x_2)| \df \mu_1(x_1) \df \mu_2(x_2)$\\ [8pt]
		$\phantom{aaaaaaaaaaaaaa} = \left[\int |\frac{\df \nu_1}{\df \mu_1}(x_1)| \df \mu_1(x_1)\right] \cdot \left[\int |\frac{\df \nu_2}{\df \mu_2}(x_2)|\df \mu_2(x_2)\right] < \infty$
	\end{tabular}\retTwo\par}

	Thus, $\frac{\df \nu_1}{\df \mu_1}(x_1) \cdot \frac{\df \nu_2}{\df \mu_2}(x_2) \in L^1(\mu_1 \times \mu_2)$ and we can conclude by Fubini's theorem that:\\ [-8pt]

	{\centering\begin{tabular}{l}
		$\iint \chi_E \frac{\df \nu_1}{\df \mu_1}(x_1) \frac{\df \nu_2}{\df \mu_2}(x_2) \df \mu_1(x_1) \df \mu_2(x_2) = \int_E \frac{\df \nu_1}{\df \mu_1}(x_1) \frac{\df \nu_2}{\df \mu_2}(x_2) \df (\mu_1 \times \mu_2)(x_1, x_2)$.
	\end{tabular}\retTwo\par}

	So, we've proven that $\nu_1 \times \nu_2(E) = \int_E \frac{\df \nu_1}{\df \mu_1}(x_1) \frac{\df \nu_2}{\df \mu_2}(x_2) \df (\mu_1 \times \mu_2)(x_1, x_2)$. And from there all the conclusions we were trying to prove before are obvious. $\blacksquare$\retTwo
\end{myIndent}

\myComment\fontsize{12}{13} One more note I'd like to make is that if for $j=1,2$ we have that $\nu_j$ is a complex measure $(X_j, \mathcal{M}_j)$, then $|\nu_1 \times \nu_2| = |\nu_1| \times |\nu_2|$. This is easily seen from the fact that:

{\center\begin{tabular}{l}
	$\nu_1 \times \nu_2 = \frac{\df \nu_1}{\df |\nu_1|}(x) \frac{\df \nu_2}{\df |\nu_2|}(y) \df(|\nu_1| \times |\nu_2|)(x, y)$\\ [8pt]
	
	$\phantom{aaaaaaaaa}\Longrightarrow |\nu_1 \times \nu_2| = \left| \frac{\df \nu_1}{\df |\nu_1|}(x) \frac{\df \nu_2}{\df |\nu_2|}(y) \right| \df(|\nu_1| \times |\nu_2|)(x, y) = 1 \cdot \df(|\nu_1| \times |\nu_2|)$
\end{tabular} \retTwo\par}

Also, since $\left|\frac{\df \nu_j}{\df |\nu_j|}\right| = 1$ a.e. we know $f \in L^1(|\nu_1| \times |\nu_2|)$ iff $f \cdot \frac{\df \nu_1}{\df |\nu_1|} \cdot \frac{\df \nu_2}{\df |\nu_2|} \in L^1(|\nu_1| \times |\nu_2|)$. And when the latter is true then we can easily see via the Fubini-Tonelli theorem that:

{\centering$\int f \df \nu_1 \times \nu_2 = \iint f \df \nu_1 \df \nu_2 = \iint f \df \nu_2 \df \nu_1$.\retTwo\par}

And since $|\nu_1 \times \nu_2| = |\nu_1| \times |\nu_2|$ is finite, it's clear that any bounded $f$ is in $L^1(\nu_1 \times \nu_2)$.\retTwo

\Hstatement\mySepTwo

\hTwo If $\mu, \nu \in M(\mathbb{R}^n)$, we define $\mu * \nu \in M(\mathbb{R}^n)$ by $\mu * \nu(E) \coloneqq \mu \times \nu(\alpha^{-1}(E))$ where $\alpha$ is addition. (Literally. $\alpha: \mathbb{R}^n \times \mathbb{R}^n \to \mathbb{R}^n$ is the map $\alpha(x, y) = x + y$.) We call $\mu * \nu$ the \udefine{convolution} of $\mu$ and $\nu$. Also note that $\alpha^{-1}(E) = \{(x, y) \in \mathbb{R}^n \times \mathbb{R}^n : x + y \in E\}$. Thus $\chi_{\alpha^{-1}(E)}(x, y) = \chi_E(x + y)$ and so:\\ [-10pt]

{\centering\begin{tabular}{l}
	$\mu * \nu(E) = \mu \times \nu(\alpha^{-1}(E)) = \int \chi_{\alpha^{-1}(E)}(x, y)\df(\mu \times \nu)(x, y)$\\ [6pt]
	$\phantom{\mu * \nu(E) = \mu \times \nu(\alpha^{-1}(E))} = \int \chi_E(x + y) \df (\mu \times \nu)(x, y)$ 
\end{tabular}\retTwo\par}

\begin{myIndent}\pracOne
	Question: How do we know that $\mu * \nu$ is a well-defined measure in $M(\mathbb{R}^n)$?\retTwo
	
	\pracTwo To start off, if $E \in \mathcalli{B}_{\mathbb{R}^n}$ then we know $\alpha^{-1}(E) \in \mathcalli{B}_{\mathbb{R}^n \times \mathbb{R}^n}$. because $\alpha$ is continuous. Next, we clearly have that $\chi_{\alpha^{-1}(E)} \in L^1(|\mu \times \nu|) = L^1(|\mu| \times |\nu|)$ since $|\mu|$ and $|\nu|$ are both finite. This proves that the integral expression $\int \chi_{\alpha^{-1}(E)}\df(\mu \times \nu)(x, y)$ is well-defined and finite.\newpage
	\begin{myIndent}\myComment
		Also, as a side note it means that:
		
		{\centering$\mu * \nu(E) = \iint \chi_E(x + y)\df\mu(x)\df \nu(y) = \iint \chi_E(x + y)\df \nu(y)\df \mu(x)$.\retTwo\par}
	\end{myIndent}

	Now showing that $\mu * \nu$ is a countably additive measure is as simple as noting that\\ $\alpha^{-1}(\emptyset) = \emptyset$ and also that if $\{E_n\}_{n \in \mathbb{N}}$ is a sequence of disjoint sets in $\mathcalli{B}_{\mathbb{R}^n}$, then\\ $\{\alpha^{-1}(E_n)\}_{n \in \mathbb{N}}$ is a sequence of disjoint sets in $\mathcalli{B}_{\mathbb{R}^n} \otimes \mathcalli{B}_{\mathbb{R}^n}$. Meanwhile, by theorem 7.8 in my math 240c notes, it suffices to show that $|\mu * \nu|(K) < \infty$ for all compact sets $K$ in order to say that $\mu * \nu$ is regular (and thus Radon). Yet this is trivially true for any complex measure.\retTwo
\end{myIndent}

\hOne\dispDate{9/11/2025}

\exTwo\ul{Proposition 8.48:}
\begin{itemize}
	\item[\hypertarget{Folland Proposition 8.48 part a}{(a)}] Convolution of measures is commutative and associative.
	
	\begin{myIndent}\exThreeP
		Proof:\\
		Commutativity is obvious. Meanwhile to show associativity note that:
		
		\begin{itemize}
			\item[\bullet] $((\lambda * \mu) * \nu)(E) = \iint \chi_{E}(x^\prime + z)\df (\lambda * \mu)(x^\prime) \df \nu(z)$\\ [6pt]
			$\phantom{((\lambda * \mu) * \nu)(E)} = \int (\lambda * \mu)(E + (-z)) \df \nu(z)$\\ [6pt]
			$\phantom{((\lambda * \mu) * \nu)(E)} = \int (\iint \chi_{E + (-z)}(x + y) \df \lambda(x) \df \mu(y)\df )\nu(z)$\\ [6pt]
			$\phantom{((\lambda * \mu) * \nu)(E)} = \iiint \chi_E(x + y + z)\df \lambda(x)\df \mu(y) \df \nu(z)$\retTwo

			\item[\bullet] Similarly, you can show that:
			
			{\centering$(\lambda * (\mu * \nu))(E) = \iiint \chi_E(x + y + z)\df \lambda(x)\df\mu(y)\df \nu(z)$.\retTwo\par}
		\end{itemize}

		Thus since both measures equal a common expression, we are done. $\blacksquare$\retTwo
	\end{myIndent}
	
	\item[(b)] For any bounded Borel measurable function $h$:
	
	{\centering$\int h \df (\mu * \nu) = \iint h(x + y)\df \mu(x) \df \nu(y)$.\retTwo\par}
	
	\begin{myIndent}\exThreeP
		Proof:\\
		This is obvious for characteristic functions, and by applying standard linearity\\ arguments we can show this is true when $h$ is a simple function.\retTwo

		For the general case, suppose $h: X \to \mathbb{C}$ is Borel measurable. Then, using a theorem on page 44 of my latex math 240a notes, we know there exists a sequence $\{\phi_n\}_{n \in \mathbb{N}}$ of simple functions such that $\{\phi_n\}_{n \in \mathbb{N}}$ is monotone increasing and $\phi_n \to h$ pointwise. If $h$ is bounded, then we can conclude by the dominated convergence theorem using $|h|$ and $\int |h(x, y)| \df \mu(x)$ as upper bounds that:

		{\centering\begin{tabular}{l}
			$\int h \df (\mu * \nu) = \lim\limits_{n \to \infty}\int \phi_n \df (\mu * \nu)$\\ [9pt]
			$\phantom{\int h \df (\mu * \nu)} = \lim\limits_{n \to \infty}\iint \phi_n(x + y)\df \mu(x)\df \nu(y) = \iint h(x + y)\df \mu(x)\df \nu(y)$ $\blacksquare$
		\end{tabular} \newpage\par}
	\end{myIndent}

	\item[(c)] $\|\mu * \nu\| \leq \|\mu\|\|\nu\|$.
	
	\begin{myIndent}\exThreeP
		Proof:\\
		Let $h = \frac{\df (\mu * \nu)}{\df |\mu * \nu|}$. Then we have that:
		
		{\centering $\|\mu * \nu\| = |\mu * \nu|(X) = \int \df |\mu * \nu| = \int \overline{h} h \df |\mu * \nu| = \int \overline{h} \df (\mu * \nu)$.\retTwo\par}

		Next, $|\overline{h}| = 1$ a.e. Therefore by applying part (b) we have that:

		{\centering\begin{tabular}{l}
			$\int \overline{h} \df (\mu * \nu) = |\iint \overline{h}(x + y)\df \mu(x)\df \nu(y)|$\\ [6pt]
			$\phantom{\int \overline{h} \df (\mu * \nu)} \leq \iint |\overline{h}(x + y)|\df |\mu|(x)\df |\nu|(y) = \iint 1 \df |\mu|(x)\df |\nu|(y)$\\ [6pt]
			$\phantom{\int \overline{h} \df (\mu * \nu) \leq \iint |\overline{h}(x + y)|\df |\mu|(x)\df |\nu|(y)} = |\mu|(\mathbb{R}^n)|\nu|(\mathbb{R}^n) = \|\mu\|\|\nu\|$ $\blacksquare$
		\end{tabular}\retTwo\par}
	\end{myIndent}
	
	\item[(d)] If $\mu = f \df m$ and $\nu = g \df m$, then $\mu * \nu = (f * g)\df m$; that's to say, if we view $L^1(m)$ as a subspace of $M(\mathbb{R}^n)$ via the mapping $f \mapsto f \df m$ (where $m$ is the Lebesgue measure), then our old and new definition of convolutions coincide.

	\begin{myIndent}\exThreeP
		Proof:\\
		Let $E \in \mathcalli{B}_{\mathbb{R}^n}$. Now $f(x)g(y) \in L^1(m \times m)$ by exercise 2.51 (in my math 240a latex notes) plus the fact that $|\mu|$ and $|\nu|$ are finite. So by Fubini's theorem plus the translation invariance of the Lebesgue measure, we have that:

		{\centering\begin{tabular}{l}
			$\mu * \nu(E) = \iint \chi_E(x + y)\df x \df y$\\ [6pt]
			$\phantom{\mu * \nu(E)} = \iint \chi_E(x + y)f(x)g(y) \df x \df y$\\ [6pt]
			$\phantom{\mu * \nu(E)} = \iint \chi_E(x)f(x - y)g(y) \df x \df y$\\ [6pt]
			$\phantom{\mu * \nu(E)} = \int_E \left(\int f(x - y)g(y)\df y \right)\df x = \int_E f * g \df x$ $\blacksquare$
		\end{tabular}\retTwo\par}
	\end{myIndent}
\end{itemize}

\hTwo From now on if I write $L^p$, let it be assumed that I'm referring to $L^p(\mathbb{R}^n, m)$ where $m$ is the Lebesgue measure. Then, we shall define convolutions of measures with functions in $L^p$ for any $p$.\retTwo

\exTwo\ul{Proposition 8.49:} If $f \in L^p(\mathbb{R}^n)$ (where $1 \leq p \leq \infty$) and $\mu \in M(\mathbb{R}^n)$, then the integral $f * \mu(x) \coloneqq \int f(x - y)\df \mu(y)$ exists for a.e. $x$, $f * \mu \in L^p$, and $\|f * \mu\|_p \leq \|f\|_p\|\mu\|$.

\begin{myIndent}\exThreeP
	Proof:\\
	To start off, we know by Minkowski's inequality that:
	
	{\centering\begin{tabular}{l}
		$\left(\int \left|\int |f(x - y)|\df |\mu|(y)^{\vphantom{\int^\int}}\right|^p \df x\right)^{1/p}\hspace{-0.5em} \leq \int \left( \int |f(x - y)|^p\df x \right)^{1/p} \df |\mu|(y)$\\

		$\phantom{\left(\int \left|\int |f(x + y)|\df |\mu|(y)^{\vphantom{\int^\int}}\right|^p \df x\right)^{1/p}\hspace{-0.5em}} = \int \left( \int |f(x)|^p\df x \right)^{1/p} \df |\mu|(y)$\\

		$\phantom{\left(\int \left|\int |f(x + y)|\df |\mu|(y)^{\vphantom{\int^\int}}\right|^p \df x\right)^{1/p}\hspace{-0.5em}} = \int \|f\|_p \df |\mu| = \|f\|_p \int \df |\mu| = \|f\|_p \|\mu\|$.
	\end{tabular}\retTwo\par}
	
	Then since $\|f\|_p \|\mu\| < \infty$, this tells us that $\int |f(x - y)|\df |\mu|(y)$ exists and is finite for a.e. $x$. In turn, this means that $\int f(x - y) \df \mu(y)$ exists and is finite for a.e. $x$. And since $|\int f(x + y)\df \mu(y)| \leq \int |f(x - y)\df |\mu|(y) = \left|\int |f(x - y)\df |\mu|(y)\right|$, we're done. $\blacksquare$\newpage
\end{myIndent}

\hTwo Note that if $p = 1$, then our two definitions of $f * \mu$ coincide (where again we are\\ considering $L^1$ as a subspace of $M(\mathbb{R}^n)$). After all,

{\centering $\int_E f * \mu(x) \df x = \iint \chi_E(x) f(x - y)\df \mu(y)\df x = \iint \chi_E(x + y) (f(x)\df x)\df \mu(y)$. \retTwo\par}

Thus, consider equipping $M(\mathbb{R}^n)$ with the convolution operator in addition to its other vector space operations. Then it's easy to see that $\mu * (\nu + \lambda) = (\mu * \nu) + (\mu * \lambda)$ for all $\mu, \nu, \lambda \in M(\mathbb{R}^n)$. Also, consider the point mass measure $\mu$ defined by $\iota(E) = 1$ if $0 \in E$ and $\iota(E) = 0$ otherwise. Clearly $\iota \in M(X)$ since $\iota$ is a measure on $\mathbb{R}^n$ that is finite on all compact sets. Also, if $\mu$ is any other measure in $M(\mathbb{R}^n)$ then we have that $\iota * \mu = \mu$. After all:

{\centering$\iota * \mu(E) = \iint \chi_E(x + y)\df \iota(x) \df \mu(y) = \int \chi_E(y)\df \mu(y) = \mu(E)$\retTwo\par}

Thus it follows that $M(\mathbb{R}^n)$ when equipped with pointwise addition and convolution is a commutative ring, and by proposition 8.49 we know that $L^1$ is an ideal in $M(\mathbb{R}^n)$.\retTwo

Also, it's clear that $c(\mu * \nu) = (c \mu) * \nu = \mu * (c \nu)$. So, $M(\mathbb{R}^n)$ is also an algebra. And it's clear either by proposition 8.48(d) or the fact that $L^1$ is an ideal that we also know that $L^1$ is a subalgebra of $M(\mathbb{R}^n)$.\retTwo

We now extend the Fourier Transform from $L^1$ to $M(\mathbb{R}^n)$ in the obvious way. Specifically, if $\mu \in M(\mathbb{R}^n)$, then define the function $\widehat{\mu}: \mathbb{R}^n \to \mathbb{C}$ by $\widehat{\mu}(\xi) = \int e^{-2\pi i \xi \cdot x} \df \mu(x)$.\retTwo

\pracOne\mySepTwo
Right now it is relevant to express a generalization of a result that Folland gave in chapter 2 of his book.\retTwo

\hypertarget{Generalization page 189}{\ul{Theorem:}} Consider any measure space $(X, \mathcal{M}, \mu)$ (where $\mu$ is positive), let $U \subseteq \mathbb{R}^n$ be an open set, and suppose that $f: X \times U \to \mathbb{C}$ is a function such that $f(\cdot, y): X \to \mathbb{C}$ is integrable for all $y \in U$. Also define $F(y) = \int_X f(x, t)\df \mu(x)$.

\begin{enumerate}
	\item[(a)] Suppose there exists $g \in L^1(\mu)$ such that $|f(x, y)| \leq g(x)$ for all $x \in X$ and $y \in U$. If $\lim_{y \to y^\prime} f(x, y) = f(x, y^\prime)$ for all $x \in X$, then $\lim_{y \to y^\prime} F(y) = F(y^\prime)$. In particular, if $f(x, \cdot)$ is continuous for each $x$, then $F$ is continuous.
	
	\item[(b)] Denote any $y \in U$ as $y = (y_1, \ldots, y_n)$. Now suppose that $\partial f / \partial y_j$ exists for all $y \in U$ and that there exists $g \in L^1(\mu)$ such that $|(\partial f / \partial y_j)(x, y)| \leq g(x)$ for all $x \in X$ and $y \in U$. Then $\partial F / \partial y_j$ exists and $\frac{\partial F}{\partial y_j}(y) = \int (\partial f / \partial y_j)(x, y)\df \mu(x)$.
\end{enumerate}

\begin{myIndent}\pracTwo
	The proof for part (a) is identical to that of theorem 2.27 in Folland (see my paper math 240a notes). Just apply dominated convergence theorem to the functions $f_m(x) \coloneqq f(x, y^{(m)})$ where $(y^{(m)})_{m \in \mathbb{N}}$ is any sequence in $U$ converging to $y^\prime$.
	\begin{myIndent}\myComment
		Side note: technically the proof for part (a) doesn't require $U$ to be open.\retTwo
	\end{myIndent}

	To prove part (b), let $e_1, \ldots, e_n$ be the standard basis vectors for $\mathbb{R}^n$ and consider for any $y \in U$ that there exists some $r > 0$ such that $B_r(y) \subseteq U$. Thus for any sequence $(t_m)_{m \in \mathbb{N}}$ in $(-r, r)$ converging to $0$ we have that:

	{\centering $\frac{\partial f}{\partial y_j}(x, y) = \lim_{m \to \infty} h_m(x)$ where $h_m(x) = \frac{f(x, y + t_m e_j) - f(x, y)}{t_m}$.\par}

	\begin{myIndent}\myComment
		(Note that this guarentees that $\frac{\partial f}{\partial y_j}$ is measurable for all $y$.)\newpage
	\end{myIndent}


	Now importantly $\tilde{f}(t) \coloneqq f(x, y + te_j)$ is differentiable with $\frac{\df \tilde{f}}{\df t}(t) = \frac{\partial f}{\partial y_j} (x, y + te_j)$.\retTwo 
	
	So, we know by the mean value theorem that $|\tilde{f}(t) - \tilde{f}(0)| \leq |t| \cdot \sup_{t \in (-r, r)} |\tilde{f}^\prime(t)|$. And in turn, we have that:

	{\centering $|h_m(x)| \leq \sup_{t \in (-r, r)}\frac{\partial f}{\partial y_j}(x, y + te_j) \leq g(x)$. \retTwo\par}

	Hence, we may apply dominated convergence theorem to say that:

	{\center\begin{tabular}{l}
		$\lim\limits_{m \to \infty}\frac{F(y + t_m) - F(y)}{t} = \lim\limits_{m \to \infty} \dfrac{\int f(x, y + t_me_j) \df \mu(x) - \int f(x, y)\df \mu(x)}{t}$\\ [10pt]
		$\phantom{\lim\limits_{m \to \infty}\frac{F(y + t_m) - F(y)}{t}} = \lim\limits_{m \to \infty} \int h_m(x)\df \mu(x) = \int \lim\limits_{m \to \infty} h_m(x)\df \mu(x) = \int \frac{\partial f}{\partial y_j}(x, y) \df \mu(x)$
	\end{tabular}\retTwo\par}

	This proves that $\frac{\partial F}{\partial y_j}(y) = \int \frac{\partial f}{\partial y_j}(x, y) \df \mu(x)$. $\blacksquare$\retTwo
\end{myIndent}

Also note that if $\mu$ is a complex measure, then by expressing $\mu = \mu_1 - \mu_2 + i(\mu_3 - \mu_4)$\\ where each $\mu_j$ is a positive measure and applying what we already proved to each\\ $\int f(x,y)\df \mu_j(x)$, we can show that the previous theorem holds for any function\\ $F(y) = \int f(x, y)\df \mu(x)$ provided that there exists some $g : X \to \mathbb{C}$ in $L^1(|\mu|)$ such\\ that $|f(x, y)| \leq g(x)$ for all $x \in X$ and $y \in U \subseteq \mathbb{R}^n$.\retTwo

\mySepTwo

\hTwo Since $e^{-2\pi i \xi \cdot x}$ is continuous with respect to $x$ and $|e^{-2\pi i \xi \cdot x}| = 1 \in L^1(|\mu|)$, we know by the prior reasoning that $\widehat{\mu}$ is continuous. Also, it is clear that $|\widehat{\mu}(\xi)| \leq \|\mu\|$. And, we can see for any $\mu, \nu \in M(\mathbb{R}^n)$ that $(\mu * \nu)^\wedge = \widehat{\mu}\widehat{\nu}$. After all:

{\centering \begin{tabular}{l}
	$(\mu * \nu)^\wedge(\xi) = \int e^{-2\pi i \xi \cdot x^\prime} \df (\mu * \nu)(x^\prime)$\\ [6pt]
	$\phantom{(\mu * \nu)^\wedge(\xi)} = \iint e^{-2\pi i \xi \cdot (x + y)} \df \mu(x)\df \nu(y)$ \\ [6pt]
	$\phantom{(\mu * \nu)^\wedge(\xi)} = \left(\int e^{-2\pi i \xi \cdot x} \df \mu(x)\right)\left(\int e^{-2\pi i \xi \cdot y} \df \nu(y)\right) = \widehat{\mu}(\xi)\widehat{\nu}(\xi)$
\end{tabular} \retTwo\par}

\exTwo\hypertarget{Folland proposition 8.50}{\ul{Proposition 8.50:}} Suppose that $\{\mu_k\}_{k\in \mathbb{N}}$ is a sequence in $M(\mathbb{R}^n)$ and $\mu \in M(\mathbb{R}^n)$. If\\ $\|\mu_k\| \leq C < \infty$ for all $k$ and $\widehat{\mu}_k \to \widehat{\mu}$ pointwise, then $\mu_k \to \mu$ vaguely.

\begin{myIndent}\exThreeP
	Proof:\\
	If $f \in \mathcalli{S}$ (where $\mathcalli{S}$ is the collection of Schwarts functions), then $f^\vee \in \mathcalli{S}$. So by the Fourier inversion theorem, we have that:

	{\centering\begin{tabular}{l}
		$\int f \df \mu_k = \int (\int f^{\vee}(x)e^{-2\pi i \xi \cdot x} \df x)\df \mu_k(\xi)$\\ [6pt]
		$\phantom{\int f \df \mu_k} = \int f^\vee(x) (\int e^{-2\pi i \xi \cdot x}\df \mu_k(\xi)) \df x = \int f^\vee(x) \widehat{\mu}_k(x) \df x$
	\end{tabular}\retTwo\par}

	By similar resoning we know that $\int f \df \mu = \int f^\vee(x) \widehat{\mu}(x) \df x$. But now since $\|\widehat{\mu_k}\|_u \leq C$ for all $k$ and $f^\vee \widehat{\mu}_k \to f^\vee \widehat{\mu}$ pointwise, we can conclude by the dominated convergence theorem that:

	{\centering\begin{tabular}{l}
		$\lim\limits_{k\to \infty} \int f \df \mu_k = \lim\limits_{k \to \infty}\int f^\vee(x) \widehat{\mu_k}(x) \df x$\\ [10pt]
		$\phantom{\lim\limits_{k\to \infty} \int f \df \mu_k} = \int \lim\limits_{k \to \infty} f^\vee(x)\widehat{\mu_k}(x)\df x = \int f^\vee(x)\widehat{\mu}(x) \df x = \int f \df \mu$
	\end{tabular}\retTwo\par}

	To finish off, note that since $\mathcalli{S}$ is dense in $C_0(\mathbb{R}^n)$, we know by Folland  proposition 5.17 (see my math 240b notes) that $\int f \df \mu_k \to \int f \df \mu$ as $k \to \infty$ for all $f \in C_0(\mathbb{R}^n)$. $\blacksquare$\newpage
\end{myIndent}

\hTwo Here's a relevant exercise from Folland:\retTwo

\Hstatement\hypertarget{Folland exercise 7.26}{\blab{Exercise 7.26:}} Suppose $X$ is an LCH space. If $\{\mu_k\}_{k \in \mathbb{N}} \subseteq M(X)$, $\mu_k \to \mu$ vaguely, and\\ $\|\mu_k\| \to \|\mu\|$, then $\int f \df \mu_k \to \int f \df \mu$ for every $f \in BC(X)$.

\begin{myIndent}\HexOne
	In the case that $\mu = 0$, then this is trivial from the fact that $\|\mu_k\| = |\mu_k|(X) \to 0$ as\\ $k \to \infty$. Meanwhile, if $\mu \neq 0$, then consider any $\varepsilon > 0$. We can show  that there exists\\ a compact set $F \subseteq X$ such that $|\mu_k|(F^\comp) < 4 \varepsilon$ for all $k$ and that $|\mu|(F^\comp) < 4\varepsilon$.\\ [-8pt]
	\begin{myIndent}\HexTwoP
		By Lusin's theorem, we know there exists a function $h \in C_c(X, [0, 1])$ such that\\ [2pt] $\|h\|_u \leq 1$ and $h = \frac{\df \mu}{\df |\mu|}$ everwhere outside a set $E$ of $|\mu|$-measure less than $\varepsilon$. Thus\\ we have that $|\int h \df \mu_k | \leq \int |h| \df |\mu_k| \leq \|\mu_k\|$, that $|\int h \df \mu| \leq \|\mu\|$, and that:

		{\center $|\int h \df \mu| = |\int_{E^\comp} \df |\mu| + \int_E h \df \mu| \geq |\mu|(E^\comp) - |\mu|(E) > \|\mu\| - 2\varepsilon$ \retTwo\par}

		Also, since $\mu_k \to \mu$ vaguely, we know that $\int h \df \mu_k \to \int h \df \mu$ as $k \to \infty$. Hence, there is some $N_1 \in \mathbb{N}$ such that for all $k > N_1$, $|\int h \df \mu_k - \int h \df \mu| < \varepsilon$. But in turn we have that $|\int h \df \mu_k| > \|\mu\| - 3\varepsilon$ for all $k > N_1$. Meanwhile, since $\|\mu_k\| \to \|\mu\|$, we can find some $N_2 \in \mathbb{N}$ greater than $N_1$ such that $|\int h \df \mu_k| > \|\mu_k\| - 4\varepsilon$ for all $k > N_2$. Thus, by setting $F^\prime = \supp(h)$ it is clear that $|\mu_k|(X - F^\prime) < 4\varepsilon$ for all $k > N_2$ and that $|\mu|(X - F^\prime) < \varepsilon < 4\varepsilon$.\retTwo

		Now to finish off, for all $k \in \mathbb{N}$ with $k \leq N_2$, we can find by the inner regularity\\ of $|\mu_k|$ on $X$ a compact set $F_k$ such that $|\mu_k|(F_k^\comp) < 4\varepsilon$. And by setting\\ [-2pt] $F = F^\prime \cup (\bigcup_{k=1}^{N_2} F_k )$, we are done.\retTwo 
	\end{myIndent}

	Now by Urysohn's lemma, we know there exists a function $g \in C_c(X, [0, 1])$ such that\\ $g = 1$ on $F$. It follows for any $f \in BC(X)$ that:
	
	{\centering$|\int (f - gf) \df \mu_k| = |\int_{F^\comp} (1 - g)f \df \mu_k| \leq \int_{F^\comp} 1\df |\mu_k| < 4\varepsilon$\retTwo\par}
	
	Similarly we have that $|\int (f - gf) \df \mu| < 4\varepsilon$. And importantly since $fg \in C_c(X)$ for any\\ $f \in BC(X)$ and $\mu_k \to \mu$ vaguely, we know that $\lim_{k \to \infty} |\int gf \df \mu_k - \int gf \df \mu| = 0$. So, we may conclude that:

	{\center\begin{tabular}{l}
		$\lim\limits_{k \to \infty}|\int f \df \mu_k - \int f \df \mu|$\\
		$\phantom{aaaaa} \leq \lim\limits_{k \to \infty}\left(|\int (f - gf)\df \mu_k| + |\int gf \df \mu_k - \int gf \df \mu| + |\int (f - gf) \df \mu| \right) < 8\varepsilon$
	\end{tabular} \retTwo\par}

	And taking $\varepsilon \to 0$ we have shown what we wanted.\retTwo
\end{myIndent}

Moreover, the hypothesis that $\|\mu_k\| \to \|\mu\|$ can't be omitted.

\begin{myIndent}\HexOne
	Consider $X = \mathbb{R}$ and let $\mu = 0$ and $\mu_k$ be the point mass centered at $k$ (i.e. $\mu_k(E) = 1$\\ if $k \in E$ and $0$ otherwise). Now clearly $\|\mu_k\| \not\to \|\mu\|$ as $k \to \infty$ since $\|\mu_k\| = 1$ for all $k$ but $\|\mu\| = 0$. Also it's clear that $\mu_k \to \mu$ vaguely since for all $f \in C_0(\mathbb{R})$, we have that $\int f \df \mu_k = f(k) \to 0 = \int f \df \mu$ as $k \to \infty$ since $f$ vanishes at infinity. That said, it is not the case that $\int f \df \mu_k \to \int f \df \mu$ for all $f \in BC(\mathbb{R})$, and a simple example illustrating that is $f = 1$. $\blacksquare$\retTwo
\end{myIndent}

\hTwo A corollary of the above exercise is that if $\{\mu_k\}_{k \in \mathbb{N}}$ is a sequence in $M(\mathbb{R}^n)$, $\mu \in M(\mathbb{R}^n)$, $\mu_k \to \mu$ vaguely, and $\|\mu_k\| \to \|\mu\|$, then $\widehat{\mu}_k \to \widehat{\mu}$ pointwise. This is because $e^{-2pi i \xi \cdot x}$ is a bounded continuous function for all $\xi \in \mathbb{R}^n$.\newpage

\Hstatement\blab{Exercise 8.40:} $L^1(\mathbb{R}^n)$ is vaguely dense in $M(\mathbb{R}^n)$.

\begin{myIndent}\HexOne
	Let $\varphi \in L^1(\mathbb{R}^n)$ such that $\varphi \geq 0$ and $\int \varphi(x) \df x = 1$. Then put $\varphi_t(x) \coloneqq \frac{1}{t^n}\varphi(x/t)$ for all $t > 0$. Now our claim is that for any $\mu \in M(\mathbb{R}^n)$, we have that $(\varphi_t * \mu)\df x \to \mu$ vaguely as $t \to 0$.
	\begin{myIndent}\HexTwoP
		To start off, we know that $\|(\varphi_t * \mu)\df x\| \leq \|\varphi_t\|_1 \|\mu\| = \|\mu\| < \infty$ for all $t > 0$.\retTwo
		
		Also note that $((\varphi_t * \mu)\df x)^\wedge(\xi) = \widehat{\varphi}_t(\xi)\widehat{\mu}(\xi)$. But consider that $\widehat{\varphi}$ is continuous and $\widehat{\varphi}_t(\xi) = \int \frac{1}{t^n} \varphi(\frac{x}{t}) e^{-2\pi i \xi \cdot x} \df x = \int \varphi(u)e^{-2\pi i \xi \cdot t u} \df u = \widehat{\varphi}(t\xi)$. Thus it's clear that $\widehat{\varphi}_t(\xi) \to \widehat{\varphi}(0)$ pointwise as $t \to 0$. And since $\int \varphi(x)\df x = 1$, we know that $\widehat{\varphi}(0) = 1$. Hence, $((\varphi_t*\mu)\df x)^\wedge \to \widehat{\mu}$ pointwise.\retTwo

		Applying \inLinkRap{Folland proposition 8.50}{proposition 8.50}, we are done. $\blacksquare$\retTwo
	\end{myIndent}
\end{myIndent}

\hOne\dispDate{9/14/2025}

I'd been visiting my parents in Ohio for the entire past week and didn't manage\\ to get that much done. But now that I'm heading back to California, I ought to start doing work again. Today I'll be starting on Folland chapter 10 (where he covers\\ probability). But before covering any theorems, I'd just like to shamelessly rip off\\ this table that Folland put together:

{\centering\includegraphics[scale=0.96]{Folland_screenshot.png}\retTwo\par}

Also, given some proposition $Q$, probabilists typically write $\{Q(\omega)\}$ as opposed to $\{\omega \in \Omega : Q(\omega) = \text{true}\}$. And given a probability measure $P$ they write $P(Q(\omega))$ as opposed to $P(\{Q(\omega)\})$. As long as I'm studying probability theory, I'm going to use probability terminology and notation.\retTwo

\hTwo Let $X$ be a random variable. We define it's \udefine{variance} $\sigma^2$ and \udefine{standard deviation} $\sigma$ as:\\ [-20pt]

{\center $\sigma^2(X) \coloneqq \inf_{a \in \mathbb{R}}E[(x - a)^2]$ and $\sigma(X) \coloneqq \sqrt{\sigma^2(X)}$\retTwo\par}

If $X \notin L^2$, then we always have $E((X - a)^2) = \infty$ for all $a$ and thus $\sigma^2(X) = \infty$. Otherwise, since $L^2 \subseteq L^1$ (since $P$ is finite), we have that $X \in L^1 \cap L^2$. So:

{\centering$E((X - a)^2) = E(X^2 - 2aX + a^2) = E(X^2) - 2aE(X) + a^2$.\newpage\par} 

Note that this is a quadratic which attains it's minimum at $a = E(X)$.
\begin{myIndent}\pracTwo
	Sanity check, by completing the square we have:
	
	{\center\fontsize{11}{13}\selectfont$a^2 - 2aE(X) = a^2 - 2aE(X) + (E(X))^2 - (E(X))^2 = (a - E(X))^2 - (E(X))^2$.\retTwo\par}
\end{myIndent}

Hence, for all $X \in L^2$ we have that:

{\centering$\sigma^2(X) = E(X^2) - 2(E(X))^2 + (E(X))^2 = E(X^2) - (E(X))^2$.\retTwo\par}

Let $(\Omega, \mathcalli{B}, P)$ be a sample space and let $(\Omega^\prime, \mathcalli{B}^\prime)$ be another measurable space. Given a $(\mathcalli{B}, \mathcalli{B}^\prime)$-measurable map $\phi$, we may define the \udefine{image measure} $P_\phi(E) \coloneqq P(\phi^{-1}(E))$ on $(\Omega^\prime, \mathcalli{B}^\prime)$. This is in fact a measure because preimages commute with unions, intersections, and complements. Also importantly $P_\phi(\Omega^\prime) = P(\Omega) = 1$. So $P_\phi$ is another probability measure.\retTwo

\exTwo\hypertarget{Folland Proposition 10.1}{\ul{Proposition 10.1:}} If $f: \Omega^\prime \to \mathbb{R}$ is a random variable, then $\int f \df P_\phi = \int (f \circ \phi) \df P$\\ whenever either side exists.

\begin{myIndent}\exThreeP
	This obviously holds for simple functions and you can extend the result from there using standard methods.\retTwo
\end{myIndent}

\hTwo If $X$ is a random variable on $\Omega$, then  $P_X$ is a probability measure on $\mathbb{R}$ called the\\ \udefine{distribution} of $X$. Also, we call $F(t) = P_X((-\infty, t]) = P(X \leq t)$ the \udefine{distribution\\ function} of $X$. If $\{X_\alpha\}_{\alpha \in A}$ is a family of random variables, we say that $\{X_\alpha\}_{\alpha \in A}$ are\\ \udefine{identically distributed} if $P_{X_\alpha} = P_{X_\beta}$ for all $\alpha, \beta \in A$.\retTwo

Basically all properties of a random variable we care about can be gathered just from\\ knowing the variable's distribution. For example, by proposition 10.1 we know that:

{\centering $E(X) = \int t \df P_X(t)$ and $E[(X - a)^2] = \int (t - a)^2 \df P_X(t)$. \retTwo\par}

\begin{myTindent}\pracTwo
	And from there we can easily see if $t \in L^1(P_X) \cap L^2(P_X)$, then:
	
	{\centering$\sigma^2(X) = \int (t - E(X))^2 \df P_X(t)$.\retTwo\par}
\end{myTindent}

If $X_1, \ldots, X_n$ is a finite sequence of random variables on $\Omega$, then we can consider\\ $(X_1, \ldots, X_n)$ as a map from $\omega$ to $\mathbb{R}^n$. Then the induced measure $P_{(X_1, \ldots, X_n)}$ on $\mathbb{R}^n$ is\\ called the \udefine{joint distribution} of $X_1, \ldots, X_n$.

\begin{myTindent}\pracOne
	Hopefully it's a given but if you doubt $(X_1, \ldots, X_n)$ is measurable\\ then see the proposition on page 41 of my math 240a latex notes.\retTwo
\end{myTindent}

Note by proposition 10.1 that given any two random variables $X, Y$ on $\Omega$:

{\center\begin{tabular}{l}
	$\int (t + s) \df P_{X, Y}(t, s) = \int [\pi_1 \circ \myId] \df P_{X, Y} + \int [\pi_2 \circ \myId] \df P_{X, Y}$\\ [6pt]
	$\phantom{\int (t + s) \df P_{X, Y}(t, s)} = \int [\pi_1 \circ \myId \circ (X, Y)] \df P + \int [\pi_2 \circ \myId \circ (X, Y)]\df P$\\ [6pt]
	$\phantom{\int (t + s) \df P_{X, Y}(t, s)} = \int X \df P + \int Y \df P = E(X) + E(Y) = E(X + Y)$.
\end{tabular}\retTwo\par}

Because of the convenience of distributions, given a Borel probability measure $\lambda$ on $\mathbb{R}$\\ we'll often talk of the mean $\overline{\lambda}$ and variance $\sigma^2(\lambda)$ of $\lambda$:

{\centering $\overline{\lambda} \coloneqq \int t \df \lambda(t)$ and $\sigma^2(\lambda) \coloneqq \int (t - \overline{\lambda})^2 \df \lambda(t)$ \newpage\par}

Let $(\Omega, \mathcalli{F}, P)$ be a sample space and $E \subseteq \Omega$ be an event such that $P(E) > 0$. Then\\ $P_E(F) \coloneqq P(E \cap F) / P(E)$ is a probability measure on $E$ which gives the \udefine{conditional\\ probability} of $F$ given $E$. $P_E(F)$ is also denoted $P(F | E)$.\retTwo

It's clear if $P(E) > 0$ that we then have that $P(E \cap F) = P(E)P(F | E)$. An important case to consider is when $P(F | E) = P(F)$ (which corresponds to when knowing $\omega \in E$ doesn't give us any information about if $\omega \in F$). In that case we say that $E$ and $F$ are\\ independent. More generally, we call two events $E$ and $F$ independent whenever\\ $P(E \cap F) = P(E)P(F)$. Even more generally, we call an arbitrary collection $\{E_\alpha\}_{\alpha \in A}$\\ of events in $\Omega$ \udefine{independent} iff for all $n \in \mathbb{N}$ and $\alpha_1, \ldots, \alpha_n \in A$, we have that\\ $P(\bigcap_{j=1}^n E_{\alpha_j}) = \prod_{j=1}^n P(E_{\alpha_j})$.\retTwo

\begin{myIndent}\pracOne
	From this definition hopefully it is obvious to you that if $C \subseteq A$, then $\{E_\alpha\}_{\alpha \in C}$\\ is a collection of independent random variables.\retTwo
\end{myIndent}

An arbitrary collection of random variables $\{X_\alpha\}_{\alpha \in A}$ are called \udefine{independent} if\\ $\{X_\alpha^{-1}(B_\alpha)\}_{\alpha \in A}$ is independent for all choices of $B_\alpha \in \mathcalli{B}_{\mathbb{R}}$.\retTwo

Observe that:

{\centering\begin{tabular}{l}
	$P(X_1^{-1}(B_1) \cap \ldots \cap X_n^{-1}(B_n)) = P((X_1, \ldots, X_n)^{-1}(B_1 \times \ldots \times B_n))$\\
	$\phantom{P(X_1^{-1}(B_1) \cap \ldots \cap X_n^{-1}(B_n))} = P_{X_1, \ldots, X_n}(B_1 \times \ldots \times X_n)$
\end{tabular}\retTwo\par}

Meanwhile $\prod_{j=1}^n P(X_j^{-1}(B_j)) = \prod_{j = 1}^n P_{X_j}(B_j) = \left(\prod_{j=1}^n P_{X_j}\right)(B_1 \times \ldots \times B_n)$.\retTwo

Thus, we have that $P_{X_1, \ldots, X_n} = \prod_{j=1}^n P_{X_j}$ if $X_1, \ldots, X_n$ are independent. Also, we\\ can easily show the converse by just setting $B_j = \mathbb{R}$ for any $j$ we want to ignore when\\ proving independence.\retTwo

\exTwo\hypertarget{Folland Proposition 10.2}{\ul{Proposition 10.2:}} Let $\{X_{n,j} : 1 \leq j \leq J(n), 1 \leq n \leq N\}$ be independent random variables, and let $f_n : \mathbb{R}^{J(n)} \to \mathbb{R}$ be Borel measurable for $1 \leq n \leq N$. Then the random variables $\{Y_n = f_n(X_{n,1}, \ldots, X_{n, J(n)}) : 1 \leq n \leq N\}$ are independent.

\begin{myIndent}\exThreeP
	Proof:\\
	Let $Z_n = (X_{n,1}, \ldots, X_{n, J(n)})$. If $B_1, \ldots, B_N$ are Borel subsets of $\mathbb{R}$, we have that\\ $Y_n^{-1}(B_n) = Z_n^{-1}(f_n^{-1}(B_n))$. Hence:\\ [-10pt]

	{\centering\begin{tabular}{l}
		$(Y_1, \ldots, Y_N)^{-1}(B_1 \times \ldots \times B_n) = \bigcap_{n=1}^N Y_n^{-1}(B_n)$\\ [4pt]
		$\phantom{(Y_1, \ldots, Y_N)^{-1}(B_1 \times \ldots \times B_n)} = \bigcap_{n=1}^N Z_n^{-1}(f_n^{-1}(B_n))$ \\ [4pt]
		$\phantom{(Y_1, \ldots, Y_N)^{-1}(B_1 \times \ldots \times B_n)} = (Z_1, \ldots, Z_N)^{-1}(f_1^{-1}(B_1) \times \ldots \times f_N^{-1}(B_N))$
	\end{tabular}\retTwo\par}

	But now note that $(Z_1, \ldots, Z_n)$ and $(X_{1,1},\ldots,X_{N,J(N)})$ are identical functions. And so by the independence of the $X_{n,j}$ and \inLinkRap{Folland Exercise 2.45}{Exercise 2.45}, note that:

	{\center\begin{tabular}{l}
		$P_{Z_1, \ldots, Z_n} = P_{X_{1,1},\ldots,X_{N,J(N)}}$\\
		
		$\phantom{P_{Z_1, \ldots, Z_n}} = \hspace{-0.6em}\prod\limits_{n=1,\phantom{.} j=1}^{N,\phantom{.} J(n)}\hspace{-0.6em} P_{X_{n,j}} = \prod\limits_{n=1}^N \left( \prod\limits_{j=1}^{J(n)} P_{X_{n,j}}\right) = \prod\limits_{n=1}^N \left(P_{X_{n,1}, \ldots, X_{n, J(n)}}\right) = \prod_{n=1}^N P_{Z_n}$
	\end{tabular}\newpage\par}

	Therefore, we have that:

	{\centering\begin{tabular}{l}
		$P_{Y_1, \ldots, Y_N}(B_1 \times \ldots \times B_N) = P((Y_1, \ldots, Y_N)^{-1}(B_1 \times \ldots \times B_N))$\\ [4pt]

		$\phantom{P_{Y_1, \ldots, Y_N}(B_1 \times \ldots \times B_N)} = P((Z_1, \ldots, Z_N)^{-1}(f_1^{-1}(B_1) \times \ldots \times f_N^{-1}(B_N)))$\\ [4pt]

		$\phantom{P_{Y_1, \ldots, Y_N}(B_1 \times \ldots \times B_N)} = P_{Z_1, \ldots, Z_n}(f_1^{-1}(B_1) \times \ldots \times f_N^{-1}(B_N))$\\ [4pt]

		$\phantom{P_{Y_1, \ldots, Y_N}(B_1 \times \ldots \times B_N)} = \prod_{n=1}^N P_{Z_n}(f_n^{-1}(B_n))$\\ [4pt]

		$\phantom{P_{Y_1, \ldots, Y_N}(B_1 \times \ldots \times B_N)} = \prod_{n=1}^N P(Z_n^{-1}(f_n^{-1}(B_n)))$\\ [4pt]

		$\phantom{P_{Y_1, \ldots, Y_N}(B_1 \times \ldots \times B_N)} =  \prod_{n=1}^N P(Y_n^{-1}(B_n)) = \prod_{n=1}^N P_{Y_n}(B_n)$. $\blacksquare$
	\end{tabular}\retTwo\par}
\end{myIndent}

\dispDate{9/15/2025}

\pracOne\ul{Quick lemma:} Let $(\Omega_1, \mathcalli{F}_1, P)$ be a sample space, let $(\Omega_2, \mathcalli{F}_2)$, $(\Omega_3, \mathcalli{F}_3)$ be measurable\\ spaces, and let $\phi: \Omega_1 \to \Omega_2$, $\psi: \Omega_2 \to \Omega_3$ be measurable maps. Then $P_{\psi \circ \phi} = (P_\phi)_{\psi}$.

\begin{myIndent}\pracTwo
	Proof:

	{\centering $P_{\psi \circ \phi}(E) = P((\psi \circ \phi)^{-1}(E)) = P(\phi^{-1}(\psi^{-1}(E))) = P_\phi(\psi^{-1}(E)) = (P_\phi)_{\psi}(E)$.\retTwo\par}

	\myComment (I'm mentioning this because Folland implicitely uses this in the next proof\dots)\retTwo
\end{myIndent}

\hTwo By easy induction on our proof for \inLinkRap{Folland Proposition 8.48 part a}{proposition 8.48(a)}, we can see that if $\lambda_1, \ldots, \lambda_n$ are\\ all in $M(\mathbb{R}^n)$, then:

{\centering$\lambda_1 * \ldots * \lambda_n(E) = \int\cdots\int \chi_E(\sum_{i=1}^n t_n) \df \lambda_1(t_1)\cdots \df \lambda_n(t_n)$.\retTwo\par}

\exTwo\hypertarget{Folland Proposition 10.4}{\ul{Proposition 10.4:}} If $X_1, \ldots, X_n$  are independent random variables, then:

{\centering $P_{X_1 + \ldots + X_n} = P_{X_1} * \ldots * P_{X_n}$.\par}

\begin{myIndent}\exThreeP
	Proof:\\
	Let $A(t_1, \ldots, t_n) = \sum_{j=1}^n t_j$. Then $A$ is a Borel measurable map from $\mathbb{R}^n$ to $\mathbb{R}$. So:

	{\center\begin{tabular}{l}
		$P_{X_1 + \ldots + X_n}(E) = \left(P_{X_1, \ldots, X_n}\right)_A(E)$\\ [8pt]
		$\phantom{P_{X_1 + \ldots + X_n}(E)} = \left(\prod_{j=1}^n P_{X_j}\right)_A(E) = \left(\prod_{j=1}^n P_{X_j}\right)(A^{-1}(E))$\\ [8pt]
		$\phantom{P_{X_1 + \ldots + X_n}(E) = \left(\prod_{j=1}^n P_{X_j}\right)_A(E)} = (P_{X_1} * \ldots * P_{X_n})(E)$. $\blacksquare$
	\end{tabular}\retTwo\par}
\end{myIndent}

\ul{Propostion 10.5:} Suppose that $X_1, \ldots, X_n$ are independent random variables. If $X_j \in L^1$ for all $1 \leq j \leq n$, then $\prod_{j=1}^n X_j \in L^1$ and $E(\prod_{j=1}^n X_j) = \prod_{j=1}^n E(X_j)$.

\begin{myIndent}\exThreeP
	Proof:\\
	Let $f(t_1, \ldots, t_n) = \prod_{j=1}^n |t_j|$. Then by the independence of the $X_j$ we have that:\\ [-8pt]

	{\centering\begin{tabular}{l}
		$E(\left|\prod_{j=1}^n X_j\right|) = \int f \circ (X_1, \ldots, X_n) \df P = \int f \df P_{X_1, \ldots, X_n} = \int f \df(P_{X_1} \times \ldots \times P_{X_n})$
	\end{tabular}\retTwo\par}

	And since $f$ is nonnegative, we know by Tonelli's theorem that:\\ [-10pt]

	{\center\begin{tabular}{l}
		$\int f \df(P_{X_1} \times \ldots \times P_{X_n}) = \prod_{j=1}^n \int |t_j| \df P_{X_j} = \prod_{j=1}^n E(|X_j|)$
	\end{tabular}\retTwo\par}

	Hence if all the $X_j \in L^1$, we've shown that $E(\left|\prod_{j=1}^n X_j\right|) = \prod_{j=1}^n E(|X_j|) < \infty$. This proves our first assertion that $\prod_{j=1}^n X_j \in L^1$. And to our prove our second assertion, just do the prior reasoning all over again but remove all the absolute value signs and use Fubini's theorem instead of Tonelli's. $\blacksquare$\newpage
\end{myIndent}

\hypertarget{Folland Corollary 10.6}{\ul{Corollary 10.6:}} Suppose that $X_1, \ldots, X_n$ are independent random variables and in $L^2$. Then $\sigma^2(X_1 + \ldots + X_n) = \sum_{j=1}^n \sigma^2(X_j)$.\retTwo

\begin{myIndent}\exThreeP
	Proof:\\
	Let $Y_j = X_j - E(X_j)$. Then by proposition 10.2, we have that all the $Y_j$ are independent and have an expected value of $0$. So by our last proposition, we know for all $j \neq k$ that $E(Y_j Y_k) = E(Y_j)E(Y_k) = 0$. And hence:

	{\centering\begin{tabular}{l}
		$\sigma^2(X_1 + \ldots + X_n) = E((X_1 + \ldots + X_n - E(X_1 + \ldots + X_n))^2)$\\ [4pt]
		$\phantom{\sigma^2(X_1 + \ldots + X_n)} = E((Y_1 + \ldots + Y_n)^2)$\\ [2pt]
		$\phantom{\sigma^2(X_1 + \ldots + X_n)} = \sum\limits_{j=1}^n\sum\limits_{k=1}^n E(Y_j Y_k) = \sum\limits_{j=1}^n E(Y_j^2) = \sum\limits_{j=1}^n \sigma^2(X_j)$. $\blacksquare$
	\end{tabular}\retTwo\par}
\end{myIndent}

\hTwo These prior properties show us that independence is actually a very strong assumption to make. In fact, by propositions 10.2 and 10.5, if $X$ and $Y$ are independent random variables with $E(X) = 0$, then given any Borel measurable function $f$ such that $f \circ Y \in L^1$, we have that $E(X \cdot f(Y)) = E(X)E(f(Y)) = 0$. Hence $X$ is orthogonal in $L^2$ to every function of $Y$ with finite $2$nd moment.\retTwo

That said, fortunately there is an easy way of constructing collections of independent\\ random variables. Specifically, given a collection of sample spaces $\{(\Omega_j, \mathcalli{B}_j, P_j)\}_{j=1}^n$, let\\ $\Omega = \Omega_1 \times \ldots \times \Omega_n$, $\mathcalli{B} = \mathcalli{B}_1 \otimes \ldots \otimes \mathcalli{B}_n$, and $P = P_1 \times \ldots \times P_n$. Then any collection\\ $\{X_1, \ldots, X_n\}$ of random variables on $\Omega$ such that $X_j$ depends only on the $j$th. coordinate\\ of $\Omega$ for each $j$ is independent.

\begin{myIndent}\pracTwo
	After all, express every $X_j$ as $f_j \circ \pi_j$ where $f_j$ is some $\mathcalli{B}_j$-measurable function and $\pi_j$\\ is the projection of $\Omega$ onto $\Omega_j$. Now given any rectangle $B_1 \times \ldots \times B_n$ in $\mathcalli{B}_{\mathbb{R}^n}$, we have\\ that:

	{\centering\begin{tabular}{l}
		$P_{X_1, \ldots, X_n}(B_1 \times \ldots \times B_n) = P((X_1, \ldots, X_n)^{-1}(B_1 \times \ldots \times B_n))$\\ [1pt]
		$\phantom{P_{X_1, \ldots, X_n}(B_1 \times \ldots \times B_n)} = P(\prod\limits_{j=1}^n f_j^{-1}(B_j)) = \prod\limits_{j=1}^n P_j(f_j^{-1}(B_j)) = \prod\limits_{j=1}^N (P_j)_{f_j}(B_j)$.
	\end{tabular}\retTwo\par}

	Now it is easily seen that:
	
	{\centering\begin{tabular}{l}
		$P_{\pi_j}(E) = P(\Omega_1 \times \ldots \times \Omega_{j-1} \times E \times \Omega_{j+1} \times \ldots \times \Omega)$\\ [1pt]
		$\phantom{P_{\pi_j}(E)} = P_j(E) \cdot \prod\limits_{i\neq j}P_i(\Omega_i) = P_j(E) \cdot 1 = P_j(E)$
	\end{tabular}\retTwo\par}

	Thus $(P_j)_{f_j} = (P_{\pi_j})_{f_j} = P_{f_j \circ \pi_j} = P_{X_j}$ and we've shown that $P_{X_1, \ldots, X_n}$ and\\ [1pt] $\prod_{j=1}^n P_{X_j}$ agree on all rectangles. It now follows by standard arguments that they\\ equal for all sets in $\mathcalli{B}_{\mathbb{R}^n}$.
\end{myIndent}

\Hstatement\mySepTwo

\blab{Exercise 10.9:} Suppose that $(X_n)_{n \in \mathbb{N}}$ is a sequence of random variables. If $X_n \to X$ in\\ probability, then $P_{X_n} \to P_X$ vaguely. 

\begin{myIndent}\HexOne
	Note that we trivially have that $\sup_{n \in \mathbb{N}} \|P_{X_n}\| = 1 < \infty$ since all the $P_{X_n}$ are probability\\ [1pt] measures. Meanwhile, define $F_n(t) \coloneqq P_{X_n}((-\infty, t])$ and $F(t) \coloneqq P_{X_n}((-\infty, t])$. We\\ [1pt] want to show that if $F$ is continuous at $t_0 \in \mathbb{R}$ then we have that $F_n(t_0) \to F(t_0)$ as\\ [1pt] $n \to \infty$. This is so that we can then apply \inLinkRap{Folland Proposition 7.19}{proposition 7.19} and be done.\newpage 
	
	So given any $\varepsilon > 0$, use the continuity of $F$ at $t_0$ to pick $\delta > 0$ such that $|F(t) - F(t_0)| < \varepsilon$ whenever $|t - t_0| < \delta$. Since $X_n \to X$ in measure, we may pick $N > 0$ such that $P(|X_n - X| > \sfrac{\delta}{2}) < \varepsilon$ for all $n \geq N$. Also, let $E_n = \{|X_n - X| \leq \sfrac{\delta}{2}\}$ for all $n \geq N$. Thus we have for all $n \geq N$ that:\\ [-6pt]

	{\centering\begin{tabular}{l}
		$|F_n(t_0) - F(t_0)| = |P(X_n^{-1}((-\infty, t_0]) \cap E_n) + P(X_n^{-1}((-\infty, t_0]) \cap E_n^\comp)$\\ [2pt]
		$\phantom{|F_n(t_0) - F(t_0)| = aaaaaaa} - P(X^{-1}((-\infty, t_0 + \sfrac{\delta}{2}])) + P(X^{-1}(( t_0, t_0 + \sfrac{\delta}{2}])) |$\\ [8pt]

		$\phantom{|F_n(t_0) - F(t_0)|} \leq |P(X_n^{-1}((-\infty, t_0]) \cap E_n) - P(X^{-1}((-\infty, t_0 + \sfrac{\delta}{2}]))|$\\ [2pt]
		$\phantom{|F_n(t_0) - F(t_0)| = aaaaaaa} + |P(X_n^{-1}((-\infty, t_0]) \cap E_n^\comp) + P(X^{-1}((t_0, t_0 + \sfrac{\delta}{2}]))|$
	\end{tabular} \retTwo\par}

	Now because $P(E_n^\comp) < \varepsilon$ and $|P(X^{-1}((t_0, t_0 + \sfrac{\delta}{2}]))| = |F(t_0 + \sfrac{\delta}{2}) - F(t_0)| < \varepsilon$, we know that $|P(X_n^{-1}((-\infty, t_0]) \cap E_n^\comp) + P(X^{-1}((t_0, t_0 + \sfrac{\delta}{2}]))| < 2\varepsilon$.\retTwo

	Meanwhile, $X_n^{-1}((-\infty, t_0]) \cap E_n \subseteq X^{-1}((-\infty, t_0 + \sfrac{\delta}{2}])$. Therefore, we have that:\\ [-6pt]

	{\centering\begin{tabular}{l}
		$|P(X_n^{-1}((-\infty, t_0]) \cap E_n) - P(X^{-1}((-\infty, t_0 + \sfrac{\delta}{2}]))|$\\ [4pt]
		$\phantom{aaaaaaaaaaaa} = P(X^{-1}((-\infty, t_0 + \sfrac{\delta}{2}]) - (E_n \cap X_n^{-1}((-\infty, t_0])))$\\ [4pt]
		$\phantom{aaaaaaaaaaaa} < P(E_n^\comp) + P(E_n \cap \left(X^{-1}((-\infty, t_0 + \sfrac{\delta}{2}] - X_n^{-1}((-\infty, t_0]))\right)) < \varepsilon + 0$
	\end{tabular} \retTwo\par}

	So $|F_n(t_0) - F(t_0)| < 3\varepsilon$ for all $n \geq N$. This proves that $F_n(t_0) \to F(t_0)$ as $n \to \infty$ and we are done. $\blacksquare$\retTwo
\end{myIndent}

\blab{Exercise 10.4:} Let $X$, $Y$, and $Z$ be positive independent random variables with a common\\ distribution $\lambda$, and let $F(t) = \lambda((-\infty, t])$. The probability that the polynomial $X t^2 + Y t + Z$\\ has real roots is $\int_0^\infty \int_0^\infty F(t^2 / 4s) \df \lambda(t) \df \lambda(s)$.

\begin{myIndent}\HexOne
	This is really easy. Note that the quadratic has real roots iff $Y^2 - 4XZ \geq 0$. Rewriting this, we have that the quadratic has real roots when $0 \leq Z \leq Y^2 / 4X$. Therefore:
	
	{\centering\begin{tabular}{l}
		$P(Y^2 - 4XZ \geq 0) = \int \chi_{\{t^2 - 4rs \geq 0\}}\df P_{X, Y, Z}(r, s, t)$\\ [4pt]
		
		$\phantom{P(Y^2 - 4XZ \geq 0)} = \int \chi_{\{t^2 - 4rs \geq 0\}}\df \lambda^3(r, s, t)$\\ [2pt]
		
		$\phantom{P(Y^2 - 4XZ \geq 0)} = \int_0^\infty \int_0^\infty \int_0^{t^2/4s} \df \lambda(r)\df \lambda(t)\df \lambda(s)$\\ [4pt]

		$\phantom{P(Y^2 - 4XZ \geq 0)} = \int_0^\infty \int_0^\infty F(t^2 / 4s) \df \lambda(t) \df \lambda(s)$. $\blacksquare$
	\end{tabular}\retTwo\par}
\end{myIndent}

\blab{Exercise 10.5:} If $X$ is a random variable with distribution $P_X = f(t)\df t$ where $f(t) = f(-t)$,\\ then the distribution of $X^2$ is $P_{X^2} = t^{-1/2} f(t^{1/2})\chi_{(0, \infty)}(t) \df t$.

\begin{myIndent}\HexOne
	Note that for any $s \geq 0$, $\{X^2 \leq s\} = \{-\sqrt{s} \leq X \leq \sqrt{s}\}$. Hence:

	{\centering\begin{tabular}{l}
		$P_{X^2}((-\infty, s]) = P(X^2 \leq s) = P(-\sqrt{s} \leq X \leq \sqrt{s})$\\ [4pt]
		$\phantom{P_{X^2}((-\infty, s]) = P(X^2 \leq s)} = \int_{-\sqrt{s}}^{\sqrt{s}} \df P_X = \int_{-\sqrt{s}}^{\sqrt{s}} f(t)\df t = \int_0^{\sqrt{s}}2f(t)\df t$\\ [4pt]
		$\phantom{P_{X^2}((-\infty, s]) = P(X^2 \leq s) = \int_{-\sqrt{s}}^{\sqrt{s}} \df P_X = \int_{-\sqrt{s}}^{\sqrt{s}} f(t)\df t} = \int_0^{s}2f(u^{1/2})(\frac{1}{2}u^{-1/2}\df u)$\\ [4pt]
		$\phantom{P_{X^2}((-\infty, s]) = P(X^2 \leq s) = \int_{-\sqrt{s}}^{\sqrt{s}} \df P_X = \int_{-\sqrt{s}}^{\sqrt{s}} f(t)\df t} = \int_0^s u^{-1/2}f(u^{1/2})\df u$
	\end{tabular}\retTwo\par}

	Meanwhile if $s < 0$, then $P_{X^2}((-\infty, s]) = 0$. Hence, we have for all $s \in \mathbb{R}$ that:

	{\centering $P_{X^2}((-\infty, s]) = \int_{-\infty}^s t^{-1/2}f(t^{1/2})\chi_{(0, \infty)}\df t$ \retTwo\par}

	This is enough by theorem 3.29 (see my math 240a paper notes) that to see that\\ $P_{X^2} = t^{-1/2}f(t^{1/2})\chi_{(0, \infty)}\df t$. $\blacksquare$\newpage
\end{myIndent}

I won't be reintroducing distributions or proving things that are already in my math 180a notes. That said, there is a result I'd like to give more clarity to. To start off, Folland denotes:

\begin{itemize}
	\item $\delta_t$ (where $t \in \mathbb{R}$) is the point mass measure centered at $E$ (meaning $\delta_t(E) = 1$ if $t \in E$\\ and $0$ otherwise).
	
	\item $\beta_p^{*n} \coloneqq \sum_{k=0}^n \binom{n}{k}p^k(1 - p)^{n-k}\delta_k$ is the binomial distribution with parameters $n \in \mathbb{N}$ and $p \in [0, 1]$.
	
	\item $\lambda_a \coloneqq e^{-a}\sum_{k=0}^\infty \frac{a^k}{k!}\delta_k$ is the Poisson distribution with parameter $a > 0$.
\end{itemize}

\blab{Exercise 10.8(c):} $\beta_{a/n}^{*n} \to \lambda_a$ vaguely as $n \to \infty$.

\begin{myIndent}\HexOne
	Like in my answer to exercise 10.9, our strategy for this exercise will be to apply \inLinkRap{Folland Proposition 7.19}{proposition\\ [2pt] 7.19}. So, set $F_n(t) \coloneqq \beta_{a/n}^{*n}((-\infty, t])$ and $F(t) \coloneqq \lambda_a((-\infty, t])$ for all $t \in \mathbb{R}$. Then the\\ set of discontinuities of $F$ is precisely $\mathbb{N}$ (here I'm having $\mathbb{N}$ include $0$). So, we just need to\\ [2pt] show that $F_n(t) \to F(t)$ whenever $t \notin \mathbb{N}$.\retTwo
	
	If $t < 0$, then it's clear that $0 = F_n(t) \to F(t) = 0$ as $n \to \infty$. Meanwhile, suppose that\\ [-1pt] $N < t < N + 1$ where $N \in \mathbb{N}$. Then $F(t) = e^{-a}\sum_{k=0}^N \frac{a^k}{k!}$.\retTwo 
	
	Meanwhile $\frac{n(n-1)\cdots(n-k+1)}{n^k} \to 0$, $(1 - \frac{a}{n})^n \to e^{-a}$, and $(1 - \frac{a}{n})^{-k} \to 1^{-k} = 1$ as $n \to \infty$\\ for all $k$. Therefore for $N < t < N+1$ we have:
	
	{\centering$F_n(t) = \sum_{k=0}^N \binom{n}{k} (\sfrac{a}{n})^k (1 - \sfrac{a}{n})^{n-k} \to \sum_{k=0}^N \frac{a^ke^{-a}}{k!} = F(t)$ as $n \to \infty$. $\blacksquare$\retTwo\par}

	\myComment\fontsize{11}{13}\selectfont As a side note, since $\|\beta_{a/n}^{*n}\| = 1 = \|\lambda_a\|$ for all $n$ since all of them are probability measures, we can apply \inLinkRap{Folland exercise 7.26}{Exercise 7.26} to say that $\int f \df \beta_{a/n}^{*n} \to \int f \df \lambda_a$ as $n \to \infty$ for all $f \in BC(\mathbb{R})$.\retTwo
\end{myIndent}

\blab{Exercise 10.10: (The Moment Convergence Theorem)}
\begin{myIndent}
	Let $X_1, X_2, \ldots, X$ be random variables such that $P_{X_n} \to P_X$ vaguely and\\ $\sup_{n \in \mathbb{N}}E(|X_n|^r) < \infty$ for some $r > 0$. Then $E(|X_n|^s) \to E(|X|^s)$ for all\\ $s \in (0, r)$, and if $s \in \mathbb{N}$ also, then $E((X_n)^s) \to E(X^s)$.\retTwo

	\HexOne Proof:\\
	Fix $C \geq \sup_{n \in \mathbb{N}} E(|X_n|^r)$. Then by Chebyshev's inequality (see my math 240b notes),\\ we know that $P(|X_n| > \alpha) \leq \frac{E(|X_n|^r)}{\alpha^r}$ for all $\alpha > 0$ and $n \in \mathbb{N}$. And hence for all $n \in \mathbb{N}$ we have that $P(|X_n| > \alpha) \leq C/\alpha^r$ when $\alpha > 0$.\retTwo

	Next, given any $\alpha > 0$, let $\phi_\alpha \in C_c(\mathbb{R}, [0, 1])$ such that $\phi_{\alpha}(t) = 1$ when $|t| \leq \alpha$ and\\ $\phi_{\alpha}(t) = 0$ when $|t| > \alpha + 1$. Importantly, $\phi_\alpha(t)|t|^s$ (and $\phi_\alpha(t) t^s$ when $s$ is an\\ integer) is in $C_c(\mathbb{R})$. Thus by the vague convergence of the $P_{X_n}$ we know for all $s \in (0, r)$\\ that $\int \phi_{\alpha}(t)|t|^s\df P_{X_n} \to \int \phi_{\alpha}(t)|t|^s \df P_{X}$ as $n \to \infty$ (and similarly if $s$ is also an integer\\ we have that $\int \phi_{\alpha}(t)t^s\df P_{X_n} \to \int \phi_{\alpha}(t)t^s \df P_{X}$).\retTwo

	Meanwhile, since $s < r$, we know by Hölder's inequality that:

	{\centering\begin{tabular}{l}
		$0 \leq \int (1 - \phi_{\alpha}(t))|t|^s \df P_{X_n} \leq \left[\int (1 - \phi_{\alpha}(t))^{r/(r-s)} \df P_{X_n}\right]^{(r-s)/r} \cdot \left[\int \left(|t|^s\right)^{r/s} \df P_{X_n}\right]^{s/r}$\\ [4pt]
		$\phantom{0 \leq \int (1 - \phi_{\alpha}(t))|t|^s \df P_{X_n}} \leq [\int \chi_{\{|t| > a\}} \df P_{X_n}]^{(r-s)/r} \cdot [E(|X_n|^r)]^{s/r}$\\ [4pt]
		$\phantom{0 \leq \int (1 - \phi_{\alpha}(t))|t|^s \df P_{X_n}} \leq (P(|X_n| > a))^{(r-s)/r} \cdot C^{s/r} \leq (\frac{C^{1/r}}{\alpha})^{r-s}C^{s/r} = \frac{C}{\alpha^{r-s}}$.
	\end{tabular}\newpage\par}

	Now $\int |t|^s\df P_{X_n} = \int \phi_{\alpha}(t)|t|^s \df P_{X_n} + \int(1 - \phi_{\alpha}(t))|t|^s \df P_{X_n}$. So, we can say that:\\ [-6pt]

	{\centering\begin{tabular}{l}
		$\int_{-\alpha}^{\alpha} |t|^s \df P_{X} \leq \int \phi_{\alpha}(t)|t|^s \df P_{X} \leq \liminf\limits_{n \to \infty} \int |t|^s \df P_{X_n}$\\ [10pt] 
		
		$\phantom{\int_{-\alpha}^{\alpha} |t|^s \df P_{X} \leq \int \phi_{\alpha}(t)|t|^s \df P_{X}} \leq \limsup\limits_{n \to \infty} \int |t|^s \df P_{X_n} \leq \int \phi_{\alpha}(t)|t|^s \df P_{X} + \frac{C}{\alpha^{r-s}}$\\ [10pt]

		$\phantom{\int_{-\alpha}^{\alpha} |t|^s \df P_{X} \leq \int \phi_{\alpha}(t)|t|^s \df P_{X} \leq \limsup\limits_{n \to \infty} \int |t|^s \df P_{X_n}} \leq \int_{-\alpha-1}^{\alpha + 1} |t|^s \df P_{X} + \frac{C}{\alpha^{r-s}}$
	\end{tabular}\retTwo\par}

	And by taking $\alpha \to \infty$, we know by the monotone convergence theorem and the fact that $\frac{1}{\alpha^{r-s}} \to 0$ that $\lim_{n \to \infty}E(|X|^s) = \lim_{n\to\infty}\int |t|^s \df P_{X_n} = \int |t|^s \df P_X = E(|X|^s)$ for all $s \in (0, r)$.\retTwo

	A notable consequence of this is that $X \in L^s(P)$ for all $s \in (0, r)$. After all, by Folland proposition 6.12 (see my math 240b notes):
	
	{\centering $\|X_n\|_s \leq \|X_n\|_rP(X)^{\sfrac{1}{s} - \sfrac{1}{r}} = \|X_n\|_r \leq C^{1/r}$ for all $n \in \mathbb{N}$.\retTwo\par}

	Thus $(E(|X_n|^s))_{n \in \mathbb{N}}$ is a sequence bounded above by $C^{s/r}$, and since $E(|X_n|^s) \to E(|X|^s)$, we have that $E(|X|^s) \leq C^{s/r} < \infty$ for all $s \in (0, r)$. This now let's us address the special case that $s$ is an integer. After all, note that:

	{\centering\begin{tabular}{l}
		$\limsup_{n\to\infty}|\int t^s \df P_{X_n} - \int t^s \df P_X| = \limsup_{n\to\infty}|\int t^s\phi_\alpha(t) \df P_{X_n} + \int t^s(1-\phi_\alpha(t)) \df P_{X_n}$\\ [6pt]
		$\phantom{aaaaaaaaaaaaaaaaaaaaaaaaaaaaaaaaaa} - \int t^s \phi_{\alpha}(t) \df P_X - \int t^s (1- \phi_\alpha(t)) \df P_X|$\\ [10pt]

		$\phantom{\limsup_{n\to\infty}|\int t^s \df P_{X_n} - \int t^s \df P_X|} \leq 0 + \limsup_{n\to\infty}|\int t^s(1 - \phi_{\alpha}(t))\df P_{X_n}|$\\ [2pt]
		$\phantom{aaaaaaaaaaaaaaaaaaaaaaaaaaaaaaaaaa 0 +}  + \limsup_{n \to \infty}|\int t^s(1 - \phi_{\alpha}(t))\df P_{X}|$
	\end{tabular}\retTwo\par}

	Now $0 \leq |\int (1 - \phi_{\alpha}(t))t^s \df P_{X_n}| \leq \int (1 - \phi_{\alpha}(t))|t|^s \df P_{X_n} \leq \frac{C}{\alpha^{r-s}}$ for all $n$.\retTwo
	
	Meanwhile pick $q \in (s, r)$ and note by Chebyshev's inequality that $P(|X| > \alpha) \leq \frac{C^{q/r}}{\alpha^{q}}$\\ [2pt] for all $\alpha > 0$. And then by similar reasoning to earlier we can show using Hölder's inequality\\ [2pt] that:

	{\centering\begin{tabular}{l}
		$0 \leq |\int (1 - \phi_{\alpha}(t))t^s \df P_{X}| \leq \int (1 - \phi_{\alpha}{t})|t|^s \df P_X$\\ [6pt]
		$\phantom{0 \leq |\int (1 - \phi_{\alpha}(t))t^s \df P_{X}|} \leq (P(|X| > \alpha))^{(q-s)/q} \cdot [E(|X|^q)]^{s/q}$\\ [6pt]
		$\phantom{0 \leq |\int (1 - \phi_{\alpha}(t))t^s \df P_{X}|} \leq (\frac{C^{1/r}}{\alpha})^{q-s} \cdot [C^{q/r}]^{s/q} = \frac{1}{\alpha^{q-s}} C^{(\frac{q}{r} - \frac{s}{r} + \sfrac{s}{r})} = \frac{1}{\alpha^{q-s}} C^{q/r}$
	\end{tabular} \retTwo\par}

	Thus $\limsup_{n\to\infty}|\int t^s \df P_{X_n} - \int t^s \df P_X| \leq \frac{C}{\alpha^{r-s}} + \frac{1}{\alpha^{q-s}} C^{q/r}$ and the latter goes to zero as $\alpha \to \infty$. $\blacksquare$\retTwo
\end{myIndent}

\mySepTwo

\hTwo\dispDate{9/16/2025}

In the next section of Folland, he writes a lot of theorems involving infinite sequences of\\ independent random variables. Now unfortunately actually constructing those sequences\\ requires some theorems I skipped over earlier. So for now just I'm just going to ignore the\\ issue. However, I'll come back and address this issue later \hypertarget{page 200 reference}{on} page \_\_\_.\newpage

\exTwo\ul{Theorem 10.9: (The Weak Law of Large Numbers)}
\begin{myIndent}
	Let $\{X_j\}_{j = 1}^\infty$ be a sequence of independent $L^2$ random variables with means $\{\mu_j\}_{j \in \mathbb{N}}$ and variances $\{\sigma_j^2\}$. If $n^{-2}\sum_{j=1}^n \sigma^2_j \to 0$ as $n \to \infty$, then $n^{-1}\sum_{j=1}^n (X_j - \mu_j) \to 0$ in probability as $n \to \infty$.\retTwo

	\exThreeP Proof:\\
	$n^{-1}\sum_{j=1}^n (X_j - \mu_j)$ has mean $n \cdot 0 = 0$ and variance:
	
	{\centering$\sum_{j=1}^n (\sigma^2(\frac{X_j - \mu_j}{n})) = \sum_{j=1}^n(E(\frac{(X_j - \mu_j)^2}{n^2} - 0)) = \frac{1}{n^2}\sum_{j=1}^n \sigma^2(X_j)$.\retTwo\par}

	Hence by Chebyshev's inequality, for any $\varepsilon > 0$ we have that:

	{\centering $P(|n^{-1}\sum_{j=1}^n (X_j - \mu_j)| > \varepsilon) \leq \frac{1}{\varepsilon^2}\left(\frac{1}{n^2}\sum_{j=1}^n \sigma^2(X_j)\right)$ \retTwo\par}

	And since $n^{-2}\sum_{j=1}^n \sigma^2_j \to 0$ as $n \to \infty$, we know $P(|n^{-1}\sum_{j=1}^n (X_j - \mu_j)| > \varepsilon) \to 0$\\ as $n \to \infty$. $\blacksquare$\retTwo

	\begin{myIndent}\myComment
		As a side note: if $\sup_{j \in \mathbb{N}}\sigma^2_j \leq C < \infty$, then $n^{-2}\sum_{j=1}^n \sigma^2_j \leq n^{-1} C \to 0$ as\\ $n \to \infty$. Hence, this theorem is especially useful for the case where all the $X_j$\\ are identically distributed.\retTwo
	\end{myIndent}
\end{myIndent}

\Hstatement\mySepTwo
Before continuing, we need an exercise.\retTwo

\blab{Exercise 10.3(b):} Suppose that $\{E_\alpha\}_{\alpha \in A}$ is a collection of independent events in $\Omega$. Then so is $\{F_\alpha\}_{\alpha \in A}$ where each $F_\alpha$ is equal to either $E_\alpha$ or $E_\alpha^\comp$.

\begin{myIndent}\HexOne
	Proof:\\
	I already showed in my math 180a notes that this is true whenever $A$ is finite. So I won't be repeating that proof. In the general case, for any $n \in \mathbb{N}$ and $\alpha_1, \ldots, \alpha_n \in A$ we know that the subcollection $\{E_{\alpha_1}, \ldots, E_{\alpha_n}\}$ is also a collection of independent events. And so we can apply the finite case proved in my math 180a notes to say that $F_{\alpha_1}, \ldots, F_{\alpha_n}$ are independent events, and in turn that $P(\bigcap_{j=1}^n F_{\alpha_j}) = \prod_{j=1}^n P(F_{\alpha_j})$. $\blacksquare$
\end{myIndent}
\mySepTwo

\hTwo Given a sequence $\{A_n\}_{n \in \mathbb{N}}$ of measurable sets / events, we define:

{\centering$\limsup A_n \coloneqq \bigcap_{n \in \mathbb{N}}\left(\bigcup_{k=n}^\infty A_k\right)$ and $\liminf A_n \coloneqq \bigcup_{n \in \mathbb{N}}\left(\bigcap_{k=n}^\infty A_k\right)$\retTwo\par}

\begin{myIndent}\myComment
	For some intuition: 
	\begin{itemize}
		\item $x \in \liminf A_n \Longleftrightarrow \exists N > 0 \suchthat \forall n \geq N, \myHS x \in A_n$\\$\phantom{aaaaaaaaaaaaaaaaaaaaaaaaaaaaaaa}$ (i.e. $x$ is eventually in each $A_n$),
		
		\item $x \in \limsup A_n \Longleftrightarrow \forall N > 0,\myHS \exists n \geq N \suchthat x \in A_n$\\$\phantom{aaaaaaaaaaaaaaaaaaaaaaaaaaaaaaa}$ (i.e. $x$ is frequently in the $A_n$).\retTwo
	\end{itemize}

	Clearly $\liminf A_n \subseteq \limsup A_n$.\retTwo
\end{myIndent}

\exTwo\ul{The Borel-Cantelli Lemma:} Let $\{A_n\}_{n \in \mathbb{N}}$ be a sequence of events.
\begin{itemize}
	\item[(a)] If $\sum_{n=1}^\infty P(A_n) < \infty$, then $P(\limsup A_n) = 0$.
	
	\item[(b)] If the $A_n$ are all independent and $\sum_{n=1}^\infty P(A_n) = \infty$, then $P(\limsup A_n) = 1$.
\end{itemize}
\newpage
\begin{myIndent}\exThreeP
	Proof:\\
	Part (a) is simple. $P(\limsup A_n) \leq P(\bigcup_{k=n}^\infty A_k) \leq \sum_{k=n}^\infty A_k$ for all $n \in \mathbb{N}$. And if\\ [1pt] $\sum_{n=1}^\infty P(A_n) < \infty$, then we know that $\sum_{k=n}^\infty A_k \to 0$ as $n \to \infty$.\retTwo

	As for part (b), note that $P(\limsup A_n) = 1$ if and only if:

	{\centering $P((\limsup A_n)^\comp) = P((\bigcap\limits_{n \in \mathbb{N}}(\bigcup\limits_{k=n}^\infty A_k))^\comp) = P(\bigcup\limits_{n \in \mathbb{N}}(\bigcup\limits_{k=n}^\infty A_k)^\comp) = P(\bigcup\limits_{n \in \mathbb{N}}(\bigcap\limits_{k=n}^\infty A_k^\comp)) = 0$ \retTwo\par}

	So, it suffices to show $P(\bigcap_{k=n}^\infty A_k^\comp) = 0$ for all $n \in \mathbb{N}$. But fortunately all the $A_k^\comp$ are independent according to exercise 10.3(b) on the last page. Thus we have for all $K > n$ that $P(\bigcap_{k=n}^K A_k^\comp) = \prod_{k=n}^K (1 - P(A_k))$. And by applying the monotonicity of measures, we have that:

	{\centering $P(\bigcap_{k=n}^\infty A_k^\comp) = \lim_{K\to \infty}P(\bigcap_{k=n}^K A_k^\comp) = \lim_{K \to \infty }\prod_{k=n}^K (1 - P(A_k))$. \retTwo\par}

	But now note that $1 - t \leq e^{-t}$ for all $t \in \mathbb{R}$. 
	\begin{myIndent}\exPPP
		My apartment mate immediately recognized this trick so I guess it's commonly used. Also, it probably would have been efficient to use when I was doing the exercise on \inLinkRap{Folland Exercise 1.32}{pages 51-52} last Christmas. Anyways, it's easy to see that this inequality is true by just looking at the derivative of $e^{-t} - (1 - t)$ and seeing that the function attains a global minimum at $t = 0$.\retTwo
	\end{myIndent}

	Hence:
	
	{\centering\begin{tabular}{l}
		$\lim_{K \to \infty }\prod_{k=n}^K (1 - P(A_k)) \leq \limsup_{K \to \infty}\prod_{k=n}^K e^{-P(A_k)}$\\
		$\phantom{\lim_{K \to \infty }\prod_{k=n}^K (1 - P(A_k))} = \limsup_{K \to \infty}\exp(-\sum_{k=n}^K P(A_k))$.
	\end{tabular}\retTwo\par}
	
	And since $\sum_{k=n}^\infty P(A_k) = +\infty$ for each $n$, we thus have that:
	
	{\centering$P(\bigcap_{k=n}^\infty A_k^\comp) \leq \limsup_{K \to \infty}\exp(-\sum_{k=n}^K P(A_k)) = 0$. $\blacksquare$\retTwo\par}
\end{myIndent}

\ul{Kolmogorov's Inequality:} Let $X_1, \ldots, X_n$ be independent random variables with mean $0$ and variances $\sigma_1^2, \ldots, \sigma_n^2$, and let $S_k = X_1 + \ldots + X_k$ for all $k \in \{1, \ldots, n\}$. Then for any $\varepsilon > 0$, we have that:

{\centering$P(\max\limits_{1 \leq k \leq n} |S_k| \geq \varepsilon) \leq \varepsilon^{-2}\sum\limits_{k=1}^n \sigma_k^2$.\par}

\begin{myIndent}\exThreeP
	Proof:\\
	For each $k$ let $A_k$ be the set where $|S_j| < \varepsilon$ for $j < k$ and $|S_k| \geq \varepsilon$. That way all the\\ $A_k$ are disjoint and their union is the set where $\max_{1 \leq k \leq n} |S_k| \geq \varepsilon$. Also, note that\\ $P(A_k) = E(\chi_{A_k}) \leq \varepsilon^{-2}E(\chi_{A_k}S_k^2)$ since $S_k^2 \geq \varepsilon^2$ on $A_k$. Therefore, we know that:\\ [-9pt]

	{\centering $P(\max_{1 \leq k \leq n} |S_k| \geq \varepsilon) = \sum_{k=1}^n P(A_k) \leq \varepsilon^{-2}\sum_{k=1}^n E(\chi_{A_k}S_k^2)$. \retTwo\par}

	\myComment Now, I kinda hate how Folland does the next bit because he just expands out his expression in a way that looks completely out of nowhere. So here's my attempt at explaining what insights I think led to Kolmogorov or Folland to doing the following reasoning.
	\begin{myIndent}\pracTwo
		Note that we already know that $E(S_n^2) = \sum_{k=1}^n \sigma_k^2$ by \inLinkRap{Folland Corollary 10.6}{corollary 10.6} since\\ $E(S_n^2) = \sigma^2(S_n)$ on account of the fact that $E(S_n) = 0$. So, if we could just\\ bound $\sum_{k=1}^n E(\chi_{A_k}S_k^2)$ from above by $E(S_n^2)$ then we'd be done. Luckily, note\\ that $S_n - S_k = X_{k+1} + \ldots + X_n$. Thus, by \inLinkRap{Folland Proposition 10.2}{proposition 10.2} we know that $S_k$\\ and $S_n - S_k$ are independent.\newpage 
		
		Going a step further, note that $\chi_{A_k}S_k$ can be rewritten as $f \circ (S_1, \ldots, S_k)$ where $f(t_1, \ldots, t_k) = t_k\chi_{(-\varepsilon, \varepsilon)^{k-1} \times (-\varepsilon, \varepsilon)^\comp}$ is a Borel measurable function from $\mathbb{R}^k$ to $\mathbb{R}$. In turn there is clearly another Borel measurable function $g: \mathbb{R}^k \to \mathbb{R}^k$ such that $g \circ (X_1, \ldots, X_k) = (S_1, \ldots, S_k)$. Thus $\chi_{A_k} S_k = f \circ g \circ (X_1, \ldots, X_k)$. And by applying \inLinkRap{Folland Proposition 10.2}{proposition 10.2}, we have that $\chi_{A_k} S_k$ and $S_n - S_k$ are independent.\retTwo

		Importantly, $E(S_n - S_k) = 0$ for all $k$. Thus:
		
		{\centering$E(\chi_{A_k}S_k(S_n - S_k)) = E(\chi_{A_k}S_k)E(S_n - S_k) =  E(\chi_{A_k}S_k)\cdot 0 = 0$ for all $k$.\retTwo\par}

		And hopefully it's now obvious how to proceed.\retTwo
	\end{myIndent}

	\exThreeP Observe that:

	{\centering \begin{tabular}{l}
		$E(S_n^2) \geq \sum\limits_{k=1}^n E(\chi_{A_k}S_n^2) = \sum\limits_{k=1}^n E(\chi_{A_k}(S_n^2 + S_k^2 - S_k^2 + 2S_kS_n - 2S_kS_n))$\\ [12pt]

		$\phantom{E(S_n^2) \geq \sum\limits_{k=1}^n E(\chi_{A_k}S_n^2)} = \sum\limits_{k=1}^n E(\chi_{A_k}(S_k^2 + 2S_kS_n + (S_n - S_k)^2))$\\ [12pt]

		$\phantom{E(S_n^2) \geq \sum\limits_{k=1}^n E(\chi_{A_k}S_n^2)} = \sum\limits_{k=1}^n E(\chi_{A_k}(3S_k^2 + 2S_kS_n - 2S_kS_k + (S_n - S_k)^2))$\\ [12pt]

		$\phantom{E(S_n^2) \geq \sum\limits_{k=1}^n E(\chi_{A_k}S_n^2)} \geq \sum\limits_{k=1}^n \left[3E(\chi_{A_k}S_k^2) + 2E(\chi_{A_k}S_k(S_n - S_k)) + 0\right]$\\ [12pt]
		
		$\phantom{E(S_n^2) \geq \sum\limits_{k=1}^n E(\chi_{A_k}S_n^2)} = 3 \sum\limits_{k=1}^n E(\chi_{A_k}S_k^2)$. $\blacksquare$
	\end{tabular} \retTwo\par}

	{\center\myComment\begin{myClosureOne}{5}
	\\ [-20pt]
	As a side note: I realize I proved a tighter bound then what Folland did.\\ It doesn't help that Folland made a mistake which I fixed while doing\\ the above manipulations. But anyways, I actually showed that:

	{\centering$P(\max\limits_{1 \leq k \leq n} |S_k| \geq \varepsilon) \leq \frac{1}{3\varepsilon^2}\sum\limits_{k=1}^n \sigma_k^2$.\retTwo\par}
	\end{myClosureOne}\retTwo\par}
\end{myIndent}

\ul{Kolmogorov's Strong Law of Large Numbers:} If $\{X_n\}_{n \in \mathbb{N}}$ is a sequence of independent $L^2$ random variables with means $\{\mu_n\}_{n \in \mathbb{N}}$ and variances $\{\sigma^2_n\}$ such that $\sum_{n \in \mathbb{N}} n^{-2}\sigma_n^2 < \infty$, then $n^{-1}\sum_{j=1}^n (X_j - \mu_j) \to 0$ a.s. as $n \to \infty$.

\begin{myIndent}\exThreeP
	Proof:\\
	Let $S_n = \sum_{k=1}^n (X_k - \mu_k)$ for all $n$. Next, given any $\varepsilon > 0$, for each $k \in \mathbb{N}$ let $A_k$ be the\\ set where $n^{-1}|S_n| \geq \varepsilon$ for some $n$ such that $2^{k-1} \leq n < 2^k$. Then we know for all\\ outcomes in $A_k$ that $|S_n| \geq \varepsilon 2^{k-1}$ for some $n < 2^k$. And thus by Kolmogorov's inequality,\\ we have that:\\ [-24pt]
	
	{\centering$P(A_k) \leq \frac{1}{3}(\varepsilon 2^{k-1})^{-2} \sum\limits_{n=1}^{2^k} \sigma_n^2$\retTwo\par}

	In turn we have (and note the usage of Tonelli's theorem at the end) that:
	
	{\centering$\sum\limits_{k=1}^\infty P(A_k) \leq \sum\limits_{k=1}^\infty[\frac{1}{3}(\varepsilon 2^{k-1})^{-2} \sum\limits_{n=1}^{2^k} \sigma_n^2] = \frac{4}{3\varepsilon^2}\sum\limits_{k=1}^\infty\sum\limits_{n=1}^{2^k} 2^{-2k}\sigma_n^2 = \frac{4}{3\varepsilon^2}\sum\limits_{n=1}^\infty(\sum\limits_{k \geq \log_2(n)} \hspace{-1em}2^{-2k})\sigma_n^2$.\retTwo\par}

	Also note that:
	
	{\centering$\hspace{-0.5em}\sum\limits_{k \geq \log_2(n)} \hspace{-1em}2^{-2k} = 2^{-2\lceil \log_2(n) \rceil}\sum\limits_{k=0}^\infty (2^{-2})^k = 2^{-2\lceil \log_2(n) \rceil} \cdot \frac{4}{3} \leq 2^{-2\log_2(n)} \cdot \frac{4}{3} = \frac{4}{3n^2}$.\newpage\par}

	Therefore, $\sum\limits_{k=1}^\infty P(A_k) \leq \frac{16}{9\varepsilon^2}\sum\limits_{n=1}^\infty n^{-2}\sigma_n^2 < \infty$.\retTwo

	By the Borel-Cantelli lemma, we now know that $P(\limsup A_k) = 0$. But note that $\limsup A_k$ consists precisely of every outcome where $n^{-1}|S_n| \geq \varepsilon$ for infinitely many $n$. Therefore, $P(\limsup_{n \to \infty} n^{-1}|S_n| \geq \varepsilon) = 0$ for all $\varepsilon > 0$. And consequently, we have that:

	{\center\begin{tabular}{l}
		$P(n^{-1}S_n \not\to 0) = P(\bigcup\limits_{m \in \mathbb{N}}\{\limsup\limits_{n \to \infty} n^{-1}|S_n| \geq \sfrac{1}{m}\})$\\ [12pt]
		$\phantom{P(n^{-1}|S_n| \not\to 0)} \leq \sum\limits_{m \in \mathbb{N}}P(\limsup\limits_{n \to \infty} n^{-1}|S_n| \geq \sfrac{1}{m}) = 0$. $\blacksquare$
	\end{tabular}\retTwo\par}

	\begin{myDindent}\myComment
		As a side note: it's still clear that if $\sup_{n \in \mathbb{N}}\sigma^2_n \leq C < \infty$, then\\ $\sum_{n=1}^\infty n^{-2} \sigma^2_j \leq C\sum_{n=1}^\infty n^{-2} < \infty$. In particular, this means that\\ this theorem applies when all the $X_j$ are identically distributed.\retTwo
	\end{myDindent}
\end{myIndent}

\ul{Khinchine's Strong Law of Large Numbers:} If $\{X_n\}_{n \in \mathbb{N}}$ is a sequence of independent\\ identically distributed $L^1$ random variables with mean $\mu$, then $n^{-1}\sum_{j=1}^n X_j \to \mu$ a.s.\\ as $n \to \infty$.
\begin{myIndent}\exThreeP
	Proof:\\
	Let $\lambda$ be the common distribution of the $X_n$ and note that $\int |t| \df \lambda(t) < \infty$. Next, for each $j$ let $Y_j = X_j$ on the set where $|X_j| \leq j$ and $Y_j = 0$ elsewhere. Then:

	{\centering\begin{tabular}{l}
		$\sum\limits_{j=1}^\infty P(Y_j \neq X_j) = \sum\limits_{j=1}^\infty P(|X_j| > j) = \sum\limits_{j=1}^\infty \lambda(\{|t| > j\}) = \sum\limits_{j=1}^\infty \sum\limits_{k=j}^\infty \lambda(\{k < |t| \leq k + 1\})$.
	\end{tabular}\retTwo\par}

	By swapping the order of summation (which we can do via Tonelli's theorem), we have that:

	{\centering\begin{tabular}{l}
		$\sum\limits_{j=1}^\infty \sum\limits_{k=j}^\infty \lambda(\{k < |t| \leq k + 1\}) = \sum\limits_{k=1}^\infty \sum\limits_{j=1}^k \lambda(\{k < |t| \leq k + 1\})$.\\ [12pt]
		$\phantom{\sum\limits_{j=1}^\infty \sum\limits_{k=j}^\infty \lambda(\{k < |t| \leq k + 1\})} = \sum\limits_{k=1}^\infty k \cdot \lambda(\{k < |t| \leq k + 1\}) \leq \int |t| \df \lambda(t) < \infty$.
	\end{tabular}\retTwo\par}

	Thus by the Borel-Cantelli lemma, if $A_j = \{Y_j \neq X_j\}$, then $P(\limsup A_j) = 0$. Or in other words, $P((\limsup A_j)^\comp) = 1$. This proves that almost surely there exists $J > 0$ such that  $X_j = Y_j$ for all $j > J$. And luckily from this we can now conclude that it suffices to show $n^{-1}\sum_{j=1}^n Y_j \to \mu$ a.s. in order to show that $n^{-1}\sum_{j=1}^n X_j \to \mu$ a.s.
	\begin{myIndent}\exPPP
		Why?\\
		Suppose that $n^{-1}\sum_{j=1}^n Y_j \to \mu$, that there exists $J$ such that $Y_j = X_j$ for all\\ $j > J$, and that all the $X_j$ are finite-valued. Then for any $n > J$ we have that:
		
		{\centering$n^{-1}\sum_{j=1}^n X_j = n^{-1}(\sum_{j=1}^{J}X_j) + n^{-1}(\sum_{j = J+1}^n Y_j)$.\retTwo\par}
		
		Now it's clear that $n^{-1}(\sum_{j=1}^{J}X_j) \to 0$ as $n \to \infty$ since $\sum_{j=1}^{J}X_j$ is fixed. Also by similar reasoning we know that $n^{-1}(\sum_{j=1}^{J}Y_j) \to 0$ as $n \to \infty$. And this is enough to say that $\lim_{n \to \infty}n^{-1}(\sum_{j=J+1}^{n}Y_j) = \lim_{n \to \infty}n^{-1}(\sum_{j=1}^{n}Y_j) = \mu$ since:

		{\center\begin{tabular}{l}
			$\lim_{n \to \infty}n^{-1}(\sum_{j=1}^{n}Y_j) = \lim_{n \to \infty}n^{-1}\sum_{j=1}^J Y_j + \lim_{n \to \infty}n^{-1}(\sum_{j=J+1}^{n}Y_j)$\\ [6pt]
			$\phantom{\lim_{n \to \infty}n^{-1}(\sum_{j=1}^{n}Y_j)} = 0 + \lim_{n \to \infty}n^{-1}(\sum_{j=J+1}^{n}Y_j)$ 
		\end{tabular}\retTwo\par}

		\myComment And as a side note since it took me a while to process this, if you were wondering why it wasn't trivial that $Y_j - X_j \to 0$ a.s., start by realizing that if the range of the $X_n$ includes values greater than $j + 1$, then $\{Y_j = X_j\} \not\subseteq \{Y_{j+1} = X_{j+1}\}$.\retTwo
	\end{myIndent}

	Now $\sigma^2(Y_n) \leq E(Y_n^2) = \int_{\{|t| \leq n\}} t^2 \df \lambda(t)$. Therefore:

	{\center\begin{tabular}{l}
		$\sum\limits_{n=1}^\infty n^{-2}\sigma^2(Y_n) = \sum\limits_{n=1}^\infty \sum\limits_{j=1}^n n^{-2}\int_{\{j-1 < |t| \leq j\}} t^2 \df \lambda(t) \leq \sum\limits_{n=1}^\infty \sum\limits_{j=1}^n jn^{-2}\int_{\{j-1 < |t| \leq j\}} |t| \df \lambda(t)$
	\end{tabular}\retTwo\par}

	And by using Tonelli's theorem to swap the order of summation, we can say that:

	{\center\begin{tabular}{l}
		$\sum\limits_{n=1}^\infty \sum\limits_{j=1}^n jn^{-2}\int_{\{j-1 < |t| \leq j\}} |t| \df \lambda(t) = \sum\limits_{j=1}^\infty \sum\limits_{n=j}^\infty jn^{-2}\int_{\{j-1 < |t| \leq j\}} |t| \df \lambda(t)$\\ [10pt]
		$\phantom{\sum\limits_{n=1}^\infty \sum\limits_{j=1}^n jn^{-2}\int_{\{j-1 < |t| \leq j\}} |t| \df \lambda(t)} = \sum\limits_{j=1}^\infty [j \cdot \left(\sum\limits_{n=j}^\infty n^{-2}\right) \cdot \int_{\{j-1 < |t| \leq j\}} |t| \df \lambda(t)]$
	\end{tabular}\retTwo\par}

	Next note that if $j > 1$, then: $\sum_{n=j}^\infty n^{-2} \leq \int_{j-1}^\infty x^{-2}\df x = \frac{1}{2(j-1)}$. Furthermore, note that when $j > 1$: $(\frac{1}{j})/(\frac{1}{2(j-1)}) = \frac{2j-2}{j} = 2 - \frac{2}{j} \geq 1$. Thus $\frac{1}{2(j-1)} < \frac{1}{j}$ for all $j > 1$. Meanwhile, $\sum_{n=1}^\infty n^{-2}$ famously equals $\frac{\pi^2}{6}$ which is easily checked to be less than $2 = 2 \cdot (1)^{-1}$. Thus,\\ [3pt] we can conclude that:

	{\center\begin{tabular}{l}
		$\sum\limits_{j=1}^\infty [j \cdot \left(\sum\limits_{n=j}^\infty n^{-2}\right) \cdot \int_{\{j-1 < |t| \leq j\}} |t| \df \lambda(t)] \leq \sum\limits_{j=1}^\infty [j \cdot \frac{2}{j} \cdot \int_{\{j-1 < |t| \leq j\}} |t| \df \lambda(t)]$\\ [14pt]
		$\hphantom{\sum\limits_{j=1}^\infty [j \cdot \left(\sum\limits_{n=j}^\infty n^{-2}\right) \cdot \int_{\{j-1 < |t| \leq j\}} |t| \df \lambda(t)]} = 2\int |t|\df \lambda(t) < \infty$
	\end{tabular}\retTwo\par}

	Therefore, if $\mu_j \coloneqq E(Y_j)$ for all $j$, then we know by Kolmogorov's strong law of large numbers that $n^{-1}\sum_{j=1}^n (Y_j - \mu_j) = \left(n^{-1}\sum_{j=1}^n Y_j\right) - \left(n^{-1}\sum_{j=1}^n \mu_j\right) \to 0$ a.s. as $n \to \infty$.\retTwo

	But now note by the dominated convergence theorem that:
	
	{\centering$\mu_j = \int_j^j t \df \lambda(t) \to \int_{-\infty}^\infty t \df \lambda(t) = \mu$ as $j \to \infty$.\retTwo\par}

	Therefore, by exercise 10.12 below, we know that $n^{-1}\sum_{j=1}^n \mu_j \to \mu$ as $n \to \infty$. And this proves that $n^{-1}\sum_{j=1}^n Y_j \to \mu$ a.s. as $n \to \infty$. $\blacksquare$\retTwo
\end{myIndent}

\Hstatement\mySepTwo
\blab{Exercise 10.12:} Suppose $\mathcalli{X}$ is a normed vector space and $(a_n)_{n \in \mathbb{N}}$ is a sequence in $\mathcalli{X}$ such that $a_n \to a$ for some $a \in \mathcalli{X}$. Then $n^{-1}\sum_{j=1}^n a_j \to a$ as well. 

\begin{myIndent}\HexOne
	Proof:\\
	Suppose $J$ is any integer and note that for any $n > J$:\\ [-10pt]

	{\centering\begin{tabular}{l}
		$\|(n^{-1}\sum_{j=1}^n a_j) - a\| \leq n^{-1}\sum_{j=1}^n \|a_j - a\|$\\ [4pt]
		$\phantom{\|(n^{-1}\sum_{j=1}^n a_j) - a\|} \leq \frac{J}{n}\max_{1 \leq j \leq J}\|a_j - a\| + \frac{n - J}{n}\sup_{j > J}\|a_j - a\|$
	\end{tabular}\retTwo\par}

	Thus by taking $n \to \infty$ we clearly have that:
	
	{\centering\begin{tabular}{l}
		$\limsup_{n \to \infty} \|(n^{-1}\sum_{j=1}^n a_j) - a\| \leq \sup_{j > J}\|a_j - a\|$.
	\end{tabular}\retTwo\par}

	And since $a_j \to a$ as $j \to \infty$, we know that $\sup_{j > J}\|a_j - a\| \to 0$ as $J \to \infty$. Thus,\\ taking $J \to \infty$ shows that $\limsup_{n \to \infty} \|(n^{-1}\sum_{j=1}^n a_j) - a\| = 0$. Or in other words,\\ $n^{-1}\sum_{j=1}^n a_j \to a$ as $n \to \infty$. $\blacksquare$\retTwo
\end{myIndent}

\mySepTwo\newpage

\hTwo The physical consequence of the laws of large numbers is that as a person plays more\\ games of chance (that are independent of each other), their average outcome will\\ approach the expected average outcome. Or to put into other words, a person's luck will\\ tend to balance out the more independent games of chance they play \pracOne(although that does\\ not mean that past games of chance have any predictive power over future games of\\ chance).\retTwo

\hTwo\dispDate{9/18/2025}

I want to start today by doing some exercises to expand on some of the earlier results. Firstly, here is a proof that the hypotheses for the weak law of large numbers are weaker than the hypotheses for Kolmogorov's strong law of large numbers.\retTwo

\Hstatement\blab{Exercise 10.11:} Let $(a_n)_{n \in \mathbb{N}}$ be a sequence in $[0, \infty)$ such that $\sum_{n=1}^\infty n^{-2}a_n < \infty$. Then: \\ [-10pt]

{\centering$\lim_{n \to \infty}n^{-2}\sum_{j=1}^n a_j  = 0$.\retTwo\par}

\begin{myIndent}\HexOne
	Proof:\\
	For any $N \in \mathbb{N}$ and $n > N$, we have that:

	{\centering\begin{tabular}{l}
		$n^{-2}\sum\limits_{j=1}^n a_j = n^{-2}\left(\sum\limits_{j=1}^N a_j\right) + \left(\sum\limits_{j=N+1}^n n^{-2}a_j\right) \leq n^{-2}\left(\sum\limits_{j=1}^N a_j\right) + \left(\sum\limits_{j=N+1}^n j^{-2}a_j\right)$
	\end{tabular} \retTwo\par}

	Then taking $n \to \infty$ we have that:
	
	{\centering$0 \leq \limsup_{n \to \infty} n^{-2}\sum_{j=1}^n a_j \leq 0 + \sum_{n=N+1}^\infty n^{-2}a_n$.\retTwo\par}

	And since $\sum_{n=1}^\infty n^{-2}a_n < \infty$, we know that $\sum_{n=N+1}^\infty n^{-2}a_n \to 0$ as $N \to \infty$. Hence,\\ we've shown that $\limsup_{n \to \infty} n^{-2}\sum_{j=1}^n a_j = 0$. And since $\liminf_{n \to \infty }n^{-2}\sum_{j=1}^n a_j \geq 0$,\\ this proves that $\lim_{n \to \infty} n^{-2}\sum_{j=1}^n a_j = 0$. $\blacksquare$\retTwo
\end{myIndent}

\hTwo Also, we can actually weaken our hypotheses for the weak law of large numbers.\retTwo

\Hstatement\blab{Exercise 10.13:} The weak law of large numbers remains valid if the hypothesis of independence is replaced by the hypothesis that $E[(X_j - \mu_j)(X_k - \mu_k)] = 0$ for all $j \neq k$.

\begin{myIndent}\HexOne
	We still have that $E(n^{-1}\sum_{j=1}^n (X_j - \mu_j)) = n^{-1}\sum_{j=1}^n E(X_j - \mu_j) = 0$. And in turn:

	{\center\begin{tabular}{l}
		$\sigma^2(n^{-1}\sum_{j=1}^n (X_j - \mu_j)) = E((n^{-1}\sum_{j=1}^n (X_j - \mu_j) - 0)^2)$\\ [8pt]
		$\phantom{\sigma^2(n^{-1}\sum_{j=1}^n (X_j - \mu_j))} = \frac{1}{n^2}E((\sum_{j=1}^n (X_j - \mu_j))^2)$\\ [8pt]
		$\phantom{\sigma^2(n^{-1}\sum_{j=1}^n (X_j - \mu_j))} = \frac{1}{n^2}\left(\sum_{j=1}^n \sigma_j^2 + \sum_{j\neq k}E[(X_j - \mu_j)(X_k - \mu_k)]\right)$\\ [8pt]
		$\phantom{\sigma^2(n^{-1}\sum_{j=1}^n (X_j - \mu_j))} = \frac{1}{n^2}\sum_{j=1}^n \sigma_j^2 + 0$.
	\end{tabular} \retTwo\par}

	And now the rest of the proof is identical to our original proof. $\blacksquare$
	\begin{myIndent}\color{RawerSienna}
		This hypothesis is strictly weaker. For example, it holds as long as the random\\ variables are pairwise independent, and it's possible for random variables to be\\ pairwise independent but not independent.\retTwo
	\end{myIndent}
\end{myIndent}

\hTwo On a complete tangent, here's another interesting result.\retTwo

\Hstatement\blab{Exercise 10.14:} If $\{X_n\}_{n \in \mathbb{N}}$ is a sequence of independent random variables such that $E(X_n) = 0$ for all $n$ and $\sum_{n=1}^\infty \sigma^2(X_n) < \infty$, then $\sum_{n=1}^\infty X_n$ converges almost surely.\newpage

\begin{myIndent}\HexOne
	Proof:\\
	Fix $\varepsilon > 0$ and let $N \in \mathbb{N}$. Also denote $S_n = \sum_{j=1}^n X_j$ for all $n \in \mathbb{N}$. Then note that\\ [1pt] by Kolmogorov's inequality, we have for all $m > N$ that:\\ [-10pt]

	{\centering$P(\max\limits_{N < n \leq m}|S_n - S_N| \geq \sfrac{\varepsilon}{2}) \leq 4\varepsilon^{-2}\hspace{-0.5em}\sum\limits_{n=N + 1}^m\hspace{-0.5em} \sigma^2(X_n)$.\retTwo\par}

	And by taking $m \to \infty$, we get that:

	{\centering$P(\exists n > N \suchthat |S_n - S_N| \geq \sfrac{\varepsilon}{2}) \leq 4\varepsilon^{-2}\hspace{-0.5em}\sum\limits_{n=N + 1}^\infty\hspace{-0.5em} \sigma^2(X_n)$.\retTwo\par}

	But now since $\sum_{n=1}^\infty \sigma^2(X_n) < \infty$, we know that $\sum_{n=N + 1}^\infty \sigma^2(X_k) \to 0$ as $N \to \infty$.\\ This is enough to let us conclude that there almost surely exists some $N \in \mathbb{N}$ such that\\ $|S_n - S_N| < \sfrac{\varepsilon}{2}$ for all $n > N$.

	\begin{myIndent}\HexTwoP
		Why is this?
		
		{\centering\begin{tabular}{l}
			$P(\forall N \in \mathbb{N},\myHS \exists n > N \suchthat |S_n - S_N| \geq \sfrac{\varepsilon}{2})$\\ [6pt]

			$\phantom{aaaaaaaaaaaaaaaa} = P(\bigcap\limits_{N \in \mathbb{N}}\{\exists n > N \suchthat |S_n - S_N| \geq \sfrac{\varepsilon}{2}\})$\\ [12pt]

			$\phantom{aaaaaaaaaaaaaaaa} \leq \inf\limits_{N \in \mathbb{N}}P(\exists n > N \suchthat |S_n - S_N| \geq \sfrac{\varepsilon}{2})$\\ [6pt]

			$\phantom{aaaaaaaaaaaaaaaa}\leq \inf\limits_{N \in \mathbb{N}} 4\varepsilon^{-2}\hspace{-0.5em}\sum\limits_{n=N + 1}^\infty\hspace{-0.5em} \sigma^2(X_n) = 0$.
		\end{tabular}\retTwo\par}
	\end{myIndent}

	Also note that if $m > n > N$, $|S_m - S_N| < \sfrac{\varepsilon}{2}$, and $|S_n - S_N| < \sfrac{\varepsilon}{2}$, then:
	
	{\centering $|S_m - S_n| \leq |S_m - S_N| + |s_n - S_N| < \varepsilon$.\retTwo\par}

	Hence, we have successfully proven that for any $\varepsilon > 0$, there almost surely exists some $N \in \mathbb{N}$ such that for all $m > n > N$ we have that $|S_m - S_n| < \varepsilon$. And in turn  by considering a countable sequence $(\varepsilon_k)_{k \in \mathbb{N}}$ in $(0, \infty)$ converging to $0$ we can thus easily see that the partial sums of the $X_n$ almost surely satisfy the Cauchy criterion. $\blacksquare$\retTwo
\end{myIndent}

\blab{Corollary (still part of the prior exercise):} If the plus and minus signs in $\sum_{n=1}^\infty \pm n^{-1}$ are\\ determined by successive tosses of a fair coin, the resulting series converges almost surely.

\begin{myIndent}\HexOne
	In this case, we have for each $n$ that $X_n = +n^{-1}$ $50$\% of the time and $-n^{-1}$ the other\\ $50$\% of the time. And since each $X_n$ is a simple function, we can easily evaluate that\\ $E(X_n) = 0$ and $\sigma^2(X_n) = E((X_n - 0)^2) = n^{-2}$. Hence, it's clear that we can apply\\ what we just proved to this sequence of $X_n$. $\blacksquare$\retTwo
\end{myIndent}

\hTwo Now to finish off today, I want to do (and give a little commentary) to an exercise that is heavily relevant to statistics.\retTwo

\Hstatement\blab{Exercise 10.17:} A collection or "population" of $N$ objects may be considered a sample space in\\ which each object individually has probability $N^{-1}$. Also let $X$ be a random variable on this space\\ with mean $\mu$ and variance $\sigma^2$. A central goal of statistics is to determine what $\mu$ and $\sigma^2$ are by\\ randomly sampling our population and measuring $X$ for each object sampled.\retTwo

To mathematically model this process, we imagine generating a sequence $\{X_n\}_{n \in \mathbb{N}}$ of numbers that are values of independent random variables with the same distribution as $X$. (specifically we can think of $X_n$ as the value of $X$ for the $n$th. object we sampled). And now we want to infer what $\mu$ and $\sigma^2$ are using only the first however so many numbers of our sequence.\newpage

The \udefine{$n$th sample mean} is defined as $M_n \coloneqq n^{-1}\sum_{j=1}^n X_j$. Note that $E(M_n) = \mu$ and\\ $\sigma^2(M_n) = n^{-1}\sigma^2$ for all $n$ and that $M_n \to \mu$ a.s. as $n \to \infty$.

\begin{myIndent}\HexOne
	The final statement is easily seen by applying Khinchine's strong law of large numbers.\\ [2pt] Meanwhile, the first result can be easily evaluated using the linearity of expectation (i.e.\\ [2pt] the fact that $E(n^{-1}\sum_{j=1}^n X_j) = n^{-1}\sum_{j=1}^n E(X_j)$). And finally, note by \inLinkRap{Folland Corollary 10.6}{corollary 10.6}\\ plus the easily checked fact that $\sigma^2(aX) = a^2 \sigma^2(X)$ that:
	
	{\center$\sigma^2(M_n) = \sigma^{2}(n^{-1}\sum_{j=1}^n X_j) = n^{-2}\sigma^2(\sum_{j=1}^n X_j) = n^{-2}\sum_{j=1}^n \sigma^2 = n^{-1}\sigma^2$\retTwo\par}
\end{myIndent}

Next, the \udefine{$n$th sample variance} is defined as $S_n^2 \coloneqq (n-1)^{-1} \sum_{j=1}^n (X_j - M_n)^2$.
\begin{itemize}
	\item Why do we use $n-1$ instead of $n$?
	
	\begin{myIndent}\HexOne
		Consider letting $E_n^2 \coloneqq {\text{\exPPP $n^{-1}$}} \sum_{j=1}^n (X_j - M_n)^2$ for all $n$. Then:

		{\centering\begin{tabular}{l}
			$E(E_n^2) = {\text{\exPPP $n^{-1}$}}\sum_{j=1}^nE((X_j - M_n)^2)$\\ [6pt]

			$\phantom{E(E_n^2)} = {\text{\exPPP $n^{-1}$}}\sum_{j=1}^n E(((X_j - \mu) - (M_n - \mu))^2)$\\ [6pt]

			$\phantom{E(E_n^2)} = {\text{\exPPP $n^{-1}$}}\sum_{j=1}^n \left[E((X_j - \mu)^2) - 2E((X_j - \mu)(M_n - \mu)) + E((M_n - \mu)^2)\right]$\\ [6pt]

			$\phantom{E(E_n^2)} = {\text{\exPPP $n^{-1}$}}\left[(\sum_{j=1}^n \sigma^2) - 2(\sum_{j=1}^n E((M_n - \mu)(X_j - \mu)) + (\sum_{j=1}^n n^{-1}\sigma^2))\right]$\\ [6pt]

			$\phantom{E(E_n^2)} = {\text{\exPPP $n^{-1}$}}\left[(n + 1)\sigma^2 - 2\sum_{j=1}^n E((M_n - \mu)(X_j - \mu))\right]$\\
		\end{tabular}\retTwo\par}

		Also note for all $j$ that:

		{\centering\begin{tabular}{l}
			$E((M_n - \mu)(X_j - \mu)) = E((n^{-1}\sum_{i=1}^n X_i - \mu)\cdot (X_j - \mu))$\\ [6pt]

			$\phantom{E((M_n - \mu)(X_j - \mu))} = n^{-1}\sum_{i=1}^nE((X_i - \mu)(X_j - \mu))$\\ [6pt]

			$\phantom{E((M_n - \mu)(X_j - \mu))} = n^{-1}\left(E((X_j - \mu)^2) + \sum_{i\neq j} E(X_i - \mu)E(X_j - \mu)\right)$\\ [6pt]

			$\phantom{E((M_n - \mu)(X_j - \mu))} = n^{-1}\left(\sigma^2 + \sum_{i\neq j} 0\right)$
		\end{tabular}\retTwo\par}

		Hence, we've shown that:
		
		{\centering$E(E_n^2) = {\text{\exPPP $n^{-1}$}}((n + 1)\sigma^2 - 2(n \cdot n^{-1}\sigma^2)) = {\text{\exPPP $n^{-1}$}}(n-1)\sigma^2$.\retTwo\par}
		
		And this proves that $E_n^2$ will typically under-estimate the actual variance of the\\ population.
		\begin{myIndent}\color{RawerSienna}
			As a side note, the physical intuition for this is that when sampling a population you aren't likely to measure the outliers of a population and it is those that tend to dominate what would be the theoretical expression for the population\\ variance.\retTwo
		\end{myIndent}

		Fortunately though, notice that upon replacing {\exPPP $n^{-1}$} with {\exPPP $(n-1)^{-1}$}, then it does work out that $E(S_n^2) = \text{\exPPP $(n-1)^{-1}$}(n-1) \sigma^2 = \sigma^2$.\retTwo
	\end{myIndent}

	\item Also note that $S_n^2 \to \sigma^2$ a.s. as $n \to \infty$.
	
	\begin{myIndent}\HexOne
		To start off, this statement would be way easier to prove if we were working with $E_n^2$ instead of $S_n^2$. So let's first prove that if $E_n^2 \to \sigma^2$ a.s. as $n \to \infty$, then so does $S_n^2 \to \sigma^2$ a.s. as $n \to \infty$.
		
		\begin{myIndent}\HexTwoP
			It's clear that $\frac{1}{n} \leq \frac{1}{n-1}$. So it will always be the case that:
			
			{\centering$\liminf_{n \to \infty} S_n^2 \geq \lim_{n \to \infty} E_n^2$.\newpage\par}

			On the other hand, note that for all $a > 1$, we have that $\frac{a}{n} \geq \frac{1}{n-1}$ so long as $n \geq \frac{a}{a-1}$.

			\begin{myIndent}\HexPPP
				Why?
				For $x > 1$ and $a > 1$: 
				
				{\centering\begin{tabular}{l}
					$\frac{a}{x} - \frac{1}{x-1} = \frac{ax - a - x}{x(x-1)} \geq 0 \Longleftrightarrow ax - a - x \geq 0 \Longleftrightarrow (a-1)x \geq a$\\
					$\phantom{\frac{a}{x} - \frac{1}{x-1} = \frac{ax - a - x}{x(x-1)} \geq 0 \Longleftrightarrow ax - a - x \geq 0} \Longleftrightarrow x \geq \frac{a}{a-1}$.
				\end{tabular}\retTwo\par}
			\end{myIndent}

			This proves that $\limsup_{n \to \infty} S_n^2 \leq a\lim_{n \to \infty} E_n^2$ for all $a > 1$. And now taking\\ [2pt] $a \to 1$ we get the desired result that $\lim_{n \to \infty} S_n^2 = \lim_{n \to \infty} E_n^2$.\retTwo
		\end{myIndent}

		So now we just need to show that $E_n^2 \to \sigma^2$ almost surely. Fortunately, note that:

		{\centering\begin{tabular}{l}
			$E_n^2 = n^{-1}\sum_{j=1}^n (X_j - M_j)^2$\\ [6pt]

			$\phantom{E_n^2} = n^{-1}\sum_{j=1}^n ((X_j - \mu) - (M_j - \mu))^2$\\ [6pt]

			$\phantom{E_n^2} = n^{-1}\sum_{j=1}^n \left[(X_j - \mu)^2 - 2(X_j - \mu)(M_n - \mu) + (M_n - \mu)^2\right]$\\ [6pt]

			$\phantom{E_n^2} = \left(n^{-1}\sum_{j=1}^n (X_j - \mu)^2\right) - \left(2(M_n - \mu) \cdot n^{-1}\sum_{j=1}^n(X_j - \mu)\right) + (M_n - \mu)^2$
		\end{tabular}\retTwo\par}

		Now we already proved that $M_n \to \mu$ a.s. Therefore it's clear that, $(M_n - \mu)^2 \to 0$ and $2(M_n - \mu) \to 0$ a.s. as $n \to \infty$. Meanwhile, by a straight foward application of Khinchine's strong law of large numbers we know that almost surely:
		
		{\centering$n^{-1}\sum_{j=1}^n(X_j - \mu) \to 0$ and $n^{-1}\sum_{j=1}^n (X_j - \mu)^2 \to \sigma^2$ as $n \to \infty$\retTwo\par}

		So, we almost surely have that $E_n^2 \to \sigma^2 - (0 \cdot 0) + 0 = \sigma^2$ as $n \to \infty$. $\blacksquare$\retTwo
	\end{myIndent}
\end{itemize}

\hTwo\dispDate{9/19/2025}

Given any $\mu \in \mathbb{R}$ and $\sigma^2 > 0$, we define the measure $\nu_\mu^{\sigma^2} \coloneqq \frac{1}{\sigma\sqrt{2\pi}} e^{-(t - \mu)^2/(2\sigma^2)} \df t$. \retTwo

You can check my math 180a notes to see that $\nu_\mu^{\sigma^2}(\mathbb{R}) = 1$. Also, while I realize that my\\ [-3pt] math 180a notes don't prove that $\int |t| \df \nu_\mu^{\sigma^2}(t) < \infty$, come on it's not that hard to prove and I don't feel like entirely redoing all my notes from that class. So just note that\\ $\int t \df \nu_\mu^{\sigma^2}(t) = \mu$ and $\int(t - \mu)^2 \df \nu_\mu^{\sigma^2}(t) = \sigma^2$.\retTwo

$\nu_\mu^{\sigma^2}$ is called the \udefine{normal / Gaussian distribution} with mean $\mu$ and variance $\sigma^2$. Also $\nu_0^1$ is called the \udefine{standard normal distribution}. Interestingly, this probability distribution was observed to be common in nature before the following technical reasoning was ever come up with for why.\retTwo

\pracOne\mySepTwo
Before continuing on, I need to lay some groundwork for some notation used in Folland's next proof. I don't know of any particular resources for this so I'm just going to wing this next part and also look at wikipedia for definitions (see bibliography for a link).\retTwo

Given a function $g: \mathbb{R} \to \mathbb{R}$ and $t_0 \in \overline{\mathbb{R}}$, we say that $f(t) = o(g(t))$ as $t \to t_0$ if there exists $\alpha: \mathbb{R} \to \mathbb{R}$ (or $\mathbb{C}$) such that $f(t) = \alpha(t)g(t)$ and $\alpha(t) \to 0$ as $t \to t_0$. This is called \udefine{little $o$ notation}.\newpage

\begin{myIndent}\myComment
	Note that I'm letting $\alpha$ take on complex values because I want to let $f$ also take on complex values. Although, for the sake of simplicity, I'm requiring $g$ to be real-valued.
	\retTwo
\end{myIndent}

Note that if $g(t) \neq 0$ when $t \neq t_0$ on some neighborhood of $t_0$, then $f(t) = o(g(t))$ as\\ $t \to t_0$ if and only if $\frac{f(t)}{g(t)} \to 0$ as $t \to t_0$.\retTwo

\ul{Lemma 1:} If $f(t) = o(g(t))$ as $t \to t_0$ and $c \neq 0$, then $f(t) = o(cg(t))$ as $t \to t_0$.
\begin{myIndent}\pracTwo
	Proof:\\
	Suppose $f(t) = \alpha(t)g(t)$ where $\alpha(t) \to 0$ as $t \to t_0$. Then just let $\alpha^\prime(t) \coloneqq c^{-1}\alpha(t)$\\ and we have that $f(t) = \alpha^\prime(t) \cdot cg(t)$ where $\alpha^\prime(t) \to 0$ as $t \to t_0$.\retTwo
\end{myIndent}


\ul{Lemma 2:} If $f(t) = o(g(t))$ as $t \to t_0$ and $c \in \mathbb{C}$, then $cf(t) = o(g(t))$ as $t \to t_0$.
\begin{myIndent}\pracTwo
	Proof:\\
	Once again write $f(t) = \alpha(t)g(t)$. Then $cf(t) = c\alpha(t)g(t)$ and $c\alpha(t) \to 0$ as $t \to t_0$.\retTwo
\end{myIndent}

\ul{Lemma 3:} $f(t) = o(g(t))$ as $t \to t_0$ iff $f(t) = o(|g(t)|)$ as $t \to t_0$.
\begin{myIndent}\pracTwo
	$(\Longrightarrow)$\\
	Suppose $f(t) = o(g(t))$ and write $f(t) = \alpha(t)g(t)$. Then set $\alpha^\prime(t) \coloneqq \alpha(t)g(t)|g(t)|^{-1}$ when $g(t) \neq 0$ and $\alpha^\prime(t) \coloneqq 0$ when $g(t) = 0$. Then it's clear that $f(t) = \alpha^\prime(t)|g(t)|$ and that $|\alpha^\prime(t)| \leq |\alpha(t)|\cdot 1 \to 0$ as $t \to t_0$. Hence $f(t) = o(|g(t)|)$.\retTwo

	$(\Longleftarrow)$\\
	Suppose $f(t) = o(|g(t)|)$ and write $f(t) = \alpha(t)|g(t)|$. Then when $g(t) \neq 0$ set\\ $\alpha^\prime(t) \coloneqq \alpha(t)|g(t)|(g(t))^{-1}$ and when $g(t) = 0$ set $\alpha^\prime(t) \coloneqq 0$. And, it's clear by identical reasoning as before that $f(t) = o(|g(t)|)$.\retTwo
\end{myIndent}

\ul{Lemma 4:} If $g_1$ and $g_2$ are nonnegative and $f_1(t) = o(g_1(t))$ and $f_2(t) = o(g_2(t))$ as $t \to t_0$, then $(f_1 + f_2)(t) = o((g_1 + g_2)(t))$ as $t \to t_0$.
\begin{myIndent}\pracTwo
	Proof:\\
	Write $f_1(t) = \alpha_1(t)g_1(t)$ and $f_2(t) = \alpha_2(t)g_2(t)$, and then set:
	
	{\centering$\alpha^\prime(t) \coloneqq \frac{\alpha_1(t)g_1(t) + \alpha_2(t)g_2(t)}{g_1(t) + g_2(t)}$ when $g_1(t) \neq 0$ and $g_2(t) \neq 0$, and $\alpha^\prime(t) \coloneqq 0$ otherwise.\retTwo\par}

	Then $(f_1 + f_2)(t) = \alpha^\prime(t)(g_1 + g_2)(t)$ and:

	{\centering\begin{tabular}{l}
		$|\alpha^\prime(t)| = |\alpha_1(t)|\frac{g_1(t)}{g_1(t) + g_2(t)} + |\alpha_2(t)|\frac{g_2(t)}{g_1(t) + g_2(t)} \leq |\alpha_1(t)|\frac{g_1(t)}{g_1(t)} + |\alpha_2(t)|\frac{g_2(t)}{g_2(t)}$\\ [6pt]
		
		$\phantom{|\alpha^\prime(t)| = |\alpha_1(t)|\frac{g_1(t)}{g_1(t) + g_2(t)} + |\alpha_2(t)|\frac{g_2(t)}{g_1(t) + g_2(t)}} = |\alpha_1(t)| + |\alpha_2(t)| \to 0$ as $t \to t_0$.
	\end{tabular}\retTwo\par}
\end{myIndent}

\ul{Corollary 5:} If $g_1$ and $g_2$ always have the same sign and $f_1(t) = o(g_1(t))$ and\\ $f_2(t) = o(g_2(t))$ as $t \to t_0$, then $(f_1 + f_2)(t) = o((g_1 + g_2)(t))$ as $t \to t_0$.
\begin{myIndent}\pracTwo
	Proof:\\
	Just apply lemmas 3 and 4 and use the fact the fact $|g_1| + |g_2| = |g_1 + g_2|$.\retTwo
\end{myIndent}

\ul{Corollary 6:} If $f_1(t) = o(g(t))$ and $f_2(t) = o(g(t))$ as $t \to t_0$, then $(f_1 + f_2)(t) = o(g(t))$ as $t \to t_0$.
\begin{myIndent}\pracTwo
	Proof:\\
	Apply corollary 5 as well as lemma 1.\retTwo
\end{myIndent}

\ul{Lemma 7:} Suppose $f_1(t) = o(g_1(t))$ and $f_2(t) = o(g_2(t))$ as $t \to t_0$. Then\\ $(f_1 \cdot f_2)(t) = o((g_1 \cdot g_2)(t))$ as $t \to t_0$.\newpage
\begin{myIndent}\pracTwo
	Proof:\\
	Write $f_1(t) = \alpha_1(t)g_1(t)$ and $f_2(t) = \alpha_2(t)g_2(t)$. Then $\alpha^\prime(t) \coloneqq \alpha_1(t)\alpha_2(t)$ satisfies that $\alpha^\prime(t) \to 0$ as $t \to t_0$ and $(f_1 \cdot f_2)(t) = \alpha^\prime(t)(g_1 \cdot g_2)(t)$.\retTwo
\end{myIndent}

\ul{Lemma 8:} Suppose $f(t) = o(g(t))$ as $t \to t_0$ and $h: \mathbb{R} \to \mathbb{R}$ is another function. Then $(f \cdot h)(t) = o((g \cdot h)(t))$ as $t \to t_0$.
\begin{myIndent}\pracTwo
	Proof:\\
	If $f(t) = \alpha(t)g(t)$, then $(f \circ h)(t) = \alpha(t)(g \circ h)(t)$.\retTwo
\end{myIndent}

\ul{Lemma 9:} Suppose $f(t) = o(g(t))$ as $t \to t_0$ and $g(t) = o(h(t))$ as $t \to t_0$. Then\\ $f(t) = o(h(t))$ as $t \to t_0$.
\begin{myIndent}\pracTwo
	Proof:\\
	Write $f(t) = \alpha_1(t)g(t)$ and $g(t) = \alpha_2(t)h(t)$. Then $f(t) = (\alpha_1(t)\alpha_2(t))h(t)$ and it's clear that $\alpha_1(t)\alpha_2(t) \to 0$ as $t \to t_0$.\retTwo
\end{myIndent}

\ul{Lemma 10:} Suppose $f(t) = o(g(t) + \sum_{j=1}^n h_j(t))$ as $t \to t_0$ where each $h_j(t) = o(g(t))$ as $t \to t_0$ for each $j$. Then $f(t) = o(g(t))$ as $t \to t_0$.
\begin{myIndent}\pracTwo
	Proof:\\
	Write $h_j(t) = \alpha_j(t)g(t)$ for each $j$ and $f(t) = \alpha(t)\cdot(g(t) + \sum_{j=1}^n h_j(t))$. Then\\ $\alpha^\prime(t) \coloneqq \alpha(t)\left(1 + \sum_{j=1}^n \alpha_j(t)\right)$ satisfies that $f(t) = \alpha^\prime(t)g(t)$ and:
	
	{\centering$\alpha^\prime(t) \to 0(1 + \sum_{j=1}^n 0) = 0$ as $t \to t_0$.\retTwo\par}
\end{myIndent}

\ul{Lemma 11:} Let $t_0, t_1 \in \overline{\mathbb{R}}$ and suppose $h: \mathbb{R} \to \mathbb{R}$ is some function such that $h(t) \to t_0$ as $t \to t_1$. Also suppose either:
\begin{itemize}
	\item that $h(t) \neq t_0$ on $U - \{t_1\}$ for some open neighborhood of $t_1$,
	\item or that $f$ and $g$ are continuous at $t_0$ (so that $\alpha$ is too).
\end{itemize}
Then if $f(t) = o(g(t))$ as $t \to t_0$, we know that $(f \circ h)(t) = o((g \circ h)(t))$ as $t \to t_1$. 

\begin{myIndent}\pracTwo
	Proof:\\
	Write $f(t) = \alpha(t)g(t)$. Then $(f \circ h)(t) = \alpha(h(t))g(h(t))$. And no matter which\\ assumption proposed above that we make, we have that $\alpha(h(t)) \to 0$ as $t \to t_1$.\retTwo
\end{myIndent}

\ul{Lemma 12:} Suppose $f(t) = o(g(t))$ as $t \to t_0$ and $|g| \leq |h|$ on some neighborhood of $t_0$. Then $f(t) = o(h(t))$.
\begin{myIndent}\pracTwo
	Proof:\\
	Write $f(t) = \alpha(t)g(t)$ and define $\alpha^\prime(t) \coloneqq \alpha(t)\frac{g(t)}{h(t)}$ when $h(t) \neq 0$ and $\alpha^\prime(t) \coloneqq 0$ otherwise. Then since $|g| \leq |h|$ on a neighborhood of $t_0$, it's clear that $|\alpha^\prime(t)| \leq |\alpha(t)|$ on a neighborhood of $t_0$. Hence, $f(t) = \alpha^\prime(t)h(t)$ and $\alpha^\prime(t) \to 0$ as $t \to t_0$.\retTwo
\end{myIndent}

\myComment Now, why do we care about little $o$ notation? Essentially, there are two use cases. First, if\\ $g(t) \to \pm \infty$ as $t \to t_0$ then we can use little $o$ notation to describe a cap on how fast something\\ grows. After all, if $g$ had that property and $f(t) = o(g(t))$ as $t \to t_0$, then that would be\\ equivalent to saying that for every $\varepsilon > 0$ there exists some $\delta > 0$ such that $|f(t)| < \varepsilon |g(t)|$\\ for all $t$ satisfying that $|t - t_0| < \delta$.\newpage

Meanwhile, note that if $g(t)$ is bounded on a neighborhood of $t_0$ and $f(t) = o(g(t))$ as $t \to t_0$, then we must have that $f(t) \to 0$ as $t \to t_0$. This points to the other use case of little $o$ notation which is to rigorously keep track of how negligible an error term is as we approach an asymptotic case while not having to worry about what those error terms specifically are. In fact, typically for this use case we will just entirely forget what $f$ is and write expressions like $1 + x^2 + o(x^2)$ as $x \to 0$. And if we write $o(g(t))$ (as $t \to t_0$) in an arithmetic setting without any other context, then it is implied that you can replace $o(g(t))$ with any $f$ satisfying that $f(t) = o(g(t))$ as $t \to t_0$ \retTwo

One more note to make is that (according to wikipedia) usually it is assumed that $t_0 = \infty$ unless otherwise stated. Also, hopefully it is clear how little $o$ notation can easily be applied to sequences in the case that $t_0 = \infty$.

\pracOne\mySepTwo

\hTwo Now Folland near the end of his next proof does some cool reasoning which unfortunately is flawed. Therefore, while I still want to show off the idea (since I think it's a cool usage of little $o$ notation), I'm going to try and find a different proof.\retTwo

\exTwo\ul{Lemma:} For any fixed $b \in \mathbb{R}$, we have that $\lim\limits_{n \to \infty}\left[1 + \frac{b}{n} + o(\frac{1}{n})\right]^n = e^{b}$.

\begin{myIndent}\exThreeP
	Proof:\\
	Let $(a_n)_{n \in \mathbb{N}}$ be any sequence in $\mathbb{C}$ with $a_n \to 0$ as $n \to \infty$ (so that $o(\frac{\xi^2}{n}) = a_n\frac{\xi^2}{n}$).\\ Then note that:

	{\centering\begin{tabular}{l}
		$\left[1 + \frac{b}{n} + \frac{a_n}{n}\right]^n = (1 + \frac{b}{n})^n + \sum\limits_{k=1}^n (1 + \frac{b}{n})^{n-k} \cdot \frac{n\cdots(n-k+1)}{k!} \cdot (\frac{a_n}{n})^k$
	\end{tabular}\retTwo\par}

	It's clear that $(1 + \frac{b}{n})^n \to e^b$ as $n \to \infty$. So, we just need to show that the complicated-looking sum goes to $0$ as $n \to \infty$. To do this, note that:

	{\centering\begin{tabular}{l}
		$\left|\sum\limits_{k=1}^n (1 + \frac{b}{n})^{n-k} \cdot \frac{n\cdots(n-k+1)}{k!} \cdot (\frac{a_n}{n})^k\right| \leq \sum\limits_{k=1}^n \left|(1 + \frac{b}{n})^{n-k}\right| \cdot \frac{n\cdots(n-k+1)}{k! \cdot n^k} \cdot |a_n|^k$
	\end{tabular}\retTwo\par}

	Now $\frac{n\cdots(n-k+1)}{k! \cdot n^k} \leq \frac{1}{k!}$. Also, when $n$ is big enough, we can guarentee $(1 + \frac{b}{n}) \in (0, 2)$. Then $(1 + \frac{b}{n})^{n-k} = ((1 + \frac{b}{n})^n)^{1 - \frac{k}{n}}$. And in the case that $b$ is negative, we will have that $((1 + \frac{b}{n})^n)^{1 - \frac{k}{n}} \leq ((1 + \frac{b}{n})^n)^0 = 1$ for all $k$. Meanwhile, in the case that $b$ is nonnegative, we know that for any $C > e^b$ that $((1 + \frac{b}{n})^n)^{1 - \frac{k}{n}} \leq ((1 + \frac{b}{n})^n)^1 < C$ for all $k$ once $n$ is big enough. Either way, this proves that there exists a constant $C^\prime$ such that once $n$ is big enough, $|(1 + \frac{b}{n})^{n-k}| \leq C^\prime$.\retTwo

	So, we now have for all $n$ sufficiently large that:

	{\centering\begin{tabular}{l}
		$\left|\sum\limits_{k=1}^n (1 + \frac{b}{n})^{n-k} \cdot \frac{n\cdots(n-k+1)}{k!} \cdot (\frac{a_n}{n})^k\right| \leq C^\prime \sum\limits_{k=1}^n  \frac{1}{k!} |a_n|^k$\\
		$\phantom{\left|\sum\limits_{k=1}^n (1 + \frac{b}{n})^{n-k} \cdot \frac{n\cdots(n-k+1)}{k!} \cdot (\frac{a_n}{n})^k\right|} = C^\prime(-1 + \sum\limits_{k=0}^n \frac{1}{k!}|a_n|^k)$\\
		$\phantom{\left|\sum\limits_{k=1}^n (1 + \frac{b}{n})^{n-k} \cdot \frac{n\cdots(n-k+1)}{k!} \cdot (\frac{a_n}{n})^k\right|} \leq C^\prime(-1 + \sum\limits_{k=0}^\infty \frac{1}{k!}|a_n|^k) = C^\prime(e^{|a_n|} - 1)$.
	\end{tabular}\retTwo\par}

	And since $|a_n| \to 0$ as $n \to \infty$, we are done. $\blacksquare$\newpage
\end{myIndent}

\exTwo\ul{Theorem 10.14:} Let $\lambda$ be a Borel probability measure on $\mathbb{R}$ such that $\int t^2 \df \lambda(t) = 1$ and $\int t \df \lambda(t) = 0$ (side note: the finiteness of the first integral implies the existence of the second integral). For $n \in \mathbb{N}$ let $\lambda^{*n} \coloneqq \lambda * \cdots * \lambda$ (i.e. $\lambda$ convoluted with itself $n$ times), and $\lambda_n(E) \coloneqq \lambda^{*n}(\sqrt{n} E)$ for all $E \in \mathcalli{B}_{\mathbb{R}}$ where $\sqrt{n}E \coloneqq \{\sqrt{n} t : t \in E\}$. Then $\lambda_n \to \nu_0^1$ vaguely as $n \to \infty$.

\begin{myIndent}\exThreeP
	\begin{myIndent}\myComment
		As a side note: $\lambda_n$ is just the measure induced by $\lambda^{*n}$ and $t \mapsto n^{-\sfrac{1}{2}} t$. So $\lambda_n$ is in fact a well-defined measure. This also means that $\int f(t) \df \lambda_n(t) = \int f(n^{-\sfrac{1}{2}}t)\df \lambda^{*n}(t)$ for all $f$ where either integral exists by \inLinkRap{Folland Proposition 10.1}{proposition 10.1}.\retTwo
	\end{myIndent}

	Proof:\\
	By applying part (b) of the \inLinkRap{Generalization page 189}{theorem on page 189} of my journal twice and then part (a) once, we can show that $\widehat{\lambda} \in C^2(\mathbb{R})$ with:
	\begin{itemize}
		\item $\widehat{\lambda}(0) = \int e^{-2 \pi i (0) \cdot t} \df \lambda(t) = \lambda(\mathbb{R}) = 1$,
		\item $\widehat{\lambda}^\prime(0) = \int -2\pi it e^{-2 \pi i (0) \cdot t} \df \lambda(t) = -2\pi i \int t \cdot 1 \df \lambda(t) = 0$,
		\item $\widehat{\lambda}^\pprime(0) = \int (-2\pi it)^2 e^{-2 \pi i (0) \cdot t} \df \lambda(t) = -4\pi^2\int t^2 \cdot 1 \df \lambda(t) = -4\pi^2$.\retTwo
	\end{itemize}

	It follows by applying Taylor's theorem to the real and imaginary parts separately (see my math 240c notes) that $\lambda(\xi) = 1 - 2\pi^2 \xi^2 + o(\xi^2)$ as $\xi \to 0$.\retTwo

	Next note that: {\exPPP(where $o(\frac{\xi^2}{n})$ is by any path where $\frac{\xi^2}{n} \to 0$)}\\ [-10pt]
	
	{\centering\begin{tabular}{l}
		$\widehat{\lambda}_n(\xi) = \int e^{-2\pi i \xi t} \df \lambda_n(t) = \int e^{-2\pi i \xi n^{-\sfrac{1}{2}} t} \df \lambda^{*n}(t)$\\ [6pt]
		$\phantom{\widehat{\lambda}_n(\xi) = \int e^{-2\pi i \xi t} \df \lambda_n(t)} = \widehat{\lambda^{*n}}(n^{-\sfrac{1}{2}}\xi) = \left[\widehat{\lambda}(n^{-\sfrac{1}{2}}\xi)\right]^n = \left[1 - \frac{2\pi^2 \xi^2}{n} + o(\frac{\xi^2}{n})\right]^n$
	\end{tabular}\retTwo\par}

	In particular, upon fixing $\xi \in \mathbb{R}$ we can apply the prior lemma to get that:
	
	{\centering\begin{tabular}{l}
		$\left[1 - \frac{2\pi^2 \xi^2}{n} + o(\frac{\xi^2}{n})\right]^n = \left[1 + \frac{-2\pi^2 \xi^2}{n} + o(\frac{1}{n})\right]^n \to e^{-2\pi^2 \xi^2}$ as $n \to \infty$.
	\end{tabular}\retTwo\par}

	This shows that $\widehat{\lambda}_n \to e^{-2\pi^2\xi^2}$ pointwise as $n \to \infty$.\retTwo

	But now recall from Folland proposition 8.24 (which is in my math 240c notes) that if\\ $f(x) = e^{-\pi ax^2}$ where $a > 0$, then $\widehat{f}(\xi) = a^{-1/2} e^{-\pi \xi^2 / a}$. Therefore:\\ [-10pt]

	{\centering\begin{tabular}{l}
		$\widehat{\nu}_0^1(\xi) = (\frac{1}{\sqrt{2\pi}}e^{-\frac{1}{2}x^2})^\wedge(\xi) = \frac{1}{\sqrt{2\pi}}(e^{-\pi(\frac{1}{2\pi})x^2})^\wedge(\xi) = \frac{1}{\sqrt{2\pi}} (\frac{1}{2\pi})^{-1/2} e^{-\pi \xi^2 \cdot (\frac{1}{2\pi})^{-1}} = 1 \cdot e^{-2\pi^2 \xi^2}$
	\end{tabular}\retTwo\par}

	Hence $\widehat{\lambda}_n \to \widehat{\nu}_0^1$ pointwise as $n \to \infty$. And since $\|\lambda_k\| = 1$ for all $k$, we can thus conclude by \inLinkRap{Folland proposition 8.50}{proposition 8.50} that $\lambda_n \to \nu_0^1$ vaguely. $\blacksquare$\retTwo

	\exPPP\mySepThree
	Before moving on, I want to show off what Folland tried to do. Folland notes that\\ $\log(1 + z) = z + o(z)$ as $z \to 0$. And therefore we have that:

	{\centering\begin{tabular}{l}
		$\log(\widehat{\lambda}_n(\xi)) = n \log(1 - \frac{2\pi^2 \xi^2}{n} + o(\frac{\xi^2}{n})) = n(- \frac{2\pi^2 \xi^2}{n} + o(\frac{\xi^2}{n}) + o(- \frac{2\pi^2 \xi^2}{n} + o(\frac{\xi^2}{n})))$\\ [6pt]

		$\phantom{\log(\widehat{\lambda}_n(\xi)) = n \log(1 - \frac{2\pi^2 \xi^2}{n} + o(\frac{\xi^2}{n}))} = -2\pi^2\xi^2 + n \cdot \left(o(\frac{\xi^2}{n}) + o(- \frac{2\pi^2 \xi^2}{n} + o(\frac{\xi^2}{n}))\right)$\\ [6pt]

		$\phantom{\log(\widehat{\lambda}_n(\xi)) = n \log(1 - \frac{2\pi^2 \xi^2}{n} + o(\frac{\xi^2}{n}))} = -2\pi^2\xi^2 + n \cdot o(\frac{\xi^2}{n}) \to -2\pi^2 \xi^2$ as $n \to \infty$.\\ [6pt]
	\end{tabular}\newpage\par}

	My issue with this reasoning is that $\widehat{\lambda}_n(\xi)$ is a complex valued function. So, what exactly does it mean to take the logarithm of it? Also, even if we were to extend the logarithm function to the complex plane, how do we know that $\log(z^n) = n\log(z)$ when $z$ has an imaginary component?

	\mySepThree
\end{myIndent}

\ul{The Central Limit Theorem:} Let $\{X_j\}_{j \in \mathbb{N}}$ be a sequence of independent identically\\ distributed $L^2$ random variables with mean $\mu$ and variance $\sigma^2$. As $n \to \infty$, the distribution\\ of $(\sigma \sqrt{n})^{-1}\sum_{j=1}^n (X_j - \mu)$ converges vaguely to the standard normal distribution\\ $\nu_0^1$ as $n \to \infty$, and for all $a \in \mathbb{R}$: $$\lim\limits_{n \to \infty} P\left(\frac{1}{\sigma\sqrt{n}}\sum_{j=1}^n(X_j - \mu) \leq a\right) = \frac{1}{\sqrt{2\pi}}\int_{-\infty}^a e^{\frac{-t^2}{2}} \df t$$

\begin{myIndent}\exThreeP
	Proof:\\
	Let $\lambda$ be the common distribution of $\sigma^{-1}(X_j - \mu)$ for all $j$. Then $\lambda$ satisfies the hypotheses last theorem. So, using the notation of the last theorem, we know that $\lambda_n \to \nu_0^1$ vaguely as $n \to \infty$. But now by applying \inLinkRap{Folland Proposition 10.4}{proposition 10.4} as well as the quick lemma on the same page, we have that:

	{\centering $\lambda_n = (\lambda^{*n})_{t \mapsto n^{-1/2}t} = (P_{\sigma^{-1}\sum_{j=1}^n (X_j - \mu)})_{t \mapsto n^{-1/2}t} = P_{\frac{1}{\sigma\sqrt{n}}\sum_{j=1}^n (X_j - \mu)}$ \retTwo\par}

	This proves the first claim. The second claim follows easily from \inLinkRap{Folland Proposition 7.19}{proposition 7.19 part (b)}\\ after you notice that $F(a) = \frac{1}{\sqrt{2\pi}}\int_{-\infty}^a e^{\frac{-t^2}{2}} \df t$ is continuous at all $a$ and that we trivially\\ have that all of the distributions of the $(\sigma \sqrt{n})^{-1}\sum_{j=1}^n (X_j - \mu)$ are positive measures.\\ $\blacksquare$\retTwo
\end{myIndent}

\hTwo\dispDate{9/20/2025}

To start off, there's an exercise that applies both the central limit theorem and the law of large numbers that I want to do. However, before I tackle it I need to do an exercise from chapter 2 of Folland.\retTwo

\Hstatement\blab{Exercise 2.39:} Let $(X, \mathcal{M}, \mu)$ be a measure space and suppose $(f_n)_{n \in \mathbb{N}}$ is a sequence of functions on $X$ such that $f_n \to f$ almost uniformly as $n \to \infty$ (meaning for all $\varepsilon > 0$ there exists $E \in \mathcal{M}$ such that $\mu(E) < 0$ and $f_n \to f$ uniformly on $E^\comp$). Then $f_n \to f$ a.e. and in measure.

\begin{myIndent}\HexOne
	Proof:\\
	The first claim is simple. For each $n$ let $E_n$ be a set satisfying that $\mu(E_n) < \sfrac{1}{n}$ and $f_n \to f$ uniformly as $n \to \infty$ on $E_n^\comp$. Then it is clear that $f_n \to f$ pointwise on $\bigcup_{n \in \mathbb{N}} E_n^\comp$. And $\mu((\bigcup_{n \in \mathbb{N}} E_n^\comp)^\comp) = \mu(\bigcap_{n \in \mathbb{N}} E_n) < \mu(E_n) = \sfrac{1}{n}$ for all $n \in \mathbb{N}$. Thus it follows that\\ [1pt] $\mu((\bigcup_{n \in \mathbb{N}} E_n^\comp)^\comp) = 0$ and $f_n \to f$ a.e.\retTwo

	To show the second claim, consider any fixed $\varepsilon > 0$ and $\delta > 0$. Then let $E$ be a set satisfying that $\mu(E) < \delta$ and $f_n \to f$ uniformly on $E^\comp$. Then clearly there is some $N$ such that for all $n \geq N$, $\mu(\{x : |f_n(x) - f(x)| \geq \varepsilon\}) \leq \mu(E) < \delta$. This proves that $f_n \to f$ in measure.\newpage
\end{myIndent}

\hTwo By combining the prior exercise with Egoroff's theorem, we can see that on any finite measure space, if $f_n \to f$ pointwise a.e., then $f_n \to f$ almost uniformly and therefore also in measure. This can be especially useful to keep in mind when working on spaces equipped with probability measures.\retTwo

\Hstatement\blab{Exercise 10.20:} If $\{X_j\}_{j \in \mathbb{N}}$ is a sequence of independent identically distributed random\\ variables with mean $0$ and variance $1$, then the distributions of $\sum_{j=1}^n X_j / (\sum_{j=1}^n X_j^2)^{1/2}$ and\\ $\sqrt{n}\sum_{j=1}^n X_j / (\sum_{j=1}^n X_j^2)$ both converge vaguely to the standard normal distribution.

\begin{myIndent}\HexOne
	Proof:\\
	
\end{myIndent}


























\newpage

\hThree\dispDate{Bibliography:}

Everything is cited in the order it shows up in the journal with the exception that the first citation is for the book that got me to actually sit down and make a bibliography. Also, I decided to write my citations according to APA 7.
\begin{enumerate}
	\item Hewitt, E., \& Stromberg, K. (1975). \textit{Real and Abstract Analysis}. Springer.
	
	\item Rudin, W. (1976). \textit{Principles of Mathematical Analysis} (3rd ed.). McGraw Hill.
	
	\item Munkres, J. R. (2014). \textit{Topology}. Pearson.
	
	\item Artin, M. (2010). \textit{Algebra} (2nd ed.). Pearson.

	\item Folland, G. B. (1999). \textit{Real Analysis : Modern Techniques and Their Applications} (2nd ed.). John Wiley And Sons.
	
	\item Munkres, J. R. (1994). \textit{Analysis on manifolds}. Addison-Wesley.
	
	\item Guillemin, V., \& Haine, P. (2019). \textit{Differential Forms}. World Scientific Publishing Company.
	
	\item Spencer, J. (1981). Balancing unit vectors. \textit{Journal of Combinatorial Theory, Series A}, 30(3), 349–350. https://doi.org/10.1016/0097-3165(81)90033-9
	
	\item Lee, J. M. (2013). \textit{Introduction to smooth manifolds}. Springer.
	
	\item Big O notation. (2025, September 9). In \textit{Wikipedia}. https://en.wikipedia.org/w/index.php?title=Big\_O\_notation\&oldid=1310464597\#Little-o\_notation
\end{enumerate}
\end{document}



































% I thought it'd be fun to do this exercise from Folland's \textit{Real Analysis} tonight.\retTwo

% \exOne\blab{Exercise 2.61} If $f$ is continuous on $[0, \infty)$ for $\alpha > 0$ and $x \geq 0$, let:

% {\center $I_\alpha f(x) = \frac{1}{\Gamma(\alpha)} \int_0^x (x-t)^{\alpha-1}f(t)\df t$ \retTwo\par}

% $I_\alpha f$ is called the $\alpha$th \udefine{fractional integral} of $f$.
% \begin{itemize}
% 	\item[(a)] $I_{\alpha + \beta}f = I_\alpha(I_\beta f)$ for all $\alpha, \beta > 0$.
	
% 	\begin{myIndent}\exTwoP
% 		The book hints to use the identity $\frac{\Gamma(x)\Gamma(y)}{\Gamma(x + y)} = \int_0^1 t^{x-1}(1-t)^{y-1}\df t$ for $x, y > 0$. Proving this was exercise 2.60 and I solved out this exercise on pages 69 and 70 of my math 240A LaTeX document.\retTwo


% 	\end{myIndent}

% 	\item[(b)] If $n \in \mathbb{N}$, then $I_n f$ is the $n$th-order antiderivative of $f$. 
% \end{itemize}


% ~~~~~~~~~~~~~~~~~~~~~~~~~~~~~~~~~~~~~~~~~~~~~~~~~~~~

% \blab{Orientations:}\\
% If $L$ is a $1$-dimensional vector space, then $L - \{0\}$ has two connected components. An\\ \udefine{orientation} of $L$ is a choice of one of those two connected components. Typically, we denote that choice $L_+$ (or \udefine{the positive component}) and we denote the one not chosen $L_-$ (or \udefine{the negative component}).\retTwo

% If $(L, L^+)$ is an \udefine{oriented} $1$-dimensional vector space, then we say a vector $v \in L$ is \udefine{positively oriented} if $v \in L_+$.




% ~~~~~~~~~~~~~~~~~~~~~~~~~~~~~~~~~~~~~~~~~~~~~~~~~~~~






% If $\mu$ is a signed or complex measure on $(X, \mathcal{M})$, we call $\mu$ decomposable if $(X, \mathcal{M}, |\mu|)$ is decomposable.
% \begin{myIndent}\myComment
% 	Hopefully it is obvious to you that every $\sigma$-finite measure is decomposable. This also means that every complex measure is trivially decomposable since complex measures are finite.\retTwo

% 	Here are two lemmas which will come up in the following theorem. (I will not write out proofs cause they're fairly obvious if you just think about it for a second.)
	
% 	\begin{enumerate}
% 		\item If $\mu$ is a signed decomposable measure on $(X, \mathcal{M})$ via the decomposition $\mathcal{F}$, then we have that the positive and negative variations $\mu^+$ and $\mu^-$ of $\mu$ are also decomposable via $\mathcal{F}$.
% 		\item Suppose $(X, \mathcal{M}, \mu)$ is decomposable via the decomposition $\mathcal{F}$, and let $E \in \mathcal{M}$. Then the measure $\mu_E$ defined by $\mu_E(A) = \mu(E \cap A)$ is also decompsable via $\mathcal{F}$.\retTwo 
% 	\end{enumerate}
% \end{myIndent}

% \exTwo\ul{One generalization of the Lebesgue-Radon-Nikodym Theorem:} 
% \begin{myIndent}
% 	Let $(X, \mathcal{M})$ be a measurable space and let $\mu$ and $\nu$ be decomposable measures on $(X, \mathcal{M})$ via the decomposition $\mathcal{F}$. Then there exists measures $\lambda$ and $\rho$ on $(X, \mathcal{M})$ that are decomposable via $\mathcal{F}$ and which satisfy that $\lambda \perp \mu$, $\rho \ll \mu$, and $\nu = \lambda + \rho$. Moreover, there is an extended $\mu$-integrable function $f: X \to \mathbb{R}$ such that $\rho = f \df \mu$. And if $f^\prime$ also satisfies that $\rho = f^\prime \df \mu$, then $f^\prime = f$ $\mu$-a.e.\retTwo

% 	\exThreeP%
% 	Proof:\\
% 	By considering the positive and negative variations separately and then taking their\\ differences, it suffices to assume that $\nu$ is a positive measure.\retTwo

% 	We first consider the case that $\mu$ is finite and $\nu$ is decomposable on $(X, \mathcal{M})$ via the\\ decomposition $\mathcalli{F}$. For each $F \in \mathcalli{F}$ let $\mu_F(E) \coloneqq \mu(E \cap F)$ and $\nu_F(E) \coloneqq \nu(E \cap F)$\\ for all $E \in \mathcal{M}$. Then since both $\mu_F$ and $\nu_F$ are finite, we know by the typical Lebesgue-\\Radon-Nikodym theorem that there exists positive measures $\lambda_F$ and $f_F\df \mu_F$ such that\\ $\lambda_F \perp \mu_F$ and $\nu_F = \lambda_F + f_F \df \mu_F$. Also, since $\mu_F(F^\comp) = \nu_F(F^\comp) = 0$, we know that\\ $\lambda_F(F^\comp) = \nu_F(F^\comp) - \int_{F^\comp} f_F\df \mu_F = 0$. And, by modifying $f_F$ on the null set $F^\comp$, we can assume $f_F = 0$ outside of $F$.\retTwo

% 	Now $\mathcalli{F}$ is not necessarily countable so this step will require some carefulness. Define\\ $\lambda \coloneqq \sum_{F \in \mathcalli{F}} \lambda_F$ and $f \coloneqq \sum_{F \in \mathcalli{F}} f_F$.

% 	\begin{itemize}
% 		\item To show that $f$ is measurable, suppose $E$ is any measurable set in $\overline{\mathbb{R}}$. Then since all the $F$ are disjoint so that only one $f_F$ is contributing a nonzero amount to $f$ at a time, we know that $F \cap f^{-1}(E) = F \cap f_F^{-1}(E)$ for all $F \in \mathcalli{F}$. It follows from the fact that $f_F$ is measurable then that $F \cap f^{-1}(E) \in \mathcal{M}$ for all $F$.\newpage Hence by property (iv) of decompositions, we know that $f_F^{-1}(E) \in \mathcal{M}$.\retTwo
		
% 		\item 
% 	\end{itemize}


% \end{myIndent}