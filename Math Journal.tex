\documentclass{book}

\usepackage{fontspec} % used to import Calibri
\usepackage{anyfontsize} % used to adjust font size

% needed for inch and other length measurements
% to be recognized
\usepackage{calc}

% for colors and text effects as is hopefully obvious
\usepackage[dvipsnames]{xcolor}
\usepackage{soul}

% control over margins
\usepackage[margin=1in]{geometry}
\usepackage[strict]{changepage}

\usepackage{mathtools}
\usepackage{amsfonts}
\usepackage{bm}

\usepackage[scr=rsfso, scrscaled=.96]{mathalpha}

\usepackage{amssymb} % originally imported to get the proof square
\usepackage{xfrac}
\usepackage[overcommands]{overarrows} % Get my preferred vector arrows...
\usepackage{relsize}

% Just am using this to get a dashed line in a table...
% Also you apparently want this to be inactive if you aren't
% using it because it slows compilation.
\usepackage{arydshln} \ADLinactivate 
\newenvironment{allowTableDashes}{\ADLactivate}{\ADLinactivate}

\usepackage{graphicx}
\graphicspath{{./158_Images/}}

\usepackage{tikz}
   \usetikzlibrary{arrows.meta}
   \usetikzlibrary{graphs, graphs.standard}

\usepackage{quiver} %commutative diagrams


\newfontfamily{\calibri}{Calibri}
\setlength{\parindent}{0pt}
\definecolor{RawerSienna}{HTML}{945D27}

% ~~~~~~~~~~~~~~~~~~~~~~~~~~~~~~~~~~~~~~~~~~~~~~~~~~
%Arrow Commands:

% Thank you Bernard, gernot, and Sigur who I copied this from:
% https://tex.stackexchange.com/questions/364096/command-for-longhookrightarrow
\newcommand{\hooklongrightarrow}{\lhook\joinrel\longrightarrow}
\newcommand{\hooklongleftarrow}{\longleftarrow\joinrel\rhook}
\newcommand{\hookxlongrightarrow}[2][]{\lhook\joinrel\xrightarrow[#1]{#2}}
\newcommand{\hookxlongleftarrow}[2][]{\xleftarrow[#1]{#2}\joinrel\rhook}

% Thank you egreg who I copied from:
% https://tex.stackexchange.com/questions/260554/two-headed-version-of-xrightarrow
\newcommand{\longrightarrowdbl}{\longrightarrow\mathrel{\mkern-14mu}\rightarrow}
\newcommand{\longleftarrowdbl}{\leftarrow\mathrel{\mkern-14mu}\longleftarrow}

\newcommand{\xrightarrowdbl}[2][]{%
  \xrightarrow[#1]{#2}\mathrel{\mkern-14mu}\rightarrow
}
\newcommand{\xleftarrowdbl}[2][]{%
  \leftarrow\mathrel{\mkern-14mu}\xleftarrow[#1]{#2}
}

\newcommand{\mRoman}[1]{%
   \textrm{\MakeUppercase{\romannumeral #1}}%
}



% ~~~~~~~~~~~~~~~~~~~~~~~~~~~~~~~~~~~~~~~~~~~~~~~~~~

\newcommand{\hOne}{%
   \color{Black}%
   \fontsize{14}{16}\selectfont%
}
\newcommand{\hTwo}{%
\color{Black}%
   \fontsize{13}{15}\selectfont%
}
% \newcommand{\scratchWork}{%
%    \color{PineGreen!85!Orange}
%    \fontsize{12}{14}\selectfont%
% }
\newcommand{\hThree}{%
   \color{Black}%
   \fontsize{12}{14}\selectfont%
}
\newcommand{\myComment}{%
   \color{RawerSienna}%
   \fontsize{12}{14}\selectfont%
}
\newcommand{\pracOne}{
   \color{BrickRed}%
   \fontsize{13}{15}\selectfont%
}
\newcommand{\pracTwo}{
   \color{Orange}%
   \fontsize{12}{14}\selectfont%
}
\newcommand{\exOne}{%
   \color{Purple}%
   \fontsize{14}{16}\selectfont%
}
\newcommand{\exTwo}{%
   \color{Purple}%
   \fontsize{13}{15}\selectfont%
}
\newcommand{\exP}{%
   \color{Purple}%
   \fontsize{12}{14}\selectfont%
}
\newcommand{\exTwoP}{%
   \color{RedViolet}%
   \fontsize{13}{15}\selectfont%
}
\newcommand{\exPP}{%
   \color{RedViolet}%
   \fontsize{12}{14}\selectfont%
}
% ~~~~~~~~~~~~~~~~~~~~~~~~~~~~~~~~~~~~~~~~~~~~~~~~

\newcommand{\cyPen}[1]{{\vphantom{.}\color{Cerulean}#1}}
\newcommand{\redPen}[1]{{\vphantom{.}\color{Red}#1}}

\newenvironment{myIndent}{%
   \begin{adjustwidth}{2.5em}{0em}%
}{%
   \end{adjustwidth}%
}

\newenvironment{myDindent}{%
   \begin{adjustwidth}{5em}{0em}%
}{%
   \end{adjustwidth}%
}

\newenvironment{myTindent}{%
   \begin{adjustwidth}{7.5em}{0em}%
}{%
   \end{adjustwidth}%
}

\newenvironment{myConstrict}{%
   \begin{adjustwidth}{2.5em}{2.5em}%
}{%
   \end{adjustwidth}%
}

\newcommand{\udefine}[1]{{%
   \setulcolor{Red}%
   \setul{0.14em}{0.07em}%
   \ul{#1}%
}}

\newcommand{\blab}[1]{\textbf{#1}}

\newcommand{\uuline}[2][.]{%
{\vphantom{a}\color{#1}%
\rlap{\rule[-0.18em]{\widthof{#2}}{0.06em}}%
\rlap{\rule[-0.32em]{\widthof{#2}}{0.06em}}}%
#2}

\newcommand{\pprime}{{\prime\prime}}
\newcommand{\suchthat}{ \hspace{0.3em}s.t.\hspace{0.3em}}
\newcommand{\rea}[1]{\mathrm{Re}(#1)}
\newcommand{\ima}[1]{\mathrm{Im}(#1)}
\newcommand{\comp}{\mathsf{C}}
\newcommand{\myHS}{ \hspace{0.5em}}

\newcommand{\myId}{\mathrm{Id}}
\newcommand{\myIm}{\mathrm{im}}
\newcommand{\myObj}{\mathrm{Obj}}
\newcommand{\myHom}{\mathrm{Hom}}
\newcommand{\myEnd}{\mathrm{End}}
\newcommand{\myAut}{\mathrm{Aut}}

\newcommand{\mcateg}[1]{{\bm{\mathsf{#1}}}}

% Thank you Gonzalo Medina and Moriambar who wrote this on stack exchange:
%https://tex.stackexchange.com/questions/74125/how-do-i-put-text-over-symbols%
\newcommand{\myequiv}[1]{\stackrel{\mathclap{\mbox{\footnotesize{$#1$}}}}{\equiv}}

% Thank you chs who wrote this on stack exchange:
%https://tex.stackexchange.com/questions/89821/how-to-draw-a-solid-colored-circle%
\newcommand{\filledcirc}[1][.]{\ensuremath{\hspace{0.05em}{\color{#1}\bullet}\mathllap{\circ}\hspace{0.05em}}}

%Thank you blerbl who wrote this on stack exchange:
%https://tex.stackexchange.com/questions/25348/latex-symbol-for-does-not-divide
\newcommand{\ndiv}{\hspace{-0.3em}\not|\hspace{0.35em}}

\newcommand{\mySepOne}[1][.]{%
   {\noindent\color{#1}{\rule{6.5in}{1mm}}}\\%
}
\newcommand{\mySepTwo}[1][.]{%
   {\noindent\color{#1}{\rule{6.5in}{0.5mm}}}\\%
}

\newenvironment{myClosureOne}[2][.]{%
   \color{#1}%
   \begin{tabular}{|p{#2in}|} \hline \\%
}{%
   \\ \hline \end{tabular}%
}

\newcommand{\retTwo}{\hfill\bigbreak}

\newcommand{\dispDate}[1]{{
   \color{Black}%
   \fontsize{20}{18}\selectfont%
   #1\retTwo
}}


\title{Math Journal}
\author{Isabelle Mills}


\begin{document}
   \maketitle{}
   \setul{0.14em}{0.07em}
   \calibri\hOne
   
   \dispDate{8/31/2024}
   My goal for today is to work through the appendix to chapter 1 in Baby Rudin. This appendix focuses on constructing the real numbers using Dedikind cuts.\retTwo
   
   \hTwo
   \begin{myIndent}
      We define a \udefine{cut} to be a set $\alpha \subset \mathbb{Q}$ such that:
      \begin{enumerate}
         \item $\alpha \neq \emptyset$
         \item If $p \in \alpha$,\myHS $q \in \mathbb{Q}$, and $q < p$, then $q \in \alpha$.
         \item If $p \in \alpha$, then $p < r$ for some $r \in \alpha$\newline
      \end{enumerate}

      Point 3 tells us that $\alpha$ doesn't have a max element. Also, point 2 directly implies the following facts:
      \begin{itemize}
         \item[a.] If $p \in \alpha$,\myHS $q \in \mathbb{Q}$, and $q \notin \alpha$, then $q > p$.
         \item[b.] If $r \notin \alpha$,\myHS $r, s \in \mathbb{Q}$, and $r < s$, then $s \notin \alpha$.\newline
      \end{itemize}

      As a shorthand, I shall refer to the set of all cuts as $R$.
      \begin{myIndent}\myComment
         An example of a cut would be the set of rational numbers less than $2$.\\
      \end{myIndent}

      Firstly, we shall assign an ordering to $R$. Specifically, given any $\alpha, \beta \in R$, we say that $\alpha < \beta$ if $\alpha \subset \beta$ (a proper subset).

      \begin{myIndent}\exTwo
         Here we prove that $<$ satisfies the definition of an ordering.
         \begin{itemize}
            \item[\mRoman{1}.] It's obvious from the definition of a proper subset that at most one of the following three things can be true: $\alpha < \beta$,\myHS $\alpha = \beta$, and $\beta < \alpha$.\retTwo
            
            Now let's assume that $a \not< \beta$ and $\alpha \not= \beta$. Then $\exists p \in \alpha$ such that $p \notin \beta$. But then for any $q \in \beta$, we must have by fact b. above that $q < p$. Hence $q \in \alpha$, meaning that $\beta \subset \alpha$. This proves that at least one of the following has to be true: $\alpha < \beta$,\myHS $\alpha = \beta$, and $\beta < \alpha$.\retTwo

            \item[\mRoman{2}.] If for $\alpha, \beta, \gamma \in R$ we have that $\alpha < \beta$ and $\beta < \gamma$, then clearly $\alpha < \gamma$ becuase $\alpha \subset \beta \subset \gamma$.\retTwo
         \end{itemize}
      \end{myIndent}

      Now we claim that $R$ equipped with $<$ has the least-upper-bound property.
      \begin{myIndent}\exTwo
         Proof:\\
         Let $A \subset R$ be nonempty and $\beta \in R$ be an upper bound of $A$. Then set\\ $\gamma = \hspace{-0.2em}\bigcup\limits_{\alpha \in A}\hspace{-0.2em}\alpha$. Firstly, we want to show that $\gamma \in R$\retTwo

         Since $A \neq \emptyset$, there exists $\alpha_0 \in A$. And as $\alpha_0 \neq \emptyset$ and $\alpha_0 \subseteq \gamma$ by definition, we know that $\gamma \neq \emptyset$. At the same time, we know that $\gamma \subset \beta$ since $\forall \alpha \in A$,\myHS $\alpha \subset \beta$. Hence, $\gamma \neq \mathbb{Q}$, meaning that $\gamma$ satisfies property 1$.$ of cuts.\retTwo

         Next, let $p \in \gamma$ and $q \in \mathbb{Q}$ such that $q < p$. We know that for some $\alpha_1 \in A$, we have that $p \in \alpha_1$. Hence by property 2$.$ of cuts, we know that $q \in \alpha_1 \subset \gamma$, thus showing that $\gamma$ satisfies property 2$.$ of cuts.\newpage
         
         Thirdly, by property 3$.$ we can pick $r \in \alpha_1$ such that $p < r$ and $r \in \alpha_1 \subset \gamma$. So, $\gamma$ satisfies property 3$.$ of cuts.\retTwo

         With that, we've now shown that $\gamma \in R$. Clearly, $\gamma$ is an upper bound of $A$ since $\alpha \subset \gamma$ for all $\alpha \in A$. Meanwhile, consider any $\delta < \gamma$. Then $\exists s \in \gamma$ such that $s \notin \delta$. And since $s \in \gamma$, we know that $s \in \alpha$ for some $\alpha \in A$. Hence, $\delta < \alpha$, meaning that $\delta$ is not an upper bound of $A$. This shows that $\gamma = \sup A$.\\ [6pt]
      \end{myIndent}

      Secondly, we want to assign $+$ and $\hspace{0.1em}\cdot\hspace{0.1em}$ operations to $R$ so that $R$ is an ordered field.\retTwo
      
      To start, given any $\alpha, \beta \in R$, we shall define $\alpha + \beta$ to be the set of all sums $r + s$ such that $r \in \alpha$ and $s \in \beta$.
      \begin{myIndent}\exTwo
         Here we show that $\alpha + \beta \in R$.
         \begin{enumerate}
            \item Clearly, $\alpha + \beta \neq \emptyset$. Also, take $r^\prime \notin \alpha$ and $s^\prime \notin \beta$. Then $r^\prime + s^\prime > r + s$ for all $r \in \alpha$ and $s \in \beta$. Hence, $r^\prime + s^\prime \notin \alpha + \beta$, meaning that $\alpha + \beta \neq \mathbb{Q}$.\\ [-9pt]
         \end{enumerate}

         Now let $p \in \alpha + \beta$. Thus there exists $r \in \alpha$ and $s \in \beta$ such that $p = r + s$.\\ [-9pt]

         \begin{enumerate}
            \item[2.] Suppose $q < p$. Then $q - s < r$, meaning that $q - s \in \alpha$. Hence,\\ $q = (q - s) + s \in \alpha + \beta$.\retTwo
            
            \item[3.] Let $t \in \alpha$ so that $t > r$. Then $p = r + s < t + s$ and $t + s \in \alpha + \beta$.\retTwo
         \end{enumerate}
      \end{myIndent}

      Also, we shall define $0^*$ to be the set of all negative rational numbers. Clearly, $0^*$ is a cut. Furthermore, we claim that $+$ satisfies the addition requirements of a field with $0^*$ as its $0$ element.

      \begin{myIndent}\exTwo
         Commutativity and associativity of $+$ on $R$ follows directly from the\\ commutativity and associativity of addition on the rational numbers.\retTwo

         Also, for any $\alpha \in R$,\myHS $\alpha + 0^* = \alpha$.
         \begin{myIndent}\exP
            If $r \in \alpha$ and $s \in 0^*$, then $r + s < r$. Hence $r + s \in \alpha$, meaning that $\alpha + 0^* \subseteq \alpha$. Meanwhile, if $p \in \alpha$, then we can pick $r \in \alpha$ such that $r > p$. Then, $p - r \in 0^*$ and $p = r + (p - r) \in \alpha + 0^*$. So, $\alpha \subseteq \alpha + 0^*$.\retTwo
         \end{myIndent}

         Finally, given any $\alpha \in R$, let $\beta = \{p \in \mathbb{Q} \mid \exists\hspace{0.1em} r \in \mathbb{Q}^+ \suchthat -p-r \notin \alpha\}$.
         \begin{myIndent}\myComment
            To give some intuition on this definition, firstly we want to guarentee that for all $p \in \beta$, $-p$ is greater than all elements of $\alpha$. Secondly, we add the $-r$ term to guarentee that $\beta$ doesn't have a maximum element.\\
         \end{myIndent}

         We claim that $\beta \in R$ and $\beta + \alpha = 0^*$. Hence, we can define $-\alpha = \beta$.

         \begin{myIndent}\exP
            To start, we'll show that $\beta \in R$:
            \begin{enumerate}
               \item For $s \notin \alpha$ and $p = -s - 1$, we have that $-p - 1 \notin \alpha$. Hence, $p \in \beta$, meaning that $\beta \neq \emptyset$. Meanwhile, if $q \in \alpha$, then $-q \notin \beta$ because there does not exist $r > 0$ such that $-(-q) - r = q - r \notin \alpha$. So $\beta \neq \mathbb{Q}$.\\ [-6pt]
            \end{enumerate}

            Now let $p \in \beta$ and pick $r > 0$ such that $-p -r \notin \alpha$.\newpage

            \begin{enumerate}
               \item[2.] Suppose $q < p$. Then $-q - r > -p - r$, meaning that $-q - r \notin \alpha$. Hence, $q \in \beta$.\retTwo
               
               \item[3.] Let $t = p + \frac{r}{2}$. Then $t > p$ and $-t - \frac{r}{2} = -p - r \notin \alpha$, meaning $t \in \beta$.\retTwo
            \end{enumerate}

            Now that we've proved $\beta \in R$, we next prove that $\beta$ is the additive inverse of $\alpha$. To start, suppose $r \in \alpha$ and $s \in \beta$. Then $-s \notin \alpha$, meaning that $r < -s$. So $r + s < 0$, thus showing that $\alpha + \beta \subseteq 0^*$.\retTwo

            As for the other inclusion, pick any $v \in 0^*$ and set $w = -\frac{v}{2}$. Then because $w > 0$, we can use the archimedean property of $\mathbb{Q}$ to say that there exists $n \in \mathbb{Z}$ such that $nw \in \alpha$ but $(n+1)w \notin \alpha$. Put $p = -(n + 2)w$. Then $p \in \beta$ because $-p - w = (n+1)w
            \notin \alpha$. And finally, $v = nw + p \in \alpha + \beta$. Thus, $0^* \subseteq \alpha + \beta$.\retTwo
         \end{myIndent}
      \end{myIndent}
   \end{myIndent}
   \dispDate{9/1/2024}
   \begin{myIndent}\hTwo
      Based on the definition of $+$, it's also hopefully clear that for any $\alpha, \beta, \gamma \in R$ such that $\alpha < \beta$, we have that $\alpha + \gamma < \beta + \gamma$.\retTwo

      Next, we shall define multiplication on $R$. Except, first we're going to limit ourselves to the set $R^+$ of all cuts greater than $0^*$. So, given any $\alpha, \beta \in R^+$, we shall define $\alpha \beta$ to be the set of all $p \in \mathbb{Q}$ such that $p \leq rs$ where $r \in \alpha$,\myHS $s \in \beta$,\myHS $r > 0$, and $s > 0$.

      \begin{myIndent}\exTwo
         Here we show that $\alpha\beta \in R^+$.
         \begin{enumerate}
            \item Clearly $\alpha\beta \neq \emptyset$. Also, take any $r^\prime \notin \alpha$ and $s^\prime \notin \beta$. Then $r^\prime s^\prime > rs$ for all $r \in \alpha \cap \mathbb{Q}^+$ and $s \in \beta \cap \mathbb{Q}^+$ since all four rational numbers are positive. By extension, $r^\prime s^\prime$ is greater than all the elements (both positive and negative) of $\alpha\beta$. So, $r^\prime s^\prime \notin \alpha\beta$, meaning that $\alpha\beta \neq \mathbb{Q}$.\\ [-9pt]
         \end{enumerate}

         Now let $p \in \alpha\beta$. Based on our definition of $\alpha\beta$, we know that the conditions of a cut trivially hold for any negative $p$. So, we'll assume from now on that $p > 0$. (Also note that a positive choice of $p$ must exist because both $\alpha$ and $\beta$ by assumption have positive elements.)\retTwo

         Since $p \in \alpha\beta \cap \mathbb{Q}^+$, we know there exists $r \in \alpha$ and $s \in \beta$ such that $p = rs$ and $r, s > 0$.

         \begin{enumerate}
            \item[2.] Suppose $0 < q < p$ (the case where $q \leq 0$ is trivial). Then $\frac{q}{s} < r$, meaning that $\frac{q}{s} \in \alpha$. So, $q = \frac{q}{s} \cdot s \in \alpha\beta$.\retTwo
            
            \item[3.] Let $t \in \alpha$ so that $t > r$. Then $p = rs < ts$ and $ts \in \alpha\beta$.\retTwo
         \end{enumerate}
      \end{myIndent}

      Also, we shall define $1^*$ to be the set of all rational numbers less than $1$. Clearly, $1^*$ is a cut. And we claim that $\hspace{0.1em}\cdot\hspace{0.1em}$ satisfies the multiplication requirements of a field with $1^*$ as its $1$ element.\newpage

      \begin{myIndent}\exTwo
         As before, commutativity and associativity of $\hspace{0.1em}\cdot\hspace{0.1em}$ on $R^+$ follows directly from commutativity and associativity of multiplication on the rational numbers.\retTwo

         Next, for any $\alpha \in R^+$, we have that $\alpha 1^* = \alpha$.
         \begin{myIndent}\exP
            It's clear that for any rational number $r \leq 0$, we have that $r \in \alpha 1^*$ and $r \in \alpha$. So we can exclusively focus on positive rational numbers.\retTwo
            
            Now suppose $r \in \alpha \cap \mathbb{Q}^+$ and $s \in 1^*$. Then $rs < r$, meaning that $rs \in \alpha$. So $\alpha 1^* \subseteq \alpha$. Meanwhile, if $p \in \alpha \cap \mathbb{Q}^+$, then we can pick $r \in \alpha$ such that $r > p$. Then $\frac{p}{r} \in 1^*$ and $p = \frac{p}{r} \cdot r \in \alpha 1^*$. So, $\alpha \subseteq \alpha 1^*$.\retTwo
         \end{myIndent}

         Thirdly, given any $\alpha \in R^+$, define:

         \begin{centering}
            $\beta = \{p \in \mathbb{Q} \mid p \leq 0\} \cup \{p \in \mathbb{Q}^+ \mid \exists r \in \mathbb{Q}^+ \suchthat \frac{1}{q} - r \notin \alpha\}$\retTwo\par
         \end{centering}

         \begin{myIndent}\exP
            Here we show that $\beta \in R^+$.
            \begin{enumerate}
               \item Clearly $\beta \neq \emptyset$. Also, if $q \in \alpha$, then $\frac{1}{q} \notin \beta$. Hence, $\beta \neq \mathbb{Q}$.\retTwo
            \end{enumerate}

            Now let $p \in \beta$ and pick $r > 0$ such that $\frac{1}{p} - r \notin \alpha$. Also, assume $p > 0$ because the proof is trivial if $p \leq 0$. (The fact that $p > 0$ in $\beta$ exists is trivial to show.)\retTwo

            \begin{enumerate}
               \item[2.] If $q \leq 0 < p$, then trivially $q \in \beta$. Meanwhile, if $0 < q < p$, then\\ [2pt] $\frac{1}{q} - r > \frac{1}{p} - r$, meaning that $\frac{1}{q} - r \notin \alpha$. Hence, $q \notin \beta$.\retTwo
               
               \item[3.] Let $t = \frac{1}{\frac{1}{p} - \frac{r}{2}}$. Then since $\frac{1}{p} - r \notin \alpha$, we know that $\frac{1}{p} - \frac{r}{2} > 0$. Also since $\frac{1}{t} = \frac{1}{p} - \frac{r}{2} < \frac{1}{p}$, we have that $t > p$. But note that $\frac{1}{t} - \frac{r}{2} = \frac{1}{p} - r \notin \alpha$.\\[2pt] Hence $t \notin \beta$.\retTwo
            \end{enumerate}
         \end{myIndent}

         We claim that $\beta\alpha = 1^*$. Hence, we can define $\frac{1}{\alpha} = \beta$.

         \begin{myIndent}\exP
            To start, suppose $r \in \alpha \cap \mathbb{Q}^+$ and $s \in \beta \cap \mathbb{Q}^+$. Then $\frac{1}{s} \notin \alpha$, meaning that\\ $r < \frac{1}{s}$. So $rs < 1$, thus showing that $\alpha\beta \subseteq 1^*$.\retTwo

            The other inclusion has a more complicated proof. Firstly, take any\\ $v \in 1^* \cap \mathbb{Q}^+$ (the proof is trivial if $v \leq 0$). Then set $w = \frac{1}{v}$, meaning\\ that $w > 1$. Now since $\alpha \in R^+$, we know there exists $n \in \mathbb{Z}$ such that\\ $w^n \in \alpha$ but $w^{n+1} \notin \alpha$. Then as $w^{n+2} > w^{n+1}$, we know that $\frac{1}{w^{n+2}} \in \beta$.\\ Hence, $v^2 = w^n \frac{1}{w^{n+2}} \in \alpha\beta$.\retTwo

            Now that we've shown that the square of every $v \in 1^* \cap \mathbb{Q}^+$ is also in $\alpha\beta$,\\ [2pt] we next show that there exists $z \in 1^* \cap \mathbb{Q}^+$ such that $z^2 > v$. Suppose $v = \frac{p}{q}$ where $p, q \in \mathbb{Z}^+$. Then set $z = \frac{p + q}{2q}$. Importantly, since $p < q$, we still have that $z \in 1^*$. But also note that:

            \begin{centering}
               $z^2 - v = \frac{p^2 + 2pq + q^2}{4q^2} - \frac{4pq}{4q^2} = \frac{p^2 - 2pq + q^2}{4q^2} = \left(\frac{p - q}{2q}\right)^2 \geq 0$\retTwo\par
            \end{centering}

            Thus as $v \leq z^2$ and $z^2 \in \alpha\beta$, we have that $v \in \alpha\beta$ as well. So $1^* \subseteq \alpha\beta$.\newpage
         \end{myIndent}

         Finally, so long as $\alpha, \beta, \gamma \in R^+$, we have that  $\alpha(\beta + \gamma) = \alpha\beta + \alpha\gamma$ because the rational numbers satisfy the distributive property.\retTwo
      \end{myIndent}

      Notably, in proving that $\alpha\beta \in R^+$ before, we also guarenteed that for $\alpha, \beta > 0$, we have that $\alpha\beta > 0$.\retTwo
   \end{myIndent} 

   \dispDate{9/7/2024}

   \begin{myIndent}

      Now we still need to extend our definition of multiplication from $R^+$ to all of $R$. To do this, set $\alpha 0^* = 0^*\alpha = 0^*$ and define:

      {\centering $\alpha \beta = \left\{
      \begin{matrix}
         (-\alpha)(-\beta) & \text{ if } \alpha < 0^*, \beta < 0^* \\
         -((-\alpha)\beta) & \text { if } \alpha < 0^*, \beta > 0^* \\
         -(\alpha(-\beta)) & \text{ if } \alpha > 0^*, \beta < 0^*
      \end{matrix}\right.$ \retTwo\par}

      Having done that, reproving those properties of multiplication on all of $R$ just\\ becomes a matter of addressing many cases and using the identity that\\ $(-(-\alpha)) = \alpha$.

      \begin{myIndent}\myComment
         Note that that identity can be proven just from the addition properties of a field.\retTwo
      \end{myIndent}

      Because I'm bored with this construction at this point, I'm going to skip reproving those properties.\retTwo

      So now that we've established that $R$ is a field, all we have left to do is to show that all numbers $r, s \in \mathbb{Q}$ are represented by cuts $r^*, s^* \in R$ such that:
      
      \begin{itemize}
         \item $(r + s)^* = r^* + s^*$
         \item $(rs)^* = r^*s^*$
         \item $r < s \Longleftrightarrow r^* < s^*$\retTwo
      \end{itemize}

      Again, I'm super bored and demotivated at this point. So, I'm going to skip showing this.\retTwo

      With all that done, we've now shown that $R$ satisfies all of the properties of real numbers. That concludes the proof of the existence theorem of the real numbers.
      \newpage
   \end{myIndent}

   \dispDate{9/9/2024}

   \hOne
   Today I'm just looking at James Munkres' book \textit{Topology}. Now while I'm done with the era of my life of taking exhaustive notes on a textbook, I still want to write down some interesting proofs. I also hope to do some exercises.\retTwo

   \blab{Theorem 7.8:} Let $A$ be a nonempty set. There is no injective map $f: \mathcal{P}(A) \longrightarrow A$ and there is no surjective map $g: A \longrightarrow \mathcal{P}(A)$.

   \begin{myDindent}\myComment
      In other words, the power set of a set has strictly greater cardinality.\retTwo
   \end{myDindent}

   
   \begin{myIndent}\hTwo
      Proof:\\
      If such an injective $f$ existed, then that would imply a surjective $g$ exists. So, we just need to show that any function $g: A \longrightarrow \mathcal{P}(A)$ isn't surjective.\retTwo

      Let $g: A \longrightarrow \mathcal{P}(A)$ be any function and define $B = \{a \in A \mid a \in A - g(a)\}$.\\ Clearly, $B \subseteq A$. However, $B$ cannot be in the image of $g$. After all, suppose there exists $a_0 \in A$ such that $g(a_0) = B$. Then we get a contradiction because:

      {\centering $a_0 \in B \Longleftrightarrow a_0 \in A - g(a_0) \Longleftrightarrow a_0 \in A - B$ \retTwo\par}

      Hence, $g(A) \neq \mathcal{P}(A)$ and we conclude that $g$ can't be surjective. $\blacksquare$\retTwo\retTwo
   \end{myIndent}

   \exOne
   \blab{Exercise 7.3:} Let $X= \{0, 1\}$. Show there is a bijective correspondence between the set $\mathcal{P}(\mathbb{Z}_+)$ and the Cartesian product $X^\omega$.\retTwo

   
   \begin{myIndent}\exTwo
      For any set $A \in \mathcal{P}(\mathbb{Z}_+)$, define $f(A)$ to be the $\omega$-tuple $\mathbf{x}$ such that for all\\ $i \in \mathbb{Z}^+$, $\mathbf{x}_i = 1$ if $i \in A$ and $\mathbf{x}_i = 0$ if $i \notin A$. Then clearly $f$ is injective as\\ $\forall A, B \in \mathcal{P}(\mathbb{Z}_+)$, $f(A) = f(B) \Longrightarrow A = B$. Also, given any $\mathbf{x} \in X^\omega$, we\\ know that the set $A = \{i \in \mathbb{Z}_+ \mid \mathbf{x}_i = 1\}$ satisfies that $f(A) = \mathbf{x}$,\\ meaning $f$ is surjective.
      
      \retTwo Hence, $f$ is a bijective function between $\mathcal{P}(\mathbb{Z}_+)$ and $X^\omega$.
      \begin{myTindent}\myComment
         Note that this construction still works if $\mathbb{Z}_+$ is replaced with any\\ countably infinite set.\retTwo\retTwo
      \end{myTindent}
   \end{myIndent}

   \blab{Exercise 7.5:} Determine whether the following sets are countable or not.
   \begin{itemize}
      \item[(f)] The set $F$ of all functions $f: \mathbb{Z}_+ \longrightarrow \{0, 1\}$ that are "eventually zero", meaning there is a positive integer $N$ such that $f(n) = 0$ for all $n \geq N$.
      
      \begin{myIndent}\exTwo
         $F$ is countable. To see why, let:
         
         {\centering $A_n = \{f: \mathbb{Z}_+ \longrightarrow \{0, 1\} \mid \forall i \geq n, \myHS f(i) = 0\}$\retTwo\par}
         
         Thus each $A_n$ is finite (with $2^n$ elements) and $F = \bigcup\limits_{n = 1}^\infty A_n$.\newpage
      \end{myIndent}

      \item[(g)] The set $G$ of all functions $f: \mathbb{Z}_+ \longrightarrow \mathbb{Z}_+$ that are eventually $1$.
      
      \begin{myIndent}\exTwo
         $G$ is countable. To see why, let:

         {\centering $A_n = \{f: \mathbb{Z}_+ \longrightarrow \mathbb{Z}_+ \mid \forall i \geq n, \myHS f(i) = 1\}$\retTwo\par}

         Then each $A_n$ has a bijective correspondence with $(\mathbb{Z}_+)^n$, meaning each $A_n$ is countable, and $G = \bigcup\limits_{n = 1}^\infty A_n$.
         
         \begin{myTindent}\myComment
            The same argument applies to all functions $f: \mathbb{Z}_+ \longrightarrow \mathbb{Z}_+$ that are eventually any constant.\retTwo
         \end{myTindent}
      \end{myIndent}

      \item[(h)] The set $H$ of all functions $f: \mathbb{Z}_+ \longrightarrow \mathbb{Z}_+$ that are eventually constant.
      
      \begin{myIndent}\exTwo
         $H$ is countable. To see why, let $A_n$ be the set of all functions\\ $f: \mathbb{Z}_+ \longrightarrow \mathbb{Z}_+$ that are eventually $n$. Because of part g of\\ [-6pt] this exercise, we know that each $A_n$ is countable. Also, $H = \bigcup\limits_{n = 1}^\infty A_n$.\retTwo
      \end{myIndent}

      \item[(i)] The set $I$ of all two-element subsets of $\mathbb{Z}_+$
      \item[(j)] The set $J$ of all finite subsets of $\mathbb{Z}_+$.
      
      \begin{myIndent}\exTwo
         Both $I$ and $J$ are countably infinite. We know this because we can define\\ surjections from $(\mathbb{Z_+})^2$ to $I$ and $\bigcup\limits_{n = 1}^\infty (\mathbb{Z}_+)^n$ to $J$.
         
         \begin{myIndent}
            (Finite cartesian products of countable sets and unions of countably many countable sets are countable.)\retTwo\retTwo
         \end{myIndent}
      \end{myIndent}
   \end{itemize}

   \blab{Exercise 7.6.a:} Show that if $B \subset A$ and there is an injection $f: A \longrightarrow B$, then $|A| = |B|$.
   
   \begin{myIndent}\exTwo
      According to the hint, we set $A_1 = A$ and $A_n = f(A_{n-1})$ for all $n > 1$. Similarly, we set $B_1 = B$ and $B_n = f(B_{n-1})$ for all $n > 1$.\retTwo

      We can assume $A_2$ is a proper subset of $B_1$ because if $A_2 = B_1$, then we already have that $f$ is a bijection. Also, as $f$ is an injection, we know that $B_2 \subset A_2$. Thus by induction, we can conclude that:

      {\centering $ A_1 \supset B_1 \supset A_2 \supset B_2 \supset A_3 \supset B_3 \supset \cdots $\retTwo\par}

      Now, the textbook recommends defining $h: A \longrightarrow B$ by:

      \begin{center}
         $h(x) = \left\{
         \begin{matrix}
            f(x) & & \text{ if } x \in A_n - B_n \text{ for any } n \in \mathbb{Z}_+ \\
            x & & \text{ otherwise }
         \end{matrix}\right.$\newpage
      \end{center}

      
      \begin{myIndent}
         \myComment I want to ask a professor about this definition because it urks me. My issue with\\ this definition of $h$ is that I feel like it should be possible for:
         $$\bigcap\limits_{n=1}^\infty (A_n \cap B_n) \neq \emptyset.$$
         
         However, we wouldn't be able to know that some $x$ is in that intersection and\\ thus falls into case 2 until after an infinite number of steps.\retTwo
         
         On the other hand, $S_1 = \bigcup\limits_{n = 1}^\infty (A_n - B_n)$ is a valid definition for a set, as is\\ $S_2 = A - S_1$. So the definition $h$ is valid because it's saying that $h(x) = f(x)$\\ [6pt] if $x \in S_1$ and $h(x) = x$ if $x \in S_2$.\retTwo

         Maybe my issue is just that I have trouble trusting the validity of a function definition if I can't actually evaluate that function myself. Although, there are lots of functions like that that I don't have any problem with. For example, given $g(x) = 0$ if $x$ is rational and $g(x) = 1$ if $x$ is irrational, what is $g(\pi^2)$?\retTwo
      \end{myIndent}

      Hopefully it is clear that $h$ is in fact a valid function from $A$ to $B$. Now firstly, we shall show that $h$ is injective.

      \begin{myIndent}\exP
         Let $x, y \in A$ such that $x \neq y$. If there are integers $n$ and $m$ such that $x \in A_n - B_n$ and $y \in A_m - B_m$, then $h(x) \neq h(y)$ because $f$ is injective. Meanwhile, if no such $n$ or $m$ exists, then $h(x) \neq h(y)$ because $x \neq y$.\retTwo

         This leaves the case that there exists $n \in \mathbb{Z}_+$ such that $x \in A_n - B_n$ but for\\ all $m \in \mathbb{Z}_+,\myHS y \notin A_m - B_m$. Then, note that $f(x) \in f(A_{n} - B_{n})$. And since\\ $f$ is injective, we thus have that $f(x) \in f(A_{n}) - f(B_{n}) = A_{n+1} - B_{n+1}$.\\ Therefore, as $y \notin A_{n+1} - B_{n+1}$, we know that $h(x) \neq y = h(y)$.\retTwo
      \end{myIndent}

      Next, we show $h$ is surjective.

      \begin{myIndent}\exP
         Let $y \in B$.\retTwo
         
         Suppose there exists $n \in \mathbb{Z}_+$ such that $y \in A_n - B_n$. We know that $n \neq 1$ since $y \in B$. Thus, there must exist $x \in A_{n-1}$ such that $y = f(x) \in f(A_{n-1}) = A_n$. Furthermore, this $x$ can't be in $B_{n-1}$ because otherwise $y$ would be in $B_n$ which we know isn't true. So, $x \in A_{n-1} - B_{n-1}$, meaning that $h(x) = f(x) = y$.\retTwo

         Meanwhile, if no such $n$ exists, then we simply have that $h(y) = y$. Hence,\\ $h(A) = B$.\retTwo
      \end{myIndent}

      Thus, we've shown that $h$ is a bijection, meaning that $|A| = |B|$.\newpage
   \end{myIndent}

   \blab{Exercise 7.7:} Show that $|\{0, 1\}^\omega| = |(\mathbb{Z}_+)^\omega|$.

   
   \begin{myIndent}\exTwo
      Firstly, obviously a bijection exists between $\{0, 1\}^\omega$ and $\{1, 2\}^\omega$. Also,\\ $\{1, 2\}^\omega \subset (\mathbb{Z}_+)^\omega$. So, if we can construct an injective function from $(\mathbb{Z}_+)^\omega$\\ to $\{1, 2\}^\omega$, then we can apply the result of exercise 7.6.a to prove this\\ exercise's claim.\retTwo

      We shall create this injection using a diagonalization argument. Let $x \in (\mathbb{Z}_+)^\omega$.\\ Then we define $f(x) = y \in \{1, 2\}^\omega$ as follows:
      
      \begin{center}
         \begin{tabular}{c}
            $y(1) = 2$ if $x(1) = 1$. Otherwise $y(1) = 1$.\\ [6pt]
            $y(2) = 2$ if $x(1) = 2$. Otherwise $y(2) = 1$.\\
            $y(3) = 2$ if $x(2) = 1$. Otherwise $y(3) = 1$.\\ [6pt]
            $y(4) = 2$ if $x(1) = 3$. Otherwise $y(4) = 1$.\\
            $y(5) = 2$ if $x(2) = 2$. Otherwise $y(5) = 1$.\\
            $y(6) = 2$ if $x(3) = 1$. Otherwise $y(6) = 1$. \\ [6pt]

            $y(7) = 2$ if $x(1) = 4$. Otherwise $y(7) = 1$.\\
            $\vdots$\\ [12pt]
         \end{tabular}
      \end{center}

      Clearly $f$ is an injection since $f(x_1) = f(x_2)$ implies that $x_1$ and $x_2$ have the same integers at all indices.\retTwo\retTwo
   \end{myIndent}

   \blab{Exercise 7.6.b: (Schroeder-Bernstein theorem)} If there are injections $f: A \longrightarrow C$ and $g: C\longrightarrow A$, then $A$ and $C$ have the same cardinality.\retTwo
   \myComment
   I did my work on paper and now it's late and I don't want to write more tonight.\retTwo\retTwo

   \dispDate{9/11/2024}

   \hOne

   Since today's my day off, I'm gonna work through Munkres' textbook \textit{Topology} some more.\retTwo

   \blab{Theorem 8.4 (Principle of recursive definition):} Let $A$ be a set and let $a_0$ be an element of $A$. Suppose $\rho$ is a function assigning an element of $A$ to each function $f$ mapping a nonempty section of the positive integers onto $A$. Then there exists a unique function $h: \mathbb{Z}_+ \longrightarrow A$ such that:

   {\begin{center}
      \begin{tabular}{l c r}
         $(*)$ & \phantom{aaaa} & 
         \begin{tabular}{l r}
            $h(1) = a_0$ & \\
            $h(i) = \rho(h|_{\{1, \ldots, i-1\}})$ & $\text{for } i > 1\text{.}$
         \end{tabular}
      \end{tabular}\retTwo
   \end{center}}

   \newpage
   \begin{myIndent}\hTwo
      Proof outline:

     \begin{myIndent}\hThree
       Part 1: Given any $n \in \mathbb{Z}_+$, there exists a function $f: \{1, \ldots, n\} \longrightarrow A$ that\\ satisfies $(*)$.
 
       \begin{myIndent}\myComment
          This is obvious from induction.\\ [9pt]
       \end{myIndent}

       Part 2: Suppose that $f: \{1, \ldots, n\} \longrightarrow A$ and $g: \{1, \ldots, m\} \longrightarrow A$ both satisfy $(*)$ for all $i$ in their respective domains. Then $f(i) = g(i)$ for all $i$ in both domains.

      \begin{myIndent}
         Proof:\\
         Suppose not. Let $i$ be the smallest integer for which $f(i) \neq g(i)$.\retTwo
         
         We know $i \neq 1$ because $f(1) = a_0 = g(1)$. But then note that\\ $f|_{\{1, \ldots, i - 1\}} = g|_{\{1, \ldots, i- 1\}}$. Hence:
         
         {\centering $f(i) = \rho(f|_{\{1, \ldots, i - 1\}}) = \rho(g|_{\{1, \ldots, i - 1\}}) = g(i)$.\retTwo\par}

         This contradicts that $i$ is the smallest integer for which $f(i) \neq g(i)$.\\ [9pt]
      \end{myIndent}

      Part 3: Let $f_n: \{1, \ldots, n\} \longrightarrow A$ be the unique function satisfying  $(*)$\\ (uniqueness was proven in part 2). Then we define:

      {\centering $h = \bigcup\limits_{i = 1}^\infty f_n$ \retTwo\par}

      \begin{myIndent}\myComment
         Because of part 2, we can fairly easily show that for each $i \in \mathbb{Z}_+$, there is exactly one element in $h$ with $i$ as it's first coordinate. Hence, the set $h$ defines a functions from $\mathbb{Z}_+$ to $A$.\retTwo

         Also, hopefully it's clear that $h$ satisfies $(*)$.\retTwo
      \end{myIndent}
     \end{myIndent}
   \end{myIndent}

   \mySepTwo

   \blab{Axiom of choice}: Given a collection $\mathcal{A}$ of disjoint nonempty sets, there exists a set $C$ consisting of exactly one element from each element of $\mathcal{A}$.

   
   \begin{myIndent}\myComment
      A few notes:
      \begin{enumerate}
         \item If we restrict $\mathcal{A}$ to being a finite collection, then there is nothing controversial about this axiom. It only becomes controversial when $\mathcal{A}$ is allowed to be infinite.
         \item There are multiple instances in baby Rudin where we made an infinite number of\\ arbitrary choices. Looking at a lot of those proofs closer, I think many of them could avoid using the axiom of choice by specifying that we had to pick rational numbers in a set. However, being able to pick elements without worrying about a preexisting choice function is way easier.\retTwo
         
         My take away from this is that not only does it make proofs cleaner to not worry about using constructed choice functions, but it's also perfectly acceptable now-a-days to use this axiom.\newpage
         
         Plus, some really commonly used theorems require the axiom of choice to prove them. For example, the union of countably many countable sets being countable. This makes it really easy to accidentally use the axiom of choice in a proof.\retTwo
      \end{enumerate}
   \end{myIndent}

   \blab{Lemma 9.2: (Existence of a choice function)} Given a collection $\mathcal{B}$ of nonempty sets (not necessarily disjoint), there exists a function \[c: \mathcal{B} \longrightarrow \bigcup\limits_{B \in \mathcal{B}}B\] such that $c(B)$ is an element of $B$ for each $B \in \mathcal{B}$.\retTwo

   
   \begin{myIndent}\hTwo
      Proof:\\
      Given any set $B \in \mathcal{B}$, we define $B^\prime = \{(B, b) \mid b \in B\}$. Because $B \neq \emptyset$, we know that $B^\prime \neq \emptyset$ as well. Furthermore, given $B_1, B_2 \in \mathcal{B}$ if $B_1 \neq B_2$, then we have that the first element of all the pairs in $B_1^\prime$ are different from that of $B_2^\prime$. So $B_1^\prime$ and $B_2^\prime$ are disjoint.\retTwo

      Now form the collection $\mathcal{C} = \{B^\prime \mid B \in \mathcal{B}\}$. From before, we know that $\mathcal{C}$ is\\ a collection of disjoint sets. So by the axiom of choice, there exists a set $c$\\ consisting of exactly one element from each element of $\mathcal{C}$.\retTwo

      This set $c$ is a subset of $\mathcal{B} \times \bigcup\limits_{B \in \mathcal{B}}B$ which satisfies our definition of a choice function.\\ [-12pt]
      \begin{myTindent}\begin{myTindent}\myComment
         Hopefully it's obvious enough why $c$ satisfies\\ those properties.\retTwo\retTwo
      \end{myTindent}\end{myTindent}
   \end{myIndent}

   \mySepTwo

   A set $A$ with an order relation $<$ is said to be \udefine{well-ordered} if every nonempty subset of $A$ has a smallest element.\retTwo

   
   \begin{myIndent}\myComment
      
      {\fontsize{13}{15}\selectfont%
      \blab{Tangent: inductiveness of $\mathbb{Z}_+$ is equivalent to the well-orderedness of $\mathbb{Z}_+$}
      }
      
      \begin{myIndent}
         This proof is taken from https://math.libretexts.org/ on their page for the\\ well-ordering principle.\retTwo

         ($\Longrightarrow$)\\
         Suppose $S$ is a nonempty subset of $\mathbb{Z}_+$ with no least element. Then let $R$ be the set of lower bounds of $S$. Since $1$ is the least element of $\mathbb{Z}_+$, we know that $1 \in R$.\retTwo
         
         Now given any $k \geq 1$, if $k \in R$, we know that $\{1, \ldots, k\}$ must be a subset of $R$. Also note that $R \cap S = \emptyset$ because if that wasn't true, we'd know that $S$ has a least element. Therefore, $\{1, \ldots, k\} \cap S = \emptyset$. But then that shows that $k + 1 \notin S$ since otherwise $k + 1$ would be the least element of $S$. Furthermore, since no element of $\{1, \ldots, k\}$ is in $S$, we automatically have that $k + 1 \in R$.\retTwo

         By induction, this means that $R = \mathbb{Z}_+$. Hence, we get a contradiction as $S$ must be empty.\newpage

         ($\Longleftarrow$)\\
         Let $S$ be a subset of $\mathbb{Z}_+$ such that $1 \in S$ and $k \in S \Longrightarrow k + 1 \in S$. Then suppose that $S \neq \mathbb{Z}_+$. In that case, we know that $S^\comp \neq \emptyset$, and since $\mathbb{Z}_+$ is well-ordered, we know there is a least element $\alpha$ of $S^\comp$.\retTwo

         Because $1 \in S$, we know that $\alpha \geq 2$. But then consider that $1 \leq \alpha - 1 < \alpha$. Therefore, $\alpha - 1 \in S$, thus implying that $\alpha \in S$. This contradicts that $\alpha \in S^\comp$.\retTwo
      \end{myIndent}

      {\fontsize{13}{15}\selectfont%
      From what I've heard, when defining the positive integers, one usally takes one of the two above properties as an axiom and then proves the other as a theorem. In Munkres' book, he starts with induction and proves well-orderedness.\retTwo\retTwo
      }
   \end{myIndent}

   Facts:
   \begin{itemize}
      \item If $A$ with the order relation $<$ is well-ordered, then any subset of $A$ is well-\\ordered as well with $<$ restricted to that subset.
      \item If $A$ has the order relation $<_1$ and $B$ has the order relation $<_2$ and both are well-ordered, then $A \times B$ is well-ordered with the dictionary order.
      \item Given any countable set $A$, we know there exists a bijection $f$ from $A$ to $\mathbb{Z}_+$. Hence, given $a, b \in A$, we can say that $a < b \Longleftrightarrow f(a) < f(b)$. Then, $A$ is well-ordered by $<$ with the least element of any subset $S$ of $A$ being the element $\alpha \in A$ such that $f(\alpha)$ is the least element in $f(S)$.
      \item If a set $A$ is well-ordered, then we can make a choice function $c: \mathcal{P}(A) \longrightarrow A$ using that well-ordering.
      
      \begin{myIndent}
         Specifically, given any $B \subseteq A$, assign $c(B) = \beta$ where $\beta$\\ is the least element of $B$.

         
         \begin{myIndent}\myComment
            This is why we can pick elements of $\mathbb{Q}$ without worrying about the axiom of choice.\retTwo\retTwo
         \end{myIndent}
      \end{myIndent}
   \end{itemize}

   An important theorem (which I will hopefully prove soon) is:
   
   \begin{myIndent}
      \blab{The Well Ordering Theorem:} If $A$ is a set, there exists an order relation on $A$ that is well-ordering.
      
      \begin{myIndent}\myComment
         Note: this theorem requires the axiom of choice to prove.\retTwo
      \end{myIndent}
   \end{myIndent}

   \exOne\blab{Exercise 10.5:} Show that the well-ordering theorem implies the (infinite) axiom of choice.
   \begin{myIndent}\exTwo
      Let $\mathcal{A}$ be a collection of disjoint sets. By the well-ordering theorem, we can pick an order relation on $\bigcup\limits_{A \in \mathcal{A}}\hspace{-0.4em}A$ that is well-ordering.\newpage
      
      \begin{myIndent}\myComment
         Note that the previous sentence is carefully worded to only make use of the finite axiom of choice. Specifically, the order relation we are picking is an element of some subset of $\bigcup\limits_{A \in \mathcal{A}}A \times \bigcup\limits_{A \in \mathcal{A}}A$.\retTwo

         If we had instead picked a well-ordering for each $A \in \mathcal{A}$, then that would require the axiom of choice as we would be making potentially infinitely many arbitrary choices of order relations.\retTwo
      \end{myIndent}

      Now let $C = \{a \in \hspace{-0.2em}\bigcup\limits_{A \in \mathcal{A}}\hspace{-0.2em}A \mid \exists A \in \mathcal{A} \suchthat a \in A \text{ and } \forall b \in A,\myHS a \leq b \}$.\retTwo Then $C$ fulfils the properties of the set that the axiom of choice would guarentee exists.\retTwo
   \end{myIndent}

   \dispDate{9/14/2024}

   \blab{Exercise 10.1:} Show that every well-ordered set has the least-upper-bound\\ property.

   \begin{myIndent}\exTwo
      Let the set $A$ with the order relation $<$ be well-ordered. Then consider any nonempty $B \subseteq A$ and suppose there exists $\alpha \in A$ such that $b < \alpha$ for all $b \in B$.\retTwo

      Let $U = \{a \in A \mid \forall b \in B, \myHS b \leq a\}$. Since $\alpha \in U$, we know that $U \neq \emptyset$. So, because $A$ is well-ordered, we know that $U$ has a least element $\beta$. This $\beta$ is by definition the least upper bound of $B$. So $\sup{B} = \beta$.\retTwo\retTwo
   \end{myIndent}

   \hOne

   Let $X$ be a well-ordered set. Given $\alpha \in X$, let $S_\alpha$ denote the set $\{x \in X \mid x < \alpha\}$. We call $S_\alpha$ the \udefine{section} of $X$ by $\alpha$.\retTwo

   \blab{Lemma 10.2:} There exists a well-ordered set $A$ having a largest element $\Omega$ such that $S_\Omega$ is uncountable but every other section of $A$ is countable.

   \begin{myIndent}\hTwo
      Proof:\\
      Starting off, let $B$ be an uncountable well-ordered set. Then let $C$ be the well-\\ordered set $\{1, 2\} \times B$ with the dictionary order. Clearly, given any $b \in B$, we have that $S_{(2, b)}$ is uncountable. So the set of $c \in C$ such that $S_c$ is uncountable is not empty.\retTwo

      Let $\Omega$ be the least element of $C$ such that $S_\Omega$ is uncountable. Then define\\ $A = S_\Omega \cup \{\Omega\}$. This is called a \udefine{minimal uncountable well-ordered set}.
      \retTwo
      
      \begin{myIndent}\myComment
         The reason we are considering $\{1, 2\} \times B$ instead of just $B$ is that if we were just considering $B$, then we wouldn't be able to guarentee that there exists $b \in B$ such that $S_b$ is uncountable.\newpage

         User MJD on https://math.stackexchange.com wrote some good intuition for why\\ this is.
         \begin{myIndent}
            While the set $\mathbb{Z}_+$ is countably infinite, all sections $S_x$ of $\mathbb{Z}_+$ are finite.\\ However, when considering $\{1, 2\} \times \mathbb{Z}_+$ with the dictionary order, we\\ have that $S_{(2, 1)}$ is countably infinite. Furthermore, all sections of $S_{(2, 1)}$\\ are finite. Thus, $S_{(2, 1)}$ would be a minimal \textit{countable} well-ordered set.\retTwo
         \end{myIndent}
      \end{myIndent}
   \end{myIndent}

   We call a set described by lemma 10.2 $\overline{S}_\Omega = S_\Omega \cup \{\Omega\}$.\retTwo

   \blab{Theorem 10.3:} If $A$ is a countable subset of $S_\Omega$, then $A$ has an upper bound in $S_\Omega$.
   
   \begin{myIndent}\hTwo
      Proof:\\
      Let $A$ be a countable subset of $S_\Omega$. For all $a \in A$, we know that $S_a$ is countable. Therefore, $B = \bigcup\limits_{a \in A}S_a$ is also countable, meaning that $S_\Omega - B \neq \emptyset$.\retTwo

      If we pick $x \in S_\Omega - B$, we must have that $x$ is an upper bound to $A$ because if $x < a$ for some $a \in A$, we would have that $x \in S_a \subseteq B$.\retTwo
      
      
      \begin{myIndent}\myComment
         If you combine this with exercise 10.1, we know that $A$ has a least upper bound.\retTwo
      \end{myIndent}
   \end{myIndent}

   \exOne
   \blab{Exercise 10.6:} Let $S_\Omega$ be a minimal uncountable well-ordered set.
   
   \begin{itemize}
      \item[(a)] Show that $S_\Omega$ has no largest element.
      \begin{myIndent}\exTwo
         Suppose $\alpha \in S_\Omega$ is the largest element of $S_\Omega$. In that case, we'd have that\\ $S_\alpha = S_\Omega - \{\alpha\}$. However, by theorem 10.3, we know that $S_\alpha$ is countable. This implies that $S_\Omega = S_\alpha \cup \{\alpha\}$ must also be countable, which is a contradiction.
         \retTwo
      \end{myIndent}
      \item[(b)] Show that for every $\alpha \in S_\Omega$, the subset $\{x \in S_\Omega \mid \alpha < x\}$ is uncountable.
      \begin{myIndent}\exTwo
         Let $\alpha \in S_\Omega$. By the law of trichotomy, we know that:
   
         {\centering $S_\Omega = \{x \in S_\Omega \mid x < \alpha\} \cup \{\alpha\} \cup \{x \in S_\Omega \mid \alpha < x\}$.\retTwo\par}
         
         Now suppose $\{x \in S_\Omega \mid \alpha < x\}$ is countable. Then as both $\{x \in S_\Omega \mid x < \alpha\}$ and $\{\alpha\}$ are countable, we have a contradiction as the three's union must also be countable. But we know $S_\Omega$ isn't.\retTwo
      \end{myIndent}
      % \item[(c)] Let $X_0$ be the subset of $S_\Omega$ consisting of all elements $x$ such that $x$ has no immediate predecessor. Show that $X_0$ is uncountable.  
   \end{itemize}

   \hOne
   \mySepTwo

   Some definitions I've been lacking:
   \begin{enumerate}
      \item Let $A$ be a set and suppose $x, y, z$ are any three different elements of $A$.
      
      {\centering\fontsize{11}{13}\selectfont%
         \begin{tabular}{|l|l|}
            \udefine{Simple [Default] Order Relation}: ($<$) & \udefine{Strict Partial Order Relation}: ($\prec$) \\ [4pt] \hline &\\ [-9pt]
   
            \begin{tabular}{l}
               Nonreflexitivity: $x \not< x$ \\
               Transitivity: $x < y$ and $y < z \Rightarrow x < z$ \\
               Comparability: $x < y$ or $y < x$ is true
            \end{tabular} &
            \begin{tabular}{l}
               Nonreflexitivity: $x \not\prec x$ \\
               Transitivity: $x \prec y$ and $y \prec z \Rightarrow x \prec z$\\\phantom{a}
            \end{tabular}
         \end{tabular}
      \par}
      \newpage

      
      \begin{myIndent}\myComment
         Basically, a partial order relation is allowed to not give an order for some pairings of elements. If someone just says a set is ordered, they mean the set is simply ordered.\retTwo
      \end{myIndent}

      \item Let $A$ and $B$ be sets ordered by $<_A$ and $<_B$ respectively. We say that $A$ and\\ $B$ have the same \udefine{order type} if there exists an order-preserving bijection\\ $f: A \longrightarrow B$, meaning that $\forall a_1, a_2 \in A,\myHS a_1 <_A a_2 \Longrightarrow f(a_1) <_B f(a_2)$.
      
      \begin{myIndent}\myComment
         It is trivial to show that if $f$ is an order-preserving bijection, then $f^{-1}$ is also an order-\\preserving bijection.\retTwo
      \end{myIndent}

      \item If $A$ is an ordered set and $a$ and $b$ are two different elements, then consider\\ the set $S = \{x \in A \mid a < x < b\}$. If $S = \emptyset$ we say that $b$ is the \udefine{successor} of\\ $a$ and $a$ is the \udefine{predecessor} of $b$.\retTwo\retTwo
   \end{enumerate}

   \exOne
   \blab{Exercise 10.2}:  
   \begin{itemize}
       \item[(a)] Show that in a well-ordered set, every element except the largest (if one exists) has an immediate successor
       
      \begin{myIndent}\exTwo
         Let $A$ be a well-ordered set and let $\alpha$ be any element in $A$ such that there exists $\beta \in A$ for which $\alpha < \beta$. Then consider the set $S = \{x \in A \mid \alpha < x < \beta\}$. If $S = \emptyset$, then we know $\alpha$ has $\beta$ as its successor. Meanwhile, if $S \neq \emptyset$, then since $A$ is well-ordered, we know that $A$ has a least element $\gamma$. Thus, the set $\{x \in A \mid \alpha < x < \gamma\} = \emptyset$ and we know that $\gamma$ is the successor of $\alpha$.\retTwo
      \end{myIndent}

       \item[(b)] Find a set in which every element has an immediate successor that is not well-ordered. 
       \begin{myIndent}\exTwo
         Consider the set $\mathbb{Z}$ of all integers using the standard ordering. Then for any $n \in \mathbb{Z}$, we know that its successor is $n + 1$. At the same time though, the set of all negative integers has no least element. So $\mathbb{Z}$ is not well-ordered by $<$.\retTwo\retTwo
      \end{myIndent}
   \end{itemize}

   \blab{Exercise 10.6:}
   \begin{itemize}
      \item[(c)] Let $X_0$ be the subset of $S_\Omega$ consisting of all elements $x$ such that $x$ has no\\ immediate predecessor. Show that $X_0$ is uncountable.
      
      \begin{myIndent}\exTwo
         Suppose $X_0$ is bounded above by some $\alpha \in S_\Omega$. Thus, there is a predecessor $x \in S_\Omega$ for any $y$ in the set $T = \{z \in S_\Omega \mid z > \alpha\}$.\newpage

         Now define a function $f: \mathbb{Z}_+ \longrightarrow T$ such that $f(1) =$ the least element of $T$ and $f(n) =$ the successor of $f(n - 1)$ for all $n > 1$. We know this function is well-defined because $S_\Omega$ has no largest element according to exercise 10.6.a. So, all elements of $S_\Omega$ and thus $T$ have a successor by exercises 10.2.a, meaning our formula for $f(n)$ is always defined no matter what $f(n-1)$ is. Hence, the principle of recursive definition guarentees a unique $f$ exists.\retTwo

         Now it's easy to show that $f$ is injective. For suppose that given some $x, n \in \mathbb{Z}_+$ we had that $f(x) = f(x + n)$. Then that would mean that:

         {\centering $f(x) < f(x + 1) < \cdots < f(x + n - 1) < f(x + n) = f(x) $\retTwo\par}

         Hence we have a contradiction as $f(x) < f(x)$.\retTwo

         Next, we show that $f$ is surjective. Suppose the set $R = T - f(\mathbb{Z}_+) \neq \emptyset$. Then since $S_\Omega$ and hence $T$ is well-ordered, we know that $R$ has a least element $\beta$. But note that $\beta$ has a predecessor $\gamma$ which isn't in $R$. More specifically, since we know that the least element of $T$ is in $f(\mathbb{Z}_+)$, we know that $\gamma$ is at least the least of element of $T$. So $\gamma \in T$.\retTwo

         Thus we conclude that $\gamma \in T - (T - f(\mathbb{Z}_+)) = f(\mathbb{Z}_+)$, meaning there exists $N$ such that $f(N) = \gamma$. But this means that $f(N + 1) = \beta$, which contradicts that $\beta$ is the least element of $R$.\retTwo

         With that, we've now shown that $f: \mathbb{Z}_+ \longrightarrow T$ is a bijection, meaning that $T$ is countable. However, this contradicts exercise 10.6.b. which asserts that $T$ is uncountable.\retTwo

         Therefore, we conclude that $X_0$ cannot be bounded above. And by theorem 10.3, that means that $X_0$ can't be a countable subset of $S_\Omega$.\retTwo
      \end{myIndent}
   \end{itemize}
   
   \blab{Exercise 10.4:}
   \begin{itemize}
      \item[(a)] Let $\mathbb{Z}_-$ be the set of negative integers in the usual order. Show that a simply ordered set $A$ fails to be well-ordered if and only if it contains a subset having the same order type as $\mathbb{Z}_-$.
      
      
      \begin{myIndent}\exTwo
         ($\Longleftarrow$)\\
         If for some $B \subseteq A$, we have that $f: \mathbb{Z}_- \longrightarrow B$ is an order preserving bijection, then we must have that $B$ has no least element. Hence, not all subsets of $A$ have a least element, meaning that $A$ is not well-ordered.\retTwo

         ($\Longrightarrow$)\\
         If $A$ is not well ordered, then we know there is a set $B \subseteq A$ with no least element. Now using the axiom of choice, choose any $\beta_1 \in B$. Then for all\\ $n > 1$, choose $\beta_n \in B_{\beta_{n-1}}$. In other words, choose $\beta_n \in B$ such that\\ $\beta_n < \beta_{n-1}$.\newpage
         
         Finally, define $f: \mathbb{Z}_- \longrightarrow \{\beta_n \mid n \in \mathbb{Z}_+\}$ by the rule: $f(n) = \beta_{-n}$. This $f$ is an order preserving bijection. Thus, the set $\{\beta_n \mid n \in \mathbb{Z}_+\} \subseteq A$ has the same order type as $\mathbb{Z}_-$.\retTwo
      \end{myIndent}

      \item[(b)] Show that if $A$ is simply ordered and every countable subset of $A$ is well-ordered, then $A$ is well-ordered.
      
      \begin{myIndent}\exTwo
         It's easy to show the contrapositive of this statement.\retTwo

         If $A$ is not well-ordered, then by part a. we know there exists a set $B \subseteq A$ and a function $f: \mathbb{Z}_- \longrightarrow B$ that is an order-preserving bijection. Clearly, $B$ has no least element. Also, the function $g(n) = f(-n)$ gives a bijection from $\mathbb{Z}_+$ to $B$, meaning that $B$ is countable. Hence, we have shown that $B$ is a countable subset of $A$ that is not well-ordered.\retTwo\retTwo
      \end{myIndent}
   \end{itemize}

   \hOne
   Let $J$ be a well-ordered set. A subset $J_0$ of $J$ is said to be \udefine{inductive} if for every $\alpha \in J$, we have that $(S_\alpha \subseteq J_0) \Longrightarrow \alpha \in J_0$.\retTwo

   \exOne\blab{Exercise 10.7: (The principle of transfinite induction)} If $J$ is a well-ordered set and $J_0$ is an inductive subset of $J$, then $J_0 = J$.
   
   \begin{myIndent}\exTwo
      Proof:\\
      Suppose $J_0 \neq J$. That would mean the set $J - J_0$ is nonempty. So let $\alpha$ be the least element of $J - J_0$. We know that $S_\alpha$ must be disjoint to $J - J_0$, meaning that $S_\alpha \in J_0$. But then by the inductiveness of $J_0$, we must have that $\alpha \in J_0$. This contradicts that $\alpha$ is the least element of $J - J_0$.\retTwo\retTwo
   \end{myIndent}

   \blab{Exercise 10.10: (Theorem)} Let $J$ and $C$ be well-ordered sets; assume that there is no surjective function mapping a section of $J$ onto $C$. Then there exists a unique function $h: J \longrightarrow C$ satisfying for each $x \in J$ the equation:\\ [-14pt]
   
   {\centering $(*)$\phantom{aaaaaaaaaaaaaa}$h(x) = \text{smallest element of } C - h(S_x)$.\retTwo\par}

   \begin{myIndent}\exTwo
      Proof:
      \begin{itemize}
         \item[(a)] If $h$ and $k$ map sections of $J$ or all of $J$ into $C$ and satisfy $(*)$ for all $x$ in their domains, then $h(x) = k(x)$ for all $x$ in both domains.
         
         \begin{myIndent}\exPP
            Proof:\\
            Suppose not. Let $y$ be the smallest element of the domains of $h$ and $k$ for which $h(y) \neq k(y)$. Then note that $\forall z \in S_y$, we must have that $h(z) = k(z)$. Thus, we get a contradiction since:

            {\centering $h(y) = \text{smallest}(C - h(S_y)) = \text{smallest}(C - k(S_y)) = k(y)$.\newpage\par}
         \end{myIndent}

         \item[(b)] If there exists a function $h: S_\alpha \longrightarrow C$ satisfying $(*)$, then there exists a function $k: S_\alpha \cup \{\alpha\} \longrightarrow C$ satisfying $(*)$.
         \begin{myIndent}\exPP
            Proof:\\
            Since there is no surjective function mapping a section of $J$ onto $C$, we know that $C - h(S_\alpha) \neq \emptyset$. Hence, we can define $k(x) = h(x)$ for $x < \alpha$ and $k(\alpha) = \text{smallest}(C - h(S_\alpha))$.\retTwo
         \end{myIndent}

         \item[(c)] If $K \subseteq J$ and for all $\alpha \in K$ there exists $h_\alpha: S_\alpha \longrightarrow C$ satisfying $(*)$, then there exists a function $k: \bigcup\limits_{\alpha \in K}S_\alpha \longrightarrow C$ satisfying $(*)$.\\ [-16pt]
         \begin{myIndent}\exPP
            Proof:\\
            Define $k = \bigcup\limits_{\alpha \in K} h_\alpha$.\retTwo

            We know $k$ is a valid function definition because part (a) guarentees that for all $\alpha_1, \alpha_2 \in K$ greater than $x$, we have that $h_{\alpha_1}(x) = h_{\alpha_2}(x)$. Plus,  given any $x \in \bigcup\limits_{\alpha \in K}S_\alpha$, we know that there is $\alpha \in K$ such that $\forall y \in S_x,\myHS k(y) = h_\alpha(y)$.\\ [-16pt]
            \begin{myDindent}
               \phantom{aa.a}This shows that $k$ satisfies $(*)$ at any $x$ due to the relevant $h_\alpha$\\\phantom{a.aa}satisfying $(*)$.\retTwo
            \end{myDindent}
         \end{myIndent}

         \item[(d)] For all $\beta \in J$, there exists a function $h_\beta: S_\beta \longrightarrow C$ satisfying $(*)$.
         \begin{myIndent}\exPP
            Proof:\\
            Let $J_0$ be the set of all $\beta \in J$ for which there exists a function\\ $h_\beta: S_\beta \longrightarrow C$ satisfying $(*)$. Our goal is to show that $J_0$ is\\ inductive. That way, we can conclude by transfinite induction\\ (exercise 10.7) that $J_0 = J$.\retTwo

            Pick any $\beta\in J$ and suppose $S_\beta \in J_0$.
            \begin{myIndent}
               Case 1: $\beta$ has an immediate predecessor $\alpha$.
               \begin{myIndent}
                  Then $S_\beta = S_\alpha \cup \{\alpha\}$. So, knowing that $h_\alpha$ satisfying $(*)$ exists, we can use part (b) to define $h_\beta$ satisfying $(*)$.\retTwo
               \end{myIndent}

               Case 2: $\beta$ has no immediate predecessor.
               \begin{myIndent}
                  Then $S_\beta = \hspace{-0.2em}\bigcup\limits_{\alpha \in S_\beta}\hspace{-0.2em}S_\alpha$.\\ [0pt] And since we assumed that there exists $h_\alpha: S_\alpha \longrightarrow C$ satisfying $(*)$ for all $\alpha \in S_\beta$, we thus know by part (c) that there exists a function from $\hspace{-0.2em}\bigcup\limits_{\alpha \in S_\beta}\hspace{-0.2em}S_\alpha = S_\beta$ to $C$ satisfying $(*)$.\\
               \end{myIndent}
            \end{myIndent}

            Thus in both cases, we have shown that $S_\beta \in J_0$ implies that $h_\beta: S_\beta \longrightarrow C$ satisfying $(*)$ exists. Or in other words, $S_\beta \in J_0 \Longrightarrow \beta \in J_0$.\retTwo
         \end{myIndent}

         \item[(e)] Finally, we now finish proving this theorem.
         \begin{myIndent}\exPP
            Case 1: $J$ has a max element $\beta$.
            \begin{myIndent}
               Then since we know there exists $h_\beta: S_\beta \longrightarrow C$ satisfying $(*)$, we\\ can apply part (b) to get a function $h$ from $J = S_\beta \cup \{\beta\}$ to $C$\\ satisfying $(*)$.\newpage
            \end{myIndent}

            Case 2: $J$ has no max element.
            \begin{myIndent}
               Then $J = \bigcup\limits_{\beta \in J}S_\beta$.\\ And since there exists $h_\beta: S_\beta \longrightarrow C$ satisfying $(*)$ for all $\beta \in J$, we\\ can thus apply part (c) to get a function $h$ from $J = \bigcup\limits_{\beta \in J}S_\beta$ to $C$\\ [-9pt] satisfying $(*)$.\retTwo\retTwo
            \end{myIndent}
         \end{myIndent}
      \end{itemize}
   \end{myIndent}

   \hOne
   \mySepTwo

   \dispDate{9/17/2024}

   \blab{Theorem (The Hausdorff maximum principle):} Let $A$ be a set and let $\prec$ be a strict partial order on $A$. Then there exists a maximal simply ordered subset $B$ of $A$.
   
   \begin{myTindent}\hThree
      In other words, there exists a subset $B$ of $A$ such that $B$ is simply ordered by $\prec$ and no subset of $A$ that properly contains $B$ is simply ordered by $\prec$.\retTwo
   \end{myTindent}
   
   \begin{myIndent}\hTwo
      Proof:\\
      To start out, let $J$ be a set well-ordered by $<$ such that the elements of $A$\\ are indexed in a bijective fashion by the elements of $J$. In other words,\\ $A = \{a_\alpha \in A \mid \alpha \in J\}$.
      \begin{myIndent}\myComment
         Assuming the well-ordering theorem, we know that $J$ exists. Specifically let $J$ refer to the same set as $A$ but equip $J$ with the well-ordering $<$ that we know exists instead of the partial ordering $\prec$ which we equipped $A$.\retTwo
      \end{myIndent}

      Now our goal is to construct a function $h: J \longrightarrow \{0, 1\}$ such that $h(\alpha) = 1$ if $a_\alpha$\\ is in our maximal simply ordered subset of $A$ and $h(\alpha) = 0$ otherwise. To do this, we rely on the \textbf{general principle of recursive definition}.\retTwo

      \begin{myIndent}
         \begin{myClosureOne}{5.1}
            \\ [-20pt]\textbf{Theorem: (General principle of recursive definition):}
            
            \begin{myIndent}
               Let $J$ be a well-ordered set and $C$ be any set. Given a\newline function $\rho: \mathcal{F} \longrightarrow C$ where $\mathcal{F}$ is the set of all functions\newline mapping sections of $J$ into $C$, we have that there exists a\newline unique functon $h: J \longrightarrow C$ satisfying that $h(\alpha) = \rho(h|_{S_\alpha})$\newline for all $\alpha \in J$.
               \begin{myIndent}\myComment
                  The proof for this is supplementary exercise 1. of this\newline chapter. But I'm not going to do it because it's mostly\newline identical to exercise 10.10.
               \end{myIndent}
            \end{myIndent}
         \end{myClosureOne}\retTwo
      \end{myIndent}

      Given any $\alpha \in J$ and $f: S_\alpha \longrightarrow \{0, 1\}$, define $\rho(\alpha) = 1$ if $a_\alpha \in A$ is comparable to all $a_\beta \in A$ such that $\beta \in f^{-1}(1)$ (the preimage of $1$).
      \begin{myTindent}\myComment
         Note that $a_\alpha$ is comparable to $a_\beta$ if either $a_\alpha \prec a_\beta$ or $a_\beta \prec a_\alpha$.\newpage
      \end{myTindent}

      Then by the general principle of recursive definition, we know a unique function $h: J \longrightarrow \{0, 1\}$ exists such that for all $\alpha \in J$, we have that $h(\alpha) = 1$ only when $a_\alpha$ is comparable to all $a_\beta \in A$ such that $\beta \in S_\alpha$ and $h(\beta) = 1$.\retTwo

      Let $B = \{a_\alpha \in A \mid \alpha \in J \text{ and } h(\alpha) = 1\}$. Then given any $a_\alpha, a_\beta \in B$ such that $\alpha < \beta$, we know that either $a_\alpha \prec a_\beta$ or $a_\beta \prec a_\alpha$. Hence, $B$ is simply ordered by $\prec$. At the same time, if $a_\gamma \notin B$, then we know $h(\gamma) = 0$, meaning there exists $a_\alpha \in B$ such that $\alpha < \gamma$ and  $a_\gamma$ is not comparable to $a_\alpha$. This shows that any set properly containing $B$ is not simply ordered by $\prec$.

      
      \begin{myIndent}\myComment
         Note that the maximal simply ordered subset $B$ is not unique. In fact, choosing a different well-ordering of $J$ is likely to give a completely different maximal simply ordered subset.\retTwo
         
         Also, $B$ is not empty because any set with one element is simply ordered by $\prec$.\retTwo\retTwo
      \end{myIndent}
   \end{myIndent}

   Let $A$ be a set and let $\prec$ be a strict partial order on $A$. If $B$ is a subset of $A$, we say an \udefine{upper bound} on $B$ is an element $c$ of $A$ such that for every $b \in B$, either $b = c$ or $b \prec c$. A \udefine{maximal element} of $A$ is an element $m$ of $A$ such that for no element $a$ of $A$ does the relation $m \prec a$ hold.\retTwo

   \blab{Zorn's Lemma:} Let $A$ be a set that is strictly partially ordered. If every simply ordered subset of $A$ has an upper bound in $A$, then $A$ has a maximal element.

   \begin{myIndent}\hTwo
      Proof:\\
      By the Hausdorff maximum principle, there exists a maximal simply ordered subset $B$ of $A$. Let $c$ be an element of $A$ that is an upperbound to $B$. We claim that $c$ is a maximal element of $A$. For suppose there exists $d \in A$ such that $c \prec d$. We know $d \notin B$ since that would imply $d \prec c$. But by the transitivity of $\prec$, we know that $b \preceq c \prec d \Longrightarrow b \prec d$ for all $b \in B$. Hence, $B \cup \{d\}$ is simply ordered by $\prec$. This contradicts that $B$ is a maximal simply ordered subset of $A$.\retTwo\retTwo
   \end{myIndent}

   \exOne\blab{Exercise 11.1:} If $a$ and $b$ are real numbers, define $a \prec b$ if $b - a$ is positive and rational. 
   \begin{itemize}\exTwo
      \item It's easy to show that $\prec$ is a strict partial order. After all, for all $a \in \mathbb{R}$, we have that $a - a$ is not positive. Also, if $a \prec b$ and $b \prec c$, then we know that $b - a = p$ and $c - b = q$ where $p, q \in \mathbb{Q}_+$. But then $c - a = c - b + b - a = p + q \in \mathbb{Q}_+$. So $a \prec c$.\retTwo
      \item Clearly, given any $x \in \mathbb{R}$, the maximal simply ordered set containing $x$ is the set\\ $\{x + p \mid p \in \mathbb{Q}\}$.\newpage
   \end{itemize}

   \pracOne\mySepTwo

   \blab{Tangent:} I never got around to writing this down last quarter. So here's a proof that\\ assuming the axiom of choice, non-Lebesgue measurable sets exist.
   
   \begin{myIndent}\pracTwo
      Let $\mathcal{B}$ be the collection of sets of the form $S_x = [0, 1] \cap \{x + p \mid p \in \mathbb{Q}\}$ where $x$ is any real number. Obviously, all the sets in $\mathcal{B}$ are nonempty. We also claim that all the sets in $\mathcal{B}$ are disjoint. For suppose $S_x, S_y \in \mathcal{B}$ and $S_x \cap S_y \neq \emptyset$. Then fix $c \in S_x \cap S_y$ and consider any $a \in S_x$ and $b \in S_y$.\retTwo

      We know $c - x = p_1$, $a - x = p_2$, $c - y = q_1$, and $b - y = q_2$ where $p_1, p_2, q_1, q_2 \in \mathbb{Q}$. Thus, we have that $a - y = (a - x) + (x - c) + (c - y) = p_2 - p_1 + q_1 \in \mathbb{Q}$. Similarly, we have that $b - x = (b - y) + (y - c) + (c - x) = q_2 - q_1 + p_1 \in \mathbb{Q}$. This tells us that $a \in S_y$ and $b \in S_x$. And since this works for all $a \in S_x$ and $b \in S_y$, we thus must have that $S_x = S_y$.\retTwo

      Now using the axiom of choice, let $V$ be a set containing one element from each set in $\mathcal{B}$.\retTwo

      To show that $V$ is nonmeasurable, we'll reach a contradiction by supposing $V$ is\\ measurable. Let $q_1, q_2, \ldots$ be an enumeration of all the rational numbers in the set $[-1, 1]$. Then having defined $V + q_n = \{v + q_n \mid v \in V\}$, consider the set: $\hspace{-0.3em}\bigcup\limits_{n \in \mathbb{Z}_+}\hspace{-0.3em}(V + q_n)$.\\

      Obviously, since $V \subseteq [0,1]$, we know that $\hspace{-0.3em}\bigcup\limits_{n \in \mathbb{Z}_+}\hspace{-0.3em}(V + q_n) \subseteq [-1, 2]$.\\

      Also, consider any $x \in [0, 1]$ and let $v$ be the element of $V$ which was chosen from the set $S_x \in \mathcal{B}$. Then $v - x = p$ where $p$ is some rational number in $[-1, 1]$. So, we also know that $[0, 1] \subseteq \hspace{-0.3em}\bigcup\limits_{n \in \mathbb{Z}_+}\hspace{-0.3em}(V + q_n)$. This means that $1 \leq \mu(\hspace{-0.3em}\bigcup\limits_{n \in \mathbb{Z}_+}\hspace{-0.3em}(V + q_n)) \leq 3$.\\

      But now note that for any $n, m \in \mathbb{Z}_+$, we have that $n \neq m \Longrightarrow V + q_n \cap V + q_m = \emptyset$. To prove this, assume $V + q_n \cap V + q_m \neq \emptyset$. Thus, there would exist $v, u \in V$ such that $v + q_n = u + q_m$. In turn, we'd have that $v - u = q_m - q_n \in \mathbb{Q}$, which means that\\ $v \in S_u$. However, this contradicts that $V$ has only one element of $S_u$.\\ [-4pt]

      Now since $\mu$ is countably additive, we have that $\mu(\hspace{-0.3em}\bigcup\limits_{n \in \mathbb{Z}_+}\hspace{-0.3em}(V + q_n)) = \sum\limits_{n = 1}^\infty \mu(V + q_n)$.\\
      
      Finally, note that $\mu(V) = \mu(V + q_n)$ for all $n$. Thus $\sum\limits_{n = 1}^\infty \mu(V + q_n) = \sum\limits_{n = 1}^\infty \mu(V)$ is either\\ [-6pt] $0$ or $\infty$.\retTwo
      
      But this contradicts our earlier finding that the measure was between $1$ and $3$. So, we conclude that $V \notin \mathfrak{M}(\mu)$. $\blacksquare$
   \end{myIndent}

   \mySepTwo 

   \exOne\blab{Exercise 11.2:} 
   \begin{itemize}
      \item[(a)] Let $\prec$ be a strict partial order on the set $A$. Define a (non-strict partial) relation $\preceq$ on $A$ by letting $a \preceq b$ if either $a \prec b$ or $a = b$. Show that this relation has the following properties which are called the \textit{partial order axioms}:\newpage
      
      \begin{itemize}\exTwo
         \item[(i)] $a \preceq a$ for all $a \in A$
         \begin{myIndent}
            This is true because $a = a$ for all $x \in A$.
         \end{myIndent}

         \item[(ii)] $a \preceq b \text{ and } b \preceq a \Longrightarrow a = b$.
         \begin{myIndent}
            Given any $a, b\in A$ such that $a \preceq b \text{ and } b \preceq a$, if $a \neq b$, then we'd have that $a \prec b$ and $b \prec a$. This gives a contradiction since $a \prec b \prec a \Longrightarrow a \prec a$ which is not allowed.
         \end{myIndent}

         \item[(iii)] $a \preceq b \text{ and } b \preceq c \Longrightarrow a \preceq c$
         \begin{myIndent}
            Proving this is a matter of considering six rather trivial cases.\retTwo
         \end{myIndent}
      \end{itemize}

      \exOne\item[(b)] Let $P$ be a relation on $A$ satisfying the three axioms above. Define a relation $S$ on $A$ by letting $a\mathop{S}b$ if $a\mathop{P}b$ and $a \neq b$. Show that $S$ is a strict partial order on $A$.
      
      \begin{myIndent}\exTwo
         Obviously, $a \hspace{-0.1em}\not{\hspace{-0.24em}\mathop{S}}\hspace{0.2em} a$ for all $a \in A$ since $a = a$ for all $a \in A$. Meanwhile, suppose\\ $a \mathop{S} b$ and $b \mathop{S} c$. Then we know that $a \mathop{P} b$ and $b \mathop{P} c$, meaning that $a \mathop{P} c$. So we just need to show that $a \neq c$ and then we will have proven that $a \mathop{S} c$.

         Suppose $a = c$. Then we know that $c \mathop{P} a$ and $a \mathop{P} b$, meaning that $c \mathop{P} b$. But then since $b \mathop{P} c$, we know that $b = c$. This contradicts that $b \mathop{S} c$.\retTwo
      \end{myIndent}
   \end{itemize}

   {\hOne In the next exercises we will explore some equivalent theorems to the Hausdorff maximum principle and Zorn's lemma.}\retTwo
   \blab{Exercise 11.5:} Show that Zorn's lemma implies the following:
   
   \begin{myIndent}
      \blab{Kuratowski's Lemma:} Let $\mathcal{A}$ be a collection of sets. Suppose that for every subcollection $\mathcal{B}$ of $\mathcal{A}$ that is simply ordered by proper inclusion, the union of the elements of $\mathcal{B}$ belongs to $\mathcal{A}$. Then $\mathcal{A}$ has an element that is properly contained in no other element of $\mathcal{A}$.\retTwo
      
      \begin{myIndent}\exTwo\color{RedViolet}
         To be clear, given any $A, B \in \mathcal{A}$, we defined above that $A \prec B$ if $A \subset B$. Importantly, our assumption about $\mathcal{A}$ means that every subcollection $\mathcal{B}$ of $\mathcal{A}$ that is simply ordered by $\prec$ has an upper bound in $\mathcal{A}$: $\bigcup\limits_{B \in \mathcal{B}}\hspace{-0.2em}B$.\retTwo
         
         Thus by Zorn's lemma, we know that $\mathcal{A}$ has a maximal element $C$. And since there is no element $D \in \mathcal{A}$ such that $C \prec D$, we know that $C$ is properly contained by no sets in $\mathcal{A}$.\retTwo
      \end{myIndent}
   \end{myIndent}

   \blab{Exercise 11.6:} A collection $\mathcal{A}$ of subsets of a set $X$ is said to be of \textit{finite type}\\ provided that a subset $B$ of $X$ belongs to $\mathcal{A}$ if and only if every finite subset\\ of $B$ belongs to $\mathcal{A}$. Show that the Kuratowski lemma implies the following:

   
   \begin{myIndent}
      \blab{Tukey's Lemma:} Let $\mathcal{A}$ be a collection of sets. If $\mathcal{A}$ is of finite type, then $\mathcal{A}$ has an element that is properly contained in no other element of $\mathcal{A}$.\newpage
      
      \begin{myIndent}\exTwo\color{RedViolet}
         To start off I want to clarify that $\mathcal{A}$ being of finite types means both that:
         \begin{enumerate}
            \item[1.] For each $A \in \mathcal{A}$, every finite subset of $A$ belongs to $\mathcal{A}$.
            \item[2.] If every finite subset of a given set $A$ belongs to $\mathcal{A}$, then $A$ belongs to $\mathcal{A}$.\\
         \end{enumerate}

         Now let $\mathcal{B}$ be any subcollection of $\mathcal{A}$ that is simply ordered by proper\\ inclusion. Next, consider the set $S = \bigcup\limits_{B \in \mathcal{B}}B$. We want to show that any\\ [-8pt] finite subset of $S$ is in $\mathcal{A}$.\retTwo

         To do this, let $n \in \mathbb{Z}_+$ and consider any subset $\{b_1, b_2, \ldots, b_n\}$ of $S$ with $n$ elements. Note that for each $1 \leq i \leq n$, there exists $B_i \in \mathcal{B}$ such that $b_i \in B_i$. Then since $\{B_1, B_2, \ldots B_n\}$ is a simply ordered finite set, we know that it has a maximum element $B_m$ such that $B_i \subseteq B_m$ for all $i$. Hence, we have that $\{b_1, b_2, \ldots b_n\}$ is contained by some $B_m$ in $\{B_1, B_2, \ldots, B_n\} \subseteq \mathcal{B}$. Because $\mathcal{A}$ is of finite type, this tells us that $\{b_1, b_2, \ldots b_n\} \in \mathcal{A}$.\retTwo

         Since we showed above that any finite subset of $S$ is in $\mathcal{A}$, we can thus conclude because $\mathcal{A}$ is of finite type that $S \in \mathcal{A}$. And so, we have now proven the hypothesis of Kuratowski's lemma, meaning that $\mathcal{A}$ must have a set that is properly contained in other element of $\mathcal{A}$.\retTwo
      \end{myIndent}
   \end{myIndent}

   \blab{Exercise 11.7:} Show that the Tukey lemma implies the Hausdorff maximum\\ principle.

   \begin{myIndent}\exTwo\color{RedViolet}
      Let $A$ be a set with the strict partial order $\prec$. Then let $\mathcal{A}$ be the collection of all subsets of $A$ that are simply ordered by $\prec$. We shall show below that $\mathcal{A}$ is of finite type.
      
      \begin{enumerate}
         \item Suppose $B \in \mathcal{A}$. Then given any subset $C$ of $B$ (finite or not), we know that $C$ is also simply ordered by $\prec$. So $C \in \mathcal{A}$.
         \item Let $B \subseteq A$ and suppose every finite subset of $B$ is in $\mathcal{A}$. Then given any two different elements $b_1, b_2 \in B$, we know that $\{b_1, b_2\} \in \mathcal{A}$, meaning that either $b_1 \prec b_2$ or $b_2 \prec b_1$. In other words, $B$ is simply ordered by $\prec$, meaning that $B \in \mathcal{A}$.\retTwo
      \end{enumerate}

      Because $\mathcal{A}$ is of finite type, we know that $\mathcal{A}$ has an element that is properly\\ contained in no other element of $\mathcal{A}$. Or in other words, there exists a subset of $A$ which is simply ordered by $\prec$ and not properly contained in any other subset of $A$ that is simply ordered by $\prec$.\newpage
   \end{myIndent}

   \hOne
   \dispDate{9/19/2024}
   % I realize I have another week before I take my first abstract algebra class. However, the next problem has to do with vector spaces.\newpage
   % 
   % If $A$ is a subset of the vector space $V$, we say a vector belongs to the \textit{span} of $A$ if it equals a finite linear combination of elements of $A$.
   % 
   %    
   % \begin{myIndent}\myComment
   %    In other words, $w \in \mathrm{span}(A)$ if there exists a finite subset $\{v_1, \ldots, v_n\}$ of $A$ and scalars: $c_1, \ldots, c_n$ such that $w = c_1v_1 + \ldots c_nv_n$.\retTwo
   % 
   %    I didn't realize before now but we exclude linear combinations of infinite sets of vectors in $A$. Apparently,
   % \end{myIndent}
   % 
   % \exOne
   % \blab{Exercise 11.8:}

\begin{tabular}{p{2.2in} p{4in}}
   % In the past 14 pages we've\newline learned a lot about the\newline axiom of choice. All the\newline {\color{blue}blue arrows} in the\newline diagram to the right\newline represent proofs we've\newline already done. Meanwhile,\newline the {\color{red}red arrows} represent\newline proofs that Munkres left\newline to the supplementary\newline exercises of section 1 of\newline his book. We're gonna do\newline those proofs now.
   &
   % https://q.uiver.app/#q=WzAsNyxbMCwxLCJcXHRleHR7IEF4aW9tIG9mIENob2ljZSB9Il0sWzAsMiwiXFx0ZXh0e1dlbGwgT3JkZXJpbmcgVGhlb3JlbX0iXSxbMSwwLCJcXHRleHR7Tm9uLW1lYXN1cmFibGUgU2V0c30iXSxbMCwzLCJcXHRleHR7TWF4aW11bSBQcmluY2lwbGV9Il0sWzAsNCwiXFx0ZXh0e1pvcm4ncyBMZW1tYX0iXSxbMSw0LCJcXHRleHR7S3VyYXRvd3NraSdzIExlbW1hfSJdLFsxLDMsIlxcdGV4dHtUdWtleSdzIExlbW1hfSJdLFswLDIsIiIsMix7ImxldmVsIjoyLCJjb2xvdXIiOlsyNDAsNjAsNjBdfV0sWzAsMSwiIiwwLHsib2Zmc2V0IjotNSwibGV2ZWwiOjIsImNvbG91ciI6WzAsNjAsNjBdfV0sWzEsMCwiIiwwLHsib2Zmc2V0IjotNSwibGV2ZWwiOjIsImNvbG91ciI6WzI0MCw2MCw2MF19XSxbMSwzLCIiLDAseyJvZmZzZXQiOi01LCJsZXZlbCI6MiwiY29sb3VyIjpbMjQwLDYwLDYwXX1dLFszLDEsIiIsMCx7Im9mZnNldCI6LTUsImxldmVsIjoyLCJjb2xvdXIiOlswLDYwLDYwXX1dLFszLDQsIiIsMCx7ImxldmVsIjoyLCJjb2xvdXIiOlsyNDAsNjAsNjBdfV0sWzQsNSwiIiwwLHsibGV2ZWwiOjIsImNvbG91ciI6WzI0MCw2MCw2MF19XSxbNSw2LCIiLDAseyJsZXZlbCI6MiwiY29sb3VyIjpbMjQwLDYwLDYwXX1dLFs2LDMsIiIsMCx7ImxldmVsIjoyLCJjb2xvdXIiOlsyNDAsNjAsNjBdfV1d
   \begin{tikzcd}[sep=scriptsize]
      & {\text{Non-measurable Sets}} \\
      {\text{ Axiom of Choice }} \\
      {\text{Well Ordering Theorem}} \\
      {\text{Maximum Principle}} & {\text{Tukey's Lemma}} \\
      {\text{Zorn's Lemma}} & {\text{Kuratowski's Lemma}}
      \arrow[color={rgb,255:red,92;green,92;blue,214}, Rightarrow, from=2-1, to=1-2]
      \arrow[shift left=5, color={rgb,255:red,214;green,92;blue,92}, Rightarrow, from=2-1, to=3-1]
      \arrow[shift left=5, color={rgb,255:red,92;green,92;blue,214}, Rightarrow, from=3-1, to=2-1]
      \arrow[shift left=5, color={rgb,255:red,92;green,92;blue,214}, Rightarrow, from=3-1, to=4-1]
      \arrow[shift left=5, color={rgb,255:red,214;green,92;blue,92}, Rightarrow, from=4-1, to=3-1]
      \arrow[color={rgb,255:red,92;green,92;blue,214}, Rightarrow, from=4-1, to=5-1]
      \arrow[color={rgb,255:red,92;green,92;blue,214}, Rightarrow, from=5-1, to=5-2]
      \arrow[color={rgb,255:red,92;green,92;blue,214}, Rightarrow, from=5-2, to=4-2]
      \arrow[color={rgb,255:red,92;green,92;blue,214}, Rightarrow, from=4-2, to=4-1]
   \end{tikzcd}
\end{tabular}
\\ [-2.6in]
\begin{tabular}{p{2.2in} p{4in}}
   In the past 14 pages, we've\newline learned a lot about the\newline axiom of choice. All the\newline {\color{blue}blue arrows} in the\newline diagram to the right\newline represent proofs we've\newline already done. Meanwhile,\newline the {\color{red}red arrows} represent\newline proofs that Munkres left\newline to the supplementary\newline exercises of section 1 of\newline his book. We're gonna do\newline those proofs now.
   &
\end{tabular}\retTwo

\exOne
\blab{Exercise 1: (General principle of recursive definition)}
\begin{myIndent}\exTwo\color{RedViolet}
   We already addressed this before. I'm skipping proving this because the proof is mostly identical to exercise 10.10. In fact, exercise 10.10 is just this exercise but with a specific $\rho: \mathcal{F} \longrightarrow C$.\retTwo
\end{myIndent} 

\blab{Exercise 2:} 
\begin{itemize}
   \item[(a)] Let $J$ and $E$ be well-ordered sets and let $h: J \longrightarrow E$. Show that the following two statement are equivalent:
   \begin{enumerate}
      \item[(i)] $h$ is order preserving and its image is $E$ or a section of $E$.
      \item[(ii)] $h(\alpha) = \text{smallest}(E - h(S_\alpha))$ for all $\alpha \in J$. 
      
      \begin{myIndent}\exTwo\color{RedViolet}
         (i) $\Longrightarrow$ (ii):\\
         Given any $\alpha \in J$, we know that $h(\alpha)$ must be an upper bound to $h(S_\alpha)$. Now suppose $\exists \beta \in S_{h(\alpha)}$ such that $\beta \notin h(S_\alpha)$. Because of our assumption about the image of $h$, we know that $\beta \in h(J)$, meaning there exists $\gamma \in J$ such that $h(\gamma) = \beta$. But because $h$ is order-preserving, we must have that $\beta < f(\alpha) \Longrightarrow \gamma < \alpha$. This contradicts that $\beta \notin h(S_\alpha)$.\retTwo

         With that, we've now shown that $h(S_\alpha) = S_{h(\alpha)}$. In turn, this shows that $h(\alpha)$ is the smallest element in $E - h(S_\alpha)$.\retTwo

         (ii) $\Longrightarrow$ (i):\\
         It's easy to show $h$ is order preserving. Let $\alpha, \beta \in J$ such that $\alpha < \beta$.\\ Then $h(S_\alpha) \subset h(S_\beta)$, meaning that $E - h(S_\beta) \subset E - h(S_\alpha)$. And\\ since the least element of $E - h(S_\alpha)$ is not in $E - h(S_\beta)$, that means that\\ $h(\alpha) = \text{smallest}(E - h(S_\alpha)) < \text{smallest}(E - h(S_\beta)) = h(\beta)$.\newpage

         As for showing the other property of $h$, let $J_0 = \{\alpha \in J \mid h(S_\alpha) = S_{h(\alpha)}\}$. Now suppose that for some $\alpha \in J$, we have that $S_\alpha \subseteq J_0$. Then we can show that $\alpha \in J_0$.
         
         \begin{myIndent}\exPP
            Case 1: $\alpha$ has an immediate predecessor $\beta$.
            
            \begin{myIndent}
               Then $S_\alpha = S_\beta \cup \{\beta\}$, meaning that:
   
               {\centering $h(S_\alpha) = h(S_\beta) \cup \{h(\beta)\} = S_{h(\beta)} \cup \{h(\beta)\}$.\retTwo\par}
               
               Since $h(\alpha)$ is the least element of $E$ not in $h(S_\alpha)$. We can thus say that $S_{h(\beta)} \cup \{h(\beta)\} = S_{h(\alpha)}$.\retTwo
            \end{myIndent}

            Case 2: $\alpha$ has no immediate predecessor.

            \begin{myIndent}
               Then we have that $h(S_\alpha) = h(\hspace{-0.2em}\bigcup\limits_{\beta \in S_\alpha}\hspace{-0.4em}S_\beta) = \hspace{-0.2em}\bigcup\limits_{\beta \in S_\alpha}\hspace{-0.4em}h(S_\beta) = \hspace{-0.2em}\bigcup\limits_{\beta \in S_\alpha}\hspace{-0.4em}S_{h(\beta)}$.\retTwo

               Hence, $h(S_\alpha)$ is a section of $E$, and since $h(\alpha)$ is the least element not in that section, we can conclude that $h(S_\alpha) = S_{h(\alpha)}$.\retTwo
            \end{myIndent}
         \end{myIndent}

         By transfinite induction, we thus know that $J_0 = J$. So finally, we consider two cases.\retTwo

         \begin{myIndent}\exPP
            Case 1: $J$ has a max element $\alpha$.
            
            \begin{myIndent}
               Then $h(J) = h(S_\alpha) \cup \{h(\alpha)\} = S_{h(\alpha)} \cup \{h(\alpha)\}$. And since $h(\alpha)$ is the least element not in $S_{h(\alpha)}$, we thus know that $h(J)$ is either a section of or the whole of $E$.
            \end{myIndent}

            Case 2: $J$ has no max element.

            \begin{myIndent}
               Then $h(J) = h(\hspace{-0.1em}\bigcup\limits_{\alpha \in J}\hspace{-0.3em}S_\alpha) = \hspace{-0.1em}\bigcup\limits_{\alpha \in J}\hspace{-0.3em}h(S_\alpha) = \hspace{-0.1em}\bigcup\limits_{\alpha \in J}\hspace{-0.3em}S_{h(\alpha)}$.\retTwo
               
               So, $h(J)$ is either a section of or the whole of $E$.\retTwo
            \end{myIndent}
         \end{myIndent}
      \end{myIndent}
   \end{enumerate}

   \item[(b)] If $E$ is a well-ordered set, show that no section of $E$ has the same order type as $E$, nor do any two different sections of $E$ have the same order type.
   
   
   \begin{myIndent}\exTwo\color{RedViolet}
      Let $J$ be any well-ordered set. By combining part (a) of this exercise with\\ exercise 10.10 (which is a special case of the general principle of recursive\\ definition), we know that there is at most one order preserving map from $J$\\ to $E$ whose image is either $E$ or a section of $E$. Hence, $J$ can only have the\\ same order type as one of either the entirety of $E$ or one section of $E$.\retTwo

      Based on that fact, we can get an easy contradiction if we assume that the claim of part (b) is false.\newpage
   \end{myIndent}
\end{itemize}

\hOne
\dispDate{9/21/2024}

Unfortunately I tested positive for Covid on the two days ago. So I've been really delirious. However, right now I'm in an airport in the process of moving back out to California (great idea). And since my flight just got delayed, I feel like I might as well kill time and try to do some math.\retTwo

\exOne\blab{Exercise 3:} Let $J$ and $E$ be well-ordered sets, and suppose there is an order-preserving map $k: J \longrightarrow E$. Using exercises 1 and 2, show that $J$ has the order type of one of either $E$ or one section of $E$.\retTwo


\begin{myIndent}\exTwoP
   Pick any $e_0 \in E$. Then define $h: J \longrightarrow E$ by the rule: 
   
   {\centering$h(\alpha) = \left\{
   \begin{matrix}
      \text{smallest}(E - h(S_\alpha)) & \text{ if } h(S_\alpha) \neq E \\
      e_0 & \text{ otherwise }
   \end{matrix}\right.$\retTwo\par} 

   Note that the second case of our definition of $h$ is just included to ensure that $h$ is well-defined before we begin the proof in earnest. I mention that because our goal now is to show that the second case will never apply.\retTwo

   Let $J_0 = \{\alpha \in J \mid h(\alpha)\leq k(\alpha)\}$. Then suppose that for some $\alpha \in J$, we have that $S_\alpha \subseteq J_0$. Because $k$ is order preserving, we know that $k(\alpha) > k(\beta) \geq h(\beta)$ for all $\beta \in S_\alpha$. Hence, $k(\alpha) \notin h(S_\alpha)$, meaning that $h(S_\alpha) \neq E$. So, we conclude that $h(\alpha) = \text{smallest}(E - h(S_\alpha))$. And since $k(\alpha) \in E - h(S_\alpha)$, we thus know that $h(\alpha) \leq k(\alpha)$\retTwo
   
   Therefore, $\alpha \in J_0$. By transfinite induction, this proves that $J = J_0$. The reason this is relavent is that we can now say that $k(\alpha)$ is never in $h(S_\alpha)$ , meaning that $E - h(S_\alpha) \neq \emptyset$. So $h(\alpha)$, will never be determined by the second case of our definition above.\retTwo

   By exercise 2, we know that $h: J \longrightarrow E$ is the unique order-preserving map whose image is either $E$ or a section of $E$. Thus, $J$ has the same order type as exactly one of either the entirety of $E$ or one section of $E$.\retTwo
\end{myIndent}

\blab{Exercise 4:} Use exercises 1-3 to prove the following:

\begin{itemize}
   \item[(a)] If $A$ and $B$ are well-ordered sets, then exactly one of the following three\\ conditions holds: $A$ and $B$ have the same order type, $A$ has the order type of\\ a section of $B$, or $B$ has the order type of a section of $A$.
   
   
   \begin{myIndent}\exTwoP
      To start off, it's relatively easy to show that at most one of the above three cases is true. After all, $A$ having the same order type as $B$ as well as a section of $B$ contradicts exercise 2. Similarly $B$ having the same order type as $A$ as well as a section of $A$ contradicts exercise 2.\newpage

      Meanwhile, to find a contradiction if $A$ has the order type of $S_\beta$ and $B$ has the order type of $S_\alpha$ where $\alpha \in A$ and $\beta \in B$, let $h: A \longrightarrow S_\alpha$ be the function defined by the rule $h(a) = g(f(a))$ where $f$ is the order-preserving bijection from $A$ to $S_\beta$ and $g$ is the order-preserving bijection from $B$ to $S_\alpha$.\retTwo

      Then given any $a, b \in A$, we know that:
      
      {\centering $a < b \Rightarrow f(a) < f(b) \Rightarrow h(a) = g(f(a)) < g(f(b)) = h(b)$.\retTwo\par}

      Hence, $h$ is an order preserving map from $A$ to $S_\alpha$. This gives us a contradiction since exercise 3 would then imply that $A$ has the same order type as either $S_\alpha$ or a section of $S_\alpha$ (which would still be a section of $A$).\retTwo

      Now, what's left to show is that at least one of the three above cases must be true. Unfortunately, the hinted route for showing this uses an exercise I didn't do. And right now I really don't want to do that exercise. So I'm just going to write out the thing I was supposed to have proven earlier.\retTwo

      {\centering\color{VioletRed}
      \begin{myClosureOne}{5}
         \\ [-20pt]\textbf{Exercise 10.8.a:}
         
         \begin{myIndent}
            Let $A_1$ and $A_2$ be disjoint sets well-ordered by $<_1$ and\newline $<_2$ respectively. Then define an order relation on $A_1 \cup A_2$\newline by letting $a < b$ either if $a, b \in A_1$ and $a <_1 b$, or if\newline $a, b \in A_2$ and $a <_2 b$, or if $a \in A_1$ and $b \in A_2$. This is a\newline well-ordering of $A_1 \cup A_2$.
         \end{myIndent}
      \end{myClosureOne}\retTwo\par}

      Let $A^\prime = \{A\} \times A$ and let $B^\prime = \{B\} \times B$. That way, so long as $A \neq B$, we know that $A^\prime$ and $B^\prime$ are disjoint. (The case where $A = B$ is trivial.)\retTwo
      
      It's hopefully obvious that the well-orderings of $A$ and $B$ can be used to well-\\order $A^\prime$ and $B^\prime$. For $A^\prime$, define $(A, a_1) <_{A^\prime} (A, a_2)$ if $a_1 <_A a_2$. Similarly, define the analogous ordering for $B^\prime$. Clearly, $A$ and $A^\prime$ have the same order type, as do $B$ and $B^\prime$. Also, given any $\alpha \in A$ and $\beta \in B$,\myHS $S_\alpha$ and $S_{(A, \alpha)}$ have the same order type, as do $S_\beta$ and $S_{(B, \beta)}$\retTwo

      Next, define a well-ordering on $A^\prime \cup B^\prime$ by letting $a^\prime < b^\prime$ if either $a^\prime, b^\prime \in A^\prime$ and $a^\prime <_{A^\prime} b^\prime$, or if $a^\prime, b^\prime \in B^\prime$ and $a^\prime <_{B^\prime} b^\prime$, or if $a^\prime \in A^\prime$ and $b^\prime \in B^\prime$.\retTwo

      Note that the inclusion function from $B^\prime$ to $A^\prime \cup B^\prime$ is an order-preserving\\ map. Thus, by exercise 3, we know that $B^\prime$ has the order type of one of either $A^\prime \cup B^\prime$ or one section of $A^\prime \cup B^\prime$.

      \begin{myIndent}
         Case 1: $B^\prime$ has the order type of a section $S_\alpha$ of $A^\prime \cup B^\prime$.
         
         \begin{myIndent}\exPP
            If $\alpha \in A^\prime$, then $B^\prime$ has the order type of a section of $A^\prime$, meaning $B$ has the order type of a section of $A$.\newpage

            If $\alpha$ is the first element of $B$, then $B^\prime$ has the same order type as $A^\prime$, meaning $B$ has the same order type as $A$.\retTwo

            If $\alpha \in B^\prime$, then there exists an order preserving bijection from $B^\prime$ to $A^\prime \cup \{b \in B^\prime \mid b <_{B^\prime} \alpha\}$. So let $f$ be the inverse of that bijection but with it's domain restricted to just $A^\prime$. Since $f$ is also an order-preserving map, we know by exercise 3 that $A^\prime$ has the order type of either $B^\prime$ or a section of $B^\prime$. This would mean that $A$ has the order type of either $B$ or a section of $B$.\retTwo
         \end{myIndent}

         Case 2: $B^\prime$ has the order type of $A^\prime \cup B^\prime$.
         \begin{myIndent}\exPP
            Let $f$ be the inverse of the order preserving bijection from $B^\prime$ to $A^\prime$, except with it's inverse restricted to just $A^\prime$. Since $f$ is also an order-preserving map, we know by exercise 3 that $A^\prime$ has the order type of either $B^\prime$ or a section of $B^\prime$. This would mean that $A$ has the order type of either $B$ or a section of $B$.\retTwo
         \end{myIndent}

         With that, we've now shown that at least one of the three cases posed by the exercise will always be true.\retTwo
      \end{myIndent}
   \end{myIndent}

   \item[(b)] Suppose that $A$ and $B$ are well-ordered sets that are uncountable such that every section of $A$ and of $B$ is countable. Show that $A$ and $B$ have the same order type.
   
   \begin{myIndent}\exTwoP
      If $A$ did not have the same order type as $B$, then by part (a) of this exercise we would know that either $A$ has the order type of a section of $B$ or $B$ has the order type of a section of $A$. However, that would suggest the existence of a bijection between a countable set and an uncountable set, which by definition is not possible.\retTwo
   \end{myIndent}
\end{itemize}

\dispDate{9/23/2024}

\blab{Exercise 5:} Let $X$ be any set and let $\mathcal{A}$ be the collection of all pairs $(A, <)$ where $A$ is a subset of $X$ and $<$ is a well-ordering of $A$. Define:

{\center $(A, <) \prec (A^\prime, <^\prime)$ \retTwo\par}

if $(A, <)$ equals a section of $(A^\prime, <^\prime)$.

\begin{myIndent}\myComment
   In other words, $A = S_\alpha = \{a \in A^\prime \mid a <^\prime \alpha \}$ where $\alpha \in A^\prime$, and $<$ is the order relation $<^\prime$ restricted to $A$.
\end{myIndent}


\begin{itemize}
   \item[(a)] Show that $\prec$ is a strict partial order on $\mathcal{A}$.
   
   \begin{myIndent}\exTwoP
      Clearly no $A$ is a section of itself. So $(A, <) \not\prec (A, <)$.\retTwo Also if $(A, <_A) \prec (B, <_B) \prec (C, <_C)$, then we know that $A$ is a section of a section of $C$ (which is still a section). Plus, $<_A$ is just $<_C$ restricted to $<_A$. Hence, $(A, <) \prec (C, <_C)$.\newpage
   \end{myIndent}

   \item[(b)] Let $\mathcal{B}$ be a subcollection of $\mathcal{A}$ that is simply ordered by $\prec$. Define $B^\prime$ to be the union of the sets $B$ for all $(B, <) \in \mathcal{B}$, and define $<^\prime$ to be the union of the relations $<$ for all $(B, <) \in \mathcal{B}$. Show that $(B^\prime, <^\prime)$ is a well-ordered set.
   
   \begin{myIndent}
      \exTwoP
      To start, let's quickly double check that $<^\prime$ is a valid order relation on $B^\prime$.
      \begin{itemize}
         \item[(i)] Given any $b \in B^\prime$, if $b \in B$ for any $(B, <) \in \mathcal{B}$, then we know that\\ $(b, b) \notin {<}$. So $(b, b) \notin {<^\prime}$.\\ [-9pt]
         \item[(ii)] Suppose $a, b \in B^\prime$ such that $(a, b) \notin {<^\prime}$. Then for all $(B, <) \in \mathcal{B}$ such that $a, b \in B$, we know that $(a, b) \notin {<}$, meaning that $(b, a) \in {<}$. So $(b, a) \in {<^\prime}$.\\ [-9pt]
         \item[(iii)] Given $a, b, c \in B^\prime$, suppose $a <^\prime b <^\prime c$. Then there exists $(B_1, <_1)$ and\\ $(B_2, <_2)$ in $\mathcal{B}$ such that $(a, b) \in {<_1}$ and $(b, c) \in {<_2}$. Now by how we defined $\mathcal{B}$, we know that either ${<_1} \subset {<_2}$ or ${<_2} \subset {<_1}$. Thus, we know $(a, b), (b, c) \in \{<_i\}$ for some $i \in \{1, 2\}$. Hence, $(a, c) \in {<_i}$, meaning that $(a, c) \in {<^\prime}$.\retTwo
      \end{itemize}

      Next, we show that $B^\prime$ is well-ordered by $<^\prime$.\retTwo
      
      Let $S \subseteq B^\prime$ be nonempty and pick any element $\beta$ in $S$. Then we know there exists $(B_1, <_1) \in \mathcal{B}$ such that $\beta \in B_1$. Also, $B_1$ is well-ordered by $<$. So let $\alpha$ be the least element (using $<_1$) of $B_1 \cap S$.\retTwo

      We claim that $\alpha$ is the least element (using $<^\prime$) of $S$. To prove this, suppose there exists $c \in S$ such that $c <^\prime a$. Then we know $(c, \alpha) \in {<_2}$ for some\\ $(B_2, <_2) \in \mathcal{B}$. Importantly, $(B_1, <_1) \neq (B_2, <_2)$ since otherwise we'd have chosen $\alpha$ differently. So one must be a section of the other.
      
      \begin{itemize}
         \item[\bullet] If $(B_2, <_2)$ is a section of $(B_1, <_1)$, then we know that ${<_2} \subset {<_1}$ and\\ $c \in B_1 \cap S$. But this contradicts how we chose $\alpha$.\\ [-9pt]
         
         \item[\bullet] If $(B_1, <_1)$ is a section of $(B_2, <_2$), then we know there exists $\gamma \in B_2$ such that $B_1 = S_\gamma \subseteq B_2$. If $c <_2 \gamma$, then we know that $c \in B_1$ and thus $B_1 \cap S$. This contradicts how we chose $\alpha$. So we must have that $\gamma <_2 c$. But then this also gives us a contradiction as $\alpha <_2 \gamma <_2 c \Longrightarrow \alpha <_2 c$, meaning that $\alpha <^\prime c$.\retTwo
      \end{itemize}
   \end{myIndent}

   \item[(c)] [Not in the book...] Given any $\mathcal{B}$ from part (b) of this problem and defining\\ $(B^\prime, <^\prime)$ as before, we have that $(B, <) \preceq (B^\prime, <^\prime)$ for all $(B, <) \in \mathcal{B}$.
   \begin{myIndent}
      \exTwoP
      Consider any $(B_1, <_1) \in \mathcal{B}$. If $B_1 \neq B^\prime$, then we know there exists $\alpha \in B^\prime - B_1$, thus meaning there exists $(B_2, <_2) \in \mathcal{B}$ such that $\alpha \in B_2$. Since $B_2 \not\prec B_1$, we know that $B_1 \prec B_2$, meaning that $B_1 = S_\beta \subseteq B_2$ for some $\beta \in B_2$.\newpage

      Now we know that $\{b \in B^\prime \mid b <^\prime \beta\} \subseteq \{b \in B_2 \mid b <_2 \beta\}$. For suppose\\ there exists $a$ in the former set but not the latter set. Then there must exist\\ $(B_3, <_3)  \in \mathcal{B}$ such that $(a, b) \in {<_3}$.\retTwo

      If $a \in B_2$, then we'd have that $(b, a) \in <_2$. But that would imply that $(a, b)$ and $(b, a)$ are in $<^\prime$ which we know isn't possible. So we know that $B_3 \not\subseteq B_2$.\retTwo

      Since $\mathcal{B}$ is simply ordered by $\prec$ and we can't have that $B_3 \prec B_2$, we know that $B_2 \prec B_3$. So $B_2 = S_\gamma$ where $\gamma \in S_3$. Now $a <_3 \gamma$ would contradict that\\ $a \notin B_2$. So we must have that $\gamma <_3 a$. However, we also must have that\\ $b <_3 \gamma$, which contradicts that $a <_3 b$.\retTwo

      Hence, we've shown that $(B_1, <_1) \neq (B^\prime, <^\prime)$ implies that $(B_1, <_1) \prec (B^\prime, <^\prime)$.\retTwo
   \end{myIndent}
\end{itemize}

\blab{Exercise 6:} Use exercise 5 to prove that the maximum principle implies the well-\\ordering theorem.

\begin{myIndent}\exTwoP
   Let $X$ be any set and construct $\mathcal{A}$ and $\prec$ as before in exercise 5. By the maximal principle, we know there exists $\mathcal{B} \subseteq \mathcal{A}$ such that $\mathcal{B}$ is simply ordered by $\prec$ and no proper superset of $\mathcal{B}$ is simply ordered by $\prec$.\retTwo

   Next, construct $B^\prime$ and $<^\prime$ as in exercise 5.b. We claim that $B^\prime = X$. To see this, suppose there exists $c \in X - B^\prime$. Then $B^\prime \cup \{c\}$ is well-ordered by the order\\ relation: ${<^\prime}\cup\{(b, c) \mid b \in B^\prime\}$. Hence $(B^\prime \cup \{c\}, {<^\prime}\cup\{(b, c) \mid b \in B^\prime\}) \in \mathcal{A}$.\retTwo

   At the same time, note that $(B^\prime, <^\prime) \prec (B^\prime \cup \{c\}, {<^\prime}\cup\{(b, c) \mid b \in B^\prime\})$. And since we have that $(B, <) \preceq (B^\prime, <^\prime)$ for all $(B, <) \in \mathcal{B}$, we thus know that for any $(B, <) \in \mathcal{B}$:

   {\centering $(B, <) \prec (B^\prime \cup \{c\}, {<^\prime}\cup\{(b, c) \mid b \in B^\prime\})$.\retTwo\par}

   This tells us both that $(B^\prime \cup \{c\}, {<^\prime}\cup\{(b, c) \mid b \in B^\prime\}) \notin \mathcal{B}$ and that\\ $(B^\prime \cup \{c\}, {<^\prime}\cup\{(b, c) \mid b \in B^\prime\})$ is comparable with all elements of $\mathcal{B}$. But\\ that contradicts that $\mathcal{B}$ is a maximal simply ordered subset of $\mathcal{A}$.\retTwo

   So we must have that $B^\prime = X$. And thus by exercise 5.b, we know that a well-ordering of $X$ exists.\newpage
\end{myIndent}

\dispDate{9/24/2024}

\blab{Exercise 7:} Use exercises 1-5 to prove that the choice axiom implies the well-ordering theorem.\retTwo

Let $X$ be a set and $c$ be a fixed choice function for the nonempty subsets of $X$. If $T$ is a subset of $X$ and $<$ is a relation on $T$, we say that $(T, <)$ is a \udefine{tower} in $X$ if $<$ is a well-ordering of $T$ and if for each $x \in T$, $x = c(X - S_x(T))$ where $S_x(T)$ is the section of $T$ by $x$.
\begin{myTindent}\myComment
   Well, shit. I wish I was given that notation for specifying which set I was\\ taking a section of before I did exercise 2. $h(S_x(J)) = S_{h(x)}(E)$ is a lot\\ clearer notation than just $h(S_x) = S_{h(x)}$\retTwo
\end{myTindent}

\begin{itemize}
   \item[(a)] Let $(T_1, <_1)$ and $(T_2, <_2)$ be two towers in $X$. Show that either these two\\ ordered sets are the same or one equals a section of the other.
   
   \begin{myIndent}\exTwoP
      By applying exercise 4 and switching indices if necessary, we know that $T_1$ has the order type of one of either $T_2$ or one section of $T_2$. In other words, there exists an order preserving map $h: T_1 \longrightarrow T_2$ such that $h(T_1)$ equals $T_2$ or a section of $T_2$.\retTwo

      Now we assert that given any $x \in T_1$,\myHS $h(x) = x$. To prove this, first note\\ that because of transfinite induction, we can assume that $h(x) = x$ for all $x$ in $S_x(T_1)$. This means that we can assume $h(S_x(T_1)) = S_x(T_1)$. Also, as part of doing exercise 2, we proved that $h$ must satisfy that $h(S_x(T_1)) = S_{h(x)}(T_2)$.\\ Hence, $S_x(T_1) = S_{h(x)}(T_2)$. This let's us conclude that:

      {\centering $x = c(X - S_x(T_1)) = c(X - S_{h(x)}(T_2)) = h(x)$.\retTwo\par}

      With that we now know that $h(T_1) = T_1$. So $T_1$ equals either $T_2$ or a section of $T_2$.\retTwo
   \end{myIndent}

   \item[(b)] If $(T, <)$ is a tower in $X$ and $T \neq X$, then there is a tower in $X$ of which $(T, <)$ is a section.
   
   \begin{myIndent}\exTwoP
      Since $T \neq X$, let $y = c(X - T)$. Then define $T^\prime = T \cup \{y\}$ and\\ ${<^\prime} = {<} \cup \{(x, y) \mid x \in T\}$. Clearly, $(T^\prime, <^\prime)$ is a tower which contains\\ $(T, <)$ as a section.\retTwo

      
      \begin{myIndent}\exPP
         Clearly $T^\prime$ is well-ordered by $<^\prime$.\retTwo
         
         Also, if $x \in T^\prime - \{y\}$, then we have that $c(X - S_x(T^\prime)) = c(X - S_x(T)) = x$.\\ Plus, we know that $c(X - S_y(T^\prime)) = c(X - T) = y$.\newpage
      \end{myIndent}
   \end{myIndent}

   \item[(c)] Let $\{(T_k, <_k) \mid k \in K\}$ be the collection of all towers in $X$. Then define:
   
   {\centering $ T = \bigcup\limits_{k \in K}T_k$\hspace{0.2em} and \hspace{0.2em}${<} = \bigcup\limits_{k \in K}\hspace{-0.2em}{<_k}$. \retTwo\par}

   Show that $(T, <)$ is a tower in $X$. Conclude that $T = X$.

   \begin{myIndent}\exTwoP
      If we define $\mathcal{A}$ and $\prec$ from $X$ as we did in exercise 5, we can see from part (a) of this problem that $\{(T_k, <_k) \mid k \in K\}$ is a subset of $\mathcal{A}$ that is simply ordered by $\prec$. Thus, from part (b) of exercise 5, we know that $T$ is well-ordered by $<$.\retTwo

      To prove that $T$ is a tower, consider any $y \in T$. Then we know there exists $k \in K$ such that $y \in T_k$. Furthermore, we know that $y = c(X - S_y(T_k))$. By, part (c) of exercise 5, we know that $T_k$ is either a section of $T$ or all of $T$. Hence, $S_y(T) = S_y(T_k)$. And thus we have that $y = c(X - S_y(T))$.\retTwo

      Now that we have shown $(T, <)$ is a tower in $X$, we get an easy contradiction if $T \neq X$. This is because $T$ must contain all towers, but $T$ not equalling $X$ would imply the existence of a tower not contained by $T$ due to part (b) of this exercise.\retTwo

      And since $T = X$, we thus have that $<$ is a well-ordering of $X$. $\blacksquare$\retTwo
   \end{myIndent}
\end{itemize}

\hOne
I'm gonna skip doing exercise 8 of the supplementary exercise. Basically it shows that you can construct a well-ordered set with higher cardinality than an arbitrary well-ordered set, all without using the axiom of choice. Also, while that does mean we can construct a minimal uncountable well-ordered set without using the axiom of choice, theorem 10.3 requires the axiom of choice to prove. So almost nothing we discovered about a minimal uncountable well-ordered set can be proven without the axiom of choice.\retTwo

\mySepTwo

\dispDate{9/25/2024}

I'm gonna try to cram as much topology as I can today before class starts tomorrow. After all, I suspect and fear that a bunch of this will be necessary at some point in 240. As before, I'm shamelessly ripping off James Munkres' book.\retTwo

A \udefine{Topology} on a set $X$ is a collection $\mathcal{T}$ of subsets of $X$ having the properties:
\begin{enumerate}
   \item $\emptyset$ and $X$ are in $\mathcal{T}$.
   \item The union of the elements of any subcollection of $\mathcal{T}$ is in $\mathcal{T}$.
   \item The intersection of the elements of any finite subcollection of $\mathcal{T}$ is in $\mathcal{T}$.\newpage
\end{enumerate}

Technically, a topological space is an ordered pair $(X, \mathcal{T})$ consisting of a set $X$ and a topology $\mathcal{T}$ on $X$. But when no confusion will arise, we usually omit mentioning $\mathcal{T}$ and just call $X$ a topological space.\retTwo

Given a topological space $(X, \mathcal{T})$, we say that a subset $U$ of $X$ is an \udefine{open set} if $U \in \mathcal{T}$.\retTwo

Suppose $\mathcal{T}$ and $\mathcal{T}^\prime$ are topologies on $X$ Such that $\mathcal{T} \subseteq \mathcal{T}^\prime$. Then we say $\mathcal{T}^\prime$ is \udefine{finer} or \udefine{larger} than $\mathcal{T}$ Also, we say $\mathcal{T}$ is \udefine{coarser} or \udefine{smaller} than $\mathcal{T}^\prime$. And we say both are \udefine{comparable} with each other.

\begin{myIndent}\hThree
   If $\mathcal{T}$ is properly contained by $\mathcal{T}^\prime$, then we add the word \textit{strictly} before those adjectives.\retTwo
\end{myIndent}

\mySepTwo

If $X$ is a set, a \udefine{basis} for a topology on $X$ is a collection $\mathcal{B}$ of subsets of $X$ (called \udefine{basis elements}) such that:
\begin{enumerate}
   \item For each $x \in X$, there is at least one basis element $B$ containing $x$.
   \item If $x$ belongs to the intersection of two basis elements $B_1$ and $B_2$, then there is a basis element $B_3$ containing $x$ such that $B_3 \subseteq B_1 \cap B_2$.\retTwo
\end{enumerate}

If $\mathcal{B}$ satisfies these two conditions, then we define the \udefine{topology $\mathcal{T}$ generated by $\mathcal{B}$} as follows:
\begin{myIndent}
   $U \subseteq X$ is open if for each $x \in U$, there is a basis element $B \in \mathcal{B}$ such that $x \in B$ and $B \subseteq U$.\retTwo
\end{myIndent}

\blab{Proof that the $\mathcal{T}$ generated by $\mathcal{B}$ is a topology:}
\begin{myIndent}\hTwo
   We fairly trivially have that $\emptyset$ and $X$ are included in $\mathcal{T}$.\retTwo

   Let $\{U_\alpha\}_{\alpha \in J}$ be an indexed family of elements of $\mathcal{T}$ and define $U = \bigcup\limits_{\alpha \in J}U_\alpha$.\\ [-8pt] Given any $x \in U$, we know there exists $\alpha \in J$ such that $x \in U_\alpha$.\\ And since $U_\alpha$ is open, there exists $B \in \mathcal{B}$ such that $x \in B$ and\\ $B \subseteq U_\alpha \subseteq U$. So, we conclude that $U$ is also open.\retTwo

   Finally, we shall prove by induction that given $U_1, \ldots U_n \in \mathcal{T}$, we have that\\ $U_1 \cap \ldots \cap U_n \in \mathcal{T}$.

   
   \begin{myIndent}\hThree
      Firstly, consider any $U_1, U_2 \in \mathcal{T}$. Then, given any $x \in U_1 \cap U_2$, choose\\ basis elements $B_1, B_2 \in \mathcal{B}$ such that $x \in B_1 \subseteq U_1$ and $x \in B_2 \subseteq U_2$. \\Since $x \in B_1 \cap B_2$, we know there is a basis element $B_3 \in \mathcal{B}$ such that\\ $x \in B_3 \subseteq B_1 \cap B_2$. Then $x \in B_3 \subseteq U$.\retTwo

      With that, we've now shown that the intersection of any two elements of $\mathcal{T}$\\ is also in $\mathcal{T}$. So, we can proceed by induction.\retTwo

      Suppose for $i < n$ that $(U_1 \cap \ldots \cap U_i) \in \mathcal{T}$. Then we know that\\ $(U_1 \cap \ldots \cap U_i) \cap U_{i+1} \in \mathcal{T}$.\newpage
   \end{myIndent}
\end{myIndent}

\blab{Lemma 13.1:} Let $X$ be a set and $\mathcal{B}$ be a basis for a topology $\mathcal{T}$ on $X$. Then $\mathcal{T}$ equals the collection of all unions of elements of $\mathcal{B}$.

\begin{myIndent}\hTwo
   Proof:\\
   Let $\mathcal{T}^\prime$ be the collection of all unions of elements of $\mathcal{B}$.\retTwo

   Since every $B \in \mathcal{B}$ is an element of $\mathcal{T}$, we trivially have that $\mathcal{T}^\prime \subseteq \mathcal{T}$. Meanwhile, given any $U \in \mathcal{T}$, choose for each $x \in U$ an element $B_x$ of $\mathcal{B}$ such that $x \in B_x \subseteq U$. Then $U = \bigcup\limits_{x \in U}B_x$, meaning $U \in \mathcal{T}^\prime$.\\ [-12pt]
   \begin{myTindent}\begin{myTindent}
   \begin{myIndent}
      \color{Red}\fontsize{12}{14}\selectfont
         (Axiom of Choice usage alert!!)\retTwo\retTwo
   \end{myIndent}
   \end{myTindent}\end{myTindent}
\end{myIndent}

\blab{Lemma 13.2}: Let $X$ be a topological space. Suppose that $\mathcal{C}$ is a collection of open sets of $X$ such that for each open set $U$ of $X$ and each $x \in U$, there is an element $C \in \mathcal{C}$ such that $x \in C \subseteq U$. Then $\mathcal{C}$ is a basis for the topology of $X$.

\begin{myIndent}\hTwo
   Proof:\\
   Firstly, we need to show that $\mathcal{C}$ is a basis.
   \begin{myIndent}\hThree
      Since $X$ is an open set, we know by hypothesis that for all $x \in X$, there is $C \in \mathcal{C}$ such that $x \in C$. As for the second condition of a basis, suppose $x \in C_1 \cap C_2$\\ where $C_1, C_2 \in \mathcal{C}$. Since $C_1$ and $C_2$ are open, we know that $C_1 \cap C_2$ is open. So\\ by hypothesis, there is $C_3 \in \mathcal{C}$ such that $x \in C_3 \subseteq (C_1 \cap C_2)$.\retTwo
   \end{myIndent}

   Secondly, we need to show that $\mathcal{C}$ is a basis for the topology of $X$.
   \begin{myIndent}\hThree
      Let $\mathcal{T}$ be the collection of open sets of $X$, and let $\mathcal{T}^\prime$ be the topology generated\\ by $\mathcal{C}$. Firstly, if $U \in \mathcal{T}$ and $x \in U$, there is by hypothesis $c \in \mathcal{C}$ such that $x \in C$ and $C \subseteq U$. So $U \subseteq \mathcal{T}^\prime$. Meanwhile, if $W \in \mathcal{T}^\prime$, then $W$ equals a union of elements of $\mathcal{C}$ by lemma 13.1. Since each element of $\mathcal{C}$ is in $\mathcal{T}$, we know $W$ is the union of elements of $\mathcal{T}$, meaning $W \in \mathcal{T}$. So, we've shown that $\mathcal{T} \subseteq \mathcal{T}^\prime \subseteq \mathcal{T}$.\retTwo\retTwo
   \end{myIndent}
\end{myIndent}

\blab{Lemma 13.3:} Let $\mathcal{B}$ and $\mathcal{B}^\prime$ be bases for the topologies $\mathcal{T}$ and $\mathcal{T}^\prime$ respectively on $X$. Then $\mathcal{T}^\prime$ is finer than $\mathcal{T}$ if and only if for each $x \in X$ and each basis element $B \in \mathcal{B}$ containing $x$, there is a basis element $B^\prime \in \mathcal{B}^\prime$ such that $x \in B^\prime \subseteq B$.

\begin{myIndent}\hTwo
   Proof:\\
   ($\Longrightarrow$) Let $x \in X$ and $B \in \mathcal{B}$ such that $x \in B$. Since $B \in \mathcal{T}$ and we are assuming $\mathcal{T} \subseteq \mathcal{T}^\prime$, we know that $B \in \mathcal{T}^\prime$. Then since $\mathcal{B}^\prime$ generated $\mathcal{T}^\prime$, we know there is $B^\prime \in \mathcal{B}^\prime$ such that $x \in B^\prime \subseteq B$.\retTwo

   ($\Longleftarrow$)\\
   Given an element $U$ of $\mathcal{T}$, we need to show that $U \in \mathcal{T}^\prime$. To do this, consider any $x \in U$. Since $\mathcal{B}$ generates $\mathcal{T}$, there is an element $B \in \mathcal{B}$ such that $x \in B \subseteq U$. Now by hypothesis, there exists $B^\prime \in \mathcal{B}^\prime$ such that $x \in B^\prime \subseteq B$. So $x \in B^\prime \subseteq U$. Hence, $U \in \mathcal{T}^\prime$.\newpage
\end{myIndent}

If $\mathcal{B}$ is the collection of all open intervals $(a, b) = \{x \in \mathbb{R} \mid a < x < b\}$ in the real line, then we call the topology generated by $\mathcal{B}$ the \udefine{standard topology} on the real line.\\ [-16pt]

\begin{myDindent}\hTwo
   We assume $\mathbb{R}$ has this topology unless stated otherwise.\retTwo
\end{myDindent}

If $\mathcal{B}^\prime$ is the collection of all intervals $[a, b)$ of the real line, we call the topology\\ generated by $\mathcal{B}^\prime$ the \udefine{lower limit topology}.\\ [-15pt]
\begin{myDindent}\hTwo
   When $\mathbb{R}$ has this topology, we denote it $\mathbb{R}_l$.\retTwo
\end{myDindent}

Letting $K = \{\frac{1}{n} \mid n \in \mathbb{Z}_+\}$, if $\mathcal{B}^\pprime$ is the collection of all intervals $(a, b)$ of the real line along with all sets of the form $(a, b) - K$, then we call the topology generated by $\mathcal{B}^\pprime$ the \udefine{$K$-topology} on the real line.\\ [-15pt]
\begin{myDindent}\hTwo
   When $\mathbb{R}$ has this topology, we denote it $\mathbb{R}_K$.\retTwo\retTwo
\end{myDindent}

\blab{Lemma 13.4:} The topologies of $\mathbb{R}_l$ and $\mathbb{R}_K$ are strictly finer than the standard topology on $\mathbb{R}$. But, they aren't comparable with one another.

\begin{myIndent}\hTwo
   Proof:\\
   Let $\mathcal{T}, \mathcal{T}^\prime, \mathcal{T}^\pprime$ be the topologies of $\mathbb{R}, \mathbb{R}_l, \mathbb{R}_K$ respectively.\retTwo

   Given any $(a, b) \in \mathcal{B}$ and $x \in (a, b)$, we know that $[x, b) \in \mathcal{B}^\prime$ and that\\ $x \in [x, b) \subseteq (a, b)$. So by lemma 13.3, $\mathcal{T} \subseteq \mathcal{T}^\prime$. On the other hand, for any\\ $[x, b) \in \mathcal{B}^\prime$, there is no set $(a, b) \in \mathbb{B}$ such that $x \in (a, b) \subseteq [x, b)$. So\\ $\mathcal{T}^\prime \not\subseteq \mathcal{T}$. Hence, $\mathcal{T}^\prime$ is strictly finer than $\mathcal{T}$.\retTwo

   Also, given any $(a, b) \in \mathcal{B}$, we also know that $(a, b) \in \mathcal{B}^\pprime$. So $\mathcal{T} \subseteq \mathcal{T}^\pprime$. On\\ the other hand, given $(-1, 1) - K \in \mathcal{B}^\pprime$, we know there is no interval\\ $(a, b) \in \mathcal{B}$ such that $0 \in (a, b) \subseteq (-1, 1) - K$. So by lemma 13.3, we know\\ that $\mathcal{T}^\pprime \not\subseteq \mathcal{T}$. Hence, $\mathcal{T}^\pprime$ is strictly finer than $\mathcal{T}$.\retTwo

   Finally, we show $\mathcal{T}^\prime$ and $\mathcal{T}^\pprime$ aren't comparable. Firstly, given the set $(-1, 1) - K$ in $\mathcal{B}^\pprime$, there is no set $[a, b) \in \mathcal{B}^\prime$ such that $0 \in [a, b) \subseteq (-1, 1) - K$. After all, for any $b > 0$, we can use the archimedean property to find $\frac{1}{n} < b$. Secondly, given the set $[0, 1) \in \mathcal{B}^\prime$, no set of the form $(a, b)$ can satisfy that $0 \in (a, b) \subseteq [0, 1)$. Similarly, no set of the form $(a, b) - K$ can satisfy that $0 \in (a - b) - K \subseteq [0, 1)$. So neither $\mathcal{T}^\prime \subseteq \mathcal{T}^\pprime$ nor $\mathcal{T}^\pprime \subseteq \mathcal{T}^\prime$.\retTwo\retTwo
\end{myIndent}

A \udefine{subbasis} $\mathcal{S}$ for a topology on $X$ is a collection of subsets of $X$ whose union equals $S$. The \udefine{topology generated by the subbasis $\mathcal{S}$} is defined to be the collection $\mathcal{T}$ of all unions of finite intersections of elements of $\mathcal{S}$.\retTwo

\blab{Proof that the $\mathcal{T}$ generated by $\mathcal{S}$ is a topology:}
\begin{myIndent}\hTwo
   It suffices to show that the collection $\mathcal{B}$ of all finite intersections of elements of $\mathcal{S}$ is a basis. The first condition of a basis is trivially true for $\mathcal{B}$ since the union of the elements of $\mathcal{S}$ is all of $X$ and $\mathcal{S} \subseteq \mathcal{B}$.\newpage
   
   As for the second condition of a basis, given any $(S_1 \cap \ldots \cap S_n), (S_1^\prime \cap \ldots \cap S_m^\prime) \in \mathcal{B}$, we know that $(S_1 \cap \ldots \cap S_n) \cap (S_1^\prime \cap \ldots \cap S_m^\prime)$ is a finite intersection of elements of $\mathcal{S}$ and thus an element in $\mathcal{B}$. Thus, the condition easily follows.\retTwo\retTwo
\end{myIndent}

\dispDate{9/26/2024}

Well, it looks like I'll be able to survive 240A with the topology information I've learned so far. However, it doesn't look like I'll be able to survive 240B with what I know right now. So, I've got to study more of this. But if needed for 188 this quarter, I'll take a break to study algebra.\retTwo

\exOne
\blab{Exercise 13.3} Show that $\mathcal{T} = \{U \subseteq X \mid X - U \text{ is countable or all of } X\}$ is a\\ topology on $X$.

\begin{myIndent}\exTwoP
   Clearly $\emptyset, X \in \mathcal{T}$ since $|X - X| = 0$ and $X - \emptyset = X$.\retTwo

   Suppose $\{U_\alpha\}_{\alpha \in A}$ is a collection of sets in $\mathcal{T}$. Then $X - \bigcup\limits_{\alpha \in A}U_\alpha = \bigcap\limits_{\alpha \in A} (X - U_\alpha)$\\ [-6pt] is countable since it's a subset of a countable set.\retTwo

   Hence, $\bigcup\limits_{\alpha \in A}U_\alpha \in \mathcal{T}$.\retTwo

   Finally, consider any $\{U_1, \ldots, U_n\}$ in $\mathcal{T}$. Then $X - \bigcap\limits_{k=1}^n U_k = \bigcup\limits_{k = 1}^n (X - U_k)$\\ [-7pt] is countable since it's a union of finitely many\\ countable sets.\retTwo
   
   Hence, $\bigcap\limits_{k = 1}^n U_k \in \mathcal{T}$.\retTwo
\end{myIndent}

\blab{Exercise 13.4:}
\begin{itemize}
   \item[(a)] If $\{\mathcal{T}_\alpha\}_{\alpha \in A}$ is a family of topologies on $X$, show that $\bigcap\mathcal{T}_\alpha$ is a topology on $X$.\\ Is $\bigcup\mathcal{T}_\alpha$ a topology on $X$?
   
   \begin{myIndent}\exTwoP
      Let $\mathcal{T} = \bigcap\limits_{\alpha \in A}\mathcal{T}_\alpha$.\\

      Since $\emptyset$ and $X$ belong to all $\mathcal{T}_\alpha$, we know that $\emptyset, X \in \mathcal{T}$.\retTwo

      Next, suppose $\{U_\beta\}_{\beta \in B}$ is a collection of sets in $\mathcal{T}$. Since $\{U_\beta\}_{\beta \in B} \subseteq \mathcal{T}_\alpha$ for all $\alpha$, we know that $\bigcup\limits_{\beta \in B}U_\beta \in \mathcal{T}_\alpha$ for all $\alpha$. Hence, $\bigcup\limits_{\beta \in B}U_\beta \in \bigcap\limits_{\alpha \in A}\mathcal{T}_\alpha = \mathcal{T}$.\retTwo

      The same argument as used for arbitrary unions also shows that any finite\\ intersection of sets in $\mathcal{T}$ is also in $\mathcal{T}$.\newpage

      We've now shown that $\mathcal{T}$ is a topology. As for the other question asked, no we don't necessarily have that $\bigcup\limits_{\alpha \in A}\mathcal{T}_\alpha$ is a topology.\retTwo

      To see this, consider the set $X = \{a, b, c\}$ with the topologies\\ $\mathcal{T}_1 = \{\emptyset, \{a\}, \{a, b\}, \{a, b, c\}\}$ and $\mathcal{T}_2 = \{\emptyset, \{c\}, \{b, c\}, \{a, b, c\}\}$. Then\\ $\mathcal{T}_1 \cup \mathcal{T}_2$ is not a topology because $\{a\}, \{c\} \in \mathcal{T}_1 \cup \mathcal{T}_2$ but $\{a, c\} \notin \mathcal{T}_1 \cup \mathcal{T}_2$.\retTwo
   \end{myIndent}

   \item[(b)] Let $\{\mathcal{T}_\alpha\}_{\alpha \in A}$ be a family of topologies on $X$. Show that there is a unique smallest topology on $X$ containing all the collections $\mathcal{T}_\alpha$, and a unique largest topology contained in all $\mathcal{T}_\alpha$.
   
   \begin{myIndent}\exTwoP
      Firstly, let $\{\mathcal{T}_\beta^\prime\}_{\beta \in B}$ be the collection of all topologies on $X$ which contain $\bigcup\limits_{\alpha \in A}\hspace{-0.2em}\mathcal{T}_\alpha$.\retTwo

      We know that $\{\mathcal{T}_\beta^\prime\}_{\beta \in B}$ is not empty because it must at least have $\mathcal{P}(X)$ as an element. Hence, we can apply part (a) of the problem to know that $\bigcap\limits_{\beta \in B}\mathcal{T}_\beta^\prime$ is a\\ [-8pt] topology on $X$.\retTwo

      Importantly, by virtue of being an intersection, that topology is smaller than all other topologies containing $\bigcup\limits_{\alpha \in A}\hspace{-0.2em}\mathcal{T}_\alpha$. At the same time, we know it contains\\ [-8pt] $\bigcup\limits_{\alpha \in A}\hspace{-0.2em}\mathcal{T}_\alpha$.\retTwo
      
      So it is the unique smallest topology on $X$ containing all the collections $\mathcal{T}_\alpha$.\retTwo

      The second part of this question is trivial from part (a). If a topology $\mathcal{T}^\pprime$ is contained in all $\mathcal{T}_\alpha$, then we know that $\mathcal{T}^\pprime \subseteq \bigcap\limits_{\alpha \in A}\mathcal{T}_\alpha$. Clearly, the largest topology\\ [-8pt] satisfying this is $\bigcap\limits_{\alpha \in A}\mathcal{T}_\alpha$.\retTwo
   \end{myIndent}
\end{itemize}

\blab{Exercise 13.5:} Show that if $\mathcal{A}$ is a basis for a topology on $X$, then the topology\\ generated by $\mathcal{A}$ equals the intersection of all topologies on $X$ that contain $\mathcal{A}$.

\begin{myIndent}\exTwoP
   Let $\mathcal{T}$ be the topology generated by $\mathcal{A}$, and suppose $\mathcal{T}^\prime$ is any topology containing $\mathcal{A}$. Then consider any $U \in \mathcal{T}$. By Lemma 13.1, we know that $U = \bigcup\limits_{\beta \in B}A_\beta$ where\\ [-11pt] $\{A_\beta\}$ is some collection of sets in $\mathcal{A}$. Hence, $U$ is a union of\\ sets in $\mathcal{T}^\prime$, meaning $U \in \mathcal{T}^\prime$.\retTwo

   Since $\mathcal{T} \subseteq \mathcal{T}^\prime$ for all $\mathcal{T}^\prime$ containing $\mathcal{A}$, we thus know that $\mathcal{T}$ is the unique\\ smallest topology containing $\mathcal{A}$. At the same time, by exercise 13.4.a, we know that\\ the intersection $\mathcal{T}^\pprime$ of all topologies containing $\mathcal{A}$ is a topology. By virtue of being\\ an intersection, we know it is smaller than all topologies containing $\mathcal{A}$, and that\\ it contains $\mathcal{A}$. So, $\mathcal{T} \subseteq \mathcal{T}^\pprime \subseteq \mathcal{T} \Longrightarrow \mathcal{T} = \mathcal{T}^\pprime$.\newpage
\end{myIndent}

Prove the same if $\mathcal{A}$ is a subbasis.

\begin{myIndent}\exTwoP
   Let $\mathcal{T}$ be the topology generated by $\mathcal{A}$ and suppose $\mathcal{T}^\prime$ is any topology containing $\mathcal{A}$. Then consider any $U \in \mathcal{T}$. We know that $U = \bigcup\limits_{\beta \in B}U_\beta$ where $\{U_\beta\}_{\beta \in B}$ is a\\ [-11pt] collection of finite intersections of sets in $\mathcal{A}$.\retTwo

   Because each $U_\beta$ must be in $\mathcal{T}^\prime$, we thus know that $U \in \mathcal{T}^\prime$. So $\mathcal{T} \subseteq \mathcal{T}^\prime$.\retTwo

   The rest of the proof goes exactly the same as before.
   \begin{myIndent}\myComment
      This fact can be used as a shortcut for finding the unique smallest topology\\ containing all topologies in a collection.\retTwo
   \end{myIndent}
\end{myIndent}

\blab{Exercise 13.1:} Let $X$ be a topological space and let $A$ be a subset of $X$. Suppose that for each $x \in A$, there is an open set $U$ such that $x \in U \subseteq A$. Then $A$ is open in $X$.

\begin{myIndent}\exTwoP
   For all $x \in A$, pick an open set $U_x$ such that $x \in U_x \subseteq A$. Then $A = \bigcup\limits_{x \in A}U_x$ is a\\ [-10pt] union of open sets.
   \begin{myTindent}\myComment\color{Red}
      (A.O.C. usage!!)\retTwo
   \end{myTindent}
\end{myIndent}

\hOne
\mySepTwo




\end{document}
