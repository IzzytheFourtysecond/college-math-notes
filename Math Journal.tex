\documentclass{book}

\usepackage{fontspec} % used to import Calibri
\usepackage{anyfontsize} % used to adjust font size

% needed for inch and other length measurements
% to be recognized
\usepackage{calc}

% for colors and text effects as is hopefully obvious
\usepackage[dvipsnames]{xcolor}
\usepackage{soul}

% control over margins
\usepackage[margin=1in]{geometry}
\usepackage[strict]{changepage}

\usepackage{mathtools}
\usepackage{amsfonts}
\usepackage{bm}

\usepackage[scr=rsfso, scrscaled=.96]{mathalpha}

\usepackage{amssymb} % originally imported to get the proof square
\usepackage{xfrac}
\usepackage[overcommands]{overarrows} % Get my preferred vector arrows...
\usepackage{relsize}

% Just am using this to get a dashed line in a table...
% Also you apparently want this to be inactive if you aren't
% using it because it slows compilation.
\usepackage{arydshln} \ADLinactivate 
\newenvironment{allowTableDashes}{\ADLactivate}{\ADLinactivate}

\usepackage{graphicx}
\graphicspath{{./158_Images/}}

\usepackage{tikz}
   \usetikzlibrary{arrows.meta}
   \usetikzlibrary{graphs, graphs.standard}

\usepackage{quiver} %commutative diagrams


\newfontfamily{\calibri}{Calibri}
\setlength{\parindent}{0pt}
\definecolor{RawerSienna}{HTML}{945D27}

% ~~~~~~~~~~~~~~~~~~~~~~~~~~~~~~~~~~~~~~~~~~~~~~~~~~
%Arrow Commands:

% Thank you Bernard, gernot, and Sigur who I copied this from:
% https://tex.stackexchange.com/questions/364096/command-for-longhookrightarrow
\newcommand{\hooklongrightarrow}{\lhook\joinrel\longrightarrow}
\newcommand{\hooklongleftarrow}{\longleftarrow\joinrel\rhook}
\newcommand{\hookxlongrightarrow}[2][]{\lhook\joinrel\xrightarrow[#1]{#2}}
\newcommand{\hookxlongleftarrow}[2][]{\xleftarrow[#1]{#2}\joinrel\rhook}

% Thank you egreg who I copied from:
% https://tex.stackexchange.com/questions/260554/two-headed-version-of-xrightarrow
\newcommand{\longrightarrowdbl}{\longrightarrow\mathrel{\mkern-14mu}\rightarrow}
\newcommand{\longleftarrowdbl}{\leftarrow\mathrel{\mkern-14mu}\longleftarrow}

\newcommand{\xrightarrowdbl}[2][]{%
  \xrightarrow[#1]{#2}\mathrel{\mkern-14mu}\rightarrow
}
\newcommand{\xleftarrowdbl}[2][]{%
  \leftarrow\mathrel{\mkern-14mu}\xleftarrow[#1]{#2}
}

\newcommand{\mRoman}[1]{%
   \textrm{\MakeUppercase{\romannumeral #1}}%
}



% ~~~~~~~~~~~~~~~~~~~~~~~~~~~~~~~~~~~~~~~~~~~~~~~~~~

\newcommand{\hOne}{%
   \color{Black}%
   \fontsize{14}{16}\selectfont%
}
\newcommand{\hTwo}{%
\color{Black}%
   \fontsize{13}{15}\selectfont%
}
% \newcommand{\scratchWork}{%
%    \color{PineGreen!85!Orange}
%    \fontsize{12}{14}\selectfont%
% }
\newcommand{\hThree}{%
   \color{Black}%
   \fontsize{12}{14}\selectfont%
}
\newcommand{\myComment}{%
   \color{RawerSienna}%
   \fontsize{12}{14}\selectfont%
}
% \newcommand{\pracOne}{
%    \color{BrickRed}%
%    \fontsize{13}{15}\selectfont%
% }
% \newcommand{\pracTwo}{
%    \color{Orange}%
%    \fontsize{12}{14}\selectfont%
% }
\newcommand{\exOne}{%
   \color{Purple}%
   \fontsize{14}{16}\selectfont%
}
\newcommand{\exTwo}{%
\color{Purple}%
   \fontsize{13}{15}\selectfont%
}
\newcommand{\exP}{%
   \color{Purple}%
   \fontsize{12}{14}\selectfont%
}
% ~~~~~~~~~~~~~~~~~~~~~~~~~~~~~~~~~~~~~~~~~~~~~~~~

\newcommand{\cyPen}[1]{{\vphantom{.}\color{Cerulean}#1}}
\newcommand{\redPen}[1]{{\vphantom{.}\color{Red}#1}}

\newenvironment{myIndent}{%
   \begin{adjustwidth}{2.5em}{0em}%
}{%
   \end{adjustwidth}%
}

\newenvironment{myDindent}{%
   \begin{adjustwidth}{5em}{0em}%
}{%
   \end{adjustwidth}%
}

\newenvironment{myTindent}{%
   \begin{adjustwidth}{7.5em}{0em}%
}{%
   \end{adjustwidth}%
}

\newenvironment{myConstrict}{%
   \begin{adjustwidth}{2.5em}{2.5em}%
}{%
   \end{adjustwidth}%
}

\newcommand{\udefine}[1]{{%
   \setulcolor{Red}%
   \setul{0.14em}{0.07em}%
   \ul{#1}%
}}

\newcommand{\blab}[1]{\textbf{#1}}

\newcommand{\uuline}[2][.]{%
{\vphantom{a}\color{#1}%
\rlap{\rule[-0.18em]{\widthof{#2}}{0.06em}}%
\rlap{\rule[-0.32em]{\widthof{#2}}{0.06em}}}%
#2}

\newcommand{\pprime}{{\prime\prime}}
\newcommand{\suchthat}{ \hspace{0.3em}s.t.\hspace{0.3em}}
\newcommand{\rea}[1]{\mathrm{Re}(#1)}
\newcommand{\ima}[1]{\mathrm{Im}(#1)}
\newcommand{\comp}{\mathsf{C}}
\newcommand{\myHS}{ \hspace{0.5em}}

\newcommand{\myId}{\mathrm{Id}}
\newcommand{\myIm}{\mathrm{im}}
\newcommand{\myObj}{\mathrm{Obj}}
\newcommand{\myHom}{\mathrm{Hom}}
\newcommand{\myEnd}{\mathrm{End}}
\newcommand{\myAut}{\mathrm{Aut}}

\newcommand{\mcateg}[1]{{\bm{\mathsf{#1}}}}

% Thank you Gonzalo Medina and Moriambar who wrote this on stack exchange:
%https://tex.stackexchange.com/questions/74125/how-do-i-put-text-over-symbols%
\newcommand{\myequiv}[1]{\stackrel{\mathclap{\mbox{\footnotesize{$#1$}}}}{\equiv}}

% Thank you chs who wrote this on stack exchange:
%https://tex.stackexchange.com/questions/89821/how-to-draw-a-solid-colored-circle%
\newcommand{\filledcirc}[1][.]{\ensuremath{\hspace{0.05em}{\color{#1}\bullet}\mathllap{\circ}\hspace{0.05em}}}

%Thank you blerbl who wrote this on stack exchange:
%https://tex.stackexchange.com/questions/25348/latex-symbol-for-does-not-divide
\newcommand{\ndiv}{\hspace{-0.3em}\not|\hspace{0.35em}}

\newcommand{\mySepOne}[1][.]{%
   {\noindent\color{#1}{\rule{6.5in}{1mm}}}\\%
}
\newcommand{\mySepTwo}[1][.]{%
   {\noindent\color{#1}{\rule{6.5in}{0.5mm}}}\\%
}

\newenvironment{myClosureOne}[2][.]{%
   \color{#1}%
   \begin{tabular}{|p{#2in}|} \hline \\%
}{%
   \\ \hline \end{tabular}%
}

\newcommand{\retTwo}{\hfill\bigbreak}

\newcommand{\dispDate}[1]{{
   \color{Black}%
   \fontsize{20}{18}\selectfont%
   #1\retTwo
}}


\title{Math Journal}
\author{Isabelle Mills}


\begin{document}
   \maketitle{}
   \setul{0.14em}{0.07em}
   \calibri\hOne
   
   \dispDate{8/31/2024}
   My goal for today is to work through the appendix to chapter 1 in Baby Rudin. This appendix focuses on constructing the real numbers using Dedikind cuts.\retTwo
   
   \hTwo
   \begin{myIndent}
      We define a \udefine{cut} to be a set $\alpha \subset \mathbb{Q}$ such that:
      \begin{enumerate}
         \item $\alpha \neq \emptyset$
         \item If $p \in \alpha$,\myHS $q \in \mathbb{Q}$, and $q < p$, then $q \in \alpha$.
         \item If $p \in \alpha$, then $p < r$ for some $r \in \alpha$\newline
      \end{enumerate}

      Point 3 tells us that $\alpha$ doesn't have a max element. Also, point 2 directly implies the following facts:
      \begin{itemize}
         \item[a.] If $p \in \alpha$,\myHS $q \in \mathbb{Q}$, and $q \notin \alpha$, then $q > p$.
         \item[b.] If $r \notin \alpha$,\myHS $r, s \in \mathbb{Q}$, and $r < s$, then $s \notin \alpha$.\newline
      \end{itemize}

      As a shorthand, I shall refer to the set of all cuts as $R$.
      \begin{myIndent}\myComment
         An example of a cut would be the set of rational numbers less than $2$.\\
      \end{myIndent}

      Firstly, we shall assign an ordering to $R$. Specifically, given any $\alpha, \beta \in R$, we say that $\alpha < \beta$ if $\alpha \subset \beta$ (a proper subset).

      \begin{myIndent}\exTwo
         Here we prove that $<$ satisfies the definition of an ordering.
         \begin{itemize}
            \item[\mRoman{1}.] It's obvious from the definition of a proper subset that at most one of the following three things can be true: $\alpha < \beta$,\myHS $\alpha = \beta$, and $\beta < \alpha$.\retTwo
            
            Now let's assume that $a \not< \beta$ and $\alpha \not= \beta$. Then $\exists p \in \alpha$ such that $p \notin \beta$. But then for any $q \in \beta$, we must have by fact b. above that $q < p$. Hence $q \in \alpha$, meaning that $\beta \subset \alpha$. This proves that at least one of the following has to be true: $\alpha < \beta$,\myHS $\alpha = \beta$, and $\beta < \alpha$.\retTwo

            \item[\mRoman{2}.] If for $\alpha, \beta, \gamma \in R$ we have that $\alpha < \beta$ and $\beta < \gamma$, then clearly $\alpha < \gamma$ becuase $\alpha \subset \beta \subset \gamma$.\retTwo
         \end{itemize}
      \end{myIndent}

      Now we claim that $R$ equipped with $<$ has the least-upper-bound property.
      \begin{myIndent}\exTwo
         Proof:\\
         Let $A \subset R$ be nonempty and $\beta \in R$ be an upper bound of $A$. Then set\\ $\gamma = \hspace{-0.2em}\bigcup\limits_{\alpha \in A}\hspace{-0.2em}\alpha$. Firstly, we want to show that $\gamma \in R$\retTwo

         Since $A \neq \emptyset$, there exists $\alpha_0 \in A$. And as $\alpha_0 \neq \emptyset$ and $\alpha_0 \subseteq \gamma$ by definition, we know that $\gamma \neq \emptyset$. At the same time, we know that $\gamma \subset \beta$ since $\forall \alpha \in A$,\myHS $\alpha \subset \beta$. Hence, $\gamma \neq \mathbb{Q}$, meaning that $\gamma$ satisfies property 1$.$ of cuts.\retTwo

         Next, let $p \in \gamma$ and $q \in \mathbb{Q}$ such that $q < p$. We know that for some $\alpha_1 \in A$, we have that $p \in \alpha_1$. Hence by property 2$.$ of cuts, we know that $q \in \alpha_1 \subset \gamma$, thus showing that $\gamma$ satisfies property 2$.$ of cuts.\newpage
         
         Thirdly, by property 3$.$ we can pick $r \in \alpha_1$ such that $p < r$ and $r \in \alpha_1 \subset \gamma$. So, $\gamma$ satisfies property 3$.$ of cuts.\retTwo

         With that, we've now shown that $\gamma \in R$. Clearly, $\gamma$ is an upper bound of $A$ since $\alpha \subset \gamma$ for all $\alpha \in A$. Meanwhile, consider any $\delta < \gamma$. Then $\exists s \in \gamma$ such that $s \notin \delta$. And since $s \in \gamma$, we know that $s \in \alpha$ for some $\alpha \in A$. Hence, $\delta < \alpha$, meaning that $\delta$ is not an upper bound of $A$. This shows that $\gamma = \sup A$.\\ [6pt]
      \end{myIndent}

      Secondly, we want to assign $+$ and $\hspace{0.1em}\cdot\hspace{0.1em}$ operations to $R$ so that $R$ is an ordered field.\retTwo
      
      To start, given any $\alpha, \beta \in R$, we shall define $\alpha + \beta$ to be the set of all sums $r + s$ such that $r \in \alpha$ and $s \in \beta$.
      \begin{myIndent}\exTwo
         Here we show that $\alpha + \beta \in R$.
         \begin{enumerate}
            \item Clearly, $\alpha + \beta \neq \emptyset$. Also, take $r^\prime \notin \alpha$ and $s^\prime \notin \beta$. Then $r^\prime + s^\prime > r + s$ for all $r \in \alpha$ and $s \in \beta$. Hence, $r^\prime + s^\prime \notin \alpha + \beta$, meaning that $\alpha + \beta \neq \mathbb{Q}$.\\ [-9pt]
         \end{enumerate}

         Now let $p \in \alpha + \beta$. Thus there exists $r \in \alpha$ and $s \in \beta$ such that $p = r + s$.\\ [-9pt]

         \begin{enumerate}
            \item[2.] Suppose $q < p$. Then $q - s < r$, meaning that $q - s \in \alpha$. Hence,\\ $q = (q - s) + s \in \alpha + \beta$.\retTwo
            
            \item[3.] Let $t \in \alpha$ so that $t > r$. Then $p = r + s < t + s$ and $t + s \in \alpha + \beta$.\retTwo
         \end{enumerate}
      \end{myIndent}

      Also, we shall define $0^*$ to be the set of all negative rational numbers. Clearly, $0^*$ is a cut. Furthermore, we claim that $+$ satisfies the addition requirements of a field with $0^*$ as its $0$ element.

      \begin{myIndent}\exTwo
         Commutativity and associativity of $+$ on $R$ follows directly from the\\ commutativity and associativity of addition on the rational numbers.\retTwo

         Also, for any $\alpha \in R$,\myHS $\alpha + 0^* = \alpha$.
         \begin{myIndent}\exP
            If $r \in \alpha$ and $s \in 0^*$, then $r + s < r$. Hence $r + s \in \alpha$, meaning that $\alpha + 0^* \subseteq \alpha$. Meanwhile, if $p \in \alpha$, then we can pick $r \in \alpha$ such that $r > p$. Then, $p - r \in 0^*$ and $p = r + (p - r) \in \alpha + 0^*$. So, $\alpha \subseteq \alpha + 0^*$.\retTwo
         \end{myIndent}

         Finally, given any $\alpha \in R$, let $\beta = \{p \in \mathbb{Q} \mid \exists\hspace{0.1em} r \in \mathbb{Q}^+ \suchthat -p-r \notin \alpha\}$.
         \begin{myIndent}\myComment
            To give some intuition on this definition, firstly we want to guarentee that for all $p \in \beta$, $-p$ is greater than all elements of $\alpha$. Secondly, we add the $-r$ term to guarentee that $\beta$ doesn't have a maximum element.\\
         \end{myIndent}

         We claim that $\beta \in R$ and $\beta + \alpha = 0^*$. Hence, we can define $-\alpha = \beta$.

         \begin{myIndent}\exP
            To start, we'll show that $\beta \in R$:
            \begin{enumerate}
               \item For $s \notin \alpha$ and $p = -s - 1$, we have that $-p - 1 \notin \alpha$. Hence, $p \in \beta$, meaning that $\beta \neq \emptyset$. Meanwhile, if $q \in \alpha$, then $-q \notin \beta$ because there does not exist $r > 0$ such that $-(-q) - r = q - r \notin \alpha$. So $\beta \neq \mathbb{Q}$.\\ [-6pt]
            \end{enumerate}

            Now let $p \in \beta$ and pick $r > 0$ such that $-p -r \notin \alpha$.\newpage

            \begin{enumerate}
               \item[2.] Suppose $q < p$. Then $-q - r > -p - r$, meaning that $-q - r \notin \alpha$. Hence, $q \in \beta$.\retTwo
               
               \item[3.] Let $t = p + \frac{r}{2}$. Then $t > p$ and $-t - \frac{r}{2} = -p - r \notin \alpha$, meaning $t \in \beta$.\retTwo
            \end{enumerate}

            Now that we've proved $\beta \in R$, we next prove that $\beta$ is the additive inverse of $\alpha$. To start, suppose $r \in \alpha$ and $s \in \beta$. Then $-s \notin \alpha$, meaning that $r < -s$. So $r + s < 0$, thus showing that $\alpha + \beta \subseteq 0^*$.\retTwo

            As for the other inclusion, pick any $v \in 0^*$ and set $w = -\frac{v}{2}$. Then because $w > 0$, we can use the archimedean property of $\mathbb{Q}$ to say that there exists $n \in \mathbb{Z}$ such that $nw \in \alpha$ but $(n+1)w \notin \alpha$. Put $p = -(n + 2)w$. Then $p \in \beta$ because $-p - w = (n+1)w
            \notin \alpha$. And finally, $v = nw + p \in \alpha + \beta$. Thus, $0^* \subseteq \alpha + \beta$.\retTwo
         \end{myIndent}
      \end{myIndent}
   \end{myIndent}
   \dispDate{9/1/2024}
   \begin{myIndent}\hTwo
      Based on the definition of $+$, it's also hopefully clear that for any $\alpha, \beta, \gamma \in R$ such that $\alpha < \beta$, we have that $\alpha + \gamma < \beta + \gamma$.\retTwo

      Next, we shall define multiplication on $R$. Except, first we're going to limit ourselves to the set $R^+$ of all cuts greater than $0^*$. So, given any $\alpha, \beta \in R^+$, we shall define $\alpha \beta$ to be the set of all $p \in \mathbb{Q}$ such that $p \leq rs$ where $r \in \alpha$,\myHS $s \in \beta$,\myHS $r > 0$, and $s > 0$.

      \begin{myIndent}\exTwo
         Here we show that $\alpha\beta \in R^+$.
         \begin{enumerate}
            \item Clearly $\alpha\beta \neq \emptyset$. Also, take any $r^\prime \notin \alpha$ and $s^\prime \notin \beta$. Then $r^\prime s^\prime > rs$ for all $r \in \alpha \cap \mathbb{Q}^+$ and $s \in \beta \cap \mathbb{Q}^+$ since all four rational numbers are positive. By extension, $r^\prime s^\prime$ is greater than all the elements (both positive and negative) of $\alpha\beta$. So, $r^\prime s^\prime \notin \alpha\beta$, meaning that $\alpha\beta \neq \mathbb{Q}$.\\ [-9pt]
         \end{enumerate}

         Now let $p \in \alpha\beta$. Based on our definition of $\alpha\beta$, we know that the conditions of a cut trivially hold for any negative $p$. So, we'll assume from now on that $p > 0$. (Also note that a positive choice of $p$ must exist because both $\alpha$ and $\beta$ by assumption have positive elements.)\retTwo

         Since $p \in \alpha\beta \cap \mathbb{Q}^+$, we know there exists $r \in \alpha$ and $s \in \beta$ such that $p = rs$ and $r, s > 0$.

         \begin{enumerate}
            \item[2.] Suppose $0 < q < p$ (the case where $q \leq 0$ is trivial). Then $\frac{q}{s} < r$, meaning that $\frac{q}{s} \in \alpha$. So, $q = \frac{q}{s} \cdot s \in \alpha\beta$.\retTwo
            
            \item[3.] Let $t \in \alpha$ so that $t > r$. Then $p = rs < ts$ and $ts \in \alpha\beta$.\retTwo
         \end{enumerate}
      \end{myIndent}

      Also, we shall define $1^*$ to be the set of all rational numbers less than $1$. Clearly, $1^*$ is a cut. And we claim that $\hspace{0.1em}\cdot\hspace{0.1em}$ satisfies the multiplication requirements of a field with $1^*$ as its $1$ element.\newpage

      \begin{myIndent}\exTwo
         As before, commutativity and associativity of $\hspace{0.1em}\cdot\hspace{0.1em}$ on $R^+$ follows directly from commutativity and associativity of multiplication on the rational numbers.\retTwo

         Next, for any $\alpha \in R^+$, we have that $\alpha 1^* = \alpha$.
         \begin{myIndent}\exP
            It's clear that for any rational number $r \leq 0$, we have that $r \in \alpha 1^*$ and $r \in \alpha$. So we can exclusively focus on positive rational numbers.\retTwo
            
            Now suppose $r \in \alpha \cap \mathbb{Q}^+$ and $s \in 1^*$. Then $rs < r$, meaning that $rs \in \alpha$. So $\alpha 1^* \subseteq \alpha$. Meanwhile, if $p \in \alpha \cap \mathbb{Q}^+$, then we can pick $r \in \alpha$ such that $r > p$. Then $\frac{p}{r} \in 1^*$ and $p = \frac{p}{r} \cdot r \in \alpha 1^*$. So, $\alpha \subseteq \alpha 1^*$.\retTwo
         \end{myIndent}

         Thirdly, given any $\alpha \in R^+$, define:

         \begin{centering}
            $\beta = \{p \in \mathbb{Q} \mid p \leq 0\} \cup \{p \in \mathbb{Q}^+ \mid \exists r \in \mathbb{Q}^+ \suchthat \frac{1}{q} - r \notin \alpha\}$\retTwo\par
         \end{centering}

         \begin{myIndent}\exP
            Here we show that $\beta \in R^+$.
            \begin{enumerate}
               \item Clearly $\beta \neq \emptyset$. Also, if $q \in \alpha$, then $\frac{1}{q} \notin \beta$. Hence, $\beta \neq \mathbb{Q}$.\retTwo
            \end{enumerate}

            Now let $p \in \beta$ and pick $r > 0$ such that $\frac{1}{p} - r \notin \alpha$. Also, assume $p > 0$ because the proof is trivial if $p \leq 0$. (The fact that $p > 0$ in $\beta$ exists is trivial to show.)\retTwo

            \begin{enumerate}
               \item[2.] If $q \leq 0 < p$, then trivially $q \in \beta$. Meanwhile, if $0 < q < p$, then\\ [2pt] $\frac{1}{q} - r > \frac{1}{p} - r$, meaning that $\frac{1}{q} - r \notin \alpha$. Hence, $q \notin \beta$.\retTwo
               
               \item[3.] Let $t = \frac{1}{\frac{1}{p} - \frac{r}{2}}$. Then since $\frac{1}{p} - r \notin \alpha$, we know that $\frac{1}{p} - \frac{r}{2} > 0$. Also since $\frac{1}{t} = \frac{1}{p} - \frac{r}{2} < \frac{1}{p}$, we have that $t > p$. But note that $\frac{1}{t} - \frac{r}{2} = \frac{1}{p} - r \notin \alpha$.\\[2pt] Hence $t \notin \beta$.\retTwo
            \end{enumerate}
         \end{myIndent}

         We claim that $\beta\alpha = 1^*$. Hence, we can define $\frac{1}{\alpha} = \beta$.

         \begin{myIndent}\exP
            To start, suppose $r \in \alpha \cap \mathbb{Q}^+$ and $s \in \beta \cap \mathbb{Q}^+$. Then $\frac{1}{s} \notin \alpha$, meaning that\\ $r < \frac{1}{s}$. So $rs < 1$, thus showing that $\alpha\beta \subseteq 1^*$.\retTwo

            The other inclusion has a more complicated proof. Firstly, take any\\ $v \in 1^* \cap \mathbb{Q}^+$ (the proof is trivial if $v \leq 0$). Then set $w = \frac{1}{v}$, meaning\\ that $w > 1$. Now since $\alpha \in R^+$, we know there exists $n \in \mathbb{Z}$ such that\\ $w^n \in \alpha$ but $w^{n+1} \notin \alpha$. Then as $w^{n+2} > w^{n+1}$, we know that $\frac{1}{w^{n+2}} \in \beta$.\\ Hence, $v^2 = w^n \frac{1}{w^{n+2}} \in \alpha\beta$.\retTwo

            Now that we've shown that the square of every $v \in 1^* \cap \mathbb{Q}^+$ is also in $\alpha\beta$,\\ [2pt] we next show that there exists $z \in 1^* \cap \mathbb{Q}^+$ such that $z^2 > v$. Suppose $v = \frac{p}{q}$ where $p, q \in \mathbb{Z}^+$. Then set $z = \frac{p + q}{2q}$. Importantly, since $p < q$, we still have that $z \in 1^*$. But also note that:

            \begin{centering}
               $z^2 - v = \frac{p^2 + 2pq + q^2}{4q^2} - \frac{4pq}{4q^2} = \frac{p^2 - 2pq + q^2}{4q^2} = \left(\frac{p - q}{2q}\right)^2 \geq 0$\retTwo\par
            \end{centering}

            Thus as $v < z^2$ and $z^2 \in \alpha\beta$, we have that $v \in \alpha\beta$ as well. So $1^* \subseteq \alpha\beta$.\newpage
         \end{myIndent}

         Finally, so long as $\alpha, \beta, \gamma \in R^+$, we have that  $\alpha(\beta + \gamma) = \alpha\beta + \alpha\gamma$ because the rational numbers satisfy the distributive property.\retTwo
      \end{myIndent}

      Notably, in proving that $\alpha\beta \in R^+$ before, we also guarenteed that for $\alpha, \beta > 0$, we have that $\alpha\beta > 0$.\retTwo
   \end{myIndent} 

   \dispDate{9/7/2024}

   \begin{myIndent}

      Now we still need to extend our definition of multiplication from $R^+$ to all of $R$. To do this, set $\alpha 0^* = 0^*\alpha = 0^*$ and define:

      {\centering $\alpha \beta = \left\{
      \begin{matrix}
         (-\alpha)(-\beta) & \text{ if } \alpha < 0^*, \beta < 0^* \\
         -((-\alpha)\beta) & \text { if } \alpha < 0^*, \beta > 0^* \\
         -(\alpha(-\beta)) & \text{ if } \alpha > 0^*, \beta < 0^*
      \end{matrix}\right.$ \retTwo\par}

      Having done that, reproving those properties of multiplication on all of $R$ just\\ becomes a matter of addressing many cases and using the identity that\\ $(-(-\alpha)) = \alpha$.

      \begin{myIndent}\myComment
         Note that that identity can be proven just from the addition properties of a field.\retTwo
      \end{myIndent}

      Because I'm bored with this construction at this point, I'm going to skip reproving those properties.\retTwo

      So now that we've established that $R$ is a field, all we have left to do is to show that all numbers $r, s \in \mathbb{Q}$ are represented by cuts $r^*, s^* \in R$ such that:
      
      \begin{itemize}
         \item $(r + s)^* = r^* + s^*$
         \item $(rs)^* = r^*s^*$
         \item $r < s \Longleftrightarrow r^* < s^*$\retTwo
      \end{itemize}

      Again, I'm super bored and demotivated at this point. So, I'm going to skip showing this.\retTwo

      With all that done, we've now shown that $R$ satisfies all of the properties of real numbers. That concludes the proof of the existence theorem of the real numbers.
      \newpage
   \end{myIndent}

   \dispDate{9/9/2024}

   \hOne
   Today I'm just looking at James Munkres' book \textit{Topology}. Now while I'm done with the era of my life of taking exhaustive notes on a textbook, I still want to write down some interesting proofs. I also hope to do some exercises.\retTwo

   \blab{Theorem 7.8:} Let $A$ be a nonempty set. There is no injective map $f: \mathcal{P}(A) \longrightarrow A$ and there is no surjective map $g: A \longrightarrow \mathcal{P}(A)$.

   \begin{myDindent}\myComment
      In other words, the power set of a set has strictly greater cardinality.\retTwo
   \end{myDindent}

   
   \begin{myIndent}\hTwo
      Proof:\\
      If such an injective $f$ existed, then that would imply a surjective $g$ exists. So, we just need to show that any function $g: A \longrightarrow \mathcal{P}(A)$ isn't surjective.\retTwo

      Let $g: A \longrightarrow \mathcal{P}(A)$ be any function and define $B = \{a \in A \mid a \in A - g(a)\}$.\\ Clearly, $B \subseteq A$. However, $B$ cannot be in the image of $g$. After all, suppose there exists $a_0 \in A$ such that $g(a_0) = B$. Then we get a contradiction because:

      {\centering $a_0 \in B \Longleftrightarrow a_0 \in A - g(a_0) \Longleftrightarrow a_0 \in A - B$ \retTwo\par}

      Hence, $g(A) \neq \mathcal{P}(A)$ and we conclude that $g$ can't be surjective. $\blacksquare$\retTwo\retTwo
   \end{myIndent}

   \exOne
   \blab{Exercise 7.3:} Let $X= \{0, 1\}$. Show there is a bijective correspondence between the set $\mathcal{P}(\mathbb{Z}_+)$ and the Cartesian product $X^\omega$.\retTwo

   
   \begin{myIndent}\exTwo
      For any set $A \in \mathcal{P}(\mathbb{Z}_+)$, define $f(A)$ to be the $\omega$-tuple $\mathbf{x}$ such that for all\\ $i \in \mathbb{Z}^+$, $\mathbf{x}_i = 1$ if $i \in A$ and $\mathbf{x}_i = 0$ if $i \notin A$. Then clearly $f$ is injective as\\ $\forall A, B \in \mathcal{P}(\mathbb{Z}_+)$, $f(A) = f(B) \Longrightarrow A = B$. Also, given any $\mathbf{x} \in X^\omega$, we\\ know that the set $A = \{i \in \mathbb{Z}_+ \mid \mathbf{x}_i = 1\}$ satisfies that $f(A) = \mathbf{x}$,\\ meaning $f$ is surjective.
      
      \retTwo Hence, $f$ is a bijective function between $\mathcal{P}(\mathbb{Z}_+)$ and $X^\omega$.
      \begin{myTindent}\myComment
         Note that this construction still works if $\mathbb{Z}_+$ is replaced with any\\ countably infinite set.\retTwo\retTwo
      \end{myTindent}
   \end{myIndent}

   \blab{Exercise 7.5:} Determine whether the following sets are countable or not.
   \begin{itemize}
      \item[(f)] The set $F$ of all functions $f: \mathbb{Z}_+ \longrightarrow \{0, 1\}$ that are "eventually zero", meaning there is a positive integer $N$ such that $f(n) = 0$ for all $n \geq N$.
      
      \begin{myIndent}\exTwo
         $F$ is countable. To see why, let:
         
         {\centering $A_n = \{f: \mathbb{Z}_+ \longrightarrow \{0, 1\} \mid \forall i \geq n, \myHS f(i) = 0\}$\retTwo\par}
         
         Thus each $A_n$ is finite (with $2^n$ elements) and $F = \bigcup\limits_{n = 1}^\infty A_n$.\newpage
      \end{myIndent}

      \item[(g)] The set $G$ of all functions $f: \mathbb{Z}_+ \longrightarrow \mathbb{Z}_+$ that are eventually $1$.
      
      \begin{myIndent}\exTwo
         $G$ is countable. To see why, let:

         {\centering $A_n = \{f: \mathbb{Z}_+ \longrightarrow \mathbb{Z}_+ \mid \forall i \geq n, \myHS f(i) = 1\}$\retTwo\par}

         Then each $A_n$ has a bijective correspondence with $(\mathbb{Z}_+)^n$, meaning each $A_n$ is countable, and $G = \bigcup\limits_{n = 1}^\infty A_n$.
         
         \begin{myTindent}\myComment
            The same argument applies to all functions $f: \mathbb{Z}_+ \longrightarrow \mathbb{Z}_+$ that are eventually any constant.\retTwo
         \end{myTindent}
      \end{myIndent}

      \item[(h)] The set $H$ of all functions $f: \mathbb{Z}_+ \longrightarrow \mathbb{Z}_+$ that are eventually constant.
      
      \begin{myIndent}\exTwo
         $H$ is countable. To see why, let $A_n$ be the set of all functions\\ $f: \mathbb{Z}_+ \longrightarrow \mathbb{Z}_+$ that are eventually $n$. Because of part g of\\ [-6pt] this exercise, we know that each $A_n$ is countable. Also, $H = \bigcup\limits_{n = 1}^\infty A_n$.\retTwo
      \end{myIndent}

      \item[(i)] The set $I$ of all two-element subsets of $\mathbb{Z}_+$
      \item[(j)] The set $J$ of all finite subsets of $\mathbb{Z}_+$.
      
      \begin{myIndent}\exTwo
         Both $I$ and $J$ are countably infinite. We know this because we can define\\ surjections from $(\mathbb{Z_+})^2$ to $I$ and $\bigcup\limits_{n = 1}^\infty (\mathbb{Z}_+)^n$ to $J$.
         
         \begin{myIndent}
            (Finite cartesian products of countable sets and unions of countably many countable sets are countable.)\retTwo\retTwo
         \end{myIndent}
      \end{myIndent}
   \end{itemize}

   \blab{Exercise 7.6.a:} Show that if $B \subset A$ and there is an injection $f: A \longrightarrow B$, then $|A| = |B|$.
   
   \begin{myIndent}\exTwo
      According to the hint, we set $A_1 = A$ and $A_n = f(A_{n-1})$ for all $n > 1$. Similarly, we set $B_1 = B$ and $B_n = f(B_{n-1})$ for all $n > 1$.\retTwo

      We can assume $A_2$ is a proper subset of $B_1$ because if $A_2 = B_1$, then we already have that $f$ is a bijection. Also, as $f$ is an injection, we know that $B_2 \subset A_2$. Thus by induction, we can conclude that:

      {\centering $ A_1 \supset B_1 \supset A_2 \supset B_2 \supset A_3 \supset B_3 \supset \cdots $\retTwo\par}

      Now, the textbook recommends defining $h: A \longrightarrow B$ by:

      \begin{center}
         $h(x) = \left\{
         \begin{matrix}
            f(x) & & \text{ if } x \in A_n - B_n \text{ for any } n \in \mathbb{Z}_+ \\
            x & & \text{ otherwise }
         \end{matrix}\right.$\newpage
      \end{center}

      
      \begin{myIndent}
         \myComment I want to ask a professor about this definition because it urks me. My issue with\\ this definition of $h$ is that I feel like it should be possible for:
         $$\bigcap\limits_{n=1}^\infty (A_n \cap B_n) \neq \emptyset.$$
         
         However, we wouldn't be able to know that some $x$ is in that intersection and\\ thus falls into case 2 until after an infinite number of steps.\retTwo
         
         On the other hand, $S_1 = \bigcup\limits_{n = 1}^\infty (A_n - B_n)$ is a valid definition for a set, as is\\ $S_2 = A - S_1$. So the definition $h$ is valid because it's saying that $h(x) = f(x)$\\ [6pt] if $x \in S_1$ and $h(x) = x$ if $x \in S_2$.\retTwo

         Maybe my issue is just that I have trouble trusting the validity of a function definition if I can't actually evaluate that function myself. Although, there are lots of functions like that that I don't have any problem with. For example, given $g(x) = 0$ if $x$ is rational and $g(x) = 1$ if $x$ is irrational, what is $g(\pi^2)$?\retTwo
      \end{myIndent}

      Hopefully it is clear that $h$ is in fact a valid function from $A$ to $B$. Now firstly, we shall show that $h$ is injective.

      \begin{myIndent}\exP
         Let $x, y \in A$ such that $x \neq y$. If there are integers $n$ and $m$ such that $x \in A_n - B_n$ and $y \in A_m - B_m$, then $h(x) \neq h(y)$ because $f$ is injective. Meanwhile, if no such $n$ or $m$ exists, then $h(x) \neq h(y)$ because $x \neq y$.\retTwo

         This leaves the case that there exists $n \in \mathbb{Z}_+$ such that $x \in A_n - B_n$ but for\\ all $m \in \mathbb{Z}_+,\myHS y \notin A_m - B_m$. Then, note that $f(x) \in f(A_{n} - B_{n})$. And since\\ $f$ is injective, we thus have that $f(x) \in f(A_{n}) - f(B_{n}) = A_{n+1} - B_{n+1}$.\\ Therefore, as $y \notin A_{n+1} - B_{n+1}$, we know that $h(x) \neq y = h(y)$.\retTwo
      \end{myIndent}

      Next, we show $h$ is surjective.

      \begin{myIndent}\exP
         Let $y \in B$.\retTwo
         
         Suppose there exists $n \in \mathbb{Z}_+$ such that $y \in A_n - B_n$. We know that $n \neq 1$ since $y \in B$. Thus, there must exist $x \in A_{n-1}$ such that $y = f(x) \in f(A_{n-1}) = A_n$. Furthermore, this $x$ can't be in $B_{n-1}$ because otherwise $y$ would be in $B_n$ which we know isn't true. So, $x \in A_{n-1} - B_{n-1}$, meaning that $h(x) = f(x) = y$.\retTwo

         Meanwhile, if no such $n$ exists, then we simply have that $h(y) = y$. Hence,\\ $h(A) = B$.\retTwo
      \end{myIndent}

      Thus, we've shown that $h$ is a bijection, meaning that $|A| = |B|$.\newpage
   \end{myIndent}

   \blab{Exercise 7.7:} Show that $|\{0, 1\}^\omega| = |(\mathbb{Z}_+)^\omega|$.

   
   \begin{myIndent}\exTwo
      Firstly, there's obviously a bijection exists between $\{0, 1\}^\omega$ and $\{1, 2\}^\omega$. Also,\\ $\{1, 2\}^\omega \subset (\mathbb{Z}_+)^\omega$. So, if we can construct an injective function from $(\mathbb{Z}_+)^\omega$ to\\ $\{1, 2\}^\omega$, then we can apply the result of exercise 7.6.a to prove this exercise's\\ claim.\retTwo

      We shall create this injection using a diagonalization argument. Let $x \in (\mathbb{Z}_+)^\omega$.\\ Then we define $f(x) = y \in \{1, 2\}^\omega$ as follows:
      
      \begin{center}
         \begin{tabular}{c}
            $y(1) = 2$ if $x(1) = 1$. Otherwise $y(1) = 1$.\\ [6pt]
            $y(2) = 2$ if $x(1) = 2$. Otherwise $y(2) = 1$.\\
            $y(3) = 2$ if $x(2) = 1$. Otherwise $y(3) = 1$.\\ [6pt]
            $y(4) = 2$ if $x(1) = 3$. Otherwise $y(4) = 1$.\\
            $y(5) = 2$ if $x(2) = 2$. Otherwise $y(5) = 1$.\\
            $y(6) = 2$ if $x(3) = 1$. Otherwise $y(6) = 1$. \\ [6pt]

            $y(7) = 2$ if $x(1) = 4$. Otherwise $y(7) = 1$.\\
            $\vdots$\\ [12pt]
         \end{tabular}
      \end{center}

      Clearly $f$ is an injection since $f(x_1) = f(x_2)$ implies that $x_1$ and $x_2$ have the same integers at all indices.\retTwo\retTwo
   \end{myIndent}

   \blab{Exercise 7.6.b: (Schroeder-Bernstein theorem)} If there are injections $f: A \longrightarrow C$ and $g: C\longrightarrow A$, then $A$ and $C$ have the same cardinality.\retTwo
   \myComment
   I did my work on paper and now it's late and I don't want to write more tonight.\retTwo\retTwo

   \dispDate{9/11/2024}

   \hOne

   Since today's my day off, I'm gonna work through Munkres' textbook \textit{Topology} some more.\retTwo

   \blab{Theorem 8.4 (Principle of recursive definition):} Let $A$ be a set and let $a_0$ be an element of $A$. Suppose $\rho$ is a function assigning an element of $A$ to each function $f$ mapping a nonempty section of the positive integers onto $A$. Then there exists a unique function $h: \mathbb{Z}_+ \longrightarrow A$ such that:

   {\begin{center}
      \begin{tabular}{l c r}
         $(*)$ & \phantom{aaaa} & 
         \begin{tabular}{l r}
            $h(1) = a_0$ & \\
            $h(i) = \rho(h|_{\{1, \ldots, i-1\}})$ & $\text{for } i > 1\text{.}$
         \end{tabular}
      \end{tabular}\retTwo
   \end{center}}

   \newpage
   \begin{myIndent}\hTwo
      Proof outline:

     \begin{myIndent}\hThree
       Part 1: Given any $n \in \mathbb{Z}_+$, there exists a function $f: \{1, \ldots, n\} \longrightarrow A$ that\\ satisfies $(*)$.
 
       \begin{myIndent}\myComment
          This is obvious from induction.\\ [9pt]
       \end{myIndent}

       Part 2: Suppose that $f: \{1, \ldots, n\} \longrightarrow A$ and $g: \{1, \ldots, m\} \longrightarrow A$ both satisfy $(*)$ for all $i$ in their respective domains. Then $f(i) = g(i)$ for all $i$ in both domains.

      \begin{myIndent}
         Proof:\\
         Suppose not. Let $i$ be the smallest integer for which $f(i) \neq g(i)$.\retTwo
         
         We know $i \neq 1$ because $f(1) = a_0 = g(1)$. But then note that\\ $f|_{\{1, \ldots, i - 1\}} = g|_{\{1, \ldots, i- 1\}}$. Hence:
         
         {\centering $f(i) = \rho(f|_{\{1, \ldots, i - 1\}}) = \rho(g|_{\{1, \ldots, i - 1\}}) = g(i)$.\retTwo\par}

         This contradicts that $i$ is the smallest integer for which $f(i) \neq g(i)$.\\ [9pt]
      \end{myIndent}

      Part 3: Let $f_n: \{1, \ldots, n\} \longrightarrow A$ be the unique function satisfying  $(*)$\\ (uniqueness was proven in part 2). Then we define:

      {\centering $h = \bigcup\limits_{i = 1}^\infty f_n$ \retTwo\par}

      \begin{myIndent}\myComment
         Because of part 2, we can fairly easily show that for each $i \in \mathbb{Z}_+$, there is exactly one element in $h$ with $i$ as it's first coordinate. Hence, the set $h$ defines a functions from $\mathbb{Z}_+$ to $A$.\retTwo

         Also, hopefully it's clear that $h$ satisfies $(*)$.\retTwo
      \end{myIndent}
     \end{myIndent}
   \end{myIndent}

   \mySepTwo

   \blab{Axiom of choice}: Given a collection $\mathcal{A}$ of disjoint nonempty sets, there exists a set $C$ consisting of exactly one element from each element of $\mathcal{A}$.

   
   \begin{myIndent}\myComment
      A few notes:
      \begin{enumerate}
         \item If we restrict $\mathcal{A}$ to being a finite collection, then there is nothing controversial about this axiom. It only becomes controversial when $\mathcal{A}$ is allowed to be infinite.
         \item There are multiple instances in baby Rudin where we made an infinite number of\\ arbitrary choices. Looking at a lot of those proofs closer, I think many of them could avoid using the axiom of choice by specifying that we had to pick rational numbers in a set. However, being able to pick elements without worrying about a preexisting choice function is way easier.\retTwo
         
         My take away from this is that not only does it make proofs cleaner to not worry about using constructed choice functions, but it's also perfectly acceptable now-a-days to use this axiom.\newpage
         
         Plus, some really commonly used theorems require the axiom of choice to prove them. For example, the union of countably many countable sets is countable. This makes it really easy to accidentally use the axiom of choice in a proof.\retTwo
      \end{enumerate}
   \end{myIndent}

   \blab{Lemma 9.2: (Existence of a choice function)} Given a collection $\mathcal{B}$ of nonempty sets (not necessarily disjoint), there exists a function \[c: \mathcal{B} \longrightarrow \bigcup\limits_{B \in \mathcal{B}}B\] such that $c(B)$ is an element of $B$ for each $B \in \mathcal{B}$.\retTwo

   
   \begin{myIndent}\hTwo
      Proof:\\
      Given any set $B \in \mathcal{B}$, we define $B^\prime = \{(B, b) \mid b \in B\}$. Because $B \neq \emptyset$, we know that $B^\prime \neq \emptyset$ as well. Furthermore, given $B_1, B_2 \in \mathcal{B}$ if $B_1 \neq B_2$, then we have that the first element of all the pairs in $B_1^\prime$ are different from that of $B_2^\prime$. So $B_1^\prime$ and $B_2^\prime$ are disjoint.\retTwo

      Now form the collection $\mathcal{C} = \{B^\prime \mid B \in \mathcal{B}\}$. From before, we know that $\mathcal{C}$ is\\ a collection of disjoint sets. So by the axiom of choice, there exists a set $c$\\ consisting of exactly one element from each element of $\mathcal{C}$.\retTwo

      This set $c$ is a subset of $\mathcal{B} \times \bigcup\limits_{B \in \mathcal{B}}B$ which satisfies our definition of a choice function.\\ [-12pt]
      \begin{myTindent}\begin{myTindent}\myComment
         Hopefully it's obvious enough why $c$ satisfies\\ those properties.\retTwo\retTwo
      \end{myTindent}\end{myTindent}
   \end{myIndent}

   \mySepTwo

   A set $A$ with an order relation $<$ is said to be \udefine{well-ordered} if every nonempty subset of $A$ has a smallest element.\retTwo

   
   \begin{myIndent}\myComment
      
      {\fontsize{13}{15}\selectfont%
      \blab{Tangent: inductiveness of $\mathbb{Z}_+$ is equivalent to the well-orderedness of $\mathbb{Z}_+$}
      }
      
      \begin{myIndent}
         This proof is taken from https://math.libretexts.org/ on their page for the\\ well-ordering principle.\retTwo

         ($\Longrightarrow$)\\
         Suppose $S$ is a nonempty subset of $\mathbb{Z}_+$ with no least element. Then let $R$ be the set of lower bounds of $S$. Since $1$ is the least element of $\mathbb{Z}_+$, we know that $1 \in R$.\retTwo
         
         Now given any $k \geq 1$, if $k \in R$, we know that $\{1, \ldots, k\}$ must be a subset of $R$. Also note that $R \cap S = \emptyset$ because if that wasn't true, we'd know that $S$ has a least element. Therefore, $\{1, \ldots, k\} \cap S = \emptyset$. But then that shows that $k + 1 \notin S$ since otherwise $k + 1$ would be the least element of $S$. Furthermore, since no element of $\{1, \ldots, k\}$ is in $S$, we automatically have that $k + 1 \in R$.\retTwo

         By induction, this means that $R = \mathbb{Z}_+$. Hence, we get a contradiction as $S$ must be empty.\newpage

         ($\Longleftarrow$)\\
         Let $S$ be a subset of $\mathbb{Z}_+$ such that $1 \in S$ and $k \in S \Longrightarrow k + 1 \in S$. Then suppose that $S \neq \mathbb{Z}_+$. In that case, we know that $S^\comp \neq \emptyset$, and since $\mathbb{Z}_+$ is well-ordered, we know there is a least element $\alpha$ of $S^\comp$.\retTwo

         Because $1 \in S$, we know that $\alpha \geq 2$. But then consider that $1 \leq \alpha - 1 < \alpha$. Therefore, $\alpha - 1 \in S$, thus implying that $\alpha \in S$. This contradicts that $\alpha \in S^\comp$.\retTwo
      \end{myIndent}

      {\fontsize{13}{15}\selectfont%
      From what I've heard, when defining the positive integers, one usally takes one of the two above properties as an axiom and then proves the other as a theorem. In Munkres' book, he starts with induction and proves well-orderedness.\retTwo\retTwo
      }
   \end{myIndent}

   Facts:
   \begin{itemize}
      \item If $A$ with the order relation $<$ is well-ordered, then any subset of $A$ is well-\\ordered as well with $<$ restricted to that subset.
      \item If $A$ has the order relation $<_1$ and $B$ has the order relation $<_2$ and both are well-ordered, then $A \times B$ is well-ordered with the dictionary order.
      \item Given any countable set $A$, we know there exists a bijection $f$ from $A$ to $\mathbb{Z}_+$. Hence, given $a, b \in A$, we can say that $a < b \Longleftrightarrow f(a) < f(b)$. Then, $A$ is well-ordered by $<$ with the least element of any subset $S$ of $A$ being the element $\alpha \in A$ such that $f(\alpha)$ is the least element in $f(S)$.
      \item If a set $A$ is well-ordered, then we can make a choice function $c: \mathcal{P}(A) \longrightarrow A$ using that well-ordering.
      
      \begin{myIndent}
         Specifically, given any $B \subseteq A$, assign $c(B) = \beta$ where $\beta$\\ is the least element of $B$.

         
         \begin{myIndent}\myComment
            This is why we can pick elements of $\mathbb{Q}$ without worrying about the axiom of choice.\retTwo\retTwo
         \end{myIndent}
      \end{myIndent}
   \end{itemize}

   An important theorem (which I will hopefully prove soon) is:
   
   \begin{myIndent}
      \blab{The Well Ordering Theorem:} If $A$ is a set, there exists an order relation on $A$ that is well-ordering.
      
      \begin{myIndent}\myComment
         Note: this theorem requires the axiom of choice to prove.\retTwo
      \end{myIndent}
   \end{myIndent}

   \exOne\blab{Exercise 10.5:} Show that the well-ordering theorem implies the (infinite) axiom of choice.
   \begin{myIndent}\exTwo
      Let $\mathcal{A}$ be a collection of disjoint sets. By the well-ordering theorem, we can pick an order relation on $\bigcup\limits_{A \in \mathcal{A}}A$ that is well-ordering.\newpage
      
      \begin{myIndent}\myComment
         Note that the previous sentence is carefully worded to only make use of the finite axiom of choice. Specifically, the order relation we are picking is an element of some subset of $\bigcup\limits_{A \in \mathcal{A}}A \times \bigcup\limits_{A \in \mathcal{A}}A$.\retTwo

         If we had instead picked a well-ordering for each $A \in \mathcal{A}$, then that would require the axiom of choice as we would be making potentially infinitely many arbitrary choices of order relations.\retTwo
      \end{myIndent}

      Now let $C = \{a \in \hspace{-0.2em}\bigcup\limits_{A \in \mathcal{A}}\hspace{-0.2em}A \mid \exists A \in \mathcal{A} \suchthat a \in A \text{ and } \forall b \in A,\myHS a \leq b \}$.\retTwo Then $C$ fulfils the properties of the set that the axiom of choice would guarentee exists.\retTwo
   \end{myIndent}

   \dispDate{9/14/2024}

   \blab{Exercise 10.1:} Show that every well-ordered set has the least-upper-bound\\ property.

   \begin{myIndent}\exTwo
      Let the set $A$ with the order relation $<$ be well-ordered. Then consider any nonempty $B \subseteq A$ and suppose there exists $\alpha \in A$ such that $b < \alpha$ for all $b \in B$.\retTwo

      Let $U = \{a \in A \mid \forall b \in B, \myHS b \leq a\}$. Since $\alpha \in U$, we know that $U \neq \emptyset$. So, because $A$ is well-ordered, we know that $U$ has a least element $\beta$. This $\beta$ is by definition the least upper bound of $B$. So $\sup{B} = \beta$.\retTwo\retTwo
   \end{myIndent}

   \hOne

   Let $X$ be a well-ordered set. Given $\alpha \in X$, let $S_\alpha$ denote the set $\{x \in X \mid x < \alpha\}$. We call $S_\alpha$ the \udefine{section} of $X$ by $\alpha$.\retTwo

   \blab{Lemma 10.2:} There exists a well-ordered set $A$ having a largest element $\Omega$ such that $S_\Omega$ is uncountable but every other section of $A$ is countable.

   \begin{myIndent}\hTwo
      Proof:\\
      Starting off, let $B$ be an uncountable well-ordered set. Then let $C$ be the well-\\ordered set $\{1, 2\} \times B$ with the dictionary order. Clearly, given any $b \in B$, we have that $S_{(2, b)}$ is uncountable. So the set of $c \in C$ such that $S_c$ is uncountable is not empty.\retTwo

      Let $\Omega$ be the least element of $C$ such that $S_\Omega$ is uncountable. Then define\\ $A = S_\Omega \cup \{\Omega\}$. This is called a \udefine{minimal uncountable well-ordered set}.
      \retTwo
      
      \begin{myIndent}\myComment
         The reason we are considering $\{1, 2\} \times B$ instead of just $B$ is that if we were just considering $B$, then we wouldn't be able to guarentee that there exists $b \in B$ such that $S_b$ is uncountable.\newpage

         User MJD on https://math.stackexchange.com wrote some good intuition for why\\ this is.
         \begin{myIndent}
            While the set $\mathbb{Z}_+$ is countably infinite, all sections $S_x$ of $\mathbb{Z}_+$ are finite.\\ However, when considering $\{1, 2\} \times \mathbb{Z}_+$ with the dictionary order, we\\ have that $S_{(2, 1)}$ is countably infinite. Furthermore, all sections of $S_{(2, 1)}$\\ are finite. Thus, $S_{(2, 1)}$ would be a minimal \textit{countable} well-ordered set.\retTwo
         \end{myIndent}
      \end{myIndent}
   \end{myIndent}

   We call a set described by lemma 10.2 $\overline{S}_\Omega = S_\Omega \cup \{\Omega\}$.\retTwo

   \blab{Theorem 10.3:} If $A$ is a countable subset of $S_\Omega$, then $A$ has an upper bound in $S_\Omega$.
   
   \begin{myIndent}\hTwo
      Proof:\\
      Let $A$ be a countable subset of $S_\Omega$. For all $a \in A$, we know that $S_a$ is countable. Therefore, $B = \bigcup\limits_{a \in A}S_a$ is also countable, meaning that $S_\Omega - B \neq \emptyset$.\retTwo

      If we pick $x \in S_\Omega - B$, we must have that $x$ is an upper bound to $A$ because if $x < a$ for some $a \in A$, we would have that $x \in S_a \subseteq B$.\retTwo
      
      
      \begin{myIndent}\myComment
         If you combine this with exercise 10.1, we know that $A$ has a least upper bound.\retTwo
      \end{myIndent}
   \end{myIndent}

   \exOne
   \blab{Exercise 10.6:} Let $S_\Omega$ be a minimal uncountable well-ordered set.
   
   \begin{itemize}
      \item[(a)] Show that $S_\Omega$ has no largest element.
      \begin{myIndent}\exTwo
         Suppose $\alpha \in S_\Omega$ is the largest element of $S_\Omega$. In that case, we'd have that\\ $S_\alpha = S_\Omega - \{\alpha\}$. However, by theorem 10.3, we know that $S_\alpha$ is countable. This implies that $S_\Omega = S_\alpha \cup \{\alpha\}$ must also be countable, which is a contradiction.
         \retTwo
      \end{myIndent}
      \item[(b)] Show that for every $\alpha \in S_\Omega$, the subset $\{x \in S_\Omega \mid \alpha < x\}$ is uncountable.
      \begin{myIndent}\exTwo
         Let $\alpha \in S_\Omega$. By the law of trichotomy, we know that:
   
         {\centering $S_\Omega = \{x \in S_\Omega \mid x < \alpha\} \cup \{\alpha\} \cup \{x \in S_\Omega \mid \alpha < x\}$.\retTwo\par}
         
         Now suppose $\{x \in S_\Omega \mid \alpha < x\}$ is countable. Then as both $\{x \in S_\Omega \mid x < \alpha\}$ and $\{\alpha\}$ are countable, we have a contradiction as the three's union must also be countable. But we know $S_\Omega$ isn't.\retTwo
      \end{myIndent}
      % \item[(c)] Let $X_0$ be the subset of $S_\Omega$ consisting of all elements $x$ such that $x$ has no immediate predecessor. Show that $X_0$ is uncountable.  
   \end{itemize}

   \hOne
   \mySepTwo

   Some definitions I've been lacking:
   \begin{enumerate}
      \item Let $A$ be a set and suppose $x, y, z$ are any three different elements of $A$.
      
      {\centering\fontsize{11}{13}\selectfont%
         \begin{tabular}{|l|l|}
            \udefine{Simple [Default] Order Relation}: ($<$) & \udefine{Strict Partial Order Relation}: ($\prec$) \\ [4pt] \hline &\\ [-9pt]
   
            \begin{tabular}{l}
               Nonreflexitivity: $x \not< x$ \\
               Associativity: $x < y$ and $y < z \Rightarrow x < z$ \\
               Comparability: $x < y$ or $y < x$ is true
            \end{tabular} &
            \begin{tabular}{l}
               Nonreflexitivity: $x \not\prec x$ \\
               Associativity: $x \prec y$ and $y \prec z \Rightarrow x \prec z$\\\phantom{a}
            \end{tabular}
         \end{tabular}
      \par}
      \newpage

      
      \begin{myIndent}\myComment
         Basically, a partial order relation is allowed to not give an order for some pairings of elements. If someone just says a set is ordered, they mean the set is simply ordered.\retTwo
      \end{myIndent}

      \item Let $A$ and $B$ be sets ordered by $<_A$ and $<_B$ respectively. We say that $A$ and\\ $B$ have the same \udefine{order type} if there exists an order-preserving bijection\\ $f: A \longrightarrow B$, meaning that $\forall a_1, a_2 \in A,\myHS a_1 <_A a_2 \Longrightarrow f(a_1) <_B f(a_2)$.
      
      \begin{myIndent}\myComment
         It is trivial to show that if $f$ is an order-preserving bijection, then $f^{-1}$ is also an order-\\preserving bijection.\retTwo
      \end{myIndent}

      \item If $A$ is an ordered set and $a$ and $b$ are two different elements, then consider\\ the set $S = \{x \in A \mid a < x < b\}$. If $S = \emptyset$ we say that $b$ is the \udefine{successor} of\\ $a$ and $a$ is the \udefine{predecessor} of $b$.\retTwo\retTwo
   \end{enumerate}

   \exOne
   \blab{Exercise 10.2}:  
   \begin{itemize}
       \item[(a)] Show that in a well-ordered set, every element except the largest (if one exists) has an immediate successor
       
      \begin{myIndent}\exTwo
         Let $A$ be a well-ordered set and let $\alpha$ be any element in $A$ such that there exists $\beta \in A$ for which $\alpha < \beta$. Then consider the set $S = \{x \in A \mid \alpha < x < \beta\}$. If $S = \emptyset$, then we know $\alpha$ has $\beta$ as its successor. Meanwhile, if $S \neq \emptyset$, then since $A$ is well-ordered, we know that $A$ has a least element $\gamma$. Thus, the set $\{x \in A \mid \alpha < x < \gamma\} = \emptyset$ and we know that $\gamma$ is the successor of $\alpha$.\retTwo
      \end{myIndent}

       \item[(b)] Find a set in which every element has an immediate successor that is not well-ordered. 
       \begin{myIndent}\exTwo
         Consider the set $\mathbb{Z}$ of all integers using the standard ordering. Then for any $n \in \mathbb{Z}$, we know that its successor is $n + 1$. At the same time though, the set of all negative integers has no least element. So $\mathbb{Z}$ is not well-ordered by $<$.\retTwo
      \end{myIndent}
   \end{itemize}

   \blab{Exercise 10.6}
      \begin{itemize}
         \item[(c)] Let $X_0$ be the subset of $S_\Omega$ consisting of all elements $x$ such that $x$ has no\\ immediate predecessor. Show that $X_0$ is uncountable.
         
         \begin{myIndent}\exTwo
            Suppose $X_0$ is countable. Then by theorem 10.3, we know that $X_0$ is bounded above by some $\alpha \in S_\Omega$. Thus, there is a predecessor $x \in S_\Omega$ for any $y$ in the set $T = \{z \in S_\Omega \mid z > \beta\}$.\newpage

            Now define a function $f: \mathbb{Z}_+ \longrightarrow T$ such that $f(1) =$ the least element of $T$ and $f(n) =$ the successor of $f(n - 1)$ for all $n > 1$. We know this function is well-defined because $S_\Omega$ has no largest element according to exercise 10.6.a. So, all elements of $S_\Omega$ and thus $T$ have a successor by exercises 10.2.a, meaning our recursion formula always uniquely determines an $f(n + 1)$ for all $n \in \mathbb{Z}_+$. Hence, the principle of recursive definition guarentees a unique $f$ exists.
         \end{myIndent}
      \end{itemize}

   \newpage

\end{document}
