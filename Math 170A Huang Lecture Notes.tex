
% Note for any github stalkers. I am currently in the process
% of learning LaTeX. I don't know what I'm doing yet. Sorry
% if my code absolutely sucks.
\documentclass{book}

\usepackage{fontspec} % used to import Calibri
\usepackage{anyfontsize} % used to adjust font size

% needed for inch and other length measurements
% to be recognized
\usepackage{calc}

% for colors and text effects as is hopefully obvious
\usepackage[dvipsnames]{xcolor}
\usepackage{soul}

% control over margins
\usepackage[margin=1in]{geometry}
\usepackage[strict]{changepage}

\usepackage{mathtools}
\usepackage{amsfonts}
\usepackage{amssymb} % originally imported to get the proof square
\usepackage[overcommands]{overarrows} % Get my preferred vector arrows...

% Just am using this to get a dashed line in a table...
% Also you apparently want this to be inactive if you aren't
% using it because it slows compilation.
\usepackage{arydshln} \ADLinactivate 
\newenvironment{allowTableDashes}{\ADLactivate}{\ADLinactivate}

\usepackage{graphicx}
%\graphicspath{{./140A_images/}}

\usepackage{tikz}
   \usetikzlibrary{arrows.meta}

\newfontfamily{\calibri}{Calibri}
\setlength{\parindent}{0pt}
\definecolor{RawerSienna}{HTML}{945D27}
\setul{0.14em}{0.07em}

\newcommand{\hOne}{%
   \color{Black}%
   \fontsize{14}{15}\selectfont%
}
\newcommand{\hTwo}{%
   \color{MidnightBlue}%
   \fontsize{13}{13}\selectfont%
}
\newcommand{\hThree}{%
   \color{PineGreen}
   \fontsize{13}{13}\selectfont%
}
\newcommand{\hFour}{%
   \color{Cerulean}
   \fontsize{12}{12}\selectfont%
}
\newcommand{\myComment}{%
   \color{RawerSienna}%
   \fontsize{12}{12}\selectfont%
}
\newcommand{\teachComment}{
   \color{Orange}%
   \fontsize{12}{12}\selectfont%
}
\newcommand{\exOne}{%
   \color{Purple}%
   \fontsize{14}{14}\selectfont%
}
\newcommand{\exTwo}{%
   \color{RedViolet}%
   \fontsize{13}{13}\selectfont%
}
\newcommand{\exP}{%
   \color{VioletRed}%
   \fontsize{12}{12}\selectfont%
}

\newenvironment{myIndent}{%
   \begin{adjustwidth}{2.5em}{0em}%
}{%
   \end{adjustwidth}%
}

\newenvironment{myConstrict}{%
   \begin{adjustwidth}{2.5em}{2.5em}%
}{%
   \end{adjustwidth}%
}

\newcommand{\udefine}[1]{%
   {\setulcolor{Red}%
   \setul{0.14em}{0.07em}%
   \ul{#1}}%
}

\newcommand{\uuline}[2][.]{%
{\vphantom{a}\color{#1}%
\rlap{\rule[-0.18em]{\widthof{#2}}{0.06em}}%
\rlap{\rule[-0.32em]{\widthof{#2}}{0.06em}}}%
#2}

\newcounter{LectureNumber}
\newcommand*{\markLecture}[1]{%
   \stepcounter{LectureNumber}%
   {\huge \color{Black} \textbf{Lecture \theLectureNumber: #1} \newline}%
}

\newcommand{\pprime}{\prime\prime}

\newcounter{PropNumber}
\newcommand{\propCount}{%
   \stepcounter{PropNumber}%
   \thePropNumber%
}

\newcommand{\mySepOne}[1][.]{%
   {\noindent\color{#1}{\rule{6.5in}{1mm}}}\\%
}
\newcommand{\mySepTwo}[1][.]{%
   {\noindent\color{#1}{\rule{6.5in}{0.5mm}}}\\%
}

\newenvironment{myClosureOne}[2][.]{%
   \color{#1}%
   \begin{tabular}{|p{#2in}|} \hline \\%
}{%
   \\ \\ \hline \end{tabular}%
}

\newcommand{\retTwo}{\hfill\bigbreak}


% Programming stuff:
% ~~~~~~~~~~~~~~~~~~~~~~~~~~~~~~~~~~~~~~~~~~~~~~~~~~~~~~~~~~~~~~~~~~~~~
\usepackage{listings} % for writing code...
\lstset{
   basicstyle=\ttfamily
}

\colorlet{currentIdentifierFromLst}{Black}

\lstdefinestyle{170test}{
   language=Matlab,
   backgroundcolor = \color{Dandelion!5},
   commentstyle=\color{OliveGreen},
   keywordstyle=\color{BlueViolet},
   numbersep=5pt,
   numbers=left,
   numberstyle=\color{Gray},
   tabsize=3,
   identifierstyle=\color{BrickRed}
}

\newcommand{\lstSetTest}{%
   \lstset{style=170test}%
   \colorlet{currentIdentifierFromLst}{BrickRed}%
}

\newcommand{\identifier}[1]{%
   {\color{currentIdentifierFromLst}\texttt{#1}}%
}
% ~~~~~~~~~~~~~~~~~~~~~~~~~~~~~~~~~~~~~~~~~~~~~~~~~~~~~~~~~~~~~~~~~~~~~



\title{Math 170A Lecture Notes (Professor: Lei Huang)}
\author{Isabelle Mills}

\begin{document}
   \maketitle
   \calibri

   {\huge \color{Black} \textbf{Week 1 Notes (1/8 - 1/12/2024)}%
   \stepcounter{LectureNumber}%
   \stepcounter{LectureNumber}%
   \stepcounter{LectureNumber} \retTwo}
   
   \hOne
   For this class, we shall define the number of \udefine{flops} an
   algoirthm takes as the number of individual $+$, $-$, $\times$, $/$,
   and $\sqrt{\phantom{x}}$ operations on \uuline{real numbers}
   used in the algorithm. \retTwo
   
   \begin{myIndent} \hTwo
      For example: taking the inner product of two vectors
      $\overrightharpoonup{u} = (u_1, u_2, \ldots, u_n)$ and
      $\overrightharpoonup{v} = (v_1, v_2, \ldots, v_n)$ requires
      $n$ multiplications and $(n-1)$ additions. So we'll say
      it has a flops count of $2n-1$
   \end{myIndent}
   \hOne
   \retTwo \retTwo

   Technically, the word "flop" stands for 
   \textul{floating (point) operation}. Based on that knowledge,
   hopefully it is easier to guess what is and is not a flop. For
   instance, observe the code written below for taking an inner
   product of two $n$-vectors.

   \lstSetTest

   \begin{tabular}{ p{3.0in} p{3.0in}}
      \retTwo
      \begin{lstlisting}
   P = 0
   for i = 1:n
      P = P + v(i) * w(i)
   end
      \end{lstlisting}
      &
      \begin{myIndent} 
         {\hTwo Neither incrementing \identifier{i} nor \newline
         initializing any other variables are counted towards the 
         flop number. Because the code does \identifier{n} additions
         and multiplications between \newline floating point
         numbers, we say this function has $2n$ flops.}
      \end{myIndent}
   \end{tabular}


   \mySepOne


   For a sequence $a_n$, we define $a_n = O(b_n)$ if there exists
   real constants\\ $C, N \geq 0$ such that for $n \geq N$, \hspace{0.25em}
   $a_n \leq Cb_n$.

\end{document}