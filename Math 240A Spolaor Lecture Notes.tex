\documentclass{book}

\usepackage{fontspec} % used to import Calibri
\usepackage{anyfontsize} % used to adjust font size

% needed for inch and other length measurements
% to be recognized
\usepackage{calc}

% for colors and text effects as is hopefully obvious
\usepackage[dvipsnames]{xcolor}
\usepackage{soul}

% control over margins
\usepackage[margin=1in]{geometry}
\usepackage[strict]{changepage}

\usepackage{mathtools}
\usepackage{amsfonts}
\usepackage{bm}

\usepackage[scr=rsfso, scrscaled=.96]{mathalpha}

\usepackage{amssymb} % originally imported to get the proof square
\usepackage{xfrac}
\usepackage[overcommands]{overarrows} % Get my preferred vector arrows...
\usepackage{relsize}

% Just am using this to get a dashed line in a table...
% Also you apparently want this to be inactive if you aren't
% using it because it slows compilation.
\usepackage{arydshln} \ADLinactivate 
\newenvironment{allowTableDashes}{\ADLactivate}{\ADLinactivate}

\usepackage{graphicx}
\graphicspath{{./158_Images/}}

\usepackage{tikz}
   \usetikzlibrary{arrows.meta}
   \usetikzlibrary{graphs, graphs.standard}

\usepackage{quiver} %commutative diagrams


\newfontfamily{\calibri}{Calibri}
\setlength{\parindent}{0pt}
\definecolor{RawerSienna}{HTML}{945D27}

% ~~~~~~~~~~~~~~~~~~~~~~~~~~~~~~~~~~~~~~~~~~~~~~~~~~
%Arrow Commands:

% Thank you Bernard, gernot, and Sigur who I copied this from:
% https://tex.stackexchange.com/questions/364096/command-for-longhookrightarrow
\newcommand{\hooklongrightarrow}{\lhook\joinrel\longrightarrow}
\newcommand{\hooklongleftarrow}{\longleftarrow\joinrel\rhook}
\newcommand{\hookxlongrightarrow}[2][]{\lhook\joinrel\xrightarrow[#1]{#2}}
\newcommand{\hookxlongleftarrow}[2][]{\xleftarrow[#1]{#2}\joinrel\rhook}

% Thank you egreg who I copied from:
% https://tex.stackexchange.com/questions/260554/two-headed-version-of-xrightarrow
\newcommand{\longrightarrowdbl}{\longrightarrow\mathrel{\mkern-14mu}\rightarrow}
\newcommand{\longleftarrowdbl}{\leftarrow\mathrel{\mkern-14mu}\longleftarrow}

\newcommand{\xrightarrowdbl}[2][]{%
  \xrightarrow[#1]{#2}\mathrel{\mkern-14mu}\rightarrow
}
\newcommand{\xleftarrowdbl}[2][]{%
  \leftarrow\mathrel{\mkern-14mu}\xleftarrow[#1]{#2}
}

\newcommand{\MRoman}[1]{%
   \textrm{\MakeUppercase{\romannumeral #1}}%
}

% ~~~~~~~~~~~~~~~~~~~~~~~~~~~~~~~~~~~~~~~~~~~~~~~~~~

\newcommand{\learnToSpot}[1]{{\color{Red}#1}}

\newcommand{\hOne}{%
   \color{Black}%
   \fontsize{14}{16}\selectfont%
}
\newcommand{\hTwo}{%
\color{MidnightBlue}%
   \fontsize{13}{15}\selectfont%
}
\newcommand{\hThree}{%
   \color{PineGreen!85!Orange}
   \fontsize{12}{14}\selectfont%
}
\newcommand{\hFour}{%
   \color{Cyan}
   \fontsize{12}{14}\selectfont%
}
\newcommand{\myComment}{%
   \color{RawerSienna}%
   \fontsize{12}{14}\selectfont%
}
\newcommand{\teachComment}{
   \color{Orange}%
   \fontsize{12}{14}\selectfont%
}
\newcommand{\exOne}{%
   \color{Purple}%
   \fontsize{13}{15}\selectfont%
}
\newcommand{\exTwo}{%
   \color{Purple}%
   \fontsize{13}{15}\selectfont%
}
\newcommand{\exP}{%
   \color{Purple}%
   \fontsize{12}{14}\selectfont%
}
\newcommand{\exTwoP}{%
   \color{RedViolet}%
   \fontsize{13}{15}\selectfont%
}
\newcommand{\exPP}{%
   \color{RedViolet}%
   \fontsize{12}{14}\selectfont%
}
\newcommand{\exPPP}{%
   \color{VioletRed}%
   \fontsize{12}{14}\selectfont%
}
% ~~~~~~~~~~~~~~~~~~~~~~~~~~~~~~~~~~~~~~~~~~~~~~~~

\newcommand{\cyPen}[1]{{\vphantom{.}\color{Cerulean}#1}}
\newcommand{\redPen}[1]{{\vphantom{.}\color{Red}#1}}

\newenvironment{myIndent}{%
   \begin{adjustwidth}{2.5em}{0em}%
}{%
   \end{adjustwidth}%
}

\newenvironment{myDindent}{%
   \begin{adjustwidth}{5em}{0em}%
}{%
   \end{adjustwidth}%
}

\newenvironment{myTindent}{%
   \begin{adjustwidth}{7.5em}{0em}%
}{%
   \end{adjustwidth}%
}

\newenvironment{myConstrict}{%
   \begin{adjustwidth}{2.5em}{2.5em}%
}{%
   \end{adjustwidth}%
}

\newcommand{\udefine}[1]{{%
   \setulcolor{Red}%
   \setul{0.14em}{0.07em}%
   \ul{#1}%
}}

\newcommand{\blab}[1]{\textbf{#1}}

\newcommand{\uuline}[2][.]{%
{\vphantom{a}\color{#1}%
\rlap{\rule[-0.18em]{\widthof{#2}}{0.06em}}%
\rlap{\rule[-0.32em]{\widthof{#2}}{0.06em}}}%
#2}

\newcommand{\pprime}{{\prime\prime}}
\newcommand{\suchthat}{ \hspace{0.3em}s.t.\hspace{0.3em}}
\newcommand{\rea}[1]{\mathrm{Re}(#1)}
\newcommand{\ima}[1]{\mathrm{Im}(#1)}
\newcommand{\comp}{\mathsf{C}}
\newcommand{\card}{\mathrm{card}}
\newcommand{\diam}{\mathrm{diam}}
\newcommand{\myHS}{ \hspace{0.5em}}

\newcommand{\myId}{\mathrm{Id}}
\newcommand{\myIm}{\mathrm{im}}
\newcommand{\myObj}{\mathrm{Obj}}
\newcommand{\myHom}{\mathrm{Hom}}
\newcommand{\myEnd}{\mathrm{End}}
\newcommand{\myAut}{\mathrm{Aut}}

\newcommand{\mcateg}[1]{{\bm{\mathsf{#1}}}}

% Thank you Gonzalo Medina and Moriambar who wrote this on stack exchange:
%https://tex.stackexchange.com/questions/74125/how-do-i-put-text-over-symbols%
\newcommand{\myequiv}[1]{\stackrel{\mathclap{\mbox{\footnotesize{$#1$}}}}{\equiv}}

% Thank you chs who wrote this on stack exchange:
%https://tex.stackexchange.com/questions/89821/how-to-draw-a-solid-colored-circle%
\newcommand{\filledcirc}[1][.]{\ensuremath{\hspace{0.05em}{\color{#1}\bullet}\mathllap{\circ}\hspace{0.05em}}}

%Thank you blerbl who wrote this on stack exchange:
%https://tex.stackexchange.com/questions/25348/latex-symbol-for-does-not-divide
\newcommand{\ndiv}{\hspace{-0.3em}\not|\hspace{0.35em}}

\newcommand{\mySepOne}[1][.]{%
   {\noindent\color{#1}{\rule{6.5in}{1mm}}}\\%
}
\newcommand{\mySepTwo}[1][.]{%
   {\noindent\color{#1}{\rule{6.5in}{0.5mm}}}\\%
}

\newenvironment{myClosureOne}[2][.]{%
   \color{#1}%
   \begin{tabular}{|p{#2in}|} \hline \\%
}{%
   \\ \hline \end{tabular}%
}

\newcommand{\retTwo}{\hfill\bigbreak}

\newcommand{\mHeader}[1]{{
   \color{Black}%
   \fontsize{20}{18}\selectfont%
   #1\retTwo
}}


\title{Math 240A Notes (Professor: Luca Spolaor)}
\author{Isabelle Mills}

\begin{document}
\maketitle{}
\setul{0.14em}{0.07em}
\calibri

\hOne
\mHeader{Lecture 1 Notes: 9/26/2024}

Given an indexed family of sets $\{X_\alpha\}_{\alpha \in A}$, we define its \udefine{Cartesian Product} to be:

{\center $\prod\limits_{\alpha \in A}X_\alpha = \{f: A \longrightarrow \bigcup\limits_{\alpha \in A}X_\alpha \mid f(\alpha \in X_\alpha)\}$ \retTwo\par}

A projection is a function $\pi_\alpha : \prod\limits_{\alpha \in A}X_\alpha \longrightarrow X_\alpha$ satisfying that $f \mapsto f(\alpha)$.\retTwo

If $X, Y$ are sets, we define:
\begin{itemize}
	\item $\card(X) \leq \card(Y)$ if there exists an injection $f: X \longrightarrow Y$.
	\item  $\card(X) \geq \card(Y)$ if there exists a surjection $f: X \longrightarrow Y$.
	\item $\card(X) = \card(Y)$ if there exists a bijection $f: X \longrightarrow Y$.
	
	\begin{myIndent}\hTwo
		Note that $\card(X) \leq \card(Y) \Longleftrightarrow \card(Y) \geq \card(X)$. After all, given an injection in one direction, we can easily make a surjection in the other direction. Or given a surjection in one direction, we can \learnToSpot{(using A.O.C (axiom of choice))} easily make an injection in the other direction.\retTwo

		Also, if $\card(X) \leq \card(Y)$ and $\card(Y) \leq \card(X)$, then we know that\\ $\card(Y) = \card(X)$.
		
		\begin{myIndent}\hThree
			Proof:\\
			We know there exists $f: X \longrightarrow Y$ and $g: Y \longrightarrow X$ which are both\\ injective. Hence, $g \circ f$ is an injection from $X$ to $g(Y) \subseteq X$. By an exercise done in my math journal on page 8, we thus there exists a bijection $h$ from $X$ to $g(Y)$. And letting $g^{-1}$ be any left-inverse of $g$, we then have that $g^{-1} \circ h$ is a bijection from $X$ to $Y$.\retTwo
		\end{myIndent}
	\end{myIndent}
\end{itemize}

We say $X$ has the \udefine{cardinality of the continuum} if $\card(X) = \card(\mathbb{R})$.

\begin{myIndent}\hTwo
	Proposition: $\card(\mathcal{P}(\mathbb{N})) = \card(\mathbb{R})$.
	\begin{myIndent}\hThree
		Our textbook goes about proving this by constructing two functions: an injection and a surjection, from $\mathcal{P}(\mathbb{N})$ to $\mathbb{R}$ based on the binary expansion of any real number. That way, we know that $\card(\mathcal{P}(\mathbb{N})) \leq \card(\mathbb{R})$ and $\card(\mathcal{P}(\mathbb{N})) \geq \card(\mathbb{R})$.\retTwo
	\end{myIndent}
\end{myIndent}

Given a sequence $(x_n)_{n \in \mathbb{N}}$ in $\mathbb{R}$ we know there exists: $\limsup x_n = \inf\limits_{k \geq 1}(\sup\limits_{n \geq k} x_n)$ and\\ [-10pt] $\liminf x_n = \sup\limits_{k \geq 1}(\inf\limits_{n \geq k} x_n)$.\retTwo

Also, given a function $f: \mathbb{R} \longrightarrow \overline{\mathbb{R}}$, we can define:

{\centering $\limsup\limits_{x \rightarrow a}f(x) = \inf\limits_{\delta > 0}\left(\sup\limits_{0 < |x-a|<\delta}\hspace{-1em}f(x)\right)$.\newpage\par}

If $X$ is an arbitrary set and $f: X \longrightarrow [0, \infty]$, we define:

{\centering $\sum\limits_{x \in X}f(x) = \sup\left\{\sum\limits_{x \in F}f(x) \mid F \subseteq X \suchthat F \text{ is finite}\right\}$.\retTwo\par}

\begin{myIndent}\hTwo
	Cool Proposition from textbook (not covered in lecture):
	\begin{myIndent}\hTwo
		Let $A = \{x \in X \mid f(x) > 0\}$. If $A$ is uncountable, then $\sum\limits_{x \in X}f(x) = \infty$.\retTwo If $A$ is countably infinite and $g:\mathbb{N} \longrightarrow A$ is a bijection, then\\ $\sum\limits_{x \in X}f(x) = \sum\limits_{n = 1}^\infty f(g(n))$.\retTwo
		
		\begin{myIndent}\hThree
			Proof of first statement:\\
			$A = \bigcup\limits_{n \in \mathbb{N}}A_n$ where $A_n = \{x \in X \mid f(x) > \frac{1}{n}\}$.\retTwo

			If $A$ is uncountable, we must have that some $A_n$ is uncountable. But then for any finite set $F \subseteq X$, we have that $\sum\limits_{x \in F}f(x) > \frac{\card(F)}{n}$. So $\sum\limits_{x \in X}f(x)$ is\\ [-7pt] unbounded.\retTwo
		\end{myIndent}
	\end{myIndent}
\end{myIndent}

A metric space $(X, \rho)$ is a set $X$ equipped with a distance function\\ $\rho: X \times X \longrightarrow [0, \infty)$. We denote the open ball of radius $r$ about $x$ to be\\ $B(r, x) = \{y \in X \mid \rho(x, y) < r\}$. And you remember our definitions from\\ 140A... right?\retTwo


\begin{myIndent}\hTwo
	\blab{Proposition 0.21:} Every open set in $\mathbb{R}$ is a countable union of disjoint open intervals.
		
	\begin{myIndent}\myComment
		We proved this as part of a homework exercise in Math 140A.\retTwo
	\end{myIndent}
\end{myIndent}

\exOne Given a metric space $(X, \rho)$, an element $x \in X$, and sets $F, E \subseteq X$, we can define:
\begin{itemize}
	\item $\rho(x, E) = \rho_E(x) = \inf\{\rho(x, y) \mid y \in E\}$.
	\item $\rho(F, E) = \inf\{\rho_E(y) \mid y \in F\}$.
\end{itemize}


\begin{myIndent}\exTwoP
	\blab{Exercise:} $g(x, E) = 0 \Longleftrightarrow x \in \overline{E}$.
	
	\begin{myIndent}\exPP
		Proof:\\
		If $\inf\{\rho(x, y) \mid y \in E\} = 0$, then there exists a sequence $\{y_n\}$ in $E$ such that\\ $\rho(x, y_n) \rightarrow 0$. This implies $x \in \overline{E}$. Similarly, if $x \in \overline{E}$, we can construct a sequence $\{y_n\}$ such that $\rho(x, y_n) < \frac{1}{n}$ for all $n$. Then:
		
		{\centering $0 \leq \inf\{\rho(x, y) \mid y \in E\} \leq \inf\{\rho(x, y_n) \mid n \in \mathbb{N}\} = 0$.\retTwo\par}
	\end{myIndent}
\end{myIndent}

\hOne
Given a subset $E$ of a metric space $(X, \rho)$, we define:

{\centering$\diam(E) = \sup\{\rho(x, y)\mid x, y \in E\}$.\retTwo\par}

If $\diam(E) < \infty$, we say $E$ is \udefine{bounded}. If $\forall \varepsilon > 0$, $E$ can be covered by finitely many balls of radius $\varepsilon$, then we say $E$ is \udefine{totally bounded}.\newpage

\begin{myIndent}\exTwo
	\blab{Exercise:} $E$ being totally bounded implies $E$ is bounded.
	\begin{myIndent}\exPP
		Pick $\varepsilon > 0$ and let $\{z_1, \ldots, z_n\}$ be the set of points such that $E \subseteq \bigcup\limits_{k = 1}^n B(\varepsilon, z_n)$.\retTwo

		Then given any $x, y \in E$, we can assume that $x \in B(\varepsilon, z_i)$ and $y \in B(\varepsilon, z_j)$. So, $\rho(x, y) \leq \rho(x, z_i) + \rho(z_i, z_j) + \rho(z_j, y) < 2\varepsilon + \max\{\rho(z_i, z_j) \mid 1 \leq i, j \leq n\}$.\retTwo
	\end{myIndent}

	The converse is not generally true. For instance, if you use the discrete metric, then any set with more than one element will have a diameter of $1$. But if $0 < \varepsilon < 1$, then it will be impossible to cover an infinite set with finitely many balls.\retTwo
\end{myIndent}

\mySepTwo

\mHeader{Lecture 2 Notes: 10/1/2024}

\begin{myIndent}\hTwo
   \blab{Proposition:} Suppose $E$ is a subset of a metric space $(X, \rho)$. Then the following are equivalent.
   
   \begin{enumerate}
      \item $E$ is complete and totally bounded
      \item All sequences $(x_n) \subseteq E$, have a convergent subsequence.
      \item For all open covers $\{V_\alpha\}_{\alpha \in A}$ of $E$, there exists $V_{\alpha_1}, \ldots, V_{\alpha_n}$ such that\\ $E \subseteq \bigcup\limits_{i=1}^n V_{\alpha_i}$.\retTwo
   \end{enumerate}
   
   \begin{myIndent}\hThree
      Proof:\\
      (1) $\Longrightarrow$ (2):
      \begin{myIndent}
         Lemma:\\
         If $E$ is totally bounded and $F \subseteq E$, then $F$ is totally bounded.
         \begin{myIndent}\hFour
            Given any $\varepsilon > 0$, let $\{x_1, \ldots, x_n\}$ be a subset of $E$ such that\\ $E \subseteq \bigcup\limits_{i = 1}^n B(\sfrac{\varepsilon}{2}, x_i)$. Then consider the collection of sets:\\ [-8pt] \phantom{aaaaaaaaaaaaaaaaaaaaaaaaaaaaaaa} $\{F \cap B(\sfrac{\varepsilon}{2}, x_i)\} - \{\emptyset\}$.\retTwo

            We know the diameter of each $F \cap B(\sfrac{\varepsilon}{2}, x_i)$ is at most $\varepsilon$. So in each set, pick $y_i \in F \cap B(\sfrac{\varepsilon}{2}, x_i)$. Then for some $m \leq n$:

            {\centering $F \subseteq \bigcup\limits_{i=1}^m B(\varepsilon, y_i)$ \retTwo\par}
         \end{myIndent}

         Let $A_1 = E$. Then for $k \geq 2$ we recursively define $A_k$ as follows:\retTwo
         
         Assuming $A_{k-1} \cap (x_n)_{n\in \mathbb{N}}$ is infinite and $A_{k-1}$ is totally bounded, choose\\ [-2pt] $\{y_1, \ldots, y_m\}$ in $A_k$ such that $A_k \subseteq \bigcup\limits_{i = 1}^m B(2^{-n}, y_i)$. Importantly, since\\ $(x_n)_{n\in \mathbb{N}} \cap A_{k-1}$ is infinite, we know one of those open balls contains\\ [4pt] infinitely many points in our sequence. So set $A_{k}$ equal to that ball\\ [4pt] intersected with $E$. Note that by our lemma, $A_k$ is totally bounded.\newpage

         Now pick any $x_{n_1}$ and then for all $k \geq 2$ pick $x_{n_k} \in A_k$ such that\\ $n_k > n_{k - 1}$. That way, $(x_{n_k})_{k \in \mathbb{Z}_+}$ is a subsequence of $(x_{n})_{n \in \mathbb{Z}_+}$. Also,\\ we know that $(x_{n_k})_{k \in \mathbb{Z}_+}$ is Cauchy. Hence, since $E$ is complete, we know\\ that it converges to some $x$ in $E$.\retTwo
      \end{myIndent}

      (2) $\Longrightarrow$ (1):
      \begin{myIndent}
         Firstly, suppose $E$ is not complete. Then there exists a sequence $(x_n)_{n \in \mathbb{N}}$ that is Cauchy but does not converge in $E$. Importantly, because $(x_n)_{n \in \mathbb{N}}$ is Cauchy, if there was a convergent subsequence, we know the limit of that subsequence would have to be the limit of the whole sequence. But that doesn't exist. So, we know (2) can't be true.\retTwo

         Secondly, suppose $E$ is not totally bounded. Then there exists $\varepsilon > 0$ such that it is impossible to cover $E$ in balls of radius $\varepsilon$. So, we can recursively define a sequence $(x_n)_{n \in \mathbb{N}}$ in $E$ satisfying that:
         
         {\centering$x_n \in E - \bigcup\limits_{i = 1}^{n-1}B(\varepsilon, x_{i})$.\retTwo\par}

         Importantly, for all natural numbers $n \neq m$, we have that $\rho(x_n, x_m) \geq \varepsilon$. So, it is impossible to find a convergent subsequence of $(x_n)$, meaning (2) is false.\retTwo
      \end{myIndent}

      (1) and (2) $\Longrightarrow$ (3):
      \begin{myIndent}
         Let $\{V_\alpha\}_{\alpha \in A}$ be an open cover of $E$.\retTwo

         Suppose for the sake of contradiction that for all $n \in \mathbb{N}$, there is a ball $B_n$ of radius $2^{-n}$ centered in $E$ such that $B_n \cap E \neq \emptyset$ but $B_n \not\subseteq V_\alpha$ for all $\alpha \in A$. Then we can construct a sequence $(x_n)_{n \in \mathbb{N}}$ in $E$ such that $x_n \in B_n \cap E$\\ for all $n \in \mathbb{N}$. By (2), we know there is a subsequence that converges to some $x \in E$. Importantly, we know $x \in V_\alpha$ for some $\alpha \in A$, and because $V_\alpha$ is open, there is $\varepsilon > 0$ such that $B(\varepsilon, x) \subseteq V_\alpha$. But now we get a contradiction because by picking $n$ such that $2^{-n} < \sfrac{\varepsilon}{3}$ and $\rho(x, x_n) < \sfrac{\varepsilon}{3}$, we have for all $y \in B_n$ that:

         {\centering $\rho(x, y) \leq \rho(x, x_n) + \rho(x_n, y) < 2^{-n} + 2^{-n+1} < \varepsilon$ \retTwo\par}

         So $B_n \subseteq B(\varepsilon, x) \subseteq V_\alpha$.\retTwo

         We've thus shown that for some $n \in N$, all balls of radius $2^{-n}$ centered\\ in $E$ are contained by some $V_\alpha$. And assuming (1), we can cover $E$ with finitely many balls of radius $2^{-n}$ It follows that by picking a $V_\alpha$ containing a ball for each ball covering $E$, we've found a finite covering $E$ using the sets in $\{V_\alpha\}_{\alpha \in A}$.\retTwo
      \end{myIndent}

      (3) $\Longrightarrow$ (2):
      \begin{myIndent}
         Suppose $(x_n)_{n \in \mathbb{N}}$ is a sequence in $E$ with no convergent subsequence. Then for each $x \in E$, there must exist $\varepsilon_x > 0$ such that $B(\varepsilon_x, x) \cap (x_n)_{n \in \mathbb{N}}$ is finite. (If $\varepsilon_x$ didn't exist, we could construct a Cauchy subsequence converging to $x$).\newpage

         But now $\{B(\varepsilon_x, x)\}_{x \in E}$ is an open cover of $E$ with no finite subcover of $E$ because it will take an infinite cover to cover all of $(x_n)_{n \in \mathbb{N}}$.\retTwo
      \end{myIndent}
   \end{myIndent}

   If $E$ satisfies all three of the above properties, we say $E$ is \udefine{compact}.\retTwo

   \blab{Corollary}: $K \subseteq \mathbb{R}^n$ is compact iff it's closed and bounded.
\end{myIndent}

\mySepTwo

Roughly speaking, we want a measure to be a function $\mu: \mathcal{P}(\mathbb{R}^n) \longrightarrow [0, \infty)$ such that $E \mapsto \mu(E) =$ "the area of $E$". Also, we would like it if:
\begin{enumerate}
   \item[(i)] $\mu([0, 1)^n) = 1$
   \item[(ii)] $\mu(\text{rotation, translation, or reflection of } A) = \mu(A)$
   \item[(iii)] $\mu(\bigcup\limits_{i=1}^\infty A_i) = \sum\limits_{i = 1}^\infty \mu(A_i)$ if $A_i \cap A_j \neq \emptyset \Longrightarrow i = j$.
\end{enumerate}

Unfortunately, the properties as written above are inconsistent.

\begin{myIndent}\hTwo
   \blab{Vitali Sets:}\\
   Defining $x \sim y$ iff $x - y \in \mathbb{Q}$, let $N \subseteq [0, 1)$ be a set such that $N \cap [x]$ has precisely one element for all $x \in \mathbb{R}$. Next let $R = [0, 1) \cap \mathbb{Q}$, and for all $r \in R$ define:
   
   {\centering$N_r = \{x + r \mid x \in N \cap [0, 1-r)\}\cup \{x + r - 1 \mid x \in N \cap [1 - r, 1)\} $.\retTwo\par}

   Importantly, note that $N_r \subseteq [0, 1)$. Plus, the two sets being unioned over to make $N_r$ are both disjoint and can be translated around so that they are still disjoint but their union forms $N$. Hence assuming $\mu : \mathcal{P}(\mathbb{R}^n) \longrightarrow [0, \infty)$ satisfying (ii) and (iii), we know $\mu(N_r) = \mu(N)$.\retTwo

   Also, for all $y \in [0, 1)$, if $x \in N \cap [y]$, we know that $y \in N_r$ where $r = x - y$ if $x \geq y$, or $r = x - y + 1$ if $x < y$. Hence, $[0, 1) = \bigcup\limits_{r \in R}N_r$.\retTwo

   Also, given any $N_r$ and $N_s$, if $x \in N_r \cap N_s$, then we'd be able to show that both $x - r$ or $x - r + 1$ and $x - s$ or $x - s + 1$ are distinct elements of $N$ in the same equivalence class, which contradicts how we defined $N$.
   \begin{myTindent}\begin{myTindent}\myComment
      You work through the scratch work of the different cases on your own! :P\retTwo
   \end{myTindent}\end{myTindent}

   So supposing $\mu$ satisfies (i) and (iii) and because $R$ is countable, we have that:
   
   {\centering $1 = \sum\limits_{r \in R}\mu(N_r) = \sum\limits_{r \in R}(N) = 0 \text{ or } \infty$.\retTwo\par}

   This is a contradiction.\retTwo
\end{myIndent}

Furthermore, the problem is not the countable union property as is demonstrated by the Banach-Tarsky paradox:\newpage


\begin{myIndent}
   \hTwo
   \blab{Theorem:} Let $U$ and $V$ be arbitrary bounded sets in $\mathbb{R}^n$ where $n \geq 3$. Then there exists $E_1,\ldots, E_N, F_1, \ldots, F_N$ in $\mathbb{R}^n$ such that:
   \begin{itemize}
      \item $E_i \cap E_j = \emptyset$ for all $i \neq j$ and $\bigcup\limits_{i = 1}^NE_i = U$
      \item $F_i \cap F_j = \emptyset$ for all $i \neq j$ and $\bigcup\limits_{i = 1}^NF_i = V$
      \item $E_i$ and $F_i$ are congruent for all $i \in \{1, \ldots, N\}$.\retTwo
   \end{itemize}

   Supposing that $\mu(E_j)$ and $\mu(F_j)$ exists for all $j$ and that $\mu$ satisfies (i), (ii), and (iii) except only for finite unions, then that would suggest all sets have the same "area", which we know doesn't make sense.
\end{myIndent}

What we will do to fix this issue is only define $\mu$ on a subset of $\mathcal{P}(\mathbb{R}^n)$.\retTwo

\mySepTwo 

Let $X \neq \emptyset$. An \udefine{algebra of sets} in $X$ is a collection $\mathcal{A} \subseteq \mathcal{P}(X)$ which is closed under finite unions and complements. If is $\mathcal{A}$ is also closed under countable unions, we say $\mathcal{A}$ is a $\sigma$-algebra.

\begin{myIndent}\hTwo
   Observations:
   \begin{enumerate}
      \item Algebras of sets are closed under finite intersections and $\sigma$-algebras are\\ closed under countable intersection.
      
      \begin{myIndent}\hThree
         This is because $\bigcap\limits_{n \in \mathbb{N}}\hspace{-0.2em}A_n = \left(\bigcup\limits_{n \in \mathbb{N}}\hspace{-0.2em}A_n^\comp\right)^\comp$\hspace{-0.4em}.
      \end{myIndent}

      \item If $\mathcal{A}$ is closed under disjoint countable union, then it's closed under arbitrary countable unions.
      
      \begin{myIndent}\hThree
         This is because $\bigcup\limits_{n \in \mathbb{N}}\hspace{-0.2em}A_n = A_1 \cup \bigcup\limits_{n \geq 2}\left(A_n \cap \left(\bigcup\limits_{i = 1}^{n-1} A_i\right)^{\hspace{-0.2em}\comp}\right)$
      \end{myIndent}

      \item If $\{\mathcal{E}_\alpha\}_{\alpha \in A}$ is a collection of $\sigma$-algebras, then $\bigcap\limits_{\alpha \in A}\mathcal{E}_\alpha$ is a $\sigma$-algebra.
      
      \begin{myIndent}\hThree
         This is pretty trivial to prove. It should remind you of topologies.\retTwo 
      \end{myIndent}
   \end{enumerate}
\end{myIndent}

\exOne

\blab{Exercise 1.1:} A family of sets $\mathcal{R} \subseteq \mathcal{P}(X)$ is called a \udefine{ring} if it is closed under finite unions and difference. If $\mathcal{R}$ is also closed under countable unions, it is called a $\sigma$-ring.
\begin{enumerate}
   \item[(a)] Rings are closed under finite intersections and $\sigma$-rings are closed under countable intersections.
   
   \begin{myIndent}\exTwoP
      If $\mathcal{R}$ is a ring and $A_1, \ldots, A_n \in \mathcal{R}$, then:\\ [-18pt]
      
      {\centering $\bigcap\limits_{i = 1}^n A_n = A_1 - \bigcup\limits_{i = 2}^n(A_1 - A_i) \in \mathcal{R}$ \retTwo\par}

      This is because each $A_1 - A_i \in \mathcal{R}$, meaning $\bigcup\limits_{i = 2}^n(A_1 - A_i) \in \mathcal{R}$, and so finally\\ [-10pt] $A_1 - \bigcup\limits_{i = 2}^n(A_1 - A_i) \in \mathcal{R}$.\retTwo

      If $\mathcal{R}$ is a $\sigma$-algebra, we can replace the finite intersection and union used in the prior reasoning with a countable intersection and union.\retTwo
   \end{myIndent}
   \item[(b)] If $\mathcal{R}$ is a ring (or $\sigma$-ring), then $\mathcal{R}$ is an algebra (or $\sigma$-algebra) iff $X \in \mathbb{R}$.
   
   \begin{myIndent}\exTwoP
      To start, while this is pedantic, technically if $\mathcal{R}$ is empty, then it is trivially an algebra and a ring despite not containing $X$. So, let's assume $\mathcal{R} \neq \emptyset$.\retTwo

      ($\Longrightarrow$) Suppose $\mathcal{R}$ is an algebra. Then note that $\emptyset \in \mathcal{R}$ because for any $A \in \mathcal{R}$, $A - A \in \mathcal{R}$. So taking complements, we get that $X \in \mathcal{R}$.\retTwo

      ($\Longleftarrow$) Suppose $X \in \mathcal{R}$. Then for any $A \in \mathcal{R}$, we know that $A^\comp = X - A \in \mathcal{R}$. So $\mathcal{R}$ is an algebra (or $\sigma$-algebra).\retTwo
   \end{myIndent}

   \item[(c)] If $\mathcal{R}$ is a $\sigma$-ring, then $\mathcal{A} = \{E \subseteq X \mid E \in \mathcal{R} \text{ or } E^\comp \in \mathcal{R}\}$ is a $\sigma$-algebra.
   
   \begin{myIndent}\exTwoP
      To start, we know that $\mathcal{A}$ is closed under complements because for any $A \in \mathcal{A}$,
      \begin{myIndent}
         $A \in \mathcal{R} \Longrightarrow (A^\comp)^\comp \in \mathcal{R} \Longrightarrow A^\comp \in \mathcal{A}$\\
         $A \notin \mathcal{R} \Longrightarrow A^\comp \in \mathcal{R} \Longrightarrow A^\comp \in \mathcal{A}$\retTwo
      \end{myIndent}

      Also, let $(E_n)_{n \in \mathbb{N}}$ be a countable collection of sets in $\mathcal{A}$. Then define\\ $A = \{n \in \mathbb{N} \mid E_n^\comp \notin \mathcal{R}\}$ and $B = \{n \in \mathbb{N} \mid E_n^\comp \in \mathcal{R}\}$. Clearly, we have that:

      {\centering $\bigcup\limits_{n \in \mathbb{N}}E_n = \bigcup\limits_{n \in A}E_n \cup \bigcup\limits_{n \in B}E_n = \bigcup\limits_{n \in A}E_n \cup \bigcup\limits_{n \in B}(E_n^\comp)^{\comp}$\\ [2pt]\par}

      Also $\bigcup\limits_{n \in B}(E_n^\comp)^\comp = \left(\bigcap\limits_{n \in B}\hspace{-0.3em}E_n^\comp\right)^{\hspace{-0.3em}\comp}$\hspace{-0.3em}, and by part (a), we know that $E_B \coloneq \bigcap\limits_{n \in B}\hspace{-0.3em}E_n^\comp \in \mathcal{R}$.\retTwo

      Similarly, we know $E_A \coloneq \bigcup\limits_{n \in A}\hspace{-0.3em} E_n \in \mathcal{R}$. So, we've shown that $\bigcup\limits_{n \in \mathbb{N}}E_n = E_A \cup E_B^\comp$\\ [-8pt] where $E_A, E_B \in \mathcal{R}$.\retTwo

      Finally, note that $E_A \cup E_B^\comp = (E_B - E_A)^\comp$. Since $E_B - E_A \in \mathcal{R}$, we know that $(E_B - E_A)^\comp \in \mathcal{A}$. \retTwo
   \end{myIndent}

   \item[(d)] If $\mathcal{R}$ is a $\sigma$-ring, then $\mathcal{A} = \{E \subseteq X \mid E \cap F \in \mathcal{R} \text{ for all } F \in \mathcal{R}\}$ is a $\sigma$-algebra.
   
   \begin{myIndent}\exTwoP
      To start if $E \in \mathcal{A}$, then $E^\comp \in \mathcal{A}$ because for all $F \in \mathcal{R}$ we have that:
      
      {\centering$E^\comp \cap F = F - E = F - (E \cap F) \in \mathcal{R}$.\retTwo\par}

      Also, let $(E_n)_{n \in \mathbb{N}}$ be a countable collection of sets in $\mathcal{A}$. Then for all $F \in \mathcal{R}$, we have that $\left(\bigcup\limits_{n \in \mathbb{N}}E_n\right) \cap F = \bigcup\limits_{n \in \mathbb{N}}(E_n \cap F) \in \mathcal{R}$. So $\mathcal{A}$ is closed under countable union.\newpage
   \end{myIndent}
\end{enumerate}

\hOne Let $\mathcal{E} \subseteq \mathcal{P}(X)$ be a collection of sets. Since the intersection of $\sigma$-algebras is still a $\sigma$-algebra, we define $\mathcal{M}(\mathcal{E})$ to be the smallest $\sigma$-algebra that contains $\mathcal{E}$. In other words, $\mathcal{M}(\mathcal{E})$ is the intersection of all $\sigma$-algebras that contain $\mathcal{E}$.\retTwo

We call $\mathcal{M}(\mathcal{E})$ the $\sigma$-algebra generated by $\mathcal{E}$.


\begin{myIndent}\hTwo
   \blab{Lemma:} if $\mathcal{E} \in \mathcal{M}(\mathcal{F})$, then $\mathcal{M}(\mathcal{E}) \subseteq \mathcal{M}(\mathcal{F})$.
\end{myIndent}

\mySepTwo

Let $(X, \rho)$ be a metric space. We define the \udefine{Borel $\sigma$-algebra} on $X$: $\mathcal{B}_X$, to be the $\sigma$-algebra generated by the collection of all open sets, or equivalently the collection of all closed sets.

\begin{itemize}
   \item A set is $G_\delta$ if it is a countable intersection of open sets.
   \item A set is $F_\sigma$ if it is a countable union of closed sets.
   \item A set is $G_{\delta\sigma}$ if it is a countable union of $G_\delta$ sets.
   \item A set is $F_{\sigma\delta}$ if it is a countable intersection of $F_\sigma$ sets.
   \begin{myTindent}\myComment
      You can hopefully see the pattern. Also the professor isn't sure how much we'll use this $\delta$ and $\sigma$ notation in class.
   \end{myTindent}
\end{itemize}

\exOne\blab{Exercise 1.2:} $\mathcal{B}_\mathbb{R}$ is generated by each of the following:
\begin{itemize}
   \item [(a)] the set of open intervals: $\mathcal{E}_1 = \{(a, b) \mid a < b\}$
   \item [(b)] the set of closed intervals: $\mathcal{E}_2 = \{[a, b] \mid a < b\}$
   \item [(c)] the set of half-open intervals:
   \begin{itemize}
      \item[(i)] $\mathcal{E}_3 = \{(a, b] \mid a < b\}$
      \item[(ii)] $\mathcal{E}_4 = \{[a, b) \mid a < b\}$
   \end{itemize}
   \item [(c)] the set of open rays:
   \begin{itemize}
      \item[(i)] $\mathcal{E}_5 = \{(a, \infty) \mid a \in \mathbb{R}\}$
      \item[(ii)] $\mathcal{E}_6 = \{(-\infty, a) \mid a \in \mathbb{R}\}$
   \end{itemize}
   \item [(d)] the set of closed rays:
   \begin{itemize}
      \item[(i)] $\mathcal{E}_7 = \{[a, \infty) \mid a \in \mathbb{R}\}$
      \item[(ii)] $\mathcal{E}_8 = \{(-\infty, a] \mid a \in \mathbb{R}\}$\retTwo
   \end{itemize}
\end{itemize}

\begin{myIndent}\exTwoP
   Proof:\\
   We trivially have that $\mathcal{M}(\mathcal{E}_1), \mathcal{M}(\mathcal{E}_2), \mathcal{M}(\mathcal{E}_5), \mathcal{M}(\mathcal{E}_6), \mathcal{M}(\mathcal{E}_7), \mathcal{M}(\mathcal{E}_8) \subseteq \mathcal{B}_{\mathbb{R}}$ since each of them contain either only open sets or only closed sets. As for the other inclusions, we must do more work.\newpage

   \begin{itemize}
      \item[(a)] Note that $\mathbb{Q}$ is a countable dense subset of $\mathbb{R}$. Hence, a countable base of $\mathbb{R}$ is the set: $\mathcal{F} = \{(p - q, p + q) \subset \mathbb{R} \mid p, q \in \mathbb{Q} \text{ and } q \neq 0\}$. In other words, given any open set $E \subseteq \mathbb{R}$, there is a countable subcollection of $\mathcal{F}$ whose union is $E$.
      \begin{myIndent}\exPPP
         To see why, let $x \in E$. Since $E$ is open, there exists $r > 0$ with $B(r, x) \subseteq E$.\\ Next, pick $p \in (x, x + \frac{r}{2}) \cap \mathbb{Q}$, followed by $q \in (p - x, r - p) \cap \mathbb{Q}$. Then $x \in (p - q, p + q) \in \mathcal{F}$ and $(p - q, p + q) \subseteq (x - r, x + r)$.\retTwo
   
         With that, we've now shown that for all $x \in E$, there exists $F \in \mathcal{F}$ such\\ that $x \in F \subseteq E$. If we choose such an $F_x$ for all $x \in E$, we then get\\ that $E = \bigcup\limits_{x \in E}F_x$. So $E$ is the union of a subcollection of $\mathcal{F}$. But since $\mathcal{F}$ is\\ [-9pt]\phantom{aaaaaaaaaaaaaaa} countable, the set $\{F_x \in \mathcal{F} \mid x \in E\}$ is also countable.\retTwo
      \end{myIndent}
   
      Importantly, $\mathcal{F} \subset \mathcal{E}_1$. So $\mathcal{M}(\mathcal{F}) \subseteq \mathcal{M}(\mathcal{E}_1)$. However as shown above, we\\ must have that $\mathcal{M}(\mathcal{F})$ includes all open sets. So by our lemma on the previous page, $\mathcal{B}_{\mathbb{R}} \subseteq \mathcal{M}(\mathcal{F}) \subseteq \mathcal{M}(\mathcal{E}_1)$.\retTwo

      \item[(b)] Given any $E = (a, b) \in \mathcal{E}_1$, we can write that $E = \hspace{-0.3em}\bigcup\limits_{n \in \mathbb{Z}_+}\hspace{-0.3em} [a+\frac{1}{n}, b - \frac{1}{n}]$. Thus,\\ [-10pt] $\mathcal{E}_1 \subseteq \mathcal{M}(\mathcal{E}_2)$, meaning $\mathcal{B}_\mathbb{R} = \mathcal{M}(\mathcal{E}_1) \subseteq \mathcal{M}(\mathcal{E}_2)$.\retTwo
      
      \item[(c)] Remember that for these two, we still need to show that $\mathcal{M}(\mathcal{E}_1), \mathcal{M}(\mathcal{E}_2) \in \mathcal{B}_\mathbb{R}$.
      \begin{itemize}
         \item[(i)] Firstly note that if $F = (a, b] \in \mathcal{E}_3$, then $ = \hspace{-0.3em}\bigcap\limits_{n \in \mathbb{Z}_+}\hspace{-0.3em} (a, b+\frac{1}{n})$. So $\mathcal{E}_3 \subseteq \mathcal{M}(\mathcal{E}_1)$.\\
         On the other hand, if $E = (a, b) \in \mathcal{E}_1$, we have that $E = \hspace{-0.3em}\bigcup\limits_{n \in \mathbb{Z}_+}\hspace{-0.3em} (a, b-\frac{1}{n}]$.\\ [-9pt] So $\mathcal{E}_1 \subseteq \mathcal{M}(\mathcal{E}_3)$.\retTwo
         
         By our lemma on the previous page, we thus have that:
         
         {\centering $\mathcal{B}_\mathbb{R} = \mathcal{M}(\mathcal{E}_1) \subseteq \mathcal{M}(\mathcal{E}_3) \subseteq \mathcal{M}(\mathcal{E}_1) = \mathcal{B}_\mathbb{R}$. \retTwo\par}

         \item[(ii)] Mostly identical reasoning as with $\mathcal{E}_3$ shows that:
         
         {\centering $\mathcal{B}_\mathbb{R} = \mathcal{M}(\mathcal{E}_1) \subseteq \mathcal{M}(\mathcal{E}_4) \subseteq \mathcal{M}(\mathcal{E}_1) = \mathcal{B}_\mathbb{R}$\retTwo\par}
      \end{itemize}

      \item[(d)] \phantom{a}\\ [-10pt]
      \begin{itemize}
         \item[(i)] If $E = (a, b) \in \mathcal{E}_1$, then we know that:
         
         {\centering$E  = (a, \infty) \cap (\hspace{-0.3em}\bigcap\limits_{n \in \mathbb{Z}_+}\hspace{-0.3em} (b - \frac{1}{n}, \infty))^\comp \in \mathcal{M}(\mathcal{E}_5)$.\\\par}

         So $\mathcal{E}_1 \subseteq \mathcal{M}(\mathcal{E}_5)$, meaning $\mathcal{B}_\mathbb{R} = \mathcal{M}(\mathcal{E}_1) \subseteq \mathcal{M}(\mathcal{E}_5)$.\retTwo

         \item[(ii)] Analogous reasoning to that with $\mathcal{E}_5$ shows that $\mathcal{B}_\mathbb{R} = \mathcal{M}(\mathcal{E}_1) \subseteq \mathcal{M}(\mathcal{E}_6)$.\retTwo
      \end{itemize}

      \item[(e)] \phantom{a}\\ [-10pt]
      \begin{itemize}
         \item[(i)] If $E = (a, \infty) \in \mathcal{E}_6$, then we have that $E = \hspace{-0.3em}\bigcup\limits_{n \in \mathbb{Z}_+}\hspace{-0.3em} [a + \frac{1}{n}, \infty)$. So $\mathcal{E}_5 \subseteq \mathcal{M}(\mathcal{E}_7)$,\\ [-10pt] meaning that $\mathcal{B}_\mathbb{R} = \mathcal{M}(\mathcal{E}_5) \subseteq \mathcal{M}(\mathcal{E}_7)$.\retTwo

         \item[(ii)] Analogous reasoning as with $\mathcal{E}_7$ shows that $\mathcal{B}_\mathbb{R} = \mathcal{M}(\mathcal{E}_6) \subseteq \mathcal{M}(\mathcal{E}_8)$.\newpage
      \end{itemize}
   \end{itemize}
\end{myIndent}

\blab{Exercise 1.3:} Let $\mathcal{A}$ be an infinite $\sigma$-algebra on $X$.
\begin{itemize}
   \item[(a)] $\mathcal{A}$ contains an infinite sequence of disjoint sets.
   
   \begin{myIndent}\exTwoP
      By the Hausdorff maximum principle, we know there is a subcollection $\mathcal{S}$\\ of $\mathcal{A}$ which is simply ordered by proper subset and is not contained in any other collection of $\mathcal{A}$ which is simply ordered by proper subset.\retTwo
      
      We claim $\mathcal{S}$ can't be finite. For suppose $\mathcal{S} = \{A_1, \ldots, A_n\}$ is a sequence of sets in $\mathcal{A}$ simply ordered by proper subset which are indexed such that:
      \begin{itemize}
         \item[\bullet] $A_1 = \emptyset$
         \item[\bullet] $A_i \subset A_{i + 1}$ for all $i \in \{1, \ldots, n-1\}$
         \item[\bullet] $A_n = X$.
         
         \begin{myTindent}\exPPP
            (If $\mathcal{S}$ is maximal, we know $\emptyset, X \in \mathcal{S}$)\retTwo
         \end{myTindent}
      \end{itemize} 
      
      Then choose $B \notin \mathcal{A} - \mathcal{M}(\{A_1, \ldots, A_n\})$. We know we can do this because $\mathcal{M}(\{A_1, \ldots, A_n\})$ is finite while $\mathcal{A}$ is infinite.\retTwo
      
      Next let $k = \min\{i \in \{1, \ldots, n\} \mid B \subset A_i\}$. In other words, let $k$ be such that $B \subset A_k$ but $B \not\subset A_{k-1}$. If $A_{k-1} \cap B \neq \emptyset$, then because $B \not\subseteq A_{k-1}$, we know that $A_{k-1} \subset A_{k-1} \cup B \subset A_{k}$. Meanwhile, if $A_{k-1} \cap B = \emptyset$, then note that:\\ $A_{k-1} \cup B = A_k \Longrightarrow B = A_k - A_{k-1} \in \mathcal{M}(\{A_1, \ldots, A_n\})$. So we know\\ that $A_{k-1} \cup B \neq A_k$. At the same time, since $B \neq \emptyset$, we know that\\ $A_{k-1} \subset A_{k-1} \cup B \subset A_k$.\retTwo

      By transitivity, we know that $A_{k-1} \cup B$ is comparable via proper subset with $A_i$ for all $i \in \{1, \ldots, n\}$. Hence, we've shown that $\mathcal{S} \cup \{A_{k-1} \cup B\}$ is a sequence of sets in $\mathcal{A}$ simply ordered by proper subset. But this contradicts that $\mathcal{S}$ is maximal.\retTwo

      Now that we know $\mathcal{S}$ is infinite, let $(E_n)_{n \in \mathbb{Z}_+}$ be a sequence of sets in $\mathcal{S}$\\ satisfying that $E_n \subset E_{n + 1}$. Then we have that $(E_{n+1} - E_n)_{n \in \mathbb{Z}_+}$ is an infinite\\ sequence of disjoint sets in $\mathcal{A}$.\retTwo
   \end{myIndent}

   \item[(b)] Show that $\card(\mathcal{A}) \geq \mathfrak{c}$.
   
   \begin{myIndent}\exTwoP
      Let $(E_n)_{n \in \mathbb{N}}$ be a sequence of disjoint sets in $\mathcal{A}$. Then if we define the map\\ [2pt] $f: [0, 1]^\mathbb{N} \longrightarrow \mathcal{A}$ such that $(a_0, a_1, a_2, \ldots)$ is mapped to the union of all $E_n$\\ [2pt] such that $a_n = 1$, we have that $f$ is an injection.\retTwo

      Hence, $\card(\mathcal{A}) \geq \card([0, 1]^{\mathbb{N}})$. And since there is a trivial bijection\\ from $[0, 1]^{\mathbb{N}}$ and $\mathcal{P}(\mathbb{N})$, plus the fact that we proved early on in the class\\ that $\card(\mathcal{P}(\mathbb{N})) = \card(\mathbb{R})$, we thus know that $\card(\mathcal{A}) \geq \mathfrak{c}$.\retTwo
   \end{myIndent}
\end{itemize}


\end{document}












% $\bigotimes\limits_{\alpha \in A}\mathcal{M}_\alpha = \{\pi^{-1}_\alpha(E_\alpha) \mid E_\alpha \in \mathcal{M}_\alpha, \alpha \in A\}$