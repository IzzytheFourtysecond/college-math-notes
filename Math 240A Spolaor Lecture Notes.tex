\documentclass{book}

\usepackage{fontspec} % used to import Calibri
\usepackage{anyfontsize} % used to adjust font size

% needed for inch and other length measurements
% to be recognized
\usepackage{calc}

% for colors and text effects as is hopefully obvious
\usepackage[dvipsnames]{xcolor}
\usepackage{soul}

% control over margins
\usepackage[margin=1in]{geometry}
\usepackage[strict]{changepage}

\usepackage{mathtools}
\usepackage{amsfonts}
\usepackage{bm}

\usepackage[scr=rsfso, scrscaled=.96]{mathalpha}

\usepackage{amssymb} % originally imported to get the proof square
\usepackage{xfrac}
\usepackage[overcommands]{overarrows} % Get my preferred vector arrows...
\usepackage{relsize}

% Just am using this to get a dashed line in a table...
% Also you apparently want this to be inactive if you aren't
% using it because it slows compilation.
\usepackage{arydshln} \ADLinactivate 
\newenvironment{allowTableDashes}{\ADLactivate}{\ADLinactivate}

\usepackage{graphicx}
\graphicspath{{./158_Images/}}

\usepackage{tikz}
   \usetikzlibrary{arrows.meta}
   \usetikzlibrary{graphs, graphs.standard}

\usepackage{quiver} %commutative diagrams


\newfontfamily{\calibri}{Calibri}
\setlength{\parindent}{0pt}
\definecolor{RawerSienna}{HTML}{945D27}

% ~~~~~~~~~~~~~~~~~~~~~~~~~~~~~~~~~~~~~~~~~~~~~~~~~~
%Arrow Commands:

% Thank you Bernard, gernot, and Sigur who I copied this from:
% https://tex.stackexchange.com/questions/364096/command-for-longhookrightarrow
\newcommand{\hooklongrightarrow}{\lhook\joinrel\longrightarrow}
\newcommand{\hooklongleftarrow}{\longleftarrow\joinrel\rhook}
\newcommand{\hookxlongrightarrow}[2][]{\lhook\joinrel\xrightarrow[#1]{#2}}
\newcommand{\hookxlongleftarrow}[2][]{\xleftarrow[#1]{#2}\joinrel\rhook}

% Thank you egreg who I copied from:
% https://tex.stackexchange.com/questions/260554/two-headed-version-of-xrightarrow
\newcommand{\longrightarrowdbl}{\longrightarrow\mathrel{\mkern-14mu}\rightarrow}
\newcommand{\longleftarrowdbl}{\leftarrow\mathrel{\mkern-14mu}\longleftarrow}

\newcommand{\xrightarrowdbl}[2][]{%
  \xrightarrow[#1]{#2}\mathrel{\mkern-14mu}\rightarrow
}
\newcommand{\xleftarrowdbl}[2][]{%
  \leftarrow\mathrel{\mkern-14mu}\xleftarrow[#1]{#2}
}

\newcommand{\MRoman}[1]{%
   \textrm{\MakeUppercase{\romannumeral #1}}%
}

% ~~~~~~~~~~~~~~~~~~~~~~~~~~~~~~~~~~~~~~~~~~~~~~~~~~

\newcommand{\learnToSpot}[1]{{\color{Red}#1}}

\newcommand{\hOne}{%
   \color{Black}%
   \fontsize{14}{16}\selectfont%
}
\newcommand{\hTwo}{%
\color{MidnightBlue}%
   \fontsize{13}{15}\selectfont%
}
\newcommand{\hThree}{%
   \color{PineGreen!85!Orange}
   \fontsize{12}{14}\selectfont%
}
\newcommand{\hFour}{%
   \color{Cyan!80!black}
   \fontsize{12}{14}\selectfont%
}
\newcommand{\myComment}{%
   \color{RawerSienna}%
   \fontsize{12}{14}\selectfont%
}
\newcommand{\teachComment}{
   \color{Orange}%
   \fontsize{12}{14}\selectfont%
}
\newcommand{\exOne}{%
   \color{Purple}%
   \fontsize{13}{15}\selectfont%
}
\newcommand{\exTwo}{%
   \color{Purple}%
   \fontsize{13}{15}\selectfont%
}
\newcommand{\exP}{%
   \color{Purple}%
   \fontsize{12}{14}\selectfont%
}
\newcommand{\exTwoP}{%
   \color{RedViolet}%
   \fontsize{13}{15}\selectfont%
}
\newcommand{\exPP}{%
   \color{RedViolet}%
   \fontsize{12}{14}\selectfont%
}
\newcommand{\exPPP}{%
   \color{VioletRed}%
   \fontsize{12}{14}\selectfont%
}
% ~~~~~~~~~~~~~~~~~~~~~~~~~~~~~~~~~~~~~~~~~~~~~~~~

\newcommand{\cyPen}[1]{{\vphantom{.}\color{Cerulean}#1}}
\newcommand{\redPen}[1]{{\vphantom{.}\color{Red}#1}}

\newenvironment{myIndent}{%
   \begin{adjustwidth}{2.5em}{0em}%
}{%
   \end{adjustwidth}%
}

\newenvironment{myDindent}{%
   \begin{adjustwidth}{5em}{0em}%
}{%
   \end{adjustwidth}%
}

\newenvironment{myTindent}{%
   \begin{adjustwidth}{7.5em}{0em}%
}{%
   \end{adjustwidth}%
}

\newenvironment{myConstrict}{%
   \begin{adjustwidth}{2.5em}{2.5em}%
}{%
   \end{adjustwidth}%
}

\newcommand{\udefine}[1]{{%
   \setulcolor{Red}%
   \setul{0.14em}{0.07em}%
   \ul{#1}%
}}

\newcommand{\blab}[1]{\textbf{#1}}

\newcommand{\uuline}[2][.]{%
{\vphantom{a}\color{#1}%
\rlap{\rule[-0.18em]{\widthof{#2}}{0.06em}}%
\rlap{\rule[-0.32em]{\widthof{#2}}{0.06em}}}%
#2}

\newcommand{\pprime}{{\prime\prime}}
\newcommand{\suchthat}{ \hspace{0.3em}s.t.\hspace{0.3em}}
\newcommand{\rea}[1]{\mathrm{Re}(#1)}
\newcommand{\ima}[1]{\mathrm{Im}(#1)}
\newcommand{\comp}{\mathsf{C}}
\newcommand{\card}{\mathrm{card}}
\newcommand{\diam}{\mathrm{diam}}
\newcommand{\myHS}{ \hspace{0.5em}}

\newcommand{\myId}{\mathrm{Id}}
\newcommand{\myIm}{\mathrm{im}}
\newcommand{\myObj}{\mathrm{Obj}}
\newcommand{\myHom}{\mathrm{Hom}}
\newcommand{\myEnd}{\mathrm{End}}
\newcommand{\myAut}{\mathrm{Aut}}

\newcommand{\mcateg}[1]{{\bm{\mathsf{#1}}}}

% Thank you Gonzalo Medina and Moriambar who wrote this on stack exchange:
%https://tex.stackexchange.com/questions/74125/how-do-i-put-text-over-symbols%
\newcommand{\myequiv}[1]{\stackrel{\mathclap{\mbox{\footnotesize{$#1$}}}}{\equiv}}

% Thank you chs who wrote this on stack exchange:
%https://tex.stackexchange.com/questions/89821/how-to-draw-a-solid-colored-circle%
\newcommand{\filledcirc}[1][.]{\ensuremath{\hspace{0.05em}{\color{#1}\bullet}\mathllap{\circ}\hspace{0.05em}}}

%Thank you blerbl who wrote this on stack exchange:
%https://tex.stackexchange.com/questions/25348/latex-symbol-for-does-not-divide
\newcommand{\ndiv}{\hspace{-0.3em}\not|\hspace{0.35em}}

\newcommand{\mySepOne}[1][.]{%
   {\noindent\color{#1}{\rule{6.5in}{1mm}}}\\%
}
\newcommand{\mySepTwo}[1][.]{%
   {\noindent\color{#1}{\rule{6.5in}{0.5mm}}}\\%
}

\newenvironment{myClosureOne}[2][.]{%
   \color{#1}%
   \begin{tabular}{|p{#2in}|} \hline \\%
}{%
   \\ \hline \end{tabular}%
}

\newcommand{\retTwo}{\hfill\bigbreak}

\newcommand{\mHeader}[1]{{
   \color{Black}%
   \fontsize{20}{18}\selectfont%
   #1\retTwo
}}


\title{Math 240A Notes (Professor: Luca Spolaor)}
\author{Isabelle Mills}

\begin{document}
\maketitle{}
\setul{0.14em}{0.07em}
\calibri

\hOne
\mHeader{Lecture 1 Notes: 9/26/2024}

Given an indexed family of sets $\{X_\alpha\}_{\alpha \in A}$, we define its \udefine{Cartesian Product} to be:

{\center $\prod\limits_{\alpha \in A}X_\alpha = \{f: A \longrightarrow \bigcup\limits_{\alpha \in A}X_\alpha \mid f(\alpha \in X_\alpha)\}$ \retTwo\par}

A projection is a function $\pi_\alpha : \prod\limits_{\alpha \in A}X_\alpha \longrightarrow X_\alpha$ satisfying that $f \mapsto f(\alpha)$.\retTwo

If $X, Y$ are sets, we define:
\begin{itemize}
	\item $\card(X) \leq \card(Y)$ if there exists an injection $f: X \longrightarrow Y$.
	\item  $\card(X) \geq \card(Y)$ if there exists a surjection $f: X \longrightarrow Y$.
	\item $\card(X) = \card(Y)$ if there exists a bijection $f: X \longrightarrow Y$.
	
	\begin{myIndent}\hTwo
		Note that $\card(X) \leq \card(Y) \Longleftrightarrow \card(Y) \geq \card(X)$. After all, given an injection in one direction, we can easily make a surjection in the other direction. Or given a surjection in one direction, we can \learnToSpot{(using A.O.C (axiom of choice))} easily make an injection in the other direction.\retTwo

		Also, if $\card(X) \leq \card(Y)$ and $\card(Y) \leq \card(X)$, then we know that\\ $\card(Y) = \card(X)$.
		
		\begin{myIndent}\hThree
			Proof:\\
			We know there exists $f: X \longrightarrow Y$ and $g: Y \longrightarrow X$ which are both\\ injective. Hence, $g \circ f$ is an injection from $X$ to $g(Y) \subseteq X$. By an exercise done in my math journal on page 8, we thus there exists a bijection $h$ from $X$ to $g(Y)$. And letting $g^{-1}$ be any left-inverse of $g$, we then have that $g^{-1} \circ h$ is a bijection from $X$ to $Y$.\retTwo
		\end{myIndent}
	\end{myIndent}
\end{itemize}

We say $X$ has the \udefine{cardinality of the continuum} if $\card(X) = \card(\mathbb{R})$.

\begin{myIndent}\hTwo
	Proposition: $\card(\mathcal{P}(\mathbb{N})) = \card(\mathbb{R})$.
	\begin{myIndent}\hThree
		Our textbook goes about proving this by constructing two functions: an injection and a surjection, from $\mathcal{P}(\mathbb{N})$ to $\mathbb{R}$ based on the binary expansion of any real number. That way, we know that $\card(\mathcal{P}(\mathbb{N})) \leq \card(\mathbb{R})$ and $\card(\mathcal{P}(\mathbb{N})) \geq \card(\mathbb{R})$.\retTwo
	\end{myIndent}
\end{myIndent}

Given a sequence $(x_n)_{n \in \mathbb{N}}$ in $\mathbb{R}$ we know there exists: $\limsup x_n = \inf\limits_{k \geq 1}(\sup\limits_{n \geq k} x_n)$ and\\ [-10pt] $\liminf x_n = \sup\limits_{k \geq 1}(\inf\limits_{n \geq k} x_n)$.\retTwo

Also, given a function $f: \mathbb{R} \longrightarrow \overline{\mathbb{R}}$, we can define:

{\centering $\limsup\limits_{x \rightarrow a}f(x) = \inf\limits_{\delta > 0}\left(\sup\limits_{0 < |x-a|<\delta}\hspace{-1em}f(x)\right)$.\newpage\par}

If $X$ is an arbitrary set and $f: X \longrightarrow [0, \infty]$, we define:

{\centering $\sum\limits_{x \in X}f(x) = \sup\left\{\sum\limits_{x \in F}f(x) \mid F \subseteq X \suchthat F \text{ is finite}\right\}$.\retTwo\par}

\begin{myIndent}\hTwo
	Cool Proposition from textbook (not covered in lecture):
	\begin{myIndent}\hTwo
		Let $A = \{x \in X \mid f(x) > 0\}$. If $A$ is uncountable, then $\sum\limits_{x \in X}f(x) = \infty$.\retTwo If $A$ is countably infinite and $g:\mathbb{N} \longrightarrow A$ is a bijection, then\\ $\sum\limits_{x \in X}f(x) = \sum\limits_{n = 1}^\infty f(g(n))$.\retTwo
		
		\begin{myIndent}\hThree
			Proof of first statement:\\
			$A = \bigcup\limits_{n \in \mathbb{N}}A_n$ where $A_n = \{x \in X \mid f(x) > \frac{1}{n}\}$.\retTwo

			If $A$ is uncountable, we must have that some $A_n$ is uncountable. But then for any finite set $F \subseteq X$, we have that $\sum\limits_{x \in F}f(x) > \frac{\card(F)}{n}$. So $\sum\limits_{x \in X}f(x)$ is\\ [-7pt] unbounded.\retTwo
		\end{myIndent}
	\end{myIndent}
\end{myIndent}

A metric space $(X, \rho)$ is a set $X$ equipped with a distance function\\ $\rho: X \times X \longrightarrow [0, \infty)$. We denote the open ball of radius $r$ about $x$ to be\\ $B(r, x) = \{y \in X \mid \rho(x, y) < r\}$. And you remember our definitions from\\ 140A... right?\retTwo


\begin{myIndent}\hTwo
	\blab{Proposition 0.21:} Every open set in $\mathbb{R}$ is a countable union of disjoint open intervals.
		
	\begin{myIndent}\myComment
		We proved this as part of a homework exercise in Math 140A.\retTwo
	\end{myIndent}
\end{myIndent}

\exOne Given a metric space $(X, \rho)$, an element $x \in X$, and sets $F, E \subseteq X$, we can define:
\begin{itemize}
	\item $\rho(x, E) = \rho_E(x) = \inf\{\rho(x, y) \mid y \in E\}$.
	\item $\rho(F, E) = \inf\{\rho_E(y) \mid y \in F\}$.
\end{itemize}


\begin{myIndent}\exTwoP
	\blab{Exercise:} $g(x, E) = 0 \Longleftrightarrow x \in \overline{E}$.
	
	\begin{myIndent}\exPP
		Proof:\\
		If $\inf\{\rho(x, y) \mid y \in E\} = 0$, then there exists a sequence $\{y_n\}$ in $E$ such that\\ $\rho(x, y_n) \rightarrow 0$. This implies $x \in \overline{E}$. Similarly, if $x \in \overline{E}$, we can construct a sequence $\{y_n\}$ such that $\rho(x, y_n) < \frac{1}{n}$ for all $n$. Then:
		
		{\centering $0 \leq \inf\{\rho(x, y) \mid y \in E\} \leq \inf\{\rho(x, y_n) \mid n \in \mathbb{N}\} = 0$.\retTwo\par}
	\end{myIndent}
\end{myIndent}

\hOne
Given a subset $E$ of a metric space $(X, \rho)$, we define:

{\centering$\diam(E) = \sup\{\rho(x, y)\mid x, y \in E\}$.\retTwo\par}

If $\diam(E) < \infty$, we say $E$ is \udefine{bounded}. If $\forall \varepsilon > 0$, $E$ can be covered by finitely many balls of radius $\varepsilon$, then we say $E$ is \udefine{totally bounded}.\newpage

\begin{myIndent}\exTwo
	\blab{Exercise:} $E$ being totally bounded implies $E$ is bounded.
	\begin{myIndent}\exPP
		Pick $\varepsilon > 0$ and let $\{z_1, \ldots, z_n\}$ be the set of points such that $E \subseteq \bigcup\limits_{k = 1}^n B(\varepsilon, z_n)$.\retTwo

		Then given any $x, y \in E$, we can assume that $x \in B(\varepsilon, z_i)$ and $y \in B(\varepsilon, z_j)$. So, $\rho(x, y) \leq \rho(x, z_i) + \rho(z_i, z_j) + \rho(z_j, y) < 2\varepsilon + \max\{\rho(z_i, z_j) \mid 1 \leq i, j \leq n\}$.\retTwo
	\end{myIndent}

	The converse is not generally true. For instance, if you use the discrete metric, then any set with more than one element will have a diameter of $1$. But if $0 < \varepsilon < 1$, then it will be impossible to cover an infinite set with finitely many balls.\retTwo
\end{myIndent}

\mySepTwo

\mHeader{Lecture 2 Notes: 10/1/2024}

\begin{myIndent}\hTwo
   \blab{Proposition:} Suppose $E$ is a subset of a metric space $(X, \rho)$. Then the following are equivalent.
   
   \begin{enumerate}
      \item $E$ is complete and totally bounded
      \item All sequences $(x_n) \subseteq E$, have a convergent subsequence.
      \item For all open covers $\{V_\alpha\}_{\alpha \in A}$ of $E$, there exists $V_{\alpha_1}, \ldots, V_{\alpha_n}$ such that\\ $E \subseteq \bigcup\limits_{i=1}^n V_{\alpha_i}$.\retTwo
   \end{enumerate}
   
   \begin{myIndent}\hThree
      Proof:\\
      (1) $\Longrightarrow$ (2):
      \begin{myIndent}
         Lemma:\\
         If $E$ is totally bounded and $F \subseteq E$, then $F$ is totally bounded.
         \begin{myIndent}\hFour
            Given any $\varepsilon > 0$, let $\{x_1, \ldots, x_n\}$ be a subset of $E$ such that\\ $E \subseteq \bigcup\limits_{i = 1}^n B(\sfrac{\varepsilon}{2}, x_i)$. Then consider the collection of sets:\\ [-8pt] \phantom{aaaaaaaaaaaaaaaaaaaaaaaaaaaaaaa} $\{F \cap B(\sfrac{\varepsilon}{2}, x_i)\} - \{\emptyset\}$.\retTwo

            We know the diameter of each $F \cap B(\sfrac{\varepsilon}{2}, x_i)$ is at most $\varepsilon$. So in each set, pick $y_i \in F \cap B(\sfrac{\varepsilon}{2}, x_i)$. Then for some $m \leq n$:

            {\centering $F \subseteq \bigcup\limits_{i=1}^m B(\varepsilon, y_i)$ \retTwo\par}
         \end{myIndent}

         Let $A_1 = E$. Then for $k \geq 2$ we recursively define $A_k$ as follows:\retTwo
         
         Assuming $A_{k-1} \cap (x_n)_{n\in \mathbb{N}}$ is infinite and $A_{k-1}$ is totally bounded, choose\\ [-2pt] $\{y_1, \ldots, y_m\}$ in $A_k$ such that $A_k \subseteq \bigcup\limits_{i = 1}^m B(2^{-n}, y_i)$. Importantly, since\\ $(x_n)_{n\in \mathbb{N}} \cap A_{k-1}$ is infinite, we know one of those open balls contains\\ [4pt] infinitely many points in our sequence. So set $A_{k}$ equal to that ball\\ [4pt] intersected with $E$. Note that by our lemma, $A_k$ is totally bounded.\newpage

         Now pick any $x_{n_1}$ and then for all $k \geq 2$ pick $x_{n_k} \in A_k$ such that\\ $n_k > n_{k - 1}$. That way, $(x_{n_k})_{k \in \mathbb{Z}_+}$ is a subsequence of $(x_{n})_{n \in \mathbb{Z}_+}$. Also,\\ we know that $(x_{n_k})_{k \in \mathbb{Z}_+}$ is Cauchy. Hence, since $E$ is complete, we know\\ that it converges to some $x$ in $E$.\retTwo
      \end{myIndent}

      (2) $\Longrightarrow$ (1):
      \begin{myIndent}
         Firstly, suppose $E$ is not complete. Then there exists a sequence $(x_n)_{n \in \mathbb{N}}$ that is Cauchy but does not converge in $E$. Importantly, because $(x_n)_{n \in \mathbb{N}}$ is Cauchy, if there was a convergent subsequence, we know the limit of that subsequence would have to be the limit of the whole sequence. But that doesn't exist. So, we know (2) can't be true.\retTwo

         Secondly, suppose $E$ is not totally bounded. Then there exists $\varepsilon > 0$ such that it is impossible to cover $E$ in balls of radius $\varepsilon$. So, we can recursively define a sequence $(x_n)_{n \in \mathbb{N}}$ in $E$ satisfying that:
         
         {\centering$x_n \in E - \bigcup\limits_{i = 1}^{n-1}B(\varepsilon, x_{i})$.\retTwo\par}

         Importantly, for all natural numbers $n \neq m$, we have that $\rho(x_n, x_m) \geq \varepsilon$. So, it is impossible to find a convergent subsequence of $(x_n)$, meaning (2) is false.\retTwo
      \end{myIndent}

      (1) and (2) $\Longrightarrow$ (3):
      \begin{myIndent}
         Let $\{V_\alpha\}_{\alpha \in A}$ be an open cover of $E$.\retTwo

         Suppose for the sake of contradiction that for all $n \in \mathbb{N}$, there is a ball $B_n$ of radius $2^{-n}$ centered in $E$ such that $B_n \cap E \neq \emptyset$ but $B_n \not\subseteq V_\alpha$ for all $\alpha \in A$. Then we can construct a sequence $(x_n)_{n \in \mathbb{N}}$ in $E$ such that $x_n \in B_n \cap E$\\ for all $n \in \mathbb{N}$. By (2), we know there is a subsequence that converges to some $x \in E$. Importantly, we know $x \in V_\alpha$ for some $\alpha \in A$, and because $V_\alpha$ is open, there is $\varepsilon > 0$ such that $B(\varepsilon, x) \subseteq V_\alpha$. But now we get a contradiction because by picking $n$ such that $2^{-n} < \sfrac{\varepsilon}{3}$ and $\rho(x, x_n) < \sfrac{\varepsilon}{3}$, we have for all $y \in B_n$ that:

         {\centering $\rho(x, y) \leq \rho(x, x_n) + \rho(x_n, y) < 2^{-n} + 2^{-n+1} < \varepsilon$ \retTwo\par}

         So $B_n \subseteq B(\varepsilon, x) \subseteq V_\alpha$.\retTwo

         We've thus shown that for some $n \in N$, all balls of radius $2^{-n}$ centered\\ in $E$ are contained by some $V_\alpha$. And assuming (1), we can cover $E$ with finitely many balls of radius $2^{-n}$ It follows that by picking a $V_\alpha$ containing a ball for each ball covering $E$, we've found a finite covering $E$ using the sets in $\{V_\alpha\}_{\alpha \in A}$.\retTwo
      \end{myIndent}

      (3) $\Longrightarrow$ (2):
      \begin{myIndent}
         Suppose $(x_n)_{n \in \mathbb{N}}$ is a sequence in $E$ with no convergent subsequence. Then for each $x \in E$, there must exist $\varepsilon_x > 0$ such that $B(\varepsilon_x, x) \cap (x_n)_{n \in \mathbb{N}}$ is finite. (If $\varepsilon_x$ didn't exist, we could construct a Cauchy subsequence converging to $x$).\newpage

         But now $\{B(\varepsilon_x, x)\}_{x \in E}$ is an open cover of $E$ with no finite subcover of $E$ because it will take an infinite cover to cover all of $(x_n)_{n \in \mathbb{N}}$.\retTwo
      \end{myIndent}
   \end{myIndent}

   If $E$ satisfies all three of the above properties, we say $E$ is \udefine{compact}.\retTwo

   \blab{Corollary}: $K \subseteq \mathbb{R}^n$ is compact iff it's closed and bounded.
\end{myIndent}

\mySepTwo

Roughly speaking, we want a measure to be a function $\mu: \mathcal{P}(\mathbb{R}^n) \longrightarrow [0, \infty)$ such that $E \mapsto \mu(E) =$ "the area of $E$". Also, we would like it if:
\begin{enumerate}
   \item[(i)] $\mu([0, 1)^n) = 1$
   \item[(ii)] $\mu(\text{rotation, translation, or reflection of } A) = \mu(A)$
   \item[(iii)] $\mu(\bigcup\limits_{i=1}^\infty A_i) = \sum\limits_{i = 1}^\infty \mu(A_i)$ if $A_i \cap A_j \neq \emptyset \Longrightarrow i = j$.
\end{enumerate}

Unfortunately, the properties as written above are inconsistent.

\begin{myIndent}\hTwo
   \blab{Vitali Sets:}\\
   Defining $x \sim y$ iff $x - y \in \mathbb{Q}$, let $N \subseteq [0, 1)$ be a set such that $N \cap [x]$ has precisely one element for all $x \in \mathbb{R}$. Next let $R = [0, 1) \cap \mathbb{Q}$, and for all $r \in R$ define:
   
   {\centering$N_r = \{x + r \mid x \in N \cap [0, 1-r)\}\cup \{x + r - 1 \mid x \in N \cap [1 - r, 1)\} $.\retTwo\par}

   Importantly, note that $N_r \subseteq [0, 1)$. Plus, the two sets being unioned over to make $N_r$ are both disjoint and can be translated around so that they are still disjoint but their union forms $N$. Hence assuming $\mu : \mathcal{P}(\mathbb{R}^n) \longrightarrow [0, \infty)$ satisfying (ii) and (iii), we know $\mu(N_r) = \mu(N)$.\retTwo

   Also, for all $y \in [0, 1)$, if $x \in N \cap [y]$, we know that $y \in N_r$ where $r = x - y$ if $x \geq y$, or $r = x - y + 1$ if $x < y$. Hence, $[0, 1) = \bigcup\limits_{r \in R}N_r$.\retTwo

   Also, given any $N_r$ and $N_s$, if $x \in N_r \cap N_s$, then we'd be able to show that both $x - r$ or $x - r + 1$ and $x - s$ or $x - s + 1$ are distinct elements of $N$ in the same equivalence class, which contradicts how we defined $N$.
   \begin{myTindent}\begin{myTindent}\myComment
      You work through the scratch work of the different cases on your own! :P\retTwo
   \end{myTindent}\end{myTindent}

   So supposing $\mu$ satisfies (i) and (iii) and because $R$ is countable, we have that:
   
   {\centering $1 = \sum\limits_{r \in R}\mu(N_r) = \sum\limits_{r \in R}(N) = 0 \text{ or } \infty$.\retTwo\par}

   This is a contradiction.\retTwo
\end{myIndent}

Furthermore, the problem is not the countable union property as is demonstrated by the Banach-Tarsky paradox:\newpage


\begin{myIndent}
   \hTwo
   \blab{Theorem:} Let $U$ and $V$ be arbitrary bounded sets in $\mathbb{R}^n$ where $n \geq 3$. Then there exists $E_1,\ldots, E_N, F_1, \ldots, F_N$ in $\mathbb{R}^n$ such that:
   \begin{itemize}
      \item $E_i \cap E_j = \emptyset$ for all $i \neq j$ and $\bigcup\limits_{i = 1}^NE_i = U$
      \item $F_i \cap F_j = \emptyset$ for all $i \neq j$ and $\bigcup\limits_{i = 1}^NF_i = V$
      \item $E_i$ and $F_i$ are congruent for all $i \in \{1, \ldots, N\}$.\retTwo
   \end{itemize}

   Supposing that $\mu(E_j)$ and $\mu(F_j)$ exists for all $j$ and that $\mu$ satisfies (i), (ii), and (iii) except only for finite unions, then that would suggest all sets have the same "area", which we know doesn't make sense.
\end{myIndent}

What we will do to fix this issue is only define $\mu$ on a subset of $\mathcal{P}(\mathbb{R}^n)$.\retTwo

\mySepTwo 

Let $X \neq \emptyset$. An \udefine{algebra of sets} in $X$ is a nonempty collection $\mathcal{A} \subseteq \mathcal{P}(X)$ which is closed under finite unions and complements. If is $\mathcal{A}$ is also closed under countable unions, we say $\mathcal{A}$ is a $\sigma$-algebra.

\begin{myIndent}\hTwo
   Observations:
   \begin{enumerate}
      \item Algebras of sets are closed under finite intersections and $\sigma$-algebras are\\ closed under countable intersection. (This also means algebras of sets are closed under set differences.)
      
      \begin{myIndent}\hThree
         This is because $\bigcap\limits_{n \in \mathbb{N}}\hspace{-0.2em}A_n = \left(\bigcup\limits_{n \in \mathbb{N}}\hspace{-0.2em}A_n^\comp\right)^\comp$\hspace{-0.4em}.
      \end{myIndent}

      \item If $\mathcal{A}$ is closed under disjoint countable union, then it's closed under arbitrary countable unions.
      
      \begin{myIndent}\hThree
         This is because $\bigcup\limits_{n \in \mathbb{N}}\hspace{-0.2em}A_n = A_1 \cup \bigcup\limits_{n \geq 2}\left(A_n \cap \left(\bigcup\limits_{i = 1}^{n-1} A_i\right)^{\hspace{-0.2em}\comp}\right)$
      \end{myIndent}

      \item If $\{\mathcal{E}_\alpha\}_{\alpha \in A}$ is a collection of $\sigma$-algebras, then $\bigcap\limits_{\alpha \in A}\mathcal{E}_\alpha$ is a $\sigma$-algebra.
      
      \begin{myIndent}\hThree
         This is pretty trivial to prove. It should remind you of topologies.\retTwo 
      \end{myIndent}
   \end{enumerate}
\end{myIndent}

\exOne

\blab{Exercise 1.1:} A family of sets $\mathcal{R} \subseteq \mathcal{P}(X)$ is called a \udefine{ring} if it is closed under finite unions and difference. If $\mathcal{R}$ is also closed under countable unions, it is called a $\sigma$-ring.
\begin{enumerate}
   \item[(a)] Rings are closed under finite intersections and $\sigma$-rings are closed under countable intersections.
   
   \begin{myIndent}\exTwoP
      If $\mathcal{R}$ is a ring and $A_1, \ldots, A_n \in \mathcal{R}$, then:\\ [-18pt]
      
      {\centering $\bigcap\limits_{i = 1}^n A_n = A_1 - \bigcup\limits_{i = 2}^n(A_1 - A_i) \in \mathcal{R}$ \retTwo\par}

      This is because each $A_1 - A_i \in \mathcal{R}$, meaning $\bigcup\limits_{i = 2}^n(A_1 - A_i) \in \mathcal{R}$, and so finally\\ [-10pt] $A_1 - \bigcup\limits_{i = 2}^n(A_1 - A_i) \in \mathcal{R}$.\retTwo

      If $\mathcal{R}$ is a $\sigma$-algebra, we can replace the finite intersection and union used in the prior reasoning with a countable intersection and union.\retTwo
   \end{myIndent}
   \item[(b)] If $\mathcal{R}$ is a ring (or $\sigma$-ring), then $\mathcal{R}$ is an algebra (or $\sigma$-algebra) iff $X \in \mathbb{R}$.
   
   \begin{myIndent}\exTwoP
      ($\Longrightarrow$) Suppose $\mathcal{R}$ is an algebra. Then note that $\emptyset \in \mathcal{R}$ because for any $A \in \mathcal{R}$, $A - A \in \mathcal{R}$. So taking complements, we get that $X \in \mathcal{R}$.\retTwo

      ($\Longleftarrow$) Suppose $X \in \mathcal{R}$. Then for any $A \in \mathcal{R}$, we know that $A^\comp = X - A \in \mathcal{R}$. So $\mathcal{R}$ is an algebra (or $\sigma$-algebra).\retTwo
   \end{myIndent}

   \item[(c)] If $\mathcal{R}$ is a $\sigma$-ring, then $\mathcal{A} = \{E \subseteq X \mid E \in \mathcal{R} \text{ or } E^\comp \in \mathcal{R}\}$ is a $\sigma$-algebra.
   
   \begin{myIndent}\exTwoP
      To start, we know that $\mathcal{A}$ is closed under complements because for any $A \in \mathcal{A}$,
      \begin{myIndent}
         $A \in \mathcal{R} \Longrightarrow (A^\comp)^\comp \in \mathcal{R} \Longrightarrow A^\comp \in \mathcal{A}$\\
         $A \notin \mathcal{R} \Longrightarrow A^\comp \in \mathcal{R} \Longrightarrow A^\comp \in \mathcal{A}$\retTwo
      \end{myIndent}

      Also, let $(E_n)_{n \in \mathbb{N}}$ be a countable collection of sets in $\mathcal{A}$. Then define\\ $A = \{n \in \mathbb{N} \mid E_n^\comp \notin \mathcal{R}\}$ and $B = \{n \in \mathbb{N} \mid E_n^\comp \in \mathcal{R}\}$. Clearly, we have that:

      {\centering $\bigcup\limits_{n \in \mathbb{N}}E_n = \bigcup\limits_{n \in A}E_n \cup \bigcup\limits_{n \in B}E_n = \bigcup\limits_{n \in A}E_n \cup \bigcup\limits_{n \in B}(E_n^\comp)^{\comp}$\\ [2pt]\par}

      Also $\bigcup\limits_{n \in B}(E_n^\comp)^\comp = \left(\bigcap\limits_{n \in B}\hspace{-0.3em}E_n^\comp\right)^{\hspace{-0.3em}\comp}$\hspace{-0.3em}, and by part (a), we know that $E_B \coloneq \bigcap\limits_{n \in B}\hspace{-0.3em}E_n^\comp \in \mathcal{R}$.\retTwo

      Similarly, we know $E_A \coloneq \bigcup\limits_{n \in A}\hspace{-0.3em} E_n \in \mathcal{R}$. So, we've shown that $\bigcup\limits_{n \in \mathbb{N}}E_n = E_A \cup E_B^\comp$\\ [-8pt] where $E_A, E_B \in \mathcal{R}$.\retTwo

      Finally, note that $E_A \cup E_B^\comp = (E_B - E_A)^\comp$. Since $E_B - E_A \in \mathcal{R}$, we know that $(E_B - E_A)^\comp \in \mathcal{A}$. \retTwo
   \end{myIndent}

   \item[(d)] If $\mathcal{R}$ is a $\sigma$-ring, then $\mathcal{A} = \{E \subseteq X \mid E \cap F \in \mathcal{R} \text{ for all } F \in \mathcal{R}\}$ is a $\sigma$-algebra.
   
   \begin{myIndent}\exTwoP
      To start if $E \in \mathcal{A}$, then $E^\comp \in \mathcal{A}$ because for all $F \in \mathcal{R}$ we have that:
      
      {\centering$E^\comp \cap F = F - E = F - (E \cap F) \in \mathcal{R}$.\retTwo\par}

      Also, let $(E_n)_{n \in \mathbb{N}}$ be a countable collection of sets in $\mathcal{A}$. Then for all $F \in \mathcal{R}$, we have that $\left(\bigcup\limits_{n \in \mathbb{N}}E_n\right) \cap F = \bigcup\limits_{n \in \mathbb{N}}(E_n \cap F) \in \mathcal{R}$. So $\mathcal{A}$ is closed under countable union.\newpage
   \end{myIndent}
\end{enumerate}

\hOne Let $\mathcal{E} \subseteq \mathcal{P}(X)$ be a collection of sets. Since the intersection of $\sigma$-algebras is still a $\sigma$-algebra, we define $\mathcal{M}(\mathcal{E})$ to be the smallest $\sigma$-algebra that contains $\mathcal{E}$. In other words, $\mathcal{M}(\mathcal{E})$ is the intersection of all $\sigma$-algebras that contain $\mathcal{E}$.\retTwo

We call $\mathcal{M}(\mathcal{E})$ the $\sigma$-algebra generated by $\mathcal{E}$.


\begin{myIndent}\hTwo
   \blab{Lemma:} if $\mathcal{E} \in \mathcal{M}(\mathcal{F})$, then $\mathcal{M}(\mathcal{E}) \subseteq \mathcal{M}(\mathcal{F})$.
\end{myIndent}

\mySepTwo

Let $(X, \rho)$ be a metric space. We define the \udefine{Borel $\sigma$-algebra} on $X$: $\mathcal{B}_X$, to be the $\sigma$-algebra generated by the collection of all open sets, or equivalently the collection of all closed sets.

\begin{itemize}
   \item A set is $G_\delta$ if it is a countable intersection of open sets.
   \item A set is $F_\sigma$ if it is a countable union of closed sets.
   \item A set is $G_{\delta\sigma}$ if it is a countable union of $G_\delta$ sets.
   \item A set is $F_{\sigma\delta}$ if it is a countable intersection of $F_\sigma$ sets.
   \begin{myTindent}\myComment
      You can hopefully see the pattern. Also the professor isn't sure how much we'll use this $\delta$ and $\sigma$ notation in class.
   \end{myTindent}
\end{itemize}

\exOne\blab{Exercise 1.2:} $\mathcal{B}_\mathbb{R}$ is generated by each of the following:
\begin{itemize}
   \item [(a)] the set of open intervals: $\mathcal{E}_1 = \{(a, b) \mid a < b\}$
   \item [(b)] the set of closed intervals: $\mathcal{E}_2 = \{[a, b] \mid a < b\}$
   \item [(c)] the set of half-open intervals:
   \begin{itemize}
      \item[(i)] $\mathcal{E}_3 = \{(a, b] \mid a < b\}$
      \item[(ii)] $\mathcal{E}_4 = \{[a, b) \mid a < b\}$
   \end{itemize}
   \item [(c)] the set of open rays:
   \begin{itemize}
      \item[(i)] $\mathcal{E}_5 = \{(a, \infty) \mid a \in \mathbb{R}\}$
      \item[(ii)] $\mathcal{E}_6 = \{(-\infty, a) \mid a \in \mathbb{R}\}$
   \end{itemize}
   \item [(d)] the set of closed rays:
   \begin{itemize}
      \item[(i)] $\mathcal{E}_7 = \{[a, \infty) \mid a \in \mathbb{R}\}$
      \item[(ii)] $\mathcal{E}_8 = \{(-\infty, a] \mid a \in \mathbb{R}\}$\retTwo
   \end{itemize}
\end{itemize}

\begin{myIndent}\exTwoP
   Proof:\\
   We trivially have that $\mathcal{M}(\mathcal{E}_1), \mathcal{M}(\mathcal{E}_2), \mathcal{M}(\mathcal{E}_5), \mathcal{M}(\mathcal{E}_6), \mathcal{M}(\mathcal{E}_7), \mathcal{M}(\mathcal{E}_8) \subseteq \mathcal{B}_{\mathbb{R}}$ since each of them contain either only open sets or only closed sets. As for the other inclusions, we must do more work.\newpage

   \begin{itemize}
      \item[(a)] Note that $\mathbb{Q}$ is a countable dense subset of $\mathbb{R}$. Hence, a countable base of $\mathbb{R}$ is the set: $\mathcal{F} = \{(p - q, p + q) \subset \mathbb{R} \mid p, q \in \mathbb{Q} \text{ and } q > 0\}$. In other words, given any open set $E \subseteq \mathbb{R}$, there is a countable subcollection of $\mathcal{F}$ whose union is $E$.
      \begin{myIndent}\exPPP
         To see why, let $x \in E$. Since $E$ is open, there exists $r > 0$ with $B(r, x) \subseteq E$.\\ Next, pick $p \in (x, x + \frac{r}{2}) \cap \mathbb{Q}$, followed by $q \in (p - x, r - p) \cap \mathbb{Q}$. Then $x \in (p - q, p + q) \in \mathcal{F}$ and $(p - q, p + q) \subseteq (x - r, x + r)$.\retTwo
   
         With that, we've now shown that for all $x \in E$, there exists $F \in \mathcal{F}$ such\\ that $x \in F \subseteq E$. If we choose such an $F_x$ for all $x \in E$, we then get\\ that $E = \bigcup\limits_{x \in E}F_x$. So $E$ is the union of a subcollection of $\mathcal{F}$. But since $\mathcal{F}$ is\\ [-9pt]\phantom{aaaaaaaaaaaaaaa} countable, the set $\{F_x \in \mathcal{F} \mid x \in E\}$ is also countable.\retTwo
      \end{myIndent}
   
      Importantly, $\mathcal{F} \subset \mathcal{E}_1$. So $\mathcal{M}(\mathcal{F}) \subseteq \mathcal{M}(\mathcal{E}_1)$. However as shown above, we\\ must have that $\mathcal{M}(\mathcal{F})$ includes all open sets. So by our lemma on the previous page, $\mathcal{B}_{\mathbb{R}} \subseteq \mathcal{M}(\mathcal{F}) \subseteq \mathcal{M}(\mathcal{E}_1)$.\retTwo

      \item[(b)] Given any $E = (a, b) \in \mathcal{E}_1$, we can write that $E = \hspace{-0.3em}\bigcup\limits_{n \in \mathbb{Z}_+}\hspace{-0.3em} [a+\frac{1}{n}, b - \frac{1}{n}]$. Thus,\\ [-10pt] $\mathcal{E}_1 \subseteq \mathcal{M}(\mathcal{E}_2)$, meaning $\mathcal{B}_\mathbb{R} = \mathcal{M}(\mathcal{E}_1) \subseteq \mathcal{M}(\mathcal{E}_2)$.\retTwo
      
      \item[(c)] Remember that for these two, we still need to show that $\mathcal{M}(\mathcal{E}_1), \mathcal{M}(\mathcal{E}_2) \in \mathcal{B}_\mathbb{R}$.
      \begin{itemize}
         \item[(i)] Firstly note that if $F = (a, b] \in \mathcal{E}_3$, then $ = \hspace{-0.3em}\bigcap\limits_{n \in \mathbb{Z}_+}\hspace{-0.3em} (a, b+\frac{1}{n})$. So $\mathcal{E}_3 \subseteq \mathcal{M}(\mathcal{E}_1)$.\\
         On the other hand, if $E = (a, b) \in \mathcal{E}_1$, we have that $E = \hspace{-0.3em}\bigcup\limits_{n \in \mathbb{Z}_+}\hspace{-0.3em} (a, b-\frac{1}{n}]$.\\ [-9pt] So $\mathcal{E}_1 \subseteq \mathcal{M}(\mathcal{E}_3)$.\retTwo
         
         By our lemma on the previous page, we thus have that:
         
         {\centering $\mathcal{B}_\mathbb{R} = \mathcal{M}(\mathcal{E}_1) \subseteq \mathcal{M}(\mathcal{E}_3) \subseteq \mathcal{M}(\mathcal{E}_1) = \mathcal{B}_\mathbb{R}$. \retTwo\par}

         \item[(ii)] Mostly identical reasoning as with $\mathcal{E}_3$ shows that:
         
         {\centering $\mathcal{B}_\mathbb{R} = \mathcal{M}(\mathcal{E}_1) \subseteq \mathcal{M}(\mathcal{E}_4) \subseteq \mathcal{M}(\mathcal{E}_1) = \mathcal{B}_\mathbb{R}$\retTwo\par}
      \end{itemize}

      \item[(d)] \phantom{a}\\ [-10pt]
      \begin{itemize}
         \item[(i)] If $E = (a, b) \in \mathcal{E}_1$, then we know that:
         
         {\centering$E  = (a, \infty) \cap (\hspace{-0.3em}\bigcap\limits_{n \in \mathbb{Z}_+}\hspace{-0.3em} (b - \frac{1}{n}, \infty))^\comp \in \mathcal{M}(\mathcal{E}_5)$.\\\par}

         So $\mathcal{E}_1 \subseteq \mathcal{M}(\mathcal{E}_5)$, meaning $\mathcal{B}_\mathbb{R} = \mathcal{M}(\mathcal{E}_1) \subseteq \mathcal{M}(\mathcal{E}_5)$.\retTwo

         \item[(ii)] Analogous reasoning to that with $\mathcal{E}_5$ shows that $\mathcal{B}_\mathbb{R} = \mathcal{M}(\mathcal{E}_1) \subseteq \mathcal{M}(\mathcal{E}_6)$.\retTwo
      \end{itemize}

      \item[(e)] \phantom{a}\\ [-10pt]
      \begin{itemize}
         \item[(i)] If $E = (a, \infty) \in \mathcal{E}_6$, then we have that $E = \hspace{-0.3em}\bigcup\limits_{n \in \mathbb{Z}_+}\hspace{-0.3em} [a + \frac{1}{n}, \infty)$. So $\mathcal{E}_5 \subseteq \mathcal{M}(\mathcal{E}_7)$,\\ [-10pt] meaning that $\mathcal{B}_\mathbb{R} = \mathcal{M}(\mathcal{E}_5) \subseteq \mathcal{M}(\mathcal{E}_7)$.\retTwo

         \item[(ii)] Analogous reasoning as with $\mathcal{E}_7$ shows that $\mathcal{B}_\mathbb{R} = \mathcal{M}(\mathcal{E}_6) \subseteq \mathcal{M}(\mathcal{E}_8)$.\newpage
      \end{itemize}
   \end{itemize}
\end{myIndent}

\blab{Exercise 1.3:} Let $\mathcal{M}$ be an infinite $\sigma$-algebra on $X$.
\begin{itemize}
   \item[(a)] $\mathcal{M}$ contains an infinite sequence of disjoint sets.
   
   \begin{myIndent}\exTwoP
      By the Hausdorff maximum principle, we know there is a subcollection $\mathcal{S}$\\ of $\mathcal{M}$ which is simply ordered by proper subset and is not contained in any other collection of $\mathcal{M}$ which is simply ordered by proper subset.\retTwo
      
      We claim $\mathcal{S}$ can't be finite. For suppose $\mathcal{S} = \{E_1, \ldots, E_n\}$ is a sequence of sets in $\mathcal{M}$ simply ordered by proper subset which are indexed such that $E_i \subset E_{i + 1}$ for all $i \in \{1, \ldots, n-1\}$.
      \begin{myDindent}\exPPP
         (Note: if $\mathcal{S}$ is maximal, then we must have $E_1 = \emptyset$ and $E_n = X$.)\retTwo
      \end{myDindent}
      
      We can partition $\mathcal{M}$ into collections $\mathcal{M}_1, \ldots \mathcal{M}_n$ such that $A \in \mathcal{M}_i$ iff $i$ is the least integer for which $A \subseteq E_i$. Importantly, all sets in $\mathcal{M}$ will fall into a\\ partition because all sets from $\mathcal{M}$ are contained in $E_n$. Also note that while there are infinitely many $A \in \mathcal{M}$, there are only $n$ many partitions. So, there must be a least integer $k$ such that $\mathcal{M}_k$ contains infinitely many $A \in \mathcal{M}$.\\ [-14pt]
      
      \begin{myTindent}\begin{myTindent}\exPPP
         And since $\mathcal{M}_1 = \{\emptyset\}$, we know $k \geq 2$.\retTwo
      \end{myTindent}\end{myTindent}

      The fact that $\mathcal{M}_i$ is finite for all $i < k$ means that there are only finitely many sets from $\mathcal{M}$ contained in $E_{k-1}$. Thus, we can pick a set $B \in \mathcal{A}_k$ such that $B \neq (E_k - E_{k-1}) \cup A$ for any $A \in \mathcal{M}$ that is a subset of $E_{k-1}$.\retTwo

      Note that since $E_{k-1} \cap B$ is a set in $\mathcal{M}$, we must have that $(E_k - E_{k-1}) \not\subseteq B$ or else $B$ would be the union of $(E_k - E_{k-1})$ and a set from $\mathcal{M}$. Thus, we know $E_k$ contains points that neither $B$ nor $E_{k-1}$ have. At the same time, we know $B$ has points that $E_{k-1}$ doesn't have. It follows that: $E_{k-1} \subset E_{k-1} \cup B \subset E_k$. \retTwo

      Via transitivity, $E_{k-1} \cup B$ is comparable via proper subset with $E_i$ for all\\ $i \in \{1, \ldots, n\}$. Hence, we've shown that $\mathcal{S} \cup \{E_{k-1} \cup B\}$ is a sequence of sets in $\mathcal{M}$ simply ordered by proper subset. But this contradicts that $\mathcal{S}$ is maximal.\retTwo

      Now that we know $\mathcal{S}$ is infinite, let $(E_n)_{n \in \mathbb{Z}_+}$ be a sequence of sets in $\mathcal{S}$\\ satisfying that $E_n \subset E_{n + 1}$. Then we have that $(E_{n+1} - E_n)_{n \in \mathbb{Z}_+}$ is an infinite\\ sequence of disjoint sets in $\mathcal{M}$.\retTwo
   \end{myIndent}

   \item[(b)] Show that $\card(\mathcal{M}) \geq \mathfrak{c}$.
   
   \begin{myIndent}\exTwoP
      Let $(E_n)_{n \in \mathbb{N}}$ be a sequence of disjoint sets in $\mathcal{M}$. Then if we define the map\\ [2pt] $f: [0, 1]^\mathbb{N} \longrightarrow \mathcal{M}$ such that $(a_0, a_1, a_2, \ldots)$ is mapped to the union of all $E_n$\\ [2pt] such that $a_n = 1$, we have that $f$ is an injection.\retTwo

      Hence, $\card(\mathcal{M}) \geq \card([0, 1]^{\mathbb{N}})$. And since there is a trivial bijection\\ from $[0, 1]^{\mathbb{N}}$ and $\mathcal{P}(\mathbb{N})$, plus the fact that we proved early on in the class\\ that $\card(\mathcal{P}(\mathbb{N})) = \card(\mathbb{R})$, we thus know that $\card(\mathcal{M}) \geq \mathfrak{c}$.\newpage
   \end{myIndent}
\end{itemize}

\blab{Exercise 1.4}: An algebra $\mathcal{A}$ is a $\sigma$-algebra if and only if $\mathcal{A}$ is closed under countable\\ increasing unions (meaning $E_1 \subseteq E_2 \subseteq \ldots$).

\begin{myIndent}\exTwoP
   The rightward implication is true since $\mathcal{A}$ being a $\sigma$-algebra means that $\mathcal{A}$ is\\ closed under all countable unions. As for showing the leftward implication,\\ suppose $\{A_n\}_{n \in \mathbb{Z}_+}$ is a countable collection of sets in $\mathcal{A}$. Then for all $n \in \mathbb{Z}_+$,\\ define $E_n = A_1 \cup \ldots \cup A_n$.\retTwo
   
   Since each $E_n$ are finite unions of sets in $\mathcal{A}$, we know that each $E_n$ is in $\mathcal{A}$. Also,\\ we clearly have that $E_1 \subseteq E_2 \subseteq E_3 \subseteq \ldots$\phantom{a} In order to make the sets strictly\\ increasing, let $S = \{1\} \cup \{k \in \mathbb{Z} \mid k > 1 \text{ and } E_{k} - E_{k-1} \neq \emptyset \}$. Then for\\ any $n, m \in S$, we know that $n < m \Longrightarrow E_n \subset E_m$.\retTwo

   Finally, $\hspace{-0.3em}\bigcup\limits_{n \in \mathbb{Z}_+}\hspace{-0.3em} A_n = \hspace{-0.3em}\bigcup\limits_{n \in \mathbb{Z}_+}\hspace{-0.3em} E_n = \bigcup\limits_{n \in S}E_n$.\retTwo Importantly, $S$ is either finite or countably infinite, and $S$ consists of strictly\\ increasing sets. So by the right hypothesis, we know $\bigcup\limits_{n \in S}E_n \in \mathcal{A}$. Hence,\\ [-9pt] the union over $\{A_n\}_{n \in \mathbb{Z}_+}$ is in $\mathcal{A}$. \retTwo
\end{myIndent}

\hOne
\mySepTwo

Let $\{X_\alpha\}_{\alpha \in A}$ be a collection of nonempty sets, and define $X = \prod\limits_{\alpha \in A}X_\alpha$.\\ If $\mathcal{M}_\alpha$ is a $\sigma$-algebra in $X_\alpha$ for all $\alpha \in A$, then we define the \udefine{product $\sigma$-algebra}\\ on $X$ to be: $\bigotimes\limits_{\alpha \in A}\mathcal{M}_\alpha = \mathcal{M}(\{\pi_\alpha^{-1}(E_\alpha) \mid E_\alpha \in \mathcal{M}_\alpha \text{ and } \alpha \in A\})$.\retTwo


\begin{myIndent}\myComment
   To get a better geometric intuition for this definition, consider if $A = \{1, 2\}$. Then:\\ [-8pt]

   {\centering 
   \begin{tabular}{l}
      $\bigotimes\limits_{\alpha \in A}\mathcal{M}_\alpha = \{\pi_1^{-1}(E_1) \mid E_1 \in \mathcal{M}_1\} \cup \{\pi_2^{-1}(E_2) \mid E_2 \in \mathcal{M}_2\}$ \\ [-4pt]$\hphantom{\bigotimes\limits_{\alpha \in A}\mathcal{M}_\alpha} = \{E_1 \times X_2 \mid E_1 \in \mathcal{M}_1\} \cup \{X_1 \times E_2 \mid E_2 \in \mathcal{M}_2\}$
   \end{tabular} \retTwo\par}

   Also, the motivation for this definition is that $\bigotimes\limits_{\alpha \in A}\mathcal{M}_\alpha$ is the smallest $\sigma$-algebra\\ [-2pt] where $\pi_\alpha$ is "measurable" for all $\alpha$. We'll learn what that means shortly...\retTwo

   \hTwo
   \blab{Proposition:}
   \begin{itemize}
      \item[(i)] $A$ is countable implies $\bigotimes\limits_{\alpha \in A}\mathcal{M}_\alpha = \mathcal{M}(\{\prod\limits_{\alpha \in A}E_\alpha \mid \forall \alpha \in A, \myHS E_\alpha \in \mathcal{M}_\alpha\})$
      
      \begin{myIndent}\hThree
         Proof:\\
         If $E_\alpha \in \mathcal{M}_\alpha$, then $\pi_\alpha^{-1}(E_\alpha) = \prod\limits_{\beta \in A}E_\beta$ where $E_\beta = X_\beta$ if $\beta \neq \alpha$ (and\\ [-8pt] $E_\beta = E_\alpha$ if $\beta = \alpha$).\retTwo
         
         So $\pi_\alpha^{-1}(E_\alpha) \in \mathcal{M}(\{\prod\limits_{\alpha \in A}E_\alpha \mid \forall \alpha \in A,\myHS E_\alpha \in \mathcal{M}_\alpha\})$

         On the other hand, $\prod\limits_{\alpha \in A}E_\alpha = \bigcap\limits_{\alpha \in A}\pi_\alpha^{-1}(E_\alpha)$.\retTwo
         
         Since $A$ is countable, we thus know that if $E_\alpha \in \mathcal{M}_\alpha$ for all $\alpha \in A$, then\\ $\prod\limits_{\alpha \in A}E_\alpha  \in \bigotimes\limits_{\alpha \in A}\mathcal{M}_\alpha$.\newpage
      \end{myIndent}

      \item[(ii)] Suppose $\mathcal{M}_\alpha = \mathcal{M}(\mathcal{E}_\alpha)$ for all $\alpha \in A$. Then $\bigotimes\limits_{\alpha \in A}\mathcal{M}_\alpha$ is generated by\\ [-8pt] $\mathcal{F} = \{\pi^{-1}_\alpha(E_\alpha) \mid E_\alpha \in \mathcal{E}_\alpha \text{ and } \alpha \in A\}$.
      
      \begin{myIndent}\hThree
         Proof:\\
         Since $\mathcal{F} \subseteq \{\pi_\alpha^{-1}(E_\alpha) \mid E_\alpha \in \mathcal{M}_\alpha \text{ and } \alpha \in A\}$, we trivially have that\\ $\mathcal{M}(\mathcal{F}) \subseteq \bigotimes\limits_{\alpha \in A} \mathcal{M}_\alpha$.\\ [-2pt]

         As for showing the other inclusion, define for each $\alpha \in A$:

         {\centering$\mathcal{F}_\alpha = \{E \subseteq X_\alpha \mid \pi_\alpha^{-1}(E) \in \mathcal{M}(\mathcal{F})\}$.\retTwo\par}

         Note that $\mathcal{F}_\alpha$ is a $\sigma$-algebra on $X_\alpha$ that contains $\mathcal{E}_\alpha$.
         \begin{myIndent}\hFour
            This is because for any $F \in \mathcal{F}_\alpha$ and $(E_n)_{n \in \mathbb{N}} \subseteq \mathcal{F}_\alpha$, we know that:
            \begin{myIndent}
               \begin{itemize}
                  \item[\bullet] $\left(\pi_{\alpha}^{-1}(F)\right)^\comp = \pi_{\alpha}^{-1}(F^\comp)$
                  \item[\bullet] $\bigcup\limits_{n \in \mathbb{N}}\pi_{\alpha}^{-1}(E_n) = \left(\pi_{\alpha}^{-1}(\bigcup\limits_{n \in \mathbb{N}}E_n)\right)$\retTwo
               \end{itemize}
            \end{myIndent}

            Also, for any $E \subseteq X_\alpha$, $E \in \mathcal{E}_\alpha \Longrightarrow \pi_\alpha^{-1}(E) \in \mathcal{M}(\mathcal{F})$.\retTwo
         \end{myIndent}

         By definition, we thus know that $\mathcal{M}_\alpha \subseteq \mathcal{F}_\alpha$. So for all $\alpha \in A$ and $E_\alpha \in \mathcal{M}_\alpha$, we know that $E_\alpha \in \mathcal{F}_\alpha$, which means that $\pi_\alpha^{-1}(E_\alpha) \in \mathcal{M}(\mathcal{F})$. So\\ $\bigotimes\limits_{\alpha \in A} \mathcal{M}_\alpha \subseteq \mathcal{M}(\mathcal{F})$.\retTwo
      \end{myIndent}

      (iii) We can also combine the first two parts of this proposition. If $A$ is countable and $\mathcal{M}_\alpha = \mathcal{M}(\mathcal{E}_\alpha)$ for all $\alpha \in A$, then $\bigotimes\limits_{\alpha \in A}\mathcal{M}_\alpha$ is generated by:
      
      {\centering $\{\prod\limits_{\alpha \in A}E_\alpha \mid \forall \alpha \in A, \myHS E_\alpha \in \mathcal{E}_\alpha\}$\retTwo\par}
   \end{itemize}
\end{myIndent}

\mHeader{Lecture 3 Notes: 10/3/2024}

\begin{myIndent}\hTwo
   \blab{Proposition}: Let $X_1, \ldots, X_n$ be metric spaces, and define $X = \prod\limits_{i=1}^n X_i$ to be the\\ [-8pt] metric space equipped with the product metric.
   
   \begin{myIndent}\myComment
      The product metric defines the distance between any $\bm{x}, \bm{y} \in \prod\limits_{i = 1}^n$ to be the max\\ distance between a coordinate of $\bm{x}$ and the corresponding coordinate in $\bm{y}$.\retTwo
   \end{myIndent}

   \begin{itemize}
      \item $\bigotimes\limits_{i = 1}^n \mathcal{B}_{X_i} \subseteq \mathcal{B}_X$.\\ [-10pt]
      \begin{myIndent}\hThree
         Proof:\\
         By the previous proposition: $\bigotimes\limits_{i = 1}^n \mathcal{B}_{X_i}$ is generated by the collection:
         
         {\centering $\{\pi_{i}^{-1}(U_i) \mid i \in \{1, \ldots, n\} \text{ and } U_i \subseteq X_i \text{ is open}\}$.\retTwo\par}

         Also, by the definition of a product topology, we know that each $\pi_{i}^{-1}(U_i)$ is\\ open in $X$. So by the lemma on page 9, we know that $\bigotimes\limits_{i = 1}^n \mathcal{B}_{X_i} \subseteq \mathcal{B}_X$.\newpage 
      \end{myIndent}

      \item If each $X_i$ is separable, then $\bigotimes\limits_{i = 1}^n \mathcal{B}_{X_i} = \mathcal{B}_X$.\\ [-10pt]
      
      \begin{myIndent}\hThree
         Proof:\\
         Let $C_i \subseteq X_i$ be countable with $\overline{C_i} = X_i$ for all $i \in \{1, \ldots, n\}$. Then define $\mathcal{E}_i = \{B(p, x) \mid x \in C_i \text{ and } p \in \mathbb{Q}_+\}$ for each $i$. Since $\mathcal{E}_i$ is countable and all open sets in $X_i$ are the union of a subcollection of $\mathcal{E}_i$, we know that any open set in $X_i$ is also in $\mathcal{M}(\mathcal{E}_i)$. So, $\mathcal{B}_{X_i} \subseteq \mathcal{M}(\mathcal{E}_i)$.  And since $\mathcal{E}_i$ contains only open sets of $X_i$, the reverse inclusion holds too.\retTwo

         Also, $C = \prod\limits_{i=1}^n C_i$ is a countable dense subset of $X$.\retTwo Defining $\mathcal{E} = \{B(p, \bm{x}) \mid \bm{x} \in C \text{ and } p \in \mathbb{Q}_+\}$, we have that $\mathcal{E}$ is countable and any open set in $X$ is also in $\mathcal{M}(\mathcal{E})$. So, $\mathcal{B}_{X} \subseteq \mathcal{M}(\mathcal{E})$. And like before since $\mathcal{E}$ contains only open sets of $X$, the reverse inclusion holds too. \retTwo

         But now note that given, $B(p, (x_1, \ldots, x_n)) \in \mathcal{E}$, we know that\\ $B(p, (x_1, \ldots, x_n)) = \prod\limits_{i = 1}^n B(p, x_i)$ where $(p, x_i) \in \mathcal{E}_i$ for all $i$.\retTwo
         
         So applying part 3 of the previous proposition and the lemma on page 9:

         {\centering\fontsize{12}{14}\selectfont $\mathcal{B}_X = \mathcal{M}(\mathcal{E}) \subseteq \mathcal{M}\left(\prod\limits_{i = 1}^n E_i \mid E_i \in \mathcal{E}_i \text{ for all } i\right) = \bigotimes\limits_{i=1}^n \mathcal{M}(\mathcal{E}_i) = \bigotimes\limits_{i = 1}^n \mathcal{B}_{X_i}$ \retTwo\par}
      \end{myIndent}
   \end{itemize}

   \blab{Corollary:} $\mathcal{B}_{\mathbb{R}^n} = \bigotimes\limits_{i = 1}^n \mathcal{B}_\mathbb{R}$.

   \begin{myIndent}\hThree
      This is because the product metric $\rho_1$ of $\prod\limits_{i=1}^n \mathbb{R}$ is \udefine{equivalent} to the standard metric\\ [-8pt] $\rho_2$ of $\mathbb{R}^n$, meaning that: 
      
      {\centering $\exists C, C^\prime > 0$ such that $C\rho_1 \leq \rho_2 \leq C^\prime\rho_1$.\retTwo\par}

      
      \begin{myIndent}\hFour
         In the specific case of this corollary, set $C = \sqrt{\sfrac{1}{n}}\hspace{0.2em}$ and $C^\prime = 1$.\retTwo
      \end{myIndent}

      The fact relevant here is that given the metrics $\rho_1, \rho_2$ on a set $X$, if $\rho_1$\\ is equivalent to $\rho_2$, then $(X, \rho_1)$ and $(X, \rho_2)$ have the same open sets (this\\ is really trivial to prove).\retTwo
   \end{myIndent}
\end{myIndent}

\mySepTwo

An \udefine{elementary family} is a collection $\mathcal{E}$ of subsets of a set $X$ such that:

\begin{enumerate}
   \item $\emptyset \in \mathcal{E}$
   \item If $E, F \in \mathcal{E}$, then $E \cap F \in \mathcal{E}$.
   \item If $E \in \mathcal{E}$, then $E^\comp$ is a finite disjoint union of members of $\mathcal{E}$.
\end{enumerate}

\newpage

\begin{myIndent}\hTwo
   If $\mathcal{E}$ is an elementary collection, then $\mathcal{A}$ equal to the collection of finite disjoint\\ unions of $\mathcal{E}$ is an algebra.\retTwo

   \begin{myIndent}\hThree
      Proof:\\
      Firstly, given any $A, B \in \mathcal{E}$, we have that $A \cup B = (A - B) \cup B$. Also,\\ [-2pt] by property 3 of elementary families, $(A - B) = (A \cap B^\comp) = (A \cap \bigcup\limits_{i=1}^k C_i)$\\ where each $C_i \in \mathcal{E}$ and disjoint. By property 2 of elementary families, we thus\\ [4pt] know $A \cap C_i \in \mathcal{E}$ for all $i$. So $(A - B)$ is a finite union of disjoint sets in $\mathcal{E}$. In turn,\\ [4pt] so is $(A - B) \cup B$. Hence, $A \cup B \in \mathcal{A}$.\retTwo

      By induction, we get that for any $A_1, \ldots, A_n \in \mathcal{E}$,\myHS $A_1 \cup \ldots \cup A_n$ is a finite\\ union of disjoint sets in $\mathcal{E}$. So $\mathcal{A}$ actually equals the set of all finite unions of $\mathcal{E}$. It\\ follows that $\mathcal{A}$ is closed under finite unions.\retTwo

      I really don't want to write down the proof that $\mathcal{A}$ is closed under complements. It's what you would expect but just heavy on notation.
   \end{myIndent}
\end{myIndent}

\exOne\mySepTwo 

\blab{Exercise 1.5:} If $\mathcal{M}$ is the $\sigma$-algebra generated by $\mathcal{E}$, then $\mathcal{M}$ is the union of the $\sigma$-algebras generated by $\mathcal{F}$ as $\mathcal{F}$ ranges over all countable subsets of $\mathcal{E}$.

\begin{myIndent}\exTwoP
   For the sake of convenience, I will write the union of $\sigma$-algebras generated by\\ countable subsets of $\mathcal{E}$ as: $\bigcup_\mathcal{F} \mathcal{M}(\mathcal{F})$.\retTwo

   To start, since each $\mathcal{M}(\mathcal{F}) \subseteq \mathcal{M}(\mathcal{E})$, we trivially know $\bigcup_\mathcal{F}\mathcal{M}(\mathcal{F}) \subseteq \mathcal{M}(\mathcal{E}) = \mathcal{M}$.\\ On the other hand, $\mathcal{E} \subseteq \bigcup_\mathcal{F}\mathcal{M}(\mathcal{F})$ since each countable $\mathcal{F} \subseteq \mathcal{E}$ is contained in\\ $\mathcal{M}(\mathcal{F}) \subseteq \bigcup_\mathcal{F}\mathcal{M}(\mathcal{F})$. So, if we can show that $\bigcup_\mathcal{F}\mathcal{M}(\mathcal{F})$ is a $\sigma$-algebra, then we will know that: $\mathcal{M} = \mathcal{M}(\mathcal{E}) \subseteq \bigcup_\mathcal{F}\mathcal{M}(\mathcal{F})$.\retTwo

   Fortunately, it's trivial to show that $\bigcup_\mathcal{F}\mathcal{M}(\mathcal{F})$ is closed under complements.\\ Given any $E \in \bigcup_\mathcal{F}\mathcal{M}(\mathcal{F})$, we know there exists $\mathcal{M}(\mathcal{F})$ with $E \in \mathcal{M}(\mathcal{F})$.\\ Then $E^\comp \in \mathcal{M}(\mathcal{F}) \subseteq \bigcup_\mathcal{F}\mathcal{M}(\mathcal{F})$. \retTwo
   
   Meanwhile, the proof that $\bigcup_\mathcal{F}\mathcal{M}(\mathcal{F})$ is closed under countable unions is\\ more involved:\\ [-9pt]
   \begin{myIndent}\exPPP
      Suppose $\{E_n\}_{n \in \mathbb{N}}$ is a countable collection of sets in $\bigcup_\mathcal{F}\mathcal{M}(\mathcal{F})$. Then for each\\ $n \in \mathbb{N}$, there exists $\mathcal{F}_n$ such that $E_n \in \mathcal{M}(\mathcal{F}_n)$. Importantly, $\bigcup\limits_{n \in \mathbb{N}}\mathcal{F}_n$ is still\\ [-7pt] countable. So, setting $\mathcal{F}^\prime = \bigcup\limits_{n \in \mathbb{N}}\mathcal{F}_n$, we have that:

      {\centering $\mathcal{M}(\mathcal{F}^\prime) \subseteq \bigcup_\mathcal{F}\mathcal{M}(\mathcal{F})$ \retTwo\par}

      Since $\mathcal{F}_n \subseteq \mathcal{F}^\prime$ for all $n$, we know that $\mathcal{M}(\mathcal{F}_n) \subseteq \mathcal{M}(\mathcal{F}^\prime)$ for all $n$. So, $\{E_n\}_{n \in \mathbb{N}}$\\ is contained in $\mathcal{M}(\mathcal{F}^\prime)$. It follows that $\bigcup\limits_{n \in \mathbb{N}}E_n \in \mathcal{M}(\mathcal{F}^\prime) \subseteq \bigcup_\mathcal{F}\mathcal{M}(\mathcal{F})$.\retTwo
   \end{myIndent}
\end{myIndent}

\mySepTwo
   
\newpage\hOne

Let $X \neq \emptyset$ and $\mathcal{M}$ be a $\sigma$-algebra on $X$. A \udefine{measure} $\mu: \mathcal{M} \longrightarrow [0, \infty]$ is a function satisfying that:
\begin{itemize}
   \item $\mu(\emptyset) = 0$
   \item $\mu(\bigcup\limits_{j=1}^\infty E_j) = \sum\limits_{j=1}^\infty \mu(E_j)$ if $E_j \in \mathcal{M}$ for all $j$ and $E_j \cap E_i = \emptyset$ for all $i \neq j$\retTwo
\end{itemize}

$(X, \mathcal{M})$ is called a \udefine{measurable space} and $(X, \mathcal{M}, \mu)$ is called a \udefine{measure space}.\retTwo

Let $(X, \mathcal{M}, \mu)$ be a measure space.
\begin{itemize}
   \item $\mu$ is called \udefine{finite} if $\mu(X) < \infty$.
   
   \begin{myIndent}\teachComment
      It follows if $\mu$ is finite that $\mu(E) < \infty$ for all $E \in \mathcal{M}$ since $E \subseteq X$.\\
      In probability theory, most measure spaces are finite.\retTwo
   \end{myIndent}

   \item $\mu$ is called \udefine{$\sigma$-finite} if $X = \bigcup\limits_{j=1}^\infty E_j$, such that $E_j \in \mathcal{M}$ and $\mu(E_j) < \infty$ for all $j$.\retTwo
   
   \item $\mu$ is called \udefine{semifinite} if $\forall E \in \mathcal{M}$ with $\mu(E) = \infty$, there exists $F \subset E$ such that $F \in \mathcal{M}$, and $ 0 < \mu(F) < \infty$.\retTwo
\end{itemize}


\begin{myIndent}\exOne
   Example: Let $X \neq \emptyset$ and $\mathcal{M} = \mathcal{P}(X)$. Then given a function $\rho: X \longrightarrow [0, \infty]$,\\ $\mu(E) = \sum\limits_{x \in E}\rho(x)$ is a measure.
   
   \begin{itemize}
      \item $\mu$ is semifinite if and only if $\rho(x) < \infty$ for all $x \in X$.
      \item $\mu$ is $\sigma$-finite if and only if it is semifinite and $\{x \in X \mid \rho(x) > 0\}$ is countable.
   \end{itemize}

   If $\rho(x) = 1$ for all $x$, then $\mu$ is called the \udefine{counting measure}.\\
   If $\rho(x) = \left\{
   \begin{matrix}
      1 & \text{ if } x = x_0 \\
      0 & \text{ if } x \neq x_0,
   \end{matrix}\right.$ then $\mu$ is called the \udefine{Dirac measure} at $x_0$: $\delta_{x_0}$.\retTwo

   \hTwo 
   \blab{Theorem:} Let $(X, \mathcal{M}, \mu)$ be a measure space. Then:
   \begin{enumerate}
      \item If $E, F \in \mathcal{M}$ with $E \subseteq F$, then $\mu(E) \leq \mu(F)$.
      \item If $(E_j)_{j\in\mathbb{N}} \subseteq \mathcal{M}$, then $\mu(\bigcup\limits_{j=1}^\infty E_j) \leq \sum\limits_{j = 1}^\infty \mu(E_j)$.\\ [-10pt]
      \item If $(E_j)_{j \in \mathbb{N}} \subseteq \mathcal{M}$ with $E_j \subseteq E_{j+1}$ for all $j \in \mathbb{N}$, then $\mu(\bigcup\limits_{j=1}^\infty E_j) = \lim\limits_{j \rightarrow \infty}\mu(E_j)$.
      \item If $(E_j)_{j \in \mathbb{N}} \subseteq \mathcal{M}$ and $\mu(E_1) < \infty$ and $E_{j+1} \subseteq E_j$ for all\\ [-6pt] \phantom{aaaaaaaaaaaaaaaaaaaaaaaaaaaaaaaaaa} $j \in \mathbb{N}$, then $\mu(\bigcap\limits_{j=1}^\infty E_j) = \lim\limits_{j\rightarrow \infty}\mu(E_j)$.
      
      \begin{myIndent}\hThree
         Proofs:\\
         (1) Suppose $E, F \in \mathcal{M}$ with $E \subseteq F$. Then $F = (F - E) \cup E$ is a disjoint union of sets in $\mathcal{M}$, meaning $\mu(F) = \mu(F - E) + \mu(E) \geq \mu(E)$.\newpage

         (2) Set $F_1 = E_1$ and $F_m = E_m - \bigcup\limits_{i=1}^{m-1}E_i$ for all $m > 1$. Then $(F_i)_{i \in \mathbb{N}}$ is\\ [-6pt] pairwise disjoint and $\bigcup\limits_{i=1}^\infty F_i = \bigcup\limits_{i=1}^\infty E_i$. So $\mu(\bigcup\limits_{i=1}^\infty E_i) = \sum\limits_{i=1}^\infty\mu(F_i)$. On the other hand, $F_i \subseteq E_i$ for all $i$. So $\sum\limits_{i=1}^\infty\mu(F_i) \leq \sum\limits_{i=1}^\infty\mu(E_i)$\retTwo\retTwo

         (3) Setting $E_0 = \emptyset$, we have that $\mu(\bigcup\limits_{i=1}^\infty E_i) = \sum\limits_{i=1}^\infty\mu(E_i - E_{i-1})$. Also,\\ [-6pt] $\mu(E_n) = \sum\limits_{i=1}^n\mu(E_i - E_{i-1})$. So:\\ [-8pt]
         
         {\centering$\lim\limits_{n \rightarrow \infty}{\mu(E_n)} = \lim\limits_{n \rightarrow \infty}{\sum\limits_{i=1}^n\mu(E_i - E_{i-1})} = \sum\limits_{i=1}^\infty \mu(E_i - E_{i-1}) = \mu(\bigcup\limits_{i=1}^\infty E_i)$.\retTwo\retTwo\par}

         (4) Let $F_j = E_1 - E_j$ for all $j \in \mathbb{N}$. Then for all $j \in \mathbb{N}$,\myHS $F_j \subseteq F_{j+1}$,\\ $\mu(E_1) = \mu(F_j) + \mu(E_j)$, and $\bigcup\limits_{j=1}^\infty F_j = E_1 - \bigcap\limits_{j = 1}^\infty E_j$. We can thus\\ conclude that:

         {\centering\begin{tabular}{l}
            $\mu(E_1) = \mu(\bigcap\limits_{j=1}^\infty E_j) + \mu(\bigcup\limits_{j=1}^\infty F_j)$\\ [6pt]
            $\phantom{\mu(E_1)} = \mu(\bigcap\limits_{j=1}^\infty E_j) + \lim\limits_{j \rightarrow \infty}(F_j) = \mu(\bigcap\limits_{j=1}^\infty E_j) + \lim\limits_{j \rightarrow \infty}(\mu(E_1) - \mu(E_j))$
         \end{tabular} \retTwo\par}

         Since $\mu(E_1) < \infty$, we can subtract it out of the expression to get:\\ [4pt] $\mu(\bigcap\limits_{j=1}^\infty E_j) - \lim\limits_{j\rightarrow\infty}(\mu(E_j)) = 0$. Also, we know $\mu(\bigcap\limits_{j=1}^\infty E_j) < \infty$ since\\ [0pt] it's a subset of $E_j$. So, we can rearrange to get: $\mu(\bigcap\limits_{j=1}^\infty E_j) = \lim\limits_{j\rightarrow\infty}(\mu(E_j))$.
      \end{myIndent}
   \end{enumerate}
\end{myIndent}

\exOne\mySepTwo

\blab{Exercise 1.9:} If $(X, \mathcal{M}, \mu)$ is a measure space and $E, F \in \mathcal{M}$, then we have that\\ $\mu(E) + \mu(F) = \mu(E \cup F) + \mu(E \cap F)$.

\begin{myIndent}\exTwoP
   We know $\mu(E) = \mu(E - f) + \mu(E \cap F)$ and $\mu(F) = \mu(F - E) + \mu(F \cap E)$.\\ Adding those equations together we get that:

   {\centering
   \begin{tabular}{l}
       $\mu(E) + \mu(F) = (\mu(E - F) + \mu(E \cap F) + \mu(F - E)) + \mu(E \cap F)$\\ $\phantom{\mu(E) + \mu(F) = (\mu(E - F) + \mu(E \cap F)) + a} = \mu(E \cup F) + \mu(E \cap F)$.
   \end{tabular}\retTwo\par}
\end{myIndent}

\blab{Exercise 1.14:} If $\mu$ is a semifinite measure and $\mu(E) = \infty$, then for any $C > 0$ there exists $F \subset E$ in $\mathcal{M}$ with $C < \mu(F) < \infty$.


\begin{myIndent}\exTwoP
   Let $S$ be the set of $C > 0$ for which there exists $F \subset E$ in $\mathcal{M}$ with $C < \mu(F) < \infty$. By the definition of semifiniteness, we know $S$ isn't empty. Meanwhile, if for some $C$ we had that there didn't exist a set $F \subset E$ in $\mathcal{M}$ with $C < \mu(F) < \infty$, then we'd know that $S$ is bounded above. Hence, we'd know there exists $\alpha = \sup(S)$.\newpage

   Now firstly, for all $n \in \mathbb{N}$, choose $G_n \subset E$ in $\mathcal{M}$ such that $\alpha - \frac{1}{n} < \mu(G_n) < \infty$. After that, define $F_n = \bigcup\limits_{i=1}^n G_i$ for all $n \in \mathbb{N}$. Since $\mathcal{M}$ is closed under finite unions,\\ [1pt] we know each $F_n$ is in  $\mathcal{M}$. So then observe:
   \begin{enumerate}
      \item $F_n \subseteq F_{n + 1}$ for all $n \in \mathbb{N}$
      \item For each $n \in \mathbb{N}$,\myHS $\alpha - \frac{1}{n} < \mu(F_n) \leq \alpha$
      \begin{myIndent}\exPPP
         This is because for each $n \in \mathbb{N}$,\myHS  $\mu(F_n) < \sum\limits_{i = 1}^n \mu(G_i)$ which is a finite sum of\\ [-2pt] finite quantities. So $\mu(F_n) < \infty$. At the same time, $F_n \subset E$ since each $G_i$ is\\ [5pt] a subset of $E$ (we know it is a proper subset because it has a different measure\\ [5pt] than $E$). So, if $\mu(F_n) > \alpha$, then $\frac{1}{2}(\mu(F_n) + \alpha)$ would be an element of\\ [5pt] $S$ greater than $\alpha$, thus contradicting that $\alpha = \sup(S)$. As for the other\\ [5pt] inequality, note that $G_n \subseteq F_n$. Thus $\mu(F_n) \geq \mu(G_n) > \alpha - \frac{1}{n}$.\retTwo
      \end{myIndent}

      Now $\bigcup\limits_{n = 1}^\infty F_n \in \mathcal{M}$ due to $\mathcal{M}$ being closed under countable sums. Also, by\\ [-6pt] the two observations above, we know $\mu(\bigcup\limits_{n = 1}^\infty F_n) = \lim\limits_{n \rightarrow \infty}\mu(F_n) = \alpha$. And finally,\\ [-6pt] note that $\bigcup\limits_{n = 1}^\infty F_n$ is a proper subset of $E$ (we know this because each $F_n \subset E$\\ [-6pt] and $\bigcup\limits_{n = 1}^\infty F_n$ can't equal $E$ since their measures are different).\retTwo

      So, we have now proven the existence of a set $F \in \mathcal{M}$ such that $F \subset E$ and\\ $\mu(F) = \alpha$. But now note that $\mu(E - F)$ must be infinite since:
      
      {\centering $\mu(E - F) + \alpha = \mu(E - F) + \mu(F) = \mu(E) = \infty$.\retTwo\par}

      Because $\mu$ is semifinite, there exists $F^\prime \subset E - F$ in $\mathcal{M}$ with $0 < \mu(F^\prime) < \infty$.\\ But because $F$ and $F^\prime$ are disjoint subsets of $E$ in $\mathcal{M}$, we know $F \cup F^\prime \in \mathcal{M}$ and $\mu(F \cup F^\prime) = \mu(F) + \mu(F^\prime) > \alpha$. Plus $F \cup F^\prime$ is a proper subset of $E$. (It can't equal $E$ because it's measure isn't equal to $E$. But, both $F$ and $F^\prime$ individually are subsets of $E$.)\retTwo

      Hence, we have that $\frac{1}{2}(\alpha, \mu(F) + \mu(F)^\prime)$ is an element of $S$ greater than $\alpha$, thus contradicting that $\alpha$ was the supremum of $S$. We conclude therefore that $\alpha$ does not exist, meaning $S$ is unbounded. 
   \end{enumerate}
\end{myIndent}

\mySepTwo

\hOne Given a measure space $(X, \mathcal{M}, \mu)$, a set $E \in \mathcal{M}$ satisfying that $\mu(E) = 0$ is called a \udefine{null set} (or $\mu$-null set if we want more precision).\retTwo

By subadditivity (a.k.a. the fact that for all $(E_j)_{j \in \mathbb{N}} \subset\mathcal{M},\myHS \mu(\bigcup\limits_{j=1}^\infty E_j) \leq \sum\limits_{j = 1}^\infty \mu(E_j)$),\\ [-8pt] we know countable unions of null sets are also null sets.\newpage

Given a proposition $P(x)$, if there exists a null set $E \in \mathcal{M}$ satisfying that $P(x)$ is true for all $x \in X - E$, then we say $P$ is true \udefine{almost everywhere} (abbreviated as $\mu$-a.e. or just a.e. if the measure being used is clear).\retTwo

A measure space is \udefine{complete} if given any $E \subseteq X$, we have that $N \in \mathcal{M}$ with\\ $\mu(N) = 0$ and $E \subseteq N$ implies that $E \in \mathcal{M}$.

\begin{myIndent}\hTwo
   \blab{Proposition:} Suppose $(X, \mathcal{M}, \mu)$ is a measure space. Let:
   \begin{itemize}
      \item $\mathcal{N} = \{N \in \mathcal{M} \mid \mu(N) = 0\}$
      \item $\overline{\mathcal{M}} = \{E \cup F \mid E \in \mathcal{M} \text{ and } F \subseteq N \text{ where } N \in \mathcal{N}\}$.\retTwo
   \end{itemize}

   Then $\overline{\mathcal{M}}$ is a $\sigma$-algebra and there is a unique extension $\overline{\mu}$ of $\mu$ to a complete measure on $\overline{\mathcal{M}}$.
   
   \begin{myIndent}\hThree
      Proof:\\
      Claim 1: $\overline{\mathcal{M}}$ is a $\sigma$-algebra.
      \begin{myIndent}\hFour
         To see that $\overline{\mathcal{M}}$ is closed under countable union, let $(E_i \cup F_i)_{i \in \mathbb{N}}$ be a sequence of sets in $\overline{\mathcal{M}}$ with each $E_i \in \mathcal{M}$ and $F_i \subseteq N_i$ for some $N_i \in \mathcal{N}$. Then\\ $\bigcup\limits_{i \in \mathbb{N}}(E_i \cup F_i) = \bigcup\limits_{i \in \mathbb{N}}E_i \cup \bigcup\limits_{i \in \mathbb{N}}F_i$.\retTwo

         Importantly, since $\mathcal{M}$ and $\mathcal{N}$ are closed under countable union, we know that $\bigcup\limits_{i \in \mathbb{N}}E_i \in \mathcal{M}$ and $\bigcup\limits_{i \in \mathbb{N}}F_i \subseteq \bigcup\limits_{i \in \mathbb{N}}N_i \in \mathcal{N}$. So, $\bigcup\limits_{i \in \mathbb{N}}(E_i \cup F_i) \in \overline{\mathcal{M}}$.\retTwo

         To show that $\overline{\mathcal{M}}$ is closed under complements, let $E \cup F \in \overline{\mathcal{M}}$ with\\ $E \in \mathcal{M}$ and $F \subseteq N$ for some $N \in \mathcal{N}$. Also note that we can assume\\ $E \cap N = \emptyset$. After all, if $E \cap N \neq \emptyset$, then define $N^\prime = N - E$ and\\ $F^\prime = F - E$. Since $N^\prime \subseteq N$ and $N^\prime \in \mathcal{M}$, we know that $\mu(N^\prime) = 0$. Also, $E \cup F = E \cup F^\prime$ with $F^\prime \subseteq N^\prime$. So, $E$, $F^\prime$, and $N^\prime$ fulfil the same properties we picked $E$, $F$, and $N$ for having. But also $E \cap N^\prime = \emptyset$.\retTwo

         Now, $(E \cup F)^\comp = (E \cup N)^\comp \cup (N - F)$ where $(E \cup N)^\comp \in \mathcal{M}$ and\\ $(N - F) \subset N$. So $(E \cup F)^\comp \in \overline{\mathcal{M}}$.\retTwo
      \end{myIndent}

      Now given any $E \cup F \in \overline{\mathcal{M}}$ with $E \in \mathcal{M}$ and $F \subset N$ for some $N \in \mathcal{N}$, define $\overline{\mu}(E \cup F) = \mu(E)$.\retTwo
      Claim 2: $\overline{\mu}$ is well-defined.
      \begin{myIndent}\hFour
         Suppose $E_1 \cup F_1 = E_2 \cup F_2$ where for $j \in \{1, 2\}$ we have $E_j \in \mathcal{M}$\\ and $F_j \subset N_j$ for some $N_j \in \mathcal{N}$. Then $E_1 \subseteq E_2 \cup N_2$, meaning that\\ $\mu(E_1) \leq \mu(E_2) + \mu(N_2) = \mu(E_2)$. By similar reasoning, we can say that $\mu(E_2) \leq \mu(E_1)$. So $\overline{\mu}(E_1 \cup F_1) = \overline{\mu}(E_2 \cup F_2)$.
         \retTwo
      \end{myIndent}

      \exP (The rest is exercise 1.6:)\\
      Claim 3: $\overline{\mu}$ is a complete measure on $\overline{\mathcal{M}}$.
      \begin{myIndent}\exPP
         It's easy to show that $\overline{\mu}$ is a measure. After all, $\emptyset \in \mathcal{M} \cap \mathcal{N}$. So,\\ $\overline{\mu}(\emptyset) = \overline{\mu}(\emptyset \cup \emptyset) = \mu(\emptyset) = 0$. Also, suppose $(E_i \cup F_i)_{i \in \mathbb{N}}$ is a\\ sequence of disjoint sets in $\overline{\mathcal{M}}$ with $E_i \in \mathcal{M}$ and $F_i \subset N_i$ for some $N_i \in \mathcal{N}$.\newpage

         Then $\bigcup\limits_{i \in \mathbb{N}}E_i \in \mathcal{M}$ where each $E_i$ is disjoint and $\bigcup\limits_{i \in \mathbb{N}}F_i \subseteq \bigcup\limits_{i \in \mathbb{N}}N_i \in \mathcal{N}$. So:

         {\center 
         \begin{tabular}{l}
            $\overline{\mu}(\bigcup\limits_{i \in \mathbb{N}}(E_i \cup F_i)) = \overline{\mu}(\bigcup\limits_{i \in \mathbb{N}}E_i \cup \bigcup\limits_{i \in \mathbb{N}}F_i)$\\ [12pt]
             $\phantom{\overline{\mu}(\bigcup\limits_{i \in \mathbb{N}}(E_i \cup F_i))} = \mu(\bigcup\limits_{i \in \mathbb{N}}E_i) = \sum\limits_{i = 1}^\infty\mu(E_i) = \sum\limits_{i = 1}^\infty\overline{\mu}(E_i \cup F_i)$
         \end{tabular}\retTwo\par}

         Finally, to show that $(X, \overline{\mathcal{M}}, \overline{\mu})$ is complete, suppose $A \subseteq X$ and $N_1 \in \overline{\mathcal{M}}$ with $\overline{\mu}(N_1) = 0$ and $A \subseteq N_1$. By definition, we know $N_1 = E \cup F$ where $E \in \mathcal{M}$ and $F \subseteq N_2$ for some $N_2 \in \mathcal{N}$. However, we can also assume $E = \emptyset$. For if $E \neq \emptyset$, then because $\mu(E \cup N_2) \leq \mu(E) + \mu(N_2) \leq 0$, we can define $N_2^\prime = E \cup N_2$ and $F^\prime = E \cup F$. Then $N_1 = \emptyset \cup F^\prime$ and\\ $F^\prime \subseteq N_2^\prime$ where $N_2^\prime \in \mathcal{N}$.\retTwo

         So $A \subset F \subset N_2$ where $N_2 \in \mathcal{N}$. It follows that $A = \emptyset \cup A \in \overline{\mathcal{M}}$.\retTwo
      \end{myIndent}

      Claim 4: $\overline{\mu}$ is the unique measure on $\overline{\mathcal{M}}$ that extends $\mu$.\\ [-9pt]
      \begin{myIndent}\exPP
         Suppose $\overline{\overline{\mu}}$ is another measure on $\overline{\mathcal{M}}$ such that $\overline{\overline{\mu}}|_\mathcal{M} = \mu$. Then consider any $E \cup F \in \overline{M}$ such that $E \in \mathcal{M}$ and $F \subseteq N$ for some $N \in \mathcal{N}$. As shown before, we can assume without loss of generality that $E \cap N = \emptyset$ and thus also $E \cap F = \emptyset$. So, we have that:

         {\centering $\overline{\overline{\mu}}(E \cup F) = \overline{\overline{\mu}}(E) + \overline{\overline{\mu}}(F)$\retTwo\par}

         Next, note that:

         {\centering\fontsize{11}{13}\selectfont $\mu(E) = \overline{\overline{\mu}}(E) \leq \overline{\overline{\mu}}(E) + \overline{\overline{\mu}}(F) \leq \overline{\overline{\mu}}(E) + \overline{\overline{\mu}}(N) = \mu(E) + \mu(N) = \mu(E)$ \retTwo\par}

         Hence, we know that $\overline{\overline{\mu}}(E \cup F) = \mu(E)$. But also $\overline{\mu}(E \cup F) = \mu(E)$. So $\overline{\overline{\mu}}(E \cup F) = \overline{\mu}(E \cup F)$ for all $E \cup F \in \overline{\mathcal{M}}$.\retTwo
      \end{myIndent}
   \end{myIndent}

   Note: We call $\overline{\mu}$ the \udefine{completion} of $\mu$ and $\overline{\mathcal{M}}$ the \udefine{completion} of $\mathcal{M}$ with respect\\ to $\mu$.
\end{myIndent}

\mySepTwo

\mHeader{Lecture 4 Notes: 10/8/2024}

An \udefine{outer measure} on a nonempty set $X$ is a function $\mu^*: \mathcal{P}(X) \longrightarrow [0, \infty]$\\ satisfying that:
\begin{enumerate}
   \item $\mu^*(\emptyset) = 0$.
   \item $\mu^*(A) \leq \mu^*(B)$ if $A \subseteq B$.
   \item $\mu^*(\bigcup\limits_{j = 1}^\infty A_j) \leq \sum\limits_{j=1}^\infty \mu^*(A_j)$. {\teachComment (this property is called subadditivity)}\newpage
\end{enumerate}

\begin{myIndent}\hTwo
   \blab{Proposition:} Let $\mathcal{E} \subseteq \mathcal{P}(X)$ be a collection of elementary sets and $\mu: \mathcal{E} \longrightarrow [0, \infty]$ be a function satisfying that $\mu(\emptyset) = 0$.
   \begin{myIndent}\teachComment
      The textbook only assumes that $\emptyset$ and $X$ are in $\mathcal{E}$.
   \end{myIndent}

   Then define $\mu^*(A) = \inf\left\{\sum\limits_{j = 1}^\infty \mu(E_j)\hspace{0.2em} \middle| \hspace{0.2em} E_j \in \mathcal{E} \text{ and } A \subseteq \bigcup\limits_{j = 1}^\infty E_j\right\}$. This is an outer\\ [-9pt] measure.\retTwo

   \begin{myIndent}\hThree
      Proof:\\
      Given any set $A \subseteq \mathcal{P}(X)$, define $A^* = \left\{\sum\limits_{j = 1}^\infty \mu(E_j)\hspace{0.2em} \middle| \hspace{0.2em} E_j \in \mathcal{E} \text{ and } A \subseteq \bigcup\limits_{j = 1}^\infty E_j\right\}$.\\ [-9pt] That way $\mu^*(A) = \inf(A^*)$.
      \begin{enumerate}
         \item $\mu^*(A)$ is well defined because $\mu(X) \in A^*$ and $\mu(\emptyset) = 0$ is a lower bound\\ of $A^*$.\\[ -8pt]
         \item Since $0 \in \emptyset^*$, we know that $\mu^*(\emptyset) = \inf(\emptyset^*) = 0$.\\ [-8pt]
         \item Suppose $A \subseteq B$. Then given any $(E_j)_{j \in \mathbb{N}}$ of sets in $\mathcal{E}$ covering $B$, we know\\ that they will also cover $A$. So, $A^* \subseteq B^*$, meaning $\mu^*(A) \leq \mu^*(B)$.\\ [-8pt]
         \item Suppose $(A_j)_{j \in \mathbb{N}} \subseteq \mathcal{P}(X)$. Then fix $\varepsilon > 0$. For all $j \in \mathbb{N}$, let $(E_j^{(k)})_{k \in \mathbb{N}}$ be a sequence of sets in $\mathcal{E}$ such that $A_j \subseteq \bigcup\limits_{k \in \mathbb{N}}E_j^{(k)}$ and:
         
         {\centering$\mu^*(A_j) \leq \sum\limits_{k = 1}^\infty \mu(E_j^{(k)}) \leq \mu^*(A_j) + \sfrac{\varepsilon}{2^j}$.\retTwo\par}

         Note that $\bigcup\limits_{j \in \mathbb{N}}A_j \subseteq \bigcup\limits_{j \in \mathbb{N}}(\bigcup\limits_{k \in \mathbb{N}}E_j^{(k)})$.\retTwo
         
         So, $(\bigcup\limits_{j \in \mathbb{N}}A_j)^* \ni \sum\limits_{j = 1}^\infty \sum\limits_{k = 1}^\infty\mu(E_j^{(k)}) \leq \sum\limits_{j = 1}^\infty (\mu^*(A_j) + \sfrac{\varepsilon}{2^j}) = \varepsilon + \sum\limits_{j = 1}^\infty \mu^*(A_j)$.\retTwo

         Since $\varepsilon$ was arbitrary, we thus know that:
         
         {\centering $\mu^*(\bigcup\limits_{j \in \mathbb{N}}A_j) = \inf (\bigcup\limits_{j \in \mathbb{N}}A_j)^* \leq \sum\limits_{j = 1}^\infty \mu^*(A_j)$.\retTwo\par}
      \end{enumerate}
   \end{myIndent}
\end{myIndent}

Let $\mu^*$ be an outer measure on a nonempty set $X$. Then $A \subseteq X$ is called\\ \udefine{$\mu^*$-measurable} if $\mu^*(E) = \mu^*(E \cap A) + \mu^*(E - A)$ for all $E \subseteq X$.
\begin{myIndent}\teachComment
   Note that $\mu^*(E) \leq \mu^*(E \cap A) + \mu^*(E - A)$ for all $E \in \mathcal{P}(X)$. So $\mu^*$-measurability just means that the other inequality holds.\retTwo

   \hTwo\blab{Carathéodory's Theorem:} If $\mu^*$ is an outer measure on $X$, then the collection $\mathcal{M}$ of $\mu^*$-measurable sets is a $\sigma$-algebra and the restriction of $\mu^*$ to $\mathcal{M}$ is a complete measure.
   
   \begin{myIndent}\hThree
      Proof:\\
      Part 1: $\mathcal{M}$ is an algebra and $\mu^*$ is additive on $\mathcal{M}$ (meaning $A, B \in \mathcal{M}$ with\\ $A \cap B = \emptyset$ implies that $\mu^*(A \cup B) = \mu^*(A) + \mu^*(B))$.
      
      \begin{myIndent}\hFour
         We know $\mathcal{M}$ is an algebra because:
         \begin{itemize}
            \item $\emptyset \in \mathcal{M}$ because $\mu^*(E) = \mu^*(E \cap \emptyset) + \mu^*(E - \emptyset) = 0 + \mu^*(\emptyset)$ for\\ all $E \subseteq X$.\newpage
            \item Both $A^\comp \in \mathcal{M}$ $A \in \mathcal{M}$ are equivalent to us having for all $E \subseteq X$ that $\mu^*(E) = \mu^*(E \cap A) + \mu^*(E \cap A^\comp)$.\\ [-6pt]
            \item Suppose $A$ and $B$ are sets in $\mathcal{M}$. Then given $E \subseteq X$, we have:
            
            {\centering 
            \begin{tabular}{l}
               $\mu^*(E) = \mu^*(E \cap A) + \mu^*(E - A)$\\
               $\phantom{\mu^*(E)} = \mu^*((E \cap A) \cap B) + \mu^*((E \cap A) - B)$\\ $\phantom{aaaaaaaaaaaaaaaaaa} + \mu^*((E - A) \cap B) + \mu^*((E - A) - B)$
            \end{tabular} \retTwo\par}

            Now $(E - A) - B = E \cap A^\comp \cap B^\comp = E \cap (A \cup B)^\comp$. Meanwhile,\\ $(E \cap A) - B = E \cap (A - B)$ and $(E - A) \cap B = E \cap (B - A)$.\\ So, by subadditivity, we have that:

            {\centering\begin{tabular}{l}
               $\mu^*(E \cap (A \cup B)) + \mu(E - (A \cup B))$\\
               $\phantom{aaaaaaaaaaaaa} \leq \mu^*((E \cap A) \cap B) + \mu^*((E \cap A) - B)$\\ $\phantom{aaaaaaaaaaaaaaaa} + \mu^*((E - A) \cap B) + \mu^*((E - A) - B)$
            \end{tabular} \retTwo\par}

            Hence, $\mu^*(E \cap (A \cup B)) + \mu^*(E - (A \cup B)) \leq \mu^*(E)$. So,\\ $A \cup B \in \mathcal{M}$.\retTwo
         \end{itemize}

         Next, to show that $\mu$ is additive on $\mathcal{M}$, consider any $A, B \in \mathcal{M}$ with\\ $A \cap B = \emptyset$. Then:

         {\centering\fontsize{11}{13}\selectfont $\mu^*(A \cup B) = \mu^*((A \cup B) \cap A) + \mu^*((A \cup B) - A) = \mu^*(A) + \mu^*(B) $ \retTwo\par}
      \end{myIndent}

      Part 2: $\mathcal{M}$ is a $\sigma$-algebra and $\mu^*$ is $\sigma$-additive (think countably additive) on $X$.

      \begin{myIndent}\hFour
         To show that $\mathcal{M}$ is a $\sigma$-algebra, it suffices to show that $\mathcal{M}$ is closed under countable disjoint unions. So let $(A_j)_{j \in \mathbb{N}}$ be a sequence of disjoint sets in $\mathcal{M}$. If $E$ is any set in $X$ and $m > 1$, then:

         {\centering $\mu^*(E) = \mu^*(E \cap (\bigcup\limits_{j=1}^mA_j)) + \mu^*(E - (\bigcup\limits_{j=1}^mA_j))$  \retTwo\par}

         But then note that because $\mathcal{M}$ is an algebra, we know $\bigcup\limits_{j=1}^mA_j \in \mathcal{M}$. So:
         
         {\centering\fontsize{11.5}{13.5}\selectfont 
         \begin{tabular}{l}
            $\mu^*(E \cap (\bigcup\limits_{j=1}^mA_j)) = \mu^*(E \cap (\bigcup\limits_{j=1}^mA_j) \cap A_m) + \mu^*(E \cap (\bigcup\limits_{j=1}^mA_j) \cap A_m^\comp)$ \\ [-2pt]
            $\phantom{\mu^*(E \cap (\bigcup\limits_{j=1}^mA_j))} = \mu^*(E \cap A_m) + \mu^*(E \cap (\bigcup\limits_{j = 1}^{m-1}\hspace{-0.2em} A_j))$
         \end{tabular} \retTwo\par}

         By induction, we thus have that $\mu^*(E \cap \bigcup\limits_{j=1}^mA_j) = \sum\limits_{j=1}^m \mu^*(E \cap A_j)$. Also, since\\ [-9pt] $E - (\bigcup\limits_{j=1}^mA_j) \supset E - (\bigcup\limits_{j\in\mathbb{N}} A_j)$, we thus know that:

         {\centering $\mu^*(E) \geq \sum\limits_{j=1}^m \mu^*(E \cap A_j) + \mu^*(E - \bigcup\limits_{j\in\mathbb{N}}A_j)$ \retTwo\par}

         Taking the limit as $m \rightarrow \infty$, we thus get that:
         
         {\centering
         \begin{tabular}{l}
             $\mu^*(E) \geq \sum\limits_{j=1}^\infty \mu^*(E \cap A_j) + \mu^*(E - \bigcup\limits_{j \in \mathbb{N}}A_j)$\\ $\phantom{aaaaaaaaaaaaaaa} \geq \mu^*(E \cap (\bigcup\limits_{j\in \mathbb{N}}A_j)) + \mu^*(E - \bigcup\limits_{j \in \mathbb{N}}A_j)\phantom{aaa}$
         \end{tabular}\newpage\par}

         So, $\bigcup\limits_{j \in \mathbb{N}}A_j$ is $\mu^*$-measurable. Hence, $\mathcal{M}$ is a $\sigma$-algebra.\retTwo

         Also, in order to show that $\mu^*(\bigcup\limits_{j \in \mathbb{N}}A_j) = \sum\limits_{j=1}^\infty\mu^*(A_j)$, just substitute\\ $E = \hspace{-0.2em}\bigcup\limits_{j \in \mathbb{N}}\hspace{-0.2em}A_j$ into the expression at the bottom of the last page.\retTwo
      \end{myIndent}

      Part 3: $(X, \mathcal{M}, \mu^*)$ is a complete measure space.
      \begin{myIndent}\hFour
         Suppose $\mu^*(A) = 0$. Then given $E \subseteq X$, we have that:

         {\centering $\mu^*(E) = \mu^*(E \cap A) + \mu^*(E - A) \leq \mu^*(A) + \mu^*(E) \leq 0 + \mu^*(E)$ \retTwo\par}

         It follows that $\mu^*(E) = \mu^*(E \cap A) + \mu^*(E - A)$ for all $E$. So $A \in \mathcal{M}$.\retTwo

         Now if $\mu^*(A) = 0$, then $\mu^*(E) = 0$ for all $E \subseteq A$. It follows that all subsets of $\mu^*$-null sets are in $\mathcal{M}$.\retTwo

         
         \begin{myTindent}\teachComment
            The moral of the story is that we'll just call $\mu^*$ a measure because it is when restricted to the right $\sigma$-algebra.
         \end{myTindent}
      \end{myIndent}
   \end{myIndent}
\end{myIndent}

\mySepTwo

A \udefine{premeasure} $\mu_0 : \mathcal{A} \longrightarrow [0, \infty]$ is a function on an algebra satisfying that:
\begin{itemize}
   \item $\mu_0(\emptyset) = 0$
   \item if $(A_j)_{j \in \mathbb{N}}$ is a sequence of disjoint sets in $\mathcal{A}$ with $\bigcup\limits_{j \in \mathbb{N}}A_j \in \mathcal{A}$, then\\ [-14pt] $\mu_0(\bigcup\limits_{j \in \mathbb{N}}A_j) = \sum\limits_{j=1}^\infty \mu_0(A_j)$.
\end{itemize}


\begin{myTindent}\teachComment
   By setting all but finitely many $A_j$ to the emptyset, we can show that $\mu_0$ must be finitely additive. In turn, this is enough to show that $\mu_0(A) \leq \mu_0(B)$ if $A \subseteq B$ for any $A, B \in \mathcal{A}$.
\end{myTindent}

We say $\mu^*$ is \udefine{induced by} $\mu_0$ if $\mu^*(A) = \inf\left\{\sum\limits_{j = 1}^\infty \mu_0(E_j)\hspace{0.2em} \middle| \hspace{0.2em} E_j \in \mathcal{A} \text{ and } A \subseteq \bigcup\limits_{j = 1}^\infty E_j\right\}$.\retTwo

Note that $\mu^*$ is an outer measure by a previous proposition.\retTwo

\begin{myIndent}\hTwo
   \blab{Proposition:} In this situation:
   \begin{enumerate}
      \item $\mu^*|_\mathcal{A} = \mu_0$
      \begin{myIndent}\hThree
         Proof:\\
         Suppose $E \in \mathcal{A}$ and let $(A_j)_{j \in \mathbb{N}}$ be a sequence of sets of in $\mathcal{A}$ covering $E$. It's trivial that $\mu^*(E) \leq \mu_0(E)$ because we could just let $A_1 = E$ and $A_n = \emptyset$ for all $n \geq 2$.\newpage
         
         On the other hand, letting $B_m = E \cap A_1$ and $B_m = E \cap A_m - \bigcup\limits_{j = 1}^{m-1}A_j$, we have\\ that $(B_j)_{j \in \mathbb{N}}$ is a sequence of disjoint sets in $\mathcal{A}$ whose union is $E$. It follows\\ [5pt] from the second property of a premeasure and the fact that $B_j \subseteq A_j$ for all\\ [5pt] $j$ that:
         
         {\centering$\mu_0(E) = \sum\limits_{j=1}^\infty \mu_0(B_j) \leq \sum\limits_{j=1}^\infty \mu_0(A_j)$\retTwo\par}
         
         Since $(A_j)_{j \in \mathbb{N}}$ was not specified, it follows that $\mu_0(E) \leq \mu^*(E)$.\retTwo
      \end{myIndent}

      \item Every set in $\mathcal{A}$ is $\mu^*$-measurable.
      \begin{myIndent}\hThree
         Proof:\\
         Suppose $A \in \mathcal{A}, E \subseteq X$, and $\varepsilon > 0$. Then there is a sequence $(B_j)_{j \in \mathbb{N}}$ of sets in $\mathcal{A}$ with $E \subseteq \bigcup\limits_{j=1}^\infty B_j$ and $\sum\limits_{j=1}^\infty \mu_0(B_j) \leq \mu^*(E) + \varepsilon$. Since $\mu_0$ is additive on $\mathcal{A}$, $(B_j \cap A)_{j \in \mathbb{N}} \subseteq \mathcal{A}$ covers $E \cap A$, and $(B_j - A)_{j \in \mathbb{N}} \subseteq \mathcal{A}$ covers $E - A$, we have\\ [5pt] that:

         {\centering 
         \begin{tabular}{l}
            $\mu^*(E) + \varepsilon \geq \sum\limits_{j = 1}^\infty \mu_0(B_j) = \sum\limits_{j = 1}^\infty \mu_0(B_j \cap A) + \sum\limits_{j = 1}^\infty \mu_0(B_j - A)$\\
            $\phantom{\mu^*(E) + \varepsilon \geq \sum\limits_{j = 1}^\infty \mu_0(B_j)} \geq \mu^*(E \cap A) + \mu^*(E - A) $
         \end{tabular} \retTwo\par}

         Taking $\varepsilon \rightarrow 0$, we get that $\mu^*(E) \geq \mu^*(E \cap A) + \mu^*(E - A)$.\retTwo
      \end{myIndent}
   \end{enumerate}

   \blab{Theorem:} Suppose $\mathcal{A} \subseteq \mathcal{P}(X)$ is an algebra and $\mu_0: \mathcal{A} \longrightarrow [0, \infty]$ is a premeasure. Then there exists $\mu: \mathcal{M}(\mathcal{A}) \longrightarrow [0, \infty]$ such that:
   \begin{itemize}
      \item $\mu|_\mathcal{A} = \mu_0$
      \item if $\nu: \mathcal{M}(\mathcal{A}) \longrightarrow [0, \infty]$ is a measure with $\nu|_\mathcal{A} = \mu|_\mathcal{A}$, then $\nu \leq \mu$ (with equality if $\mu(E) < \infty$).
      \item If $\mu_0$ is $\sigma$-finite, then $\mu$ is the unique extension of $\mu_0$ to a measure on $\mathcal{M}$.
   \end{itemize}

   
   \begin{myIndent}\hThree
      Proof:\\
      
      \begin{enumerate}
         \item The first claim is true by Carathéodory's Theorem and the last proposition.\\ Specifically, define $\mu = \mu^*|_{\mathcal{M}(\mathcal{A})}$ where $\mu^*$ is the outer measure induced by\\ $\mu_0$. Since $\mathcal{A}$ is a subset of the $\sigma$-algebra $\mathcal{M}$ of $\mu^*$ measurable sets, we know\\ that $\mathcal{M}(\mathcal{A}) \subseteq \mathcal{M}$. So $\mu$ is a measure over $\mathcal{M}(\mathcal{A})$. Also, we know that\\ $\mu(A) = \mu^*(A) = \mu_0(A)$ for all $A \in \mathcal{A}$ by the last proposition.\retTwo
   
         \item To show the second claim, let $E \in \mathcal{M}(\mathcal{A})$ and $(A_j)_{j \in \mathbb{N}} \subseteq \mathcal{A}$ be a covering of $E$ such that $\sum\limits_{j=1}^\infty \mu_0(A_j) \leq \mu(E) + \varepsilon$ for a given $\varepsilon > 0$. Then:
         
         {\centering$\nu(E) \leq \sum\limits_{j=1}^\infty \nu(A_j) = \sum\limits_{j=1}^\infty \mu_0(A_j) \leq  \mu(E) + \varepsilon$.\retTwo\par}
   
         Taking $\varepsilon \rightarrow 0$, we get that $\nu(E) \leq \mu(E)$.\newpage
   
         As for the other inequality, consider that for any $(A_j)_{j \in \mathbb{N}} \subseteq \mathcal{A}$, we know by part\\ 3 of the theorem on page 16 and the fact that $\nu(A_j) = \mu(A_j)$ for all $j$ that if\\ $A = \bigcup\limits_{j\in\mathbb{N}}A_j$, then:\\ [-17pt]
         
         {\centering $ \nu(A) = \lim\limits_{m\rightarrow \infty}(\nu(\bigcup\limits_{j=1}^m A_j)) = \lim\limits_{m\rightarrow \infty}(\mu(\bigcup\limits_{j=1}^m A_j)) = \mu(A)$. \retTwo\par}
   
         Also, if $\mu(E)$ is finite, then we can choose the covering $(A_j)_{j \in \mathbb{N}} \subseteq \mathcal{A}$ of $E$ so that all $A_j$ are disjoint and $A$ satisfies for a given $\varepsilon > 0$ that:
   
         {\centering $\mu(E) \leq \mu(A) = \sum\limits_{j=1}^\infty \mu(A_j) < \mu(E) + \varepsilon$ \retTwo\par}
   
         It follows that $\mu(A - E) < \varepsilon$. So:
         
         {\centering $\mu(E) \leq \mu(A) = \nu(A) = \nu(E) + \nu(A - E) \leq \nu(E) + \mu(A - E) \leq \nu(E) + \varepsilon$.\retTwo\par}
   
         Taking $\varepsilon \rightarrow 0$, we get that $\mu(E) \leq \nu(E)$.\retTwo

         \item For the third claim, suppose $X = \bigcup\limits_{j \in \mathbb{N}}A_j$ with $\mu_0(A_j) < \infty$ and all $A_j$ being\\ disjoint. Then for any $E \in \mathcal{M}(\mathcal{A})$:
         
         {\centering $\mu(E) = \sum\limits_{j=1}^\infty \mu(E \cap A_j) = \sum\limits_{j=1}^\infty \nu(E \cap A_j) = \nu(E) $ \retTwo\par}
      \end{enumerate}
   \end{myIndent}
\end{myIndent}

\exOne

\mySepTwo

\blab{Exercise 1.16:} Let $(X, \mathcal{M}, \mu)$ be a measure space. A set $E \subseteq X$ is called \udefine{locally\\ measurable} if $E \cap A \in \mathcal{M}$ for all $A \in \mathcal{M}$ such that $\mu(A) < \infty$. Let $\widetilde{\mathcal{M}}$ be the\\ collection of all locally measurable sets. Trivially, we know $\mathcal{M} \subseteq \widetilde{\mathcal{M}}$. If $\mathcal{M} = \widetilde{\mathcal{M}}$,\\ then $\mu$ is called \udefine{saturated}.

\begin{enumerate}
   \item[(a)] If $\mu$ is $\sigma$-finite, then $\mu$ is saturated.
   
   \begin{myIndent}\exTwoP
      Let $(A_j)_{j \in \mathbb{N}} \subseteq \mathcal{M}$ satisfy that $\mu(A_j) < \infty$ for all $j$, and that $X = \bigcup\limits_{j\in\mathbb{N}}A_j$. Then\\ [-8pt] if $E$ is locally measurable, we know: $E = \bigcup\limits_{j \in \mathbb{N}}(E \cap A_j) \in \mathcal{M}$. 
   \end{myIndent}

   \item[(b)] $\widetilde{\mathcal{M}}$ is a $\sigma$-algebra.
   \begin{myIndent}\exTwoP
      \begin{itemize}
         \item If $E \in \widetilde{\mathcal{M}}$, then given any $A \in \mathcal{M}$ with $\mu(A) < \infty$, we know $E \cap A \in \mathcal{M}$. It follows that $E^\comp \cap A = A - E = A - (E \cap A) \in \mathcal{M}$. So $E^\comp \in \widetilde{\mathcal{M}}$.\\ [-8pt]
         \item Suppose $(E_j)_{j \in \mathbb{N}}$ is a sequence of sets in $\widetilde{\mathcal{M}}$ and $A \in \mathcal{M}$ satisfies that\\ $\mu(A) < \infty$. Then: $(\bigcup\limits_{j \in \mathbb{N}}E_j) \cap A = \bigcup\limits_{j \in \mathbb{N}}(E_j \cap A) \in \mathcal{M}$.\retTwo

         So, $\bigcup\limits_{j \in \mathbb{N}}E_j \in \widetilde{\mathcal{M}}$
      \end{itemize}
   \end{myIndent}

   \item[(c)] Define $\widetilde{\mu}$ on $\widetilde{\mathcal{M}}$ by $\widetilde{\mu}(E) = \mu(E)$ if $E \in \mathcal{M}$ and $\widetilde{\mu}(E) = \infty$ otherwise. Then $\widetilde{\mu}$ is a saturated measure on $\mathcal{M}$ called the \udefine{saturation} of $\mu$.
   
   \begin{myIndent}\exTwoP
      Since $\emptyset \in \mathcal{M}$, we know $\widetilde{\mu}(\emptyset) = \mu(\emptyset) = 0$.\newpage 

      Note that if $A, B \in \widetilde{\mathcal{M}}$ with $A \subseteq B$ and $A \notin \mathcal{M}$ but $B \in \mathcal{M}$, then we\\ immediately get a contradiction since that would suggest $A = A \cap B \in \mathcal{M}$. As a result, supposing $(E_j)_{j \in \mathbb{N}}$ is a sequence of disjoint sets in $\widetilde{\mathcal{M}}$, we have that if any $E_j \notin \mathcal{M}$, then:\\ [-22pt]
      
      {\centering$\widetilde{\mu}(\bigcup\limits_{j \in \mathbb{N}}E_j) = \infty = \sum\limits_{j =1}^\infty \widetilde{\mu}(E_j)$.\retTwo\par}

      Meanwhile, if all sets of $(E_j)_{j \in \mathbb{N}}$ are in $\mathcal{M}$, then:\\ [-16pt]

      {\centering$\widetilde{\mu}(\bigcup\limits_{j \in \mathbb{N}}E_j) = \mu(\bigcup\limits_{j \in \mathbb{N}}E_j) = \sum\limits_{j =1}^\infty \mu(E_j) = \sum\limits_{j =1}^\infty \widetilde{\mu}(E_j)$.\retTwo\par}
   \end{myIndent}

   \item[(d)] If $\mu$ is complete, then so is $\widetilde{\mu}$.
   
   \begin{myIndent}\exTwoP
      This fact is obvious because by the way we defined $\widetilde{\mu}$, we know a set is $\widetilde{\mu}$-null if and only if it is $\mu$-null.\retTwo
   \end{myIndent}

   \item[(e)] Suppose that $\mu$ is semifinite. For $E \in \widetilde{\mathcal{M}}$, define:
   
   {\centering $\underline{\mu}(E) = \sup \{\mu(A) \mid A \in \mathcal{M} \text{ and } A \subseteq E \}$.\par}

   This is well defined because $\mu(\emptyset)$ is always in the above set and $\mu(X)$ is an upper-\\bound. Then $\underline{\mu}$ is a saturated measure on $\widetilde{\mathcal{M}}$ that extends $\mu$.\\ [-16pt]

   \begin{myIndent}\exTwoP
      Firstly, we show $\underline{\mu}$ is a measure. To start, it's trivial to see that $\underline{\mu}(\emptyset) = \mu(\emptyset) = 0$.\retTwo

      Lemma: If $E \in \widetilde{\mathcal{M}}$ and $\underline{\mu}(E) = \infty$, then there exists a set $A \in \mathcal{M}$ such that $A \subseteq E$ and $\mu(A) = \infty$. 
      \begin{myIndent}\exPPP
         To show this, construct a sequence of "increasing" sets $(A_j)_{j \in \mathbb{N}}$ satisfying\\ that $A_j \subseteq E$ and $\mu(A_j) \geq j$. Then the union $A$ of that sequence will satisfy\\ that $A \in \mathcal{M}$, that $A \subseteq E$, and that $\mu(A) = \infty$.\retTwo
      \end{myIndent}

      Because of that lemma, we don't need to deal with the edge case that a least upper bound equaling infinity doesn't mean a set contains infinity. So, let $(E_j)_{j \in \mathbb{N}}$ be a sequence of disjoint sets in $\widetilde{\mathcal{M}}$ with $E = \bigcup\limits_{j \in \mathbb{N}} E_j$. Then let $\varepsilon > 0$.\retTwo

      To show one inequality, pick a sequence $(A_j)_{j \in \mathbb{N}}$ of sets in $\mathcal{M}$ satisfying that $A_j \subseteq E_j$ and $\underline{\mu}(E_j) - \sfrac{\varepsilon}{2^j} \leq \mu(A_j)$. Since $A = \bigcup\limits_{j \in \mathbb{N}}A_j \subseteq E$, and each $A_j$ is\\ [-10pt] disjoint, we thus have:

      {\centering $-\varepsilon + \sum\limits_{j=1}^\infty \underline{\mu}(E_j) \leq \sum\limits_{j=1}^\infty \mu(A_j) = \mu(A) \leq \underline{\mu}(E)$ \newpage\par}

      To show the other inequality, pick $B \in \mathcal{M}$ satisfying that $B \subseteq E$ and\\ [1pt] $\underline{\mu}(E) - \varepsilon < \mu(B)$. Because $E, E_j \in \widetilde{\mathcal{M}}$ for each $j$, we know that $B \cap E$ and\\ [3pt] $B \cap E_j$ are in $\mathcal{M}$ for each $j$. So:

      {\centering $\underline{\mu}(E) - \varepsilon < \mu(B) = \mu(B \cap E) = \sum\limits_{j=1}^\infty \mu(B \cap E_j) \leq \sum\limits_{j=1}^\infty\underline{\mu}(E_j)$\\ [1pt]\par}

      Taking $\varepsilon \rightarrow 0$, we thus get that $\underline{\mu}(E) = \sum\limits_{j=1}^\infty \underline{\mu}(E_j)$.\retTwo

      Proving that $\mu(E) = \underline{\mu}(E)$ when $E \in \mathcal{M}$ is trivial. Obviously, $\mu(E) \leq \underline{\mu}(E)$.\\ [2pt] Meanwhile for any $F \in \mathcal{M}$ satisfying that $F \subseteq E$, we know that $\mu(F) \leq \mu(E)$.\\ [2pt] So, there does not exists a subset of $E$ in $\mathcal{M}$ with greater measure than $\mu(E)$.\retTwo
   \end{myIndent}

   Note that $\widetilde{\mu}$ and $\underline{\mu}$ are not necessarily equal. Part (f) of this problem gives a relatively\\ [2pt] simple counterexample.

\end{enumerate}

\mySepTwo

\blab{Exercise 1.17:} If $\mu^*$ is an outer measure on $X$ and $(A_j)_{j\in \mathbb{N}}$ is a sequence of disjoint\\ $\mu^*$-measurable sets, then:

{\centering $\mu^*(E \cap \bigcup\limits_{j \in \mathbb{N}}A_j) = \sum\limits_{j=1}^\infty \mu^*(E \cap A_j)$ for any $E \subseteq X$.\retTwo\par}


\begin{myIndent}\exTwoP
   Note that by induction, we can show that for any $n \in \mathbb{N}$:

   {\centering\exPP
   \begin{tabular}{l}
      $\mu^*(E \cap \bigcup\limits_{j=1}^\infty A_j) = \mu^*(E \cap \bigcup\limits_{j=1}^\infty A_j \cap A_1) + \mu^*(E \cap \bigcup\limits_{j=1}^\infty A_j - A_1)$\\ [4pt]
      $\phantom{\mu^*(E \cap \bigcup\limits_{j=1}^\infty A_j)} = \mu^*(E \cap A_1) + \mu^*(E \cap \bigcup\limits_{j=2}^\infty A_j)$\\ [4pt]
      $\phantom{\mu^*(E \cap \bigcup\limits_{j=1}^\infty A_j)} = \sum\limits_{j=1}^2\mu^*(E \cap A_j) + \mu^*(E \cap \bigcup\limits_{j=3}^\infty A_j)$\\ [4pt]
      $\phantom{\mu^*(E \cap \bigcup\limits_{j=1}^\infty A_j)} = \cdots = \sum\limits_{j=1}^n \mu^*(E \cap A_j) + \mu^*(E \cap \hspace{-0.5em}\bigcup\limits_{j=n+1}^\infty\hspace{-0.5em} A_j)$
   \end{tabular} \retTwo\par}

   Thus, we clearly have for all $n$ that $\sum\limits_{j=1}^n \mu^*(E \cap A_j) \leq \mu^*(E \cap \bigcup\limits_{j=1}^\infty A_j)$.\\ [-6pt]

   Taking the limit as $n \rightarrow \infty$, we thus know $\sum\limits_{j=1}^\infty \mu^*(E \cap A_j) \leq \mu^*(E \cap \bigcup\limits_{j \in \mathbb{N}}A_j)$.\retTwo

   The other inequality is obvious from the subadditivity property of outer measures.\retTwo
\end{myIndent}

\mySepTwo

\blab{Exercise 1.18:} Let $\mathcal{A} \subseteq \mathcal{P}(X)$ be an algebra, $\mathcal{A}_\sigma$ be be the collection of countable unions of sets in $\mathcal{A}$, and $\mathcal{A}_{\sigma\delta}$ be the collection of countable intersections of sets in $\mathcal{A}_{\sigma}$. Let $\mu_0$ be a premeasure on $\mathcal{A}$ and $\mu^*$ the induced outer measure.\newpage

\begin{enumerate}
   \item[(a)] For any $E \subseteq X$ and $\varepsilon > 0$, there exists $A \in \mathcal{A}_\delta$ with $E \subseteq A$ and $\mu^*(A) \leq \mu^*(E) + \varepsilon$.
   
   \begin{myIndent}\exTwoP
      Let $(A_j)_{j \in \mathbb{N}}$ be a sequence of sets in $\mathcal{A}$ which cover $E$ and satisfy that\\ $\sum\limits_{j=1}^\infty \mu_0(A_j) \leq \mu^*(E) + \varepsilon$. In turn, by the subadditivity of outer measures, and\\ [1pt] the fact that $\mu^*(A) = \mu_0(A)$ for all $A \in \mathcal{A}$, we know that:

      {\centering $E \subseteq \bigcup\limits_{j \in \mathbb{N}}A_j \in \mathcal{A}_{\sigma}$ and $\mu^*(\bigcup\limits_{j \in \mathbb{N}}A_j) \leq \sum\limits_{j=1}^\infty \mu^*(A_j) = \sum\limits_{j=1}^\infty \mu_0(A_j) \leq \mu^*(E) + \varepsilon$ \retTwo\par}
   \end{myIndent}
   
   \item[(b)] If $\mu^*(E) < \infty$, then $E$ is $\mu^*$-measurable if and only if there exists $B \in \mathcal{A}_{\sigma\delta}$ with $E \subseteq B$ and $\mu^*(B - E) = 0$.
   
   \begin{myIndent}\exTwoP
      Suppose $E$ is $\mu^*$-measurable. Then for all $j \in \mathbb{N}$, pick $A_j \in \mathcal{A}_{\sigma}$ satisfying that\\ $E \subseteq A_j$ and $\mu^*(E) \leq \mu^*(A_j) \leq \mu^*(E) + \sfrac{1}{j}$. Since $E$ is $\mu^*$-measurable, we\\ know that for all $j \in \mathbb{N}$:

      {\centering $\mu^*(A_j) = \mu^*(A_j \cap E) + \mu^*(A_j - E) = \mu^*(E) + \mu^*(A_j - E)$ \retTwo\par}

      In turn, because $\mu^*(E) < \infty$, this tell us that $\mu^*(A_j - E) \leq \sfrac{1}{j}$. Also, because $\bigcap\limits_{j \in \mathbb{N}}A_j \subseteq A_n$ for all $n \in \mathbb{N}$, we know that $\mu^*(\bigcap\limits_{j\in\mathbb{N}}A_j - E) < \sfrac{1}{n}$ for all $n \in \mathbb{N}$.\retTwo

      As a result, we know that $\bigcap\limits_{j \in \mathbb{N}}A_j \in \mathcal{A}_{\sigma\delta}$, $E \subseteq \bigcap\limits_{j \in \mathbb{N}}A_j$, and $\mu^*(\bigcap\limits_{j\in\mathbb{N}}A_j - E) = 0$.\retTwo

      To prove the reverse implication, suppose there exists a $\mu^*$-separable set $B$\\ satisfying that $E \subseteq B$ and $\mu^*(B - E) = 0$ (any set in $\mathcal{A}_{\sigma\delta}$ will be $\mu^*$-separable). Then given any set $F$, we have that:

      {\centering 
      \begin{tabular}{l}
         $\mu^*(F - E) = \mu^*(F \cap E^\comp \cap B) + \mu^*(F \cap E^\comp \cap B^\comp)$\\ [2pt]
         $\phantom{\mu^*(F - E)} = \mu^*(F \cap (B - E)) + \mu^*(F - B) = \mu^*(F - B)$
      \end{tabular} \retTwo\par}

      Also, since $F \cap E \subseteq F \cap B$, we know that $\mu^*(F \cap E) \leq \mu^*(F \cap B)$.\retTwo

      So, $\mu^*(F \cap E) + \mu^*(F - E) \leq \mu^*(F \cap B) + \mu^*(F - B)  = \mu^*(F)$. Hence, $E$ is $\mu^*$-measurable.\retTwo      
   \end{myIndent}

   \item[(c)] If $\mu_0$ is $\sigma$-finite, we can remove the requirement in part (b) that $\mu^*(E) < \infty$.
   
   \begin{myIndent}\exTwoP
      Because the backwards implication proof never required $\mu^*(E)$ to be finite, it suffices to show that $E$ being $\mu^*$-measurable implies there exists $B \in A_{\sigma\delta}$ with $E \subseteq B$ and $\mu^*(B - E) = 0$.\retTwo

      To start, let $(C_i)_{i \in \mathbb{N}}$ satisfy that $\mu_0(C_i) < \infty$ and $\bigcup\limits_{i = 1}^\infty C_i = X$.\newpage
      
      Next for all $j, i \in \mathbb{N}$, pick $A_j^{(i)} \in \mathcal{A}_\sigma$ satisfying that $E \cap C_i \subseteq A_j^{(i)}$ and\\ $\mu^*(E) \leq \mu^*(A_j^{(i)}) \leq \mu^*(E) + \sfrac{1}{j2^{i}}$. Since $\mu^*(E \cap C_i)$ is finite, we can use the same reasoning as in part (b) to say that $\mu^*(A_j^{(i)} - (E \cap C_i)) \leq \sfrac{1}{j2^i}$.\retTwo

      Importantly, $A_j^{(i)} - E \subseteq A_j^{(i)} - (E \cap C_i)$. for all $i$. Therefore:
      
      {\centering
      \begin{tabular}{l}
         $\mu^*((\bigcup\limits_{i\in\mathbb{N}}A_j^{(i)}) - E) = \mu^*(\bigcup\limits_{i\in\mathbb{N}}(A_j^{(i)} - E))$\\ [6pt]
         
         $\phantom{\mu^*((\bigcup\limits_{i\in\mathbb{N}}A_j^{(i)}) - E) } \leq \mu^*(\bigcup\limits_{i\in\mathbb{N}}(A_j^{(i)} - (E \cap C_i)))$\\
         
         $\phantom{\mu^*((\bigcup\limits_{i\in\mathbb{N}}A_j^{(i)}) - E) } \leq \sum\limits_{i \in \mathbb{N}}\mu^*(A_j^{(i)} - (E \cap C_i)) \leq \sum\limits_{i \in \mathbb{N}}\frac{1}{j2^i} = \frac{1}{j}$
      \end{tabular} \retTwo\par}

      Since $E \subseteq \bigcup\limits_{i \in \mathbb{N}}A_j^{(i)} \in \mathcal{A}_\sigma$, we've thus shown for all $j \in \mathbb{N}$ that there exists a set\\ [-8pt]\phantom{Aaaaaaaaaaaaaaaaaa} $A_j \in \mathcal{A}_\sigma$ satisfying that $\mu^*(A_j - E) \leq \sfrac{1}{j}$ and $E \subseteq A_j$.\retTwo

      Finally, intersecting all those $A_j$ like in part (b), we get our set satisfying the right-side of the implication.\retTwo
   \end{myIndent}
\end{enumerate}

\mySepTwo

\blab{Exercise 1.19:} Let $\mu^*$ be an outer measure on $X$ induced from a finite premeasure $\mu_0$. If $E \subseteq X$, define the \udefine{inner measure} of $E$ to be $\mu_*(E) = \mu_0(X) - \mu^*(E^\comp)$. Then $E$ is $\mu^*$-measurable if and only if $\mu^*(E) = \mu_*(E)$.

\begin{myIndent}\exTwo
   
   
\end{myIndent}


\end{document}












% $\bigotimes\limits_{\alpha \in A}\mathcal{M}_\alpha = \{\pi^{-1}_\alpha(E_\alpha) \mid E_\alpha \in \mathcal{M}_\alpha, \alpha \in A\}$