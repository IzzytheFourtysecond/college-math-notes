\documentclass{book}

\usepackage{fontspec} % used to import Calibri
\usepackage{anyfontsize} % used to adjust font size

% needed for inch and other length measurements
% to be recognized
\usepackage{calc}

% for colors and text effects as is hopefully obvious
\usepackage[dvipsnames]{xcolor}
\usepackage{soul}

% control over margins
\usepackage[margin=1in]{geometry}
\usepackage[strict]{changepage}

\usepackage{mathtools}
\usepackage{amsfonts}
\usepackage{bm}

\usepackage[scr=rsfso, scrscaled=.96]{mathalpha}

\usepackage{amssymb} % originally imported to get the proof square
\usepackage{xfrac}
\usepackage[overcommands]{overarrows} % Get my preferred vector arrows...
\usepackage{relsize}

% Just am using this to get a dashed line in a table...
% Also you apparently want this to be inactive if you aren't
% using it because it slows compilation.
\usepackage{arydshln} \ADLinactivate 
\newenvironment{allowTableDashes}{\ADLactivate}{\ADLinactivate}

\usepackage{graphicx}
\graphicspath{{./158_Images/}}

\usepackage{tikz}
   \usetikzlibrary{arrows.meta}
   \usetikzlibrary{graphs, graphs.standard}

\usepackage{quiver} %commutative diagrams


\newfontfamily{\calibri}{Calibri}
\setlength{\parindent}{0pt}
\definecolor{RawerSienna}{HTML}{945D27}

% ~~~~~~~~~~~~~~~~~~~~~~~~~~~~~~~~~~~~~~~~~~~~~~~~~~
%Arrow Commands:

% Thank you Bernard, gernot, and Sigur who I copied this from:
% https://tex.stackexchange.com/questions/364096/command-for-longhookrightarrow
\newcommand{\hooklongrightarrow}{\lhook\joinrel\longrightarrow}
\newcommand{\hooklongleftarrow}{\longleftarrow\joinrel\rhook}
\newcommand{\hookxlongrightarrow}[2][]{\lhook\joinrel\xrightarrow[#1]{#2}}
\newcommand{\hookxlongleftarrow}[2][]{\xleftarrow[#1]{#2}\joinrel\rhook}

% Thank you egreg who I copied from:
% https://tex.stackexchange.com/questions/260554/two-headed-version-of-xrightarrow
\newcommand{\longrightarrowdbl}{\longrightarrow\mathrel{\mkern-14mu}\rightarrow}
\newcommand{\longleftarrowdbl}{\leftarrow\mathrel{\mkern-14mu}\longleftarrow}

\newcommand{\xrightarrowdbl}[2][]{%
  \xrightarrow[#1]{#2}\mathrel{\mkern-14mu}\rightarrow
}
\newcommand{\xleftarrowdbl}[2][]{%
  \leftarrow\mathrel{\mkern-14mu}\xleftarrow[#1]{#2}
}

\newcommand{\MRoman}[1]{%
   \textrm{\MakeUppercase{\romannumeral #1}}%
}

% ~~~~~~~~~~~~~~~~~~~~~~~~~~~~~~~~~~~~~~~~~~~~~~~~~~

\newcommand{\learnToSpot}[1]{{\color{Red}#1}}

\newcommand{\hOne}{%
   \color{Black}%
   \fontsize{14}{16}\selectfont%
}
\newcommand{\hTwo}{%
\color{MidnightBlue}%
   \fontsize{13}{15}\selectfont%
}
\newcommand{\hThree}{%
   \color{PineGreen!85!Orange}
   \fontsize{12}{14}\selectfont%
}
\newcommand{\hFour}{%
   \color{Cyan!80!black}
   \fontsize{12}{14}\selectfont%
}
\newcommand{\myComment}{%
   \color{RawerSienna}%
   \fontsize{12}{14}\selectfont%
}
\newcommand{\teachComment}{
   \color{Orange}%
   \fontsize{12}{14}\selectfont%
}
\newcommand{\exOne}{%
   \color{Purple}%
   \fontsize{13}{15}\selectfont%
}
\newcommand{\exTwo}{%
   \color{Purple}%
   \fontsize{13}{15}\selectfont%
}
\newcommand{\exP}{%
   \color{Purple}%
   \fontsize{12}{14}\selectfont%
}
\newcommand{\exTwoP}{%
   \color{RedViolet}%
   \fontsize{13}{15}\selectfont%
}
\newcommand{\exPP}{%
   \color{RedViolet}%
   \fontsize{12}{14}\selectfont%
}
\newcommand{\exPPP}{%
   \color{VioletRed}%
   \fontsize{12}{14}\selectfont%
}
% ~~~~~~~~~~~~~~~~~~~~~~~~~~~~~~~~~~~~~~~~~~~~~~~~

\newcommand{\cyPen}[1]{{\vphantom{.}\color{Cerulean}#1}}
\newcommand{\redPen}[1]{{\vphantom{.}\color{Red}#1}}

\newenvironment{myIndent}{%
   \begin{adjustwidth}{2.5em}{0em}%
}{%
   \end{adjustwidth}%
}

\newenvironment{myDindent}{%
   \begin{adjustwidth}{5em}{0em}%
}{%
   \end{adjustwidth}%
}

\newenvironment{myTindent}{%
   \begin{adjustwidth}{7.5em}{0em}%
}{%
   \end{adjustwidth}%
}

\newenvironment{myConstrict}{%
   \begin{adjustwidth}{2.5em}{2.5em}%
}{%
   \end{adjustwidth}%
}

\newcommand{\udefine}[1]{{%
   \setulcolor{Red}%
   \setul{0.14em}{0.07em}%
   \ul{#1}%
}}

\newcommand{\blab}[1]{\textbf{#1}}

\newcommand{\uuline}[2][.]{%
{\vphantom{a}\color{#1}%
\rlap{\rule[-0.18em]{\widthof{#2}}{0.06em}}%
\rlap{\rule[-0.32em]{\widthof{#2}}{0.06em}}}%
#2}

\newcommand{\pprime}{{\prime\prime}}
\newcommand{\suchthat}{ \hspace{0.3em}s.t.\hspace{0.3em}}
\newcommand{\comp}{\mathsf{C}}
\newcommand{\card}{\mathrm{card}}
\newcommand{\diam}{\mathrm{diam}}
\newcommand{\myHS}{ \hspace{0.5em}}

\newcommand{\df}{\mathrm{d}}

\newcommand{\myId}{\mathrm{Id}}
\newcommand{\myIm}{\mathrm{im}}
\newcommand{\myObj}{\mathrm{Obj}}
\newcommand{\myHom}{\mathrm{Hom}}
\newcommand{\myEnd}{\mathrm{End}}
\newcommand{\myAut}{\mathrm{Aut}}

\newcommand{\loc}{\mathsf{loc}}

\newcommand{\mcateg}[1]{{\bm{\mathsf{#1}}}}

\DeclareMathOperator{\rea}{Re}
\DeclareMathOperator{\ima}{Im}
\DeclareMathOperator{\sgn}{sgn}

% Thank you Gonzalo Medina and Moriambar who wrote this on stack exchange:
%https://tex.stackexchange.com/questions/74125/how-do-i-put-text-over-symbols%
\newcommand{\myequiv}[1]{\stackrel{\mathclap{\mbox{\footnotesize{$#1$}}}}{\equiv}}

% Thank you chs who wrote this on stack exchange:
%https://tex.stackexchange.com/questions/89821/how-to-draw-a-solid-colored-circle%
\newcommand{\filledcirc}[1][.]{\ensuremath{\hspace{0.05em}{\color{#1}\bullet}\mathllap{\circ}\hspace{0.05em}}}

%Thank you blerbl who wrote this on stack exchange:
%https://tex.stackexchange.com/questions/25348/latex-symbol-for-does-not-divide
\newcommand{\ndiv}{\hspace{-0.3em}\not|\hspace{0.35em}}

\newcommand{\mySepOne}[1][.]{%
   {\noindent\color{#1}{\rule{6.5in}{1mm}}}\\%
}
\newcommand{\mySepTwo}[1][.]{%
   {\noindent\color{#1}{\rule{6.5in}{0.5mm}}}\\%
}

\newenvironment{myClosureOne}[2][.]{%
   \color{#1}%
   \begin{tabular}{|p{#2in}|} \hline \\%
}{%
   \\ \hline \end{tabular}%
}

\newcommand{\retTwo}{\hfill\bigbreak}

\newcommand{\mHeader}[1]{{
   \color{Black}%
   \fontsize{20}{18}\selectfont%
   #1\retTwo
}}


\title{Math 240A Notes (Professor: Luca Spolaor)}
\author{Isabelle Mills}

\begin{document}
\maketitle{}
\setul{0.14em}{0.07em}
\calibri

\hOne
\mHeader{Lecture 1 Notes: 9/26/2024}

Given an indexed family of sets $\{X_\alpha\}_{\alpha \in A}$, we define its \udefine{Cartesian Product} to be:

{\center $\prod\limits_{\alpha \in A}X_\alpha = \{f: A \longrightarrow \bigcup\limits_{\alpha \in A}X_\alpha \mid f(\alpha \in X_\alpha)\}$ \retTwo\par}

A projection is a function $\pi_\alpha : \prod\limits_{\alpha \in A}X_\alpha \longrightarrow X_\alpha$ satisfying that $f \mapsto f(\alpha)$.\retTwo

If $X, Y$ are sets, we define:
\begin{itemize}
	\item $\card(X) \leq \card(Y)$ if there exists an injection $f: X \longrightarrow Y$.
	\item  $\card(X) \geq \card(Y)$ if there exists a surjection $f: X \longrightarrow Y$.
	\item $\card(X) = \card(Y)$ if there exists a bijection $f: X \longrightarrow Y$.
	
	\begin{myIndent}\hTwo
		Note that $\card(X) \leq \card(Y) \Longleftrightarrow \card(Y) \geq \card(X)$. After all, given an injection in one direction, we can easily make a surjection in the other direction. Or given a surjection in one direction, we can \learnToSpot{(using A.O.C (axiom of choice))} easily make an injection in the other direction.\retTwo

		Also, if $\card(X) \leq \card(Y)$ and $\card(Y) \leq \card(X)$, then we know that\\ $\card(Y) = \card(X)$.
		
		\begin{myIndent}\hThree
			Proof:\\
			We know there exists $f: X \longrightarrow Y$ and $g: Y \longrightarrow X$ which are both\\ injective. Hence, $g \circ f$ is an injection from $X$ to $g(Y) \subseteq X$. By an exercise done in my math journal on page 8, we thus there exists a bijection $h$ from $X$ to $g(Y)$. And letting $g^{-1}$ be any left-inverse of $g$, we then have that $g^{-1} \circ h$ is a bijection from $X$ to $Y$.\retTwo
		\end{myIndent}
	\end{myIndent}
\end{itemize}

We say $X$ has the \udefine{cardinality of the continuum} if $\card(X) = \card(\mathbb{R})$.

\begin{myIndent}\hTwo
	Proposition: $\card(\mathcal{P}(\mathbb{N})) = \card(\mathbb{R})$.
	\begin{myIndent}\hThree
		Our textbook goes about proving this by constructing two functions: an injection and a surjection, from $\mathcal{P}(\mathbb{N})$ to $\mathbb{R}$ based on the binary expansion of any real number. That way, we know that $\card(\mathcal{P}(\mathbb{N})) \leq \card(\mathbb{R})$ and $\card(\mathcal{P}(\mathbb{N})) \geq \card(\mathbb{R})$.\retTwo
	\end{myIndent}
\end{myIndent}

Given a sequence $(x_n)_{n \in \mathbb{N}}$ in $\mathbb{R}$ we know there exists: $\limsup x_n = \inf\limits_{k \geq 1}(\sup\limits_{n \geq k} x_n)$ and\\ [-10pt] $\liminf x_n = \sup\limits_{k \geq 1}(\inf\limits_{n \geq k} x_n)$.\retTwo

Also, given a function $f: \mathbb{R} \longrightarrow \overline{\mathbb{R}}$, we can define:

{\centering $\limsup\limits_{x \rightarrow a}f(x) = \inf\limits_{\delta > 0}\left(\sup\limits_{0 < |x-a|<\delta}\hspace{-1em}f(x)\right)$.\newpage\par}

If $X$ is an arbitrary set and $f: X \longrightarrow [0, \infty]$, we define:

{\centering $\sum\limits_{x \in X}f(x) = \sup\left\{\sum\limits_{x \in F}f(x) \mid F \subseteq X \suchthat F \text{ is finite}\right\}$.\retTwo\par}

\begin{myIndent}\hTwo
	Cool Proposition from textbook (not covered in lecture):
	\begin{myIndent}\hTwo
		Let $A = \{x \in X \mid f(x) > 0\}$. If $A$ is uncountable, then $\sum\limits_{x \in X}f(x) = \infty$.\retTwo If $A$ is countably infinite and $g:\mathbb{N} \longrightarrow A$ is a bijection, then\\ $\sum\limits_{x \in X}f(x) = \sum\limits_{n = 1}^\infty f(g(n))$.\retTwo
		
		\begin{myIndent}\hThree
			Proof of first statement:\\
			$A = \bigcup\limits_{n \in \mathbb{N}}A_n$ where $A_n = \{x \in X \mid f(x) > \frac{1}{n}\}$.\retTwo

			If $A$ is uncountable, we must have that some $A_n$ is uncountable. But then for any finite set $F \subseteq X$, we have that $\sum\limits_{x \in F}f(x) > \frac{\card(F)}{n}$. So $\sum\limits_{x \in X}f(x)$ is\\ [-7pt] unbounded.\retTwo
		\end{myIndent}
	\end{myIndent}
\end{myIndent}

A metric space $(X, \rho)$ is a set $X$ equipped with a distance function\\ $\rho: X \times X \longrightarrow [0, \infty)$. We denote the open ball of radius $r$ about $x$ to be\\ $B(r, x) = \{y \in X \mid \rho(x, y) < r\}$. And you remember our definitions from\\ 140A... right?\retTwo


\begin{myIndent}\hTwo
	\blab{Proposition 0.21:} Every open set in $\mathbb{R}$ is a countable union of disjoint open intervals.
		
	\begin{myIndent}\myComment
		We proved this as part of a homework exercise in Math 140A.\retTwo
	\end{myIndent}
\end{myIndent}

\exOne Given a metric space $(X, \rho)$, an element $x \in X$, and sets $F, E \subseteq X$, we can define:
\begin{itemize}
	\item $\rho(x, E) = \rho_E(x) = \inf\{\rho(x, y) \mid y \in E\}$.
	\item $\rho(F, E) = \inf\{\rho_E(y) \mid y \in F\}$.
\end{itemize}


\begin{myIndent}\exTwoP
	\blab{Exercise:} $g(x, E) = 0 \Longleftrightarrow x \in \overline{E}$.
	
	\begin{myIndent}\exPP
		Proof:\\
		If $\inf\{\rho(x, y) \mid y \in E\} = 0$, then there exists a sequence $\{y_n\}$ in $E$ such that\\ $\rho(x, y_n) \rightarrow 0$. This implies $x \in \overline{E}$. Similarly, if $x \in \overline{E}$, we can construct a sequence $\{y_n\}$ such that $\rho(x, y_n) < \frac{1}{n}$ for all $n$. Then:
		
		{\centering $0 \leq \inf\{\rho(x, y) \mid y \in E\} \leq \inf\{\rho(x, y_n) \mid n \in \mathbb{N}\} = 0$.\retTwo\par}
	\end{myIndent}
\end{myIndent}

\hOne
Given a subset $E$ of a metric space $(X, \rho)$, we define:

{\centering$\diam(E) = \sup\{\rho(x, y)\mid x, y \in E\}$.\retTwo\par}

If $\diam(E) < \infty$, we say $E$ is \udefine{bounded}. If $\forall \varepsilon > 0$, $E$ can be covered by finitely many balls of radius $\varepsilon$, then we say $E$ is \udefine{totally bounded}.\newpage

\begin{myIndent}\exTwo
	\blab{Exercise:} $E$ being totally bounded implies $E$ is bounded.
	\begin{myIndent}\exPP
		Pick $\varepsilon > 0$ and let $\{z_1, \ldots, z_n\}$ be the set of points such that $E \subseteq \bigcup\limits_{k = 1}^n B(\varepsilon, z_n)$.\retTwo

		Then given any $x, y \in E$, we can assume that $x \in B(\varepsilon, z_i)$ and $y \in B(\varepsilon, z_j)$. So, $\rho(x, y) \leq \rho(x, z_i) + \rho(z_i, z_j) + \rho(z_j, y) < 2\varepsilon + \max\{\rho(z_i, z_j) \mid 1 \leq i, j \leq n\}$.\retTwo
	\end{myIndent}

	The converse is not generally true. For instance, if you use the discrete metric, then any set with more than one element will have a diameter of $1$. But if $0 < \varepsilon < 1$, then it will be impossible to cover an infinite set with finitely many balls.\retTwo
\end{myIndent}

\mySepTwo

\mHeader{Lecture 2 Notes: 10/1/2024}

\begin{myIndent}\hTwo
   \blab{Proposition:} Suppose $E$ is a subset of a metric space $(X, \rho)$. Then the following are equivalent.
   
   \begin{enumerate}
      \item $E$ is complete and totally bounded
      \item All sequences $(x_n) \subseteq E$, have a convergent subsequence.
      \item For all open covers $\{V_\alpha\}_{\alpha \in A}$ of $E$, there exists $V_{\alpha_1}, \ldots, V_{\alpha_n}$ such that\\ $E \subseteq \bigcup\limits_{i=1}^n V_{\alpha_i}$.\retTwo
   \end{enumerate}
   
   \begin{myIndent}\hThree
      Proof:\\
      (1) $\Longrightarrow$ (2):
      \begin{myIndent}
         Lemma:\\
         If $E$ is totally bounded and $F \subseteq E$, then $F$ is totally bounded.
         \begin{myIndent}\hFour
            Given any $\varepsilon > 0$, let $\{x_1, \ldots, x_n\}$ be a subset of $E$ such that\\ $E \subseteq \bigcup\limits_{i = 1}^n B(\sfrac{\varepsilon}{2}, x_i)$. Then consider the collection of sets:\\ [-8pt] \phantom{aaaaaaaaaaaaaaaaaaaaaaaaaaaaaaa} $\{F \cap B(\sfrac{\varepsilon}{2}, x_i)\} - \{\emptyset\}$.\retTwo

            We know the diameter of each $F \cap B(\sfrac{\varepsilon}{2}, x_i)$ is at most $\varepsilon$. So in each set, pick $y_i \in F \cap B(\sfrac{\varepsilon}{2}, x_i)$. Then for some $m \leq n$:

            {\centering $F \subseteq \bigcup\limits_{i=1}^m B(\varepsilon, y_i)$ \retTwo\par}
         \end{myIndent}

         Let $A_1 = E$. Then for $k \geq 2$ we recursively define $A_k$ as follows:\retTwo
         
         Assuming $A_{k-1} \cap (x_n)_{n\in \mathbb{N}}$ is infinite and $A_{k-1}$ is totally bounded, choose\\ [-2pt] $\{y_1, \ldots, y_m\}$ in $A_k$ such that $A_k \subseteq \bigcup\limits_{i = 1}^m B(2^{-n}, y_i)$. Importantly, since\\ $(x_n)_{n\in \mathbb{N}} \cap A_{k-1}$ is infinite, we know one of those open balls contains\\ [4pt] infinitely many points in our sequence. So set $A_{k}$ equal to that ball\\ [4pt] intersected with $E$. Note that by our lemma, $A_k$ is totally bounded.\newpage

         Now pick any $x_{n_1}$ and then for all $k \geq 2$ pick $x_{n_k} \in A_k$ such that\\ $n_k > n_{k - 1}$. That way, $(x_{n_k})_{k \in \mathbb{Z}_+}$ is a subsequence of $(x_{n})_{n \in \mathbb{Z}_+}$. Also,\\ we know that $(x_{n_k})_{k \in \mathbb{Z}_+}$ is Cauchy. Hence, since $E$ is complete, we know\\ that it converges to some $x$ in $E$.\retTwo
      \end{myIndent}

      (2) $\Longrightarrow$ (1):
      \begin{myIndent}
         Firstly, suppose $E$ is not complete. Then there exists a sequence $(x_n)_{n \in \mathbb{N}}$ that is Cauchy but does not converge in $E$. Importantly, because $(x_n)_{n \in \mathbb{N}}$ is Cauchy, if there was a convergent subsequence, we know the limit of that subsequence would have to be the limit of the whole sequence. But that doesn't exist. So, we know (2) can't be true.\retTwo

         Secondly, suppose $E$ is not totally bounded. Then there exists $\varepsilon > 0$ such that it is impossible to cover $E$ in balls of radius $\varepsilon$. So, we can recursively define a sequence $(x_n)_{n \in \mathbb{N}}$ in $E$ satisfying that:
         
         {\centering$x_n \in E - \bigcup\limits_{i = 1}^{n-1}B(\varepsilon, x_{i})$.\retTwo\par}

         Importantly, for all natural numbers $n \neq m$, we have that $\rho(x_n, x_m) \geq \varepsilon$. So, it is impossible to find a convergent subsequence of $(x_n)$, meaning (2) is false.\retTwo
      \end{myIndent}

      (1) and (2) $\Longrightarrow$ (3):
      \begin{myIndent}
         Let $\{V_\alpha\}_{\alpha \in A}$ be an open cover of $E$.\retTwo

         Suppose for the sake of contradiction that for all $n \in \mathbb{N}$, there is a ball $B_n$ of radius $2^{-n}$ centered in $E$ such that $B_n \cap E \neq \emptyset$ but $B_n \not\subseteq V_\alpha$ for all $\alpha \in A$. Then we can construct a sequence $(x_n)_{n \in \mathbb{N}}$ in $E$ such that $x_n \in B_n \cap E$\\ for all $n \in \mathbb{N}$. By (2), we know there is a subsequence that converges to some $x \in E$. Importantly, we know $x \in V_\alpha$ for some $\alpha \in A$, and because $V_\alpha$ is open, there is $\varepsilon > 0$ such that $B(\varepsilon, x) \subseteq V_\alpha$. But now we get a contradiction because by picking $n$ such that $2^{-n} < \sfrac{\varepsilon}{3}$ and $\rho(x, x_n) < \sfrac{\varepsilon}{3}$, we have for all $y \in B_n$ that:

         {\centering $\rho(x, y) \leq \rho(x, x_n) + \rho(x_n, y) < 2^{-n} + 2^{-n+1} < \varepsilon$ \retTwo\par}

         So $B_n \subseteq B(\varepsilon, x) \subseteq V_\alpha$.\retTwo

         We've thus shown that for some $n \in N$, all balls of radius $2^{-n}$ centered\\ in $E$ are contained by some $V_\alpha$. And assuming (1), we can cover $E$ with finitely many balls of radius $2^{-n}$ It follows that by picking a $V_\alpha$ containing a ball for each ball covering $E$, we've found a finite covering $E$ using the sets in $\{V_\alpha\}_{\alpha \in A}$.\retTwo
      \end{myIndent}

      (3) $\Longrightarrow$ (2):
      \begin{myIndent}
         Suppose $(x_n)_{n \in \mathbb{N}}$ is a sequence in $E$ with no convergent subsequence. Then for each $x \in E$, there must exist $\varepsilon_x > 0$ such that $B(\varepsilon_x, x) \cap (x_n)_{n \in \mathbb{N}}$ is finite. (If $\varepsilon_x$ didn't exist, we could construct a Cauchy subsequence converging to $x$).\newpage

         But now $\{B(\varepsilon_x, x)\}_{x \in E}$ is an open cover of $E$ with no finite subcover of $E$ because it will take an infinite cover to cover all of $(x_n)_{n \in \mathbb{N}}$.\retTwo
      \end{myIndent}
   \end{myIndent}

   If $E$ satisfies all three of the above properties, we say $E$ is \udefine{compact}.\retTwo

   \blab{Corollary}: $K \subseteq \mathbb{R}^n$ is compact iff it's closed and bounded.
\end{myIndent}

\mySepTwo

Roughly speaking, we want a measure to be a function $\mu: \mathcal{P}(\mathbb{R}^n) \longrightarrow [0, \infty)$ such that $E \mapsto \mu(E) =$ "the area of $E$". Also, we would like it if:
\begin{enumerate}
   \item[(i)] $\mu([0, 1)^n) = 1$
   \item[(ii)] $\mu(\text{rotation, translation, or reflection of } A) = \mu(A)$
   \item[(iii)] $\mu(\bigcup\limits_{i=1}^\infty A_i) = \sum\limits_{i = 1}^\infty \mu(A_i)$ if $A_i \cap A_j \neq \emptyset \Longrightarrow i = j$.
\end{enumerate}

Unfortunately, the properties as written above are inconsistent.

\begin{myIndent}\hTwo
   \blab{Vitali Sets:}\\
   Defining $x \sim y$ iff $x - y \in \mathbb{Q}$, let $N \subseteq [0, 1)$ be a set such that $N \cap [x]$ has precisely one element for all $x \in \mathbb{R}$. Next let $R = [0, 1) \cap \mathbb{Q}$, and for all $r \in R$ define:
   
   {\centering$N_r = \{x + r \mid x \in N \cap [0, 1-r)\}\cup \{x + r - 1 \mid x \in N \cap [1 - r, 1)\} $.\retTwo\par}

   Importantly, note that $N_r \subseteq [0, 1)$. Plus, the two sets being unioned over to make $N_r$ are both disjoint and can be translated around so that they are still disjoint but their union forms $N$. Hence assuming $\mu : \mathcal{P}(\mathbb{R}^n) \longrightarrow [0, \infty)$ satisfying (ii) and (iii), we know $\mu(N_r) = \mu(N)$.\retTwo

   Also, for all $y \in [0, 1)$, if $x \in N \cap [y]$, we know that $y \in N_r$ where $r = x - y$ if $x \geq y$, or $r = x - y + 1$ if $x < y$. Hence, $[0, 1) = \bigcup\limits_{r \in R}N_r$.\retTwo

   Also, given any $N_r$ and $N_s$, if $x \in N_r \cap N_s$, then we'd be able to show that both $x - r$ or $x - r + 1$ and $x - s$ or $x - s + 1$ are distinct elements of $N$ in the same equivalence class, which contradicts how we defined $N$.
   \begin{myTindent}\begin{myTindent}\myComment
      You work through the scratch work of the different cases on your own! :P\retTwo
   \end{myTindent}\end{myTindent}

   So supposing $\mu$ satisfies (i) and (iii) and because $R$ is countable, we have that:
   
   {\centering $1 = \sum\limits_{r \in R}\mu(N_r) = \sum\limits_{r \in R}(N) = 0 \text{ or } \infty$.\retTwo\par}

   This is a contradiction.\retTwo
\end{myIndent}

Furthermore, the problem is not the countable union property as is demonstrated by the Banach-Tarsky paradox:\newpage


\begin{myIndent}
   \hTwo
   \blab{Theorem:} Let $U$ and $V$ be arbitrary bounded sets in $\mathbb{R}^n$ where $n \geq 3$. Then there exists $E_1,\ldots, E_N, F_1, \ldots, F_N$ in $\mathbb{R}^n$ such that:
   \begin{itemize}
      \item $E_i \cap E_j = \emptyset$ for all $i \neq j$ and $\bigcup\limits_{i = 1}^NE_i = U$
      \item $F_i \cap F_j = \emptyset$ for all $i \neq j$ and $\bigcup\limits_{i = 1}^NF_i = V$
      \item $E_i$ and $F_i$ are congruent for all $i \in \{1, \ldots, N\}$.\retTwo
   \end{itemize}

   Supposing that $\mu(E_j)$ and $\mu(F_j)$ exists for all $j$ and that $\mu$ satisfies (i), (ii), and (iii) except only for finite unions, then that would suggest all sets have the same "area", which we know doesn't make sense.
\end{myIndent}

What we will do to fix this issue is only define $\mu$ on a subset of $\mathcal{P}(\mathbb{R}^n)$.\retTwo

\mySepTwo 

Let $X \neq \emptyset$. An \udefine{algebra of sets} in $X$ is a nonempty collection $\mathcal{A} \subseteq \mathcal{P}(X)$ which is closed under finite unions and complements. If is $\mathcal{A}$ is also closed under countable unions, we say $\mathcal{A}$ is a $\sigma$-algebra.

\begin{myIndent}\hTwo
   Observations:
   \begin{enumerate}
      \item Algebras of sets are closed under finite intersections and $\sigma$-algebras are\\ closed under countable intersection. (This also means algebras of sets are closed under set differences.)
      
      \begin{myIndent}\hThree
         This is because $\bigcap\limits_{n \in \mathbb{N}}\hspace{-0.2em}A_n = \left(\bigcup\limits_{n \in \mathbb{N}}\hspace{-0.2em}A_n^\comp\right)^\comp$\hspace{-0.4em}.
      \end{myIndent}

      \item If $\mathcal{A}$ is closed under disjoint countable union, then it's closed under arbitrary countable unions.
      
      \begin{myIndent}\hThree
         This is because $\bigcup\limits_{n \in \mathbb{N}}\hspace{-0.2em}A_n = A_1 \cup \bigcup\limits_{n \geq 2}\left(A_n \cap \left(\bigcup\limits_{i = 1}^{n-1} A_i\right)^{\hspace{-0.2em}\comp}\right)$
      \end{myIndent}

      \item If $\{\mathcal{E}_\alpha\}_{\alpha \in A}$ is a collection of $\sigma$-algebras, then $\bigcap\limits_{\alpha \in A}\mathcal{E}_\alpha$ is a $\sigma$-algebra.
      
      \begin{myIndent}\hThree
         This is pretty trivial to prove. It should remind you of topologies.\retTwo 
      \end{myIndent}
   \end{enumerate}
\end{myIndent}

\exOne

\blab{Exercise 1.1:} A family of sets $\mathcal{R} \subseteq \mathcal{P}(X)$ is called a \udefine{ring} if it is closed under finite unions and difference. If $\mathcal{R}$ is also closed under countable unions, it is called a $\sigma$-ring.
\begin{enumerate}
   \item[(a)] Rings are closed under finite intersections and $\sigma$-rings are closed under countable intersections.
   
   \begin{myIndent}\exTwoP
      If $\mathcal{R}$ is a ring and $A_1, \ldots, A_n \in \mathcal{R}$, then:\\ [-18pt]
      
      {\centering $\bigcap\limits_{i = 1}^n A_n = A_1 - \bigcup\limits_{i = 2}^n(A_1 - A_i) \in \mathcal{R}$ \retTwo\par}

      This is because each $A_1 - A_i \in \mathcal{R}$, meaning $\bigcup\limits_{i = 2}^n(A_1 - A_i) \in \mathcal{R}$, and so finally\\ [-10pt] $A_1 - \bigcup\limits_{i = 2}^n(A_1 - A_i) \in \mathcal{R}$.\retTwo

      If $\mathcal{R}$ is a $\sigma$-algebra, we can replace the finite intersection and union used in the prior reasoning with a countable intersection and union.\retTwo
   \end{myIndent}
   \item[(b)] If $\mathcal{R}$ is a ring (or $\sigma$-ring), then $\mathcal{R}$ is an algebra (or $\sigma$-algebra) iff $X \in \mathbb{R}$.
   
   \begin{myIndent}\exTwoP
      ($\Longrightarrow$) Suppose $\mathcal{R}$ is an algebra. Then note that $\emptyset \in \mathcal{R}$ because for any $A \in \mathcal{R}$, $A - A \in \mathcal{R}$. So taking complements, we get that $X \in \mathcal{R}$.\retTwo

      ($\Longleftarrow$) Suppose $X \in \mathcal{R}$. Then for any $A \in \mathcal{R}$, we know that $A^\comp = X - A \in \mathcal{R}$. So $\mathcal{R}$ is an algebra (or $\sigma$-algebra).\retTwo
   \end{myIndent}

   \item[(c)] If $\mathcal{R}$ is a $\sigma$-ring, then $\mathcal{A} = \{E \subseteq X \mid E \in \mathcal{R} \text{ or } E^\comp \in \mathcal{R}\}$ is a $\sigma$-algebra.
   
   \begin{myIndent}\exTwoP
      To start, we know that $\mathcal{A}$ is closed under complements because for any $A \in \mathcal{A}$,
      \begin{myIndent}
         $A \in \mathcal{R} \Longrightarrow (A^\comp)^\comp \in \mathcal{R} \Longrightarrow A^\comp \in \mathcal{A}$\\
         $A \notin \mathcal{R} \Longrightarrow A^\comp \in \mathcal{R} \Longrightarrow A^\comp \in \mathcal{A}$\retTwo
      \end{myIndent}

      Also, let $(E_n)_{n \in \mathbb{N}}$ be a countable collection of sets in $\mathcal{A}$. Then define\\ $A = \{n \in \mathbb{N} \mid E_n^\comp \notin \mathcal{R}\}$ and $B = \{n \in \mathbb{N} \mid E_n^\comp \in \mathcal{R}\}$. Clearly, we have that:

      {\centering $\bigcup\limits_{n \in \mathbb{N}}E_n = \bigcup\limits_{n \in A}E_n \cup \bigcup\limits_{n \in B}E_n = \bigcup\limits_{n \in A}E_n \cup \bigcup\limits_{n \in B}(E_n^\comp)^{\comp}$\\ [2pt]\par}

      Also $\bigcup\limits_{n \in B}(E_n^\comp)^\comp = \left(\bigcap\limits_{n \in B}\hspace{-0.3em}E_n^\comp\right)^{\hspace{-0.3em}\comp}$\hspace{-0.3em}, and by part (a), we know that $E_B \coloneq \bigcap\limits_{n \in B}\hspace{-0.3em}E_n^\comp \in \mathcal{R}$.\retTwo

      Similarly, we know $E_A \coloneq \bigcup\limits_{n \in A}\hspace{-0.3em} E_n \in \mathcal{R}$. So, we've shown that $\bigcup\limits_{n \in \mathbb{N}}E_n = E_A \cup E_B^\comp$\\ [-8pt] where $E_A, E_B \in \mathcal{R}$.\retTwo

      Finally, note that $E_A \cup E_B^\comp = (E_B - E_A)^\comp$. Since $E_B - E_A \in \mathcal{R}$, we know that $(E_B - E_A)^\comp \in \mathcal{A}$. \retTwo
   \end{myIndent}

   \item[(d)] If $\mathcal{R}$ is a $\sigma$-ring, then $\mathcal{A} = \{E \subseteq X \mid E \cap F \in \mathcal{R} \text{ for all } F \in \mathcal{R}\}$ is a $\sigma$-algebra.
   
   \begin{myIndent}\exTwoP
      To start if $E \in \mathcal{A}$, then $E^\comp \in \mathcal{A}$ because for all $F \in \mathcal{R}$ we have that:
      
      {\centering$E^\comp \cap F = F - E = F - (E \cap F) \in \mathcal{R}$.\retTwo\par}

      Also, let $(E_n)_{n \in \mathbb{N}}$ be a countable collection of sets in $\mathcal{A}$. Then for all $F \in \mathcal{R}$, we have that $\left(\bigcup\limits_{n \in \mathbb{N}}E_n\right) \cap F = \bigcup\limits_{n \in \mathbb{N}}(E_n \cap F) \in \mathcal{R}$. So $\mathcal{A}$ is closed under countable union.\newpage
   \end{myIndent}
\end{enumerate}

\hOne Let $\mathcal{E} \subseteq \mathcal{P}(X)$ be a collection of sets. Since the intersection of $\sigma$-algebras is still a $\sigma$-algebra, we define $\mathcal{M}(\mathcal{E})$ to be the smallest $\sigma$-algebra that contains $\mathcal{E}$. In other words, $\mathcal{M}(\mathcal{E})$ is the intersection of all $\sigma$-algebras that contain $\mathcal{E}$.\retTwo

We call $\mathcal{M}(\mathcal{E})$ the $\sigma$-algebra generated by $\mathcal{E}$.


\begin{myIndent}\hTwo
   \blab{Lemma:} if $\mathcal{E} \in \mathcal{M}(\mathcal{F})$, then $\mathcal{M}(\mathcal{E}) \subseteq \mathcal{M}(\mathcal{F})$.
\end{myIndent}

\mySepTwo

Let $(X, \rho)$ be a metric space. We define the \udefine{Borel $\sigma$-algebra} on $X$: $\mathcal{B}_X$, to be the $\sigma$-algebra generated by the collection of all open sets, or equivalently the collection of all closed sets.

\begin{itemize}
   \item A set is $G_\delta$ if it is a countable intersection of open sets.
   \item A set is $F_\sigma$ if it is a countable union of closed sets.
   \item A set is $G_{\delta\sigma}$ if it is a countable union of $G_\delta$ sets.
   \item A set is $F_{\sigma\delta}$ if it is a countable intersection of $F_\sigma$ sets.
   \begin{myTindent}\myComment
      You can hopefully see the pattern. Also the professor isn't sure how much we'll use this $\delta$ and $\sigma$ notation in class.
   \end{myTindent}
\end{itemize}

\exOne\blab{Exercise 1.2:} $\mathcal{B}_\mathbb{R}$ is generated by each of the following:
\begin{itemize}
   \item [(a)] the set of open intervals: $\mathcal{E}_1 = \{(a, b) \mid a < b\}$
   \item [(b)] the set of closed intervals: $\mathcal{E}_2 = \{[a, b] \mid a < b\}$
   \item [(c)] the set of half-open intervals:
   \begin{itemize}
      \item[(i)] $\mathcal{E}_3 = \{(a, b] \mid a < b\}$
      \item[(ii)] $\mathcal{E}_4 = \{[a, b) \mid a < b\}$
   \end{itemize}
   \item [(c)] the set of open rays:
   \begin{itemize}
      \item[(i)] $\mathcal{E}_5 = \{(a, \infty) \mid a \in \mathbb{R}\}$
      \item[(ii)] $\mathcal{E}_6 = \{(-\infty, a) \mid a \in \mathbb{R}\}$
   \end{itemize}
   \item [(d)] the set of closed rays:
   \begin{itemize}
      \item[(i)] $\mathcal{E}_7 = \{[a, \infty) \mid a \in \mathbb{R}\}$
      \item[(ii)] $\mathcal{E}_8 = \{(-\infty, a] \mid a \in \mathbb{R}\}$\retTwo
   \end{itemize}
\end{itemize}

\begin{myIndent}\exTwoP
   Proof:\\
   We trivially have that $\mathcal{M}(\mathcal{E}_1), \mathcal{M}(\mathcal{E}_2), \mathcal{M}(\mathcal{E}_5), \mathcal{M}(\mathcal{E}_6), \mathcal{M}(\mathcal{E}_7), \mathcal{M}(\mathcal{E}_8) \subseteq \mathcal{B}_{\mathbb{R}}$ since each of them contain either only open sets or only closed sets. As for the other inclusions, we must do more work.\newpage

   \begin{itemize}
      \item[(a)] Note that $\mathbb{Q}$ is a countable dense subset of $\mathbb{R}$. Hence, a countable base of $\mathbb{R}$ is the set: $\mathcal{F} = \{(p - q, p + q) \subset \mathbb{R} \mid p, q \in \mathbb{Q} \text{ and } q > 0\}$. In other words, given any open set $E \subseteq \mathbb{R}$, there is a countable subcollection of $\mathcal{F}$ whose union is $E$.
      \begin{myIndent}\exPPP
         To see why, let $x \in E$. Since $E$ is open, there exists $r > 0$ with $B(r, x) \subseteq E$.\\ Next, pick $p \in (x, x + \frac{r}{2}) \cap \mathbb{Q}$, followed by $q \in (p - x, r - p) \cap \mathbb{Q}$. Then $x \in (p - q, p + q) \in \mathcal{F}$ and $(p - q, p + q) \subseteq (x - r, x + r)$.\retTwo
   
         With that, we've now shown that for all $x \in E$, there exists $F \in \mathcal{F}$ such\\ that $x \in F \subseteq E$. If we choose such an $F_x$ for all $x \in E$, we then get\\ that $E = \bigcup\limits_{x \in E}F_x$. So $E$ is the union of a subcollection of $\mathcal{F}$. But since $\mathcal{F}$ is\\ [-9pt]\phantom{aaaaaaaaaaaaaaa} countable, the set $\{F_x \in \mathcal{F} \mid x \in E\}$ is also countable.\retTwo
      \end{myIndent}
   
      Importantly, $\mathcal{F} \subset \mathcal{E}_1$. So $\mathcal{M}(\mathcal{F}) \subseteq \mathcal{M}(\mathcal{E}_1)$. However as shown above, we\\ must have that $\mathcal{M}(\mathcal{F})$ includes all open sets. So by our lemma on the previous page, $\mathcal{B}_{\mathbb{R}} \subseteq \mathcal{M}(\mathcal{F}) \subseteq \mathcal{M}(\mathcal{E}_1)$.\retTwo

      \item[(b)] Given any $E = (a, b) \in \mathcal{E}_1$, we can write that $E = \hspace{-0.3em}\bigcup\limits_{n \in \mathbb{Z}_+}\hspace{-0.3em} [a+\frac{1}{n}, b - \frac{1}{n}]$. Thus,\\ [-10pt] $\mathcal{E}_1 \subseteq \mathcal{M}(\mathcal{E}_2)$, meaning $\mathcal{B}_\mathbb{R} = \mathcal{M}(\mathcal{E}_1) \subseteq \mathcal{M}(\mathcal{E}_2)$.\retTwo
      
      \item[(c)] Remember that for these two, we still need to show that $\mathcal{M}(\mathcal{E}_1), \mathcal{M}(\mathcal{E}_2) \in \mathcal{B}_\mathbb{R}$.
      \begin{itemize}
         \item[(i)] Firstly note that if $F = (a, b] \in \mathcal{E}_3$, then $F = \hspace{-0.3em}\bigcap\limits_{n \in \mathbb{Z}_+}\hspace{-0.3em} (a, b+\frac{1}{n})$. So $\mathcal{E}_3 \subseteq \mathcal{M}(\mathcal{E}_1)$.\\
         On the other hand, if $E = (a, b) \in \mathcal{E}_1$, we have that $E = \hspace{-0.3em}\bigcup\limits_{n \in \mathbb{Z}_+}\hspace{-0.3em} (a, b-\frac{1}{n}]$.\\ [-9pt] So $\mathcal{E}_1 \subseteq \mathcal{M}(\mathcal{E}_3)$.\retTwo
         
         By our lemma on the previous page, we thus have that:
         
         {\centering $\mathcal{B}_\mathbb{R} = \mathcal{M}(\mathcal{E}_1) \subseteq \mathcal{M}(\mathcal{E}_3) \subseteq \mathcal{M}(\mathcal{E}_1) = \mathcal{B}_\mathbb{R}$. \retTwo\par}

         \item[(ii)] Mostly identical reasoning as with $\mathcal{E}_3$ shows that:
         
         {\centering $\mathcal{B}_\mathbb{R} = \mathcal{M}(\mathcal{E}_1) \subseteq \mathcal{M}(\mathcal{E}_4) \subseteq \mathcal{M}(\mathcal{E}_1) = \mathcal{B}_\mathbb{R}$\retTwo\par}
      \end{itemize}

      \item[(d)] \phantom{a}\\ [-10pt]
      \begin{itemize}
         \item[(i)] If $E = (a, b) \in \mathcal{E}_1$, then we know that:
         
         {\centering$E  = (a, \infty) \cap (\hspace{-0.3em}\bigcap\limits_{n \in \mathbb{Z}_+}\hspace{-0.3em} (b - \frac{1}{n}, \infty))^\comp \in \mathcal{M}(\mathcal{E}_5)$.\\\par}

         So $\mathcal{E}_1 \subseteq \mathcal{M}(\mathcal{E}_5)$, meaning $\mathcal{B}_\mathbb{R} = \mathcal{M}(\mathcal{E}_1) \subseteq \mathcal{M}(\mathcal{E}_5)$.\retTwo

         \item[(ii)] Analogous reasoning to that with $\mathcal{E}_5$ shows that $\mathcal{B}_\mathbb{R} = \mathcal{M}(\mathcal{E}_1) \subseteq \mathcal{M}(\mathcal{E}_6)$.\retTwo
      \end{itemize}

      \item[(e)] \phantom{a}\\ [-10pt]
      \begin{itemize}
         \item[(i)] If $E = (a, \infty) \in \mathcal{E}_6$, then we have that $E = \hspace{-0.3em}\bigcup\limits_{n \in \mathbb{Z}_+}\hspace{-0.3em} [a + \frac{1}{n}, \infty)$. So $\mathcal{E}_5 \subseteq \mathcal{M}(\mathcal{E}_7)$,\\ [-10pt] meaning that $\mathcal{B}_\mathbb{R} = \mathcal{M}(\mathcal{E}_5) \subseteq \mathcal{M}(\mathcal{E}_7)$.\retTwo

         \item[(ii)] Analogous reasoning as with $\mathcal{E}_7$ shows that $\mathcal{B}_\mathbb{R} = \mathcal{M}(\mathcal{E}_6) \subseteq \mathcal{M}(\mathcal{E}_8)$.\newpage
      \end{itemize}
   \end{itemize}
\end{myIndent}

\blab{Exercise 1.3:} Let $\mathcal{M}$ be an infinite $\sigma$-algebra on $X$.
\begin{itemize}
   \item[(a)] $\mathcal{M}$ contains an infinite sequence of disjoint sets.
   
   \begin{myIndent}\exTwoP
      By the Hausdorff maximum principle, we know there is a subcollection $\mathcal{S}$\\ of $\mathcal{M}$ which is simply ordered by proper subset and is not contained in any other collection of $\mathcal{M}$ which is simply ordered by proper subset.\retTwo
      
      We claim $\mathcal{S}$ can't be finite. For suppose $\mathcal{S} = \{E_1, \ldots, E_n\}$ is a sequence of sets in $\mathcal{M}$ simply ordered by proper subset which are indexed such that $E_i \subset E_{i + 1}$ for all $i \in \{1, \ldots, n-1\}$.
      \begin{myDindent}\exPPP
         (Note: if $\mathcal{S}$ is maximal, then we must have $E_1 = \emptyset$ and $E_n = X$.)\retTwo
      \end{myDindent}
      
      We can partition $\mathcal{M}$ into collections $\mathcal{M}_1, \ldots \mathcal{M}_n$ such that $A \in \mathcal{M}_i$ iff $i$ is the least integer for which $A \subseteq E_i$. Importantly, all sets in $\mathcal{M}$ will fall into a\\ partition because all sets from $\mathcal{M}$ are contained in $E_n$. Also note that while there are infinitely many $A \in \mathcal{M}$, there are only $n$ many partitions. So, there must be a least integer $k$ such that $\mathcal{M}_k$ contains infinitely many $A \in \mathcal{M}$.\\ [-14pt]
      
      \begin{myTindent}\begin{myTindent}\exPPP
         And since $\mathcal{M}_1 = \{\emptyset\}$, we know $k \geq 2$.\retTwo
      \end{myTindent}\end{myTindent}

      The fact that $\mathcal{M}_i$ is finite for all $i < k$ means that there are only finitely many sets from $\mathcal{M}$ contained in $E_{k-1}$. Thus, we can pick a set $B \in \mathcal{A}_k$ such that $B \neq (E_k - E_{k-1}) \cup A$ for any $A \in \mathcal{M}$ that is a subset of $E_{k-1}$.\retTwo

      Note that since $E_{k-1} \cap B$ is a set in $\mathcal{M}$, we must have that $(E_k - E_{k-1}) \not\subseteq B$ or else $B$ would be the union of $(E_k - E_{k-1})$ and a set from $\mathcal{M}$. Thus, we know $E_k$ contains points that neither $B$ nor $E_{k-1}$ have. At the same time, we know $B$ has points that $E_{k-1}$ doesn't have. It follows that: $E_{k-1} \subset E_{k-1} \cup B \subset E_k$. \retTwo

      Via transitivity, $E_{k-1} \cup B$ is comparable via proper subset with $E_i$ for all\\ $i \in \{1, \ldots, n\}$. Hence, we've shown that $\mathcal{S} \cup \{E_{k-1} \cup B\}$ is a sequence of sets in $\mathcal{M}$ simply ordered by proper subset. But this contradicts that $\mathcal{S}$ is maximal.\retTwo

      Now that we know $\mathcal{S}$ is infinite, let $(E_n)_{n \in \mathbb{Z}_+}$ be a sequence of sets in $\mathcal{S}$\\ satisfying that $E_n \subset E_{n + 1}$. Then we have that $(E_{n+1} - E_n)_{n \in \mathbb{Z}_+}$ is an infinite\\ sequence of nonempty disjoint sets in $\mathcal{M}$.\retTwo
   \end{myIndent}

   \item[(b)] Show that $\card(\mathcal{M}) \geq \mathfrak{c}$.
   
   \begin{myIndent}\exTwoP
      Let $(E_n)_{n \in \mathbb{N}}$ be a sequence of nonempty disjoint sets in $\mathcal{M}$. Then if we\\ [2pt] define the map $f: [0, 1]^\mathbb{N} \longrightarrow \mathcal{M}$ such that $(a_0, a_1, a_2, \ldots)$ is mapped to the\\ [2pt]  union of all $E_n$ such that $a_n = 1$, we have that $f$ is an injection.\retTwo

      Hence, $\card(\mathcal{M}) \geq \card([0, 1]^{\mathbb{N}})$. And since there is a trivial bijection\\ from $[0, 1]^{\mathbb{N}}$ and $\mathcal{P}(\mathbb{N})$, plus the fact that we proved early on in the class\\ that $\card(\mathcal{P}(\mathbb{N})) = \card(\mathbb{R})$, we thus know that $\card(\mathcal{M}) \geq \mathfrak{c}$.\newpage
   \end{myIndent}
\end{itemize}

\blab{Exercise 1.4}: An algebra $\mathcal{A}$ is a $\sigma$-algebra if and only if $\mathcal{A}$ is closed under countable\\ increasing unions (meaning $E_1 \subseteq E_2 \subseteq \ldots$).

\begin{myIndent}\exTwoP
   The rightward implication is true since $\mathcal{A}$ being a $\sigma$-algebra means that $\mathcal{A}$ is\\ closed under all countable unions. As for showing the leftward implication,\\ suppose $\{A_n\}_{n \in \mathbb{Z}_+}$ is a countable collection of sets in $\mathcal{A}$. Then for all $n \in \mathbb{Z}_+$,\\ define $E_n = A_1 \cup \ldots \cup A_n$.\retTwo
   
   Since each $E_n$ are finite unions of sets in $\mathcal{A}$, we know that each $E_n$ is in $\mathcal{A}$. Also,\\ we clearly have that $E_1 \subseteq E_2 \subseteq E_3 \subseteq \ldots$\phantom{a} In order to make the sets strictly\\ increasing, let $S = \{1\} \cup \{k \in \mathbb{Z} \mid k > 1 \text{ and } E_{k} - E_{k-1} \neq \emptyset \}$. Then for\\ any $n, m \in S$, we know that $n < m \Longrightarrow E_n \subset E_m$.\retTwo

   Finally, $\hspace{-0.3em}\bigcup\limits_{n \in \mathbb{Z}_+}\hspace{-0.3em} A_n = \hspace{-0.3em}\bigcup\limits_{n \in \mathbb{Z}_+}\hspace{-0.3em} E_n = \bigcup\limits_{n \in S}E_n$.\retTwo Importantly, $S$ is either finite or countably infinite, and $S$ consists of strictly\\ increasing sets. So by the right hypothesis, we know $\bigcup\limits_{n \in S}E_n \in \mathcal{A}$. Hence,\\ [-9pt] the union over $\{A_n\}_{n \in \mathbb{Z}_+}$ is in $\mathcal{A}$. \retTwo
\end{myIndent}

\hOne
\mySepTwo

Let $\{X_\alpha\}_{\alpha \in A}$ be a collection of nonempty sets, and define $X = \prod\limits_{\alpha \in A}X_\alpha$.\\ If $\mathcal{M}_\alpha$ is a $\sigma$-algebra in $X_\alpha$ for all $\alpha \in A$, then we define the \udefine{product $\sigma$-algebra}\\ on $X$ to be: $\bigotimes\limits_{\alpha \in A}\mathcal{M}_\alpha = \mathcal{M}(\{\pi_\alpha^{-1}(E_\alpha) \mid E_\alpha \in \mathcal{M}_\alpha \text{ and } \alpha \in A\})$.\retTwo


\begin{myIndent}\myComment
   To get a better geometric intuition for this definition, consider if $A = \{1, 2\}$. Then:\\ [-8pt]

   {\centering 
   \begin{tabular}{l}
      $\bigotimes\limits_{\alpha \in A}\mathcal{M}_\alpha = \{\pi_1^{-1}(E_1) \mid E_1 \in \mathcal{M}_1\} \cup \{\pi_2^{-1}(E_2) \mid E_2 \in \mathcal{M}_2\}$ \\ [-4pt]$\hphantom{\bigotimes\limits_{\alpha \in A}\mathcal{M}_\alpha} = \{E_1 \times X_2 \mid E_1 \in \mathcal{M}_1\} \cup \{X_1 \times E_2 \mid E_2 \in \mathcal{M}_2\}$
   \end{tabular} \retTwo\par}

   Also, the motivation for this definition is that $\bigotimes\limits_{\alpha \in A}\mathcal{M}_\alpha$ is the smallest $\sigma$-algebra\\ [-2pt] where $\pi_\alpha$ is "measurable" for all $\alpha$. We'll learn what that means shortly...\retTwo

   \hTwo
   \blab{Proposition:}
   \begin{itemize}
      \item[(i)] $A$ is countable implies $\bigotimes\limits_{\alpha \in A}\mathcal{M}_\alpha = \mathcal{M}(\{\prod\limits_{\alpha \in A}E_\alpha \mid \forall \alpha \in A, \myHS E_\alpha \in \mathcal{M}_\alpha\})$
      
      \begin{myIndent}\hThree
         Proof:\\
         If $E_\alpha \in \mathcal{M}_\alpha$, then $\pi_\alpha^{-1}(E_\alpha) = \prod\limits_{\beta \in A}E_\beta$ where $E_\beta = X_\beta$ if $\beta \neq \alpha$ (and\\ [-8pt] $E_\beta = E_\alpha$ if $\beta = \alpha$).\retTwo
         
         So $\pi_\alpha^{-1}(E_\alpha) \in \mathcal{M}(\{\prod\limits_{\alpha \in A}E_\alpha \mid \forall \alpha \in A,\myHS E_\alpha \in \mathcal{M}_\alpha\})$

         On the other hand, $\prod\limits_{\alpha \in A}E_\alpha = \bigcap\limits_{\alpha \in A}\pi_\alpha^{-1}(E_\alpha)$.\retTwo
         
         Since $A$ is countable, we thus know that if $E_\alpha \in \mathcal{M}_\alpha$ for all $\alpha \in A$, then\\ $\prod\limits_{\alpha \in A}E_\alpha  \in \bigotimes\limits_{\alpha \in A}\mathcal{M}_\alpha$.\newpage
      \end{myIndent}

      \item[(ii)] Suppose $\mathcal{M}_\alpha = \mathcal{M}(\mathcal{E}_\alpha)$ for all $\alpha \in A$. Then $\bigotimes\limits_{\alpha \in A}\mathcal{M}_\alpha$ is generated by\\ [-8pt] $\mathcal{F} = \{\pi^{-1}_\alpha(E_\alpha) \mid E_\alpha \in \mathcal{E}_\alpha \text{ and } \alpha \in A\}$.
      
      \begin{myIndent}\hThree
         Proof:\\
         Since $\mathcal{F} \subseteq \{\pi_\alpha^{-1}(E_\alpha) \mid E_\alpha \in \mathcal{M}_\alpha \text{ and } \alpha \in A\}$, we trivially have that\\ $\mathcal{M}(\mathcal{F}) \subseteq \bigotimes\limits_{\alpha \in A} \mathcal{M}_\alpha$.\\ [-2pt]

         As for showing the other inclusion, define for each $\alpha \in A$:

         {\centering$\mathcal{F}_\alpha = \{E \subseteq X_\alpha \mid \pi_\alpha^{-1}(E) \in \mathcal{M}(\mathcal{F})\}$.\retTwo\par}

         Note that $\mathcal{F}_\alpha$ is a $\sigma$-algebra on $X_\alpha$ that contains $\mathcal{E}_\alpha$.
         \begin{myIndent}\hFour
            This is because for any $F \in \mathcal{F}_\alpha$ and $(E_n)_{n \in \mathbb{N}} \subseteq \mathcal{F}_\alpha$, we know that:
            \begin{myIndent}
               \begin{itemize}
                  \item[\bullet] $\left(\pi_{\alpha}^{-1}(F)\right)^\comp = \pi_{\alpha}^{-1}(F^\comp)$
                  \item[\bullet] $\bigcup\limits_{n \in \mathbb{N}}\pi_{\alpha}^{-1}(E_n) = \left(\pi_{\alpha}^{-1}(\bigcup\limits_{n \in \mathbb{N}}E_n)\right)$\retTwo
               \end{itemize}
            \end{myIndent}

            Also, for any $E \subseteq X_\alpha$, $E \in \mathcal{E}_\alpha \Longrightarrow \pi_\alpha^{-1}(E) \in \mathcal{M}(\mathcal{F})$.\retTwo
         \end{myIndent}

         By definition, we thus know that $\mathcal{M}_\alpha \subseteq \mathcal{F}_\alpha$. So for all $\alpha \in A$ and $E_\alpha \in \mathcal{M}_\alpha$, we know that $E_\alpha \in \mathcal{F}_\alpha$, which means that $\pi_\alpha^{-1}(E_\alpha) \in \mathcal{M}(\mathcal{F})$. So\\ $\bigotimes\limits_{\alpha \in A} \mathcal{M}_\alpha \subseteq \mathcal{M}(\mathcal{F})$.\retTwo
      \end{myIndent}

      (iii) We can also combine the first two parts of this proposition. If $A$ is countable and $\mathcal{M}_\alpha = \mathcal{M}(\mathcal{E}_\alpha)$ for all $\alpha \in A$, then $\bigotimes\limits_{\alpha \in A}\mathcal{M}_\alpha$ is generated by:
      
      {\centering $\{\prod\limits_{\alpha \in A}E_\alpha \mid \forall \alpha \in A, \myHS E_\alpha \in \mathcal{E}_\alpha\}$\retTwo\par}
   \end{itemize}
\end{myIndent}

\mHeader{Lecture 3 Notes: 10/3/2024}

\begin{myIndent}\hTwo
   \blab{Proposition}: Let $X_1, \ldots, X_n$ be metric spaces, and define $X = \prod\limits_{i=1}^n X_i$ to be the\\ [-8pt] metric space equipped with the product metric.
   
   \begin{myIndent}\myComment
      The product metric defines the distance between any $\bm{x}, \bm{y} \in \prod\limits_{i = 1}^n$ to be the max\\ distance between a coordinate of $\bm{x}$ and the corresponding coordinate in $\bm{y}$.\retTwo
   \end{myIndent}

   \begin{itemize}
      \item $\bigotimes\limits_{i = 1}^n \mathcal{B}_{X_i} \subseteq \mathcal{B}_X$.\\ [-10pt]
      \begin{myIndent}\hThree
         Proof:\\
         By the previous proposition: $\bigotimes\limits_{i = 1}^n \mathcal{B}_{X_i}$ is generated by the collection:
         
         {\centering $\{\pi_{i}^{-1}(U_i) \mid i \in \{1, \ldots, n\} \text{ and } U_i \subseteq X_i \text{ is open}\}$.\retTwo\par}

         Also, by the definition of a product topology, we know that each $\pi_{i}^{-1}(U_i)$ is\\ open in $X$. So by the lemma on page 9, we know that $\bigotimes\limits_{i = 1}^n \mathcal{B}_{X_i} \subseteq \mathcal{B}_X$.\newpage 
      \end{myIndent}

      \item If each $X_i$ is separable, then $\bigotimes\limits_{i = 1}^n \mathcal{B}_{X_i} = \mathcal{B}_X$.\\ [-10pt]
      
      \begin{myIndent}\hThree
         Proof:\\
         Let $C_i \subseteq X_i$ be countable with $\overline{C_i} = X_i$ for all $i \in \{1, \ldots, n\}$. Then define $\mathcal{E}_i = \{B(p, x) \mid x \in C_i \text{ and } p \in \mathbb{Q}_+\}$ for each $i$. Since $\mathcal{E}_i$ is countable and all open sets in $X_i$ are the union of a subcollection of $\mathcal{E}_i$, we know that any open set in $X_i$ is also in $\mathcal{M}(\mathcal{E}_i)$. So, $\mathcal{B}_{X_i} \subseteq \mathcal{M}(\mathcal{E}_i)$.  And since $\mathcal{E}_i$ contains only open sets of $X_i$, the reverse inclusion holds too.\retTwo

         Also, $C = \prod\limits_{i=1}^n C_i$ is a countable dense subset of $X$.\retTwo Defining $\mathcal{E} = \{B(p, \bm{x}) \mid \bm{x} \in C \text{ and } p \in \mathbb{Q}_+\}$, we have that $\mathcal{E}$ is countable and any open set in $X$ is also in $\mathcal{M}(\mathcal{E})$. So, $\mathcal{B}_{X} \subseteq \mathcal{M}(\mathcal{E})$. And like before since $\mathcal{E}$ contains only open sets of $X$, the reverse inclusion holds too. \retTwo

         But now note that given, $B(p, (x_1, \ldots, x_n)) \in \mathcal{E}$, we know that\\ $B(p, (x_1, \ldots, x_n)) = \prod\limits_{i = 1}^n B(p, x_i)$ where $(p, x_i) \in \mathcal{E}_i$ for all $i$.\retTwo
         
         So applying part 3 of the previous proposition and the lemma on page 9:

         {\centering\fontsize{12}{14}\selectfont $\mathcal{B}_X = \mathcal{M}(\mathcal{E}) \subseteq \mathcal{M}\left(\prod\limits_{i = 1}^n E_i \mid E_i \in \mathcal{E}_i \text{ for all } i\right) = \bigotimes\limits_{i=1}^n \mathcal{M}(\mathcal{E}_i) = \bigotimes\limits_{i = 1}^n \mathcal{B}_{X_i}$ \retTwo\par}
      \end{myIndent}
   \end{itemize}

   \blab{Corollary:} $\mathcal{B}_{\mathbb{R}^n} = \bigotimes\limits_{i = 1}^n \mathcal{B}_\mathbb{R}$.

   \begin{myIndent}\hThree
      This is because the product metric $\rho_1$ of $\prod\limits_{i=1}^n \mathbb{R}$ is \udefine{equivalent} to the standard metric\\ [-8pt] $\rho_2$ of $\mathbb{R}^n$, meaning that: 
      
      {\centering $\exists C, C^\prime > 0$ such that $C\rho_1 \leq \rho_2 \leq C^\prime\rho_1$.\retTwo\par}

      
      \begin{myIndent}\hFour
         In the specific case of this corollary, set $C = \sqrt{\sfrac{1}{n}}\hspace{0.2em}$ and $C^\prime = 1$.\retTwo
      \end{myIndent}

      The fact relevant here is that given the metrics $\rho_1, \rho_2$ on a set $X$, if $\rho_1$\\ is equivalent to $\rho_2$, then $(X, \rho_1)$ and $(X, \rho_2)$ have the same open sets (this\\ is really trivial to prove).\retTwo
   \end{myIndent}
\end{myIndent}

\mySepTwo

An \udefine{elementary family} is a collection $\mathcal{E}$ of subsets of a set $X$ such that:

\begin{enumerate}
   \item $\emptyset \in \mathcal{E}$
   \item If $E, F \in \mathcal{E}$, then $E \cap F \in \mathcal{E}$.
   \item If $E \in \mathcal{E}$, then $E^\comp$ is a finite disjoint union of members of $\mathcal{E}$.
\end{enumerate}

\newpage

\begin{myIndent}\hTwo
   If $\mathcal{E}$ is an elementary collection, then $\mathcal{A}$ equal to the collection of finite disjoint\\ unions of $\mathcal{E}$ is an algebra.\retTwo

   \begin{myIndent}\hThree
      Proof:\\
      Firstly, given any $A, B \in \mathcal{E}$, we have that $A \cup B = (A - B) \cup B$. Also,\\ [-2pt] by property 3 of elementary families, $(A - B) = (A \cap B^\comp) = (A \cap \bigcup\limits_{i=1}^k C_i)$\\ where each $C_i \in \mathcal{E}$ and disjoint. By property 2 of elementary families, we thus\\ [4pt] know $A \cap C_i \in \mathcal{E}$ for all $i$. So $(A - B)$ is a finite union of disjoint sets in $\mathcal{E}$. In turn,\\ [4pt] so is $(A - B) \cup B$. Hence, $A \cup B \in \mathcal{A}$.\retTwo

      By induction, we get that for any $A_1, \ldots, A_n \in \mathcal{E}$,\myHS $A_1 \cup \ldots \cup A_n$ is a finite\\ union of disjoint sets in $\mathcal{E}$. So $\mathcal{A}$ actually equals the set of all finite unions of $\mathcal{E}$. It\\ follows that $\mathcal{A}$ is closed under finite unions.\retTwo

      I really don't want to write down the proof that $\mathcal{A}$ is closed under complements. It's what you would expect but just heavy on notation.
   \end{myIndent}
\end{myIndent}

\exOne\mySepTwo 

\blab{Exercise 1.5:} If $\mathcal{M}$ is the $\sigma$-algebra generated by $\mathcal{E}$, then $\mathcal{M}$ is the union of the $\sigma$-algebras generated by $\mathcal{F}$ as $\mathcal{F}$ ranges over all countable subsets of $\mathcal{E}$.

\begin{myIndent}\exTwoP
   For the sake of convenience, I will write the union of $\sigma$-algebras generated by\\ countable subsets of $\mathcal{E}$ as: $\bigcup_\mathcal{F} \mathcal{M}(\mathcal{F})$.\retTwo

   To start, since each $\mathcal{M}(\mathcal{F}) \subseteq \mathcal{M}(\mathcal{E})$, we trivially know $\bigcup_\mathcal{F}\mathcal{M}(\mathcal{F}) \subseteq \mathcal{M}(\mathcal{E}) = \mathcal{M}$.\\ On the other hand, $\mathcal{E} \subseteq \bigcup_\mathcal{F}\mathcal{M}(\mathcal{F})$ since each countable $\mathcal{F} \subseteq \mathcal{E}$ is contained in\\ $\mathcal{M}(\mathcal{F}) \subseteq \bigcup_\mathcal{F}\mathcal{M}(\mathcal{F})$. So, if we can show that $\bigcup_\mathcal{F}\mathcal{M}(\mathcal{F})$ is a $\sigma$-algebra, then we will know that: $\mathcal{M} = \mathcal{M}(\mathcal{E}) \subseteq \bigcup_\mathcal{F}\mathcal{M}(\mathcal{F})$.\retTwo

   Fortunately, it's trivial to show that $\bigcup_\mathcal{F}\mathcal{M}(\mathcal{F})$ is closed under complements.\\ Given any $E \in \bigcup_\mathcal{F}\mathcal{M}(\mathcal{F})$, we know there exists $\mathcal{M}(\mathcal{F})$ with $E \in \mathcal{M}(\mathcal{F})$.\\ Then $E^\comp \in \mathcal{M}(\mathcal{F}) \subseteq \bigcup_\mathcal{F}\mathcal{M}(\mathcal{F})$. \retTwo
   
   Meanwhile, the proof that $\bigcup_\mathcal{F}\mathcal{M}(\mathcal{F})$ is closed under countable unions is\\ more involved:\\ [-9pt]
   \begin{myIndent}\exPPP
      Suppose $\{E_n\}_{n \in \mathbb{N}}$ is a countable collection of sets in $\bigcup_\mathcal{F}\mathcal{M}(\mathcal{F})$. Then for each\\ $n \in \mathbb{N}$, there exists $\mathcal{F}_n$ such that $E_n \in \mathcal{M}(\mathcal{F}_n)$. Importantly, $\bigcup\limits_{n \in \mathbb{N}}\mathcal{F}_n$ is still\\ [-7pt] countable. So, setting $\mathcal{F}^\prime = \bigcup\limits_{n \in \mathbb{N}}\mathcal{F}_n$, we have that:

      {\centering $\mathcal{M}(\mathcal{F}^\prime) \subseteq \bigcup_\mathcal{F}\mathcal{M}(\mathcal{F})$ \retTwo\par}

      Since $\mathcal{F}_n \subseteq \mathcal{F}^\prime$ for all $n$, we know that $\mathcal{M}(\mathcal{F}_n) \subseteq \mathcal{M}(\mathcal{F}^\prime)$ for all $n$. So, $\{E_n\}_{n \in \mathbb{N}}$\\ is contained in $\mathcal{M}(\mathcal{F}^\prime)$. It follows that $\bigcup\limits_{n \in \mathbb{N}}E_n \in \mathcal{M}(\mathcal{F}^\prime) \subseteq \bigcup_\mathcal{F}\mathcal{M}(\mathcal{F})$.\retTwo
   \end{myIndent}
\end{myIndent}

\mySepTwo
   
\newpage\hOne

Let $X \neq \emptyset$ and $\mathcal{M}$ be a $\sigma$-algebra on $X$. A \udefine{measure} $\mu: \mathcal{M} \longrightarrow [0, \infty]$ is a function satisfying that:
\begin{itemize}
   \item $\mu(\emptyset) = 0$
   \item $\mu(\bigcup\limits_{j=1}^\infty E_j) = \sum\limits_{j=1}^\infty \mu(E_j)$ if $E_j \in \mathcal{M}$ for all $j$ and $E_j \cap E_i = \emptyset$ for all $i \neq j$\retTwo
\end{itemize}

$(X, \mathcal{M})$ is called a \udefine{measurable space} and $(X, \mathcal{M}, \mu)$ is called a \udefine{measure space}.\retTwo

Let $(X, \mathcal{M}, \mu)$ be a measure space.
\begin{itemize}
   \item $\mu$ is called \udefine{finite} if $\mu(X) < \infty$.
   
   \begin{myIndent}\teachComment
      It follows if $\mu$ is finite that $\mu(E) < \infty$ for all $E \in \mathcal{M}$ since $E \subseteq X$.\\
      In probability theory, most measure spaces are finite.\retTwo
   \end{myIndent}

   \item $\mu$ is called \udefine{$\sigma$-finite} if $X = \bigcup\limits_{j=1}^\infty E_j$, such that $E_j \in \mathcal{M}$ and $\mu(E_j) < \infty$ for all $j$.\retTwo
   
   \item $\mu$ is called \udefine{semifinite} if $\forall E \in \mathcal{M}$ with $\mu(E) = \infty$, there exists $F \subset E$ such that $F \in \mathcal{M}$, and $ 0 < \mu(F) < \infty$.\retTwo
\end{itemize}


\begin{myIndent}\exOne
   Example: Let $X \neq \emptyset$ and $\mathcal{M} = \mathcal{P}(X)$. Then given a function $\rho: X \longrightarrow [0, \infty]$,\\ $\mu(E) = \sum\limits_{x \in E}\rho(x)$ is a measure.
   
   \begin{itemize}
      \item $\mu$ is semifinite if and only if $\rho(x) < \infty$ for all $x \in X$.
      \item $\mu$ is $\sigma$-finite if and only if it is semifinite and $\{x \in X \mid \rho(x) > 0\}$ is countable.
   \end{itemize}

   If $\rho(x) = 1$ for all $x$, then $\mu$ is called the \udefine{counting measure}.\\
   If $\rho(x) = \left\{
   \begin{matrix}
      1 & \text{ if } x = x_0 \\
      0 & \text{ if } x \neq x_0,
   \end{matrix}\right.$ then $\mu$ is called the \udefine{Dirac measure} at $x_0$: $\delta_{x_0}$.\retTwo

   \hTwo 
   \blab{Theorem:} Let $(X, \mathcal{M}, \mu)$ be a measure space. Then:
   \begin{enumerate}
      \item If $E, F \in \mathcal{M}$ with $E \subseteq F$, then $\mu(E) \leq \mu(F)$.
      \item If $(E_j)_{j\in\mathbb{N}} \subseteq \mathcal{M}$, then $\mu(\bigcup\limits_{j=1}^\infty E_j) \leq \sum\limits_{j = 1}^\infty \mu(E_j)$.\\ [-10pt]
      \item If $(E_j)_{j \in \mathbb{N}} \subseteq \mathcal{M}$ with $E_j \subseteq E_{j+1}$ for all $j \in \mathbb{N}$, then $\mu(\bigcup\limits_{j=1}^\infty E_j) = \lim\limits_{j \rightarrow \infty}\mu(E_j)$.
      \item If $(E_j)_{j \in \mathbb{N}} \subseteq \mathcal{M}$ and $\mu(E_1) < \infty$ and $E_{j+1} \subseteq E_j$ for all\\ [-6pt] \phantom{aaaaaaaaaaaaaaaaaaaaaaaaaaaaaaaaaa} $j \in \mathbb{N}$, then $\mu(\bigcap\limits_{j=1}^\infty E_j) = \lim\limits_{j\rightarrow \infty}\mu(E_j)$.
      
      \begin{myIndent}\hThree
         Proofs:\\
         (1) Suppose $E, F \in \mathcal{M}$ with $E \subseteq F$. Then $F = (F - E) \cup E$ is a disjoint union of sets in $\mathcal{M}$, meaning $\mu(F) = \mu(F - E) + \mu(E) \geq \mu(E)$.\newpage

         (2) Set $F_1 = E_1$ and $F_m = E_m - \bigcup\limits_{i=1}^{m-1}E_i$ for all $m > 1$. Then $(F_i)_{i \in \mathbb{N}}$ is\\ [-6pt] pairwise disjoint and $\bigcup\limits_{i=1}^\infty F_i = \bigcup\limits_{i=1}^\infty E_i$. So $\mu(\bigcup\limits_{i=1}^\infty E_i) = \sum\limits_{i=1}^\infty\mu(F_i)$. On the other hand, $F_i \subseteq E_i$ for all $i$. So $\sum\limits_{i=1}^\infty\mu(F_i) \leq \sum\limits_{i=1}^\infty\mu(E_i)$\retTwo\retTwo

         (3) Setting $E_0 = \emptyset$, we have that $\mu(\bigcup\limits_{i=1}^\infty E_i) = \sum\limits_{i=1}^\infty\mu(E_i - E_{i-1})$. Also,\\ [-6pt] $\mu(E_n) = \sum\limits_{i=1}^n\mu(E_i - E_{i-1})$. So:\\ [-8pt]
         
         {\centering$\lim\limits_{n \rightarrow \infty}{\mu(E_n)} = \lim\limits_{n \rightarrow \infty}{\sum\limits_{i=1}^n\mu(E_i - E_{i-1})} = \sum\limits_{i=1}^\infty \mu(E_i - E_{i-1}) = \mu(\bigcup\limits_{i=1}^\infty E_i)$.\retTwo\retTwo\par}

         (4) Let $F_j = E_1 - E_j$ for all $j \in \mathbb{N}$. Then for all $j \in \mathbb{N}$,\myHS $F_j \subseteq F_{j+1}$,\\ $\mu(E_1) = \mu(F_j) + \mu(E_j)$, and $\bigcup\limits_{j=1}^\infty F_j = E_1 - \bigcap\limits_{j = 1}^\infty E_j$. We can thus\\ conclude that:

         {\centering\begin{tabular}{l}
            $\mu(E_1) = \mu(\bigcap\limits_{j=1}^\infty E_j) + \mu(\bigcup\limits_{j=1}^\infty F_j)$\\ [6pt]
            $\phantom{\mu(E_1)} = \mu(\bigcap\limits_{j=1}^\infty E_j) + \lim\limits_{j \rightarrow \infty}(\mu(F_j)) = \mu(\bigcap\limits_{j=1}^\infty E_j) + \lim\limits_{j \rightarrow \infty}(\mu(E_1) - \mu(E_j))$
         \end{tabular} \retTwo\par}

         Since $\mu(E_1) < \infty$, we can subtract it out of the expression to get:\\ [4pt] $\mu(\bigcap\limits_{j=1}^\infty E_j) - \lim\limits_{j\rightarrow\infty}(\mu(E_j)) = 0$. Also, we know $\mu(\bigcap\limits_{j=1}^\infty E_j) < \infty$ since\\ [0pt] it's a subset of $E_j$. So, we can rearrange to get: $\mu(\bigcap\limits_{j=1}^\infty E_j) = \lim\limits_{j\rightarrow\infty}(\mu(E_j))$.
      \end{myIndent}
   \end{enumerate}
\end{myIndent}

\exOne\mySepTwo

\blab{Exercise 1.9:} If $(X, \mathcal{M}, \mu)$ is a measure space and $E, F \in \mathcal{M}$, then we have that\\ $\mu(E) + \mu(F) = \mu(E \cup F) + \mu(E \cap F)$.

\begin{myIndent}\exTwoP
   We know $\mu(E) = \mu(E - f) + \mu(E \cap F)$ and $\mu(F) = \mu(F - E) + \mu(F \cap E)$.\\ Adding those equations together we get that:

   {\centering
   \begin{tabular}{l}
       $\mu(E) + \mu(F) = (\mu(E - F) + \mu(E \cap F) + \mu(F - E)) + \mu(E \cap F)$\\ $\phantom{\mu(E) + \mu(F) = (\mu(E - F) + \mu(E \cap F)) + a} = \mu(E \cup F) + \mu(E \cap F)$.
   \end{tabular}\retTwo\par}
\end{myIndent}

\blab{Exercise 1.14:} If $\mu$ is a semifinite measure and $\mu(E) = \infty$, then for any $C > 0$ there exists $F \subset E$ in $\mathcal{M}$ with $C < \mu(F) < \infty$.


\begin{myIndent}\exTwoP
   Let $S$ be the set of $C > 0$ for which there exists $F \subset E$ in $\mathcal{M}$ with $C < \mu(F) < \infty$. By the definition of semifiniteness, we know $S$ isn't empty. Meanwhile, if for some $C$ we had that there didn't exist a set $F \subset E$ in $\mathcal{M}$ with $C < \mu(F) < \infty$, then we'd know that $S$ is bounded above. Hence, we'd know there exists $\alpha = \sup(S)$.\newpage

   Now firstly, for all $n \in \mathbb{N}$, choose $G_n \subset E$ in $\mathcal{M}$ such that $\alpha - \frac{1}{n} < \mu(G_n) < \infty$. After that, define $F_n = \bigcup\limits_{i=1}^n G_i$ for all $n \in \mathbb{N}$. Since $\mathcal{M}$ is closed under finite unions,\\ [1pt] we know each $F_n$ is in  $\mathcal{M}$. So then observe:
   \begin{enumerate}
      \item $F_n \subseteq F_{n + 1}$ for all $n \in \mathbb{N}$
      \item For each $n \in \mathbb{N}$,\myHS $\alpha - \frac{1}{n} < \mu(F_n) \leq \alpha$
      \begin{myIndent}\exPPP
         This is because for each $n \in \mathbb{N}$,\myHS  $\mu(F_n) < \sum\limits_{i = 1}^n \mu(G_i)$ which is a finite sum of\\ [-2pt] finite quantities. So $\mu(F_n) < \infty$. At the same time, $F_n \subset E$ since each $G_i$ is\\ [5pt] a subset of $E$ (we know it is a proper subset because it has a different measure\\ [5pt] than $E$). So, if $\mu(F_n) > \alpha$, then $\frac{1}{2}(\mu(F_n) + \alpha)$ would be an element of\\ [5pt] $S$ greater than $\alpha$, thus contradicting that $\alpha = \sup(S)$. As for the other\\ [5pt] inequality, note that $G_n \subseteq F_n$. Thus $\mu(F_n) \geq \mu(G_n) > \alpha - \frac{1}{n}$.\retTwo
      \end{myIndent}

      Now $\bigcup\limits_{n = 1}^\infty F_n \in \mathcal{M}$ due to $\mathcal{M}$ being closed under countable sums. Also, by\\ [-6pt] the two observations above, we know $\mu(\bigcup\limits_{n = 1}^\infty F_n) = \lim\limits_{n \rightarrow \infty}\mu(F_n) = \alpha$. And finally,\\ [-6pt] note that $\bigcup\limits_{n = 1}^\infty F_n$ is a proper subset of $E$ (we know this because each $F_n \subset E$\\ [-6pt] and $\bigcup\limits_{n = 1}^\infty F_n$ can't equal $E$ since their measures are different).\retTwo

      So, we have now proven the existence of a set $F \in \mathcal{M}$ such that $F \subset E$ and\\ $\mu(F) = \alpha$. But now note that $\mu(E - F)$ must be infinite since:
      
      {\centering $\mu(E - F) + \alpha = \mu(E - F) + \mu(F) = \mu(E) = \infty$.\retTwo\par}

      Because $\mu$ is semifinite, there exists $F^\prime \subset E - F$ in $\mathcal{M}$ with $0 < \mu(F^\prime) < \infty$.\\ But because $F$ and $F^\prime$ are disjoint subsets of $E$ in $\mathcal{M}$, we know $F \cup F^\prime \in \mathcal{M}$ and $\mu(F \cup F^\prime) = \mu(F) + \mu(F^\prime) > \alpha$. Plus $F \cup F^\prime$ is a proper subset of $E$. (It can't equal $E$ because it's measure isn't equal to $E$. But, both $F$ and $F^\prime$ individually are subsets of $E$.)\retTwo

      Hence, we have that $\frac{1}{2}(\alpha, \mu(F) + \mu(F)^\prime)$ is an element of $S$ greater than $\alpha$, thus contradicting that $\alpha$ was the supremum of $S$. We conclude therefore that $\alpha$ does not exist, meaning $S$ is unbounded. 
   \end{enumerate}
\end{myIndent}

\mySepTwo

\hOne Given a measure space $(X, \mathcal{M}, \mu)$, a set $E \in \mathcal{M}$ satisfying that $\mu(E) = 0$ is called a \udefine{null set} (or $\mu$-null set if we want more precision).\retTwo

By subadditivity (a.k.a. the fact that for all $(E_j)_{j \in \mathbb{N}} \subset\mathcal{M},\myHS \mu(\bigcup\limits_{j=1}^\infty E_j) \leq \sum\limits_{j = 1}^\infty \mu(E_j)$),\\ [-8pt] we know countable unions of null sets are also null sets.\newpage

Given a proposition $P(x)$, if there exists a null set $E \in \mathcal{M}$ satisfying that $P(x)$ is true for all $x \in X - E$, then we say $P$ is true \udefine{almost everywhere} (abbreviated as $\mu$-a.e. or just a.e. if the measure being used is clear).\retTwo

A measure space is \udefine{complete} if given any $E \subseteq X$, we have that $N \in \mathcal{M}$ with\\ $\mu(N) = 0$ and $E \subseteq N$ implies that $E \in \mathcal{M}$.

\begin{myIndent}\hTwo
   \blab{Proposition:} Suppose $(X, \mathcal{M}, \mu)$ is a measure space. Let:
   \begin{itemize}
      \item $\mathcal{N} = \{N \in \mathcal{M} \mid \mu(N) = 0\}$
      \item $\overline{\mathcal{M}} = \{E \cup F \mid E \in \mathcal{M} \text{ and } F \subseteq N \text{ where } N \in \mathcal{N}\}$.\retTwo
   \end{itemize}

   Then $\overline{\mathcal{M}}$ is a $\sigma$-algebra and there is a unique extension $\overline{\mu}$ of $\mu$ to a complete measure on $\overline{\mathcal{M}}$.
   
   \begin{myIndent}\hThree
      Proof:\\
      Claim 1: $\overline{\mathcal{M}}$ is a $\sigma$-algebra.
      \begin{myIndent}\hFour
         To see that $\overline{\mathcal{M}}$ is closed under countable union, let $(E_i \cup F_i)_{i \in \mathbb{N}}$ be a sequence of sets in $\overline{\mathcal{M}}$ with each $E_i \in \mathcal{M}$ and $F_i \subseteq N_i$ for some $N_i \in \mathcal{N}$. Then\\ $\bigcup\limits_{i \in \mathbb{N}}(E_i \cup F_i) = \bigcup\limits_{i \in \mathbb{N}}E_i \cup \bigcup\limits_{i \in \mathbb{N}}F_i$.\retTwo

         Importantly, since $\mathcal{M}$ and $\mathcal{N}$ are closed under countable union, we know that $\bigcup\limits_{i \in \mathbb{N}}E_i \in \mathcal{M}$ and $\bigcup\limits_{i \in \mathbb{N}}F_i \subseteq \bigcup\limits_{i \in \mathbb{N}}N_i \in \mathcal{N}$. So, $\bigcup\limits_{i \in \mathbb{N}}(E_i \cup F_i) \in \overline{\mathcal{M}}$.\retTwo

         To show that $\overline{\mathcal{M}}$ is closed under complements, let $E \cup F \in \overline{\mathcal{M}}$ with\\ $E \in \mathcal{M}$ and $F \subseteq N$ for some $N \in \mathcal{N}$. Also note that we can assume\\ $E \cap N = \emptyset$. After all, if $E \cap N \neq \emptyset$, then define $N^\prime = N - E$ and\\ $F^\prime = F - E$. Since $N^\prime \subseteq N$ and $N^\prime \in \mathcal{M}$, we know that $\mu(N^\prime) = 0$. Also, $E \cup F = E \cup F^\prime$ with $F^\prime \subseteq N^\prime$. So, $E$, $F^\prime$, and $N^\prime$ fulfil the same properties we picked $E$, $F$, and $N$ for having. But also $E \cap N^\prime = \emptyset$.\retTwo

         Now, $(E \cup F)^\comp = (E \cup N)^\comp \cup (N - F)$ where $(E \cup N)^\comp \in \mathcal{M}$ and\\ $(N - F) \subset N$. So $(E \cup F)^\comp \in \overline{\mathcal{M}}$.\retTwo
      \end{myIndent}

      Now given any $E \cup F \in \overline{\mathcal{M}}$ with $E \in \mathcal{M}$ and $F \subset N$ for some $N \in \mathcal{N}$, define $\overline{\mu}(E \cup F) = \mu(E)$.\retTwo
      Claim 2: $\overline{\mu}$ is well-defined.
      \begin{myIndent}\hFour
         Suppose $E_1 \cup F_1 = E_2 \cup F_2$ where for $j \in \{1, 2\}$ we have $E_j \in \mathcal{M}$\\ and $F_j \subset N_j$ for some $N_j \in \mathcal{N}$. Then $E_1 \subseteq E_2 \cup N_2$, meaning that\\ $\mu(E_1) \leq \mu(E_2) + \mu(N_2) = \mu(E_2)$. By similar reasoning, we can say that $\mu(E_2) \leq \mu(E_1)$. So $\overline{\mu}(E_1 \cup F_1) = \overline{\mu}(E_2 \cup F_2)$.
         \retTwo
      \end{myIndent}

      \exP (The rest is exercise 1.6:)\\
      Claim 3: $\overline{\mu}$ is a complete measure on $\overline{\mathcal{M}}$.
      \begin{myIndent}\exPP
         It's easy to show that $\overline{\mu}$ is a measure. After all, $\emptyset \in \mathcal{M} \cap \mathcal{N}$. So,\\ $\overline{\mu}(\emptyset) = \overline{\mu}(\emptyset \cup \emptyset) = \mu(\emptyset) = 0$. Also, suppose $(E_i \cup F_i)_{i \in \mathbb{N}}$ is a\\ sequence of disjoint sets in $\overline{\mathcal{M}}$ with $E_i \in \mathcal{M}$ and $F_i \subset N_i$ for some $N_i \in \mathcal{N}$.\newpage

         Then $\bigcup\limits_{i \in \mathbb{N}}E_i \in \mathcal{M}$ where each $E_i$ is disjoint and $\bigcup\limits_{i \in \mathbb{N}}F_i \subseteq \bigcup\limits_{i \in \mathbb{N}}N_i \in \mathcal{N}$. So:

         {\center 
         \begin{tabular}{l}
            $\overline{\mu}(\bigcup\limits_{i \in \mathbb{N}}(E_i \cup F_i)) = \overline{\mu}(\bigcup\limits_{i \in \mathbb{N}}E_i \cup \bigcup\limits_{i \in \mathbb{N}}F_i)$\\ [12pt]
             $\phantom{\overline{\mu}(\bigcup\limits_{i \in \mathbb{N}}(E_i \cup F_i))} = \mu(\bigcup\limits_{i \in \mathbb{N}}E_i) = \sum\limits_{i = 1}^\infty\mu(E_i) = \sum\limits_{i = 1}^\infty\overline{\mu}(E_i \cup F_i)$
         \end{tabular}\retTwo\par}

         Finally, to show that $(X, \overline{\mathcal{M}}, \overline{\mu})$ is complete, suppose $A \subseteq X$ and $N_1 \in \overline{\mathcal{M}}$ with $\overline{\mu}(N_1) = 0$ and $A \subseteq N_1$. By definition, we know $N_1 = E \cup F$ where $E \in \mathcal{M}$ and $F \subseteq N_2$ for some $N_2 \in \mathcal{N}$. However, we can also assume $E = \emptyset$. For if $E \neq \emptyset$, then because $\mu(E \cup N_2) \leq \mu(E) + \mu(N_2) \leq 0$, we can define $N_2^\prime = E \cup N_2$ and $F^\prime = E \cup F$. Then $N_1 = \emptyset \cup F^\prime$ and\\ $F^\prime \subseteq N_2^\prime$ where $N_2^\prime \in \mathcal{N}$.\retTwo

         So $A \subset F \subset N_2$ where $N_2 \in \mathcal{N}$. It follows that $A = \emptyset \cup A \in \overline{\mathcal{M}}$.\retTwo
      \end{myIndent}

      Claim 4: $\overline{\mu}$ is the unique measure on $\overline{\mathcal{M}}$ that extends $\mu$.\\ [-9pt]
      \begin{myIndent}\exPP
         Suppose $\overline{\overline{\mu}}$ is another measure on $\overline{\mathcal{M}}$ such that $\overline{\overline{\mu}}|_\mathcal{M} = \mu$. Then consider any $E \cup F \in \overline{M}$ such that $E \in \mathcal{M}$ and $F \subseteq N$ for some $N \in \mathcal{N}$. As shown before, we can assume without loss of generality that $E \cap N = \emptyset$ and thus also $E \cap F = \emptyset$. So, we have that:

         {\centering $\overline{\overline{\mu}}(E \cup F) = \overline{\overline{\mu}}(E) + \overline{\overline{\mu}}(F)$\retTwo\par}

         Next, note that:

         {\centering\fontsize{11}{13}\selectfont $\mu(E) = \overline{\overline{\mu}}(E) \leq \overline{\overline{\mu}}(E) + \overline{\overline{\mu}}(F) \leq \overline{\overline{\mu}}(E) + \overline{\overline{\mu}}(N) = \mu(E) + \mu(N) = \mu(E)$ \retTwo\par}

         Hence, we know that $\overline{\overline{\mu}}(E \cup F) = \mu(E)$. But also $\overline{\mu}(E \cup F) = \mu(E)$. So $\overline{\overline{\mu}}(E \cup F) = \overline{\mu}(E \cup F)$ for all $E \cup F \in \overline{\mathcal{M}}$.\retTwo
      \end{myIndent}
   \end{myIndent}

   Note: We call $\overline{\mu}$ the \udefine{completion} of $\mu$ and $\overline{\mathcal{M}}$ the \udefine{completion} of $\mathcal{M}$ with respect\\ to $\mu$.
\end{myIndent}

\mySepTwo

\mHeader{Lecture 4 Notes: 10/8/2024}

An \udefine{outer measure} on a nonempty set $X$ is a function $\mu^*: \mathcal{P}(X) \longrightarrow [0, \infty]$\\ satisfying that:
\begin{enumerate}
   \item $\mu^*(\emptyset) = 0$.
   \item $\mu^*(A) \leq \mu^*(B)$ if $A \subseteq B$.
   \item $\mu^*(\bigcup\limits_{j = 1}^\infty A_j) \leq \sum\limits_{j=1}^\infty \mu^*(A_j)$. {\teachComment (this property is called subadditivity)}\newpage
\end{enumerate}

\begin{myIndent}\hTwo
   \blab{Proposition:} Let $\mathcal{E} \subseteq \mathcal{P}(X)$ be a collection of elementary sets and $\mu: \mathcal{E} \longrightarrow [0, \infty]$ be a function satisfying that $\mu(\emptyset) = 0$.
   \begin{myIndent}\teachComment
      The textbook only assumes that $\emptyset$ and $X$ are in $\mathcal{E}$.
   \end{myIndent}

   Then define $\mu^*(A) = \inf\left\{\sum\limits_{j = 1}^\infty \mu(E_j)\hspace{0.2em} \middle| \hspace{0.2em} E_j \in \mathcal{E} \text{ and } A \subseteq \bigcup\limits_{j = 1}^\infty E_j\right\}$. This is an outer\\ [-9pt] measure.\retTwo

   \begin{myIndent}\hThree
      Proof:\\
      Given any set $A \subseteq \mathcal{P}(X)$, define $A^* = \left\{\sum\limits_{j = 1}^\infty \mu(E_j)\hspace{0.2em} \middle| \hspace{0.2em} E_j \in \mathcal{E} \text{ and } A \subseteq \bigcup\limits_{j = 1}^\infty E_j\right\}$.\\ [-9pt] That way $\mu^*(A) = \inf(A^*)$.
      \begin{enumerate}
         \item $\mu^*(A)$ is well defined because $\mu(X) \in A^*$ and $\mu(\emptyset) = 0$ is a lower bound\\ of $A^*$.\\[ -8pt]
         \item Since $0 \in \emptyset^*$, we know that $\mu^*(\emptyset) = \inf(\emptyset^*) = 0$.\\ [-8pt]
         \item Suppose $A \subseteq B$. Then given any $(E_j)_{j \in \mathbb{N}}$ of sets in $\mathcal{E}$ covering $B$, we know\\ that they will also cover $A$. So, $A^* \subseteq B^*$, meaning $\mu^*(A) \leq \mu^*(B)$.\\ [-8pt]
         \item Suppose $(A_j)_{j \in \mathbb{N}} \subseteq \mathcal{P}(X)$. Then fix $\varepsilon > 0$. For all $j \in \mathbb{N}$, let $(E_j^{(k)})_{k \in \mathbb{N}}$ be a sequence of sets in $\mathcal{E}$ such that $A_j \subseteq \bigcup\limits_{k \in \mathbb{N}}E_j^{(k)}$ and:
         
         {\centering$\mu^*(A_j) \leq \sum\limits_{k = 1}^\infty \mu(E_j^{(k)}) \leq \mu^*(A_j) + \sfrac{\varepsilon}{2^j}$.\retTwo\par}

         Note that $\bigcup\limits_{j \in \mathbb{N}}A_j \subseteq \bigcup\limits_{j \in \mathbb{N}}(\bigcup\limits_{k \in \mathbb{N}}E_j^{(k)})$.\retTwo
         
         So, $(\bigcup\limits_{j \in \mathbb{N}}A_j)^* \ni \sum\limits_{j = 1}^\infty \sum\limits_{k = 1}^\infty\mu(E_j^{(k)}) \leq \sum\limits_{j = 1}^\infty (\mu^*(A_j) + \sfrac{\varepsilon}{2^j}) = \varepsilon + \sum\limits_{j = 1}^\infty \mu^*(A_j)$.\retTwo

         Since $\varepsilon$ was arbitrary, we thus know that:
         
         {\centering $\mu^*(\bigcup\limits_{j \in \mathbb{N}}A_j) = \inf (\bigcup\limits_{j \in \mathbb{N}}A_j)^* \leq \sum\limits_{j = 1}^\infty \mu^*(A_j)$.\retTwo\par}
      \end{enumerate}
   \end{myIndent}
\end{myIndent}

Let $\mu^*$ be an outer measure on a nonempty set $X$. Then $A \subseteq X$ is called\\ \udefine{$\mu^*$-measurable} if $\mu^*(E) = \mu^*(E \cap A) + \mu^*(E - A)$ for all $E \subseteq X$.
\begin{myIndent}\teachComment
   Note that we trivially have $\mu^*(E) \leq \mu^*(E \cap A) + \mu^*(E - A)$ for all $E \in \mathcal{P}(X)$. Also, $\mu^*(E \cap A) + \mu^*(E - A) \leq \mu^*(E)$ holds trivially if $\mu^*(E)$. So $\mu^*$-measurability just means that $\mu^*(E \cap A) + \mu^*(E - A) \leq \mu^*(E)$ holds even if $\mu^*(E) < \infty$.\retTwo

   \hTwo\blab{Carathéodory's Theorem:} If $\mu^*$ is an outer measure on $X$, then the collection $\mathcal{M}$ of $\mu^*$-measurable sets is a $\sigma$-algebra and the restriction of $\mu^*$ to $\mathcal{M}$ is a complete measure.
   
   \begin{myIndent}\hThree
      Proof:\\
      Part 1: $\mathcal{M}$ is an algebra and $\mu^*$ is additive on $\mathcal{M}$ (meaning $A, B \in \mathcal{M}$ with\\ $A \cap B = \emptyset$ implies that $\mu^*(A \cup B) = \mu^*(A) + \mu^*(B))$.
      
      \begin{myIndent}\hFour
         We know $\mathcal{M}$ is an algebra because:
         \begin{itemize}
            \item $\emptyset \in \mathcal{M}$ because $\mu^*(E) = \mu^*(E \cap \emptyset) + \mu^*(E - \emptyset) = 0 + \mu^*(\emptyset)$ for\\ all $E \subseteq X$.\newpage
            \item Both $A^\comp \in \mathcal{M}$ and $A \in \mathcal{M}$ are equivalent to us having for all $E \subseteq X$ that $\mu^*(E) = \mu^*(E \cap A) + \mu^*(E \cap A^\comp)$.\\ [-6pt]
            \item Suppose $A$ and $B$ are sets in $\mathcal{M}$. Then given $E \subseteq X$, we have:
            
            {\centering 
            \begin{tabular}{l}
               $\mu^*(E) = \mu^*(E \cap A) + \mu^*(E - A)$\\
               $\phantom{\mu^*(E)} = \mu^*((E \cap A) \cap B) + \mu^*((E \cap A) - B)$\\ $\phantom{aaaaaaaaaaaaaaaaaa} + \mu^*((E - A) \cap B) + \mu^*((E - A) - B)$
            \end{tabular} \retTwo\par}

            Now $(E - A) - B = E \cap A^\comp \cap B^\comp = E \cap (A \cup B)^\comp$. Meanwhile,\\ $(E \cap A) - B = E \cap (A - B)$ and $(E - A) \cap B = E \cap (B - A)$.\\ So, by subadditivity, we have that:

            {\centering\begin{tabular}{l}
               $\mu^*(E \cap (A \cup B)) + \mu^*(E - (A \cup B))$\\
               $\phantom{aaaaaaaaaaaaa} \leq \mu^*((E \cap A) \cap B) + \mu^*((E \cap A) - B)$\\ $\phantom{aaaaaaaaaaaaaaaa} + \mu^*((E - A) \cap B) + \mu^*((E - A) - B)$
            \end{tabular} \retTwo\par}

            Hence, $\mu^*(E \cap (A \cup B)) + \mu^*(E - (A \cup B)) \leq \mu^*(E)$. So,\\ $A \cup B \in \mathcal{M}$.\retTwo
         \end{itemize}

         Next, to show that $\mu$ is additive on $\mathcal{M}$, consider any $A, B \in \mathcal{M}$ with\\ $A \cap B = \emptyset$. Then:

         {\centering\fontsize{11}{13}\selectfont $\mu^*(A \cup B) = \mu^*((A \cup B) \cap A) + \mu^*((A \cup B) - A) = \mu^*(A) + \mu^*(B) $ \retTwo\par}
      \end{myIndent}

      Part 2: $\mathcal{M}$ is a $\sigma$-algebra and $\mu^*$ is $\sigma$-additive (think countably additive) on $X$.

      \begin{myIndent}\hFour
         To show that $\mathcal{M}$ is a $\sigma$-algebra, it suffices to show that $\mathcal{M}$ is closed under countable disjoint unions. So let $(A_j)_{j \in \mathbb{N}}$ be a sequence of disjoint sets in $\mathcal{M}$. If $E$ is any set in $X$ and $m > 1$, then:

         {\centering $\mu^*(E) = \mu^*(E \cap (\bigcup\limits_{j=1}^mA_j)) + \mu^*(E - (\bigcup\limits_{j=1}^mA_j))$  \retTwo\par}

         But then note that because $\mathcal{M}$ is an algebra, we know $\bigcup\limits_{j=1}^mA_j \in \mathcal{M}$. So:
         
         {\centering\fontsize{11.5}{13.5}\selectfont 
         \begin{tabular}{l}
            $\mu^*(E \cap (\bigcup\limits_{j=1}^mA_j)) = \mu^*(E \cap (\bigcup\limits_{j=1}^mA_j) \cap A_m) + \mu^*(E \cap (\bigcup\limits_{j=1}^mA_j) \cap A_m^\comp)$ \\ [-2pt]
            $\phantom{\mu^*(E \cap (\bigcup\limits_{j=1}^mA_j))} = \mu^*(E \cap A_m) + \mu^*(E \cap (\bigcup\limits_{j = 1}^{m-1}\hspace{-0.2em} A_j))$
         \end{tabular} \retTwo\par}

         By induction, we thus have that $\mu^*(E \cap \bigcup\limits_{j=1}^mA_j) = \sum\limits_{j=1}^m \mu^*(E \cap A_j)$. Also, since\\ [-9pt] $E - (\bigcup\limits_{j=1}^mA_j) \supset E - (\bigcup\limits_{j\in\mathbb{N}} A_j)$, we thus know that:

         {\centering $\mu^*(E) \geq \sum\limits_{j=1}^m \mu^*(E \cap A_j) + \mu^*(E - \bigcup\limits_{j\in\mathbb{N}}A_j)$ \retTwo\par}

         Taking the limit as $m \rightarrow \infty$, we thus get that:
         
         {\centering
         \begin{tabular}{l}
             $\mu^*(E) \geq \sum\limits_{j=1}^\infty \mu^*(E \cap A_j) + \mu^*(E - \bigcup\limits_{j \in \mathbb{N}}A_j)$\\ $\phantom{aaaaaaaaaaaaaaa} \geq \mu^*(E \cap (\bigcup\limits_{j\in \mathbb{N}}A_j)) + \mu^*(E - \bigcup\limits_{j \in \mathbb{N}}A_j)\phantom{aaa}$
         \end{tabular}\newpage\par}

         So, $\bigcup\limits_{j \in \mathbb{N}}A_j$ is $\mu^*$-measurable. Hence, $\mathcal{M}$ is a $\sigma$-algebra.\retTwo

         Also, in order to show that $\mu^*(\bigcup\limits_{j \in \mathbb{N}}A_j) = \sum\limits_{j=1}^\infty\mu^*(A_j)$, just substitute\\ $E = \hspace{-0.2em}\bigcup\limits_{j \in \mathbb{N}}\hspace{-0.2em}A_j$ into the expression at the bottom of the last page.\retTwo
      \end{myIndent}

      Part 3: $(X, \mathcal{M}, \mu^*)$ is a complete measure space.
      \begin{myIndent}\hFour
         Suppose $\mu^*(A) = 0$. Then given $E \subseteq X$, we have that:

         {\centering $\mu^*(E) = \mu^*(E \cap A) + \mu^*(E - A) \leq \mu^*(A) + \mu^*(E) \leq 0 + \mu^*(E)$ \retTwo\par}

         It follows that $\mu^*(E) = \mu^*(E \cap A) + \mu^*(E - A)$ for all $E$. So $A \in \mathcal{M}$.\retTwo

         Now if $\mu^*(A) = 0$, then $\mu^*(E) = 0$ for all $E \subseteq A$. It follows that all subsets of $\mu^*$-null sets are in $\mathcal{M}$.\retTwo

         
         \begin{myTindent}\teachComment
            The moral of the story is that we'll just call $\mu^*$ a measure because it is when restricted to the right $\sigma$-algebra.
         \end{myTindent}
      \end{myIndent}
   \end{myIndent}
\end{myIndent}

\mySepTwo

A \udefine{premeasure} $\mu_0 : \mathcal{A} \longrightarrow [0, \infty]$ is a function on an algebra satisfying that:
\begin{itemize}
   \item $\mu_0(\emptyset) = 0$
   \item if $(A_j)_{j \in \mathbb{N}}$ is a sequence of disjoint sets in $\mathcal{A}$ with $\bigcup\limits_{j \in \mathbb{N}}A_j \in \mathcal{A}$, then\\ [-14pt] $\mu_0(\bigcup\limits_{j \in \mathbb{N}}A_j) = \sum\limits_{j=1}^\infty \mu_0(A_j)$.
\end{itemize}


\begin{myTindent}\teachComment
   By setting all but finitely many $A_j$ to the emptyset, we can show that $\mu_0$ must be finitely additive. In turn, this is enough to show that $\mu_0(A) \leq \mu_0(B)$ if $A \subseteq B$ for any $A, B \in \mathcal{A}$.
\end{myTindent}

We say $\mu^*$ is \udefine{induced by} $\mu_0$ if $\mu^*(A) = \inf\left\{\sum\limits_{j = 1}^\infty \mu_0(E_j)\hspace{0.2em} \middle| \hspace{0.2em} E_j \in \mathcal{A} \text{ and } A \subseteq \bigcup\limits_{j = 1}^\infty E_j\right\}$.\retTwo

Note that $\mu^*$ is an outer measure by a previous proposition.\retTwo

\begin{myIndent}\hTwo
   \blab{Proposition:} In this situation:
   \begin{enumerate}
      \item $\mu^*|_\mathcal{A} = \mu_0$
      \begin{myIndent}\hThree
         Proof:\\
         Suppose $E \in \mathcal{A}$ and let $(A_j)_{j \in \mathbb{N}}$ be a sequence of sets of in $\mathcal{A}$ covering $E$. It's trivial that $\mu^*(E) \leq \mu_0(E)$ because we could just let $A_1 = E$ and $A_n = \emptyset$ for all $n \geq 2$.\newpage
         
         On the other hand, letting $B_1 = E \cap A_1$ and $B_m = E \cap A_m - \bigcup\limits_{j = 1}^{m-1}A_j$, we have\\ that $(B_j)_{j \in \mathbb{N}}$ is a sequence of disjoint sets in $\mathcal{A}$ whose union is $E$. It follows\\ [5pt] from the second property of a premeasure and the fact that $B_j \subseteq A_j$ for all\\ [5pt] $j$ that:
         
         {\centering$\mu_0(E) = \sum\limits_{j=1}^\infty \mu_0(B_j) \leq \sum\limits_{j=1}^\infty \mu_0(A_j)$\retTwo\par}
         
         Since $(A_j)_{j \in \mathbb{N}}$ was not specified, it follows that $\mu_0(E) \leq \mu^*(E)$.\retTwo
      \end{myIndent}

      \item Every set in $\mathcal{A}$ is $\mu^*$-measurable.
      \begin{myIndent}\hThree
         Proof:\\
         Suppose $A \in \mathcal{A}, E \subseteq X$, and $\varepsilon > 0$. Then there is a sequence $(B_j)_{j \in \mathbb{N}}$ of sets in $\mathcal{A}$ with $E \subseteq \bigcup\limits_{j=1}^\infty B_j$ and $\sum\limits_{j=1}^\infty \mu_0(B_j) \leq \mu^*(E) + \varepsilon$. Since $\mu_0$ is additive on $\mathcal{A}$, $(B_j \cap A)_{j \in \mathbb{N}} \subseteq \mathcal{A}$ covers $E \cap A$, and $(B_j - A)_{j \in \mathbb{N}} \subseteq \mathcal{A}$ covers $E - A$, we have\\ [5pt] that:

         {\centering 
         \begin{tabular}{l}
            $\mu^*(E) + \varepsilon \geq \sum\limits_{j = 1}^\infty \mu_0(B_j) = \sum\limits_{j = 1}^\infty \mu_0(B_j \cap A) + \sum\limits_{j = 1}^\infty \mu_0(B_j - A)$\\
            $\phantom{\mu^*(E) + \varepsilon \geq \sum\limits_{j = 1}^\infty \mu_0(B_j)} \geq \mu^*(E \cap A) + \mu^*(E - A) $
         \end{tabular} \retTwo\par}

         Taking $\varepsilon \rightarrow 0$, we get that $\mu^*(E) \geq \mu^*(E \cap A) + \mu^*(E - A)$.\retTwo
      \end{myIndent}
   \end{enumerate}

   \blab{Theorem:} Suppose $\mathcal{A} \subseteq \mathcal{P}(X)$ is an algebra and $\mu_0: \mathcal{A} \longrightarrow [0, \infty]$ is a premeasure. Then there exists $\mu: \mathcal{M}(\mathcal{A}) \longrightarrow [0, \infty]$ such that:
   \begin{itemize}
      \item $\mu|_\mathcal{A} = \mu_0$
      \item if $\nu: \mathcal{M}(\mathcal{A}) \longrightarrow [0, \infty]$ is a measure with $\nu|_\mathcal{A} = \mu|_\mathcal{A}$, then $\nu \leq \mu$ (with equality if $\mu(E) < \infty$).
      \item If $\mu_0$ is $\sigma$-finite, then $\mu$ is the unique extension of $\mu_0$ to a measure on $\mathcal{M}(\mathcal{A})$.
   \end{itemize}

   
   \begin{myIndent}\hThree
      Proof:\\
      
      \begin{enumerate}
         \item The first claim is true by Carathéodory's Theorem and the last proposition.\\ Specifically, define $\mu = \mu^*|_{\mathcal{M}(\mathcal{A})}$ where $\mu^*$ is the outer measure induced by\\ $\mu_0$. Since $\mathcal{A}$ is a subset of the $\sigma$-algebra $\mathcal{M}$ of $\mu^*$ measurable sets, we know\\ that $\mathcal{M}(\mathcal{A}) \subseteq \mathcal{M}$. So $\mu$ is a measure over $\mathcal{M}(\mathcal{A})$. Also, we know that\\ $\mu(A) = \mu^*(A) = \mu_0(A)$ for all $A \in \mathcal{A}$ by the last proposition.\retTwo
   
         \item To show the second claim, let $E \in \mathcal{M}(\mathcal{A})$ and $(A_j)_{j \in \mathbb{N}} \subseteq \mathcal{A}$ be a covering of $E$ such that $\sum\limits_{j=1}^\infty \mu_0(A_j) \leq \mu(E) + \varepsilon$ for a given $\varepsilon > 0$. Then:
         
         {\centering$\nu(E) \leq \sum\limits_{j=1}^\infty \nu(A_j) = \sum\limits_{j=1}^\infty \mu_0(A_j) \leq  \mu(E) + \varepsilon$.\retTwo\par}
   
         Taking $\varepsilon \rightarrow 0$, we get that $\nu(E) \leq \mu(E)$.\newpage
   
         As for the other inequality, consider that for any $(A_j)_{j \in \mathbb{N}} \subseteq \mathcal{A}$, we know by part\\ 3 of the theorem on page 16 and the fact that $\nu(A_j) = \mu(A_j)$ for all $j$ that if\\ $A = \bigcup\limits_{j\in\mathbb{N}}A_j$, then:\\ [-17pt]
         
         {\centering $ \nu(A) = \lim\limits_{m\rightarrow \infty}(\nu(\bigcup\limits_{j=1}^m A_j)) = \lim\limits_{m\rightarrow \infty}(\mu(\bigcup\limits_{j=1}^m A_j)) = \mu(A)$. \retTwo\par}
   
         Also, if $\mu(E)$ is finite, then we can choose the covering $(A_j)_{j \in \mathbb{N}} \subseteq \mathcal{A}$ of $E$ so that all $A_j$ are disjoint and $A = \bigcup\limits_{j \in \mathbb{N}}A_j$ satisfies for a given $\varepsilon > 0$ that:\\ [-9pt]\phantom{Aaaaaaaaaaaaaaaaaaaaaaaaaaaaaaaa}$\mu(E) \leq \mu(A) = \sum\limits_{j = 1}^\infty \mu(A_j) \leq \mu(E) + \varepsilon$\retTwo
   
         It follows that $\mu(A - E) < \varepsilon$. So:
         
         {\centering $\mu(E) \leq \mu(A) = \nu(A) = \nu(E) + \nu(A - E) \leq \nu(E) + \mu(A - E) \leq \nu(E) + \varepsilon$.\retTwo\par}
   
         Taking $\varepsilon \rightarrow 0$, we get that $\mu(E) \leq \nu(E)$.\retTwo

         \item For the third claim, suppose $X = \bigcup\limits_{j \in \mathbb{N}}A_j$ with $\mu_0(A_j) < \infty$ and all $A_j$ being\\ disjoint. Then for any $E \in \mathcal{M}(\mathcal{A})$:
         
         {\centering $\mu(E) = \sum\limits_{j=1}^\infty \mu(E \cap A_j) = \sum\limits_{j=1}^\infty \nu(E \cap A_j) = \nu(E) $ \retTwo\par}
      \end{enumerate}
   \end{myIndent}
\end{myIndent}

\exOne

\mySepTwo

\blab{Exercise 1.16:} Let $(X, \mathcal{M}, \mu)$ be a measure space. A set $E \subseteq X$ is called \udefine{locally\\ measurable} if $E \cap A \in \mathcal{M}$ for all $A \in \mathcal{M}$ such that $\mu(A) < \infty$. Let $\widetilde{\mathcal{M}}$ be the\\ collection of all locally measurable sets. Trivially, we know $\mathcal{M} \subseteq \widetilde{\mathcal{M}}$. If $\mathcal{M} = \widetilde{\mathcal{M}}$,\\ then $\mu$ is called \udefine{saturated}.

\begin{enumerate}
   \item[(a)] If $\mu$ is $\sigma$-finite, then $\mu$ is saturated.
   
   \begin{myIndent}\exTwoP
      Let $(A_j)_{j \in \mathbb{N}} \subseteq \mathcal{M}$ satisfy that $\mu(A_j) < \infty$ for all $j$, and that $X = \bigcup\limits_{j\in\mathbb{N}}A_j$. Then\\ [-8pt] if $E$ is locally measurable, we know: $E = \bigcup\limits_{j \in \mathbb{N}}(E \cap A_j) \in \mathcal{M}$. 
   \end{myIndent}

   \item[(b)] $\widetilde{\mathcal{M}}$ is a $\sigma$-algebra.
   \begin{myIndent}\exTwoP
      \begin{itemize}
         \item If $E \in \widetilde{\mathcal{M}}$, then given any $A \in \mathcal{M}$ with $\mu(A) < \infty$, we know $E \cap A \in \mathcal{M}$. It follows that $E^\comp \cap A = A - E = A - (E \cap A) \in \mathcal{M}$. So $E^\comp \in \widetilde{\mathcal{M}}$.\\ [-8pt]
         \item Suppose $(E_j)_{j \in \mathbb{N}}$ is a sequence of sets in $\widetilde{\mathcal{M}}$ and $A \in \mathcal{M}$ satisfies that\\ $\mu(A) < \infty$. Then: $(\bigcup\limits_{j \in \mathbb{N}}E_j) \cap A = \bigcup\limits_{j \in \mathbb{N}}(E_j \cap A) \in \mathcal{M}$.\retTwo

         So, $\bigcup\limits_{j \in \mathbb{N}}E_j \in \widetilde{\mathcal{M}}$
      \end{itemize}
   \end{myIndent}

   \item[(c)] Define $\widetilde{\mu}$ on $\widetilde{\mathcal{M}}$ by $\widetilde{\mu}(E) = \mu(E)$ if $E \in \mathcal{M}$ and $\widetilde{\mu}(E) = \infty$ otherwise. Then $\widetilde{\mu}$ is a saturated measure on $\mathcal{M}$ called the \udefine{saturation} of $\mu$.
   
   \begin{myIndent}\exTwoP
      Since $\emptyset \in \mathcal{M}$, we know $\widetilde{\mu}(\emptyset) = \mu(\emptyset) = 0$.\newpage 

      Note that if $A, B \in \widetilde{\mathcal{M}}$ with $A \subseteq B$ and $A \notin \mathcal{M}$ but $B \in \mathcal{M}$, then we\\ immediately get a contradiction since that would suggest $A = A \cap B \in \mathcal{M}$. As a result, supposing $(E_j)_{j \in \mathbb{N}}$ is a sequence of disjoint sets in $\widetilde{\mathcal{M}}$, we have that if any $E_j \notin \mathcal{M}$, then:\\ [-22pt]
      
      {\centering$\widetilde{\mu}(\bigcup\limits_{j \in \mathbb{N}}E_j) = \infty = \sum\limits_{j =1}^\infty \widetilde{\mu}(E_j)$.\retTwo\par}

      Meanwhile, if all sets of $(E_j)_{j \in \mathbb{N}}$ are in $\mathcal{M}$, then:\\ [-16pt]

      {\centering$\widetilde{\mu}(\bigcup\limits_{j \in \mathbb{N}}E_j) = \mu(\bigcup\limits_{j \in \mathbb{N}}E_j) = \sum\limits_{j =1}^\infty \mu(E_j) = \sum\limits_{j =1}^\infty \widetilde{\mu}(E_j)$.\retTwo\par}
   \end{myIndent}

   \item[(d)] If $\mu$ is complete, then so is $\widetilde{\mu}$.
   
   \begin{myIndent}\exTwoP
      This fact is obvious because by the way we defined $\widetilde{\mu}$, we know a set is $\widetilde{\mu}$-null if and only if it is $\mu$-null.\retTwo
   \end{myIndent}

   \item[(e)] Suppose that $\mu$ is semifinite. For $E \in \widetilde{\mathcal{M}}$, define:
   
   {\centering $\underline{\mu}(E) = \sup \{\mu(A) \mid A \in \mathcal{M} \text{ and } A \subseteq E \}$.\par}

   This is well defined because $\mu(\emptyset)$ is always in the above set and $\mu(X)$ is an upper-\\bound. Then $\underline{\mu}$ is a saturated measure on $\widetilde{\mathcal{M}}$ that extends $\mu$.\\ [-16pt]

   \begin{myIndent}\exTwoP
      Firstly, we show $\underline{\mu}$ is a measure. To start, it's trivial to see that $\underline{\mu}(\emptyset) = \mu(\emptyset) = 0$.\retTwo

      Lemma: If $E \in \widetilde{\mathcal{M}}$ and $\underline{\mu}(E) = \infty$, then there exists a set $A \in \mathcal{M}$ such that $A \subseteq E$ and $\mu(A) = \infty$. 
      \begin{myIndent}\exPPP
         To show this, construct a sequence of "increasing" sets $(A_j)_{j \in \mathbb{N}}$ satisfying\\ that $A_j \subseteq E$ and $\mu(A_j) \geq j$. Then the union $A$ of that sequence will satisfy\\ that $A \in \mathcal{M}$, that $A \subseteq E$, and that $\mu(A) = \infty$.\retTwo
      \end{myIndent}

      Because of that lemma, we don't need to deal with the edge case that a least upper bound equaling infinity doesn't mean a set contains infinity. So, let $(E_j)_{j \in \mathbb{N}}$ be a sequence of disjoint sets in $\widetilde{\mathcal{M}}$ with $E = \bigcup\limits_{j \in \mathbb{N}} E_j$. Then let $\varepsilon > 0$.\retTwo

      To show one inequality, pick a sequence $(A_j)_{j \in \mathbb{N}}$ of sets in $\mathcal{M}$ satisfying that $A_j \subseteq E_j$ and $\underline{\mu}(E_j) - \sfrac{\varepsilon}{2^j} \leq \mu(A_j)$. Since $A = \bigcup\limits_{j \in \mathbb{N}}A_j \subseteq E$, and each $A_j$ is\\ [-10pt] disjoint, we thus have:

      {\centering $-\varepsilon + \sum\limits_{j=1}^\infty \underline{\mu}(E_j) \leq \sum\limits_{j=1}^\infty \mu(A_j) = \mu(A) \leq \underline{\mu}(E)$ \newpage\par}

      To show the other inequality, pick $B \in \mathcal{M}$ satisfying that $B \subseteq E$ and\\ [1pt] $\underline{\mu}(E) - \varepsilon < \mu(B)$. Because $E, E_j \in \widetilde{\mathcal{M}}$ for each $j$, we know that $B \cap E$ and\\ [3pt] $B \cap E_j$ are in $\mathcal{M}$ for each $j$. So:

      {\centering $\underline{\mu}(E) - \varepsilon < \mu(B) = \mu(B \cap E) = \sum\limits_{j=1}^\infty \mu(B \cap E_j) \leq \sum\limits_{j=1}^\infty\underline{\mu}(E_j)$\\ [1pt]\par}

      Taking $\varepsilon \rightarrow 0$, we thus get that $\underline{\mu}(E) = \sum\limits_{j=1}^\infty \underline{\mu}(E_j)$.\retTwo

      Proving that $\mu(E) = \underline{\mu}(E)$ when $E \in \mathcal{M}$ is trivial. Obviously, $\mu(E) \leq \underline{\mu}(E)$.\\ [2pt] Meanwhile for any $F \in \mathcal{M}$ satisfying that $F \subseteq E$, we know that $\mu(F) \leq \mu(E)$.\\ [2pt] So, there does not exists a subset of $E$ in $\mathcal{M}$ with greater measure than $\mu(E)$.\retTwo
   \end{myIndent}

   Note that $\widetilde{\mu}$ and $\underline{\mu}$ are not necessarily equal. Part (f) of this problem gives a relatively\\ [2pt] simple counterexample.

\end{enumerate}

\mySepTwo

\blab{Exercise 1.17:} If $\mu^*$ is an outer measure on $X$ and $(A_j)_{j\in \mathbb{N}}$ is a sequence of disjoint\\ $\mu^*$-measurable sets, then:

{\centering $\mu^*(E \cap \bigcup\limits_{j \in \mathbb{N}}A_j) = \sum\limits_{j=1}^\infty \mu^*(E \cap A_j)$ for any $E \subseteq X$.\retTwo\par}


\begin{myIndent}\exTwoP
   Note that by induction, we can show that for any $n \in \mathbb{N}$:

   {\centering\exPP
   \begin{tabular}{l}
      $\mu^*(E \cap \bigcup\limits_{j=1}^\infty A_j) = \mu^*(E \cap \bigcup\limits_{j=1}^\infty A_j \cap A_1) + \mu^*(E \cap \bigcup\limits_{j=1}^\infty A_j - A_1)$\\ [4pt]
      $\phantom{\mu^*(E \cap \bigcup\limits_{j=1}^\infty A_j)} = \mu^*(E \cap A_1) + \mu^*(E \cap \bigcup\limits_{j=2}^\infty A_j)$\\ [4pt]
      $\phantom{\mu^*(E \cap \bigcup\limits_{j=1}^\infty A_j)} = \sum\limits_{j=1}^2\mu^*(E \cap A_j) + \mu^*(E \cap \bigcup\limits_{j=3}^\infty A_j)$\\ [4pt]
      $\phantom{\mu^*(E \cap \bigcup\limits_{j=1}^\infty A_j)} = \cdots = \sum\limits_{j=1}^n \mu^*(E \cap A_j) + \mu^*(E \cap \hspace{-0.5em}\bigcup\limits_{j=n+1}^\infty\hspace{-0.5em} A_j)$
   \end{tabular} \retTwo\par}

   Thus, we clearly have for all $n$ that $\sum\limits_{j=1}^n \mu^*(E \cap A_j) \leq \mu^*(E \cap \bigcup\limits_{j=1}^\infty A_j)$.\\ [-6pt]

   Taking the limit as $n \rightarrow \infty$, we thus know $\sum\limits_{j=1}^\infty \mu^*(E \cap A_j) \leq \mu^*(E \cap \bigcup\limits_{j \in \mathbb{N}}A_j)$.\retTwo

   The other inequality is obvious from the subadditivity property of outer measures.\retTwo
\end{myIndent}

\mySepTwo

\blab{Exercise 1.18:} Let $\mathcal{A} \subseteq \mathcal{P}(X)$ be an algebra, $\mathcal{A}_\sigma$ be be the collection of countable unions of sets in $\mathcal{A}$, and $\mathcal{A}_{\sigma\delta}$ be the collection of countable intersections of sets in $\mathcal{A}_{\sigma}$. Let $\mu_0$ be a premeasure on $\mathcal{A}$ and $\mu^*$ the induced outer measure.\newpage

\begin{enumerate}
   \item[(a)] For any $E \subseteq X$ and $\varepsilon > 0$, there exists $A \in \mathcal{A}_\sigma$ with $E \subseteq A$ and $\mu^*(A) \leq \mu^*(E) + \varepsilon$.
   
   \begin{myIndent}\exTwoP
      Let $(A_j)_{j \in \mathbb{N}}$ be a sequence of sets in $\mathcal{A}$ which cover $E$ and satisfy that\\ $\sum\limits_{j=1}^\infty \mu_0(A_j) \leq \mu^*(E) + \varepsilon$. In turn, by the subadditivity of outer measures, and\\ [1pt] the fact that $\mu^*(A) = \mu_0(A)$ for all $A \in \mathcal{A}$, we know that:

      {\centering $E \subseteq \bigcup\limits_{j \in \mathbb{N}}A_j \in \mathcal{A}_{\sigma}$ and $\mu^*(\bigcup\limits_{j \in \mathbb{N}}A_j) \leq \sum\limits_{j=1}^\infty \mu^*(A_j) = \sum\limits_{j=1}^\infty \mu_0(A_j) \leq \mu^*(E) + \varepsilon$ \retTwo\par}
   \end{myIndent}
   
   \item[(b)] If $\mu^*(E) < \infty$, then $E$ is $\mu^*$-measurable if and only if there exists $B \in \mathcal{A}_{\sigma\delta}$ with $E \subseteq B$ and $\mu^*(B - E) = 0$.
   
   \begin{myIndent}\exTwoP
      Suppose $E$ is $\mu^*$-measurable. Then for all $j \in \mathbb{N}$, pick $A_j \in \mathcal{A}_{\sigma}$ satisfying that\\ $E \subseteq A_j$ and $\mu^*(E) \leq \mu^*(A_j) \leq \mu^*(E) + \sfrac{1}{j}$. Since $E$ is $\mu^*$-measurable, we\\ know that for all $j \in \mathbb{N}$:

      {\centering $\mu^*(A_j) = \mu^*(A_j \cap E) + \mu^*(A_j - E) = \mu^*(E) + \mu^*(A_j - E)$ \retTwo\par}

      In turn, because $\mu^*(E) < \infty$, this tell us that $\mu^*(A_j - E) \leq \sfrac{1}{j}$. Also, because $\bigcap\limits_{j \in \mathbb{N}}A_j \subseteq A_n$ for all $n \in \mathbb{N}$, we know that $\mu^*(\bigcap\limits_{j\in\mathbb{N}}A_j - E) \leq \sfrac{1}{n}$ for all $n \in \mathbb{N}$.\retTwo

      As a result, we know that $\bigcap\limits_{j \in \mathbb{N}}A_j \in \mathcal{A}_{\sigma\delta}$, $E \subseteq \bigcap\limits_{j \in \mathbb{N}}A_j$, and $\mu^*(\bigcap\limits_{j\in\mathbb{N}}A_j - E) = 0$.\retTwo

      To prove the reverse implication, suppose there exists a $\mu^*$-separable set $B$\\ satisfying that $E \subseteq B$ and $\mu^*(B - E) = 0$ (any set in $\mathcal{A}_{\sigma\delta}$ will be $\mu^*$-separable because $\mathcal{A}_{\sigma\delta} \subseteq \mathcal{M}(\mathcal{A})$). Then given any set $F$, we have that:

      {\centering 
      \begin{tabular}{l}
         $\mu^*(F - E) = \mu^*(F \cap E^\comp \cap B) + \mu^*(F \cap E^\comp \cap B^\comp)$\\ [2pt]
         $\phantom{\mu^*(F - E)} = \mu^*(F \cap (B - E)) + \mu^*(F - B) = \mu^*(F - B)$
      \end{tabular} \retTwo\par}

      Also, since $F \cap E \subseteq F \cap B$, we know that $\mu^*(F \cap E) \leq \mu^*(F \cap B)$.\retTwo

      So, $\mu^*(F \cap E) + \mu^*(F - E) \leq \mu^*(F \cap B) + \mu^*(F - B)  = \mu^*(F)$. Hence, $E$ is $\mu^*$-measurable.\retTwo      
   \end{myIndent}

   \item[(c)] If $\mu_0$ is $\sigma$-finite, we can remove the requirement in part (b) that $\mu^*(E) < \infty$.
   
   \begin{myIndent}\exTwoP
      Because the backwards implication proof never required $\mu^*(E)$ to be finite, it suffices to show that $E$ being $\mu^*$-measurable implies there exists $B \in A_{\sigma\delta}$ with $E \subseteq B$ and $\mu^*(B - E) = 0$.\retTwo

      To start, let $(C_i)_{i \in \mathbb{N}}$ satisfy that $\mu_0(C_i) < \infty$ and $\bigcup\limits_{i = 1}^\infty C_i = X$.\newpage
      
      Next for all $j, i \in \mathbb{N}$, pick $A_j^{(i)} \in \mathcal{A}_\sigma$ satisfying that $E \cap C_i \subseteq A_j^{(i)}$ and\\ $\mu^*(E) \leq \mu^*(A_j^{(i)}) \leq \mu^*(E) + \sfrac{1}{j2^{i}}$. Since $\mu^*(E \cap C_i)$ is finite, we can use the same reasoning as in part (b) to say that $\mu^*(A_j^{(i)} - (E \cap C_i)) \leq \sfrac{1}{j2^i}$.\retTwo

      Importantly, $A_j^{(i)} - E \subseteq A_j^{(i)} - (E \cap C_i)$. for all $i$. Therefore:
      
      {\centering
      \begin{tabular}{l}
         $\mu^*((\bigcup\limits_{i\in\mathbb{N}}A_j^{(i)}) - E) = \mu^*(\bigcup\limits_{i\in\mathbb{N}}(A_j^{(i)} - E))$\\ [6pt]
         
         $\phantom{\mu^*((\bigcup\limits_{i\in\mathbb{N}}A_j^{(i)}) - E) } \leq \mu^*(\bigcup\limits_{i\in\mathbb{N}}(A_j^{(i)} - (E \cap C_i)))$\\
         
         $\phantom{\mu^*((\bigcup\limits_{i\in\mathbb{N}}A_j^{(i)}) - E) } \leq \sum\limits_{i \in \mathbb{N}}\mu^*(A_j^{(i)} - (E \cap C_i)) \leq \sum\limits_{i \in \mathbb{N}}\frac{1}{j2^i} = \frac{1}{j}$
      \end{tabular} \retTwo\par}

      Since $E \subseteq \bigcup\limits_{i \in \mathbb{N}}A_j^{(i)} \in \mathcal{A}_\sigma$, we've thus shown for all $j \in \mathbb{N}$ that there exists a set\\ [-8pt]\phantom{Aaaaaaaaaaaaaaaaaa} $A_j \in \mathcal{A}_\sigma$ satisfying that $\mu^*(A_j - E) \leq \sfrac{1}{j}$ and $E \subseteq A_j$.\retTwo

      Finally, intersecting all those $A_j$ like in part (b), we get our set satisfying the right-side of the implication.\retTwo
   \end{myIndent}
\end{enumerate}

\mySepTwo

\blab{Exercise 1.19:} Let $\mu^*$ be an outer measure on $X$ induced from a finite premeasure\\ $\mu_0$ defined on an algebra $\mathcal{A}$. If $E \subseteq X$, define the \udefine{inner measure} of $E$ to be\\ $\mu_*(E) = \mu_0(X) - \mu^*(E^\comp)$. Then $E$ is $\mu^*$-measurable if and only if $\mu^*(E) = \mu_*(E)$.\\ [-9pt]

\begin{myIndent}\exTwoP
   ($\Longrightarrow$)\\
   If $E$ is $\mu^*$-measurable, then we have $\mu^*(X \cap E) + \mu^*(X - E) = \mu^*(X)$. Because $\mu^*(X) = \mu_0(X)$, we thus have that $\mu_*(E) = \mu_0(X) - \mu^*(X - E) = \mu_*(E)$.\retTwo

   ($\Longleftarrow$)\\
   By part (a) of the previous exercise, we know there exists $A_j \in \mathcal{A}_\sigma$ satisfying that\\ $E \subseteq A_j$ and $\mu^*(A_j) \leq \mu^*(E) + \sfrac{1}{j}$.\retTwo

   Note that $A_j$ is $\mu^*$-measurable because $\mathcal{A}_\sigma \subseteq \mathcal{M}(\mathcal{A})$. This means that:
   \begin{itemize}
      \item $\mu_0(X) = \mu^*(X) = \mu^*(X \cap A_j) + \mu^*(X - A_j) = \mu^*(A_j) + \mu^*(A_j^\comp)$.
      \item $\mu^*(E^\comp) = \mu^*(E^\comp \cap A_j) + \mu^*(A^\comp_j \cap E^\comp) = \mu^*(A_j - E) + \mu^*(A^\comp_j)$.\retTwo
   \end{itemize}

   Supposing that $\mu^*(E) = \mu_*(E) = \mu_0(X) - \mu^*(X - E)$ and plugging in the first bullet-pointed identity, we get that:
   
   {\centering $\mu^*(E^\comp) = \mu^*(A_j) + \mu^*(A_j^\comp) - \mu^*(E)$.\retTwo\par}

   Substituting that into the second bullet-pointed identity, we have:

   {\centering $\mu^*(A_j) - \mu^*(E) = \mu^*(A_j - E)$.\newpage\par}

   And finally, using the inequality: $\mu^*(A_j) \leq \mu^*(E) + \sfrac{1}{j}$, we get $\mu^*(A_j - E) \leq \sfrac{1}{j}$.\retTwo

   Hence, we've shown that there exists a set $A_j \in \mathcal{A}_\sigma$ such that $E \subseteq A_j$ and\\ $\mu^*(A_j - E) < \sfrac{1}{j}$ for all $j \in \mathbb{N}$. From there, we can proceed exactly like in part\\ (b) of exercise 1.18. Pick such an $A_j$ for all $j \in \mathbb{N}$ and then intersect them together. The result will be a set $B \in \mathcal{A}_{\sigma\delta}$ satisfying that $E \subseteq B$ and $\mu^*(B - E) = 0$. Since such a set exists, we know by the conclusion of part (b) of exercise 1.18 that $E$ is $\mu^*$-measurable.
\end{myIndent}

\mySepTwo

\blab{Exercise 1.21:} Let $\mu^*$ be an outer measure induced from a premeasure defined on an\\ algebra $\mathcal{A}$ and $\overline{\mu}$ be the restriction of $\mu^*$ to the collection $\mathcal{M}$ of $\mu^*$-measurable sets. Then $\overline{\mu}$ is saturated.\\ [-9pt]


\begin{myIndent}\exTwoP
   Let $E$ be a locally $\overline{\mu}$-measurable set and choose any $F \subseteq X$. Given any $\varepsilon > 0$, by part (a) of exercise 1.18, we know that there exists a $\mu^*$-measurable set $A \in \mathcal{A}_{\sigma} \subseteq \mathcal{M}$\\ such that $F \subseteq A$ and $\mu^*(A) \leq \mu^*(F) + \varepsilon$.\retTwo

   Assuming without loss of generality that $\mu^*(F)$ is finite, we thus know that\\ $\mu^*(A) < \infty$. So, since $E$ is locally $\overline{\mu}$-measurable, we have that $E \cap A \in \mathcal{M}$.\retTwo

   Now, we first note that because $F \cap E \subseteq A \cap E$ and $F - E \subseteq A - E$, we have that:

   {\centering $\mu^*(F \cap E) + \mu^*(F - E) \leq \mu^*(A \cap E) + \mu^*(A - E) $\retTwo\par}

   Next we note that: 
   \begin{itemize}
      \item $A \cap (A \cap E) = A \cap E$
      \item $A \cap (A \cap E)^\comp = (A \cap A^\comp) \cup (A \cap E^\comp) = A - E$\retTwo
   \end{itemize}

   So: $\mu^*(A \cap E) + \mu^*(A - E) = \mu^*(A \cap (A \cap E)) + \mu^*(A \cap (A \cap E)^\comp) = \mu^*(A)$.\retTwo

   And finally, since $\mu^*(A) \leq \mu^*(F) + \varepsilon$, we can thus conclude that:
   
   {\centering$\mu^*(F \cap E) + \mu^*(F - E) \leq \mu^*(F) + \varepsilon$.\retTwo\par}

   Taking $\varepsilon \rightarrow 0$, we get that $\mu^*(F \cap E) + \mu^*(F - E) \leq \mu^*(F)$ So, $E$ is\\ $\mu^*$-measurable, meaning $E \in \mathcal{M}$.
\end{myIndent}

\hOne\mySepTwo

Consider the collection $H = \{\emptyset, (a, b], (a, \infty) \mid -\infty \leq a < b < \infty\}$ of "half-open-intervals" of $\mathbb{R}$.

\begin{myIndent}\hTwo
   This forms an elementary family.
   \begin{itemize}
      \item We specified in the definition that $\emptyset \in H$.
      \item If $x \in (a, b] \cap (c, d] \neq \emptyset$, then we know $a < x < d$ and $c < x< b$. So\\ $(a, b] \cap (c, d] = (\max(a, c), \min(b, d)] \in H$.\newpage
      \item Given $(a, b] \in H$, we have that $(a , b]^\comp = (-\infty, a] \cup (b, \infty)$.
      \begin{myIndent}\myComment
         For the sake of time, I'm ignoring edge cases of a right bound of infinity since they are still trivial.\retTwo
      \end{myIndent}
   \end{itemize}
\end{myIndent}

By exercise 1.2, we know that $\mathcal{M}(H) = \mathcal{B}_{\mathbb{R}}$. And, by a previous proposition, we know that $\mathcal{A}$ equal to the collection of finite disjoint unions of $H$ is an algebra.\retTwo

\begin{myIndent}\hTwo
   \blab{Proposition:} Let $F: \mathbb{R} \longrightarrow \mathbb{R}$ be a monotonically increasing and right continuous function (meaning $\lim\limits_{t\rightarrow x^+}F(t) = F(x)$ for all $x \in \mathbb{R}$). Also, define:\\ [-6pt]\phantom{aaaaaaaaaaaaaaaaaaaaaaaa} $F(-\infty) = \lim\limits_{t\rightarrow -\infty}F(t)$ and $F(\infty) = \lim\limits_{t\rightarrow \infty}F(t)$.\retTwo If $(a_j, b_j]$ for $j = 1,\ldots, n$ are disjoint intervals in $H$, define:

   {\centering $\mu_0(\bigcup\limits_{j=1}^n (a_j, b_j]) = \sum\limits_{j=1}^n F(b_j) - F(a_j)$ \\ [1pt]\par}

   Also let $\mu_0(\emptyset) = 0$. And if $F(\infty) = \infty$ and $F(-\infty) = -\infty$, define $\mu_0(\mathbb{R}) = \infty$. Then this is a premeasure.
   
   \begin{myIndent}\hThree
      Proof:
      \begin{enumerate}
         \item $\mu_0$ is well defined.
         \begin{myIndent}\hFour
            Suppose $(a_j, b_j]$ for $j = 1, \ldots, n$ are disjoint intervals in $H$ satisfying that $(a, b] = \bigcup\limits_{j=1}^n (a_j b_j] = (a, b]$. Then after indexing those half intervals in a\\ certain way, we must have that:\\ [-10pt]

            {\centering $a = a_1 < b_1 = a_2 < \ldots < b_{n-1} = a_{n} < b_{n-1} = b $ \retTwo\par}

            It follows that:
            
            {\centering$F(b) - F(a) = \mu_0(\bigcup\limits_{j=1}^n(a_j, b_j]) = \sum\limits_{j=1}^n \mu_0((a_j, b_j]) = \sum\limits_{j=1}^n F(b_j) - F(a_j) $ \retTwo\par}

            In other words, $\mu_0$ is well defined for individual intervals.\retTwo

            Now suppose $I_i$ and $J_j$ for $i = 1,\ldots,m$ and $j = 1, \ldots, n$ are disjoint intervals in $H$ satisfying that $\bigcup\limits_{i=1}^m I_i = \bigcup\limits_{j=1}^n J_j$.\retTwo
            
            For each $I_i$, we can repeat the same reasoning as above with the collection of sets $I_i \cap J_j$ for $j = 1, \ldots, n$ in order to get that:

            {\centering $\mu_0(I_i) = \sum\limits_{j=1}^n \mu_0(I_i \cap J_j)$ \retTwo\par}

            Similarly, we can show for each $J_j$ that:

            {\centering $\mu_0(J_j) = \sum\limits_{i=1}^m \mu_0(I_i \cap J_j)$ \retTwo\par}

            Thus: $\sum\limits_{i=1}^m \mu_0(I_i) = \sum\limits_{i=1}^m\sum\limits_{j=1}^n \mu_0(I_i \cap J_j) = \sum\limits_{j=1}^n \mu(J_j)$.\newpage
         \end{myIndent}

         \item If $(I_j)_{j \in \mathbb{N}} \subseteq \mathcal{A}$ is a disjoint sequence satisfying that $\bigcup\limits_{j\in\mathbb{N}}I_j \in \mathcal{A}$, then:\\ [-8pt]
         
         {\centering$\mu_0(\bigcup\limits_{j\in\mathbb{N}}I_j) = \sum\limits_{j=1}^\infty \mu_0(I_j)$.\retTwo\par}

         \begin{myIndent}\hFour
            Since $\bigcup\limits_{j\in\mathbb{N}}I_j \in \mathcal{A}$, we know it is equal to a finite union of disjoint intervals of $H$. By considering those intervals separately, we can thus assume without\\ [6pt] loss of generality that $\bigcup\limits_{j \in \mathbb{N}}I_j = (a, b]$ where not both $a = -\infty$ and $b = \infty$.\retTwo

            Also, without loss of generality we can assume each $I_j$ is one interval.\retTwo

            Now it's obvious from the construction of $\mu_0$ that $\mu_0$ is additive. And since $(a, b] - \bigcup\limits_{j=1}^m I_j \in \mathcal{A}$ for all $m$, we thus know that:\\ [-5pt]

            {\centering $\mu_0((a, b]) = \mu_0(\bigcup\limits_{j=1}^m I_j) + \mu_0((a, b] - \bigcup\limits_{j=1}^m I_j) \geq \mu_0(\bigcup\limits_{j=1}^m I_j) = \sum\limits_{j=1}^m \mu_0(I_j)$ \\\par}

            Taking the limit as $m \rightarrow \infty$, we get that: $\sum\limits_{j=1}^\infty \mu_0(I_j) \leq \mu_0(\bigcup\limits_{j\in \mathbb{N}} I_j)$.\\ [-6pt]
         \end{myIndent}
      \end{enumerate}
   \end{myIndent}
\end{myIndent}

\mHeader{Lecture 6 Notes: 10/15/2024}

\begin{myIndent}\hTwo
   \begin{myIndent}\hFour
      To show the reverse inequality, suppose $a, b \in \mathbb{R}$ (a.k.a. finite), and let $\varepsilon > 0$. Since\\ $F$ is right-continuous, there exists $\delta > 0$ such that $F(a + \delta) - F(a) < \varepsilon$. Similarly, given that $I_j = (a_j, b_j]$, there exists $\delta_j > 0$ such that $F(b_j + \delta_j) - F(b_j) < \frac{\varepsilon}{2^j}$ for all $j \in \mathbb{N}$.\retTwo

      Next, note that the collection $\{ (a_j, b_j + \delta_j) \}_{j \in \mathbb{N}}$ of open intervals covers the set\\ $[a + \delta, b]$. Thus by compactness, there is a finite subcover. In other words,
      
      {\centering $(a_1, b_1 + \delta_1),\ldots,(a_N, b_N + \delta_N)$ cover $[a + \delta, b]$ \retTwo\par}
      
      Furthermore, by removing intervals in that finite subcover which are subsets of other intervals and by reindexing, we can assume that:

      {\centering $b_j + \delta_j \in (a_{j+1}, b_{j+1} + \delta_{j+1})$ for all $j = 1, \ldots, N - 1$. \retTwo\par}

      Then:

      {\centering 
      \begin{tabular}{l}
         $\mu_0((a, b]) = F(b) - F(a)$\\
         $\phantom{\mu_0((a, b])} < F(b) - F(a + \delta) + \varepsilon$\\
         $\phantom{\mu_0((a, b])} \leq F(b_N + \delta_N) - F(a_1) + \varepsilon$\phantom{aaaaaaa} {\teachComment(since $F$ is monotone increasing)}\\ [6pt]

         $\phantom{\mu_0((a, b])} = F(b_N + \delta_N) - f(a_N) + \sum\limits_{j=1}^{N-1}(F(a_{j+1}) - F(a_j)) + \varepsilon$\\ [14pt]

         $\phantom{\mu_0((a, b])} \leq F(b_N + \delta_N) - f(a_N) + \sum\limits_{j=1}^{N-1}(F(b_j + \delta_j) - F(a_j)) + \varepsilon$\\ [-6pt]
         \phantom{aaaaaaaaaaaaaaaaaaaaaaaaaaaaaaaaaaa} {\teachComment(again since $F$ is monotone increasing)}\\ [0pt]

         $\phantom{\mu_0((a, b])} = \sum\limits_{j=1}^N (F(b_j + \delta_j) - F(a_j)) + \varepsilon$\\ [14pt]
         
         $\phantom{\mu_0((a, b])} < \sum\limits_{j=1}^N(F(b_j) + \frac{\varepsilon}{2^j} - F(a_j)) + \varepsilon < \sum\limits_{j=1}^\infty \mu(I_j) + 2\varepsilon$
      \end{tabular} \retTwo\par}

      Since $\varepsilon$ is arbitrary, we've now shown the reverse inequality when $a$ and $b$ are finite. To extend this result to when $a = -\infty$ or $b = \infty$, note that the invervals $(a_j, b_j + \delta_j)$ cover $[-M  + \delta, b]$ or $[a + \delta, M]$ for all $M$ in either $(-\infty, b]$ or $(a , \infty)$.\retTwo
      
      So, doing the same manipulations as before, since $\sum\limits_{j=1}^\infty \mu(I_j) + 2\varepsilon$ is an upper bound of $\mu_0((-M, b])$ or $\mu_0((a, M])$, we know that the limit of $\mu_0((-M, b])$ or $\mu_0((a, M])$ as\\ [6pt] $M \rightarrow \infty$ will not exceed that upper bound. Then taking $\varepsilon \rightarrow 0$, we get the same\\ [6pt] result as before.\retTwo

      Plus, based on how we defined $F(\infty)$ and $F(-\infty)$, we know that\\ $\lim\limits_{M \rightarrow \infty}\mu_0((-M, b]) = \mu_0((-\infty, b])$ and $\lim\limits_{M \rightarrow \infty}\mu_0((a, M]) = \mu_0((a, \infty))$.\retTwo
      \end{myIndent}

   \blab{Theorem:} 
   \begin{enumerate}
      \item If $F$ is a monotone increasing and right-continuous like above, there is a unique\\ Borel measure $\mu_F$ on $\mathcal{B}_\mathbb{R}$ such that $\mu_F((a, b]) = F(b) - F(a)$ for all $a < b$.
      \item If $G$ is another such monotone increasing and right-continuous, then $\mu_G = \mu_F$ if and only if $F - G$ is a constant.
      \item If $\mu$ is a $\mathcal{B}_{\mathbb{R}}$ measure that is finite on all bounded sets, then we can define:
      
      {\centering $F(x) = \left\{
      \begin{matrix}
         \mu((0, x]) & \text{ if } x > 0 \\
         0 & \text{ if } x = 0 \\
         -\mu((x, 0]) & \text{ if } x < 0
      \end{matrix}\right.$ \retTwo\par}

      Then $F$ is a monotone increasing and right-continuous like above with $\mu = \mu_F$.
   \end{enumerate}
   
   \begin{myIndent}\hThree
      Proof:
      \begin{enumerate}
         \item By the previous proposition, we know $F$ induces a premeasure $\mu_F$. Also, $\mu_F$ is a\\ $\sigma$-finite premeasure. Thus, by the theorem on page 24, we know $\mu_F$ induces a unique measure on $\mathcal{M}(H) = \mathcal{B}_{\mathbb{R}}$.\retTwo
   
         \item Clearly, if $G(b) - G(a) = F(b) - F(a)$ for all $-\infty \leq a < b \leq \infty$, then we must have that $G - F$ is constant.\retTwo
   
         \item Finally, by the theorem at the bottom of page 16, we can fairly easily show that $F$ is right-continuous and monotone increasing.\retTwo
      \end{enumerate}
   \end{myIndent}
\end{myIndent}

Given a monotone increasing and right-continuous function $F$, we call $\mu_F$ the\\ \udefine{Lesbesgue-Stieltjes measure}. If $F(x) = x$, we just write $m$ and call it the\\ \udefine{Lesbesgue measure}.\retTwo

Also, we'll almost always use $\mu_F$ to refer to the completion of $(\mathbb{R}, \mathcal{B}_{\mathbb{R}}, \mu_F)$. So, this measure is actually defined on more than just $\mathcal{B}_{\mathbb{R}}$.\newpage

Let $\mu$ be a Lesbesgue-Stieltjes measure associated with a monotone increasing, right-continuous function $F$, and let $\mathcal{M}_{\mu}$ be the set of $\mu$-measureable sets. Here are some nice properties of $\mu$:

\begin{myIndent}\hTwo
   \blab{Lemma:} Suppose $\nu(E) = \inf(\{\sum\limits_{j=1}^\infty \mu((a_j, b_j)) \mid E \subseteq \bigcup\limits_{j=1}^\infty (a_j, b_j)\})$. Then\\ [-8pt] $\nu(E) = \mu(E)$ for all $E \in \mathcal{M}_{\mu}$.\retTwo
   
   \begin{myIndent}\hThree
      Proof:\\
      Fix $\varepsilon > 0$. Then, there exists $((a_j, b_j])_{j \in \mathbb{N}}$ such that $E \subseteq \bigcup\limits_{j \in \mathbb{N}}(a_j, b_j]$ and\\ [-9pt] $\sum\limits_{j=1}^\infty \mu((a_j, b_j]) \leq \mu(E) + \varepsilon$.\retTwo

      Then for all $j \in \mathbb{N}$, pick $\delta_j > 0$ such that $F(b_j + \delta_j) - F(b_j) \leq \sfrac{\varepsilon}{2^j}$. Thus:

      {\centering
      \begin{tabular}{l}
          $\nu(E) \leq \sum\limits_{j=1}^\infty \mu((a_j, b_j + \delta_j)) \leq \sum\limits_{j=1}^\infty \mu((a_j, b_j + \delta_j])$\\ [10pt] $\phantom{\nu(E) \leq \sum\limits_{j=1}^\infty \mu((a_j, b_j + \delta_j))} \leq \sum\limits_{j=1}^\infty \mu((a_j, b_j]) + \varepsilon \leq \mu(E) + 2\varepsilon$ 
      \end{tabular}\retTwo\par}

      Taking $\varepsilon \rightarrow 0$, we get that $\nu(E) \leq \mu(E)$.\retTwo

      To get the reverse inequality, note that we can write any open interval as the\\ countable union of a sequence of disjoint half-open-intervals. It follows that\\ $\mu(E) \leq \nu(E)$.\retTwo
   \end{myIndent}

   \blab{Theorem:} If $E \subseteq \mathcal{M}_\mu$, then:
   \begin{itemize}
      \item $\mu(E) = \inf(\{\mu(U) \mid E \subseteq U \text{ and } U \text{ is open}\})$
      \item $\mu(E) = \sup(\{\mu(K) \mid K \subseteq E \text{ and } K \text{ is compact}\})$
      
      \begin{myIndent}\hThree
         Proof:\\
         The first bullet point follows almost immediately from the previous lemma\\ because unions of open sets are open.\retTwo

         To show the second bullet point, first suppose $E$ is bounded. If $E$ is closed, then the equality is trivial. So suppose $\overline{E} - E \neq \emptyset$. Then given any $\varepsilon > 0$, let $U$ be an open set such that $\overline{E} - E \subseteq U$ and $\mu(U) \leq \mu(\overline{E} - E) + \varepsilon$. (Note that $\overline{E} \in \mathcal{B}_\mathbb{R}$ and so $\overline{E} - E \in \mathcal{M}_\mu$.)\retTwo

         We have that $K = \overline{E} - U$ is a closed bounded set satisfying that $K \subseteq E$ and:

         {\centering 
         \begin{tabular}{l}
            $\mu(K) = \mu(E) - \mu(E \cap U) = \mu(E) - (\mu(U) - \mu(U - E))$\\ $\phantom{\mu(K) = \mu(E) - \mu(E \cap U)} \geq \mu(E) - \mu(U) + \mu(\overline{E} - E) \geq \mu(E) - \varepsilon$
         \end{tabular} \retTwo\par}

         If $E$ is unbounded, then consider $E_j \cap (j, j+1]$ for all $j \in \mathbb{Z}$. By the previous reasoning, given any $\varepsilon > 0$ there exists a compact set $K_j$ such that $K_j \subseteq E_j$ and $\mu(K_j) \geq \mu(E) - \sfrac{\varepsilon}{2^{|j|}}$.\newpage

         Now define $H_n = \bigcup\limits_{j=-n}^nK_j$. That way $H_n$ is a compact subset of $E$ and:\\ [-10pt]
         
         {\centering $\mu(H_n) \geq \mu(\hspace{-0.2em}\bigcup\limits_{j=-n}^n\hspace{-0.4em} E_j) - 3\varepsilon$.\\ [2pt]\par}

         Since $\mu(E) = \lim\limits_{n \rightarrow \infty}(\mu(\hspace{-0.2em}\bigcup\limits_{j=-n}^n\hspace{-0.4em} E_j)) \leq \lim\limits_{n \rightarrow \infty}(\mu(H_n)) + 3\varepsilon$, the bulllet pointed claim\\ [-8pt] follows.\retTwo
      \end{myIndent}
   \end{itemize}

   \blab{Theorem:} If $E \subseteq \mathbb{R}$, then the following are equivalent:
   \begin{itemize}
      \item[(a)] $E \in \mathcal{M}_\mu$.
      \item[(b)] $E = V - N_1$ where $V$ is a $G_\delta$ set and $\mu(N_1) = 0$.
      \item[(c)] $E = H \cup N_2$ where $H$ is an $F_\sigma$ set and $\mu(N_2) = 0$.\retTwo
      
      \begin{myIndent}\hThree
         Proof:\\
         We trivially have that (b) $\Longrightarrow$ (a) and (c) $\Longrightarrow$ (a) because $V, H \in \mathcal{B}_\mathbb{R}$ and\\ $N_1, N_2 \in \mathcal{M}_\mu$.\retTwo

         To show the other direction, suppose $E \subseteq \mathcal{M}_{\mu}$. Firstly, we'll suppose\\ $\mu(E) < \infty$. Then for all $j \in \mathbb{N}$, we can choose an open set $U_j \supseteq E$ and\\ a compact set $K_j \subseteq E$ such that:

         {\centering $\mu(U_j) - \sfrac{1}{2^j} \leq \mu(E) \leq \mu(K_j) + \sfrac{1}{2^j}$ \retTwo\par}

         Define $V = \bigcap\limits_{j\in\mathbb{N}}U_j$ and $H = \bigcup\limits_{j \in \mathbb{N}}K_j$.\retTwo
         
         Then $H \subseteq E \subseteq V$ and $\mu(H) = \mu(E) = \mu(V)$. It follows since $\mu(E) < \infty$ that:
         
         \begin{itemize}
            \item[\bullet] Letting $N_1 = V - E$, we have that $E = V - N_1$ and
            
            {\centering $\mu(V) = \mu(N_1) + \mu(E) \Longrightarrow 0 = \mu(V) - \mu(E) = \mu(N_1)$.\par}

            \item[\bullet] Letting $N_2 = E - H$, we have that $E = H \cup N_2$ and
            
            {\centering $\mu(E) = \mu(H) + \mu(N_2) \Longrightarrow 0 = \mu(E) - \mu(H) =\mu(N_2)$. \retTwo\par}
         \end{itemize}

         \exP The rest is \blab{Exercise 1.25}.\\\exPP
         Now suppose $\mu(E) = \infty$.\retTwo
         
         For all $j \in \mathbb{Z}$, define $E_j = E \cap (j, j+1]$. That way $E$ is the disjoint union of all $E_j$. Importantly, each $\mu(E_j) \leq \mu((j, j+1]) < \infty$. So, using the previous logic, for all $j$ we can find an $F_\sigma$ set $H_j$ satisfying that $H_j \subseteq E_j$ and:

         \begin{itemize}
            \item[\bullet] Letting $N_2^{(j)} = E_j - H_j$, we have that $E_j = H_j \cup N_2^{(j)}$ and $\mu(N_2^{(j)}) = 0$.\retTwo
         \end{itemize}

         Setting $H = \bigcup\limits_{j \in \mathbb{Z}} H_j$, we have that $H$ is a countable union of countable unions of closed sets. Thus, $H$ is an $F_\sigma$ set.\retTwo
         
         Also $N_2 = \bigcup\limits_{j \in \mathbb{Z}}N_2^{(j)}$ is still a null set since the set of null sets is closed under\\ [-8pt] \phantom{aaaaaaaaaaaaaaaa} countable union. And finally, $H \cup N_2 = E$ becuase\\ \phantom{aaaaaaaaaaaaaaaa} each $E_j = H_j \cup N_2^{(j)}$. Thus, we've shown (c).\newpage

         Meanwhile, to show (b), consider that if $E \in \mathcal{M}_\mu$, then $E^\comp \in \mathcal{M}_\mu$. So, we can find an $F_\sigma$ set $H$ and null set $N_1$ such that $E^\comp = H \cup N_1$. Taking complements of both sides, we get that $E = H^\comp - N_1$. And importantly, the complement of a union of sets is equivalent to the intersection of the complements of those sets. So, since the complement of a closed set is open,  $H^\comp$ is a $G_\delta$ set.\retTwo
      \end{myIndent}
   \end{itemize}
\end{myIndent}

Note that we call measures with the property described in the above two theorems \udefine{regular}.\retTwo


\begin{myIndent}\hTwo
   \blab{Proposition:} If $E \in \mathcal{M}_\mu$ and $\mu(E) < \infty$, then for all $\varepsilon > 0$, there is a set $A$ that is a finite union of open intervals such that $\mu(E \mathop{\Delta} A) < \varepsilon$.
   
   \begin{myIndent}\hThree
      Note that $E \mathop(\Delta) A = (E - A) \cup (A - E) = (E \cup A) - (E \cap A)$.\retTwo
      
      \exP The proof is \blab{Exercise 1.26}\\
      \exPP Let $U$ be an open set satisfying that $E \subseteq U$ and $\mu(U) \leq \mu(E) + \sfrac{\varepsilon}{2}$. Then, a fact about the standard metric space of $\mathbb{R}$ is that any open set is a countable union of disjoint open intervals.

      
      \begin{myIndent}\exPPP
         (writing this so I have it in my notes)\\
         To see this, given the open set $U \subseteq \mathbb{R}$, and any $x \in U$, define $I_x = (a, b)$\\ where $a = \inf({\alpha \in \overline{\mathbb{R}}} \mid (\alpha, x) \subset U)$ and $b = \sup({\beta \in \overline{\mathbb{R}}} \mid (x, \beta) \subset U)$.\retTwo

         Then consider the collection $\{I_x\}_{x \in U}$. Given $I_x = (a_x, b_x)$ and $I_y = (a_y, b_y)$, supposing $t \in I_x \cap I_y$, we must have that:
         
         {\centering $(\min(a_x, a_y), t] \cup [t, \max(b_x, b_y)) = (\min(a_x, a_y), \max(b_x, b_y)) \subseteq U$. \retTwo\par}

         Because of how we defined $a_x, b_x, a_y, b_y$, and the fact that obviously\\ $I_x, I_y \subseteq (\min(a_x, a_y), \max(b_x, b_y))$, we must have that $a_x = \min(a_x, a_y) = a_y$ and $b_x = \min(b_x, b_y) = b_y$. So, $I_x = I_y$.\retTwo

         Hence, $\{I_x\}_{x \in U}$ is a disjoint collection of intervals. Also, because each interval can be injectively mapped to a rational number inside that interval, we know this collection is countable.\retTwo
      \end{myIndent}

      So, let $(I_j)_{j \in \mathbb{N}}$ be a sequence of disjoint open intervals whose union is $U$. Then observe that $\lim\limits_{n \rightarrow \infty}(\mu(\bigcup\limits_{j=1}^n)I_j) = \mu(U)$. So, we can pick $n$ large enough that for\\ [-9pt] $A = \bigcup\limits_{j=1}^n I_j$, we have that $A$ is a finite union of open intervals and:

      {\centering $\mu(U) - \sfrac{\varepsilon}{2} \leq \mu(A) \leq \mu(U) + \sfrac{\varepsilon}{2}$ \retTwo\par}

      Now since $A \subseteq U$ and $E \subseteq U$, we know that $E - A \subseteq U - A$ and $A - E \subseteq U - A$. So:
      \begin{itemize}
         \item $\mu(E) + \mu(A - E) \leq \mu(E) + \mu(U - E) = \mu(U) \leq \mu(E) + \sfrac{\varepsilon}{2}$.
         \item $\mu(E - A) \leq \mu(U - A) = \mu(U) - \mu(A) \leq \mu(U) - \mu(U) + \sfrac{\varepsilon}{2} = \sfrac{\varepsilon}{2}$\\ \phantom{aaaaaaaaaaaaaaaaaaaaaaaaaaaaaaa} {\exPPP (because $\mu(U) \leq \mu(E) + \sfrac{\varepsilon}{2}$ is finite)}\retTwo
      \end{itemize}

      Since $\mu(E)$ is finite, we can subtract it out of the first bulleted inequality and add it to the second to get:

      {\centering $\mu(A \mathop{\Delta} E) = \mu(A - E) + \mu(E - A) \leq \sfrac{\varepsilon}{2} + \sfrac{\varepsilon}{2} = \varepsilon$ \newpage\par}

   \end{myIndent}
\end{myIndent}

\exOne
\blab{Exercise 1.28:} Let $F$ be increasing and right continuous, and let $\mu_F$ be the associated\\ measure. Then:

\begin{itemize}
   \item[(a)] $\mu_F({a}) = F(a) - F(a-)$ {\exPPP (Note that this is a special case of (c))}
   \item[(b)] $\mu_F([a, b)) = F(b-) - F(a-)$
   \item[(c)] $\mu_F([a, b]) = F(b) - F(a-)$
   \item[(d)] $\mu_F((a, b)) = F(b-) - F(a)$
\end{itemize}


\begin{myIndent}\exTwoP
   To start let's show (c).\\ Note that $[a, b] = \bigcap\limits_{n \in \mathbb{N}}(a - \frac{1}{n}, b]$ and $((a - \frac{1}{n}, b])_{n \in \mathbb{N}}$ is a decreasing sequence of sets.\\ [-8pt] So:

   {\centering\exPP $\mu_F([a, b]) = \lim\limits_{n \rightarrow \infty}(\mu_F((a - \frac{1}{n}, b])) = \lim\limits_{n \rightarrow \infty}(F(b) - F(a - \frac{1}{n})) = F(b) - F(a-)$\retTwo\par}

   Next, let's show (b).\\
   Note that $[a, b) = \bigcup\limits_{n \in \mathbb{N}}[a, b - \frac{1}{n}]$ and $([a, b- \frac{1}{n}])_{n \in \mathbb{N}}$ is an increasing sequence of sets.\\ [-8pt] So:

   {\centering\exPP $\mu_F([a, b)) = \lim\limits_{n \rightarrow \infty}(\mu_F([a, b-\frac{1}{n}])) = \lim\limits_{n \rightarrow \infty}(F(b - \frac{1}{n}) - F(a-)) = F(b-) - F(a-)$\retTwo\par}

   Finally, let's show (d).\\
   Note that $(a, b) = \bigcup\limits_{n \in \mathbb{N}}(a, b-\frac{1}{n}]$ and $((a, b- \frac{1}{n}])_{n \in \mathbb{N}}$ is an increasing sequence of sets.\\ [-8pt] So:

   {\centering\exPP $\mu_F((a, b)) = \lim\limits_{n \rightarrow \infty}(\mu_F((a, b-\frac{1}{n}])) = \lim\limits_{n \rightarrow \infty}(F(b - \frac{1}{n}) - F(a)) = F(b-) - F(a)$\retTwo\par}
\end{myIndent}

\mySepTwo\hOne

If $\mu = m$ is the Lesbesgue measure (a.k.a when $F(x) = x$), we denote the collection of Lesbesgue measurable sets $\mathcal{L}$ to be the completion of $\mathcal{B}_\mathbb{R}$.\retTwo

\begin{myIndent}\hTwo
   \blab{Theorem:} If $E \in \mathcal{L}$ and $s, r \in \mathbb{R}$, we define:

   {\centering 
   \begin{tabular}{l c l}
      $E + s = \{x + s \mid x \in E\}$, & \phantom{a} & $rE = \{rx \mid x \in E\}$.
   \end{tabular} \retTwo\par}

   Then $E + s, rE \in \mathcal{L}$, $m(E + s) = m(E)$ and $m(rE) = |r|m(E)$.

   
   \begin{myIndent}\hThree
      Proof: {\myComment(the professor skipped doing this proof and the book doesn't seem to prove this well. So I'm going on my own here.)}\retTwo

      Supposing $E + s \in \mathcal{L}$, define $m_s(E) = m(E + s)$. Similarly define $m^r(E) = m(rE)$ if $rE \in \mathcal{L}$.\retTwo

      Let $\mathcal{A}$ be the collection of disjoint unions of half-open-intervals, and define $m^*$ as the outer measure induced by $m|_{\mathcal{A}}$. Then you can fairly trivially (albeit with a lot of notation) show that $m^*(E + s) = m^*(E)$ and $m^*(rE) = |r|m^*(E)$. Also, we know that $m^*(E)$ must equal $m(E)$ for all $E \in \mathcal{B}_{\mathbb{R}}$ since any measure on $\mathcal{B}_{\mathbb{R}}$ is uniquely determined by it's restriction to $\mathcal{A}$ and $m^*$ is a measure on $\mathcal{B}_{\mathbb{R}}$ (see the theorem on the bottom of page 24).\newpage

      Next, note that $E \in \mathcal{B}_\mathbb{R} \Longrightarrow E + s, rE \in \mathcal{B}_\mathbb{R}$. The reason why is that as seen in a bonus proposition below, we have that a set $E \in \mathcal{B}_\mathbb{R}$ if and only if\\ it can be gained by taking a countable amount of countable unions, countable\\ intersections, and complements of sets of $\mathcal{A}$. But note that given any $A \in \mathbb{R}$:
      \begin{itemize}
         \item $(A + s)^\comp = A^\comp + s$ and $(rA)^\comp = rA^\comp$
         \item $\bigcup\limits_{j \in \mathbb{N}}(A_j + s) = (\bigcup\limits_{j \in \mathbb{N}}A_j) + s$ and $\bigcup\limits_{j \in \mathbb{N}}(rA_j) = r(\bigcup\limits_{j \in \mathbb{N}}A_j)$
         \item $\bigcap\limits_{j \in \mathbb{N}}(A_j + s) = (\bigcap\limits_{j \in \mathbb{N}}A_j) + s$ and $\bigcap\limits_{j \in \mathbb{N}}(rA_j) = r(\bigcap\limits_{j \in \mathbb{N}}A_j)$\retTwo
      \end{itemize}

      So, by replacing all the sets from $\mathcal{A}$ in our expression for $E$ with their translated or scaled sets which are also in $\mathcal{A}$, we can show that $E + s$ or $rE$ is also in $\mathcal{B}_\mathbb{R}$.\retTwo

      Combining that with the reasoning on the previous page, we know that\\ $m(E) = m(E + s)$ and $m(rE) = |r|m(E)$ for all $E \in \mathcal{B}_\mathbb{R}$. What's left to show now is that for all sets in the completion $\mathcal{L}$ of $\mathcal{B}_{\mathbb{R}}$, we have that $E \in \mathcal{L} \Longrightarrow E + s, rE \in \mathcal{L}$, $m(E + s) = m(E)$, and $m(rE) = |r|m(E)$.\retTwo

      \myComment This is not hard to show and I'm bored of this. So do it yourself.\retTwo
   \end{myIndent}
\end{myIndent}

{\teachComment
Some fairly trivial observations:
\begin{enumerate}
   \item $m(\{x\}) = 0$ for all $x \in \mathbb{R}$.
   \item All countable sets have a Lesbesgue measure of zero (this includes $\mathbb{Q}$).
   \item There exists uncountable sets with a Lesbesgue measure of zero (example: the Cantor set).
\end{enumerate}}

\mySepTwo

Given a collection of sets: $\mathcal{A}$, here is how to actually construct $\mathcal{M}(\mathcal{A})$.\retTwo

\begin{myIndent}\hTwo
   \blab{Bonus Proposition:} Let $\Omega$ be a minimal uncountable well-ordered set. Denoting $0$ as the least element of $\Omega$, define $\mathcal{E}_0 = \mathcal{A}$. Then for all $\alpha \in \Omega - \{0\}$, if $\alpha$ has an immediate predecessor $\beta$, define $\mathcal{E}_\alpha$ as the collection of all countable unions of sets and complements of sets in $\mathcal{E}_\beta$. Otherwise, define $\mathcal{E}_\alpha = \bigcup\limits_{\beta \in S_\alpha} \mathcal{E}_\beta$ {\myComment(where $S_\alpha$ is the set\\ [-9pt] $\{\beta \in \Omega \mid \beta < \alpha\}$)}.\retTwo

   Then $\mathcal{M} = \mathcal{M}(\mathcal{A}) = \bigcup\limits_{\alpha \in \Omega}\mathcal{E}_\alpha$.
   
   \begin{myIndent}\hThree
      Proof:\\
      Suppose for all $\beta \in S_\alpha$, we have that $\mathcal{E}_\beta \subseteq \mathcal{M}$.
      \begin{itemize}
         \item Suppose $\alpha$ has no immediate predecessor. If $S_\alpha = \emptyset$, then we know that\\ $\mathcal{E} = \mathcal{A} \subseteq \mathcal{M}$. Otherwise, we trivially have that:
         
         {\centering $\mathcal{E}_\alpha = \bigcup\limits_{\beta \in S_\alpha}\mathcal{E}_\beta \subseteq \mathcal{M}$.\\ [0pt]\par}

         \item If $\alpha$ has a predecessor $\beta^\prime$, then because $\mathcal{M}$ is closed under countable unions and complements and $\mathcal{E}_{\beta^\prime} \subseteq \mathcal{M}$, we know that $\mathcal{E}_\alpha \subseteq \mathcal{M}$.\retTwo
      \end{itemize}

      It follows by transfinite induction that $\mathcal{E}_\alpha \subseteq \mathcal{M}$ for all $\alpha \in \Omega$. So, we've shown that $\bigcup\limits_{\alpha \in \Omega}\mathcal{E}_\alpha \subseteq \mathcal{M}$.\newpage

      To show the other inclusion, it suffices to show that $\bigcup\limits_{\alpha \in \Omega}\mathcal{E}_\alpha$ is a $\sigma$-algebra (we\\ [-6pt] know it contains $\mathcal{A}$ since $\mathcal{E}_0 = \mathcal{A}$).\retTwo

      Firstly, note that if $A \in \mathcal{E}_\alpha$ for any $\alpha \in \Omega$, then denoting $\beta$ be the immediate\\ successor of $\alpha$, we have that $A^\comp \in \mathcal{E}_\beta$.\retTwo

      Next, suppose $(A_j)_{j \in \mathbb{N}}$ is a sequence of sets in $\bigcup\limits_{\alpha \in \Omega}\mathcal{E}_\alpha$ satisfying that $A_j \in \mathcal{E}_{\alpha_j}$ and\\ [-6pt] $\alpha_j \in \Omega$ for all $j \in \mathbb{N}$.\retTwo

      Now every countable sequence in $\Omega$ has a supremum. So let $\alpha^\prime$ be the supremum of the sequence $(\alpha_j)_{j \in \mathbb{N}}$. Next, let $\beta^\prime$ be the least element of $\Omega$ which is both greater than $\alpha^\prime$ and has no immediate successor. That way $\mathcal{E}_{\alpha_j} \subseteq \mathcal{E}_{\beta^\prime}$ for all $j \in \mathbb{N}$. Finally, let $\gamma^\prime$ be the successor of $\beta^\prime$. Then:

      {\centering $\bigcup\limits_{j \in \mathbb{N}}A_j \in \mathcal{E}_{\gamma^\prime} \subseteq \bigcup\limits_{\alpha \in \Omega}\mathcal{E}_\alpha$ \retTwo\par}

      So, we've shown that $\bigcup\limits_{\alpha \in \Omega}\mathcal{E}_\alpha$ is a $\sigma$-algebra containing $\mathcal{A}$.\retTwo
   \end{myIndent}
\end{myIndent}

It follows from this proposition that if a set is in $\mathcal{M}(\mathcal{A})$, then it can be gained through a procedure of countably many steps of taking countable unions, intersections, and complements of sets in $\mathcal{A}$.

\mySepTwo 

\blab{Measurable Functions:}\\

$f: (X, \mathcal{M}) \longrightarrow (Y, \mathcal{N})$ is called \udefine{$(\mathcal{M}, \mathcal{N})$ measurable} if $f^{-1}(E) \in \mathcal{M}$ for all $E \in \mathcal{N}$.\retTwo


\begin{myIndent}\hTwo
   \blab{Proposition:} If $\mathcal{N}$ is generated by $\mathcal{E}$, then $f: (X, \mathcal{M}) \longrightarrow (Y, \mathcal{M})$ is $(\mathcal{M}, \mathcal{N})$ measurable if and only if $f^{-1}(E) \in \mathcal{M}$ for all $E \in \mathcal{E}$.

   \begin{myIndent}\hThree
      Proof:\\
      The rightward implication is trivial. Meanwhile suppose the right statement holds. Then consider the collection $\{E \subseteq Y \mid f^{-1}(E) \in \mathcal{M} \}$. By assumption we know it contains $\mathcal{E}$.\retTwo

      Firstly, this collection is closed under countable unions because:
      
      {\centering $f^{-1}(\bigcup\limits_{j \in \mathbb{N}}A_j) = \bigcup\limits_{j \in \mathbb{N}}f^{-1}(A_j)$.\retTwo\par}

      Secondly, this collection is closed under complements. To prove this, first note that since $\emptyset \in \mathcal{E}$, we have that $Y - \emptyset = Y$ is a finite union of sets in $\mathcal{E}$. It follows then that $F^{-1}(Y) \in \mathcal{M}$. Next consider any $E \subseteq Y$ satisfying that $f^{-1}(E) \in \mathcal{M}$. Then $f^{-1}(E^\comp) = f^{-1}(Y) - f^{-1}(E) \in \mathcal{M}$.\retTwo

      So, we've shown that the collection $\{E \subseteq Y \mid f^{-1}(E) \in \mathcal{M} \}$ is a $\sigma$-algebra\\ containing $\mathcal{E}$. It follows that it contains $\mathcal{N} = \mathcal{M}(\mathcal{E})$.\newpage
   \end{myIndent}

   \blab{Corollary:} If $X, Y$ are topological spaces, then every continuous function\\ $f: X \longrightarrow Y$ is $(\mathcal{B}_X, \mathcal{B}_y)$ measurable.\retTwo
\end{myIndent}

We say a function $f: (X, \mathcal{M}) \longrightarrow \mathbb{R}$ (or $\mathbb{C}$ or $\mathbb{R}^n$) is \udefine{$\mathcal{M}$-measurable} if it is\\ $(\mathcal{M}, \mathcal{B}_{\mathbb{R}}$ (or $\mathcal{B}_{\mathbb{C}}$ or $\mathcal{B}_{\mathbb{R}^n}$)$)$ measurable. If it's obvious what $\mathcal{M}$ is, we can omit\\ saying it.
\begin{myIndent}
   In other words, the measurable sets of the range of a real or complex function is assumed to be the collection of Borel sets.\retTwo
\end{myIndent}

A function $f: \mathbb{R} \longrightarrow \mathbb{R}$ is \udefine{Lesbesgue measurable} if $f$ is $(\mathcal{L}, \mathcal{B}_\mathbb{R})$ measurable.\retTwo

Meanwhile, $f: \mathbb{R} \longrightarrow \mathbb{R}$ is \udefine{Borel measurable} if $f$ is $(\mathcal{B}_{\mathbb{R}}, \mathcal{B}_\mathbb{R})$ measurable.\\[6pt]

\mHeader{Lecture 7 Notes: 10/17/2024}

\begin{myIndent}\exTwo
   \blab{Exercise:} Suppose $f: (X, \mathcal{M}) \longrightarrow (Y, \mathcal{N})$ is $(\mathcal{M}, \mathcal{N})$ measurable and\\ $g: (Y, \mathcal{N}) \longrightarrow (Z, \mathcal{O})$ is $(\mathcal{N}, \mathcal{O})$ measurable. Then $g \circ f$ is $(\mathcal{M}, \mathcal{O})$ measurable.

   \begin{myIndent}
      \exPP Proof:\\
      Given any $E \in \mathcal{O}$, because $g$ is measurable, we know that $g^{-1}(E) \in \mathcal{N}$. Therefore, because $f$ is measurable, we know $f^{-1}(g^{-1}(E)) = (g \circ f)^{-1}(E) \in \mathcal{M}$.\retTwo
   \end{myIndent}

   \hTwo\blab{Proposition:} Suppose $(X, \mathcal{M})$ is a measurable space. Given $f: X \longrightarrow \mathbb{R}$, the following are equivalent:
   \begin{itemize}
      \item $f$ is measurable.
      \item $f^{-1}((a, \infty)) \in \mathcal{M}$ for all $a \in \mathbb{R}$.
      \item $f^{-1}([a, \infty)) \in \mathcal{M}$ for all $a \in \mathbb{R}$.
      \item $f^{-1}((-\infty, a)) \in \mathcal{M}$ for all $a \in \mathbb{R}$.
      \item $f^{-1}((-\infty, a]) \in \mathcal{M}$ for all $a \in \mathbb{R}$.\retTwo
   \end{itemize}

   \begin{myIndent}\hThree
      Proof:\\
      It's trivial that the first bullet point implies the others. Meanwhile, to go from any other bullet point back to the first, recall exercise 1.2 and the proposition on the bottom of page 39.\retTwo
   \end{myIndent}\retTwo
\end{myIndent}

Suppose $(Y_\alpha, \mathcal{N}_\alpha)_{\alpha \in A}$ is a collection of measurable spaces. Then given a set $X$ and functions $f_\alpha: X \longrightarrow Y_\alpha$ for all $\alpha \in A$, there exists a smallest $\sigma$-algebra $\mathcal{M}$ on $X$ such that each $f_\alpha$ is measurable. Specifically:

{\centering $\mathcal{M} = \mathcal{M}(\{f_\alpha^{-1}(E_\alpha) \mid \alpha \in A \text{ and } E_\alpha \in \mathcal{N}_\alpha\})$.\retTwo\par}

Note that if $X = \prod\limits_{\alpha \in A}Y_\alpha$ and $f_\alpha = \pi_\alpha$ for all $\alpha \in A$, then the $\mathcal{M}$ defined above is\\ [-8pt] equal to $\bigotimes\limits_{\alpha \in A} \mathcal{N}_\alpha$.\newpage

\begin{myIndent}\hTwo
   \blab{Proposition:} Suppose $(Y_\alpha, \mathcal{N}_\alpha)_{\alpha \in A}$ is a collection of measurable spaces. Then\\ define $Y = \prod\limits_{\alpha \in A}Y_\alpha$ and $\mathcal{N} = \bigotimes\limits_{\alpha \in A}\mathcal{N}_\alpha$.\retTwo

   We claim $f: (X, \mathcal{M}) \longrightarrow (Y, \mathcal{N})$ is $(\mathcal{M}, \mathcal{N})$ measurable if and only if $f_\alpha \coloneq \pi_\alpha \circ f$ is $(\mathcal{M}, \mathcal{N}_\alpha)$ measurable for all $\alpha \in A$.
   
   \begin{myIndent}\hThree
      Proof:\\
      We already showed at the beginning of lecture that the composition of measurable functions is measurable. So, the rightward implication is obvious. Conversely, if $f_\alpha \coloneq \pi_\alpha \circ f$ is $(\mathcal{M}, \mathcal{N}_\alpha)$ measurable for all $\alpha \in A$, then given any $E_\alpha \in \mathcal{N}_\alpha$ for any $\alpha \in A$, we know that $f_\alpha^{-1}(E_\alpha) = f^{-1}(\pi_\alpha^{-1}(E_\alpha)) \in \mathcal{M}$. Now $\mathcal{N}$ is generated by the collection of all such $\pi_\alpha^{-1}(E_\alpha)$. So by the proposition on the bottom of page 39, we know that $f$ is $(\mathcal{M}, \mathcal{N})$ measurable.\retTwo
   \end{myIndent}

   \blab{Corollary:} $f : X \longrightarrow \mathbb{C}$ is measurable if and only if $\rea(f)$, $\ima(f)$ are measurable.\retTwo
\end{myIndent}

In the extended real numbers $\overline{\mathbb{R}}$, we define $\mathcal{B}_{\overline{\mathbb{R}}} = \{E \in \overline{\mathbb{R}} \mid E \cap \mathbb{R} \in \mathcal{B}_{\mathbb{R}}\}$. In\\ other words, $\mathcal{B}_{\overline{\mathbb{R}}} = \{A, A \cup\{\infty\}, A \cup\{-\infty\}, A \cup\{\infty, -\infty\} \mid A \in \mathcal{B}_{\mathbb{R}} \}$.\retTwo

We define $f: (X, \mathcal{M}) \longrightarrow \overline{\mathbb{R}}$ to be $\mathcal{M}$-measurable if $f$ is $(\mathcal{M}, \mathcal{B}_{\overline{\mathbb{R}}})$ measurable.\retTwo

Given a measurable space $(X, \mathcal{M})$ and set $E \in \mathcal{M}$, we say $f: X \longrightarrow \mathbb{R}$ (or $\mathbb{C}$ or $\mathbb{R}^n$) is \udefine{measurable on $E$} if $f^{-1}(B) \cap E \in \mathcal{M}$ for each Borel set $B$.\retTwo

\begin{myIndent}\exOne
   \blab{Exercise 2.1:} Suppose $(X, \mathcal{M})$ is a measurable space. Let $f: X \longrightarrow \overline{\mathbb{R}}$ and\\ $Y = f^{-1}(\mathbb{R})$. Then $f$ is measurable if and only if $f^{-1}(\{-\infty\}) \in \mathcal{M}$,\\ $f^{-1}(\{\infty\}) \in M$, and $f$ is measurable on $Y$.\retTwo

   \begin{myIndent}\exTwoP
      The forward implication is trivial. After all, $\{\infty\}$ and $\{-\infty\}$ are in $\mathcal{B}_{\overline{\mathbb{R}}}$. So, $f$ being measurable implies $f^{-1}(\{-\infty\}) \in \mathcal{M}$ and $f^{-1}(\{\infty\}) \in \mathcal{M}$. Also,\\ we know $\mathbb{R} \in \mathcal{B}_{\overline{\mathbb{R}}}$. So if $f^{-1}(B) \in \mathcal{M}$ for all $B \in \mathcal{B}_{\overline{\mathbb{R}}}$, then: 
      
      {\center $f^{-1}(B) \cap Y = f^{-1}(B) \cap f^{-1}(\mathbb{R}) \in \mathcal{M}$ for all $B \in \mathcal{B}_{\overline{\mathbb{R}}}$.\retTwo\par}

      Now suppose $f^{-1}(\{-\infty\}), f^{-1}(\{\infty\}) \in \mathcal{M}$ and $f$ is measurable on $Y$. Then given any set $B \in \mathcal{B}_{\overline{\mathbb{R}}}$, we have that:

      {\center $f^{-1}(B) = f^{-1}(B \cap \{-\infty\}) \cup f^{-1}(B \cap \{\infty\}) \cup (f^{-1}(B \cap \mathbb{R}))$ \retTwo\par}

      Note that: $f^{-1}(B \cap \mathbb{R}) = f^{-1}(B) \cap Y$. So, since we know that\\ $f^{-1}(B \cap \{-\infty\})$, $f^{-1}(B \cap \{\infty\})$, and $(f^{-1}(B \cap \mathbb{R}))$ are in $\mathcal{M}$ for all $B \in \mathcal{B}_{\overline{\mathbb{R}}}$,\\ we thus know $f^{-1}(B) \in \mathcal{M}$ for all $B \in \mathcal{B}_{\overline{\mathbb{R}}}$, meaning $f$ is measurable.\newpage
   \end{myIndent}
\end{myIndent}

\mySepTwo

Now it's a bit frustrating but I don't think I have enough knowledge of topology right now to show this all fully rigorously (I'm running into the problem that $\overline{\mathbb{R}}$ is not a metric space). So right now I don't have a working definition of what it means for a subset of $\overline{\mathbb{R}}$ to be open in this context. I'll just leave these as notes and maybe in the future I'll come back and prove them.

\begin{myIndent}\myComment
   Also neither the professor nor Folland seem to care\dots
\end{myIndent}

\begin{myIndent}\hTwo 
   \blab{Lemma 1:} The collections $\{(a, \infty] \mid a \in \mathbb{R}\}$ and $\{[\infty, a) \mid a \in \mathbb{R}\}$ are both bases of the $\sigma$-algebra $\mathcal{B}_{\mathbb{R}}$.\retTwo

   \blab{Lemma 2:} Suppose $(X, \mathcal{M})$ is a measurable space. Given $f: X \longrightarrow \overline{\mathbb{R}}$, the following are equivalent:
   \begin{itemize}
      \item $f$ is measurable.
      \item $f^{-1}((a, \infty]) \in \mathcal{M}$ for all $a \in \mathbb{R}$.
      \item $f^{-1}([-\infty, a)) \in \mathcal{M}$ for all $a \in \mathbb{R}$.\retTwo
   \end{itemize}
\end{myIndent}

\mySepTwo

\begin{myIndent}\hTwo
   \blab{Proposition:} Suppose $f, g: X \longrightarrow \mathbb{C}$ are measurable. Then $f + g$ and $fg$ are measurable.

   \begin{myIndent}\hThree
      Proof:\\ [-6pt]
      Define $F(x) = (f(x), g(x))$. Then $F$ is $(X, \mathcal{B}_{\mathbb{C}^2} = \bigotimes\limits_{i=1}^2 \mathcal{B}_{\mathbb{C}})$ measurable by the\\ [-6pt] above proposition.\retTwo

      Also, defining $G(z, w) = z + w$ and $H(z, w) = zw$ for all $z, w \in \mathbb{C}$, we have that $G$ and $H$ are continuous functions and thus $(\mathcal{B}_{\mathbb{C}^2}, \mathcal{B}_{\mathbb{C}})$ measurable. Since the composition of measurable functions is also measurable, this proposition follows.\retTwo
   \end{myIndent}

   \blab{Proposition:} Suppose $(f_j)_{j \in \mathbb{N}}$ is a sequence of measurable functions from $(X, \mathcal{M})$\\ to $\overline{\mathbb{R}}$. Then the following functions are measurable.
   
   {\centering 
   \begin{tabular}{l l}
      $g_1(x) = \sup\limits_{j \in \mathbb{N}}(f_j(x))$,& $g_2(x) = \inf\limits_{j \in \mathbb{N}}(f_j(x))$,\\ [12pt] $g_3(x) = \limsup\limits_{j \rightarrow \infty} (f_j(x))$,& $g_4(x) = \liminf\limits_{j \rightarrow \infty} (f_j(x))$.
   \end{tabular}\retTwo\par}

   \begin{myIndent}\hThree
      Proof:\\ [-3pt]
      Note that $g_1^{-1}((a, \infty]) = \bigcup\limits_{j=1}^\infty f_j^{-1}((a, \infty])$ and $g_2^{-2}([-\infty, a)) = \bigcup\limits_{j=1}^\infty f_j^{-1}([-\infty, a))$.
      
      \begin{myIndent}\hFour
         I'll show where the first of those equalities come from. Showing the other would be basically the same.\newpage

         Suppose $x \in \bigcup\limits_{j=1}^\infty f_j^{-1}((a, \infty])$. Then there exists $f_j$ with $f_j(x) > a$. It follows\\ [-8pt]\phantom{aaaaaaaaaaaaaaaa} that $g_1(x) \geq f_j(x) > a$ and thus $x \in g_1^{-1}((a, \infty])$. So:
         
         {\centering\phantom{aaaaaaaaaaaaaaaaaaaaaa}$\bigcup\limits_{j=1}^\infty f_j^{-1}((a, \infty]) \subseteq g_1^{-1}((a, \infty])$.\retTwo\par}

         On the other hand, suppose $x \in g_1^{-1}((a, \infty])$. We'll first address if $x \neq \infty$.\\ [5pt] Then letting $0 < \varepsilon < x - a$, we know there exists $f_j$ in the sequence such that $f_j(x) \geq g(x) - \varepsilon$. It follows that $x \in \bigcup\limits_{j=1}^\infty f_j^{-1}((a, \infty])$. On the other hand, if $x = \infty$, then we know there exists $f_j$ satisfying $f_j(x) > M$ for any $M \in \mathbb{R}$. Setting $M = a$, we get that $x \in \bigcup\limits_{j=1}^\infty f_j^{-1}((a, \infty])$. In both cases, we've thus shown that:

         {\centering $g_1^{-1}((a, \infty]) \subseteq \bigcup\limits_{j=1}^\infty f_j^{-1}((a, \infty])$. \retTwo\par}
      \end{myIndent}

      Since all $f_j$ are measurable, it follows that $g_1^{-1}((a, \infty])$ and $g_2^{-1}([\infty, a))$ are in $\mathcal{M}$ for all $a \in \mathbb{R}$. By our lemma on the previous page, we thus have that $g_1$ and $g_2$ are measurable.\retTwo

      To show that $g_3$ is measurable, define $h_k(x) = \sup\limits_{n \geq k}(f_n(x))$ for all $k \in \mathbb{N}$. Then note\\ [-6pt] that $g_3(x) = \inf\limits_{k \in \mathbb{N}}(h_k(x))$.\retTwo

      By our previous reasoning about $g_1$, we know all $h_k$ are measurable functions. Hence, it follows that $g_3$ is measurable by the same reasoning that $g_2$ is measurable. An\\ analogous argument shows that $g_4$ is measurable.\retTwo
   \end{myIndent}

   Also if $f(x) = \lim\limits_{j \rightarrow \infty} (f_j(x))$ exists for all $x \in X$, then $f(x)$ is measurable. This is\\ because $f(x)$ existing for all $x \in X$ means that $f(x) = g_3(x) = g_4(x)$.\\ [8pt]

   \blab{Corollary 1:} If $f, g: X \longrightarrow \overline{\mathbb{R}}$ are measurable, then $\max(f, g)$ and $\min(f, g)$ are\\ measurable.\retTwo

   \blab{Corollary 2:} If $(f_j)_{j \in \mathbb{N}}$ is a sequence of measurable functions from $X$ to $\mathbb{C}$, and $f(x) = \lim\limits_{j \rightarrow \infty}(f_j(x))$ for all $x \in X$, then $f$ is measurable.

   \begin{myIndent}\hThree
      To show this, apply the corollary on page 41 to consider the components of $f$ and of each $f_j$. Then use the previous proposition.\retTwo
   \end{myIndent}
\end{myIndent}

Here are two important decompositions of functions:
\begin{enumerate}
   \item Given a function $f: X \longrightarrow \overline{\mathbb{R}}$, define $f^+(x) \coloneq \max(f(x), 0)$ and\\ $f^-(x) \coloneq \max(-f(x), 0)$. Then $f = f^+(x) - f^-(x)$ and\\ $|f| = f^+(x) + f^-(x)$.\newpage
   \item Given $f: X \longrightarrow \mathbb{C}$, we define it's \udefine{polar decomposition}:
   
   {\centering $f(z) = \sgn(f(z))|f(z)|$ where $\sgn(z) = \left\{
   \begin{matrix}
      \frac{z}{|z|} & \text{ if } z \neq 0 \\ 0 & \text{ if } z = 0
   \end{matrix}\right.$ \retTwo\par}
\end{enumerate}

A function $f: X \longrightarrow \mathbb{C}$ is called \udefine{simple} if it is measurable and has a finite image. A standard representation of $f$ is given by:

{\centering $\sum\limits_{j=1}^n z_j \chi_{E_j}$ \retTwo\par}

\dots where $E_j = f^{-1}(\{z_j\})$ and $\chi_E = \left\{
\begin{matrix}
   1 & \text{ if } x \in E\\
   0 & \text{ otherwise }
\end{matrix}\right.$\retTwo

\begin{myIndent}\hTwo
   \blab{Observation:} If $f, g: X \longrightarrow \mathbb{C}$ are simple functions, then so are $f + g$\\ and $fg$.\retTwo

   \blab{Theorem:} Let $(X, \mathcal{M})$ be a measurable space.
   \begin{enumerate}
      \item If $f: X \longrightarrow [0, \infty]$ be a measurable function,  then there exists a sequence $(\phi_n)_{n \in \mathbb{N}}$ of simple functions such that $0 \leq \phi_1 \leq \ldots \leq f$ and $\phi_n \rightarrow f$ pointwise as $n \rightarrow \infty$. If $f$ is bounded, then $\phi_n \rightarrow f$ uniformly.\retTwo
      
      \item If $f: X \longrightarrow \mathbb{C}$ is measurable, then there exists a sequence $(\phi_n)_{n \in \mathbb{N}}$ of simple functions such that $0 \leq |\phi_1| \leq \ldots \leq |f|$ and $\phi_n \rightarrow f$ pointwise. If $f$ is bounded, then $\phi_n \rightarrow f$ uniformly.
   \end{enumerate}
   
   \begin{myIndent}\hThree
      Proof:
      \begin{enumerate}
         \item For all $n \in \mathbb{N}$, define $\phi_n$ as follows:
         \begin{myIndent}
            Define $E_n^{k} = f^{-1}((k2^{-n}, (k+1)2^{-n}])$ for all $k \in \{0, \ldots, 2^{2n} - 1\}$. Also\\ define $F_n = f^{-1}((2^n, \infty])$. Then finally, set:

            {\centering $\phi_n = \sum\limits_{k=0}^{2^{2n} - 1}k2^{-n}\chi_{E_n^{k}} + 2^m\chi_{F_m}$ \retTwo\par}
         \end{myIndent}

         To check that $\phi_n \leq \phi_{n + 1}$, first consider the case that $f(x) \leq 2^n$. Note that we can rewrite the expression $\phi_n(x) = k2^{-n} < f(x) \leq (k + 1)2^{-n}$ as:
         
         {\centering $(2k)2^{-(n+1)} < f(x) \leq (2k+2)2^{-(n+1)}$.\retTwo\par}
         
         But then note that $\phi_{n+1}(x)$ is either $(2k)2^{-(n + 1)}$ or $(2k + 1)2^{-(n + 1)}$. So,\\ $\phi_{n + 1}(x) \geq \phi_n(x)$. As for if $f(x) > 2^n$, then note that $\phi_{n+1}(x) \geq 2^n = \phi_n(x)$\\ because $E_{n+1}^{(2^{2n + 1})} = f^{-1}(((2^{2n + 1})2^{-(n+1)}, (2^{2n + 1} + 1)2^{-(n+1)}])$\retTwo

         Additionally, note that $\phi_n(x) \leq f(x)$ for all $x \in X$. And, when $f(x) < 2^{n}$, we clearly have that $|f(x) - \phi_n(x)| < 2^{-n}$. So it's clear that $(\phi_n)_{n \in \mathbb{N}}$ meets our convergence requirements.\newpage

         \item Given $f(x) = g(x) + ih(x)$, apply part 1 of this theorem to $g^+$, $g^-$, $h^+$, and $h^-$ to get the sequences $(\psi^+_n)$, $(\psi^-_n)$, $(\zeta^+_n)$, and $(\zeta^-_n)$ respectively of simple functions. Then letting $\phi_n = (\psi^+_n - \psi^-_n) + i(\zeta^+_n - \zeta^-_n)$ for all $n$ satisfies our theorem.\retTwo
      \end{enumerate}
   \end{myIndent}
\end{myIndent}

\mySepTwo

We've mostly been working without regard to a particular measure. Here are some theorems which do require that we are working with a complete measure space $(X, \mathcal{M}, \mu)$:\retTwo

\exOne\blab{Exercise 2.10:} Prove the following are true if and only if $(X, \mathcal{M}, \mu)$ is complete:
\begin{enumerate}
   \item[(a)] Suppose $f$ is measurable and $f = g$ $\mu$-a.e., then $g$ is measurable.
   
   \begin{myTindent}\begin{myIndent}\myComment
      Note that if we say $f$ is "measurable", it's implied that $f$ is a\\ function to $\mathbb{R}$, $\overline{\mathbb{R}}$, $\mathbb{C}$, or $\mathbb{R}^n$. We'll work here in $\mathbb{R}$ (and $\overline{\mathbb{R}}$) since this can all be extended to $\mathbb{C}$ and $\mathbb{R}^n$ by working component wise.
   \end{myIndent}\end{myTindent}
   
   \begin{myIndent}\exTwoP
      Proof:\\
      ($\Longrightarrow$)\\
      Suppose $(X, \mathcal{M}, \mu)$ is complete. Then given such an $f$ and $g$ above, we know $\{x \mid f(x) \neq g(x)\}$ is a subset of a null set, meaning that all it's subsets are measurable. Next note that given any $a \in \mathbb{R}$, we have that:

      {\centering\exPP $\{x \mid g(x) > a\} = \{x \mid g(x) > a, f(x) = g(x)\} \cup \{x \mid g(x) > a, f(x) \neq g(x)\}$. \retTwo\par}

      Importantly, $\{x \mid g(x) > a, f(x) \neq g(x)\}$ is measurable since it's a subset of $\{x \mid f(x) \neq g(x)\}$. On the other hand, note that:
      
      {\centering\exPP $\{x \mid g(x) > a, f(x) = g(x)\} = \{x \mid f(x) > a\} - \{x \mid f(x) > a, f(x) \neq g(x)\}$ \retTwo\par}
      
      Thus $\{x \mid g(x) > a, f(x) = g(x)\}$ is measurable because $f$ is a measurable function and $\{x \mid f(x) > a, f(x) \neq g(x)\}$ is a subset of $\{x \mid f(x) \neq g(x)\}$. Hence, we've shown that $\{x \mid g(x) > a\}$ is measurable for all $a \in \mathbb{R}$, meaning $g$ is measurable.\retTwo

      ($\Longleftarrow$)\\
      Let $N$ be a null set of $(X, \mathcal{M}, \mu)$ and $F \subseteq N$. Then, let $f = \chi_{N}$ and $g = \chi_F$. Clearly, $f(x) = g(x)$ for all $x \in X - N$. And looking back at the definition of something being true $\mu$-a.e., that means that $f(x) = g(x)$ $\mu$-a.e. So by assumption, we know $g$ is measurable, meaning that $g^{-1}(1) = F \in \mathcal{M}$. 
   \end{myIndent}

   \item[(b)] If $f_n$ is measurable for all $n \in \mathbb{N}$ and $\lim\limits_{n \rightarrow \infty}f_n(x) = f(x)$ $\mu$-a.e., then $f$ is measurable.
   
   \begin{myIndent}\exTwoP
      ($\Longrightarrow$)\\
      Given $a \in \mathbb{R}$, note that:
      
      {\centering\exPP $\{x \mid f(x) > a\} = \{x \mid f(x) > a, f_n(x) \rightarrow f(x)\} \cup \{x \mid f(x) > a, f_n(x) \not\rightarrow f(x)\}$.\newpage\par}

      The first of those two sets is measurable by the proposition on pages 42 and 43. The second is a subset of a null set. So since $(X, \mathcal{M}, \mu)$ is complete, it follows that $\{x \mid f(x) > a\} \in \mathcal{M}$.\retTwo

      ($\Longleftarrow$)\\
      Let $N$ be a null set of $(X, \mathcal{M}, \mu)$ and $F \subseteq N$. Then define $f_n = 0$ for all $n \in \mathbb{N}$ and $f = \chi_F$. Then $f_n(x) \rightarrow f(x)$ for all $x \in N^\comp$, meaning $f_n(x) \rightarrow f(x)$ $\mu$-a.e. So by assumption, $f$ is measurable, meaning that $f^{-1}(1) = F \in \mathcal{M}$. 
      \retTwo
   \end{myIndent}
\end{enumerate}

\hOne
\begin{myIndent}\hTwo
   \blab{Proposition:} If $(X, \mathcal{M}, \mu)$ is a measure space and $(X, \overline{\mathcal{M}}, \overline{\mu})$ is its completion. If $f$ is $\overline{\mathcal{M}}$-measurable, then there exists $g$ that is $\mathcal{M}$-measurable such that $f = g$ $\overline{\mu}$-a.e.

   \begin{myIndent}\hThree
      If $f = \chi_E$ for $E \in \mathcal{M}$, then this is obvious.
      \begin{myIndent}\hFour
         $E = E^\prime \cup F^\prime$ where $E^\prime \in \mathcal{M}$ and $F^\prime \subseteq N^\prime \in \mathcal{M}$ with $\mu(N^\prime) = 0$. So define $g = \chi_{E^\prime}$. Then $g$ is $\mathcal{M}$-measurable and $f(x) = g(x)$ for all $x \in (N^\prime)^\comp$. So $f(x) = g(x)$ $\mu$-a.e. and thus also $\overline{\mu}$-a.e.\retTwo
      \end{myIndent}

      Now let $f$ be any $\overline{\mathcal{M}}$-measurable function and let $(\phi_n)_{n\in\mathbb{N}}$ be a sequence of $\overline{\mathcal{M}}$-measurable simple functions converging pointwise to $f$. Next, for each $n$, let $\psi_n$ be an $\mathcal{M}$-simple function satisfying that $\psi_n(x) = \phi_n(x)$ except on a null set $E_n$ of $\overline{\mathcal{M}}$.
      
      \begin{myTindent}\myComment
         Remember above definition for why we can do that.\retTwo
      \end{myTindent}

      Choose a $\mu$-null set $N \in \mathcal{M}$ such that $\bigcup\limits_{n \in \mathbb{N}}E_n \subseteq N$. Then define $g = \lim\limits_{n \rightarrow \infty} \chi_{N^\comp}\psi_n(x)$.\retTwo

      By the second corollary on page 43, we thus know that $g$ is $\mathcal{M}$-measurable with $g(x) = f(x)$ for all $x \in N^\comp$.
   \end{myIndent}
\end{myIndent}

\mySepTwo

\mHeader{Lecture 8 Notes: 10/22/2024}

Fixing our measure space as $(X, \mathcal{M}, \mu)$, define $L^+ \coloneq \{f: X \to [0, \infty] : f \text{ is measurable}\}$. We say the function $\phi$ is \udefine{simple} if $\phi = \sum_{j=1}^m a_j \chi_{E_j}$ where all $a_j \in \mathbb{C}$ and all $E_j \in \mathcal{M}$.\retTwo

If $\phi = \sum_{j=1}^m a_j\chi_{E_j}$ where all $a_j \in [0, \infty]$ and all $E_j \in \mathcal{M}$ are disjoint, we define:

{\centering $\int \phi = \int \phi \df \mu = \int \phi(x) \df \mu(x) \coloneq \sum\limits_{j=1}^m a_j \mu(E_j)$ \retTwo\par}

If $A \in \mathcal{M}$, then we define $\int_A \phi \coloneq \int \chi_A \phi$. Note that if $\phi, \psi$ are simple functions in $L^+$, then you can show:

\begin{enumerate}\hTwo
	\item $c \geq 0 \Longrightarrow \int c \phi = c \int \phi$
\item $\int(\phi + \psi) = \int \phi + \int \psi$
\item $\phi \leq \psi \Longrightarrow \int \phi \leq \int \psi$.
\end{enumerate}

\newpage

Given any $f \in L^+$, we now define:

{\centering$\int f \df \mu \coloneq \sup \left\{\int \phi \df \mu : 0 \leq \phi \leq f \text{ and } \phi \text{ is simple}\right\}$\retTwo\par}



\newpage






















% Homework Dungeon

\begin{myIndent}\hTwo 
   \blab{Proposition:} If $f \in L^+$ and $\int f < \infty$, then $\{x \mid f(x) = \infty\}$ is a null set and\\ $\{x \mid f(x) > 0\}$ is $\sigma$-finite.
   \begin{myIndent}\exTwo
      The proof is \blab{Exercise 2.12}:\exTwoP\\
      Firstly, suppose $\mu(\{x \mid f(x) = \infty\}) = \alpha > 0$. Then for all $n \in \mathbb{N}$, we can\\ define: $\phi_n = n\chi_{\{x \mid f(x) = \infty\}}$. Thus, $(\phi_n)$ is an increasing sequence of simple functions less than $f$ satisfying that $\int f \geq \int \phi_n = n\alpha$ for all $n$. It follows that $\int f \geq \lim\limits_{n \rightarrow \infty} (n\alpha) = \infty$, meaning that $\int f = \infty$.\retTwo
      
      Hence, we've shown that $\int f < \infty \Longrightarrow \mu(\{x \mid f(x) = \infty\}) = 0$.\retTwo

      Next, suppose that for some $\beta > 0$, we have that $\mu(\{x \mid f(x) > \beta\}) = \infty$. Then $\phi = \beta \chi_{\{x \mid f(x) > \beta\}}$ is a simple function less than $f$ satisfying that:
       
      {\centering $\infty = \beta\infty = \int \phi \leq \int f$.\retTwo\par}

      It follows that $\int f < \infty \Longrightarrow \mu(\{x \mid f(x) > \beta\}) < \infty$ for all $\beta > 0$.\retTwo

      Finally, note that $\{x \mid f(x) > 0\} = \bigcup\limits_{n \in \mathbb{N}}\{x \mid f(x) > \frac{1}{n}\}$ where each\\ [-8pt] $\mu(\{x \mid f(x) > \frac{1}{n}\}) < \infty$.\retTwo

      So, $\{x \mid f(x) > 0\}$ is $\sigma$-finite.\retTwo
   \end{myIndent}
\end{myIndent}


\retTwo


\exOne
\blab{Exercise 2.13:} Suppose $(f_n)_{n \in \mathbb{N}} \subset L^+$, $f_n \rightarrow f$ pointwise, and $\int f = \lim\limits_{n \rightarrow \infty} \int f_n < \infty$.\\ [-3pt] Then $\int_E f = \lim\limits_{n \rightarrow \infty} \int_E f_n$ for all $E \in \mathcal{M}$.\\ [-8pt]


\begin{myIndent}\exTwoP
   Firstly let $E \in \mathcal{M}$. Then $\int f\chi_E + \int f \chi_{E^\comp} = \int f$ and $\int f_n\chi_E + \int f_n\chi_{E^\comp} = \int f_n$ for all $n \in \mathbb{N}$. Based on that, we can show:

   {\centering $\limsup\limits_{n \rightarrow \infty} \int f_n\chi_E + \liminf\limits_{n \rightarrow \infty} \int f_n\chi_{E^\comp} = \lim\limits_{n \rightarrow \infty} \int f_n = \int f$ \retTwo\par}

   \begin{myIndent}\exPPP
      To see this, consider any subsequence $(\int f_{n_k}\chi_E)_{k \in \mathbb{N}}$ converging to $\limsup\limits_{n \rightarrow \infty} \int f_n\chi_E$.\retTwo

      Because $\int f$ is finite, we know there is $N_1 \in \mathbb{N}$ such that $\int f_n < \infty$ for all $n > N_1$.\\ In turn, we know that $\int f_n\chi_E < \infty$ for all $n > N_1$ and thus $\limsup\limits_{n \rightarrow \infty} \int f_n\chi_E < \infty$.\retTwo

      Now since $(\int f_{n_k}\chi_E)_{k \in \mathbb{N}}$ and $(\int f_n)_{n \in \mathbb{N}}$ have finite limits, given any $\varepsilon > 0$, we can pick $N_2 \in \mathbb{N}$ greater than $N_1$ such that for all $k \geq N_2$, we have that:

      {\centering $|\int f_{n_k}\chi_E - \limsup\limits_{n \rightarrow \infty} \int f_n\chi_E| < \sfrac{\varepsilon}{2}$ and $|\int f_{n_k} - \int f| < \sfrac{\varepsilon}{2}$ \retTwo\par}

      It follows that because $\int f_{n_k} - \int f_{n_k}\chi_E = \int f_{n_k}\chi_{E^\comp}$, we have that:\\ [-10pt]

      {\centering $(\int f - \limsup\limits_{n \rightarrow \infty} \int f_n\chi_E) - \varepsilon < \int f_{n_k}\chi_{E^\comp} < (\int f - \limsup\limits_{n \rightarrow \infty} \int f_n\chi_E) + \varepsilon$ \retTwo\par}

      So $\lim\limits_{k \rightarrow \infty} \int f_{n_k}\chi_{E^\comp}$ exists and:

      {\centering $\liminf\limits_{n \rightarrow \infty} \int f_n\chi_{E^\comp} \leq \lim\limits_{k \rightarrow \infty} \int f_{n_k}\chi_{E^\comp} = \int f - \limsup\limits_{n \rightarrow \infty} \int f_n\chi_E$.\newpage\par}

      Meanwhile, consider any subsequence $(\int f_{n_j}\chi_{E^\comp})_{j \in \mathbb{N}}$ converging to $\liminf\limits_{n \rightarrow \infty} \int f_n\chi_{E^\comp}$.\retTwo

      By analogous reasoning to before, we know that $\liminf\limits_{n \rightarrow \infty} \int f_n\chi_{E^\comp}$ is finte.\retTwo
      
      Also similarly to before, we can show that $\lim\limits_{j \rightarrow \infty} \int f_{n_j}\chi_{E}$ exists and:\\ [-8pt]

      {\centering $ \int f - \liminf\limits_{n \rightarrow \infty} \int f_n\chi_{E^\comp} = \lim\limits_{j \rightarrow \infty} \int f_{n_j}\chi_{E} \leq \limsup\limits_{n \rightarrow \infty} \int f_n\chi_{E}$ \retTwo\par}

      Finally, by rearranging terms, we get that:
      
      {\centering $\int f \leq \limsup\limits_{n \rightarrow \infty} \int f_n\chi_{E} + \liminf\limits_{n \rightarrow \infty} \int f_n\chi_{E^\comp} \leq \int f$.\retTwo\par}
   \end{myIndent}

   Next, note that $f_n\chi_E \rightarrow f\chi_E$ and $f_n \chi_{E^\comp} \rightarrow f\chi_{E^\comp}$ pointwise as $n \rightarrow \infty$. Thus by Fatou's lemma, we immediately have that:

   {\centering $\int f\chi_E = \int \lim\limits_{n \rightarrow \infty}(f_n\chi_E) \leq \liminf\limits_{n\rightarrow \infty} \int f_n\chi_E$  \retTwo\par}

   Also, more round-aboutly we get that:

   {\centering
   \begin{tabular}{l}
      $\int f \chi_E = \int f - \int f \chi_{E^\comp} = \int f - \int \lim\limits_{n \rightarrow \infty} f_n\chi_{E^\comp}$\\ [9pt]
      $\phantom{\int f \chi_E = \int f - \int f \chi_{E^\comp}} \geq \int f - \liminf\limits_{n \rightarrow \infty} \int f_n\chi_{E^\comp} = \limsup\limits_{n \rightarrow \infty} \int f_n\chi_E$
   \end{tabular}\retTwo\par}

   Thus, $\int f\chi_E \leq \liminf\limits_{n\rightarrow \infty} f_n\chi_E \leq \limsup\limits_{n\rightarrow \infty} f_n\chi_E \leq \int f\chi_E$.\retTwo And so, we have shown that $\lim\limits_{n \rightarrow \infty} \int f_n\chi_E$ exists and that $\int f\chi_E = \lim\limits_{n \rightarrow \infty} \int f_n\chi_E$
\end{myIndent}

\retTwo

However, this need not be true if $\int f = \lim\limits_{n \rightarrow \infty} \int f_n = \infty$.


\begin{myIndent}\exTwoP
   Suppose $(X, \mathcal{M}, \mu) = (\mathbb{R}, \mathcal{L}, m)$ and then define $f_n = \chi_{(-\infty, 0)} + \chi_{(n, n+1]}$ for all\\ $n \in \mathbb{N}$ and $f = \chi_{(-\infty, 0)}$. Then note that $f_n \rightarrow f$ pointwise as $n \rightarrow \infty$. Also, for\\ all $n \in \mathbb{N}$ we have that $\int f = \infty = \int f_n$. However, $E = [0, \infty) \in \mathcal{L}$ and $\int_E f = 0$\\ while $\int_E f_n = 1$ for all $n$. So, $\int_E f \neq \lim\limits_{n \rightarrow \infty} \int_E f_n$.\retTwo
\end{myIndent}

\retTwo

\blab{Exercise 2.14:} If $f \in L^+$, let $\lambda(E) = \int_E fd\mu$ for $E \in \mathcal{M}$. Then $\lambda$ is a measure on $\mathcal{M}$ and for any $g \in L^+$, $\int g d\lambda = \int fgd\mu$.

\begin{myIndent}\exTwoP
   To start, we show $\lambda$ is measure. Clearly, $\lambda(\emptyset) = \int_\emptyset fd\mu = \int f\chi_{\emptyset}d\mu = \int 0d\mu = 0$.\\ Meanwhile, suppose $(E_n)_{n \in \mathbb{N}}$ is a sequence of disjoint sets in $\mathcal{M}$ whose union is $E$.\retTwo
   
   Note that $\chi_E = \sum\limits_{n = 0}^\infty \chi_{E_n}$. So:\\ [-10pt]
   
   {\centering $\lambda(E) = \int_{E} fd\mu = \int f\chi_E d\mu = \int \left(\sum\limits_{n = 0}^\infty f\chi_{E_n}\right)d\mu = \sum\limits_{n=0}^\infty \int f\chi_{E_n}d\mu = \sum\limits_{n = 0}^\infty \lambda(E_n)$\retTwo\par}

   Hence, we've shown that $\lambda$ is a measure on $\mathcal{M}$.\newpage

   Now, in order to show that $\int g d\lambda = \int fgd\mu$ for all $g \in L^+$, let's first consider a\\ simple function $\phi \in L^+$.\retTwo

   Suppose $\phi = \sum\limits_{k=1}^N x_k \chi_{E_k}$ where each $E_k$ is disjoint. Then:

   {\centering 
   \begin{tabular}{l}
      $\int \sum\limits_{k=1}^N \chi_{E_k}d\lambda = \sum\limits_{k=1}^N x_k\lambda(E_k)$\\ [10pt]
      $\phantom{\int \sum\limits_{k=1}^N \chi_{E_k}d\lambda} = \sum\limits_{k=1}^N x_k\int f\chi_{E_k}d\mu = \sum\limits_{k=1}^N \int x_k f\chi_{E_k}d\mu$\\ [10pt]
      $\phantom{\int \sum\limits_{k=1}^N \chi_{E_k}d\lambda = \sum\limits_{k=1}^N x_k\int f\chi_{E_k}d\mu} = \int \sum\limits_{k=1}^N x_k f\chi_{E_k}d\mu = \int f\sum\limits_{k=1}^N x_k \chi_{E_k}d\mu$
   \end{tabular} \retTwo\par}

   Hence, $\int \phi d\lambda = \int f \phi d\mu$.\retTwo

   To extend this to any function $g \in L^+$, let $(\phi_n)_{n \in \mathbb{N}}$ be an increasing sequence of simple functions such that $\phi_n \rightarrow g$ pointwise as $n \rightarrow \infty$. Importantly, we also have that $f\phi_n \rightarrow fg$ pointwise as $n \rightarrow \infty$ with $f\phi_{n} \leq f\phi_{n + 1}$ for all $n \in \mathbb{N}$. So, applying the monotone convergence theorem, we have that:

   {\centering $\int gd\lambda = \lim\limits_{n \rightarrow \infty}(\int \phi_nd\lambda) = \lim\limits_{n \rightarrow \infty}(\int \phi_nfd\mu) = \int fg d\mu$ \retTwo\par}
\end{myIndent}

\newpage

\blab{Exercise 2.25:} Let $f(x) = x^{-\sfrac{1}{2}}$ if $0 < x < 1$ and $f(x) = 0$ otherwise. Let $(r_n)_{n \in \mathbb{N}}$ be an enumeration of the rational numbers, and set $g(x) = \sum\limits_{n = 1}^\infty 2^{-n}f(x - r_n)$. \\ [-14pt]
\begin{enumerate}
   \item[(a)] $g \in L^1(m)$ and in particular $g < \infty$ a.e.
   
   \begin{myIndent}\exTwoP
      Note that $\int |2^{-n}f(x - r_n)|dm = 2^{-n}\int |f(x - r_n)|dm$ since $2^{-n} > 0$. Next, note that by the monotone convergence theorem, if $(a_k)_{k \in \mathbb{N}}$ is any decreasing sequence of numbers converging to $r_n$, then:

      {\centering 
      \begin{tabular}{l}
         $\int |f(x - r_n)|dm = \lim\limits_{k \rightarrow \infty}\int |f(x - r_n)|\chi_{\{x \mid x > a_k\}}dm$\\
         
         $\phantom{\int |f(x - r_n)|dm} = \lim\limits_{k \rightarrow \infty}\int_{a_k}^{r_n + 1}|f(x - r_n)|dx$\\

         $\phantom{\int |f(x - r_n)|dm} = \lim\limits_{a \rightarrow r_n+}\int_{a}^{r_n + 1}f(x - r_n)dx$\\

         $\phantom{\int |f(x - r_n)|dm} = \lim\limits_{a \rightarrow 0+}\int_{a}^{1}f(x)dx = \lim\limits_{a \rightarrow 0+}\int_{a}^{1}x^{-\sfrac{1}{2}}dx = \lim\limits_{a \rightarrow 0+} (2x|_a^1) = 2$\\
      \end{tabular} \retTwo\par}

      It follows that:
      
      {\centering $\int \sum\limits_{n = 1}^\infty |2^{-n}f(x - r_n)|dm = \sum\limits_{n = 1}^\infty \int |2^{-n}f(x - r_n)|dm = \sum\limits_{n = 1}^\infty 2^{-n} \cdot 2 = 1$.\retTwo\par}

      So by theorem 2.25, we know that $\sum\limits_{n = 1}^\infty 2^{-n} f(x - r_n) \in L^1(m)$.\retTwo

      By an earlier exercise, since $\sum\limits_{n = 1}^\infty 2^{-n} f(x - r_n) \geq 0$ for all $x$, we know that it has a finite integral if and only if it equals infinity on a null set. Hence, it follows\\ [5pt] that:

      {\centering $g(x) = \sum\limits_{n = 1}^\infty 2^{-n} f(x - r_n) < \infty$ almost everywhere.\retTwo\par}
   \end{myIndent}

   \item[(b)] $g$ is discontinuous at every point and unbounded on every interval, and it remains so after any modification on a Lesbesgue null set.
   
   \begin{myIndent}\exTwoP
      To start, let's fix $x \in \mathbb{R}$ and $\delta \in (0, 1)$. Then consider any $N > 0$.\retTwo

      Note that there exists $r_n \in (r_n)_{n \in \mathbb{N}}$ such that $x < r_n < x + \delta$. Then, note that if $r_n < x < \min(r_n + \frac{1}{N^2}2^{-2n} ,r_n + 1)$, we have that $2^{-n}f(x - r_n) > N$. Since $x + \delta < r_n + 1$, we thus know that $g(s) > N$ for all $s$ in
      
      {\centering $(r_n, \min(x + \delta, r_n + \frac{1}{N^2}2^{-2n})) \subseteq (x, x + \delta)$.\retTwo\par}

      Also note that we can pick $s$ in that interval such that $g(s)$ is finite. Furthermore, if we change $g$ on a null set $N$, then we can also select $s$ to be not in $N$.
      \begin{myIndent}\exPPP
         To see this, first note that $m(r_n, \min(r_n + N^22^{2n}, x + \delta))$ is an open interval and thus has positive measure. Also, by the completeness of $m$, we know that $\{x \mid g(x) = \infty\}$ is a subset of a null set and thus itself a null set.\newpage Finally, suppose we have another null set $N$. Then taking the difference of the interval with the two null sets sets must give a set with the positive measure of the original interval. And thus the difference can't be the empty set which has a measure of zero.\retTwo
      \end{myIndent}

      It immediately follows that even after modifying $g$ on a null set, we still have that $g$ is unbounded on all non-singleton intervals since we can fix $x$ to be any interior point of the interval and $\delta$ such that $(x, x + \delta)$ is a subset of that interval.\retTwo

      Also, if $g(x)$ is finite and $\varepsilon > 0$. Then we've shown that even if we modify\\ $g$, we still have that for all $\delta > 0$, there exists $s \in (x - \delta, x + \delta)$ with\\ $g(x) + \varepsilon < g(s) < \infty$. So $g$ is not continuous at $x$.\retTwo

      
      \begin{myTindent}\myComment
         We haven't defined continuity yet for when $g$ can equal $\infty$ on an interval. But here's my current intuition on what being continuous at $x$ when $g(x)  = \infty$ should mean.
      \end{myTindent}
      Finally if $g(x) = \infty$, then by the previous reasoning, even if we modify $g$ on a null set, we can still find $s$ in any neighborhood of $x$ such that $g(s) < \infty$. Thus, $g$ is not continuous at $x$.\retTwo
   \end{myIndent}

   \item[(c)] $g^2 < \infty$ a.e. but $g^2$ is not integrable on any interval.  

   \begin{myIndent}\exTwoP
      We know $g^2 < \infty$ a.e. because $g < \infty$ a.e.\retTwo

      Now note that since $2^{-n}f(x - r_n) \geq 0$ for all $n \in \mathbb{N}$, we can say that for all\\ $N \in \mathbb{N}$:\\ [-20pt]

      {\centering $\sum\limits_{n = 1}^N (2^{-n}f(x - r_n))^2 \leq \left(\sum\limits_{n = 1}^N 2^{-n}f(x - r_n)\right)^2$ \retTwo\par}

      Taking $N \rightarrow \infty$, we thus get that $\sum\limits_{n = 1}^\infty (2^{-n}f(x - r_n))^2 \leq (g(x))^2$.\retTwo

      Also, given any interval $I$, we can multiply our above inequality by $\chi_I$ and it won't flip because $\chi_I \geq 0$. Then as $(2^{-n}f(x - r_n))^2\chi_I \geq 0$ for all $n \in \mathbb{N}$, we can thus conclude that:

      {\centering $\sum\limits_{n = 1}^\infty \int_I (2^{-n}f(x - r_n))^2dm = \int_I \sum\limits_{n = 1}^\infty (2^{-n}f(x - r_n))^2 dm \leq \int_I (g(x))^2dm$\retTwo\par}

      But now fix $n$ such that there exists $\varepsilon \in (0, 1)$ with $[r_n - \varepsilon, r_n + \varepsilon] \subseteq I$. Then note that:

      {\centering 
      \begin{tabular}{l}
         $\int_I (2^{-n}f(x - r_n))^2 dm \geq 2^{-2n}\cdot \lim\limits_{a \rightarrow r_n+}\int_{a}^{r_n + \varepsilon} (f(x - r_n))^2dx$ \\
         $\phantom{\int_I (2^{-n}f(x - r_n))^2 dm} = 2^{-2n} \cdot \lim\limits_{a \rightarrow r_n+}\int_{a}^{r_n + \varepsilon} \frac{1}{x - r_n}dx$\\
         $\phantom{\int_I (2^{-n}f(x - r_n))^2 dm} = 2^{-2n}\cdot\lim\limits_{a \rightarrow r_n^+}(\log(\varepsilon) - \log(a - r_n)) = \infty$
      \end{tabular} \newpage\par}
      
      So $\int_I (2^{-n}f(x - r_n))^2 dm = \infty$. Hence, so does $\int_I(g(x))^2dm = \infty$. And thus, $(g(x))^2$ is not integrable on the interval $I$.
      \retTwo
   \end{myIndent}
\end{enumerate}

\retTwo

\blab{Exercise 2.26:} If $f \in L^1(m)$ and $F(x) = \int_{-\infty}^x f(t)dt$, then $F$ is continuous on $\mathbb{R}$.

\begin{myIndent}\exTwoP
   Let $(x_k)_{k \in \mathbb{N}}$ be any sequence of points converging to a given point $x$. Importantly, $f \in L^1(m)$ implies that $\int |f| < \infty$. Also,  we have that $|f\chi_{(-\infty, x_k)}| \leq |f|$ for all $k \in \mathbb{N}$. Plus, $f\chi_{(-\infty, x_k)} \rightarrow f\chi_{(-\infty, x)}$ as $k \rightarrow \infty$.\retTwo

   So, by applying the dominated convergence theorem, we have that:

   {\centering $\lim\limits_{k \rightarrow \infty}(F(x_k)) = \lim\limits_{k\rightarrow \infty} (\int f\chi_{(-\infty, x_k)}) = \int f\chi_{(-\infty, x)} = F(x)$\retTwo\par}

   Hence, $\lim\limits_{t \rightarrow x}(F(t)) = F(x)$ and thus $F$ is continuous at $x$.
\end{myIndent}

\newpage

\blab{Exercise 2.20: Generalized Dominated Convergence Theorem} If $f_n, g_n, f, g \in L^1$, $f_n \rightarrow f$ and $g_n \rightarrow g$ a.e., $|f_n| \leq g_n$, and $\int g_n \rightarrow \int g$. Then $\int f_n \rightarrow \int f$.\\ [-12pt]

\begin{myIndent}\exTwoP
   The most murky part of this theorem was saying that we could assume $f$ is\\ measurable. Luckily, the problem statement tells me out right that all the\\ functions are measurable so I don't need to struggle to justify that.\retTwo

	Now by seperating the functions into their real and imaginary parts, we can without loss of generality assume $f_n$ and $f$ are real-valued. Also, the problem statement doesn't make sense if the $g_n$ aren't real-valued.\retTwo

	Since $|f_n| < g_n$ for all $n$, we know that $g_n + f_n > 0$ and $g_n - f_n > 0$ for all $n$. Hence, applying Fatou's lemma we know that:

	\begin{itemize}
		\item $\int g + \int f = \int \lim\limits_{n \rightarrow \infty} (g_n + f_n) = \int \liminf\limits_{n \rightarrow \infty} (g_n + f_n) \leq \liminf\limits_{n \rightarrow \infty}\int (g_n + f_n)$
		\item $\int g - \int f = \int \lim\limits_{n \rightarrow \infty} (g_n - f_n) = \int \liminf\limits_{n \rightarrow \infty} (g_n - f_n) \leq \liminf\limits_{n \rightarrow \infty}\int (g_n - f_n)$\retTwo
	\end{itemize}



	Next, here's a lemma. If $(a_n), (b_n)$ are sequences of real numbers satisfying that $\lim\limits_{n \rightarrow \infty}a_n \rightarrow a \in \mathbb{R}$ and $\liminf\limits_{n\rightarrow \infty} b_n = \beta \in \overline{\mathbb{R}}$, then $\liminf\limits_{n \rightarrow \infty}(a_n + b_n) = a + \beta$.

	\begin{myIndent}\exPPP
		Proof:\\
		Let $(a_{n_k} + b_{n_k})$ be a subsequence of $(a_n + b_n)$ converging to $\liminf\limits_{n \rightarrow \infty}(a_n + b_n)$. Then because $a_{n_k} \rightarrow a$ as $n \rightarrow \infty$, we must have that $b_{n_k} \rightarrow \liminf\limits_{n \rightarrow \infty}(a_n + b_n) - a \geq \beta$. In other words, $\liminf\limits_{n \rightarrow \infty}(a_n + b_n) \geq a + \beta$\retTwo
		
		On the other hand, let $(b_{n_j})$ be a subsequence of $(b_n)$ converging to $\beta$. Then\\ $a_{n_j} + b_{n_j} \rightarrow a + \beta \geq \liminf\limits_{n \rightarrow \infty}(a_n + b_n)$ as $n \rightarrow \infty$.\retTwo
	\end{myIndent}

	So: 
	\begin{itemize}
		\item $\liminf\limits_{n \rightarrow \infty}\int (g_n + f_n) = \liminf\limits_{n \rightarrow \infty}(\int g_n + \int f_n) = \int g + \liminf\limits_{n \rightarrow \infty} \int f_n$
		\item $\liminf\limits_{n \rightarrow \infty}\int (g_n - f_n) = \liminf\limits_{n \rightarrow \infty}(\int g_n - \int f_n) = \int g + \liminf\limits_{n \rightarrow \infty} (-\int f_n)$\retTwo
	\end{itemize}

	Finally, $\liminf\limits_{n \rightarrow \infty}(-\int f_n) = -\limsup\limits_{n\rightarrow \infty}\int f_n$.\retTwo
	
	Therefore subtracting out $\int g$ which we can do because $g \in L^1$ and thus $\int g$ is finite:

	\begin{itemize}
		\item $\int g + \int f \leq \int g + \liminf\limits_{n \rightarrow \infty} \int f_n \Longrightarrow \int f \leq\liminf\limits_{n \rightarrow \infty} \int f_n $
		\item $\int g - \int f \leq \int g + \liminf\limits_{n \rightarrow \infty} (-\int f_n) \Longrightarrow \limsup\limits_{n\rightarrow\infty}\int f_n \leq \int f$\retTwo
	\end{itemize}

	Hence $\lim\limits_{n \rightarrow \infty} \int f_n = \int f$.\newpage
\end{myIndent}

\blab{Exercise 2.21}: Suppose $f_n, f \in L^1$ and $f_n \rightarrow f$ a.e. Then $\int |f_n - f| \rightarrow 0$ if and only if $\int |f_n| \rightarrow \int |f|$.

\begin{myIndent}\exTwoP
	($\Longrightarrow$)\\
	Note that $|f_n| \leq |f_n - f| + |f|$ for all $n$, $|f_n - f| + |f| \rightarrow 0 + |f| = |f|$ a.e., and $\int (|f_n - f| + |f|) \rightarrow \int 0 + \int |f| = \int |f|$ by assumption. Then by exercise 2.20, we know that $\lim\limits_{n \rightarrow \infty}\int |f_n| = \int \lim\limits_{n \rightarrow \infty} |f_n| = \int |f|$.\retTwo
	
	($\Longleftarrow$)\\
	Note that $|f_n - f| \leq |f_n| + |f|$ for all $n$, $|f_n| + |f| \rightarrow 2|f|$ a.e., and by\\ [3pt] assumption $\int (|f_n| + |f|) \rightarrow \int 2|f| = \int \lim\limits_{n \rightarrow \infty}(|f_n| + |f|)$. So by exercise 2.20,\\ we know that $\lim\limits_{n\rightarrow \infty}\int |f_n - f| = \int \lim\limits_{n\rightarrow\infty}|f_n - f| = \int 0 = 0$.\retTwo
\end{myIndent}

\blab{Exercise 2.28}: Compute the following limits:
\begin{enumerate}
	\item[(a)] $\lim\limits_{n \rightarrow \infty} \int_0^\infty (1+\sfrac{x}{n})^{-n}\sin(\sfrac{x}{n})dx$
	\begin{myIndent}\exTwoP
		Note that $|(1 + \sfrac{x}{n})^{-n}\sin(\sfrac{x}{n})| \leq (1 + \sfrac{x}{n})^{-n}$ for all $n \geq 1$ and $x \geq 0$ because $|\sin(\sfrac{x}{n})| \leq 1$ and $(1 + \sfrac{x}{n})^{-n} \geq 0$ when $n$ and $x$ satisfy the above inequalities.\retTwo

		Next, note that $(1 + \sfrac{x}{n})^n$ is a strictly increasing sequence for all $x > 0$. 
		\begin{myIndent}\exPPP
			To see this, note that because $\frac{x}{n+1} < \frac{x}{n}$, we have that:
			
			{\centering\fontsize{11}{13}\selectfont $(1 + \frac{x}{n + 1})^{n + 1} - (1 + \frac{x}{n})^n \geq (1 + \frac{x}{n + 1})^{n + 1} - (1 + \frac{x}{n+1})^n = (1 + \frac{x}{n+1})^n\cdot \frac{x}{n+1} > 0$\retTwo\par}
		\end{myIndent}

		As a result, $(1 + \sfrac{x}{n})^{-n}$ is a strictly decreasing sequence for all $x > 0$.\retTwo

		Now, while $\int_{0}^\infty(1 + x)^{-1}dx = \infty$ and so we can't use that as our upperbound, note that $(1 + \sfrac{x}{n})^{-n} \leq (1 + \sfrac{x}{2})^{-2}$ for all $n \geq 2$ by our above reasoning and:

		{\centering $\int_0^\infty(1 + \sfrac{x}{2})^{-2}dx = 2\int_1^\infty u^{-2}du = 2\lim\limits_{N \rightarrow \infty}(\frac{-1}{N} + 1) = 2$ \retTwo\par}

		Hence, $g(x) = (1 + \frac{x}{2})^{-2} \in L^1$ with $|f_n| \leq g$ for all $n \geq 2$.\retTwo

		Clearly the subsequence leaving out when $n = 1$ will still have identical limiting behavior as the full sequence including $n = 1$. So by applying D.C.T (technically to the subsequence without $n = 1$), we get that:

		{\centering $\lim\limits_{n \rightarrow \infty} \int_0^\infty (1+\sfrac{x}{n})^{-n}\sin(\sfrac{x}{n})dx = \int_0^\infty(\lim\limits_{n \rightarrow \infty} (1+\sfrac{x}{n})^{-n}\sin(\sfrac{x}{n}))dx$ \retTwo\par}

		Now $\sin$ is continuous and $\sfrac{x}{n} \rightarrow 0$ pointwise. Thus $\sin(\sfrac{x}{n}) \rightarrow \sin(0) = 0$ pointwise as $n \rightarrow \infty$. At the same time, note that $(1 + \sfrac{x}{n})^{n} \rightarrow e^{x}$ as\\ $n \rightarrow \infty$.
		
		\begin{myIndent}\exPPP
			Small proof:\\
			By L'Hôpital's rule: $\lim\limits_{y \rightarrow 0}\frac{\log(1 + ay)}{y} = \lim\limits_{y \rightarrow 0}\frac{a}{1 + ay} = a$.\\ So, since $\exp$ is continuous, we have that:

			{\centering $\lim\limits_{y \rightarrow 0}(1 + ay)^{\sfrac{1}{y}} = \lim\limits_{y \rightarrow 0}\exp(\frac{\log(1 + ay)}{y}) = \exp(\lim\limits_{y \rightarrow 0}\frac{\log(1 + ay)}{y}) = \exp(a)$ \newpage\par}
		\end{myIndent}

		Thus $(1 + \sfrac{x}{n})^{-n}\sin(\sfrac{x}{n}) \rightarrow \frac{0}{e^{-x}} = 0$. And so:
		
		{\centering$\int_0^\infty(\lim\limits_{n \rightarrow \infty} (1+\sfrac{x}{n})^{-n}\sin(\sfrac{x}{n}))dx = \int_0^\infty0dx = 0$.\retTwo\par}
	\end{myIndent}

	\item[(b)] $\lim\limits_{n \rightarrow \infty} \int_0^1 (1 + nx^2)(1 + x^2)^{-n}dx$
	
	\begin{myIndent}\exTwoP
		Note that $(1+nx^2)(1+x^2)^{-n}$ is a decreasing sequence of functions.
		
		\begin{myIndent}\exPPP
			Proof: $\frac{1+(n+1)x^2}{(1+x^2)^{n+1}} - \frac{1 + nx^2}{(1 + x^2)^n} = \frac{1 + nx^2 + x^2 - 1 - x^2 - nx^2 - nx^4}{(1 + x^2)^{n+1}} = \frac{-nx^4}{(1 + x^2)^{n+1}} \leq 0$
		\end{myIndent}

		Also, when $n = 1$, we have that $(1 + nx^2)(1+x^2)^{-n} = 1$ and $\int_0^1 1dx = 1$. So, by applying D.C.T, we get that:

		{\centering $\lim\limits_{n \rightarrow \infty} \int_0^1 (1 + nx^2)(1 + x^2)^{-n}dx = \int_0^1 \lim\limits_{n \rightarrow \infty}((1 + nx^2)(1 + x^2)^{-n})dx$ \retTwo\par}

		If $x = 0$, then $(1 + nx^2)(1+x^2)^{-n} = 1$ for all $n$. Meanwhile if $x \neq 0$, then $1 + x > 0$ and thus:

		{\centering $\frac{1 + nx^2}{(1 + x^2)^n} = \frac{1}{(1 + x^2)^n} + x^2\frac{n}{(1 + x^2)^n} \rightarrow 0 + x^2 \cdot 0 = 0$ \retTwo\par}

		It follows that $\int_0^1 \lim\limits_{n \rightarrow \infty}((1 + nx^2)(1 + x^2)^{-n})dx = \int 0 dx = 0$ (we can change the inside function at the point $x = 0$ without changing the value of the integral).\retTwo
	\end{myIndent}
\end{enumerate}

\blab{Exercise 2.29}: Show that $\int_0^\infty x^n e^{-x}dx = n!$ by differentiating the equation $\int_0^\infty e^{-tx}dx = \sfrac{1}{t}$. Similarly show that $\int_{-\infty}^\infty x^{2n}e^{-x^2}dx = \frac{(2n)!\sqrt{\pi}}{4^n n!}$ by differentiating the equation\\ $\int_{-\infty}^\infty e^{-tx^2}dx = \sqrt{\sfrac{\pi}{t}}$.

\begin{myIndent}\exTwoP

	Note that for all $x \geq 0$ and $t \geq 1$, $x^{k}e^{-tx} \leq x^ke^{-x} \leq \frac{x^k}{\frac{x^k}{k!} + \frac{x^{k+2}}{(k+2)!}} \leq \frac{(k+2)!}{(k+2)(k+1) + x^2}$.\retTwo

	Similarly, for all $x \geq 0$ and $t \geq 1$, we have:
	
	{\centering $x^{2k}e^{-tx^2} \leq x^{2k}e^{-x^2} \leq \frac{x^{2k}}{\frac{x^{2k}}{k!} + \frac{x^{2k+2}}{(k+1)!}} = \frac{(k+1)!}{1 + k + x^2}$.\par}

	\begin{myIndent}\exPPP
		Note that these upper bounds are gotten by considering the taylor series of $e^x$.\retTwo
	\end{myIndent}

	Now importantly $\int_0^\infty\frac{1}{a^2 + x^2}dx = \frac{1}{a}(\arctan(\sfrac{x}{a}))\hspace{-0.2em}\left.\right|_0^\infty = \frac{\pi}{2a}$ for all real nonzero $a$.\\ Similarly, $\int_{-\infty}^\infty \frac{1}{a^2 + x^2}dx = \frac{1}{a}(\arctan(\sfrac{x}{a}))\hspace{-0.2em}\left.\right|_{-\infty}^\infty = \frac{\pi}{a}$\retTwo

	Thus, $\frac{(k+2)!}{(k+2)(k+1) + x^2}dx$ is integrable on $[0, \infty)$ and $\int_{-\infty}^\infty \frac{(k+1)!}{1 + k + x^2}dx$ is integrable on $(-\infty, \infty)$ for all $k$.\retTwo

	So, we've shown that for all $t \geq 1$:
	\begin{itemize}
		\item $\frac{d}{dt}\int_0^\infty x^{k-1}e^{-tx}dx = \int_0^\infty \frac{\partial}{\partial t}x^{k-1}e^{-tx}dx = \int_0^\infty -x^{k}e^{-tx}dx$
		\item $\frac{d}{dt}\int_0^\infty x^{2k - 2}e^{-tx^2}dx = \int_0^\infty \frac{\partial}{\partial t}x^{2k-2}e^{-tx^2}dx = \int_0^\infty -x^{2k}e^{-tx^2}dx$\retTwo
	\end{itemize}

	And now the rest of the problem is just repeatedly differentiating the equations the problem tells you to.


	\newpage

	Firstly, note that:

	{\centering\begin{tabular}{l}
		$\int_0^\infty e^{-tx}dx = t^{-1} \xRightarrow{\partial / \partial t} \int_0^\infty -xe^{-tx}dx = -t^{-2} \Longrightarrow \int_0^\infty -1(x)e^{-tx}dx = 1t^{-2}$ \\
		$\phantom{\int_0^\infty e^{-tx}dx = t^{-1}} \xRightarrow{\partial / \partial t} \int_0^\infty -x^2e^{-tx}dx = -2!t^{-3} \Longrightarrow \int_0^\infty x^2e^{-tx}dx = 2!t^{-2}$\\
		$\phantom{\int_0^\infty e^{-tx}dxaa.} \cdots \xRightarrow{\partial / \partial t} \int_0^\infty -x^ne^{-tx}dx = -n!t^{-n-1} \Longrightarrow \int_0^\infty x^ne^{-tx}dx = n!t^{-n-1}$
	\end{tabular}\retTwo\par}

	Plugging in $t = 1$, we then get $\int_0^\infty x^n e^{-x}dx = n!$.\retTwo

	Secondly, note that:

	{\centering\begin{tabular}{l}
		$\int_{-\infty}^\infty e^{-tx^2}dx = \sqrt{\sfrac{\pi}{t}} \xRightarrow{\partial / \partial t} \int_{-\infty}^\infty -x^2e^{-tx^2}dx = -\sqrt{\pi}(\frac{1}{2})t^{-\sfrac{3}{2}}$\\
		$\phantom{\int_{-\infty}^\infty e^{-tx^2}dx = \sqrt{\sfrac{\pi}{t}}a aaaaaaa} \Longrightarrow \int_{-\infty}^\infty x^2e^{-tx^2}dx = \sqrt{\pi}(\frac{1}{2})t^{-\sfrac{3}{2}}$ \\

		$\phantom{\int_{-\infty}^\infty e^{-tx^2}dx = \sqrt{\sfrac{\pi}{t}}} \xRightarrow{\partial / \partial t} \int_{-\infty}^\infty -x^{2(2)}e^{-tx^2}dx = -\sqrt{\pi}(\frac{1}{2})(\frac{3}{2})t^{-\sfrac{5}{2}}$\\
		$\phantom{\int_{-\infty}^\infty e^{-tx^2}dx = \sqrt{\sfrac{\pi}{t}}a aaaaaaa} \Longrightarrow \int_{-\infty}^\infty x^{2(2)}e^{-tx^2}dx = \sqrt{\pi}(\frac{1}{2})(\frac{3}{2})t^{-\sfrac{5}{2}}$ \\

		$\phantom{\int_{-\infty}^\infty e^{-tx}dxaa aa.} \cdots \xRightarrow{\partial / \partial t} \int_{-\infty}^\infty -x^{2(n)}e^{-tx^2}dx = -\sqrt{\pi}(\frac{1}{2})(\frac{3}{2})\cdots(\frac{2n-1}{2})t^{-\sfrac{(2n+1)}{2}}$\\
		$\phantom{\int_{-\infty}^\infty e^{-tx^2}dx = \sqrt{\sfrac{\pi}{t}}a aaaaaaa} \Longrightarrow \int_{-\infty}^\infty x^{2(n)}e^{-tx^2}dx = \sqrt{\pi}(\frac{1}{2})(\frac{3}{2})\cdots (\frac{2n-1}{2})t^{-\sfrac{(2n+1)}{2}}$ \\
	\end{tabular}\retTwo\par}

	Plugging in $t = 1$, we get:
	
	{\centering\begin{tabular}{l}
		$\int_{-\infty}^\infty x^{2n}e^{-x^2}dx = \sqrt{\pi}(\frac{1}{2})(\frac{3}{2})\cdots (\frac{2n-1}{2})$\\
		$\phantom{\int_{-\infty}^\infty x^{2n}e^{-x^2}dx} = \frac{\sqrt{\pi}}{2^n}(2n-1)(2n-3)\cdots(3)(1) = \frac{\sqrt{\pi}}{2^n} \cdot \frac{(2n)!}{2^n n!} = \frac{\sqrt{\pi}(2n)!}{4^{n} n!}$
	\end{tabular}\retTwo\par}
\end{myIndent}

\blab{Exercise 2.31.a}: Derive the following formula by expanding part of the integrand into an infinite series and justifying the term-by term integration:

{\centering For $a > 0$, $\int_{-\infty}^\infty e^{-x^2}\cos(ax)dx = \sqrt{\pi}e^{\sfrac{-a^2}{4}}$ \retTwo\par}

\begin{myIndent}\exTwoP

	Note that:
	
	{\centering $\int_{-\infty}^\infty e^{-x^2}\cos(ax)dx = \int_{-\infty}^\infty e^{-x^2}\sum\limits_{i = 0}^{\infty}\frac{(-1)^i a^{2i}x^{2i}}{(2i)!}dx = \int_{-\infty}^\infty\sum\limits_{i=0}^\infty \frac{a^{2i}(-1)^i}{(2i)!}e^{-x^2}x^{2i}dx$\retTwo\par}

	Next, consider the sequence of functions: $\left(\frac{a^{2i}(-1)^i}{(2i)!}e^{-x^2}x^{2i}\right)_{i}$.\retTwo

	By exercise 29, we know:
	
	{\centering $\sum\limits_{i=0}^\infty \int_{-\infty}^\infty \left|\frac{a^{2i}(-1)^i}{(2i)!}e^{-x^2}x^{2i}\right|dx = \sum\limits_{i=0}^\infty \frac{a^{2i}}{(2i)!} \cdot \frac{\sqrt{\pi}(2i)!}{4^i i!} = \sqrt{\pi}\sum\limits_{i=0}^\infty \frac{a^{2i}}{4^i i!} = \sqrt{\pi} e^{\sfrac{a^2}{4}} < \infty$\retTwo\par}

	Therefore:
	
	{\centering 
	\begin{tabular}{l}
		$\int_{-\infty}^\infty\sum\limits_{i=0}^\infty \frac{a^{2i}(-1)^i}{(2i)!}e^{-x^2}x^{2i}dx = \sum\limits_{i=0}^\infty\frac{a^{2i}(-1)^i}{(2i)!}\int_{-\infty}^\infty e^{-x^2}x^{2i}dx$\\ [8pt]
		$\phantom{\int_{-\infty}^\infty\sum\limits_{i=0}^\infty \frac{a^{2i}(-1)^i}{(2i)!}e^{-x^2}x^{2i}dx} = \sum\limits_{i=0}^\infty \frac{a^{2i}(-1)^i}{4^i i!} = \sum\limits_{i = 0}^\infty \frac{1}{i!}(\frac{-a^2}{4})^i = e^{\sfrac{-a^2}{4}}$
	\end{tabular}\newpage\par}
\end{myIndent}

\blab{Exercise 2.32}: Suppose $\mu(X) < \infty$. If $f$ and $g$ are complex-valued measurable functions on $X$, define

{\centering $\rho(f, g) = \int\frac{|f - g|}{1 + |f - g|}d\mu$ \retTwo\par}

Then $\rho$ is a metric on the space of measurable functions if we identify functions are equal a.e., and $f_n \rightarrow f$ with respect to this metric iff $f_n \rightarrow f$ in measure.

\begin{myIndent}\exTwoP
	First we show $\rho$ is a metric. To start, a reoccuring relevant fact is that $\frac{|f - g|}{1 + |f - g|} \geq 0$. So:
	
	\begin{itemize}
		\item $\rho(f, g) \geq 0$
		\item $\rho(f, g) = 0 \Longleftrightarrow \frac{|f-g|}{1 + |f-g|} = 0$ a.e. $\Longleftrightarrow |f - g| = 0$ a.e. $\Longleftrightarrow f = g$ a.e.\retTwo
	\end{itemize}

	It's also trivial that $\rho(f, g) = \rho(g, f)$ since $|f - g| = |g - f|$.\retTwo

	Finally, we show the triangle inequality:
	\begin{myIndent}\exPPP
		Consider any $x, y, z \in \mathbb{C}$. 
		\begin{itemize}
			\item If $|x - y| \geq |x - z|$, then we have that $\frac{|x - z|}{1 + |x - z|} \leq \frac{|x - y|}{1 + |x - y|} \leq \frac{|x - y|}{1 + |x - y|} + \frac{|y - z|}{1 + |y - z|}$.
			\item If $|y - z| \geq |x - z|$, then we have that $\frac{|x - z|}{1 + |x - z|} \leq \frac{|y - z|}{1 + |y - z|} \leq \frac{|x - y|}{1 + |x - y|} + \frac{|y - z|}{1 + |y - z|}$.
			\item If both $|x - y| < |x - z|$ and $|y - z| < |x - z|$, then by triangle inequality:
			
			{\centering $\frac{|x - z|}{1 + |x - z|} \leq \frac{|x - y|}{1 + |x - z|} + \frac{|y - z|}{1 + |x - z|} \leq \frac{|x - y|}{1 + |x - y|} + \frac{|y - z|}{1 + |x - z|} \leq \frac{|x - y|}{1 + |x - y|} + \frac{|y - z|}{1 + |y - z|}$ \retTwo\par}
		\end{itemize}

		Thus in all cases, we have that $\frac{|x - z|}{1 + |x - z|} \leq \frac{|x - y|}{1 + |x - y|} + \frac{|y - z|}{1 + |y - z|}$.\retTwo

		It follows that $\frac{|f - h|}{1 + |f - h|} \leq \frac{|f - g|}{1 + |f - g|} + \frac{|g-h|}{1 + |g-h|}$. And since all these terms are nonnegative:

		{\centering 
		\begin{tabular}{l}
			$\rho(f, h) = \int \frac{|f - h|}{1 + |f - h|} d\mu \leq  \int \frac{|x - y|}{1 + |x - y|} + \frac{|y - z|}{1 + |y - z|}d\mu$\\ [8pt]
			
			$\phantom{\rho(f, h) = \int \frac{|f - h|}{1 + |f - h|} d\mu } = \int \frac{|x - y|}{1 + |x - y|}d\mu + \int\frac{|y - z|}{1 + |y - z|}d\mu = \rho(f, g) + \rho(g, h)$
		\end{tabular} \retTwo\par}
	\end{myIndent}

	Having shown $\rho$ is a metric, we now prove the two implications.\\
	\begin{myIndent}
		($\Longleftarrow$)\\
		Fix $\varepsilon > 0$ and define for some $\gamma > 0$ the sets $E_n = \{x \mid |f_n(x) - f(x)| > \gamma\}$ for all $n$. Then note that for all $n$:

		{\center 
		\begin{tabular}{l}
			$\int \frac{|f_n-f|}{1 + |f_n-f|}d\mu = \int_{E_n}\frac{|f_n-f|}{1 + |f_n-f|}d\mu + \int_{E_n^{\comp}}\frac{|f_n-f|}{1 + |f_n-f|}d\mu$\\ [8pt]

			$\phantom{\int \frac{|f_n-g|}{1 + |f_n-f|}d\mu} \leq \int_{E_n}\frac{|f_n-f|}{1 + |f_n-f|}d\mu + \int_{E_n^{\comp}}\frac{\gamma}{1 + \gamma}d\mu$\\ [8pt]

			$\phantom{\int \frac{|f_n-g|}{1 + |f_n-f|}d\mu} \leq \int_{E_n}1d\mu + \int_{E_n^{\comp}}\frac{\gamma}{1 + \gamma}d\mu = \mu(E_n) + \mu(E_n^\comp)\frac{\gamma}{1 + \gamma}$\\ [4pt]

			$\phantom{\int \frac{|f_n-g|}{1 + |f_n-f|}d\mu \leq \int_{E_n}1d\mu + \int_{E_n^{\comp}}\frac{\gamma}{1 + \gamma}d\mu} \leq \mu(E_n) + \mu(X)\frac{\gamma}{1 + \gamma}$
		\end{tabular}\retTwo\par}

		Note that if $\gamma < \frac{\varepsilon}{1 - \varepsilon}$ (assuming that $\varepsilon < 1$), then $\frac{\gamma}{1 + \gamma} < \varepsilon$. So by fixing $\gamma$\\ sufficiently small (and because $\mu(X)$ is finite), we can force $\mu(X)\frac{\gamma}{1 + \gamma} < \mu(X)\varepsilon$. Then, because $f_n \rightarrow f$ in measure, we know that $\mu(E_n) \rightarrow 0$ as $n \rightarrow \infty$. So, we can pick $N$ such that $\forall n \geq N$, $\mu(E_n) < \varepsilon$.\newpage
		
		Therefore, for all $n \geq N$, we have that:

		{\center $\int \frac{|f_n - f|}{1 + |f_n - f|}d\mu \leq \varepsilon( 1 + \mu(X))$ \retTwo\par}

		$1 + \mu(X)$ is a finite constant and $\varepsilon$ was arbitrary. So, we've shown that\\ $\int \frac{|f_n - f|}{1 + |f_n - f|}d\mu \rightarrow 0$ as $n \rightarrow \infty$.\retTwo

		($\Longrightarrow$)\\
		Fix $\varepsilon > 0$ and then for all $n$ define $E_n = \{x \mid |f_n(x) - f(x)| > \varepsilon\}$. Then note that:

		{\centering\begin{tabular}{l}
			$\int \frac{|f_n-f|}{1 + |f_n-f|}d\mu = \int_{E_n}\frac{|f_n-f|}{1 + |f_n-f|}d\mu + \int_{E_n^{\comp}}\frac{|f_n-f|}{1 + |f_n-f|}d\mu$\\ [8pt]

			$\phantom{\int \frac{|f_n-g|}{1 + |f_n-f|}d\mu} \geq \int_{E_n}\frac{|f_n-f|}{1 + |f_n-f|}d\mu \geq \int_{E_n} \frac{\varepsilon}{1 + \varepsilon} = \mu(E_n)\frac{\varepsilon}{1 + \varepsilon}$
		\end{tabular} \retTwo\par}

		Hence, we have that $0 \leq \mu(E_n) \leq \frac{1 + \varepsilon}{\varepsilon}\int \frac{|f_n-f|}{1 + |f_n-f|}d\mu$.\retTwo

		And now we are done because by our hypothesis, $\frac{1 + \varepsilon}{\varepsilon}\int \frac{|f_n-f|}{1 + |f_n-f|}d\mu \rightarrow 0$ as $n \rightarrow \infty$. Therefore, so must $\mu(E_n)$.
		
		\begin{myIndent}\exPPP
			Side note: only the backward direction required the fact that $\mu(X)$ is finite.\retTwo
		\end{myIndent}
	\end{myIndent}
\end{myIndent}

\blab{Exercise 2.34}: Suppoose $|f_n| \leq g \in L^1$ and $f_n \rightarrow f$ in measure.

\begin{enumerate}
	\item[(a)] $\int f = \lim\limits_{n \rightarrow \infty}\int f_n$.
	
	\begin{myIndent}\exTwoP
		A hopefully clear fact is that $f_n \rightarrow f$ in measure only if $(f_n)$ is Cauchy in\\ measure. 
		\begin{myIndent}\exPPP
			To prove this, fix any $\varepsilon > 0$ and note that:
	
			{\centering $\{x \mid |f_n - f_m| \geq \varepsilon\} \subseteq \{x \mid |f_n - f| \geq \sfrac{\varepsilon}{2}\} \cup \{x \mid |f_m - f| \geq \sfrac{\varepsilon}{2}\}$ \retTwo\par}
	
			Since $f_n \rightarrow f$ in measure, for all $\delta > 0$ there exists $N$ such that when $n, m \geq N$, then:
	
			
			{\centering\begin{tabular}{l}
				$\mu(\{x \mid |f_n - f| \geq \sfrac{\varepsilon}{2}\} \cup \{x \mid |f_m - f| \geq \sfrac{\varepsilon}{2}\})$\\
				$\phantom{aaaa} \leq \mu(\{x \mid |f_n - f| \geq \sfrac{\varepsilon}{2}\}) + \mu(\{x \mid |f_m - f| \geq \sfrac{\varepsilon}{2}\}) < \frac{\delta}{2} + \frac{\delta}{2} = \delta$
			\end{tabular} \retTwo\par}
		\end{myIndent}

		Another relavent fact is that if $f_n \rightarrow f$ in measure, then we must also have that that $f_{n_j} \rightarrow f$ in measure for any subsequence $(f_{n_j})$ of $(f_n)$. As a result, we know that given any subsequence $(f_{n_j})$ of $f_n$, it must be Cauchy and thus it must have subsequence $(f_{n_{j_k}})$ such that there exists a function $g$ such that $f_{n_{j_k}} \rightarrow h$ pointwise and $h = f$ a.e.. Importantly, by D.C.T we then know that $f \in L^1$ and $\int f_{n_{j_k}} \rightarrow \int h = \int f$.\retTwo

		Now consider the sequence $(\int f_n)_{n \in \mathbb{N}}$. Then consider two subsequences $(\int f_{n_j})_{j \in \mathbb{N}}$ and $(\int f_{n_J})_{J \in \mathbb{N}}$ such that $\int f_{n_j} \rightarrow \liminf \int f_n$ and $\int f_{n_J} \rightarrow \limsup \int f_n$.\newpage

		By prior reasoning, we know there exists a subsequence of $(f_{n_{j_k}})$ such that\\ $\int f_{n_{j_k}} \rightarrow \int f$. But at the same time, since $\int f_{n_{j_k}}$ is a subsequence of an already\\ convergent sequence, we know that $\int f_{n_{j_k}} \rightarrow \lim\limits_{j \rightarrow \infty}\int f_{n_j} = \liminf \int f_n$. So, we\\ [-6pt] must have that $\int f = \liminf \int f_n$.\retTwo

		By analogous reasoning to $(\int f_{n_J})$, we can show that $\int f = \limsup \int f_n$.\\ So, $\liminf \int f_n = \int f = \limsup \int f_n \Longrightarrow \lim\limits_{n \rightarrow \infty}\int f_n = \int f$.\retTwo
	\end{myIndent}

	\item[(b)] $f_n \rightarrow f$ in $L^1$. 
	
	\begin{myIndent}\exTwoP
		Note that $|f_n - f| \leq |f_n| + |f| \leq g + |f|$. In the previous part, we showed that $f \in L^1$. Also $g$ is assumed to be nonnegative and in $L^1$. So
		
		{\centering $\int (g + |f|) = \int g + \int |f| < \infty$.\retTwo\par}

		Now recall that given any subsequence $(f_{n_j})$ of $(f_n)$ we can find a sub-subsequence $(f_{n_{j_k}})$ such that $f_{n_{j_k}}$ converges to a function $h$ pointwise a.e and $h = f$ a.e. In simpler terms we can just say that $f_{n_{j_k}} \rightarrow f$ pointwise a.e.. But then we have that $|f_{n_{j_k}} - f| \rightarrow 0$ pointwise a.e., and by applying D.C.T (which we can do because of the previous paragraph), we know that $\int |f_{n_{j_k}} - f| \rightarrow \int 0 = 0$.\retTwo

		Finally, consider the sequence $(\int |f_n - f|)_n$ and let $(\int |f_{n_j} - f|)_j$ and\\ $(\int |f_{n_J} - f|)_J$ be subsequences converging to $\liminf \int |f_n - f|$ and\\ $\limsup \int |f_n - f|$ respectively.\retTwo

		From before, we know there exists sub-subsequences $(\int |f_{n_{j_k}} - f|)_k$ and\\ $(\int |f_{n_{J_K}} - f|)_K$ which both converge to $0$. But those subsequences must also converge to $\liminf \int |f_n - f|$ and $\limsup \int |f_n - f|$. So:

		{\centering $\liminf \int |f_n - f| = 0 = \limsup\int |f_n - f| \Longrightarrow \lim\limits_{n \rightarrow \infty} \int |f_n - f| = 0$ \retTwo\par}
	\end{myIndent}
\end{enumerate}

\blab{Exercise 2.38}: Suppose $f_n \rightarrow f$ in measure and $g_n \rightarrow g$ in measure.
\begin{enumerate}
	\item[(a)] $f_n + g_n \rightarrow f + g$ in measure.
	
	\begin{myIndent}\exTwoP
		Fix $\varepsilon > 0$ and note that:

		{\centering\begin{tabular}{l}
			$\{x \mid |f_n + g_n - f - g| \geq \varepsilon\} \subseteq \{x \mid |f_n - f| + |g_n - g|  \geq \varepsilon\}$\\

			$\phantom{\{x \mid |f_n + g_n - f - g| \geq \varepsilon\}} \subseteq \{x \mid |f_n - f| \geq \sfrac{\varepsilon}{2} \} \cup \{x \mid |g_n - g| \geq \sfrac{\varepsilon}{2}\} $
		\end{tabular}\retTwo\par}

		Therefore, we have that:
		
		{\centering\exPP
		\begin{tabular}{l}
			$\mu(\{x \mid |f_n + g_n - f - g| \geq\varepsilon\}) \leq \mu(\{x \mid |f_n - f| \geq \sfrac{\varepsilon}{2} \} \cup \{x \mid |g_n - g| \geq \sfrac{\varepsilon}{2}\})$\\

			$\phantom{\mu(\{x \mid |f_n + g_n - f - g| \geq\varepsilon\})} \leq \mu(\{x \mid |f_n - f| \geq \sfrac{\varepsilon}{2} \}) + \mu(\{x \mid |g_n - g| \geq \sfrac{\varepsilon}{2}\})$
		\end{tabular}\newpage\par}

		Since $f_n \rightarrow f$ in measure and $g_n \rightarrow g$ in measure, the two terms on the right go to $0$ as $n \rightarrow \infty$. Thus, we also have that $\mu(\{x \mid |f_n + g_n - f - g| >\varepsilon\}) \rightarrow 0$ as $n \rightarrow \infty$.\retTwo
	\end{myIndent}

	\item[(b)] $f_ng_n \rightarrow fg$ in measure if $\mu(X) < \infty$ but not necessarily if $\mu(X) = \infty$. 
	
	\begin{myIndent}\exTwoP
		Note that:

		{\centering
		\begin{tabular}{l}
			$|f_ng_n - fg| = |f_ng_n + f_ng + fg_n + fg - f_ng - fg_n - fg -fg|$\\ [3pt]

			$\phantom{|f_ng_n - fg|} \leq |f_ng_n + fg - f_ng - fg_n| + |f_ng - fg| + |fg_n - fg|$\\ [3pt]

			$\phantom{|f_ng_n - fg|} \leq |f_n(g_n - g) - f(g_n - g)| + |g(f_n - f)| + |f(g_n - g)|$\\ [3pt]

			$\phantom{|f_ng_n - fg|} \leq |f - f_n||g - g_n| + |g||f_n - f| + |f||g_n - g|$
		\end{tabular} \retTwo\par}

		Thus, fixing $\varepsilon > 0$ we have that:

		{\centering\exPP
		\begin{tabular}{l}
			$\mu(\{x : |f_ng_n - fg| > \varepsilon\}) \leq \mu(\{x : |f - f_n||g - g_n| \geq \sfrac{\varepsilon}{3}\})$\\
			$\phantom{\mu(\{x : |f_ng_n - fg| > \varepsilon\}) aaa} + \mu(\{x : |g||f_n - f| \geq \sfrac{\varepsilon}{3}\}) + \mu(\{x : |f||g_n - g| \geq \sfrac{\varepsilon}{3}\})$
		\end{tabular} \retTwo\par}

		Now, define $E_n \coloneq \{|f| > n\}$ and note that $E_n$ is a decreasing sequence\\ of sets. Also $\mu(E_1) < \mu(X) < \infty$ and $\bigcap\limits_{n \geq 1} E_n = \emptyset$ It follows that\\ [-8pt] $\lim\limits_{n \rightarrow \infty}\mu(E_n) = \mu(\bigcap\limits_{n \geq 1} E_n) = 0$.\retTwo

		Doing the same reasoning with $F_n \coloneq \{|g| > n\}$, we get that $\mu(F_n) \rightarrow 0$ as $n \rightarrow \infty$. So given any $\delta > 0$, there exists integers $M_1$ and $M_2$ such that for all $n \geq M_1$ and $m \geq M_2$, $\mu(E_n) < \delta$ and $\mu(F_n) < \delta$. Set $N_1 = \max(M_1, M_2)$. Thus $\mu(E_{N_1}) < \delta$, $\mu(F_{N_2}) < \delta$, and outside $E_{N_1}$ and $F_{N_1}$ we have that $|f| \leq N_1$ and $|g| \leq N_1$ respectively.\retTwo

		In rapid succession, use the fact that $f_n \rightarrow f$ and $g_n \rightarrow g$ in measure in order to find integers $M_3, M_4, M_5, M_6$ such that:

		\begin{itemize}
			\item $n \geq M_3 \Longrightarrow \mu(\{x : |f_n - f| \geq \frac{\varepsilon}{3N_1}\}) < \delta$
			\item $n \geq M_4 \Longrightarrow \mu(\{x : |g_n - g| \geq \frac{\varepsilon}{3N_1}\}) < \delta$
			\item $n \geq M_5 \Longrightarrow \mu(\{x : |f_n - f| \geq \sqrt{\sfrac{\varepsilon}{3}}\}) < \delta$
			\item $n \geq M_6 \Longrightarrow \mu(\{x : |g_n - g| \geq \sqrt{\sfrac{\varepsilon}{3}}\}) < \delta$
		\end{itemize}

		Take $N_2 = \max(M_3, M_4, M_5, M_6)$. Then finally, note that:
		{\begin{itemize}\exPP
			\item $\mu(\{x : |f - f_n||g - g_n| \geq \sfrac{\varepsilon}{3}\}) \leq \mu(\{x : |f - f_n| \geq \sqrt{\sfrac{\varepsilon}{3}}\} \cup \{x : |g - g_n| \geq \sqrt{\sfrac{\varepsilon}{3}}\})$\\ [3pt]
			$\phantom{\mu(\{x : |f - f_n||g - g_n| \geq\})} \leq \mu(\{x : |f - f_n| \geq \sqrt{\sfrac{\varepsilon}{3}}\}) + \mu (\{x : |g - g_n| \geq \sqrt{\sfrac{\varepsilon}{3}}\})$\\ [3pt]
			$\phantom{\mu(\{x : |f - f_n||g - g_n| \geq\})} < \delta + \delta = 2\delta$ when $n \geq N_2$.\retTwo

			\item $\mu(\{x : |g||f - f_n| \geq \sfrac{\varepsilon}{3}\}) \leq \mu(\{x : |g| \geq N_1\} \cup \{x : |f - f_n| \geq \frac{\varepsilon}{3N_1}\})$\\ [3pt]
			$\phantom{\mu(\{x : |f - f_n||g - g_n| \})} \leq \mu(\{x : |g| \geq N_1 \}) + \mu (\{x : |f - f_n| \geq \frac{\varepsilon}{3N_1}\})$\\ [3pt]
			$\phantom{\mu(\{x : |f - f_n||g - g_n|\})} < \delta + \delta = 2\delta$ when $n \geq N_2$.\newpage

			\item $\mu(\{x : |f||g - g_n| \geq \sfrac{\varepsilon}{3}\}) \leq \mu(\{x : |f| \geq N_1\} \cup \{x : |g - g_n| \geq \frac{\varepsilon}{3N_1}\})$\\ [3pt]
			$\phantom{\mu(\{x : |f - f_n||g - g_n| \})} \leq \mu(\{x : |f| \geq N_1 \}) + \mu (\{x : |g - g_n| \geq \frac{\varepsilon}{3N_1}\})$\\ [3pt]
			$\phantom{\mu(\{x : |f - f_n||g - g_n|\})} < \delta + \delta = 2\delta$ when $n \geq N_2$.\retTwo
		\end{itemize}}

		So for $n \geq N_2$, we have that $\mu(\{x : |f_ng_n - fg| > \varepsilon\}) < 6\delta$. And since $\delta$ is arbitrary, we thus know that $\mu(\{x : |f_ng_n - fg| > \varepsilon\})\rightarrow 0$ as $n \rightarrow \infty$.\retTwo\retTwo


		Now in our work above we assumed $\mu(X) < \infty$. To see that this isn't\\ necessarily true if $\mu(X) = \infty$, let us consider the real Lesbesgue measure\\ space restricted to $(0, \infty)$.

		\begin{myIndent}
			Define $f_n(x) = x + \frac{1}{n}\chi_{(n, n+1)} = g_n(X)$. Then clearly $f_n \rightarrow x$ and $g_n \rightarrow x$ in measure. However, $f_ng_n = x^2 + \frac{2x}{n}\chi_{(n, n+1)} + \frac{1}{n^2}\chi_{n, n+1}$.\retTwo
			
			For all $x \in (n, n+1)$, we have that
			
			{\centering $x^2 + \frac{2x}{n}\chi_{(n, n+1)} + \frac{1}{n^2}\chi_{n, n+1} - x^2 \geq \frac{2x}{n}\chi_{(n, n+1)} > 2$.\retTwo\par}

			So $\mu(|f_ng_n - x^2| \geq 2) > \mu((n, n+1)) = 1$ for all $n \in \mathbb{N}$.\retTwo
		\end{myIndent}
	\end{myIndent}
\end{enumerate}

\blab{Exercise 2.51:} Let $(X, \mathcal{M}, \mu)$ and $(Y, \mathcal{N}, \nu)$ be arbitrary measure spaces (not necessarily $\sigma$-finite).

\begin{enumerate}
	\item[(a)] If $f: X \rightarrow \mathbb{C}$ is $\mathcal{M}$-measurable, $g: Y \rightarrow \mathbb{C}$ is $\mathcal{N}$-measurable, and\\ $h(x, y) = f(x)g(y)$, then $h$ is $\mathcal{M} \otimes \mathcal{N}$-measurable.
	
	\begin{myIndent}\exTwoP
		Define $F(x, y) = f(x)$ and $G(x, y) = g(y)$. Then given any set $B \in \mathcal{B}_{\mathbb{C}}$, we have that $F^{-1}(B) = f^{-1}(B) \times Y$ and $G^{-1}(B) = X \times g^{-1}(B)$. And since both $f$ and $g$ are measurable, we know that $f^{-1}(B) \in \mathcal{M}$ and $g^{-1}(B) \in \mathcal{N}$. So,\\ $F^{-1}(B)$ and $G^{-1}(B)$ are both rectangles and thus in $\mathcal{M} \otimes \mathcal{N}$.\retTwo

		Next, note that $h(x, y) = F(x, y)G(x, y)$. Thus, $h$ is the product of two\\ $(\mathcal{M} \otimes \mathcal{N}, \mathcal{B}_{\mathbb{C}})$-measurable functions and thus itself also a\\ $(\mathcal{M} \otimes \mathcal{N}, \mathcal{B}_{\mathbb{C}})$-measurable function.
		\retTwo
	\end{myIndent}

	\item[(b)] If $f \in L^1(\mu)$ and $g \in L^1(\nu)$, then $h \in L^1(\mu \times \nu)$ and $\int h d(\mu \times \nu) = [\int f d\mu][\int g d\nu]$.
	
	\begin{myIndent}\exTwoP
		To start, if $f$ and $g$ are nonnegative simple functions, then this is true. To see this, write $\phi_n = \sum\limits_{i = 1}^n a_i\chi_{E_i}$ and $\varphi_n = \sum\limits_{j = 1}^m b_j \chi_{F_j}$ where all $a_i, b_j$ are complex numbers, all $E_i \subseteq X$, and all $F_i \subseteq Y$. Then:
		
		{\centering $fg = \sum\limits_{i = 1}^n \sum\limits_{j = 1}^m a_ib_j\chi_{E_i}(x)\chi_{F_j}(y) = \sum\limits_{i=1}^n\sum\limits_{j=1}^m a_i b_j \chi_{E_i \times F_j}((x, y))$ \retTwo\par}

		\newpage

		So, $\int fg d(\mu \times \nu) = \sum\limits_{i=1}^n \sum\limits_{j=1}^m a_i b_j \mu \times \nu(E_i \times F_j) = \sum\limits_{i=1}^n \sum\limits_{j=1}^m a_i b_j \mu(E_i)\nu(F_j)$.\retTwo

		But at the same time, 
		
		{\centering $(\int f d\mu)(\int g d\nu) = (\sum\limits_{i=1}^n a_i\mu(E_i))(\sum\limits_{j=1}^m b_i\nu(F_j)) = \sum\limits_{i=1}^n \sum\limits_{j=1}^m a_i b_j \mu(E_i)\nu(F_j)$.\retTwo\par}

		To extend this to nonnegative real-valued functions, let $(\phi_n)$ and $(\varphi_n)$ be increasing sequences of nonnegative simple functions converging pointwise to $f$ and $g$ respectively. Then by M.C.T., we know that $\int \phi_n d\mu \rightarrow \int f d\mu$ and $\int \varphi_n d\nu \rightarrow \int gd\nu$ as $n \rightarrow \infty$.\retTwo

		Next, note that $\phi_n\varphi_n \rightarrow fg$ pointwise as $n \rightarrow \infty$. Also, each $\phi_n\varphi_n$ is a simple function as is demontrated on the previous line, and $\phi_n(x)\varphi_n(y)$ monotonically increases as $n \rightarrow \infty$. So by applying M.C.T., we know that:

		{\centering $\int \phi_n \varphi_n d(\mu \times \nu) \rightarrow \int fg d(\mu \times \nu) = \int h d(\mu \times \nu)$ \retTwo\par}

		Thus, we can say that:

		{\centering $\int f d\mu \int g d\nu = \lim\limits_{n \rightarrow \infty} (\int \phi_n d\mu \int \varphi_n d\nu) = \lim\limits_{n \rightarrow \infty} (\int \phi_n \varphi_n d(\mu \times \nu)) = \int h d(\mu \times \nu)$. \retTwo\par}

		Now consider if $f, g$ are real-valued functions. Then $|h(x, y)| = |f(x)||g(y)|$ and so from before: $\int |h| d(\mu \times \nu) = (\int |f| d\mu)(\int |g|d\nu)$. If $f \in L^1(\mu)$ and $g \in L^1(\nu)$, then the right side of that expression is finite. So $\int |h| < \infty$ and thus $h \in L^1(\mu \times \nu)$.\retTwo

		Assuming $f, g$ are integrable, then note that $h^+ = f^+g^+ + f^-g^-$ and $h^- = f^+g^- + f^-g^+$. So applying our prior reasoning to those parts:

		{\centering\exPP
		\begin{tabular}{l}
			 $\int h d(\mu \times \nu) = \int h^+ d(\mu \times \nu) - \int h^- d(\mu \times \nu)$\\ [4pt]

			 $\phantom{\int h d(\mu \times \nu) } = \int f^+ d\mu \int g^+ d\nu + \int f^- d\mu \int g^- d\nu - \int f^+ d\mu \int g^- d\nu - \int f^- d\mu \int g^+ d\nu$\\ [4pt]

			 $\phantom{\int h d(\mu \times \nu) } = \int f^+ d\mu(\int g^+d\nu - \int g^- d\nu) - \int f^-d\mu (\int g^+ d\nu - \int g^- d\nu)$\\ [4pt]

			 $\phantom{\int h d(\mu \times \nu) } = (\int f^+d\mu - \int f^-d\mu)(\int g^+ d\nu - \int g^- d\nu) = (\int fd\mu)(\int gd\nu)$
		\end{tabular}\retTwo\par}

		Finally, we consider when $f, g$ are complex valued functions. Like before, we have that $|h(x, y)| = |f(x)||g(y)|$ and so by the same reasoning as when $f, g$ were real valued, if $f \in L^1(\mu)$ and $g \in L^1(\nu)$, then $\int |h|d(\mu \times \nu) < \infty$ and thus $h \in L^1(\mu \times \nu)$.

		Finally, supposing that $f, g$ are integrable one last time, note that:
		
		{\centering\exPP $\rea(h) = \rea(f)\rea(g) - \ima(f)\ima(g)$ and $\ima(h) = \rea(f)\ima(g) + \ima(f)\rea(g)$.\retTwo\par}

		By doing a bunch of similar manipulations to the real-valued case, you can then show that:
		
		{\centering\exPP $\int \rea(h) d(\mu \times \nu) + i \int \ima(h) d(\mu \times \nu) = (\int \rea(f)d\mu + i \int \ima(f)d\mu)(\int \rea(g)d\mu + i \int \ima(g)d\mu)$.\retTwo\par}

		But I'm out of time so do them yourself :P.
		\retTwo
	\end{myIndent}
\end{enumerate}

\blab{Exercise 2.55}: Let $E = [0, 1] \times [0, 1]$. Investigate the existence and equality of $\int_E fdm^2$, $\int_0^1 \int_0^1 f(x, y)dxdy$, and $\int_0^1 \int_0^1 f(x, y)dydx$ for the following $f$.
\begin{enumerate}
	\item[(a)] $f(x, y) = (x^2 - y^2)(x^2 + y^2)^{-2}$

	\begin{myIndent}\exTwoP
		Note that:

		{\centering 
		\begin{tabular}{l l}
			$\int_0^1 \frac{x^2 - y^2}{(x^2 + y^2)^2}dx = \int_0^{\tan^{-1}(1/y)} \frac{y^3(\tan^2(\theta) - 1)\sec^2(\theta)}{(y^2(\tan^2(\theta) + 1))^2}d\theta$ & {\exPPP (substituting $x = y\tan(\theta)$)} \\ [8pt]

			$\phantom{\int_0^1 \frac{x^2 - y^2}{(x^2 + y^2)^2}dx} = \frac{1}{y}\int_0^{\tan^{-1}(1/y)} \frac{(\tan^2(\theta) - 1)\sec^2(\theta)}{\sec^4(\theta)}d\theta$\\ [8pt]

			$\phantom{\int_0^1 \frac{x^2 - y^2}{(x^2 + y^2)^2}dx} = \frac{1}{y}\int_0^{\tan^{-1}(1/y)} \frac{\tan^2(\theta) - 1}{\sec^2(\theta)}d\theta$\\ [8pt]

			$\phantom{\int_0^1 \frac{x^2 - y^2}{(x^2 + y^2)^2}dx} = \frac{1}{y}\int_0^{\tan^{-1}(1/y)} \sin^2(\theta) - \cos^2(\theta)d\theta$\\ [8pt]

			$\phantom{\int_0^1 \frac{x^2 - y^2}{(x^2 + y^2)^2}dx} = \frac{1}{y}\int_0^{\tan^{-1}(1/y)} -\cos(2\theta)d\theta$\\ [8pt]
			
			$\phantom{\int_0^1 \frac{x^2 - y^2}{(x^2 + y^2)^2}dx} = \frac{-1}{2y}(\sin(2\tan^{-1}(1/y)) - 0)$\\ [8pt]

			$\phantom{\int_0^1 \frac{x^2 - y^2}{(x^2 + y^2)^2}dx} = \frac{-1}{y}\sin(\tan^{-1}(1/y))\cos(\tan^{-1}(1/y))$ & {\exPPP (getting rid of the 2)}\\ [8pt]

			$\phantom{\int_0^1 \frac{x^2 - y^2}{(x^2 + y^2)^2}dx} = \frac{-1}{y}\tan(\tan^{-1}(1/y))\cos^2(\tan^{-1}(1/y))$\\ [8pt]

			$\phantom{\int_0^1 \frac{x^2 - y^2}{(x^2 + y^2)^2}dx} = \frac{-1}{y^2}\frac{1}{1 + \tan^2(\tan^{-1}(1/y))}$ & {\exPPP ($\cos^2 = \frac{1}{\sec^2} = \frac{1}{1 + \tan^2}$)}\\ [8pt]

			$\phantom{\int_0^1 \frac{x^2 - y^2}{(x^2 + y^2)^2}dx} = \frac{-1}{y^2} \cdot \frac{1}{1 + y^{-2}} = \frac{-1}{y^2 + 1}$
		\end{tabular} \retTwo\par}

		Luckily we don't need to do that again because:
		
		{\centering $\int_0^1 \frac{x^2 - y^2}{(x^2 + y^2)^2}dy = -\int_0^1 \frac{y^2 - x^2}{(x^2 + y^2)^2}dy = \frac{1}{x^2 + 1}$.\retTwo\par}

		Next note that $\int_0^1 \frac{-1}{y^2 + 1}dy = -\tan^{-1}(1) + \tan^{-1}(0) = -\sfrac{\pi}{4}$. In turn this means that $\int_0^1 \frac{1}{x^2 + 1}dx = \sfrac{\pi}{4}$.\retTwo

		So, we've shown that $\int_0^1 \int_0^1 \frac{x^2 - y^2}{(x^2 + y^2)^2}dydx \neq \int_0^1 \int_0^1 \frac{x^2 - y^2}{(x^2 + y^2)^2}dxdy$\retTwo

		The reason why is that $\frac{x^2 - y^2}{(x^2 + y^2)^2} \notin L^1(m^2)$.\newpage
		
		To see why, note that:
		
		{\centering\exPP
		\begin{tabular}{l}
			$\int_0^1 \left|\frac{x^2 - y^2}{(x^2 + y^2)^2}\right|dx = -\int_0^y \frac{x^2 - y^2}{(x^2 + y^2)^2}dx + \int_y^1\frac{x^2 - y^2}{(x^2 + y^2)^2}$\\ [8pt]

			$\phantom{\int_0^1 \left|\frac{x^2 - y^2}{(x^2 + y^2)^2}\right|dx} = -\frac{-1}{2y}\sin(2\tan^{-1}(1)) + \frac{-1}{2y}\sin(2\tan^{-1}(1/y)) -\frac{-1}{2y}\sin(2\tan^{-1}(1))$\\ [8pt]

			$\phantom{\int_0^1 \left|\frac{x^2 - y^2}{(x^2 + y^2)^2}\right|dx} = \frac{1}{y} - \frac{1}{1 + y^2}$\\ [8pt]
		\end{tabular}\retTwo\par}

		But then $\int_0^1 \frac{1}{y} - \frac{1}{1 + y^2}dy = \ln(1) - \ln(0) - \tan^{-1}(1) + \tan^{-1}(0) = \infty$.\retTwo

		Assuming $\left|\frac{x^2 - y^2}{(x^2 + y^2)^2}\right| \in L^+$, we know by Tonelli-theorem that:
		
		{\centering $\int_E \left|\frac{x^2 - y^2}{(x^2 + y^2)^2}\right|dm^2 = \int_0^1\int_0^1\left|\frac{x^2 - y^2}{(x^2 + y^2)^2}\right|dxdy$.\retTwo\par}

		So, our function is not integrable with respect to $m^2$.\retTwo
	\end{myIndent}

	\item[(b)] $f(x, y) = (1 - xy)^{-a}$ with $a > 0$.
	
	\begin{myIndent}\exTwoP
		$(1 - xy)^{-a}$ is nonnegative for all $x, y \in [0, 1]$ and continuous almost everywhere, we know $(1 - xy)^{-a} \in L^+(E)$. By Tonelli's theorem, we thus know that the following three integrals exist and:

		{\centering $\int (1-xy)^{-a}dm^2 = \int_0^1 \int_0^1 (1-xy)^{-a}dxdy = \int_0^1 \int_0^1 (1-xy)^{-a}dydx$ \retTwo\par}

		That said, it could be that they are all infinite.\retTwo
	\end{myIndent}

	\item[(c)] $f(x, y) = (x - \frac{1}{2})^{-3}$ if $0 < y < |x - \frac{1}{2}|$ and $f(x, y) = 0$ otherwise.  
	
	\begin{myIndent}\exTwoP
		Note that when $y \leq \frac{1}{2}$, then:
		
		{\centering 
		\begin{tabular}{l}
			$\int_0^1 f(x, y)dx = \int_0^{\frac{1}{2}-y}(x-\frac{1}{2})^{-3}dx + \int_{\frac{1}{2} + y}^1(x-\frac{1}{2})^{-3}dx$\\

			$\phantom{\int_0^1 f(x, y)dx} = \frac{1}{-2}(x-\frac{1}{2})^{-2}\left.\right|_0^{\sfrac{1}{2}-y} + \frac{1}{-2}(x-\frac{1}{2})^{-2}\left.\right|_{y+\sfrac{1}{2}}^{1} = \frac{-1}{2y^2} + 2 - 2 - \frac{-1}{2y^2} = 0$
		\end{tabular}\retTwo\par}

		Also, when $y \geq \frac{1}{2}$, then we trivially have that $\int_0^1 f(x, y)dx = \int_0^1 0dx = 0$. So $\int_0^1 \int_0^1 f(x, y)dxdy = 0$.
		\retTwo

		On the other hand, note that $\int_0^1 f(x, y)dy = \int_0^{|x - \sfrac{1}{2}|} (x - \frac{1}{2})^{-3}dy = \frac{|x-\frac{1}{2}|}{(x-\frac{1}{2})^3}$.\retTwo

		I didn't have time to do the calculations but plugging it into an online calculator: $\int_0^1 \frac{|x-\frac{1}{2}|}{(x-\frac{1}{2})^3}dx = \infty$

		So $\int_0^1\int_0^1 f(x, y)dxdy \neq \int_0^1\int_0^1 f(x, y)dydx$ and $f(x, y) \notin L^1(m^2)$

		\newpage

	\end{myIndent}

\end{enumerate}

\retTwo

\blab{Exercise 2.59}: Let $f(x) = x^{-1}\sin(x)$.

\begin{enumerate}
	\item[(a)] Show that $\int_0^\infty |f(x)|dx = \infty$
	\item[(b)] Show that $\lim\limits_{b \rightarrow \infty} \int_0^b f(x)dx = \frac{1}{2}\pi$ by integrating $e^{-xy}\sin(x)$ with respect to $x$ and $y$. 
\end{enumerate}

\newpage








{\hOne\mHeader{Homework 5}}

\blab{Exercise 3.8}: Given a signed measure $\nu$ and positive measure $\mu$ on the measurable space $(X, \mathcal{M})$, we have that: $\nu \ll \mu$ iff $|\nu| \ll \mu$ iff $\nu^+ \ll \mu$ and $\nu^- \ll \mu$.

\begin{myIndent}\exTwoP
	Let $\nu^+$ and $\nu^-$ be the positive and negative variations of $\nu$. Also let $P$ and $N$ be a Hahn decomposition for $\nu$. Finally, let $E \in \mathcal{M}$ be such that $\mu(E) = 0$.\retTwo

	Suppose $\nu \ll \mu$. Thus $\nu(E) = 0$. Then consider if $\nu^+(E) \neq 0$. Since\\ $\nu(E) = \nu^+(E) - \nu^-(E)$ and either $\nu^+$ or $\nu^-$ is finite, we have that\\ $\nu^+(E) = M = \nu^-(E)$ where $0 < M < \infty$. But $\nu^+(E) = \nu(E \cap P)$ and\\ $\nu^-(E) = \nu(E \cap N)$. So $\nu(E \cap P) = m \neq 0$ and $\nu(E \cap N) = m \neq 0$. But at the same time, $\mu(E \cap P) = \mu(E \cap N) = 0$ since $E \cap P \subseteq E$ and $E \cap N \subseteq N$. This contradicts that $\nu \ll \mu$. So, $\nu \ll \mu \Longrightarrow \nu^+ \ll \mu$ and $\nu^- \ll \mu$.\retTwo

	Next, suppose $\mu(E) = 0 \Longrightarrow |\nu|(E) = 0$. Then since $\nu^+(E) \geq 0$, $\nu^-(E) \geq 0$,\\ and $|\nu|(E) = \nu^+(E) + \nu^-(E)$, we must have that $\nu^+(E) = 0 \nu^-(E)$. So\\ $|\nu|(E) \ll \mu \Longrightarrow \nu^+ \ll \mu$ and $\nu^- \ll \mu$.\retTwo

	Finally, it's trivial that $\nu^+ \ll \mu$ and $\nu^- \ll \mu \Longrightarrow |\nu| \ll \mu$ because:
	
	{\centering $\mu(E) \Longrightarrow \nu^+(E) = 0 = \nu^-(E) \Longrightarrow |\nu|(E) = \nu^+(E) + \nu^-(E) = 0 + 0$.\retTwo\par}

	Similarly, $\nu^+ \ll \mu$ and $\nu^- \ll \mu \Longrightarrow \nu \ll \mu$ because:

	{\centering $\mu(E) \Longrightarrow \nu^+(E) = 0 = \nu^-(E) \Longrightarrow \nu(E) = \nu^+(E) - \nu^-(E) = 0 - 0$.\retTwo\par}
\end{myIndent}

\blab{Exercise 3.9}: Let $(X, \mathcal{M})$ be a measurable space and suppose $\mu$ is a signed measure and $(\nu_j)_{j \in \mathbb{N}}$ is a sequence of of positive measures.\\ [2pt] If $\nu_j \perp \mu$ for all $j$, then $\sum\limits_{j=1}^\infty \nu_j \perp \mu$, and if $\nu_j \ll \mu$ for all $j$, then $\sum\limits_{j=1}^\infty v_j \ll \mu$.\\ [2pt]

\begin{myIndent}\exTwoP
	First let's consider if $\nu_j \perp \mu$ for all $j$.
	
	\begin{myIndent}\exPP
		For each $j \in \mathbb{N}$, let $E_j$ and $F_j$ be such that $E_j \cup F_j = X$,\phantom{a} $E_j \cap F_j = \emptyset$,\\ $\nu_j(F_j) = 0$, and $E_j$ is a $\mu$-null set.\retTwo
	
		Define $E \coloneq \bigcup\limits_{j \in \mathbb{N}} E_j$. Then $E$ is a $\mu$-null set.\retTwo
	
		Also, $F \coloneq X - E = X - \bigcup\limits_{j \in \mathbb{N}} E_j = \bigcap\limits_{j \in \mathbb{N}}X - E_j = \bigcap\limits_{j \in \mathbb{N}} F_j$.\retTwo
	
		Importantly, $E \cap F = \emptyset$, $E \cup F = X$ and $F \subseteq F_j$ for all $j \in \mathbb{N}$. Thus $\nu_j(F) = 0$ for all $j \in \mathbb{N}$, and so $\sum\limits_{j = 1}^\infty \nu_j(F) = 0$. We've thus shown that $\sum\limits_{j = 1}^\infty \nu_j \perp \mu$.\retTwo
	\end{myIndent}

	Next, let's consider if $\nu_j \ll \mu$ for all $j$.
	
	\begin{myIndent}\exPP
		Then $\mu(E) = 0 \Longrightarrow \nu_j(E) = 0$ for all $j \in \mathbb{N} \Longrightarrow \sum\limits_{j=1}^\infty \nu_j(E) = 0$. So $\sum\limits_{j=1}^\infty \nu_j \ll \mu$.\newpage
	\end{myIndent}
\end{myIndent}

\blab{Exercise 3.3}: Let $\nu$ be a signed measure on $(X, \mathcal{M})$. Show the following:
\begin{enumerate}
	\item[(a)] $L^1(\nu) = L^1(|\nu|)$
	
	\begin{myIndent}\exTwoP
		Let $\nu^+$ and $\nu^-$ be the positive and negative variations of $\nu$. Recall that\\ $f \in L^1(\nu)$ iff $\int f d\nu = \int f d\nu^+ - \int fd\nu^-$ exists and is finite. The latter\\ statement is true if and only if $f \in L^1(\nu^+) \cap L^1(\nu^-)$.\retTwo

		Meanwhile $|\nu| = \nu^+ + \nu^-$. As a result, when $f = \chi_E$, then:
		
		{\centering$\int f d|\nu| = |\nu|(E) = \nu^+(E) + \nu^-(E) = \int f d\nu^+ + \int fd\nu^-$.\retTwo\par}

		By linearity, we thus conclude that $\int f d|\nu| = \int f d\nu^+ + \int f d\nu^-$ when $f$ is\\ a simple function. And by considering increasing sequences of simple functions and using the M.C.T, we can conclude that $\int f d|\nu| = \int f d\nu^+ + \int f d\nu^-$ holds for all nonnegative $f$.\retTwo

		Now suppose $f$ is a real-valued function. If all the terms below are finite, then we have that:

		{\centering\exPP\begin{tabular}{l}
			$\int fd|\nu| = \int f^+d|\nu| - \int f^-d|\nu| = (\int f^+d\nu^+ + \int f^+d\nu^-) - (\int f^-d\nu^+ + \int f^-d\nu^-) $\\ [6pt]

			$\phantom{\int fd|\nu| = \int f^+d|\nu| - \int f^-d|\nu|} =  (\int f^+ d\nu^+ - \int f^-d\nu^+) + (\int f^+ d\nu^- - \int f^-d\nu^-)$\\ [6pt]

			$\phantom{\int fd|\nu| = \int f^+d|\nu| - \int f^-d|\nu|} = \int f d\nu^+ + \int fd\nu^-$
		\end{tabular}\retTwo\par}

		Importantly, $\int f^+ d|\nu| < \infty$ if and only if $\int f^+ d\nu^+ < \infty$ and $\int f^+ d\nu^- < \infty$. Similarly, $\int f^- d|\nu| < \infty$ if and only if $\int f^- d\nu^+ < \infty$ and $\int f^- d\nu^- < \infty$.\\ Thus:

		{\centering\exPP\begin{tabular}{l}
			$f \in L^1(|\nu|) \Longleftrightarrow \int f^+ d|\nu| < \infty$ and $\int f^- d|\nu| < \infty$ \\ [6pt]

			$\phantom{f \in L^1(|\nu|)} \Longleftrightarrow \int f^+ d\nu^+ < \infty$, $\int f^+ d\nu^- < \infty$, $\int f^- d\nu^+ < \infty$, and $\int f^- d\nu^- < \infty$\\ [6pt]

			$\phantom{f \in L^1(|\nu|)} \Longleftrightarrow f \in L^1(\nu^+)$ and $f \in L^1(\nu^-) \Longleftrightarrow f \in L^1(\nu^+) \cap L^1(\nu^-)$.
		\end{tabular} \retTwo\par}

		Finally, note that if $f$ is complex-valued, then assuming none of the integrals below are infinite:

		{\centering\exPP\begin{tabular}{l}
			$\int fd|\nu| = \int \rea(f)d|\nu| + i\int \ima(f)d|\nu|$\\ [6pt]
			
			$\phantom{\int fd|\nu|} = (\int \rea(f)d\nu^+ + \int \rea(f)d\nu^-) + i(\int \ima(f)d\nu^+ + \int \ima(f)d\nu^-) $\\ [6pt]

			$\phantom{\int fd|\nu|} =  (\int \rea(f) d\nu^+ + i\int \ima(f)d\nu^+) + (\int \rea(f) d\nu^- + i\int \ima(f)d\nu^-)$\\ [6pt]

			$\phantom{\int fd|\nu|} = \int f d\nu^+ + \int fd\nu^-$
		\end{tabular}\retTwo\par}

		Importantly:

		{\centering\exPP\begin{tabular}{l}
			$f \in L^1(|\nu|) \Longleftrightarrow \int \rea(f) d|\nu| < \infty$ and $\int \ima(f) d|\nu| < \infty$ \\ [6pt]

			$\phantom{f \in L^1(|\nu|)} \Longleftrightarrow \int \rea(f) d\nu^+ < \infty$, $\int \rea(f) d\nu^- < \infty$, $\int \ima(f) d\nu^+ < \infty$,\\
			
			\phantom{aaaaaaaaaaaaaaaaaaaaaaaaaaaaaaaaaaaaaaaaaaaaaaaaa} and $\int \ima(f) d\nu^- < \infty$\\ [6pt]

			$\phantom{f \in L^1(|\nu|)} \Longleftrightarrow f \in L^1(\nu^+)$ and $f \in L^1(\nu^-) \Longleftrightarrow f \in L^1(\nu^+) \cap L^1(\nu^-)$.
		\end{tabular} \newpage\par}

		So $L^1(|\nu|) = L^1(\nu^+) \cap L^1(\nu^-) = L^1(\nu)$ and if $f \in L^1(|\nu|)$, then\\ $\int f d|\nu| = \int f d\nu^+ + \int f d\nu^-$. \retTwo
	\end{myIndent}

	\item[(b)] If $f \in L^1(\nu)$, then $| \int f d\nu | \leq \int |f| d|\nu|$.
	
	\begin{myIndent}\exTwoP
		If $f \in L^1(\nu)$, then from before we know $f \in L^1(|\nu|)$. So:

		{\centering \begin{tabular}{l}
			$|\int f d\nu| = |\int f d\nu^+ - \int f d\nu^-| \leq |\int f d\nu^+ - 0| + |\int f d\nu^- - 0|$\\ [6pt]
			
			$\phantom{|\int f d\nu| = |\int f d\nu^+ - \int f d\nu^-|} = \int f d\nu^+ + \int f d\nu^- = \int f d|\nu|$
		\end{tabular}\retTwo\par}

		Next, from our theorems on integrals using positive measure, we know that $|f| \in L^1(|\nu|)$ and $\left|\int f d|\nu|\right| \leq \int |f| d|\nu|$. So:
		
		{\centering$|\int f d\nu| \leq \int f d|\nu| \leq \left|\int f d|\nu|\right| \leq \int |f| d|\nu|$\retTwo\par}
	\end{myIndent}

	\item[(c)] If $E \in \mathcal{M}$, $|\nu|(E) = \sup\{\int_E f d\nu : |f| \leq 1\}$.  
	
	\begin{myIndent}\exTwoP
		Let $P$ and $N$ give a Hahn decomposition for $\nu$ on $X$ with $P$ being positive and\\ $N$ being negative. Then define $f = \chi_P - \chi_N$. Importantly, $|f| \leq 1$ and\\ $\int_E fd\nu = \nu(E \cap P) - \nu(E \cap N) = \nu^+(E) + \nu^-(E) = |\nu|(E)$.\retTwo

		Hence, $|\nu|(E) \leq \sup\{\int_E f d\nu : |f| \leq 1\}$\retTwo

		On the other hand, if $|f| \leq 0$, then by part (b) we know:

		{\centering $\int_E f d\nu \leq |\int_E f d\nu| \leq \int_E |f|d|\nu| \leq \int_E 1\cdot d|\nu| = |\nu|(E)$ \retTwo\par}

		So $\sup\{\int_E f d\nu : |f| \leq 1\} \leq |\nu|(E)$.\retTwo
	\end{myIndent}
\end{enumerate}

\blab{Exercise 3.6}: Suppose $\nu(E) = \int_E f d\mu$ where $\mu$ is a positive measure on $(X, \mathcal{M})$ and $f$ is an extended $\mu$-integrable function. Describe the Hahn decompositions of $\nu$ and the positive, negative, and total variations of $\nu$ in terms of $f$ and $\nu$.

\begin{myIndent}\exTwoP
	Recall that $\int_E f d\mu = \int_E f^+ d\mu - \int_E f^-d\mu$. Thus, defining $\nu^+(E) = \int_E f^+ d\mu$ and\\ [2pt] $\nu^-(E) = \int_E f^-d\mu$, we have that $\nu^+$ and $\nu^-$ are positive measures such that\\ $\nu = \nu^+ - \nu^-$.\retTwo
	
	Additionally, let $P = f^{-1}( [0, \infty])$ and $N = f^{-1}( [-\infty, 0))$. Since $f$ is\\ measurable, we know both $P$ and $N$ are in $\mathcal{M}$. Also, $P \cap N = \emptyset$ and\\ $P \cup N = \mathrm{domain}(f) = X$. And thirdly, $f^+ = 0$ on $N$ and $f^- = 0$ on $P$.\\ Thus, if $E \subseteq N$, then $\nu^+(E) = \int_E f^+d\mu = \int 0 d\mu = 0$.  Similarly, if $E \subseteq P$,\\ then $\nu^-(E) = \int_E f^-d\mu = \int 0 d\mu = 0$.\retTwo

	Hence, we've shown that $\nu^+(E) = \int_E f^+ d\mu$ and $\nu^-(E) = \int_E f^-d\mu$ are positive measures such that $\nu = \nu^+ - \nu^-$ and $\nu^+ \perp \nu^-$. By the Jordan decomposition theorem, we thus know that $\nu^+$ and $\nu^-$ give the unique Jordan decomposition of $\nu$. In other words, the positive variation of $\nu$ is given by $\int_E f^+ d\mu$ and the negative variation of $\nu$ is given by $\int_E f^- d\mu$.\newpage

	In turn, the total variation of $\nu$ is $\int_E f^+ d\mu + \int_E f^- d\mu = \int_E |f|d\mu$\retTwo

	Now we turn to describing the Hahn decompositions for $\nu$.
	\begin{myIndent}
		Claim: $E$ is positive iff $\mu(E \cap f^{-1}([-\infty, 0))) = 0$.
		
		\begin{myIndent}\exPPP
			Proof:\\
			Suppose $\mu(E \cap f^{-1}([-\infty, 0))) = 0$. Then given any $F \in \mathcal{M}$ with $F \subseteq E$, we have that $\int_F fd\mu = \int_F f^+d\mu - \int_F f^- d\mu \geq 0 - \int_E f^-d\mu = 0$ since $f^- = 0$ a.e. on $E$. So $\nu(F) \geq 0$. Since $F$ was arbitrary, we've shown $E$ is positive.\retTwo

			Conversely, suppose $\mu(E \cap f^{-1}([-\infty, 0))) > 0$. Then because $f$ is measurable, we can fix $F = E \cap f^{-1}([-\infty, 0))$ and have that $F \in \mathcal{M}$ and $F \subseteq E$. But now note that $\nu(F) = \int_F fd\mu = 0 - \int_F f^{-1}d\mu$. And since $f^- > 0$ on all of $F$ (with $F$ not being null), we do not have that $f^{-1} = 0$ a.e. on $F$. That combined with the fact that $f^{-1} \in L^+$ means that $\int_F f^{-1}d\mu > 0$ and thus $\nu(F) < 0$. So, $E$ isn't positive.\retTwo
		\end{myIndent}

		Analogously, we can show that $E$ is negative iff $\mu(E \cap f^{-1}((0, \infty])) = 0$.\retTwo

		So two subsets $P$ and $N$ of $X$ will be a Hahn decomposition for $\nu$ if and only if $P = A_1 \cup f^{-1}((0, \infty]) \cup B_2 - B_1$ and $N = A_2 \cup f^{-1}([-\infty, 0))  \cup B_1 - B_2$\\ where $B_1$ and $B_2$ are null-subsets of $f^{-1}((0, \infty])$ and $f^{-1}([-\infty, 0))$\\ respectively and $A_1$ and $A_2$ are disjoint sets in $\mathcal{M}$ partitioning $f^{-1}(\{0\})$.\retTwo

		
		\begin{myIndent}
			Recall that earlier in this problem, we defined $P = f^{-1}([0, \infty])$ and $N = f^{-1}([-\infty, 0))$. We know from before that $P \cap N = \emptyset$ and\\ $P \cup N = X$. Also, we know that $P$ is positive and $N$ is negative due to our claims right above. So, $f^{-1}([0, \infty])$ and $f^{-1}([-\infty, 0))$ give a Hahn decomposition for $\nu$.\retTwo
		\end{myIndent}
	\end{myIndent}
\end{myIndent}

\blab{Exercise 2.60}: Show that $\frac{\Gamma(x)\Gamma(y)}{\Gamma(x + y)} = \int_0^1 t^{x-1}(1 - t)^{y-1}dt$ for $x, y > 0$.

\begin{myIndent}\exTwoP
	As a consequence of exercise 2.51, we can rewrite:
	
	{\center \begin{tabular}{l}
		$\Gamma(x)\Gamma(y) = \int_0^\infty s^{x-1}e^{-s}\mathrm{d}s \int_0^\infty s^{y-1}e^{-s}\mathrm{d}s = \int_0^\infty \int_0^\infty t^{x-1}e^{-t}s^{y-1}e^{-s}\mathrm{d}s\mathrm{d}t$\\ [6pt]

		$\phantom{\Gamma(x)\Gamma(y) = \int_0^\infty s^{x-1}e^{-s}\mathrm{d}s \int_0^\infty s^{y-1}e^{-s}\mathrm{d}s} = \int_0^\infty \int_0^\infty t^{x-1}s^{y-1}e^{-(s + t)}\mathrm{d}s\mathrm{d}t$
	\end{tabular}\retTwo\par}

	Now we want to do a change of variables. The clues to focus on is that we want to consolidate the exponential by making $(t + s) = u$, and we want to have the term $(1 - w)^{y-1}$ in our final integral (meaning $s$ is a multiple of $1 - w$).\retTwo

	Define $G(u, w) = (uw, u(1 - w))$. Then $D_{(u,w)}G = \left[
	\begin{smallmatrix}
		w & u \\ 1-w & -u
	\end{smallmatrix}\right]$. Thus, $G$ is $\mathcal{C}^1$\\ continuous and $|\det(D_{(u,w)}G)| = |{-}uw -u + uw| = |u|$.\retTwo

	Also, $G$ is injective when $u \neq 0$.
	\begin{myIndent}\exPPP
		Proof:\\
		Suppose $uw = u^\prime w^\prime$ and $u(1 - w) = u^\prime(1 - w^\prime)$.\newpage
		
		If $w = 0$, then $u^\prime w^\prime = 0 = uw$. So $u(1 - w) = u^\prime(1 - w^\prime)$ becomes $u = u^\prime$ and in turn $u^\prime w^\prime = uw \Longrightarrow w^\prime = w$.\retTwo

		Meanwhile, if $w \neq 0$, then $u = \frac{u^\prime w^\prime}{w}$. So:
		
		{\centering $u^\prime - u^\prime w^\prime = u^\prime(1 - w^\prime) = u(1 - w) = \frac{u^\prime w^\prime}{w}(1 - w) = \frac{u^\prime w^\prime}{w} - u^\prime w^\prime$.\retTwo\par}

		This says that $u^\prime w = u^\prime w^\prime$. By assumption, $u^\prime \neq 0$. So, canceling $u^\prime$ we get that $w = w^\prime$. Then since $w \neq 0$, $uw = u^\prime w^\prime \Longrightarrow u = u^\prime$.\retTwo
	\end{myIndent}

	Finally, $G$ maps $(0, \infty) \times (0, 1)$ to $(0, \infty) \times (0, \infty)$.
	\begin{myIndent}\exPPP
		Proof:\\
		Given $(t, s) \in (0, \infty) \times (0, \infty)$, let $u = t + s$ and $w = \frac{t}{t + s}$. Then $u \in (0, \infty)$, $v \in (0, 1)$, and $g(u, v) = ((t + s)\frac{t}{t + s}, (t + s)(1 - \frac{t}{t + s})) = (t, t + s - t) = (t, s)$.\retTwo
	\end{myIndent}

	So:

	{\centering
	\begin{tabular}{l}
		$\int_0^\infty \int_0^\infty t^{x-1}s^{y-1}e^{-(s + t)}dsdt = \int_0^1 \int_0^\infty u^{x-1}w^{x-1}u^{y-1}(1-w)^{y-1}e^{-u}|u|\mathrm{d}u\mathrm{d}w$\\ [6pt]

		$\phantom{\int_0^\infty \int_0^\infty t^{x-1}s^{y-1}e^{-(s + t)}dsdt} = \int_0^1 w^{x-1}(1-w)^{y-1}\mathrm{d}w\int_0^\infty u^xu^{y-1} e^{-u}\mathrm{d}u$\\ [6pt]

		$\phantom{\int_0^\infty \int_0^\infty t^{x-1}s^{y-1}e^{-(s + t)}dsdt} = \int_0^1 w^{x-1}(1-w)^{y-1}\mathrm{d}w \cdot \Gamma(x + y)$
	\end{tabular} \retTwo\par}
\end{myIndent}

\blab{Exercise 2.63}: The technique used to prove proposition 2.54 can also be used to integrate any polynomial over $S^{n-1}$. Suppose $f(x) = \prod\limits_{j=1}^n x_j^{\alpha_j}$. Then $\int f d\sigma = 0$ if any $\alpha_j$ is odd, and\\ [-2pt] if all $\alpha_j$'s are even, then:

{\centering ${\displaystyle\int_{S_n} f d\sigma = \frac{2\Gamma(\beta_1)\cdots\Gamma(\beta_n)}{\Gamma(\beta_1 + \ldots + \beta_n)}}$ where ${\displaystyle \beta_j = \frac{\alpha_j + 1}{2}}$.\retTwo\par}

\begin{myIndent}\exTwoP
	Note that $\int_{\mathbb{R}^n} f(x)e^{-|x|^2}dx = \int_{\mathbb{R}^n} x_1^{\alpha_1}\cdots x_n^{\alpha_n}e^{-x_1^2 - \cdots - x_n^2}dx$.\retTwo

	If we use the Fubini-Tonelli theorem, we can simplify the above expression a lot. But first we need to prove that $f(x)e^{-|x|^2}$ is integrable on $\mathbb{R}^n$.\retTwo

	One way to do this is show that $|t|^\alpha e^{-t^2} \in L^1$ for all integers $\alpha \in \mathbb{N} \cup \{0\}$.
	
	\begin{myIndent}\exPPP
		If $\alpha = 2\beta - 1$, meaning that $\beta = \frac{\alpha + 1}{2}$, then:
		
		{\centering $\int_{-\infty}^\infty |t|^\alpha e^{-t^2}dt = \int_{-\infty}^\infty |t|^{2\beta - 1} e^{-t^2}dt = 2\int_0^\infty t^{2\beta - 1}e^{-t^2}dt$.\retTwo\par}
		
		Now subsitute in $s = t^2$. Then our expression becomes $\frac{2}{2}\int_0^\infty t^{2\beta - 2}e^{-s}ds = \int_0^\infty s^{\beta - 1}e^{-s}ds$. When $\alpha \geq 0$, then $\beta \geq \frac{1}{2} > 0$. So $\int_0^\infty s^{\beta - 1}e^{-s}ds = \Gamma(\beta) < \infty$.\retTwo
	\end{myIndent}

	Then by exercise 2.51, since $x_1^{\alpha_1} e^{-x_1^2} \in L^1(m)$ and $x_2^{\alpha_2} e^{-x_2^2} \in L^1(m)$, we have that $\left(x_1^{\alpha_1} e^{-x_1^2}\right)\left(x_2^{\alpha_2} e^{-x_2^2}\right) = x_1^{\alpha_1}x_2^{\alpha_2}e^{-x_1^2 - x_2^2} \in L^1(m \times m)$. Additionally, since $x_3^{\alpha_3} e^{-x_3^2} \in L^1(m)$, we thus have that:
	
	{\centering $\left(x_1^{\alpha_1}x_2^{\alpha_2}e^{-x_1^2 - x_2^2}\right)\left(x_3^{\alpha_3} e^{-x_3^2}\right) = x_1^{\alpha_1}x_2^{\alpha_2}x_3^{\alpha_3}e^{-x_1^2 - x_2^2 - x_3^2} \in L^1((m \times m) \times m)$.
	\newpage\par}
	
	Because $m$ is $\sigma$-finite, if we equate $(A \times B) \times C$ and $A \times B \times C$, we have that $(m \times m) \times m = m \times m \times m$. Thus $x_1^{\alpha_1}x_2^{\alpha_2}x_3^{\alpha_3}e^{-x_1^2 - x_2^2 - x_3^2} \in L^1(m \times m \times m)$.\retTwo
	
	Continuing likewise, we get that $f(x)e^{-|x|^2} \in L^1(m \times \ldots \times m) \subseteq L^1(m^n)$.\retTwo

	With that, we can now apply Fubini's theorem to get that:

	{\centering $\int_{\mathbb{R}^n} x_1^{\alpha_1}\cdots x_n^{\alpha_n}e^{-x_1^2 - \cdots - x_n^2}dx = \prod\limits_{j=1}^n \int_{-\infty}^\infty x_j^{\alpha_j}e^{-x_j^2}dx_j$ \retTwo\par}

	If some $\alpha_j$ is odd, then for that $j$ we have $\int_{-\infty}^\infty x_j^{\alpha_j}e^{-x_j^2}dx_j = 0$. So the entire product cancels.\retTwo

	Meanwhile, if $\alpha_j$ is even, then $x_j^{\alpha_j} = |x_j|^{\alpha_j}$ and so from before we know that $\int_{-\infty}^\infty x_j^{\alpha_j}e^{-x_j^2}dx_j = \Gamma(\beta_j)$ where $\beta_j = \frac{\alpha_j + 1}{2}$. Hence, if all $\alpha_j$ are even, then our\\ [2pt] original integral evaluates to $\Gamma(\beta_1)\cdots\Gamma(\beta_n)$.\retTwo
	
	Next, we integrate $f(x)e^{-|x|^2}$ with respect to polar coordinates. To start, note that if $x^\prime = (x^\prime_1, \ldots, x^\prime_n)$, then:

	{\centering $f(rx^\prime) = r^{\alpha_1}(x^\prime_1)^{\alpha_1}\cdots r^{\alpha_n}(x^\prime_n)^{\alpha_n} = r^{\alpha_1 + \cdots + \alpha_n}(x^\prime_1)^{\alpha_1}\cdots (x^\prime_n)^{\alpha_n}$ \retTwo\par}

	Therefore: 
	
	{\centering 
	\begin{tabular}{l}
		$\Gamma(\beta_1)\cdots\Gamma(\beta_n) = \int_{S^{n-1}} \int_0^\infty r^{\alpha_1 + \cdots + \alpha_n}(x^\prime_1)^{\alpha_1}\cdots (x^\prime_n)^{\alpha_n} r^{n-1} e^{-r^2}drd\sigma(x^\prime)$\\ [6pt]

		$\phantom{\Gamma(\beta_1)\cdots\Gamma(\beta_n)} = \int_{S^{n-1}} (x^\prime_1)^{\alpha_1}\cdots (x^\prime_n)^{\alpha_n} d\sigma(x^\prime) \cdot \int_0^\infty r^{\alpha_1 + \cdots + \alpha_n} r^{n-1} e^{-r^2}dr$\\ [6pt]

		$\phantom{\Gamma(\beta_1)\cdots\Gamma(\beta_n)} = \int_{S^{n-1}} f d\sigma \cdot \int_0^\infty r^{(\alpha_1 + 1) + \cdots + (\alpha_n + 1) - 1} e^{-r^2}dr$\\ [6pt]

		$\phantom{\Gamma(\beta_1)\cdots\Gamma(\beta_n)} = \int_{S^{n-1}} f d\sigma \cdot \int_0^\infty r^{2(\beta_1 + \cdots + \beta_n) - 1} e^{-r^2}dr$\\ [6pt]

		$\phantom{\Gamma(\beta_1)\cdots\Gamma(\beta_n)} = \frac{1}{2}\int_{S^{n-1}} f d\sigma \cdot \int_0^\infty r^{2(\beta_1 + \cdots + \beta_n) - 2} e^{-s}ds$\\ [6pt]

		$\phantom{\Gamma(\beta_1)\cdots\Gamma(\beta_n)} = \frac{1}{2}\int_{S^{n-1}} f d\sigma \cdot \int_0^\infty s^{(\beta_1 + \cdots + \beta_n) - 1} e^{-s}ds$\\ [6pt]
		
		$\phantom{\Gamma(\beta_1)\cdots\Gamma(\beta_n)} = \frac{1}{2}\int_{S^{n-1}} f d\sigma \cdot \Gamma(\beta_1 + \ldots + \beta_n)$\\ [6pt]
	\end{tabular}\retTwo\par}
\end{myIndent}

\blab{Exercise 2.64:} For which real values of $a$ and $b$ is $\left|x\right|^a \left|\log\left|x\right|\right|^b$ integrable over\\ $\{x \in \mathbb{R}^n : |x| < \frac{1}{2}\}$ or over $\{x \in \mathbb{R}^n : |x| > 2\}$.

\begin{myIndent}\exTwoP
	Let $B_1 = \{x \in \mathbb{R}^n : |x| < \frac{1}{2}\}$ and $B_2 = \{x \in \mathbb{R}^n : |x| > 2\}$. Note that\\ $|\log|x|| > \log(2)$ for all $x \in B_1 \cup B_2$. As a consequence, when $b > 0$, we\\ have that $|\log|x||^b > (\log(2))^b > 0$. Meanwhile, when $b \leq 0$, we have that\\ $0 \leq |\log|x||^b < (\log(2))^b$\retTwo

	As a result, assuming $a > -n$ and $b > 0$, then:
	
	{\centering $|x|^a|\log|x||^b \geq |x|^a(\log(2))^b \geq |x|^{-n}(\log(2))^b$ for all $x \in B_1$.\retTwo\par}
	
	So by corollary 2.52, $|x|^a|\log|x||^b$ wouldn't be integrable on $B_1$.\retTwo

	Similarly, assuming $a > -n$ and $b \leq 0$, we have:
	
	{\centering $|x|^a\left|\log|x|\right|^b \leq (\log(2))^b|x|^a$ for all $x \in B^1$.\newpage\par}

	So by the same corollary, $|x|^a|\log|x||^b$ would be integrable on $B_1$.\retTwo

	Next suppose $a \leq -n$. If $b > 0$. Then $|x|^a|\log|x||^b \geq |x|^a(\log(2))^b$. But\\ $\int_0^{\sfrac{1}{2}} r^a(\log(2))^br^{n-1}dr = (\log(2))^b\int_0^{\sfrac{1}{2}} r^{a + n-1}dr \leq (\log(2))^b\int_0^{\sfrac{1}{2}} r^{-1}dr = +\infty$.\\ So $|x|^a|\log|x||^b$ wouldn't be integrable on $B_1$.\retTwo

	Fourthly, if $a \leq -n$ and $b \leq 0$ then:
	
	{\centering 
	\begin{tabular}{l}
		$\int_{\mathbb{R}^n} |x|^a|\log|x||^b dx = \sigma(S^{n-1})\int_{0}^{\sfrac{1}{2}}r^{a}|\log(r)|^b r^{n-1}dr$\\ [4pt]

		$\phantom{\int_{\mathbb{R}^n} |x|^a|\log|x||^b dx} = \sigma(S^{n-1})\int_{0}^{\sfrac{1}{2}}r^{a+n-1}(\log(r^{-1}))^b dr$\\ [4pt]

		$\phantom{\int_{\mathbb{R}^n} |x|^a|\log|x||^b dx} = \sigma(S^{n-1})\int_{2}^{\infty}y^{-a-n+1}(\log(y))^b \frac{1}{y^2}dy$ (substituting $r = \frac{1}{y^2}$)\\ [4pt]

		$\phantom{\int_{\mathbb{R}^n} |x|^a|\log|x||^b dx} = \sigma(S^{n-1})\int_{2}^{\infty}y^{-a-n-1}(\log(y))^bdy$\\ [4pt]

		$\phantom{\int_{\mathbb{R}^n} |x|^a|\log|x||^b dx} \leq (\log(2))^b\sigma(S^{n-1})\int_{2}^{\infty}y^{-a-n-1}dy$ (since $b \leq 0$)
	\end{tabular}\retTwo\par}

	But $-a \geq n$ and so $y^{-a-n-1} \geq y^{-1}$ for all $y > 2$. Hence, $\int_{2}^{\infty}y^{-a-n-1}dy \geq \int_{2}^{\infty}y^{-1}dy = \infty$.  Thus, if $a \leq -n$ and $b \leq 0$, then $|x|^a|\log|x||^b$ is not integrable on $B_1$.\retTwo

	We conclude that $|x|^a|\log|x||^b$ is integrable on $B_1$ if and only if $a > -n$ and $b \leq 0$.\retTwo

	Finally, note that:

	{\centering 
	\begin{tabular}{l}
		$\int_{B_2}|x|^a|\log|x||^bdx = \sigma(S^{n-1})\int_2^\infty r^a (\log(r))^b r^{n-1}dr$\\ [4pt]

		$\phantom{\int_{B_2}|x|^a|\log|x||^bdx} = \sigma(S^{n-1})\int_0^{\sfrac{1}{2}} y^{-a} (\log(y^{-1}))^b y^{-n+1}\frac{1}{y^2}dr$\\ [4pt]

		$\phantom{\int_{B_2}|x|^a|\log|x||^bdx} = \sigma(S^{n-1})\int_0^{\sfrac{1}{2}} y^{-a} (\log(y^{-1}))^b y^{-n-1}dr$\\ [4pt]

		$\phantom{\int_{B_2}|x|^a|\log|x||^bdx} = \sigma(S^{n-1})\int_0^{\sfrac{1}{2}} y^{-a - 2n} (\log(y^{-1}))^b y^{-n+2n-1}dr$\\ [4pt]

		$\phantom{\int_{B_2}|x|^a|\log|x||^bdx} = \sigma(S^{n-1})\int_0^{\sfrac{1}{2}} y^{-a - 2n} (\log(y^{-1}))^b y^{n-1}dr$\\ [4pt]

		$\phantom{\int_{B_2}|x|^a|\log|x||^bdx} = \int_{B_1} |x|^{-a-2n}|\log|x||^bdx$\\ [4pt]
	\end{tabular}\retTwo\par}

	So $|x|^a|\log|x||^b$ on $B_2$ if and only if $|x|^{-a-2n}|\log|x||^b$ on $B_1$, and that happens when $b \leq 0$ and $-a-2n > -n \Longrightarrow -a > n \Longrightarrow a < -n$.\retTwo

	Thus, $|x|^a|\log|x||^b$ is integrable on $B_1$ if and only if $a > -n$ and $b \leq 0$. Also, $|x|^a|\log|x||^b$ is integrable on $B_2$ if and only if $a < -n$ and $b \leq 0$.
\end{myIndent}

\newpage
\exOne
\mHeader{Homework 6}

\blab{Exercise 3.13}: Let $X = [0, 1]$, $\mathcal{M} = \mathcal{B}_{[0, 1]}$, $m = \text{Lebesgue measure}$ and $\mu = \text{ counting measure}$ on $\mathcal{M}$.
\begin{enumerate}
	\item[(a)] $m \ll \mu$ but $\df m \neq f\df\mu$ for any $f$.
	
	\begin{myIndent}\exTwoP
		The only null set of $\mu$ is the emptyset and the empty set has measure zero in all measures. So all measures are absolutely continuous with respect to $\mu$.\retTwo

		Next, assume such an $f$ as in the prompt exists. Then for all $x_0 \in X$, we have:

		{\centering\begin{tabular}{l}
			 $0 = m(\{x_0\}) = \int_{\{x_0\}}f\df\mu = \int f\chi_{\{x_0\}}\df \mu$\\
			 $\phantom{0 = m(\{x_0\}) = \int_{\{x_0\}}f\df\mu} = \int f(x_0)\chi_{\{x_0\}}\df \mu = f(x_0)\mu(\{x_0\}) = f(x_0)$
		\end{tabular} \retTwo\par}

		The second line of those manipulations follows because $f(x_0)\chi_{\{x_0\}}$ is a simple function.\retTwo

		So we've shown that we must have $f = 0$. In turn, for all $E \in \mathcal{M}$ we have that:

		{\centering $\int_E f \df\mu = \sup\left\{\sum\limits_{x \in F}f(x) : F\subseteq E \text{ with } F \text{ finite}\right\} = 0$.\retTwo\par}

		But if $E = X$, then $m(E) = 1 \neq 0$ (a contradiction).\retTwo
	\end{myIndent}

	\item[(b)] $\mu$ has no Lebesgue decomposition with respect to $m$.
	
	\begin{myIndent}\exTwoP
		Suppose $\lambda$ and $\rho$ are signed measures with $\mu = \lambda + \rho$,\myHS $\lambda \perp m$, and $\rho \ll m$.\retTwo

		Since $m(E) = 0$ for all at most finite sets $F$, we must have that $\rho(F) = 0$ as well. In turn for any finite set $F$ we know:

		{\centering $\mu(F) = \lambda(F) + \rho(F) \Longrightarrow \mu(F) = \lambda(F) + 0 \Longrightarrow \mu(F) = \lambda(F)$ \retTwo\par}

		In turn, we've shown that the only set $\lambda$ is null on is the emptyset, because given any nonempty set $E \in \mathcal{M}$, if we consider a finite nonempty subset of $E$, then $\lambda$ will be positive on that subset (and defined since all finite subsets are closed and thus in $\mathcal{B}_{[0, 1]} = \mathcal{M}$). So, $\lambda \perp m$ implies $m$ is null on $X$. But this contradicts that $m(X) = m([0, 1]) = 1$.\retTwo
	\end{myIndent}
\end{enumerate}

\blab{Exercise 3.16:} Suppose that $\mu, \nu$ are measures on $(X, \mathcal{M})$ with $\nu \ll \mu$, and let $\lambda = \mu + \nu$. If $f = \frac{\df \nu}{ \df \lambda}$, then $0 \leq f < 1$ $\mu$-a.e. and $\frac{\df \nu}{ \df \mu} = \frac{f}{ (1 - f)}$.\\ [-8pt]

\begin{myIndent}\exTwoP
	To start, here's a quick sanity test:
	\begin{myIndent}\exPPP
		If $g = \chi_E$ where $E \in \mathcal{M}$, then:
		
		{\centering $\int g \df(\mu + \nu) = \mu + \nu(E) = \mu(E)  + \nu(E) = \int g \df \mu + \int g \df \nu$.\retTwo\par}

		By linearity this extends to all simple functions. Then by M.C.T. this extends to all measurable nonnegative extended-real-valued functions.\newpage

		Next note that if $g$ is $\mathbb{C}$-valued or $\overline{\mathbb{R}}$-valued, then by the previous reasoning:\\ $\int |g|\df(\mu + \nu)= \int |g|\df\mu + \int |g|\df\nu$. This tells us that $L^1(\mu + \nu) = L^1(\mu) \cap L^1(\nu)$.\\ So assuming $g \in L^1(\mu)$ and $g \in L^1(\nu)$, by splitting into positive / negative and\\ real / imaginary components, we can show that $\int g \df(\mu + \nu) = \int g \df\mu + \int g \df \nu$.\retTwo
	\end{myIndent}

	Now note $\nu(E) = \int_E f d\lambda$. This tells us that $\int_E f d\lambda$ is nonnegative for all $E \in \mathcal{M}$ since $\nu(E)$ is nonnegative for all $E \in \mathcal{M}$. In turn, this means that $f \geq 0$ $\lambda$-a.e.
	
	\begin{myIndent}\exPPP
		Suppose $f < 0$ on a set $E$ with $\lambda(E) > 0$. Then define $A_1 = \{x \in E : f^-(x) \geq 1\}$ and $A_n = \{x \in E : \frac{1}{n-1} > f^-(x) \geq \frac{1}{n}\}$ for all $n > 1$.\\
		
		Then $\lambda(E) = \sum\limits_{n = 1}^\infty \lambda(A_n) > 0$. So there exists $A_n$ with $\lambda(A_n) > 0$.\\
		
		But then $\int_E f \df\lambda = -\int_E f^- \df\lambda \leq -\int_{A_n} f\df\lambda \leq \frac{-1}{n} \lambda(A_n) < 0$, thus contradicting that $\int_E f\df\lambda$ is nonnegative for all $E \in \mathcal{M}$.\retTwo
	\end{myIndent}

	Also observe that if $\mathcal{N}(\lambda)$, $\mathcal{N}(\mu)$, $\mathcal{N}(\nu)$ are the collections of null sets of $\lambda$, $\mu$, and $\nu$ respectively, then $\mathcal{N}(\lambda) = \mathcal{N}(\mu) \cap \mathcal{N}(\nu)$. Thus, we've shown that something being true $\lambda$-a.e. implies it is true $\mu$-a.e. and $\nu$-a.e.\retTwo

	So $f \geq 0$ $\mu$-a.e. and this means that for any $E \in \mathcal{M}$:

	{\centering\begin{tabular}{l}
		$\int_E d\nu = \nu(E) = \int_E f\df\lambda = \int_E f \df\mu + \int_E f \df\nu \Longrightarrow \int_E(1 - f)\df\nu = \int_E f \df\mu \geq 0$
	\end{tabular}\retTwo\par}

	In turn, we have $1 - f \geq 0$\phantom{..}$\nu$-a.e. But from before we know $f \geq 0$\phantom{..}$\nu$-a.e. Hence, $0 \leq f \leq 1$ $\nu$-a.e.\retTwo

	Then, let $A = \{x \in X : f(x) = 1\}$, $B = \{x \in X : f(x) > 1\}$. We can show:
	\begin{itemize}
		\item $0 = 0 \cdot \nu(A) = \int_A(1 - f)\df\nu = \int_A f\df \mu = \int_A \df\mu = \mu(E)$
		\item $0 = \int_B (1-f)\df\nu = \int_B f \df \mu \geq \int_B 1\df\mu = \mu(B) \geq 0$\\
	\end{itemize}

	Hence $0 \leq f < 1$\phantom{..}$\mu$-a.e.\retTwo

	Now by similar logic to how we showed $f \geq 0$ $\lambda$-a.e., we can show $\frac{\df \nu}{\df \mu} \geq 0$ $\mu$-a.e. Also, we must have $\int (1 - f)\frac{\df \nu}{\df \mu}\df \mu = \int f \df \mu$. And since the integrands are all nonnegative (after redefining on a $\mu$-null set), we thus know: $(1 - f)\frac{\df \nu}{\df \mu} = f$\phantom{..}$\mu$-a.e.\retTwo

	Therefore, $\frac{\df \nu}{\df \mu}$ is given by the equivalence class of $\frac{f}{1 - f}$ with respect to $\mu$.\retTwo
\end{myIndent}

\blab{Exercise 3.17:} Let $(X, \mathcal{M}, \mu)$ be a $\sigma$-finite measure space, $\mathcal{N}$ a sub-$\sigma$-algebra of $\mathcal{M}$,\\ and $\nu = \mu|_{\mathcal{N}}$. If $f \in L^1(\mu)$, there exists $g \in L^1(\nu)$ (thus $g$ is $\mathcal{N}$-measurable) such that\\ $\int_E f\df\mu = \int_E g \df\nu$ for all $E \in \mathcal{N}$; if $g^\prime$ is another such function then $g = g^\prime$ $\nu$-a.e.\\ (In probability theory, $g$ is called the \udefine{conditional expectation} of $f$ on $\mathcal{N}$.)\\ [-8pt]

\begin{myIndent}\exTwoP
	If $f \in L^1(\mu)$, then we know that $\rho(E) = \int_E f \df \mu$ is a finite signed measure on $\mathcal{M}$. Also set $\lambda(E) = \rho(E)|_{\mathcal{N}}$. Importantly, $\lambda(X) = \rho(X) < \infty$. So $\lambda$ is finite.\newpage

	Also if we know $\nu$ is $\sigma$-finite, then we know there exists a Radon-Nikodym derivative of $\lambda$ with respect to $\nu$. So we define $g = \frac{\df \lambda}{\df \nu}$. Then $g \in L^1(\nu)$ and for all $E \in \mathcal{N}$ we have $\int_E f\df\mu = \lambda(E) = \int_E g\df\nu$. Additionally, the Lebesgue-Radon-Nikodym guarentees that given another function $g^\prime$ satisfying this, $g^\prime = g$\phantom{..}$\nu$-a.e.\retTwo
	
	\begin{myIndent}\exPPP
		But you see, here Folland is a dumb idiot who doesn't proof-read his own writing. And hell there's nothing in the book's errata either to address this.\retTwo

		Firstly, we can't prove that $\nu$ is $\sigma$-finite based on the information given.
		\begin{myIndent}
			Let $\mathcal{N} = \{X, \emptyset\}$ be our sub-$\sigma$-algebra of $\mathcal{M}$. If $\mu$ is finite, then we\\ still have that $\nu$ is finite. But if $\mu(X) = \infty$, then $\nu(X) = \infty$ and there is no\\ sequence of sets of finite measure in $\mathcal{N}$ whose union is $X$. Hence $\nu$ is\\ not $\sigma$-finite.\retTwo
		\end{myIndent}

		Secondly, if don't restrict $\nu$ in some way (such as demanding it's $\sigma$-finite), then the prompt is false.
		\begin{myIndent}
			Keeping $\mathcal{N} = \{X, \emptyset\}$, note that the only $\mathcal{N}$-measurable functions are constant functions. After all, if $x$ and $y$ are two distinct points in $X$ and $h(x) = a$ and\\ $h(y) = b \neq a$, then $\emptyset \subset \{y\} \subseteq h^{-1}(\{b\}) \subseteq X - \{x\} \subset X$. Hence, $h$ is not $\mathcal{N}$-measurable since $h^{-1}(\{b\}) \notin \mathcal{N}$.\retTwo

			As a consequence, the only function $g \in L^1(\nu)$ is $g = 0$. Thus we must have $\int g d\nu = 0$. However, this gives a contradiction since it's possible $\int f d\mu \neq 0$.\retTwo
		\end{myIndent}

		Finally, I think Folland overlooked this issue because he was blinded by the\\ probability application. After all in probability $\nu$ will be finite because $\mu$ is finite\\ and then we don't have any issues.\retTwo
	\end{myIndent}
\end{myIndent}

\blab{Exercise 3.22} If $f \in L^1(\mathbb{R}^n)$, $f \neq 0$, there exists $C, R > 0$ such that $Hf(x) \geq C|x|^{-n}$ for $|x| > R$. Hence $m(\{x : Hf(x) > \alpha\}) \geq \frac{C^\prime}{\alpha}$ when $\alpha$ is small, so the estimate in the maximal theorem is essentially sharp.

\begin{myIndent}\exTwoP
	I'm going to asssume Folland means $f \neq 0$ a.e. cause otherwise the problem doesn't make sense. Plus, I really should be getting use to thinking about equivalence classes of functions.\retTwo

	Since $f \neq 0$ a.e., there exists $R > 1$ such that $\int_{B(R, 0)} |f(y)|dy > c > 0$ for some $c$ in $\mathbb{R}^n$. Then when $|x| > R$, because $B(R, 0) \subseteq B(2|x|, x)$, we have:\\ [-8pt]

	{\centering\begin{tabular}{l}
		$Hf(x) \geq \frac{1}{m(B(2|x|, x))}\int_{B(2|x|, x)}|f(y)|\df y$\\ [12pt]
		$\phantom{Hf(x)} \geq \frac{1}{m(B(2|x|, x))}\int_{B(R, 0)}|f(y)|\df y$\\ [12pt]
		$\phantom{Hf(x)} = \frac{1}{2^n|x|^n m(B(1, 0))}\int_{B(R, 0)}|f(y)|\df y > \frac{c}{2^n m(B(1, 0))}|x|^{-n}$
	\end{tabular}\retTwo\par}

	Setting $C = \frac{c}{2^n m(B(1, 0))}$, we've thus shown the first claim.\newpage

	Next, consider when $\alpha < \frac{C}{2R^n} < \frac{C}{R^n}$. Then since $R > 1$, we know that $\sqrt[n]{\sfrac{C}{\alpha}} > R$. So for $R < |x| < \sqrt[n]{\sfrac{C}{\alpha}}$, we have:
	
	{\centering $Hf(x) > C|x|^{-n} > C\left(\sqrt[n]{\sfrac{C}{\alpha}}\right)^{-n} = C\frac{\alpha}{C} = \alpha$ \retTwo\par}

	In turn:
	
	{\centering\begin{tabular}{l}
		$m(\{x: Hf(x) > \alpha\}) \geq m(R < |x| < \sqrt[n]{\sfrac{C}{\alpha}})$\\
		$\phantom{m(\{x: Hf(x) > \alpha\})} = m(\sqrt[n]{\sfrac{C}{\alpha}}, 0) - m(R, 0) = (\frac{C}{\alpha} - R^n)m(B(1,0))$
	\end{tabular}\retTwo\par}
	
	Now $\alpha < \frac{C}{2R^n} \Longrightarrow R^n < \frac{C}{2\alpha}$. So finally we have that:

	{\centering $(\frac{C}{\alpha} - R^n)m(B(1,0)) \geq (\frac{C}{\alpha} - \frac{C}{2\alpha})m(B(1,0)) = \frac{1}{\alpha}\frac{Cm(B(1,0))}{2}$ \retTwo\par}

	Set $C^\prime = \frac{Cm(B(1,0))}{2}$ and we are done.\retTwo
\end{myIndent}

\blab{Exercise 3.23:} A useful variant of the Hardy-Littlewood maximal function is:

{\centering $H^*f(x) = \sup\left\{\frac{1}{m(B)}\int_B |f(y)|dy : B \text{ is a ball and } x \in B\right\}$ \retTwo\par}

Show that $Hf \leq H^*f \leq 2^n Hf$.

\begin{myIndent}\exTwoP
	It's trivial that $Hf(x) \leq H^*f(x)$ because all the balls $Hf(x)$ is taking a sup over are included in the set of balls that $H^*f(x)$ is taking a sup over.\retTwo

	Meanwhile, given any ball $B(R, z)$ containing $x$, we know $B(R, z)\subseteq B(2R, x)$, and thus:

	{\centering\begin{tabular}{l}
		$ \frac{1}{m(B(R, z))}\int_{B(R, z)}|f(y)|\df y = \frac{1}{m(B(R, x))}\int_{B(R, z)}|f(y)|\df y $\\ [8pt]
		$\phantom{ \frac{1}{m(B(R, z))}\int_{B(R, z)}|f(y)|\df y} \leq \frac{1}{m(B(R, x))}\int_{B(2R, x)}|f(y)|\df y $
		\\ [8pt]
		$\phantom{ \frac{1}{m(B(R, z))}\int_{B(R, z)}|f(y)|\df y} = \frac{2^n}{m(B(2R, x))}\int_{B(2R, x)}|f(y)|\df y \leq 2^n Hf(x)$
	\end{tabular}\retTwo\par}

	So $H^*f(x) \leq 2^nHf(x)$.\retTwo
\end{myIndent}

\blab{Exercise 3.25:} If $E$ is a Borel set in $\mathbb{R}^n$, the \udefine{density} $D_E(x)$ of $E$ at $x$ is defined (whenever the limit exists) as:

{\centering $D_E(x) = \lim\limits_{r \rightarrow 0} \frac{m(E \cap B(r, x))}{m(B(r, x))}$\par}

\begin{enumerate}
	\item[(a)] Show that $D_E(x) = 1$ for a.e. $x \in E$ and $D_E(x)$ for a.e. $x \in E^\comp$.
	
	\begin{myIndent}\exTwoP
		Define the measure $\nu(F) = m(E \cap F)$. Importantly, $\nu$ is still $\sigma$-finite and also $m(F) = 0 \Longrightarrow m(E \cap F) = \nu(F) = 0$, meaning $\nu \ll m$. Thus by the Radon-Nikodym Theorem (and the fact that $\nu$ is positive), there exists $f \in L^+$ such that $\df\nu = f \df m$.\retTwo

		We also have $f \in L^1_{\loc}$. After all, if $K$ is a bounded measurable set, then\\ $0 \leq \int_K f \df m = \nu(K) < m(K) < \infty$. 
		\newpage

		Thus, by a result in the book $\nu$ is regular, and in turn by theorem 3.22, we know that:
		
		{\centering $\lim\limits_{r \rightarrow 0}\frac{\nu(B(r,x))}{m(B(r,x))} = \lim\limits_{r \rightarrow 0}\frac{m(E \cap B(r,x))}{m(B(r,x))} = f(x)$ for $m$-a.e. $x$.\retTwo\par}

		As for what $f(x)$ is, note:
		\begin{itemize}
			\item $\int_{E^\comp}f\df m = \nu(E^\comp) = 0$. Thus $f = 0$ m-a.e. on $E^\comp$.
			\item Let $\{V_n\}_n$ be a sequence of subsets of $E$ with finite measure such that\\ $E = \bigcup_{n \in \mathbb{N}} V_n$. Then for any measurable $F \subseteq V_n$, we can say that:\\ [-8pt]
			
			{\centering $\int_{F}|f - 1|\df m = \int_{F}f \df m - \int_{F}\df m = \nu(V_n) - m(V_n) = 0$.\\ [4pt]\par}

			This tells us that $f - 1 = 0$ $m$-a.e. on $V_n$. So $f = 1$ $m$-a.e. on $V_n$. Then repeating this for all $V_n$ tells us that $f = 1$ $m$-a.e. on $E$.
		\end{itemize}
		\retTwo
	\end{myIndent}

	\item[(b)] Find examples of $E$ and $x$ such that $D_E(x)$ is a given number $\alpha \in (0, 1)$, or such that $D_E(x)$ does not exist.
	
	\begin{myIndent}\exTwoP
		
		\retTwo
	\end{myIndent}
\end{enumerate}

\blab{Exercise 3.26:} If $\lambda$ and $\mu$ are positive, mutually singular Borel measures on $\mathbb{R}^n$ and $\lambda + \mu$ is regular, then so are $\lambda$ and $\mu$.

\begin{myIndent}\exTwoP
	Let $K$ be a compact subset of $E \in \mathcal{B}_{\mathbb{R}^n}$. Thus we know that $\lambda(K) + \mu(K) < \infty$. And because $0 \leq \lambda(K)$ and $0 \leq \mu(K)$, we thus must have that:
	
	{\centering $\lambda(K), \mu(K) \leq \lambda(K) + \mu(K) <\infty$.\retTwo\par}

	Thus the requirement of regularity is satisfied. Technically this is enough because the first requirement implies the second according to the textbook. However, we haven't proven that yet. So I'm going to try and prove the second requirement.\retTwo

	Let $E, F \in \mathcal{B}_{\mathbb{R}^n}$ be such that $E \cap F = \emptyset$, $E \cup F = \mathbb{R}^n$, $\lambda(E) = 0$ and $\mu(E) = 0$.\retTwo

	\dots actually nevermind I don't have time to prove this before my extended deadline runs out\dots
\end{myIndent}


% \blab{Exercise 2.11:}



\end{document}








% $\bigotimes\limits_{\alpha \in A}\mathcal{M}_\alpha = \{\pi^{-1}_\alpha(E_\alpha) \mid E_\alpha \in \mathcal{M}_\alpha, \alpha \in A\}$