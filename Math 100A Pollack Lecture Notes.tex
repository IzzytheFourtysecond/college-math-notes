\documentclass{book}

\usepackage{fontspec} % used to import Calibri
\usepackage{anyfontsize} % used to adjust font size

% needed for inch and other length measurements
% to be recognized
\usepackage{calc}

% for colors and text effects as is hopefully obvious
\usepackage[dvipsnames]{xcolor}
\usepackage{soul}

% control over margins
\usepackage[margin=1in]{geometry}
\usepackage[strict]{changepage}

\usepackage{mathtools}
\usepackage{amsfonts}
\usepackage{bm}

\usepackage[scr=rsfso, scrscaled=.96]{mathalpha}

\usepackage{amssymb} % originally imported to get the proof square
\usepackage{xfrac}
\usepackage[overcommands]{overarrows} % Get my preferred vector arrows...
\usepackage{relsize}

% Just am using this to get a dashed line in a table...
% Also you apparently want this to be inactive if you aren't
% using it because it slows compilation.
\usepackage{arydshln} \ADLinactivate 
\newenvironment{allowTableDashes}{\ADLactivate}{\ADLinactivate}

\usepackage{graphicx}
\graphicspath{{./158_Images/}}

\usepackage{tikz}
   \usetikzlibrary{arrows.meta}
   \usetikzlibrary{graphs, graphs.standard}

\usepackage{quiver} %commutative diagrams


\newfontfamily{\calibri}{Calibri}
\setlength{\parindent}{0pt}
\definecolor{RawerSienna}{HTML}{945D27}

% ~~~~~~~~~~~~~~~~~~~~~~~~~~~~~~~~~~~~~~~~~~~~~~~~~~
%Arrow Commands:

% Thank you Bernard, gernot, and Sigur who I copied this from:
% https://tex.stackexchange.com/questions/364096/command-for-longhookrightarrow
\newcommand{\hooklongrightarrow}{\lhook\joinrel\longrightarrow}
\newcommand{\hooklongleftarrow}{\longleftarrow\joinrel\rhook}
\newcommand{\hookxlongrightarrow}[2][]{\lhook\joinrel\xrightarrow[#1]{#2}}
\newcommand{\hookxlongleftarrow}[2][]{\xleftarrow[#1]{#2}\joinrel\rhook}

% Thank you egreg who I copied from:
% https://tex.stackexchange.com/questions/260554/two-headed-version-of-xrightarrow
\newcommand{\longrightarrowdbl}{\longrightarrow\mathrel{\mkern-14mu}\rightarrow}
\newcommand{\longleftarrowdbl}{\leftarrow\mathrel{\mkern-14mu}\longleftarrow}

\newcommand{\xrightarrowdbl}[2][]{%
  \xrightarrow[#1]{#2}\mathrel{\mkern-14mu}\rightarrow
}
\newcommand{\xleftarrowdbl}[2][]{%
  \leftarrow\mathrel{\mkern-14mu}\xleftarrow[#1]{#2}
}

\newcommand{\MRoman}[1]{%
   \textrm{\MakeUppercase{\romannumeral #1}}%
}

% ~~~~~~~~~~~~~~~~~~~~~~~~~~~~~~~~~~~~~~~~~~~~~~~~~~

\newcommand{\learnToSpot}[1]{{\color{Red}#1}}

\newcommand{\hOne}{%
   \color{Black}%
   \fontsize{14}{16}\selectfont%
}
\newcommand{\hTwo}{%
\color{MidnightBlue}%
   \fontsize{13}{15}\selectfont%
}
\newcommand{\hThree}{%
   \color{PineGreen!85!Orange}
   \fontsize{12}{14}\selectfont%
}
\newcommand{\myComment}{%
   \color{RawerSienna}%
   \fontsize{12}{14}\selectfont%
}
\newcommand{\teachComment}{
   \color{Orange}%
   \fontsize{12}{14}\selectfont%
}
\newcommand{\exOne}{%
   \color{Purple}%
   \fontsize{13}{15}\selectfont%
}
\newcommand{\exTwo}{%
   \color{Purple}%
   \fontsize{13}{15}\selectfont%
}
\newcommand{\exP}{%
   \color{Purple}%
   \fontsize{12}{14}\selectfont%
}
\newcommand{\exTwoP}{%
   \color{RedViolet}%
   \fontsize{13}{15}\selectfont%
}
\newcommand{\exPP}{%
   \color{RedViolet}%
   \fontsize{12}{14}\selectfont%
}
% ~~~~~~~~~~~~~~~~~~~~~~~~~~~~~~~~~~~~~~~~~~~~~~~~

\newcommand{\cyPen}[1]{{\vphantom{.}\color{Cerulean}#1}}
\newcommand{\redPen}[1]{{\vphantom{.}\color{Red}#1}}

\newenvironment{myIndent}{%
   \begin{adjustwidth}{2.5em}{0em}%
}{%
   \end{adjustwidth}%
}

\newenvironment{myDindent}{%
   \begin{adjustwidth}{5em}{0em}%
}{%
   \end{adjustwidth}%
}

\newenvironment{myTindent}{%
   \begin{adjustwidth}{7.5em}{0em}%
}{%
   \end{adjustwidth}%
}

\newenvironment{myConstrict}{%
   \begin{adjustwidth}{2.5em}{2.5em}%
}{%
   \end{adjustwidth}%
}

\newcommand{\udefine}[1]{{%
   \setulcolor{Red}%
   \setul{0.14em}{0.07em}%
   \ul{#1}%
}}

\newcommand{\blab}[1]{\textbf{#1}}

\newcommand{\uuline}[2][.]{%
{\vphantom{a}\color{#1}%
\rlap{\rule[-0.18em]{\widthof{#2}}{0.06em}}%
\rlap{\rule[-0.32em]{\widthof{#2}}{0.06em}}}%
#2}

\newcommand{\pprime}{{\prime\prime}}
\newcommand{\suchthat}{ \hspace{0.3em}s.t.\hspace{0.3em}}
\newcommand{\rea}[1]{\mathrm{Re}(#1)}
\newcommand{\ima}[1]{\mathrm{Im}(#1)}
\newcommand{\comp}{\mathsf{C}}
\newcommand{\card}{\mathrm{card}}
\newcommand{\diam}{\mathrm{diam}}
\newcommand{\Sym}{\mathrm{Sym}}
\newcommand{\myHS}{ \hspace{0.5em}}

\newcommand{\myId}{\mathrm{Id}}
\newcommand{\myIm}{\mathrm{im}}
\newcommand{\myObj}{\mathrm{Obj}}
\newcommand{\myHom}{\mathrm{Hom}}
\newcommand{\myEnd}{\mathrm{End}}
\newcommand{\myAut}{\mathrm{Aut}}

\newcommand{\mcateg}[1]{{\bm{\mathsf{#1}}}}

% Thank you Gonzalo Medina and Moriambar who wrote this on stack exchange:
%https://tex.stackexchange.com/questions/74125/how-do-i-put-text-over-symbols%
\newcommand{\myequiv}[1]{\stackrel{\mathclap{\mbox{\footnotesize{$#1$}}}}{\equiv}}

% Thank you chs who wrote this on stack exchange:
%https://tex.stackexchange.com/questions/89821/how-to-draw-a-solid-colored-circle%
\newcommand{\filledcirc}[1][.]{\ensuremath{\hspace{0.05em}{\color{#1}\bullet}\mathllap{\circ}\hspace{0.05em}}}

%Thank you blerbl who wrote this on stack exchange:
%https://tex.stackexchange.com/questions/25348/latex-symbol-for-does-not-divide
\newcommand{\ndiv}{\hspace{-0.3em}\not|\hspace{0.35em}}

\newcommand{\mySepOne}[1][.]{%
   {\noindent\color{#1}{\rule{6.5in}{1mm}}}\\%
}
\newcommand{\mySepTwo}[1][.]{%
   {\noindent\color{#1}{\rule{6.5in}{0.5mm}}}\\%
}

\newenvironment{myClosureOne}[2][.]{%
   \color{#1}%
   \begin{tabular}{|p{#2in}|} \hline \\%
}{%
   \\ \hline \end{tabular}%
}

\newcommand{\retTwo}{\hfill\bigbreak}

\newcommand{\mHeader}[1]{{
   \color{Black}%
   \fontsize{20}{18}\selectfont%
   #1\retTwo
}}


\title{Math 100A Notes (Professor: Aaron Pollack)}
\author{Isabelle Mills}

\begin{document}
\maketitle{}
\setul{0.14em}{0.07em}
\calibri

\hOne
\mHeader{Lecture 1 Notes: 9/27/2024}

\blab{Motivation for this class:}

Let $\mathcal{F}$ be any figure in $\mathbb{R}^2$. We want some way of talking about the symmetries of $\mathcal{F}$.\retTwo

Letting $d$ be the standard metric for $\mathbb{R}^2$, we say $f: \mathbb{R}^2 \longrightarrow \mathbb{R}^2$ is \udefine{distance preserving}\\ if $d(P, Q) = d(f(P), f(Q))$ for all $P, Q \in \mathbb{R}^2$. If $f$ is distance-preserving and\\ $f(\mathcal{F}) = \mathcal{F}$, then we call $f$ a \udefine{symmetry} of $\mathcal{F}$.\retTwo

We define $\Sym(\mathcal{F})$ to be the set of symmetries of $\mathcal{F}$.
\begin{myIndent}\hTwo
	Lemma 2: The set $\Sym(\mathcal{F})$ has the following properties:
	
	\begin{enumerate}
		\item The identity map $\myId$ is in $\Sym(\mathcal{F})$
		\item If $f \in \Sym(\mathcal{F})$, then $f^{-1} \in \Sym(\mathcal{F})$.
		\begin{myIndent}\hThree
			I realize we haven't yet shown that every $f \in \Sym(\mathcal{F})$ is a bijection. Given such an $f$, it's easy to see that $f$ must be injective. After all, the distance\\ preserving property of $f$ means that $f(P) = f(Q) \Longrightarrow P = Q$. Showing that $f$ is surjective is harder. By assumption, we know that $f$ is surjective when restricted to $\mathcal{F}$. More complicatedly, we can show that $f$ must have a certain form which happens to be surjective. Perhaps I'll prove that later.\retTwo

			Once, you've accepted that $f^{-1}$ exists, then it's clearly true that $f^{-1}$ is also order-preserving with $f^{-1}(\mathcal{F}) = \mathcal{F}$.
		\end{myIndent}
		\item If $f_1, f_2 \in \Sym(\mathcal{F})$, then $f_1 \circ f_2 \in \Sym(\mathcal{F})$ and $f_2 \circ f_1 \in \Sym(\mathcal{F})$.
		\begin{myIndent}\hThree
			This is pretty trivial to show.\retTwo
		\end{myIndent}
	\end{enumerate}
\end{myIndent}

Now while it's all good that we have a concrete way of describing the symmetries of a figure, our current terminology is not the most useful. After all, suppose $\mathcal{S}$ and $\mathcal{S}^\prime$ are two squares such that $\mathcal{S}$ is centered at the origin and $\mathcal{S}^\prime$ is centered at the point $(5, 5)$. Then even though we know both $\mathcal{S}$ and $\mathcal{S}^\prime$ have symmetries in the form of rotating and reflecting, the particular functions in $\Sym(\mathcal{S})$ and $\Sym(\mathcal{S})$ will be different (except for $\myId$). So, how do we compare the symmetries of those two squares?\retTwo

\mySepTwo

\blab{Proof that all symmetries are surjective (from our textbook)}:\\
\begin{myIndent}\myComment
	Note:\\ [-18pt]
	\begin{itemize}
		\item Our textbook calls a distance-preserving function $f: \mathbb{R}^n \longrightarrow \mathbb{R}^n$ an \udefine{isometry}.
		\item Rather than writing $f_1 \circ f_2$ to represent function composition, our textbook just\\ writes $f_1f_2$.\newpage
	\end{itemize}

	\hTwo
	\blab{Some Facts:}
	\begin{itemize}
		\item[(a)] Orthogonal linear operators are isometries.
		
		\begin{myIndent}\hThree
			Let $\varphi$ be n orthogonal linear map. $\varphi$ being linear means that\\ $\varphi(u) - \varphi(v) = \varphi(u - v)$. Meanwhile, $\varphi$ being orthogonal means that\\ $|\varphi(u - v)| = \sqrt{\varphi(u - v) \cdot \varphi(u -v)} = \sqrt{(u - v) \cdot (u - v)} = |u - v|$.\\ So, for any $u, v \in \mathbb{R}^n$, we have that $|\varphi(u) - \varphi(v)| = |u - v|$.\retTwo
		\end{myIndent}

		\item[(b)] The translation $t_a$ by a vector $a$ defined by $t_a(x) = x + a$ is an isometry.
		\begin{myIndent}\hThree
			For any $u, v \in \mathbb{R}^n$, we have $|t_a(u) - t_a(v)| = |u + a - v - a| = |u - v|$.\retTwo
		\end{myIndent}

		\item[(c)] The composition of isometries is an isometry.
		\begin{myIndent}\hThree
			If $f_1, f_2$ are isometries, then for all $u, v \in \mathbb{R}^n$, we have that\\ $|f_1(f_2(u)) - f_1(f_2(v))| = |f_2(u) - f_2(v)| = |u - v|$.
		\end{myIndent}
	\end{itemize}

\end{myIndent}

\newpage

\exOne Our textbook is \textit{Algebra, Second Edition} by Michael Artin.



\end{document}