\documentclass{book}

\usepackage{fontspec} % used to import Calibri
\usepackage{anyfontsize} % used to adjust font size

% needed for inch and other length measurements
% to be recognized
\usepackage{calc}

% for colors and text effects as is hopefully obvious
\usepackage[dvipsnames]{xcolor}
\usepackage{soul}

% control over margins
\usepackage[margin=1in]{geometry}
\usepackage[strict]{changepage}

\usepackage{mathtools}
\usepackage{amsfonts}
\usepackage{bm}

\usepackage[scr=rsfso, scrscaled=.96]{mathalpha}

\usepackage{amssymb} % originally imported to get the proof square
\usepackage{xfrac}
\usepackage[overcommands]{overarrows} % Get my preferred vector arrows...
\usepackage{relsize}

% Just am using this to get a dashed line in a table...
% Also you apparently want this to be inactive if you aren't
% using it because it slows compilation.
\usepackage{arydshln} \ADLinactivate 
\newenvironment{allowTableDashes}{\ADLactivate}{\ADLinactivate}

\usepackage{graphicx}
\graphicspath{{./158_Images/}}

\usepackage{tikz}
   \usetikzlibrary{arrows.meta}
   \usetikzlibrary{graphs, graphs.standard}

\usepackage{quiver} %commutative diagrams


\newfontfamily{\calibri}{Calibri}
\setlength{\parindent}{0pt}
\definecolor{RawerSienna}{HTML}{945D27}

% ~~~~~~~~~~~~~~~~~~~~~~~~~~~~~~~~~~~~~~~~~~~~~~~~~~
%Arrow Commands:

% Thank you Bernard, gernot, and Sigur who I copied this from:
% https://tex.stackexchange.com/questions/364096/command-for-longhookrightarrow
\newcommand{\hooklongrightarrow}{\lhook\joinrel\longrightarrow}
\newcommand{\hooklongleftarrow}{\longleftarrow\joinrel\rhook}
\newcommand{\hookxlongrightarrow}[2][]{\lhook\joinrel\xrightarrow[#1]{#2}}
\newcommand{\hookxlongleftarrow}[2][]{\xleftarrow[#1]{#2}\joinrel\rhook}

% Thank you egreg who I copied from:
% https://tex.stackexchange.com/questions/260554/two-headed-version-of-xrightarrow
\newcommand{\longrightarrowdbl}{\longrightarrow\mathrel{\mkern-14mu}\rightarrow}
\newcommand{\longleftarrowdbl}{\leftarrow\mathrel{\mkern-14mu}\longleftarrow}

\newcommand{\xrightarrowdbl}[2][]{%
  \xrightarrow[#1]{#2}\mathrel{\mkern-14mu}\rightarrow
}
\newcommand{\xleftarrowdbl}[2][]{%
  \leftarrow\mathrel{\mkern-14mu}\xleftarrow[#1]{#2}
}

\newcommand{\MRoman}[1]{%
   \textrm{\MakeUppercase{\romannumeral #1}}%
}

% ~~~~~~~~~~~~~~~~~~~~~~~~~~~~~~~~~~~~~~~~~~~~~~~~~~

\newcommand{\learnToSpot}[1]{{\color{Red}#1}}

\newcommand{\hOne}{%
   \color{Black}%
   \fontsize{14}{16}\selectfont%
}
\newcommand{\hTwo}{%
\color{MidnightBlue}%
   \fontsize{13}{15}\selectfont%
}
\newcommand{\hThree}{%
   \color{PineGreen!85!Orange}
   \fontsize{12}{14}\selectfont%
}
\newcommand{\myComment}{%
   \color{RawerSienna}%
   \fontsize{12}{14}\selectfont%
}
\newcommand{\teachComment}{
   \color{Orange}%
   \fontsize{12}{14}\selectfont%
}
\newcommand{\exOne}{%
   \color{Purple}%
   \fontsize{13}{15}\selectfont%
}
\newcommand{\exTwo}{%
   \color{Purple}%
   \fontsize{13}{15}\selectfont%
}
\newcommand{\exP}{%
   \color{Purple}%
   \fontsize{12}{14}\selectfont%
}
\newcommand{\exTwoP}{%
   \color{RedViolet}%
   \fontsize{13}{15}\selectfont%
}
\newcommand{\exPP}{%
   \color{RedViolet}%
   \fontsize{12}{14}\selectfont%
}
% ~~~~~~~~~~~~~~~~~~~~~~~~~~~~~~~~~~~~~~~~~~~~~~~~

\newcommand{\cyPen}[1]{{\vphantom{.}\color{Cerulean}#1}}
\newcommand{\redPen}[1]{{\vphantom{.}\color{Red}#1}}

\newenvironment{myIndent}{%
   \begin{adjustwidth}{2.5em}{0em}%
}{%
   \end{adjustwidth}%
}

\newenvironment{myDindent}{%
   \begin{adjustwidth}{5em}{0em}%
}{%
   \end{adjustwidth}%
}

\newenvironment{myTindent}{%
   \begin{adjustwidth}{7.5em}{0em}%
}{%
   \end{adjustwidth}%
}

\newenvironment{myConstrict}{%
   \begin{adjustwidth}{2.5em}{2.5em}%
}{%
   \end{adjustwidth}%
}

\newcommand{\udefine}[1]{{%
   \setulcolor{Red}%
   \setul{0.14em}{0.07em}%
   \ul{#1}%
}}

\newcommand{\blab}[1]{\textbf{#1}}

\newcommand{\uuline}[2][.]{%
{\vphantom{a}\color{#1}%
\rlap{\rule[-0.18em]{\widthof{#2}}{0.06em}}%
\rlap{\rule[-0.32em]{\widthof{#2}}{0.06em}}}%
#2}

\newcommand{\pprime}{{\prime\prime}}
\newcommand{\suchthat}{ \hspace{0.3em}s.t.\hspace{0.3em}}
\newcommand{\rea}[1]{\mathrm{Re}(#1)}
\newcommand{\ima}[1]{\mathrm{Im}(#1)}
\newcommand{\comp}{\mathsf{C}}
\newcommand{\card}{\mathrm{card}}
\newcommand{\diam}{\mathrm{diam}}
\newcommand{\Sym}{\mathrm{Sym}}
\newcommand{\myHS}{ \hspace{0.5em}}

\newcommand{\myId}{\mathrm{Id}}
\newcommand{\myIm}{\mathrm{im}}
\newcommand{\myObj}{\mathrm{Obj}}
\newcommand{\myHom}{\mathrm{Hom}}
\newcommand{\myEnd}{\mathrm{End}}
\newcommand{\myAut}{\mathrm{Aut}}

\newcommand{\mcateg}[1]{{\bm{\mathsf{#1}}}}

% Thank you Gonzalo Medina and Moriambar who wrote this on stack exchange:
%https://tex.stackexchange.com/questions/74125/how-do-i-put-text-over-symbols%
\newcommand{\myequiv}[1]{\stackrel{\mathclap{\mbox{\footnotesize{$#1$}}}}{\equiv}}

% Thank you chs who wrote this on stack exchange:
%https://tex.stackexchange.com/questions/89821/how-to-draw-a-solid-colored-circle%
\newcommand{\filledcirc}[1][.]{\ensuremath{\hspace{0.05em}{\color{#1}\bullet}\mathllap{\circ}\hspace{0.05em}}}

%Thank you blerbl who wrote this on stack exchange:
%https://tex.stackexchange.com/questions/25348/latex-symbol-for-does-not-divide
\newcommand{\ndiv}{\hspace{-0.3em}\not|\hspace{0.35em}}

\newcommand{\mySepOne}[1][.]{%
   {\noindent\color{#1}{\rule{6.5in}{1mm}}}\\%
}
\newcommand{\mySepTwo}[1][.]{%
   {\noindent\color{#1}{\rule{6.5in}{0.5mm}}}\\%
}

\newenvironment{myClosureOne}[2][.]{%
   \color{#1}%
   \begin{tabular}{|p{#2in}|} \hline \\%
}{%
   \\ \hline \end{tabular}%
}

\newcommand{\retTwo}{\hfill\bigbreak}

\newcommand{\mHeader}[1]{{
   \color{Black}%
   \fontsize{20}{18}\selectfont%
   #1\retTwo
}}


\title{Math 100A Notes (Professor: Aaron Pollack)}
\author{Isabelle Mills}

\begin{document}
\maketitle{}
\setul{0.14em}{0.07em}
\calibri

\hOne
\mHeader{Lecture 1 Notes: 9/27/2024}

\blab{Motivation for this class:}

Let $\mathcal{F}$ be any figure in $\mathbb{R}^2$. We want some way of talking about the symmetries of $\mathcal{F}$.\retTwo

Letting $d$ be the standard metric for $\mathbb{R}^2$, we say $f: \mathbb{R}^2 \longrightarrow \mathbb{R}^2$ is \udefine{distance preserving}\\ if $d(P, Q) = d(f(P), f(Q))$ for all $P, Q \in \mathbb{R}^2$. If $f$ is distance-preserving and\\ $f(\mathcal{F}) = \mathcal{F}$, then we call $f$ a \udefine{symmetry} of $\mathcal{F}$.\retTwo

We define $\Sym(\mathcal{F})$ to be the set of symmetries of $\mathcal{F}$.
\begin{myIndent}\hTwo
	Lemma 2: The set $\Sym(\mathcal{F})$ has the following properties:
	
	\begin{enumerate}
		\item The identity map $\myId$ is in $\Sym(\mathcal{F})$
		\item If $f \in \Sym(\mathcal{F})$, then $f^{-1} \in \Sym(\mathcal{F})$.
		\begin{myIndent}\hThree
			I realize we haven't yet shown that every $f \in \Sym(\mathcal{F})$ is a bijection. Given such an $f$, it's easy to see that $f$ must be injective. After all, the distance\\ preserving property of $f$ means that $f(P) = f(Q) \Longrightarrow P = Q$. Showing that $f$ is surjective is harder. By assumption, we know that $f$ is surjective when restricted to $\mathcal{F}$. More complicatedly, we can show that $f$ must have a certain form which happens to be surjective. Perhaps I'll prove that later.\retTwo

			Once, you've accepted that $f^{-1}$ exists, then it's clearly true that $f^{-1}$ is also distance preserving with $f^{-1}(\mathcal{F}) = \mathcal{F}$.
		\end{myIndent}
		\item If $f_1, f_2 \in \Sym(\mathcal{F})$, then $f_1 \circ f_2 \in \Sym(\mathcal{F})$ and $f_2 \circ f_1 \in \Sym(\mathcal{F})$.
		\begin{myIndent}\hThree
			This is pretty trivial to show.\retTwo
		\end{myIndent}
	\end{enumerate}
\end{myIndent}

Now while it's all good that we have a concrete way of describing the symmetries of a figure, our current terminology is not the most useful. After all, suppose $\mathcal{S}$ and $\mathcal{S}^\prime$ are two squares such that $\mathcal{S}$ is centered at the origin and $\mathcal{S}^\prime$ is centered at the point $(5, 5)$. Then even though we know both $\mathcal{S}$ and $\mathcal{S}^\prime$ have symmetries in the form of rotating and reflecting, the particular functions in $\Sym(\mathcal{S})$ and $\Sym(\mathcal{S})$ will be different (except for $\myId$). So, how do we compare the symmetries of those two squares?\retTwo

\mySepTwo

Aside start\dots\retTwo

\blab{Proof that all symmetries are surjective (taken from our textbook)}:\\
\begin{myIndent}\myComment
	Note:\\ [-18pt]
	\begin{itemize}
		\item Our textbook calls a distance-preserving function $f: \mathbb{R}^n \longrightarrow \mathbb{R}^n$ an \udefine{isometry}.
		\item Rather than writing $f_1 \circ f_2$ to represent function composition, our textbook just\\ writes $f_1f_2$.\newpage
	\end{itemize}

	\hTwo
	\blab{Some Facts:}
	\begin{itemize}
		\item[(a)] Orthogonal linear operators are isometries.
		
		\begin{myIndent}\hThree
			Let $\varphi$ be n orthogonal linear map. $\varphi$ being linear means that\\ $\varphi(u) - \varphi(v) = \varphi(u - v)$. Meanwhile, $\varphi$ being orthogonal means that\\ $|\varphi(u - v)| = \sqrt{\varphi(u - v) \cdot \varphi(u -v)} = \sqrt{(u - v) \cdot (u - v)} = |u - v|$.\\ So, for any $u, v \in \mathbb{R}^n$, we have that $|\varphi(u) - \varphi(v)| = |u - v|$.\retTwo
		\end{myIndent}

		\item[(b)] The translation $t_a$ by a vector $a$ defined by $t_a(x) = x + a$ is an isometry.
		\begin{myIndent}\hThree
			For any $u, v \in \mathbb{R}^n$, we have $|t_a(u) - t_a(v)| = |u + a - v - a| = |u - v|$.\retTwo
		\end{myIndent}

		\item[(c)] The composition of isometries is an isometry.
		\begin{myIndent}\hThree
			If $f_1, f_2$ are isometries, then for all $u, v \in \mathbb{R}^n$, we have that\\ $|f_1(f_2(u)) - f_1(f_2(v))| = |f_2(u) - f_2(v)| = |u - v|$.\retTwo
		\end{myIndent}
	\end{itemize}

	\blab{Theorem 6.2.3:} The following conditions on a map $\varphi: \mathbb{R}^n \longrightarrow \mathbb{R}^n$ are equivalent:
	\begin{itemize}
		\item[(a)] $\varphi$ is an isometry such that $\varphi(0) = 0$.
		\item[(b)] $\varphi$ preserves dot products: $\varphi(u) \cdot \varphi(w) = u \cdot w$ for all $u, w \in \mathbb{R}^n$.
		\item[(c)] $\varphi$ is an orthogonal linear operator.
		
		\begin{myIndent}\hThree
			Proof:\\
			(c) $\Longrightarrow$ (a)\\
			This comes both from the first fact on this page plus the fact that all linear\\ operators map $0$ to $0$.\retTwo

			(b) $\Longrightarrow$ (c)\\
			Our challenge here is to show that such a $\varphi$ has to be linear operator.\retTwo

			\blab{Lemma:} For $x, y \in \mathbb{R}^n$, if $(x \cdot x) = (x \cdot y) = (y \cdot y)$, then $x = y$.
			
			\begin{myIndent}\hThree
				Proof: $|x - y|^2 = (x - y) \cdot (x - y) = (x \cdot x) - 2(x \cdot y) + (y \cdot y)$.\retTwo
			\end{myIndent}

			Consider any $u, v \in \mathbb{R}^n$ and set $w = u + v$. Then set $u^\prime = \varphi(u)$,\\ $v^\prime = \varphi(v)$, and $w^\prime = \varphi(w)$. To show that $w^\prime = v^\prime + u^\prime$, we shall show\\ that $(w^\prime \cdot w^\prime) = (w^\prime \cdot (u^\prime + v^\prime)) = ((u^\prime + v^\prime) \cdot (u^\prime + v^\prime))$.\retTwo

			Firstly, simplify our equation to:

			{\centering $(w^\prime \cdot w^\prime) = (w^\prime \cdot u^\prime) + (w^\prime \cdot v^\prime) = (u^\prime \cdot u^\prime) + 2(u^\prime \cdot v^\prime) + (v^\prime \cdot v^\prime)$ \retTwo\par}

			Next, since $\varphi$ is assumed to preserve dot products, we can thus simplify our\\ equation to:

			{\centering $(w \cdot w) = (w \cdot u) + (w \cdot v) = (u \cdot u) + 2(u\cdot v) + (v \cdot v)$ \retTwo\par}

			And since $w = u + b$, all of those equalities are true. Hence, we know by our lemma above that $w^\prime = u^\prime + v^\prime$.\newpage

			Meanwhile, let $v \in \mathbb{R}^n$ and set $u = cv$ where $c$ is a constant. Then define $u^\prime$ and $v^\prime$ as before. Then we can do a few trivial simplications to show that $(u^\prime \cdot u^\prime)$, $(u^\prime \cdot cv^\prime)$ and $(cv^\prime \cdot cv^\prime)$ are all equal to $c^2(v \cdot v)$. So, $u^\prime = cv^\prime$.\retTwo

			(a) $\Longrightarrow$ (b)\\
			Since $\varphi$ is distance preserving, we know that $\forall u, v \in \mathbb{R}^n$, 

			{\centering$(\varphi(u) - \varphi(v)) \cdot (\varphi(u) - \varphi(v)) = (u - v) \cdot (u -v)|$.\retTwo\par}

			By plugging in $v = 0$, this simplifies to $(\varphi(u) \cdot \varphi(u)) = (u \cdot u)$. Similarly, by plugging in $u = 0$, we can get that $(\varphi(v) \cdot \varphi(v)) = (v \cdot v)$. So, by expanding and canceling out parts of our above expression, we get that:

			{\centering$-2(\varphi(u) \cdot \varphi(v)) = - 2(u \cdot v)$.\retTwo\par}
		\end{myIndent}
	\end{itemize}

	\blab{Corollary 6.2.7:} Every isometry $f$ of $\mathbb{R}^n$ is the composition of an orthogonal linear operator and a translation. Specifically, if $f(0) = a$, then $f = t_a\varphi$ where $t_a$ is a translation and $\varphi$ is an orthogonal linear operator.

	\begin{myIndent}\hThree
		Proof:\\
		Let $f$ be an isometry, let $a = f(0)$, and define $\varphi = t_{-a}f$. Then clearly $t_a\varphi = f$. So, we just need to show that $\varphi$ is an orthogonal linear operator. To prove this, first note that $\varphi$ is the composition of two isometries, and is thus an isometry itself. Also, $\varphi(0) = -a + f(0) = -a + a = 0$. So applying theorem 6.2.3, we know that $\varphi$ is an orthogonal linear operator.\retTwo
	\end{myIndent}
\end{myIndent}

Now we've proven in other classes that both translations and linear orthogonal\\ operators on $\mathbb{R}^n$ are surjective. So, all isometries are the composition of surjections, meaning they are surjective themselves. And since we also previously proved that all isometries are injective, we know they are bijective and have inverses.\retTwo

Aside over\dots

\mySepTwo
\newpage

\exOne Our textbook is \textit{Algebra, Second Edition} by Michael Artin.



\end{document}