\documentclass{book}

\usepackage{fontspec} % used to import Calibri
\usepackage{anyfontsize} % used to adjust font size

% needed for inch and other length measurements
% to be recognized
\usepackage{calc}

% for colors and text effects as is hopefully obvious
\usepackage[dvipsnames]{xcolor}
\usepackage{soul}

% control over margins
\usepackage[margin=1in]{geometry}
\usepackage[strict]{changepage}

\usepackage{mathtools}
\usepackage{amsfonts}
\usepackage{bm}

\usepackage[scr=rsfso, scrscaled=.96]{mathalpha}

\usepackage{amssymb} % originally imported to get the proof square
\usepackage{xfrac}
\usepackage[overcommands]{overarrows} % Get my preferred vector arrows...
\usepackage{relsize}

% Just am using this to get a dashed line in a table...
% Also you apparently want this to be inactive if you aren't
% using it because it slows compilation.
\usepackage{arydshln} \ADLinactivate 
\newenvironment{allowTableDashes}{\ADLactivate}{\ADLinactivate}

\usepackage{graphicx}
\graphicspath{{./158_Images/}}

\usepackage{tikz}
   \usetikzlibrary{arrows.meta}
   \usetikzlibrary{graphs, graphs.standard}

\usepackage{quiver} %commutative diagrams


\newfontfamily{\calibri}{Calibri}
\setlength{\parindent}{0pt}
\definecolor{RawerSienna}{HTML}{945D27}

% ~~~~~~~~~~~~~~~~~~~~~~~~~~~~~~~~~~~~~~~~~~~~~~~~~~
%Arrow Commands:

% Thank you Bernard, gernot, and Sigur who I copied this from:
% https://tex.stackexchange.com/questions/364096/command-for-longhookrightarrow
\newcommand{\hooklongrightarrow}{\lhook\joinrel\longrightarrow}
\newcommand{\hooklongleftarrow}{\longleftarrow\joinrel\rhook}
\newcommand{\hookxlongrightarrow}[2][]{\lhook\joinrel\xrightarrow[#1]{#2}}
\newcommand{\hookxlongleftarrow}[2][]{\xleftarrow[#1]{#2}\joinrel\rhook}

% Thank you egreg who I copied from:
% https://tex.stackexchange.com/questions/260554/two-headed-version-of-xrightarrow
\newcommand{\longrightarrowdbl}{\longrightarrow\mathrel{\mkern-14mu}\rightarrow}
\newcommand{\longleftarrowdbl}{\leftarrow\mathrel{\mkern-14mu}\longleftarrow}

\newcommand{\xrightarrowdbl}[2][]{%
  \xrightarrow[#1]{#2}\mathrel{\mkern-14mu}\rightarrow
}
\newcommand{\xleftarrowdbl}[2][]{%
  \leftarrow\mathrel{\mkern-14mu}\xleftarrow[#1]{#2}
}

\newcommand{\MRoman}[1]{%
   \textrm{\MakeUppercase{\romannumeral #1}}%
}

% ~~~~~~~~~~~~~~~~~~~~~~~~~~~~~~~~~~~~~~~~~~~~~~~~~~

\newcommand{\learnToSpot}[1]{{\color{Red}#1}}

\newcommand{\hOne}{%
   \color{Black}%
   \fontsize{14}{16}\selectfont%
}
\newcommand{\hTwo}{%
\color{MidnightBlue}%
   \fontsize{13}{15}\selectfont%
}
\newcommand{\hThree}{%
   \color{PineGreen!85!Orange}
   \fontsize{12}{14}\selectfont%
}
\newcommand{\myComment}{%
   \color{RawerSienna}%
   \fontsize{12}{14}\selectfont%
}
\newcommand{\teachComment}{
   \color{Orange}%
   \fontsize{12}{14}\selectfont%
}
\newcommand{\exOne}{%
   \color{Purple}%
   \fontsize{13}{15}\selectfont%
}
\newcommand{\exTwo}{%
   \color{Purple}%
   \fontsize{13}{15}\selectfont%
}
\newcommand{\exP}{%
   \color{Purple}%
   \fontsize{12}{14}\selectfont%
}
\newcommand{\exTwoP}{%
   \color{RedViolet}%
   \fontsize{13}{15}\selectfont%
}
\newcommand{\exPP}{%
   \color{RedViolet}%
   \fontsize{12}{14}\selectfont%
}
% ~~~~~~~~~~~~~~~~~~~~~~~~~~~~~~~~~~~~~~~~~~~~~~~~

\newcommand{\cyPen}[1]{{\vphantom{.}\color{Cerulean}#1}}
\newcommand{\redPen}[1]{{\vphantom{.}\color{Red}#1}}

\newenvironment{myIndent}{%
   \begin{adjustwidth}{2.5em}{0em}%
}{%
   \end{adjustwidth}%
}

\newenvironment{myDindent}{%
   \begin{adjustwidth}{5em}{0em}%
}{%
   \end{adjustwidth}%
}

\newenvironment{myTindent}{%
   \begin{adjustwidth}{7.5em}{0em}%
}{%
   \end{adjustwidth}%
}

\newenvironment{myConstrict}{%
   \begin{adjustwidth}{2.5em}{2.5em}%
}{%
   \end{adjustwidth}%
}

\newcommand{\udefine}[1]{{%
   \setulcolor{Red}%
   \setul{0.14em}{0.07em}%
   \ul{#1}%
}}

\newcommand{\blab}[1]{\textbf{#1}}

\newcommand{\uuline}[2][.]{%
{\vphantom{a}\color{#1}%
\rlap{\rule[-0.18em]{\widthof{#2}}{0.06em}}%
\rlap{\rule[-0.32em]{\widthof{#2}}{0.06em}}}%
#2}

\newcommand{\pprime}{{\prime\prime}}
\newcommand{\suchthat}{ \hspace{0.3em}s.t.\hspace{0.3em}}
\newcommand{\rea}[1]{\mathrm{Re}(#1)}
\newcommand{\ima}[1]{\mathrm{Im}(#1)}
\newcommand{\comp}{\mathsf{C}}
\newcommand{\card}{\mathrm{card}}
\newcommand{\diam}{\mathrm{diam}}
\newcommand{\Sym}{\mathrm{Sym}}
\newcommand{\myHS}{ \hspace{0.5em}}

\newcommand{\myId}{\mathrm{Id}}
\newcommand{\myObj}{\mathrm{Obj}}
\newcommand{\myHom}{\mathrm{Hom}}
\newcommand{\myEnd}{\mathrm{End}}
\newcommand{\myAut}{\mathrm{Aut}}

\newcommand{\mcateg}[1]{{\bm{\mathsf{#1}}}}

\newcommand{\divides}{\mathop{\mid}}


\DeclareMathOperator{\lcm}{lcm}
\DeclareMathOperator{\ord}{ord}
\DeclareMathOperator{\im}{im}
\DeclareMathOperator{\sgn}{sgn}

% Thank you Gonzalo Medina and Moriambar who wrote this on stack exchange:
%https://tex.stackexchange.com/questions/74125/how-do-i-put-text-over-symbols%
\newcommand{\myequiv}[1]{\stackrel{\mathclap{\mbox{\footnotesize{$#1$}}}}{\equiv}}

% Thank you chs who wrote this on stack exchange:
%https://tex.stackexchange.com/questions/89821/how-to-draw-a-solid-colored-circle%
\newcommand{\filledcirc}[1][.]{\ensuremath{\hspace{0.05em}{\color{#1}\bullet}\mathllap{\circ}\hspace{0.05em}}}

%Thank you blerbl who wrote this on stack exchange:
%https://tex.stackexchange.com/questions/25348/latex-symbol-for-does-not-divide
\newcommand{\ndiv}{\hspace{-0.3em}\not|\hspace{0.35em}}

\newcommand{\mySepOne}[1][.]{%
   {\noindent\color{#1}{\rule{6.5in}{1mm}}}\\%
}
\newcommand{\mySepTwo}[1][.]{%
   {\noindent\color{#1}{\rule{6.5in}{0.5mm}}}\\%
}

\newenvironment{myClosureOne}[2][.]{%
   \color{#1}%
   \begin{tabular}{|p{#2in}|} \hline \\%
}{%
   \\ \hline \end{tabular}%
}

\newcommand{\retTwo}{\hfill\bigbreak}

\newcommand{\mHeader}[1]{{
   \color{Black}%
   \fontsize{20}{18}\selectfont%
   #1\retTwo
}}


\title{Math 100A Notes (Professor: Aaron Pollack)}
\author{Isabelle Mills}

\begin{document}
\maketitle{}
\setul{0.14em}{0.07em}
\calibri

\hOne
\mHeader{Lecture 1 Notes: 9/27/2024}

\blab{Motivation for this class:}

Let $\mathcal{F}$ be any figure in $\mathbb{R}^2$. We want some way of talking about the symmetries of $\mathcal{F}$.\retTwo

Letting $d$ be the standard metric for $\mathbb{R}^2$, we say $f: \mathbb{R}^2 \longrightarrow \mathbb{R}^2$ is \udefine{distance preserving}\\ if $d(P, Q) = d(f(P), f(Q))$ for all $P, Q \in \mathbb{R}^2$. If $f$ is distance-preserving and\\ $f(\mathcal{F}) = \mathcal{F}$, then we call $f$ a \udefine{symmetry} of $\mathcal{F}$.\retTwo

We define $\Sym(\mathcal{F})$ to be the set of symmetries of $\mathcal{F}$.
\begin{myIndent}\hTwo
	Lemma 2: The set $\Sym(\mathcal{F})$ has the following properties:
	
	\begin{enumerate}
		\item The identity map $\myId$ is in $\Sym(\mathcal{F})$
		\item If $f \in \Sym(\mathcal{F})$, then $f^{-1} \in \Sym(\mathcal{F})$.
		\begin{myIndent}\hThree
			I realize we haven't yet shown that every $f \in \Sym(\mathcal{F})$ is a bijection. Given such an $f$, it's easy to see that $f$ must be injective. After all, the distance\\ preserving property of $f$ means that $f(P) = f(Q) \Longrightarrow P = Q$. Showing that $f$ is surjective is harder. By assumption, we know that $f$ is surjective when restricted to $\mathcal{F}$. More complicatedly, we can show that $f$ must have a certain form which happens to be surjective. Perhaps I'll prove that later.\retTwo

			Once, you've accepted that $f^{-1}$ exists, then it's clearly true that $f^{-1}$ is also distance preserving with $f^{-1}(\mathcal{F}) = \mathcal{F}$.
		\end{myIndent}
		\item If $f_1, f_2 \in \Sym(\mathcal{F})$, then $f_1 \circ f_2 \in \Sym(\mathcal{F})$ and $f_2 \circ f_1 \in \Sym(\mathcal{F})$.
		\begin{myIndent}\hThree
			This is pretty trivial to show.\retTwo
		\end{myIndent}
	\end{enumerate}
\end{myIndent}

Now while it's all good that we have a concrete way of describing the symmetries of a figure, our current terminology is not the most useful. After all, suppose $\mathcal{S}$ and $\mathcal{S}^\prime$ are two squares such that $\mathcal{S}$ is centered at the origin and $\mathcal{S}^\prime$ is centered at the point $(5, 5)$. Then even though we know both $\mathcal{S}$ and $\mathcal{S}^\prime$ have symmetries in the form of rotating and reflecting, the particular functions in $\Sym(\mathcal{S})$ and $\Sym(\mathcal{S})$ will be different (except for $\myId$). So, how do we compare the symmetries of those two squares?\retTwo

\mySepTwo

Aside start\dots\retTwo

\blab{Proof that all symmetries are surjective (taken from our textbook)}:\\
\begin{myIndent}\myComment
	Note:\\ [-18pt]
	\begin{itemize}
		\item Our textbook calls a distance-preserving function $f: \mathbb{R}^n \longrightarrow \mathbb{R}^n$ an \udefine{isometry}.
		\item Rather than writing $f_1 \circ f_2$ to represent function composition, our textbook just\\ writes $f_1f_2$.\newpage
	\end{itemize}

	\hTwo
	\blab{Some Facts:}
	\begin{itemize}
		\item[(a)] Orthogonal linear operators are isometries.
		
		\begin{myIndent}\hThree
			Let $\varphi$ be n orthogonal linear map. $\varphi$ being linear means that\\ $\varphi(u) - \varphi(v) = \varphi(u - v)$. Meanwhile, $\varphi$ being orthogonal means that\\ $|\varphi(u - v)| = \sqrt{\varphi(u - v) \cdot \varphi(u -v)} = \sqrt{(u - v) \cdot (u - v)} = |u - v|$.\\ So, for any $u, v \in \mathbb{R}^n$, we have that $|\varphi(u) - \varphi(v)| = |u - v|$.\retTwo
		\end{myIndent}

		\item[(b)] The translation $t_a$ by a vector $a$ defined by $t_a(x) = x + a$ is an isometry.
		\begin{myIndent}\hThree
			For any $u, v \in \mathbb{R}^n$, we have $|t_a(u) - t_a(v)| = |u + a - v - a| = |u - v|$.\retTwo
		\end{myIndent}

		\item[(c)] The composition of isometries is an isometry.
		\begin{myIndent}\hThree
			If $f_1, f_2$ are isometries, then for all $u, v \in \mathbb{R}^n$, we have that\\ $|f_1(f_2(u)) - f_1(f_2(v))| = |f_2(u) - f_2(v)| = |u - v|$.\retTwo
		\end{myIndent}
	\end{itemize}

	\blab{Theorem 6.2.3:} The following conditions on a map $\varphi: \mathbb{R}^n \longrightarrow \mathbb{R}^n$ are equivalent:
	\begin{itemize}
		\item[(a)] $\varphi$ is an isometry such that $\varphi(0) = 0$.
		\item[(b)] $\varphi$ preserves dot products: $\varphi(u) \cdot \varphi(w) = u \cdot w$ for all $u, w \in \mathbb{R}^n$.
		\item[(c)] $\varphi$ is an orthogonal linear operator.
		
		\begin{myIndent}\hThree
			Proof:\\
			(c) $\Longrightarrow$ (a)\\
			This comes both from the first fact on this page plus the fact that all linear\\ operators map $0$ to $0$.\retTwo

			(b) $\Longrightarrow$ (c)\\
			Our challenge here is to show that such a $\varphi$ has to be linear operator.\retTwo

			\blab{Lemma:} For $x, y \in \mathbb{R}^n$, if $(x \cdot x) = (x \cdot y) = (y \cdot y)$, then $x = y$.
			
			\begin{myIndent}\hThree
				Proof: $|x - y|^2 = (x - y) \cdot (x - y) = (x \cdot x) - 2(x \cdot y) + (y \cdot y)$.\retTwo
			\end{myIndent}

			Consider any $u, v \in \mathbb{R}^n$ and set $w = u + v$. Then set $u^\prime = \varphi(u)$,\\ $v^\prime = \varphi(v)$, and $w^\prime = \varphi(w)$. To show that $w^\prime = v^\prime + u^\prime$, we shall show\\ that $(w^\prime \cdot w^\prime) = (w^\prime \cdot (u^\prime + v^\prime)) = ((u^\prime + v^\prime) \cdot (u^\prime + v^\prime))$.\retTwo

			Firstly, simplify our equation to:

			{\centering $(w^\prime \cdot w^\prime) = (w^\prime \cdot u^\prime) + (w^\prime \cdot v^\prime) = (u^\prime \cdot u^\prime) + 2(u^\prime \cdot v^\prime) + (v^\prime \cdot v^\prime)$ \retTwo\par}

			Next, since $\varphi$ is assumed to preserve dot products, we can thus simplify our\\ equation to:

			{\centering $(w \cdot w) = (w \cdot u) + (w \cdot v) = (u \cdot u) + 2(u\cdot v) + (v \cdot v)$ \retTwo\par}

			And since $w = u + b$, all of those equalities are true. Hence, we know by our lemma above that $w^\prime = u^\prime + v^\prime$.\newpage

			Meanwhile, let $v \in \mathbb{R}^n$ and set $u = cv$ where $c$ is a constant. Then define $u^\prime$ and $v^\prime$ as before. Then we can do a few trivial simplications to show that $(u^\prime \cdot u^\prime)$, $(u^\prime \cdot cv^\prime)$ and $(cv^\prime \cdot cv^\prime)$ are all equal to $c^2(v \cdot v)$. So, $u^\prime = cv^\prime$.\retTwo

			(a) $\Longrightarrow$ (b)\\
			Since $\varphi$ is distance preserving, we know that $\forall u, v \in \mathbb{R}^n$, 

			{\centering$(\varphi(u) - \varphi(v)) \cdot (\varphi(u) - \varphi(v)) = (u - v) \cdot (u -v)|$.\retTwo\par}

			By plugging in $v = 0$, this simplifies to $(\varphi(u) \cdot \varphi(u)) = (u \cdot u)$. Similarly, by plugging in $u = 0$, we can get that $(\varphi(v) \cdot \varphi(v)) = (v \cdot v)$. So, by expanding and canceling out parts of our above expression, we get that:

			{\centering$-2(\varphi(u) \cdot \varphi(v)) = - 2(u \cdot v)$.\retTwo\par}
		\end{myIndent}
	\end{itemize}

	\blab{Corollary 6.2.7:} Every isometry $f$ of $\mathbb{R}^n$ is the composition of an orthogonal linear operator and a translation. Specifically, if $f(0) = a$, then $f = t_a\varphi$ where $t_a$ is a translation and $\varphi$ is an orthogonal linear operator.

	\begin{myIndent}\hThree
		Proof:\\
		Let $f$ be an isometry, let $a = f(0)$, and define $\varphi = t_{-a}f$. Then clearly $t_a\varphi = f$. So, we just need to show that $\varphi$ is an orthogonal linear operator. To prove this, first note that $\varphi$ is the composition of two isometries, and is thus an isometry itself. Also, $\varphi(0) = -a + f(0) = -a + a = 0$. So applying theorem 6.2.3, we know that $\varphi$ is an orthogonal linear operator.\retTwo
	\end{myIndent}
\end{myIndent}

Now we've proven in other classes that both translations and linear orthogonal\\ operators on $\mathbb{R}^n$ are surjective. So, all isometries are the composition of surjections, meaning they are surjective themselves. And since we also previously proved that all isometries are injective, we know they are bijective and have inverses.\retTwo

Aside over\dots

\mySepTwo

\mHeader{Lecture 2 Notes: 9/30/2024}

I already covered everything from this lecture in my math journal (pages 40-42).\\

\mySepTwo

\mHeader{Lecture 3 Notes: 10/2/2024}

Suppose $G_1$ and $G_2$ are groups. A map $\rho: G_1 \longrightarrow G_2$ is called a \udefine{group\\ homomorphism} if $\rho(xy) = \rho(x)\rho(y)$ for all $x, y \in G_1$. If $\rho$ is bijective, we say that\\ $\rho$ is an \udefine{isomorphism}, and that $G_1$ and $G_2$ are \udefine{isormophic.} Also if $\rho$ is bijective, we\\ have that $\rho^{-1}$ is also a group homomorphism.\newpage

If two groups are isomorphic, then we can say they are in a sense equivalent.\retTwo

Suppose $G$ is a group and $H \subseteq G$. Then $H$ equipped with the law of composition of $G$ restricted to $H \times H$ is a \udefine{subgroup} if:\\ [-20pt]
\begin{itemize}
	\item $1 \in H$\\ [-20pt]
	\item $x \in H \Longrightarrow x^{-1} \in H$\\ [-20pt]
	\item $x, y \in H \Longrightarrow xy \in H$\retTwo
\end{itemize}

\exOne

Example: If $\mathbb{R}^\times = (\mathbb{R} - \{0\}, \times)$, then some non-trivial subgroups of $\mathbb{R}^x$ are:\\ [-20pt]

\begin{itemize}
	\item $M_2 = \{1, -1\}$\\ [-20pt]
	\item $\mathbb{Z}^x = \mathbb{Z} - \{0\}$\\ [-20pt]
	\item $\mathbb{Q}^x = \mathbb{Q} - \{0\}$\\ [-20pt]
	\item $H = \{a^n \in \mathbb{R} \mid n \in \mathbb{Z}\}$.
\end{itemize}

\begin{myIndent}\hTwo
	\blab{Theorem:} Let $S$ be a subgroup of $(\mathbb{Z}, +)$ (the set of integers equipped with integer addition). Then either $S = \{0\}$ or $S = \mathbb{Z}a = \{na \mid n \in \mathbb{Z}\}$ where $a$ is the least positive element of $S$.

	
	\begin{myIndent}\hThree
		Proof:\\
		We clearly have that $\{0\}$ and $\mathbb{Z}a$ are groups under addition for any $a \in \mathbb{Z}_+$.\\ Meanwhile, suppose $S \neq \{0\}$ is a subgroup of $(\mathbb{Z}, +)$. Then, by taking inverses if necessary, we know $S \cap \mathbb{Z}_+$ is nonempty. Since $\mathbb{Z}_+$ is well-ordered, there exists a least element in $S \cap \mathbb{Z}_+$ which we'll call $a$.\retTwo

		Trivially, we have that $\mathbb{Z}a \subseteq S$. Meanwhile consider any $n \in S$. Then $n = qa + r$ for some $q \in \mathbb{Z}$ and $r \in \{0, 1, \ldots, a - 1\}$. However, since $r = n - qa$ and $n, -qa \in S$, we must have that $r \in S$. And, the only allowed value for $r$ such that $r \in S$ is $r = 0$. Thus, $n \in \mathbb{Z}a$, meaning we've shown that $S \subseteq \mathbb{Z}_a$.\retTwo
	\end{myIndent}
\end{myIndent}

\hOne\mySepTwo

\mHeader{Lecture 4 Notes: 10/4/2024}

As an immediate application of the above theorem, note that $S = \mathbb{Z}a + \mathbb{Z}b = \{ma + nb \mid m, n \in \mathbb{Z}\}$ is subgroup of $\mathbb{Z}$ under addition.
\begin{myTindent}\myComment
	This is trivial to prove.\retTwo
\end{myTindent}

By our previous theorem, we know that $S = \mathbb{Z}d$ for some unique positive integer $d$. So, we define the \udefine{greatest common divisor} of $a$ and $b$ to be $\gcd(a, b) \coloneq d$.

\begin{myIndent}\hTwo
	\blab{Proposition:} Let $a, b \in \mathbb{Z}$ be not both $0$ and $d = \gcd(a, b)$.
	\begin{enumerate}
		\item There exists $r, s \in \mathbb{Z}$ such that $d = ra + sb$
		\item $d$ divides $a$ and $b$ (written $d \divides a$ and $d \divides b$).
		\begin{myIndent}\hThree
			Both of these claims are trivially true by our definition of $S$.\newpage
		\end{myIndent}
		\item If $e \in \mathbb{Z}$ and $e$ divides $a$ and $b$, then $e$ divides $d$. This is why $d$ is called the\\ "greatest common divisor" of $a$ and $b$.
		\begin{myIndent}\hThree
			Let $r, s \in \mathbb{Z}$ such that $d = ra + sb$. Then letting $a = en$ and $b = em$, we have that $d = (rn + sm)e$, meaning $e \divides d$.\retTwo
		\end{myIndent}
	\end{enumerate}
\end{myIndent}

An algorithm for finding $\gcd(a, b)$ is given as follows:
\begin{enumerate}
	\item Assume without loss of generality that $a \geq b \geq 0$ and $a \neq 0$.
	\item If $b = 0$, then $\gcd(a, b) = \gcd(b, a) = a$
	\item Else, there exists $q, r \in \mathbb{Z}$ with $0 \leq r < b$ and $a = qb + r$. We claim\\ that $\gcd(a, b) = \gcd(b, r)$.
	
	\begin{myIndent}\hTwo
		This is because if $d \divides a$ and $d \divides b$, then we know $d \divides (qb + r)$ and $d \divides qb$,\\ meaning that $d \divides (qb + r - qb) = r$. On the other hand, if $e \divides r$ and $e \divides b$,\\ then $e \divides (qb + r) = a$. So $a$ and $b$ have the same common factors as $b$ and $c$.\retTwo
	\end{myIndent}
\end{enumerate}

Suppose $a, b \in \mathbb{Z}$. We say $a$ and $b$ are \udefine{relatively prime} iff $\gcd(a, b) = 1$.

\begin{myIndent}\hTwo
	\blab{Corollary:} $\gcd(a, b) = 1$ if and only if there exists $r, s \in \mathbb{Z}$ such that $ra + sb = 1$.
	\begin{myIndent}\hThree
		Proof:\\
		($\Longrightarrow$) By definition, $\gcd(a, b) \in \mathbb{Z}a + \mathbb{Z}b$.\\
		($\Longleftarrow$) If $ra + sb = 1$, then $1$ must be the least positive element of $\mathbb{Z}a + \mathbb{Z}b$.\\ So $\gcd(a, b) = 1$.\retTwo
	\end{myIndent}

	\blab{Lemma:} Suppose $\gcd(a, b) = 1$ and $a \divides bc$. Then $a \divides c$.
	\begin{myIndent}\hThree
		Proof:\\
		Let $1 = ra + sb$ where $r, s \in \mathbb{Z}$. Then $c = rac + sbc = (rc + s\frac{bc}{a})a$ where $\frac{bc}{a}$\\ [-2pt] is an integer. So $a \divides c$.\retTwo
	\end{myIndent}

	\blab{Corollary:} Suppose $p$ is a prime integer. If $a, b \in \mathbb{Z}$ and $p \divides ab$, then either $p \divides a$\\ or $p \divides b$.
	\begin{myIndent}\hThree
		Proof:\\
		Suppose $p {\not\divides a}$. Then $\gcd(p, a) = 1$ because the only positive divisor of $p$ other\\ than $p$ is $1$. So there exists $r, s \in \mathbb{Z}$ such that $1 = rp + sa$. In turn, since $\frac{ab}{p}$ is an\\ [-3pt] integer, we have $b = rpb + sab = p(rb +s\frac{ab}{p})$, meaning $p \divides b$.\retTwo
	\end{myIndent}
\end{myIndent}

\exOne\mySepTwo

\blab{Problem:} Suppose $p$ is prime and that $a \in \mathbb{Z}$ is not a multiple of $p$. Then there exists $x \in \mathbb{Z}$ so that $ax$ is one more than some multiple of $p$.

\begin{myIndent}\exTwoP
	Proof:\\
	Like before, we must have that $\gcd(a, p) = 1$, meaning that there exists $r, s \in \mathbb{Z}$\\ such that $rp + sa = 1$. So, if we set $x = s$, we'd be done cause $xa = (-r)p + 1$.\retTwo

	More interestingly, we can guarentee that $xa$ is one more than a nonnegative multiple of $p$ as follows:\newpage
	\begin{myIndent}
		Note that $sa = -rp + 1 \Longrightarrow (s^2a)a = (r^2p - 2r)p + 1 = r(rp - 2)p + 1$.\\ Since $p \geq 2$, we have that $r \geq 1 \Longrightarrow (rp - 2) > 0$, meaning $r(rp - 2) > 0$.\\ Meanwhile, we have that $r \leq 0 \Longrightarrow (rp - 2) < 0$, which in turn means $r(rp - 2) \geq 0$.\retTwo

		Setting $x = s^2a$ and $n = r^2p - 2r$, we thus have that $xa = np + 1$ where\\ $np$ is a nonnegative multiple of $p$.
	\end{myIndent}
\end{myIndent}

\mySepTwo

\hTwo
\blab{Lemma:} Suppose $G$ is a group and $\{H_\alpha\}_{\alpha \in A}$ are subgroups of $G$. Then $\hspace{-0.3em}\bigcap\limits_{\alpha \in A}\hspace{-0.3em}H_\alpha$ is a\\ [-10pt] subgroup of $G$.
\begin{myIndent}\hThree
	This is rather trivial to prove. So do it yourself! :3\retTwo
\end{myIndent}

\hOne

Because of the above lemma, given $a, b \in \mathbb{Z}$, we have that $\mathbb{Z}a \cap \mathbb{Z}b = \mathbb{Z}m$ for some integer $m \geq 0$. We call $m$ the \udefine{least common multiple} of $a$ and $b$, and we denote $\lcm(a, b) \coloneq m$.

\begin{myIndent}\hTwo
	\blab{Proposition:} Let $a$ and $b$ be nonzero integers and $m = \lcm(a, b)$.
	\begin{enumerate}
		\item $m$ is nonzero.
		\item $m$ is divisible by both $a$ and $b$
		\begin{myIndent}\hThree
			Both of these points are trivial from the fact that $\mathbb{Z}a \cap \mathbb{Z}b = \mathbb{Z}m$ and\\ $ab \in \mathbb{Z}m$, meaning that $\mathbb{Z}m - \{0\} \neq \emptyset$.
		\end{myIndent}
		\item If $n \in \mathbb{Z}$ such that $a \divides n$ and $b \divides n$, then $m \divides n$.
		\begin{myIndent}\hThree
			This comes trivially from the fact that $n \in \mathbb{Z}a$ and $n \in \mathbb{Z}b$ means that\\ $n \in \mathbb{Z}a \cap \mathbb{Z}b = \mathbb{Z}m$\retTwo
		\end{myIndent}
	\end{enumerate}
\end{myIndent}


Suppose $G$ is a group and $x \in G$. Then let $H = \{x^k \mid k \in \mathbb{Z}\} \subseteq G$. We clearly have that $H$ is a subgroup of $G$. We call it the \udefine{cyclic subgroup} of $G$ generated by $x$, and denote it $H = \langle x \rangle$.

\begin{myIndent}\hTwo
	\blab{Proposition:} Let $S = \{k \in \mathbb{Z} \mid x^k = 1\}$
	\begin{enumerate}
		\item $S$ is a subgroup of $(\mathbb{Z}, +)$.
		\begin{myIndent}\hThree
			This is rather trivial to show. So do it yourself!!
		\end{myIndent}
		\item Suppose $S \neq \{0\}$, meaning $S = \mathbb{Z}n$ for some positive integer $n$. Then\\ $1, x, \ldots, x^{n-1}$ are the distinct elements of $\langle x \rangle$, meaning the order of $\langle x \rangle$ is $n$.
		
		\begin{myIndent}\hThree
			Proof:\\
			$x^{k} = x^{l} \Longleftrightarrow x^{k-l} = 1$. Hence, since $n$ is the minimum positive integer such that $x^n = 1$, we know that $1, x, \ldots, x^{n-1}$ are distinct. On the other hand, if $k = qn + r$ for any $q, r \in \mathbb{Z}$ with $0 \leq r < n$, then $x^k = (x^n)^qx^r = x^r$. So the only elements of $\langle x \rangle$ are $1, x, \ldots, x^{n-1}$.\retTwo
		\end{myIndent}
	\end{enumerate}
\end{myIndent}

\newpage

\begin{myIndent}\hTwo
	\blab{Corollary}: If $S = \{k \in \mathbb{Z} \mid x^k = 1\} = \{0\}$, then $x^{k} = x^{l} \Longrightarrow k-l = 0 \Longrightarrow k = l$.\retTwo 
\end{myIndent}

\mHeader{Lecture 5 Notes: 10/7/2024}

If $G$ is a group and $x \in G$, one says $x$ has order $n$ if $n$ is the smallest positive integer for which $x^n = 1$. If there is no such integer, then we say $x$ has infinite order.\retTwo


\begin{myIndent}\hTwo
	\blab{Lemma:} Suppose that $G$ is a group, that $x \in G$ has order $n$, and that $\gcd(k , n) = d$. Then $x^k$ has order $\sfrac{n}{d}$.

	\begin{myIndent}\hThree
		Proof:\\
		Let $r = \ord(x^k)$. Then $x^{kr} = 1$, meaning $n \divides kr$. Since $d$ divides both $n$ and $k$, we thus have that $\frac{n}{d} \divides \frac{k}{d} r$. But $\gcd(\frac{n}{d}, \frac{k}{d}) = 1$ since $\gcd(n , k) = d$. So, we must have that $\frac{n}{d} \divides r$. Conversely, $(x^k)^{\sfrac{n}{d}} = (x^n)^{\frac{k}{d}} = 1$. So $r \divides \frac{n}{d}$. This means that $r = \frac{n}{d}$\retTwo
	\end{myIndent}
\end{myIndent}

If $G$ is a group and $U \subseteq G$, one can form the subgroup $H = \langle U\rangle$ of $G$ generated by $U$, meaning that $H$ is the intersection of all subgroups of $G$ containing $U$.\retTwo

\exOne\blab{Some Example Groups:}

\begin{itemize}
	\item The \udefine{Klein-4 Group} consists of the matrices with the form: $
	\left[\begin{smallmatrix}
		\pm 1 & 0 \\ 0 & \pm 1
	\end{smallmatrix}\right]$ or $
	\left[\begin{smallmatrix}
		\pm 1 & 0 \\ 0 & \mp 1
	\end{smallmatrix}\right]$.\\ [-16pt]

	It has four elements and is not cyclic.\retTwo

	\item The \udefine{Quaternion Group} consists of the 8 elements in $\mathrm{GL}_2(\mathbb{C})$: $\pm \bm{1} = \left[\begin{smallmatrix}
		 1 & 0 \\ 0 & 1
	\end{smallmatrix}\right]$,\\ $\pm \bm{I} = 
	\left[\begin{smallmatrix}
		i & 0 \\ 0 & -i
	\end{smallmatrix}\right]$, $\pm \bm{J} = 
	\left[\begin{smallmatrix}
		0 & 1 \\ -1 & 0
	\end{smallmatrix}\right]$, and $\pm \bm{K} = 
	\left[\begin{smallmatrix}
		 i & 0 \\ 0 & i
	\end{smallmatrix}\right]$.
\end{itemize}

\hOne

\mySepTwo

\begin{myIndent}\hTwo
	\blab{Proposition}: Suppose $\varphi: G \longrightarrow G^\prime$ is a group homomorphism. Then:
	\begin{enumerate}
		\item If $a_1, \cdots, a_k \in G_1$, then $\varphi(a_1\cdots a_k)=\varphi(a_1)\cdots\varphi(a_2)$.
		\item $\varphi(1_G) = 1_{G^\prime}$
		\item $\varphi(a^{-1}) = \varphi(a)^{-1}$
		
		\begin{myIndent}\hThree
			Proof:\\
			(1) This is true by induction. For example:
			
			{\centering $\varphi(a_1a_2a_3) = \varphi(a_1a_2)\varphi(a_3) = \varphi(a_1)\varphi(a_2)\varphi(a_3)$.\retTwo\par}

			(2) $\varphi(1_G) = \varphi(1_{G}1_{G}) = \varphi(1_G)\varphi(1_G)$. By multiplying $\varphi(1_G)^{-1}$ to both\\ sides, we  get that $\varphi(1_G) = 1_{G^\prime}$.\retTwo

			(3) $1_{G^\prime} = \varphi(1_G) = \varphi(aa^{-1}) = \varphi(a)\varphi(a^{-1})$. By multiplying $\varphi(a)^{-1}$ to both\\ sides, we get that $\varphi(a)^{-1} = \varphi(a^{-1})$.\newpage
		\end{myIndent}
	\end{enumerate}
\end{myIndent}

Suppose $\varphi: G \longrightarrow G^\prime$ is a group homomorphism. 
\begin{itemize}
	\item The \udefine{image} of $\varphi$ is: $\im(\varphi) = \varphi(G) = \{x \in G^\prime \mid x = \varphi(a) \text{ for some } a \in G\}$.
	\item The \udefine{kernel} of $\varphi$ is $\ker(\varphi) = \{x \in G \mid \varphi(x) = 1_{G^\prime}\}$.\retTwo
\end{itemize}


\begin{myIndent}\hTwo
	\blab{Proposition:} Let $\varphi: G \longrightarrow G^\prime$ be a group homomorphism. Then $\ker(\varphi) \subseteq G$ is a subgroup and $\im(\varphi)$ is a subgroup.
	
	\begin{myIndent}\hThree
		The kernel is a subgroup because if $\varphi(a) = 1_{G^\prime} = \varphi(b)$, then $\varphi(ab) = 1_{G^\prime}$. Also, if\\ $\varphi(a) = 1_{G^\prime}$, then $\varphi(a^{-1}) = \varphi(a)^{-1} = 1_{G^\prime}$. And finally, $\varphi(1_G) = 1_{G^\prime}$ as we showed earlier.\retTwo

		The image is subgroup because if $a^\prime, b^\prime \in \im(\varphi)$, then there exists $a, b \in G$ with $\varphi(a) = a^\prime$ and $\varphi(b) = b^\prime$. Then $\varphi(ab) = a^\prime b^\prime$, meaning $a^\prime b^\prime \in \im(\varphi)$. Also,\\ $\varphi(a^{-1}) = (a^\prime)^{-1}$, meaning $(a^\prime)^{-1} \in \im(\varphi)$. Finally, we know $1_{G^\prime} \in \im(\varphi)$\\ because $\varphi(1_G) = 1_{G^\prime}$.\retTwo
	\end{myIndent}

	\blab{Proposition:} If $\rho_1: G_1 \longrightarrow G_2$ and $\rho_2: G_2 \longrightarrow G_3$ are group homomorphisms,\\ then $\rho_2 \circ \rho_1: G_1 \longrightarrow G_3$ is a group homomorphism.
\end{myIndent}

\mySepTwo

\mHeader{Lecture 6 Notes: 10/9/2024}

Let $b_1, \ldots, b_n$ be the standard basis of $\mathbb{R}^n$. Given any $\sigma \in S_n$, define a linear map $\rho(\sigma)$ on $\mathbb{R}^n$ such that $\rho(\sigma)(b_i) = b_{\sigma(i)}$. Or equivalently:

{\centering $\rho(\sigma)(\alpha_1b_1 + \ldots + \alpha_nb_n) = \alpha_{\sigma^{-1}(1)}b(1) + \ldots + \alpha_{\sigma^{-1}(n)}b(n)$ \retTwo\par}

Then $\rho$ is a group homomorphism from $S_n$ to $GL_n(\mathbb{R})$.
\begin{myIndent}\hTwo
	The proof for this is hopefully obvious.\retTwo
\end{myIndent}

Noting that $\det: GL_n(\mathbb{R}) \longrightarrow \mathbb{R}^{\times}$ is a group homomorphism, given any $\sigma \in S_n$\\ we define the \udefine{sign} of the permutation: $\sgn(\sigma) = \det(\rho(\sigma))$. Note that by the\\ proposition at the end of the last lecture, we know $\sgn$ is a group homomorphism.

\begin{myIndent}\hTwo
	\blab{Claim:} $\im(\sgn) = \{1, -1\}$.
	
	\begin{myIndent}\hThree
		Proof:\\
		Because $S_n$ is finite, we know all $\sigma \in S_n$ have finite order. Thus, consider any\\ $\sigma \in S_n$ with order $k$. Then we have that:
		
		{\centering $\sigma^k = 1 \Longrightarrow \rho(\sigma^k) = \rho(\sigma)^k = \rho(1)$.\retTwo\par}
		
		In turn, $\det(\rho(\sigma)^k) = \det(\rho(\sigma))^k = \det(\rho(1))$. So $\sgn(\sigma)^k = 1$. But since\\ $\sgn(\sigma) \in \mathbb{R}$, we must have that $\sgn(\sigma) = \pm 1$.\retTwo
	\end{myIndent}
\end{myIndent}

The kernel of the determinant homomorphism: $\det: GL_n(\mathbb{R}) \longrightarrow \mathbb{R}^\times$ is called the \udefine{special linear group} $SL_n(\mathbb{R})$.\retTwo

The kernel of the sign homomorphism $\sgn: S_n \longrightarrow \{-1, 1\}$ is called the \udefine{alternating group}: $A_n$. Also, we call the elements of $A_n$ \udefine{even permutations}.\newpage

Suppose $H \subseteq G$ is a subgroup and $a \in G$. Then:\\ [-12pt]

{\centering$aH = \{g \in G \mid \exists h \in H \suchthat g = ah\}$,\\ [4pt]\par}

is called a \udefine{left coset} of $H$ in $G$. One can similarly define a \udefine{right coset} $Ha$.\retTwo


\begin{myIndent}\hTwo
	\blab{Proposition}: Suppose $\varphi: G \longrightarrow G^\prime$ is a group homomorphism, and let\\ $K = \ker(\varphi)$. Then the following statements are equivalent for all $a, b \in G$:
	\begin{itemize}
		\item[$1$.] $\varphi(a) = \varphi(b)$
		\item[$2$.] $a^{-1}b \in K$
		\item[$3$.] $b \in aK$
		\item[$4$.] $aK = bK$
	\end{itemize}

	\begin{myIndent}\hThree
		Proof:\\
		($1 \Longrightarrow 2$) If $\varphi(a) = \varphi(b)$, then:

		{\centering $\varphi(a^{-1}b) = \varphi(a^{-1})\varphi(b) = \varphi(a^{-1})\varphi(a) = \varphi(a^{-1}a) = 1$.\retTwo\par}
		
		So $a^{-1}b \in K$.\retTwo

		($2 \Longrightarrow 3$) If $a^{-1}b \in K$, then $b = a(a^{-1}b) \in aK$.\retTwo

		($3 \Longrightarrow 4$) Suppose $b = ak$ for some $k \in K$. Then firstly, note that for all $c \in aK$, if $h \in K$ satisfies $c = ah$, then $c = akk^{-1}h = b(k^{-1}h)$. This shows that $aK \subseteq bK$. As for the other inclusion, note that $b = ak \Longrightarrow a = bk^{-1}$. So $a \in bK$ and we can repeat the same reasoning as before.
		
		\begin{myTindent}\teachComment
			This is actually a special case of the first corollary below.\retTwo
		\end{myTindent}

		$(4 \Longrightarrow 1)$ If $aK = bK$, then we know there exists constants $k_1, k_2 \in K$ such that $ak_1 = bk_2$. In turn, $\varphi(a) = \varphi(ak_1) = \varphi(bk_2) = \varphi(b)$.\retTwo
	\end{myIndent}

	\blab{Lemma:} Suppose $H \subseteq G$ is a subgroup, $x \in G$, and $g \in xH$. Then $xH = gH$.

	\begin{myIndent}\hThree
		Proof:\\
		Let $g = xh^\prime$ where $h_1 \in H$. Then $gh = xh^\prime h \in xH$ for all $h \in H$. Hence,\\ $gH \subseteq xH$. Conversely $x = g(h^\prime)^{-1}$. So $x \in gH$ and we can do the same\\ reasoning as before to show that $xH \subseteq gH$.\retTwo
	\end{myIndent}

	\blab{Corollary:} Suppose $H \subseteq G$ is a subgroup and $x, y \in G$. If $xH \cap yH \neq \emptyset$, then $xH = yH$.

	\begin{myIndent}\hThree
		Proof:\\
		Suppose $xh_1 = g = yh_2$ with $h_1, h_2 \in H$. Then $xH = gH = yH$ by the previous lemma.\retTwo
	\end{myIndent}

	\blab{Corollary:} A group homomorphism $\varphi: G \longrightarrow G^\prime$ is injective if and only if its kernel is trivial (i.e. $\ker(\varphi) = \{1\})$.

	\begin{myIndent}\hThree
		Proof:\\
		The forward implication is trivial by the definition of injectivity. As for the reverse\\ implication, suppose $\ker(\varphi) = \{1\}$. Then:
		
		{\centering $\varphi(a) = \varphi(b) \Longrightarrow a^{-1}b \in \ker(\varphi) = \{1\} \Longrightarrow a^{-1}b = 1$.\retTwo\par} It follows that $a = b$.\newpage
	\end{myIndent}
\end{myIndent}

Suppose $G$ is a group and $a, g \in G$. Then $gag^{-1}$ is called the \udefine{conjugate} of $a$ by $g$.\retTwo

Suppose $G$ is a group and $N \subseteq G$ is a subgroup. The subgroup $N$ is \udefine{normal} if $gng^{-1} \in N$ for all $n \in N$ and $g \in G$.

\begin{myIndent}\hTwo
	\blab{Proposition:} Suppose $\varphi: G \longrightarrow G^\prime$ is a group homomorphism. Then $\ker(\varphi) \subseteq G$ is a normal subgroup.
	
	\begin{myIndent}\hThree
		Proof:\\
		Suppose $a \in \ker(\varphi)$ and $g \in G$. Then $gag^{-1} \in \ker(\varphi)$ because:

		{\center $\varphi(gag^{-1}) = \varphi(g)\varphi(a)\varphi(g)^{-1} = \varphi(g)\varphi(g)^{-1} = 1$ \retTwo\par}
	\end{myIndent}
\end{myIndent}

\mHeader{Lecture 7 Notes: 10/11/2024}

You already know what an \udefine{abelian group} is. Note that every subgroup of an abelian group is normal because $ga = ag \Longrightarrow gag^{-1} = a$\retTwo

Given a group $G$, define $Z(G) \coloneq \{z \in G \mid zx = xz \text{ for all } x \in G\}$, Then $Z(G)$ is a normal subgroup of $G$ called the \udefine{center} of $G$.

\begin{myIndent}\hTwo
	Proof that $Z(G)$ is a subgroup:
	\begin{myIndent}\hThree
		We know $1 \in Z(G)$.\\ 
		Also if $z \in Z(G)$, then for all $x \in G$ we have that:
		
		{\centering $zx = xz \Rightarrow z^{-1}zxz^{-1} = z^{-1}xzz^{-1} \Rightarrow xz^{-1} = z^{-1}x$.\retTwo\par}
	
		Finally if $y, z \in Z(G)$, then for all $x \in G$ we have that:
	
		{\centering $(zy)x = z(yx) = z(xy) = (zx)y = (xz)y = x(zy)$ \retTwo\par}
	\end{myIndent}
\end{myIndent}

Suppose $n \in \mathbb{Z}_+$. Then $\mu_n = \{z \in \mathbb{C}^\times \mid z^n = 1\}$ is a subgroup under complex\\ multiplication. ($\mu_n$ is called the \udefine{$n$th roots of unity}.)

\begin{myIndent}\hTwo
	Note that the elements of $\mu_n$ are all the numbers of the form $e^{\frac{2\pi ia }{n}}$ where $a \in \mathbb{Z}$.\retTwo

	Also, $\mu_n$ has $n$ elements and is cyclic (it is generated by $e^{\frac{2pi}{n}}$). This shows that for all $n \in \mathbb{Z}_+$ there is a cyclic group with $n$ elements.\retTwo
\end{myIndent}

\exOne

\mySepTwo

\blab{Examples of group isomorphisms:} (I'm skipping writing down most of these cause they're not interesting)\retTwo

Let $G$ be an arbitrary group and $g \in G$. Then define $\rho_g: G \longrightarrow G$ such that\\ $\rho_g(x) = gxg^{-1}$. Then $\rho_g$ is a group isomorphism.

\begin{myIndent}\exTwoP
	Proof:\\ [-20pt]
	\begin{itemize}
		\item Homomorphism: $\rho_g(a)\rho_g(b) = gag^{-1}gbg^{-1} = gabg^{-1} = \rho_g(ab)$.
		\item Surjectivity: given $y \in G$, set $x = g^{-1}yg$. Then $\rho_g(x) = y$.
		\item Injectivity: $gag^{-1} = gbg^{-1} \Rightarrow g^{-1}gag^{-1}g = g^{-1}gbg^{-1}g \Rightarrow a = b$.\newpage
	\end{itemize}
\end{myIndent}

Fix a positive integer $n$ and let $a$ be an integer with $\gcd(a, n) = 1$. Then define\\ $\varphi_a: \mu_n \longrightarrow \mu_n$ by $\varphi_a(\zeta) = \zeta^a$. This is an isomorphism.

\begin{myIndent}\exTwoP
	Proof:\\ [-20pt]
	\begin{itemize}
		\item homomorphism: since multiplication in $\mathbb{C}$ is commutative, 
		
		{\centering $\varphi_a(\zeta_1\zeta_2) = (\zeta_1\zeta_2)^a = \zeta_1^a\zeta_2^a = \varphi_a(\zeta_1)\varphi_a(\zeta_2)$.\retTwo\par}

		\item Bijectivity: we know there exists $r, s \in \mathbb{Z}$ such that $ar + ns = 1$. So define\\ $\varphi_r: \mu_n \longrightarrow \mu_n$. Then note that: $\varphi_r(\varphi_a(\zeta)) = \zeta^{ar} = \varphi_a(\varphi_r(\zeta))$ and\\ $\zeta^{ar} = \zeta^{1-ns} = \zeta(\zeta^n)^{-s} = \zeta \cdot 1^{-s} = \zeta$. So, $\varphi_r = \varphi_a^{-1}$. Hence, $\varphi_a$ is bijective.\retTwo
	\end{itemize}
\end{myIndent}

\hOne
If two groups are \udefine{isomorphic}, we write $G \approx G^\prime$.\retTwo

An isomorphism from a group $G$ to itself is called an \udefine{automorphism}.\retTwo

Two elements $x, y$ of a group $G$ are \udefine{conjugate} if there exists $g \in G$ such that $y = gxg^{-1}$.

\begin{myIndent}\hTwo
	Conjugates behave similar. For example, conjugates have the same order.\retTwo
\end{myIndent}

\mySepTwo

\begin{myIndent}\hTwo
	\blab{Lemma}: Suppose $n \geq 1$ is an integer and $C_n = \langle x \rangle$ is a cyclic group generated by an element $x \in C_n$ (to be clear, this notation tells us that $C_n$ has $n$ elements). Suppose $G$ is also a group and $y \in G$ satisfies that $y^n = 1$. Then there is a unique group homomorphism $\varphi: C_n \longrightarrow G$ with $\varphi(x) = y$.

	\begin{myIndent}\hThree
		Proof:\\
		Define $\varphi(x^k) = y^k$ for all $k \in \mathbb{Z}$. This is well defined because\\ $x^r = x^s \Longrightarrow r - s \in n\mathbb{Z}$. So given that $x^r = x^s$, there exists\\ $q \in \mathbb{Z}$ with $r = s + qn$ and $y^r = y^{s + qn} = y^s(y^n)^q = y^s$.\retTwo

		Having shown that this is well-defined, it's now trivial to see this is a group\\ homomorphism.

		{\centering $\varphi(x^jx^k) = \varphi(x^{j + k}) = y^{j+k} = y^jy^k = \varphi(y^j)\varphi(y^k)$ \retTwo\par}

		It should also be noted that $\varphi$ is unique. This is because the fact that $\varphi$ is a\\ homomorphism means that $\varphi(x^k) = \varphi(x^{k-1})\varphi(x) = \ldots = (\varphi(x))^k = y^k$.
	\end{myIndent}

	\blab{Proposition}: Suppose $G = \langle x \rangle$ and $G^\prime = \langle y \rangle$ are both cyclic of size $n$. Then $G$ is isomorphic to $G^\prime$.\retTwo

	\begin{myIndent}\hThree
		Proof:\\
		Let $\varphi: G\longrightarrow G^\prime$ be the group homomorphism with $\varphi(x) = \varphi(y)$. It is clearly sujective, and since both $G$ and $G^\prime$ have $n$ elements, it must also be injective.\retTwo
	\end{myIndent}

	\blab{Corollary}: Every cyclic group of size $n$ is isomorphic to $\mu_n$.\retTwo
\end{myIndent}

In a similar fashion, we can show every infinite cyclic group to be isomorphic to the integers $\mathbb{Z}$. (Note on notation: if I just write $\mathbb{Z}$, $\mathbb{R}$, or $\mathbb{C}$, assume I'm refering to the groups under addition.)\newpage

\exOne
Proof that $\mathbb{R}$ is not isomorphic to $\mathbb{R}^\times$.
\begin{myIndent}\exTwoP
	Suppose $\rho: \mathbb{R} \longrightarrow \mathbb{R}^\times$ is a group homomorphism. Then\\ $\rho(x) = \rho(\frac{x}{2} + \frac{x}{2}) = \rho(\frac{x}{2})^2 > 0$. So $\rho(x) > 0$ for all $x$.\retTwo
\end{myIndent}

\hOne
\mySepTwo

\mHeader{Lecture 8 Notes: 10/14/2024}






\newpage

\phantom{a} % Barrier so that I don't get struck with academic integrity violation on accident for flashing the homework answers

\newpage

\hOne
\mHeader{Homework 1: Due 10/8/2024}

\begin{enumerate}
	\item Let $S$ be a set with an associative law of composition and with an identity\\ element. Let $G = \{x \in S \mid x \text{ has an inverse}\}$. Prove that $G$ is a group with the law of composition from $S$.
	
	\begin{myIndent}\exOne
		I'll be using multiplicative notation for composition on $S$. Firstly, to prove that the law of composition on $S$ is closed over $G$, suppose $a, b \in G$, meaning there exists $a^{-1}, b^{-1} \in S$ which are inverses of $a$ and $b$ respectively. Then since $\cdot$ is associative on $S$, we know that $(b^{-1}a^{-1})ab = 1 = ab(b^{-1}a^{-1})$. So $ab$ also has an inverse, meaning $ab$.\retTwo
		
		Next, since $1$ is its own inverse, we know $1 \in G$. Also, if $x \in G$, meaning that there exists $x^{-1} \in S$, then $(x^{-1})^{-1} = x$. So $x^{-1} \in G$ as well. Finally, we know\\ that the law of composition on $G$ is associative because we assumed it was associative on $S$. Hence, we've shown that $(G, \cdot)$ is a group.\retTwo
	\end{myIndent}

	\item Let $\mathrm{SL}_2(\mathbb{Z}) = \left\{\gamma = 
	\left[\begin{smallmatrix}
		a & b \\ c & d
	\end{smallmatrix}\right] \mid a, b, c, d \in \mathbb{Z} \text{ and } \det(\gamma) = 1\right\}$. Prove that\\ multiplication of matrices makes $\mathrm{SL}_2(\mathbb{Z})$ a group.

	\begin{myIndent}\exOne
		To start, let's show that $\mathrm{SL}_2(\mathbb{Z})$ is closed under matrix multiplication.
		
		
		\begin{myIndent}\exPP
			Suppose $\gamma_1 = \left[
			\begin{smallmatrix}
				a & b \\ c & d
			\end{smallmatrix}\right]$ and $\gamma_2 = \left[
			\begin{smallmatrix}
				e & f \\ g & h
			\end{smallmatrix}\right]$ are elements of $\mathrm{SL}_2(\mathbb{Z})$. Then\\ $\gamma_1\gamma_2 = \left[
				\begin{smallmatrix}
					ae + bg & af + bh \\ ce + dg & cf + dh
				\end{smallmatrix}\right]$. Since the integers are closed under addition and\\ [2pt] multiplication, we know that all the elements of $\gamma_1\gamma_2$ are integers. Also,\\ [3pt] a fact from linear algebra is that $\det(\gamma_1\gamma_2) = \det(\gamma_1)\det(\gamma_2) = 1^2 = 1$.\\ [3pt] Hence $\gamma_1 \gamma_2 \in \mathrm{SL}_2(\mathbb{Z})$.\retTwo

				If you don't trust that fact about determinants, then you can expand out\\ the expression $(ae + bg)(cf + dh) - (ce + dg)(af + bh)$ yourself.\\ Four of the terms cancel out and the other four can be factored as\\ $(ad - bc)(eh - gf) = \det(\gamma_1)\det(\gamma_2)$.\retTwo
		\end{myIndent}

		Next, observe that $\mathrm{SL}_2(\mathbb{Z})$ satisfies the rules of a group.
		\begin{enumerate}
			\item[1.] $\bm{1} = \left[
				\begin{smallmatrix}
					1 & 0 \\ 0 & 1
				\end{smallmatrix}\right]$ is a multiplicative identity element in $\mathrm{SL}_2(\mathbb{Z})$ since $\det(\bm{1}) = 1$.\\ [-10pt]
			\item[2.]  If $\gamma \in \mathrm{SL}_2(\mathbb{Z})$, then $\gamma^{-1}$ exists and is in $\mathrm{SL}_2(\mathbb{Z})$.
			\begin{myIndent}\exTwoP
				To start, we know that the matrix $\gamma^{-1}$ exists because $\det(\gamma) \neq 0$.\\ Also, note that:
				
				{\centering $1 = \det(\bm{1}) = \det(\gamma\gamma^{-1}) = \det(\gamma)\det(\gamma^{-1}) = 1\cdot \det(\gamma^{-1})$\retTwo\par}

				Finally, if $\gamma = \left[
				\begin{smallmatrix}
					a & b \\ c & d
				\end{smallmatrix}\right]$, then we know that $\gamma^{-1} = \frac{1}{\det(\gamma)}\left[
					\begin{smallmatrix}
						d & -b \\ -c & a
					\end{smallmatrix}\right]$. Since $\det(\gamma) = 1$ and $a, b, c, d \in \mathbb{Z}$, this tells us that all the elements of $\gamma^{-1}$ are integers.\retTwo

					We conclude that $\gamma^{-1} \in \mathrm{SL}_2(\mathbb{Z})$.\\ [-10pt]
			\end{myIndent}

			\item[3.] Matrix multiplication is associative on $\mathrm{SL}_2(\mathbb{Z})$ because it's associative on $\mathcal{M}(2, \mathbb{R})$.\retTwo
		\end{enumerate}
	\end{myIndent}

	\item A group homomorphism $\rho: G_1 \longrightarrow G_2$ is said to be \textit{trivial} if $\rho(g) = 1$ for all $g \in G_1$. Otherwise, the homomorphism is said to be \textit{nontrivial}. If $\mathbb{R}$ is the group of real numbers under addition and $\mathbb{R}^{\times}$ is the group of nonzero real numbers under multiplication, then find a non-trivial homomorphism $\rho: \mathbb{R} \longrightarrow \mathbb{R}^{\times}$.

	\begin{myIndent}\exOne
		Given any $\alpha \in \mathbb{R}$ such that $\alpha > 0$, define $\rho(x) = \alpha^x$ for all $x \in \mathbb{R}$. Note that $\rho(x) \neq 0$ for all $x \in \mathbb{R}$, meaing $\rho(x) \in \mathbb{R}^{\times}$ for all $x \in \mathbb{R}$. Then for all $x, y \in \mathbb{R}$, we have that:

		{\centering $\rho(x + y) = \alpha^{x + y} = \alpha^x\alpha^y = \rho(x)\rho(y)$ \retTwo\par}
	\end{myIndent}
\end{enumerate}

\mySepTwo

\mHeader{Homework 2:}


\begin{enumerate}
	\item (Chapter 2, Problem 4.1) Let $a$ and $b$ be elements of a group $G$. Suppose that $a$ has order $7$ and $a^3b = ba^3$. Prove that $ab = ba$.
	
	\begin{myIndent}\exOne
		Since $ba^3 = a^3b$ and $a^7 = 1$, we have that:
		
		{\centering $b = a^3ba^4 = a^3(ba^3)a = a^3(a^3b)a = a^6ba$. \retTwo\par}

		Composing both sides by $a$ on the left, we get that $ab = 1ba = ba$.\retTwo
	\end{myIndent}
	
	\item (Chapter 2, Problem 4.3) Let $a$ and $b$ be elements of a group $G$. Prove that $ab$ and $ba$ have the same order.
	
	\begin{myIndent}\exOne
		Suppose $n$ is the least positive integer for which $(ab)^n = 1$. Then note that\\ $(ba)^k = b(ab)^{k-1}a$ for all $k \in \mathbb{Z}_+$. So, $(ba)^{n+1} = b(ab)^na = ba$, which\\ in turn means $(ba)^n = 1$. Also, from an earlier proposition, we know\\ $(ab)^k = (ab)^{-1} = b^{-1}a^{-1}$ if and only if $k + 1 = n$. So $n$ is the least positive integer for which $(ab)^n = 1$.\retTwo
	\end{myIndent}

	\item (Chapter 2, Problem 4.4) Suppose $G$ is group that contains no proper (nontrivial) subgroup. Prove $G$ is finite and has order $1$ or order $p$ where $p$ is prime.
	
	\begin{myIndent}\exOne
		To start, obviously a trivial group contains no proper subgroup. So, we'll now assume that $\exists x \in G$ such that $x \neq 1$. We know that the cyclic group $\langle x \rangle$ is a nontrivial subgroup of $G$. Therefore, by our assumption about $G$, we know that $\langle x \rangle = G$.\newpage
		
		Suppose $x$ has infinite order. Then we have a contradiction because $\langle x^2 \rangle$ is a subgroup of $\langle x \rangle = G$ which doesn't contain $x \in G$ (by a previous proposition, if $0$ is the only integer for which $x^0 = 1$, then $x^k = x^1 \Rightarrow k = 1$).\retTwo

		So, we know $x$ has finite order $p \in \mathbb{Z}_+$. Furthermore, since $\langle x^k \rangle = G$ for all $k \in \mathbb{Z}_+$ by assumption, we know that $x^k$ must also have order $p$ for all $k \in \mathbb{Z}_+$. But by a previous proposition, we know that $x^k$ has order $\frac{p}{\gcd(p, k)}$. So, if $k$ is not a multiple of $p$, we must have that $\gcd(p, k) = 1$. Thus, $p$ is coprime with every positive integer less than it, meaning that $p$ must be prime.\retTwo

		Hence, if $G$ is nontrivial, it must have a prime number of elements.\retTwo
	\end{myIndent}

	\item (Chapter 2, Problem 4.10) Suppose $G$ is a group and $a, b \in G$ have finite order.
	\begin{enumerate}
		\item[(a)] Suppose $G$ is abelian. Then $ab$ has finite order.
		
		\begin{myIndent}\exOne
			Suppose $a$ and $b$ have orders $n$ and $m$ respectively. Then becaues $G$ is abelian:

			{\centering $(ab)^{nm} = a^{nm}b^{nm} = (a^n)^m(b^m)^n = 1^m1^n = 1$ \retTwo\par}

			So, the set of integers $N$ such that $(ab)^N = 1$ contains a nonzero element.\retTwo
		\end{myIndent}

		\item[(b)] Show by example that if $G$ is not abelian, then $ab$ need not have finite order.
		
		\begin{myIndent}\exOne
			Consider the group of bijective functions on $\mathbb{Z}$ with function composition as it's rule of composition. Then define the functions:
			
			{\centering $f(n) = |n|$ and $g(n) = \left\{
			\begin{matrix}
				n - 1 & \text{ if } n \text{ is even } \\
				n + 1 & \text{ if } n \text{ if odd  }
			\end{matrix}\right.$ \retTwo\par}

			Using multiplicative notation, we clearly have that $f^2 = 1 = g^2$. On the other hand, we can by induction show that $(fg)^N(1) \neq 1$ for all $N \in \mathbb{Z}_+$.

			
			\begin{myIndent}\exTwoP
				Proof:\\ [-20pt]
				\begin{itemize}
					\item If $n$ is odd and positive, then $fg(n) = f(n + 1) = -n - 1$ which is negative, even, and satisfies that $|fg(n)| > |n|$.
					\item If $n$ is even and negative, then $fg(n) = f(n - 1) = -n + 1$ which is positive, odd, and satisfies that $|fg(n)| > |n|$\retTwo
				\end{itemize}

				Since $1$ is a positive odd number, we know that those will be the only two cases we run into when composing $fg$ with itself. It follows that $(fg)^N(1) \neq 1$ for any $N \in \mathbb{Z}_+$ since $|(fg)^N(1)| \neq 1$ for any $N$.
			\end{myIndent}\exTwoP
		\end{myIndent}

	\end{enumerate}
	
\end{enumerate}


\newpage

\exOne Our textbook is \textit{Algebra, Second Edition} by Michael Artin.



\end{document}