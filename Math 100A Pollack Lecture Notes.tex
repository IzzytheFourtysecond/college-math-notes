\documentclass{book}

\usepackage{fontspec} % used to import Calibri
\usepackage{anyfontsize} % used to adjust font size

% needed for inch and other length measurements
% to be recognized
\usepackage{calc}

% for colors and text effects as is hopefully obvious
\usepackage[dvipsnames]{xcolor}
\usepackage{soul}

% control over margins
\usepackage[margin=1in]{geometry}
\usepackage[strict]{changepage}

\usepackage{mathtools}
\usepackage{amsfonts}
\usepackage{bm}

\usepackage[scr=rsfso, scrscaled=.96]{mathalpha}

\usepackage{amssymb} % originally imported to get the proof square
\usepackage{xfrac}
\usepackage[overcommands]{overarrows} % Get my preferred vector arrows...
\usepackage{relsize}

% Just am using this to get a dashed line in a table...
% Also you apparently want this to be inactive if you aren't
% using it because it slows compilation.
\usepackage{arydshln} \ADLinactivate 
\newenvironment{allowTableDashes}{\ADLactivate}{\ADLinactivate}

\usepackage{graphicx}
\graphicspath{{./158_Images/}}

\usepackage{tikz}
   \usetikzlibrary{arrows.meta}
   \usetikzlibrary{graphs, graphs.standard}

\usepackage{quiver} %commutative diagrams


\newfontfamily{\calibri}{Calibri}
\setlength{\parindent}{0pt}
\definecolor{RawerSienna}{HTML}{945D27}

% ~~~~~~~~~~~~~~~~~~~~~~~~~~~~~~~~~~~~~~~~~~~~~~~~~~
%Arrow Commands:

% Thank you Bernard, gernot, and Sigur who I copied this from:
% https://tex.stackexchange.com/questions/364096/command-for-longhookrightarrow
\newcommand{\hooklongrightarrow}{\lhook\joinrel\longrightarrow}
\newcommand{\hooklongleftarrow}{\longleftarrow\joinrel\rhook}
\newcommand{\hookxlongrightarrow}[2][]{\lhook\joinrel\xrightarrow[#1]{#2}}
\newcommand{\hookxlongleftarrow}[2][]{\xleftarrow[#1]{#2}\joinrel\rhook}

% Thank you egreg who I copied from:
% https://tex.stackexchange.com/questions/260554/two-headed-version-of-xrightarrow
\newcommand{\longrightarrowdbl}{\longrightarrow\mathrel{\mkern-14mu}\rightarrow}
\newcommand{\longleftarrowdbl}{\leftarrow\mathrel{\mkern-14mu}\longleftarrow}

\newcommand{\xrightarrowdbl}[2][]{%
  \xrightarrow[#1]{#2}\mathrel{\mkern-14mu}\rightarrow
}
\newcommand{\xleftarrowdbl}[2][]{%
  \leftarrow\mathrel{\mkern-14mu}\xleftarrow[#1]{#2}
}

\newcommand{\MRoman}[1]{%
   \textrm{\MakeUppercase{\romannumeral #1}}%
}

% ~~~~~~~~~~~~~~~~~~~~~~~~~~~~~~~~~~~~~~~~~~~~~~~~~~

\newcommand{\learnToSpot}[1]{{\color{Red}#1}}

\newcommand{\hOne}{%
   \color{Black}%
   \fontsize{14}{16}\selectfont%
}
\newcommand{\hTwo}{%
\color{MidnightBlue}%
   \fontsize{13}{15}\selectfont%
}
\newcommand{\hThree}{%
   \color{PineGreen!85!Orange}
   \fontsize{12}{14}\selectfont%
}
\newcommand{\myComment}{%
   \color{RawerSienna}%
   \fontsize{12}{14}\selectfont%
}
\newcommand{\teachComment}{
   \color{Orange}%
   \fontsize{12}{14}\selectfont%
}
\newcommand{\exOne}{%
   \color{Purple}%
   \fontsize{13}{15}\selectfont%
}
\newcommand{\exTwo}{%
   \color{Purple}%
   \fontsize{13}{15}\selectfont%
}
\newcommand{\exP}{%
   \color{Purple}%
   \fontsize{12}{14}\selectfont%
}
\newcommand{\exTwoP}{%
   \color{RedViolet}%
   \fontsize{13}{15}\selectfont%
}
\newcommand{\exPP}{%
   \color{RedViolet}%
   \fontsize{12}{14}\selectfont%
}
% ~~~~~~~~~~~~~~~~~~~~~~~~~~~~~~~~~~~~~~~~~~~~~~~~

\newcommand{\cyPen}[1]{{\vphantom{.}\color{Cerulean}#1}}
\newcommand{\redPen}[1]{{\vphantom{.}\color{Red}#1}}

\newenvironment{myIndent}{%
   \begin{adjustwidth}{2.5em}{0em}%
}{%
   \end{adjustwidth}%
}

\newenvironment{myDindent}{%
   \begin{adjustwidth}{5em}{0em}%
}{%
   \end{adjustwidth}%
}

\newenvironment{myTindent}{%
   \begin{adjustwidth}{7.5em}{0em}%
}{%
   \end{adjustwidth}%
}

\newenvironment{myConstrict}{%
   \begin{adjustwidth}{2.5em}{2.5em}%
}{%
   \end{adjustwidth}%
}

\newcommand{\udefine}[1]{{%
   \setulcolor{Red}%
   \setul{0.14em}{0.07em}%
   \ul{#1}%
}}

\newcommand{\blab}[1]{\textbf{#1}}

\newcommand{\uuline}[2][.]{%
{\vphantom{a}\color{#1}%
\rlap{\rule[-0.18em]{\widthof{#2}}{0.06em}}%
\rlap{\rule[-0.32em]{\widthof{#2}}{0.06em}}}%
#2}

\newcommand{\pprime}{{\prime\prime}}
\newcommand{\suchthat}{ \hspace{0.3em}s.t.\hspace{0.3em}}
\newcommand{\rea}[1]{\mathrm{Re}(#1)}
\newcommand{\ima}[1]{\mathrm{Im}(#1)}
\newcommand{\comp}{\mathsf{C}}
\newcommand{\card}{\mathrm{card}}
\newcommand{\diam}{\mathrm{diam}}
\newcommand{\Sym}{\mathrm{Sym}}
\newcommand{\myHS}{ \hspace{0.5em}}

\newcommand{\myId}{\mathrm{Id}}
\newcommand{\myIm}{\mathrm{im}}
\newcommand{\myObj}{\mathrm{Obj}}
\newcommand{\myHom}{\mathrm{Hom}}
\newcommand{\myEnd}{\mathrm{End}}
\newcommand{\myAut}{\mathrm{Aut}}

\newcommand{\mcateg}[1]{{\bm{\mathsf{#1}}}}

\newcommand{\divides}{\mathop{\mid}}


\DeclareMathOperator{\lcm}{lcm}

% Thank you Gonzalo Medina and Moriambar who wrote this on stack exchange:
%https://tex.stackexchange.com/questions/74125/how-do-i-put-text-over-symbols%
\newcommand{\myequiv}[1]{\stackrel{\mathclap{\mbox{\footnotesize{$#1$}}}}{\equiv}}

% Thank you chs who wrote this on stack exchange:
%https://tex.stackexchange.com/questions/89821/how-to-draw-a-solid-colored-circle%
\newcommand{\filledcirc}[1][.]{\ensuremath{\hspace{0.05em}{\color{#1}\bullet}\mathllap{\circ}\hspace{0.05em}}}

%Thank you blerbl who wrote this on stack exchange:
%https://tex.stackexchange.com/questions/25348/latex-symbol-for-does-not-divide
\newcommand{\ndiv}{\hspace{-0.3em}\not|\hspace{0.35em}}

\newcommand{\mySepOne}[1][.]{%
   {\noindent\color{#1}{\rule{6.5in}{1mm}}}\\%
}
\newcommand{\mySepTwo}[1][.]{%
   {\noindent\color{#1}{\rule{6.5in}{0.5mm}}}\\%
}

\newenvironment{myClosureOne}[2][.]{%
   \color{#1}%
   \begin{tabular}{|p{#2in}|} \hline \\%
}{%
   \\ \hline \end{tabular}%
}

\newcommand{\retTwo}{\hfill\bigbreak}

\newcommand{\mHeader}[1]{{
   \color{Black}%
   \fontsize{20}{18}\selectfont%
   #1\retTwo
}}


\title{Math 100A Notes (Professor: Aaron Pollack)}
\author{Isabelle Mills}

\begin{document}
\maketitle{}
\setul{0.14em}{0.07em}
\calibri

\hOne
\mHeader{Lecture 1 Notes: 9/27/2024}

\blab{Motivation for this class:}

Let $\mathcal{F}$ be any figure in $\mathbb{R}^2$. We want some way of talking about the symmetries of $\mathcal{F}$.\retTwo

Letting $d$ be the standard metric for $\mathbb{R}^2$, we say $f: \mathbb{R}^2 \longrightarrow \mathbb{R}^2$ is \udefine{distance preserving}\\ if $d(P, Q) = d(f(P), f(Q))$ for all $P, Q \in \mathbb{R}^2$. If $f$ is distance-preserving and\\ $f(\mathcal{F}) = \mathcal{F}$, then we call $f$ a \udefine{symmetry} of $\mathcal{F}$.\retTwo

We define $\Sym(\mathcal{F})$ to be the set of symmetries of $\mathcal{F}$.
\begin{myIndent}\hTwo
	Lemma 2: The set $\Sym(\mathcal{F})$ has the following properties:
	
	\begin{enumerate}
		\item The identity map $\myId$ is in $\Sym(\mathcal{F})$
		\item If $f \in \Sym(\mathcal{F})$, then $f^{-1} \in \Sym(\mathcal{F})$.
		\begin{myIndent}\hThree
			I realize we haven't yet shown that every $f \in \Sym(\mathcal{F})$ is a bijection. Given such an $f$, it's easy to see that $f$ must be injective. After all, the distance\\ preserving property of $f$ means that $f(P) = f(Q) \Longrightarrow P = Q$. Showing that $f$ is surjective is harder. By assumption, we know that $f$ is surjective when restricted to $\mathcal{F}$. More complicatedly, we can show that $f$ must have a certain form which happens to be surjective. Perhaps I'll prove that later.\retTwo

			Once, you've accepted that $f^{-1}$ exists, then it's clearly true that $f^{-1}$ is also distance preserving with $f^{-1}(\mathcal{F}) = \mathcal{F}$.
		\end{myIndent}
		\item If $f_1, f_2 \in \Sym(\mathcal{F})$, then $f_1 \circ f_2 \in \Sym(\mathcal{F})$ and $f_2 \circ f_1 \in \Sym(\mathcal{F})$.
		\begin{myIndent}\hThree
			This is pretty trivial to show.\retTwo
		\end{myIndent}
	\end{enumerate}
\end{myIndent}

Now while it's all good that we have a concrete way of describing the symmetries of a figure, our current terminology is not the most useful. After all, suppose $\mathcal{S}$ and $\mathcal{S}^\prime$ are two squares such that $\mathcal{S}$ is centered at the origin and $\mathcal{S}^\prime$ is centered at the point $(5, 5)$. Then even though we know both $\mathcal{S}$ and $\mathcal{S}^\prime$ have symmetries in the form of rotating and reflecting, the particular functions in $\Sym(\mathcal{S})$ and $\Sym(\mathcal{S})$ will be different (except for $\myId$). So, how do we compare the symmetries of those two squares?\retTwo

\mySepTwo

Aside start\dots\retTwo

\blab{Proof that all symmetries are surjective (taken from our textbook)}:\\
\begin{myIndent}\myComment
	Note:\\ [-18pt]
	\begin{itemize}
		\item Our textbook calls a distance-preserving function $f: \mathbb{R}^n \longrightarrow \mathbb{R}^n$ an \udefine{isometry}.
		\item Rather than writing $f_1 \circ f_2$ to represent function composition, our textbook just\\ writes $f_1f_2$.\newpage
	\end{itemize}

	\hTwo
	\blab{Some Facts:}
	\begin{itemize}
		\item[(a)] Orthogonal linear operators are isometries.
		
		\begin{myIndent}\hThree
			Let $\varphi$ be n orthogonal linear map. $\varphi$ being linear means that\\ $\varphi(u) - \varphi(v) = \varphi(u - v)$. Meanwhile, $\varphi$ being orthogonal means that\\ $|\varphi(u - v)| = \sqrt{\varphi(u - v) \cdot \varphi(u -v)} = \sqrt{(u - v) \cdot (u - v)} = |u - v|$.\\ So, for any $u, v \in \mathbb{R}^n$, we have that $|\varphi(u) - \varphi(v)| = |u - v|$.\retTwo
		\end{myIndent}

		\item[(b)] The translation $t_a$ by a vector $a$ defined by $t_a(x) = x + a$ is an isometry.
		\begin{myIndent}\hThree
			For any $u, v \in \mathbb{R}^n$, we have $|t_a(u) - t_a(v)| = |u + a - v - a| = |u - v|$.\retTwo
		\end{myIndent}

		\item[(c)] The composition of isometries is an isometry.
		\begin{myIndent}\hThree
			If $f_1, f_2$ are isometries, then for all $u, v \in \mathbb{R}^n$, we have that\\ $|f_1(f_2(u)) - f_1(f_2(v))| = |f_2(u) - f_2(v)| = |u - v|$.\retTwo
		\end{myIndent}
	\end{itemize}

	\blab{Theorem 6.2.3:} The following conditions on a map $\varphi: \mathbb{R}^n \longrightarrow \mathbb{R}^n$ are equivalent:
	\begin{itemize}
		\item[(a)] $\varphi$ is an isometry such that $\varphi(0) = 0$.
		\item[(b)] $\varphi$ preserves dot products: $\varphi(u) \cdot \varphi(w) = u \cdot w$ for all $u, w \in \mathbb{R}^n$.
		\item[(c)] $\varphi$ is an orthogonal linear operator.
		
		\begin{myIndent}\hThree
			Proof:\\
			(c) $\Longrightarrow$ (a)\\
			This comes both from the first fact on this page plus the fact that all linear\\ operators map $0$ to $0$.\retTwo

			(b) $\Longrightarrow$ (c)\\
			Our challenge here is to show that such a $\varphi$ has to be linear operator.\retTwo

			\blab{Lemma:} For $x, y \in \mathbb{R}^n$, if $(x \cdot x) = (x \cdot y) = (y \cdot y)$, then $x = y$.
			
			\begin{myIndent}\hThree
				Proof: $|x - y|^2 = (x - y) \cdot (x - y) = (x \cdot x) - 2(x \cdot y) + (y \cdot y)$.\retTwo
			\end{myIndent}

			Consider any $u, v \in \mathbb{R}^n$ and set $w = u + v$. Then set $u^\prime = \varphi(u)$,\\ $v^\prime = \varphi(v)$, and $w^\prime = \varphi(w)$. To show that $w^\prime = v^\prime + u^\prime$, we shall show\\ that $(w^\prime \cdot w^\prime) = (w^\prime \cdot (u^\prime + v^\prime)) = ((u^\prime + v^\prime) \cdot (u^\prime + v^\prime))$.\retTwo

			Firstly, simplify our equation to:

			{\centering $(w^\prime \cdot w^\prime) = (w^\prime \cdot u^\prime) + (w^\prime \cdot v^\prime) = (u^\prime \cdot u^\prime) + 2(u^\prime \cdot v^\prime) + (v^\prime \cdot v^\prime)$ \retTwo\par}

			Next, since $\varphi$ is assumed to preserve dot products, we can thus simplify our\\ equation to:

			{\centering $(w \cdot w) = (w \cdot u) + (w \cdot v) = (u \cdot u) + 2(u\cdot v) + (v \cdot v)$ \retTwo\par}

			And since $w = u + b$, all of those equalities are true. Hence, we know by our lemma above that $w^\prime = u^\prime + v^\prime$.\newpage

			Meanwhile, let $v \in \mathbb{R}^n$ and set $u = cv$ where $c$ is a constant. Then define $u^\prime$ and $v^\prime$ as before. Then we can do a few trivial simplications to show that $(u^\prime \cdot u^\prime)$, $(u^\prime \cdot cv^\prime)$ and $(cv^\prime \cdot cv^\prime)$ are all equal to $c^2(v \cdot v)$. So, $u^\prime = cv^\prime$.\retTwo

			(a) $\Longrightarrow$ (b)\\
			Since $\varphi$ is distance preserving, we know that $\forall u, v \in \mathbb{R}^n$, 

			{\centering$(\varphi(u) - \varphi(v)) \cdot (\varphi(u) - \varphi(v)) = (u - v) \cdot (u -v)|$.\retTwo\par}

			By plugging in $v = 0$, this simplifies to $(\varphi(u) \cdot \varphi(u)) = (u \cdot u)$. Similarly, by plugging in $u = 0$, we can get that $(\varphi(v) \cdot \varphi(v)) = (v \cdot v)$. So, by expanding and canceling out parts of our above expression, we get that:

			{\centering$-2(\varphi(u) \cdot \varphi(v)) = - 2(u \cdot v)$.\retTwo\par}
		\end{myIndent}
	\end{itemize}

	\blab{Corollary 6.2.7:} Every isometry $f$ of $\mathbb{R}^n$ is the composition of an orthogonal linear operator and a translation. Specifically, if $f(0) = a$, then $f = t_a\varphi$ where $t_a$ is a translation and $\varphi$ is an orthogonal linear operator.

	\begin{myIndent}\hThree
		Proof:\\
		Let $f$ be an isometry, let $a = f(0)$, and define $\varphi = t_{-a}f$. Then clearly $t_a\varphi = f$. So, we just need to show that $\varphi$ is an orthogonal linear operator. To prove this, first note that $\varphi$ is the composition of two isometries, and is thus an isometry itself. Also, $\varphi(0) = -a + f(0) = -a + a = 0$. So applying theorem 6.2.3, we know that $\varphi$ is an orthogonal linear operator.\retTwo
	\end{myIndent}
\end{myIndent}

Now we've proven in other classes that both translations and linear orthogonal\\ operators on $\mathbb{R}^n$ are surjective. So, all isometries are the composition of surjections, meaning they are surjective themselves. And since we also previously proved that all isometries are injective, we know they are bijective and have inverses.\retTwo

Aside over\dots

\mySepTwo

\mHeader{Lecture 2 Notes: 9/30/2024}

I already covered everything from this lecture in my math journal (pages 40-42).\\

\mySepTwo

\mHeader{Lecture 3 Notes: 10/2/2024}

Suppose $G_1$ and $G_2$ are groups. A map $\rho: G_1 \longrightarrow G_2$ is called a \udefine{group\\ homomorphism} if $\rho(xy) = \rho(x)\rho(y)$ for all $x, y \in G_1$. If $\rho$ is bijective, we say that\\ $\rho$ is an \udefine{isomorphism}, and that $G_1$ and $G_2$ are \udefine{isormophic.} Also if $\rho$ is bijective, we\\ have that $\rho^{-1}$ is also a group homomorphism.\newpage

If two groups are isomorphic, then we can say they are in a sense equivalent.\retTwo

Suppose $G$ is a group and $H \subseteq G$. Then $H$ equipped with the law of composition of $G$ restricted to $H \times H$ is a \udefine{subgroup} if:\\ [-20pt]
\begin{itemize}
	\item $1 \in H$\\ [-20pt]
	\item $x \in H \Longrightarrow x^{-1} \in H$\\ [-20pt]
	\item $x, y \in H \Longrightarrow xy \in H$\retTwo
\end{itemize}

\exOne

Example: If $\mathbb{R}^\times = (\mathbb{R} - \{0\}, \times)$, then some non-trivial subgroups of $\mathbb{R}^x$ are:\\ [-20pt]

\begin{itemize}
	\item $M_2 = \{1, -1\}$\\ [-20pt]
	\item $\mathbb{Z}^x = \mathbb{Z} - \{0\}$\\ [-20pt]
	\item $\mathbb{Q}^x = \mathbb{Q} - \{0\}$\\ [-20pt]
	\item $H = \{a^n \in \mathbb{R} \mid n \in \mathbb{Z}\}$.
\end{itemize}

\begin{myIndent}\hTwo
	\blab{Theorem:} Let $S$ be a subgroup of $(\mathbb{Z}, +)$ (the set of integers equipped with integer addition). Then either $S = \{0\}$ or $S = \mathbb{Z}a = \{na \mid n \in \mathbb{Z}\}$ where $a$ is the least positive element of $S$.

	
	\begin{myIndent}\hThree
		Proof:\\
		We clearly have that $\{0\}$ and $\mathbb{Z}a$ are groups under addition for any $a \in \mathbb{Z}_+$.\\ Meanwhile, suppose $S \neq \{0\}$ is a subgroup of $(\mathbb{Z}, +)$. Then, by taking inverses if necessary, we know $S \cap \mathbb{Z}_+$ is nonempty. Since $\mathbb{Z}_+$ is well-ordered, there exists a least element in $S \cap \mathbb{Z}_+$ which we'll call $a$.\retTwo

		Trivially, we have that $\mathbb{Z}a \subseteq S$. Meanwhile consider any $n \in S$. Then $n = qa + r$ for some $q \in \mathbb{Z}$ and $r \in \{0, 1, \ldots, a - 1\}$. However, since $r = n - qa$ and $n, -qa \in S$, we must have that $r \in S$. And, the only allowed value for $r$ such that $r \in S$ is $r = 0$. Thus, $n \in \mathbb{Z}a$, meaning we've shown that $S \subseteq \mathbb{Z}_a$.\retTwo
	\end{myIndent}
\end{myIndent}

\hOne\mySepTwo

\mHeader{Lecture 4 Notes: 10/4/2024}

As an immediate application of the above theorem, note that $S = \mathbb{Z}a + \mathbb{Z}b = \{ma + nb \mid m, n \in \mathbb{Z}\}$ is subgroup of $\mathbb{Z}$ under addition.
\begin{myTindent}\myComment
	This is trivial to prove.\retTwo
\end{myTindent}

By our previous theorem, we know that $S = \mathbb{Z}d$ for some unique positive integer $d$. So, we define the \udefine{greatest common divisor} of $a$ and $b$ to be $\gcd(a, b) \coloneq d$.

\begin{myIndent}\hTwo
	\blab{Proposition:} Let $a, b \in \mathbb{Z}$ be not both $0$ and $d = \gcd(a, b)$.
	\begin{enumerate}
		\item There exists $r, s \in \mathbb{Z}$ such that $d = ra + sb$
		\item $d$ divides $a$ and $b$ (written $d \divides a$ and $d \divides b$).
		\begin{myIndent}\hThree
			Both of these claims are trivially true by our definition of $S$.\newpage
		\end{myIndent}
		\item If $e \in \mathbb{Z}$ and $e$ divides $a$ and $b$, then $e$ divides $d$. This is why $d$ is called the\\ "greatest common divisor" of $a$ and $b$.
		\begin{myIndent}\hThree
			Let $r, s \in \mathbb{Z}$ such that $d = ra + sb$. Then letting $a = en$ and $b = em$, we have that $d = (rn + sm)e$, meaning $e \divides d$.\retTwo
		\end{myIndent}
	\end{enumerate}
\end{myIndent}

An algorithm for finding $\gcd(a, b)$ is given as follows:
\begin{enumerate}
	\item Assume without loss of generality that $a \geq b \geq 0$ and $a \neq 0$.
	\item If $b = 0$, then $\gcd(a, b) = \gcd(b, a) = a$
	\item Else, there exists $q, r \in \mathbb{Z}$ with $0 \leq r < b$ and $a = qb + r$. We claim\\ that $\gcd(a, b) = \gcd(b, r)$.
	
	\begin{myIndent}\hTwo
		This is because if $d \divides a$ and $d \divides b$, then we know $d \divides (qb + r)$ and $d \divides qb$,\\ meaning that $d \divides (qb + r - qb) = r$. On the other hand, if $e \divides r$ and $e \divides b$,\\ then $e \divides (qb + r) = a$. So $a$ and $b$ have the same common factors as $b$ and $c$.\retTwo
	\end{myIndent}
\end{enumerate}

Suppose $a, b \in \mathbb{Z}$. We say $a$ and $b$ are \udefine{relatively prime} iff $\gcd(a, b) = 1$.

\begin{myIndent}\hTwo
	\blab{Corollary:} $\gcd(a, b) = 1$ if and only if there exists $r, s \in \mathbb{Z}$ such that $ra + sb = 1$.
	\begin{myIndent}\hThree
		Proof:\\
		($\Longrightarrow$) By definition, $\gcd(a, b) \in \mathbb{Z}a + \mathbb{Z}b$.\\
		($\Longleftarrow$) If $ra + sb = 1$, then $1$ must be the least positive element of $\mathbb{Z}a + \mathbb{Z}b$.\\ So $\gcd(a, b) = 1$.\retTwo
	\end{myIndent}

	\blab{Lemma:} Suppose $\gcd(a, b) = 1$ and $a \divides bc$. Then $a \divides c$.
	\begin{myIndent}\hThree
		Proof:\\
		Let $1 = ra + sb$ where $r, s \in \mathbb{Z}$. Then $c = rac + sbc = (rc + s\frac{bc}{a})a$ where $\frac{bc}{a}$\\ [-2pt] is an integer. So $a \divides c$.\retTwo
	\end{myIndent}

	\blab{Corollary:} Suppose $p$ is a prime integer. If $a, b \in \mathbb{Z}$ and $p \divides ab$, then either $p \divides a$\\ or $p \divides b$.
	\begin{myIndent}\hThree
		Proof:\\
		Suppose $p {\not\divides a}$. Then $\gcd(p, a) = 1$ because the only positive divisor of $p$ other\\ than $p$ is $1$. So there exists $r, s \in \mathbb{Z}$ such that $1 = rp + sa$. In turn, since $\frac{ab}{p}$ is an\\ [-3pt] integer, we have $b = rpb + sab = p(rb +s\frac{ab}{p})$, meaning $p \divides b$.\retTwo
	\end{myIndent}
\end{myIndent}

\exOne\mySepTwo

\blab{Problem:} Suppose $p$ is prime and that $a \in \mathbb{Z}$ is not a multiple of $p$. Then there exists $x \in \mathbb{Z}$ so that $ax$ is one more than some multiple of $p$.

\begin{myIndent}\exTwoP
	Proof:\\
	Like before, we must have that $\gcd(a, p) = 1$, meaning that there exists $r, s \in \mathbb{Z}$\\ such that $rp + sa = 1$. So, if we set $x = s$, we'd be done cause $xa = (-r)p + 1$.\retTwo

	More interestingly, we can guarentee that $xa$ is one more than a nonnegative multiple of $p$ as follows:\newpage
	\begin{myIndent}
		Note that $sa = -rp + 1 \Longrightarrow (s^2a)a = (r^2p - 2r)p + 1 = r(rp - 2)p + 1$.\\ Since $p \geq 2$, we have that $r \geq 1 \Longrightarrow (rp - 2) > 0$, meaning $r(rp - 2) > 0$.\\ Meanwhile, we have that $r \leq 0 \Longrightarrow (rp - 2) < 0$, which in turn means $r(rp - 2) \geq 0$.\retTwo

		Setting $x = s^2a$ and $n = r^2p - 2r$, we thus have that $xa = np + 1$ where\\ $np$ is a nonnegative multiple of $p$.
	\end{myIndent}
\end{myIndent}

\mySepTwo

\hTwo
\blab{Lemma:} Suppose $G$ is a group and $H_1, H_2$ are subgroups of $G$. Then $H_1 \cap H_2$ is a subgroup of $G$.
\begin{myIndent}\hThree
	This is rather trivial to prove. So do it yourself! :3\retTwo
\end{myIndent}

\hOne

Because of the above lemma, given $a, b \in \mathbb{Z}$, we have that $\mathbb{Z}a \cap \mathbb{Z}b = \mathbb{Z}m$ for some integer $m \geq 0$. We call $m$ the \udefine{least common multiple} of $a$ and $b$, and we denote $\lcm(a, b) \coloneq m$.

\begin{myIndent}\hTwo
	\blab{Proposition:} Let $a$ and $b$ be nonzero integers and $m = \lcm(a, b)$.
	\begin{enumerate}
		\item $m$ is nonzero.
		\item $m$ is divisible by both $a$ and $b$
		\begin{myIndent}\hThree
			Both of these points are trivial from the fact that $\mathbb{Z}a \cap \mathbb{Z}b = \mathbb{Z}m$ and\\ $ab \in \mathbb{Z}m$, meaning that $\mathbb{Z}m - \{0\} \neq \emptyset$.
		\end{myIndent}
		\item If $n \in \mathbb{Z}$ such that $a \divides n$ and $b \divides n$, then $m \divides n$.
		\begin{myIndent}\hThree
			This comes trivially from the fact that $n \in \mathbb{Z}a$ and $n \in \mathbb{Z}b$ means that\\ $n \in \mathbb{Z}a \cap \mathbb{Z}b = \mathbb{Z}m$\retTwo
		\end{myIndent}
	\end{enumerate}
\end{myIndent}


Suppose $G$ is a group and $x \in G$. Then let $H = \{x^k \mid k \in \mathbb{Z}\} \subseteq G$. We clearly have that $H$ is a subgroup of $G$. We call it the \udefine{cyclic subgroup} of $G$ generated by $x$, and denote it $H = \langle x \rangle$.

\begin{myIndent}\hTwo
	\blab{Proposition:} Let $S = \{k \in \mathbb{Z} \mid x^k = 1\}$
	\begin{enumerate}
		\item $S$ is a subgroup of $(\mathbb{Z}, +)$.
		\begin{myIndent}\hThree
			This is rather trivial to show. So do it yourself!!
		\end{myIndent}
		\item Suppose $S \neq \{0\}$, meaning $S = \mathbb{Z}n$ for some positive integer $n$. Then\\ $1, x, \ldots, x^{n-1}$ are the distinct elements of $\langle x \rangle$, meaning the order of $\langle x \rangle$ is $n$.
		
		\begin{myIndent}\hThree
			Proof:\\
			$x^{k} = x^{l} \Longleftrightarrow x^{k-l} = 1$. Hence, since $n$ is the minimum positive integer such that $x^n = 1$, we know that $1, x, \ldots, x^{n-1}$ are distinct. On the other hand, if $k = qn + r$ for any $q, r \in \mathbb{Z}$ with $0 \leq r < n$, then $x^k = (x^n)^qx^r = x^r$. So the only elements of $\langle x \rangle$ are $1, x, \ldots, x^{n-1}$.\retTwo
		\end{myIndent}
	\end{enumerate}
\end{myIndent}

\newpage



\newpage

\exOne Our textbook is \textit{Algebra, Second Edition} by Michael Artin.



\end{document}