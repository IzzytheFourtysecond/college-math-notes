% Note for any github stalkers. I am currently in the process
% of learning LaTeX. I don't know what I'm doing yet. Sorry
% if my code absolutely sucks.


\documentclass{book}

\usepackage{fontspec} % used to import Calibri
\usepackage{anyfontsize} % used to adjust font size

% needed for inch and other length measurements
% to be recognized
\usepackage{calc}

% for colors and text effects as is hopefully obvious
\usepackage[dvipsnames]{xcolor}
\usepackage{soul}

% control over margins
\usepackage[margin=1in]{geometry}
\usepackage[strict]{changepage}

\usepackage{mathtools}
\usepackage{amsfonts}
\usepackage{amssymb} % originally imported to get the proof square

\setmainfont{Calibri}
\setlength{\parindent}{0pt}
\definecolor{RawerSienna}{HTML}{945D27}

\newcommand{\hOne}{%
   \color{Black}%
   \fontsize{14}{14}\selectfont%
}
\newcommand{\hTwo}{%
   \color{MidnightBlue}%
   \fontsize{13}{13}\selectfont%
}
\newcommand{\hThree}{%
   \color{PineGreen}
   \fontsize{13}{13}\selectfont%
}
\newcommand{\myComment}{%
   \color{RawerSienna}%
   \fontsize{12}{12}\selectfont%
}

\newenvironment{myIndent}{%
   \begin{adjustwidth}{2.5em}{0em}%
}{%
   \end{adjustwidth}%
}

\newcommand{\udefine}[1]{%
   \setulcolor{Red}%
   \setul{0.1ex}{0.15ex}%
   \ul{#1}%
}

\newcommand*{\markDate}[1]{
   {\huge \color{Black} \textbf{#1} \newline}
}

\newcommand{\pprime}{\prime\prime}

\newcounter{PropNumber}
\newcommand{\propCount}{%
   \stepcounter{PropNumber}%
   \thePropNumber%
}



\title{My Notes on Paolo Aluffi's Algebra Chapter 0}
\author{Isabelle Mills}

\begin{document}
   \maketitle{}

   \markDate{1/7/2024}

   \hOne
   A \udefine{multiset} is a collection of elements 
   which like a set is unordered but unlike a set can
   contain duplicate elements.
   
   \begin{myIndent}
      \hTwo
      One way we can define a multiset is as a function
      \( f: A\rightarrow\mathbb{N} \) such that each
      \(\alpha \in A \) is mapped to the number of times that 
      \(\alpha\) appears in the multiset. Then, given the multisets 
      \(f_{1}: A\rightarrow \mathbb{N}\) and \(f_{2}: 
      B \rightarrow \mathbb{N}\), we can define the following
      operations:

      \hThree
      \begin{itemize}
         \item \(\alpha \in f_{1} \leftrightarrow \alpha \in A\)

         \item \(f_{1} \subseteq f_{2} \leftrightarrow 
               \forall \alpha \in f_{1}, \hspace{0.25em} \alpha \in f_{2} 
               \text{ and } f_{1}(\alpha) \leq f_{2}(\alpha)\)

         \item \(f_{1} \cup f_{2}: (A \cup B) \rightarrow
               \mathbb{N} \) such that for \(\alpha
               \in A \cup B\), if \(\alpha \in A \cap B \),
               then\\ \((f_{1} \cup f_{2})(\alpha) = f_{1}(\alpha)
               + f_{2}(\alpha)\). As for if \(\alpha \notin 
               A \cap B \), then \((f_{1} \cup f_{2})(\alpha)\)
               equals whatever $\alpha$ was mapped to 
               in the multiset it originally came from.
         
         \item \(f_{1} \cap f_{2}: (A \cap B) \rightarrow
               \mathbb{N} \) such that for \(\alpha
               \in A \cap B\), we have that\\
               \((f_{1} \cap f_{2})(\alpha) = \min(f_{1}(\alpha)
               , f_{2}(\alpha))\)
         
         \item \(f_{1} \setminus f_{2}: ((A \setminus B)
               \cup \{\alpha \in A \cap B \mid f_{1}(\alpha) >
               f_{2}(\alpha)\}) \rightarrow \mathbb{N} \) 
               such that for each \(\alpha \in f_{1} \setminus f_{2}\), 
               if \(\alpha \in f_{2} \), then \((f_{1} \setminus f_{2})(\alpha) 
               = f_{1}(\alpha) - f_{2}(\alpha)\). As for if \(\alpha 
               \notin f_{2} \), then \((f_{1} \setminus f_{2})(\alpha) = 
               f_{1}(\alpha)\)
         
      \end{itemize}
   
      \begin{myIndent}\begin{myIndent}\begin{myIndent}
      \begin{myIndent}\begin{myIndent}
         \myComment
         \hfill \break
         A practical example of a multiset is the prime
         \\factorization of any positive integer.
      \end{myIndent}\end{myIndent}\end{myIndent}
      \end{myIndent}\end{myIndent}
   
   \end{myIndent}

   \hrulefill

   \hOne
   \hfill \break
   We say that two sets $A$ and $B$ are \udefine{isomorphic} if and only
   if there exists a bijection between $A$ and $B$. We denote this by
   writing $A \cong B$. Additionally, we can refer to any bijection $f$
   between $A$ and $B$ as an \udefine{isomorphism} between the two sets
   (also written as \(f: A \xlongrightarrow{\sim} B\))

   \hfill \break
   A function $f: A \rightarrow B$ is a \udefine{monomorphism} (a.k.a 
   a \udefine{monic}) if for all sets $Z$ and all functions $a\sp{\prime}$
   and $a\sp{\pprime}: Z \rightarrow A$, we have that $f \circ a\sp{\prime}
   = f \circ a\sp{\pprime} \Rightarrow a\sp{\prime} = a\sp{\pprime}$

   
   % Note to future self, I just made it so that if you
   % type ctrl + shift + e followed by space, then it will
   % surround a highlighted text with begin{}...end{}
   \begin{myIndent}
      \hTwo
      Proposition \propCount: A function is injective if and only if
      it is a monomorphism.

      \begin{myIndent}
         \hThree
         Proof: Let's say we have a function $f: A \rightarrow B$.
         \hfill \bigbreak

         First, let us assume $f$ is injective.
         
         \begin{myIndent}
            Then let us assume we have two functions $a\sp{\prime}$ 
            and $a\sp{\pprime}$ from some set $Z$ to $A$ such that 
            $f \circ a\sp{\prime} = f \circ a\sp{\pprime}$. Then, 
            because $f$ is injective, we know there exists a function
            $g: B \rightarrow A$ such that $g \circ f = \mathrm{Id}_A$.
            Composing $g$ with the previous equation, we get that:
            \[g\circ(f \circ a\sp{\prime}) = g \circ (f \circ 
            a\sp{\pprime}) \Longrightarrow \mathrm{Id}_A \circ a\sp{\prime}
            = \mathrm{Id}_A \circ a\sp{\pprime} \Longrightarrow 
            a\sp{\prime} = a\sp{\pprime}\]
            Thus by our assumptions, we have shown $f$ to be a\\
            monomorphism.
         \end{myIndent}
         \newpage

         Next, we shall assume $f$ is a monomorphism.

         \begin{myIndent}
            Based on this, we can say that for any two functions 
            $a\sp{\prime}$ and $a\sp{\pprime}$ mapping a set
            $Z$ to $A$, we have that $f \circ a\sp{\prime}
            = f \circ a\sp{\pprime} \Rightarrow a\sp{\prime} = 
            a\sp{\pprime}$. However, now note that if we make $Z$ a 
            \udefine{singleton}, meaning it only contains one element, 
            then $a\sp{\prime}$ and $a\sp{\pprime}$ can each
            only take on one value. So, we can effectively rewrite
            $f \circ a\sp{\prime} = f \circ a\sp{\pprime}
            \Rightarrow a\sp{\prime} = a\sp{\pprime}$ as:
            \[f(a\sp{\prime}) = f(a\sp{\pprime}) \Rightarrow 
            a\sp{\prime} = a\sp{\pprime}\]
            This is the definition of an injective function. $\blacksquare$
         \end{myIndent}
      \end{myIndent}
   \end{myIndent}

   \hfill \bigbreak
   \markDate{1/8/2024}
   \hfill \bigbreak

   \hOne
   A function $f: A \rightarrow B$ is an \udefine{epimorphism} 
   (a.k.a an \udefine{epi}) if for all sets $Z$ and all functions $a\sp{\prime}$
   and $a\sp{\pprime}: B \rightarrow Z$, we have that 
   $a\sp{\prime} \circ f = a\sp{\pprime} \circ f \Rightarrow 
   a\sp{\prime} = a\sp{\pprime}$

   
   \begin{myIndent}
      \hTwo
      Proposition \propCount: A function is a surjection if and
      only if it is an epimorphism.
      
      \hThree
      \begin{myIndent}
         Proof: Let's say we have a function $f: A \rightarrow B$.
         \hfill \bigbreak

         First, let us assume $f$ is surjective.

      \end{myIndent}
   \end{myIndent}

\end{document}