% Note for any github stalkers. I am currently in the process
% of learning LaTeX. I don't know what I'm doing yet. Sorry
% if my code absolutely sucks.


\documentclass{book}

\usepackage{fontspec} % used to import Calibri
\usepackage{anyfontsize} % used to adjust font size

% needed for inch and other length measurements
% to be recognized
\usepackage{calc}

% for colors and text effects as is hopefully obvious
\usepackage[dvipsnames]{xcolor}
\usepackage{soul}

% control over margins
\usepackage[margin=1in]{geometry}
\usepackage[strict]{changepage}

\usepackage{mathtools}
\usepackage{amsfonts}
\usepackage{amssymb} % originally imported to get the proof square
\usepackage{xfrac}
\usepackage[overcommands]{overarrows} % Get my preferred vector arrows...
\usepackage{relsize}

% Just am using this to get a dashed line in a table...
% Also you apparently want this to be inactive if you aren't
% using it because it slows compilation.
\usepackage{arydshln} \ADLinactivate 
\newenvironment{allowTableDashes}{\ADLactivate}{\ADLinactivate}

\usepackage{graphicx}
\graphicspath{{./158_Images/}}

\usepackage{tikz}
   \usetikzlibrary{arrows.meta}
   \usetikzlibrary{graphs, graphs.standard}

\usepackage{quiver} %commutative diagrams


\newfontfamily{\calibri}{Calibri}
\setlength{\parindent}{0pt}
\definecolor{RawerSienna}{HTML}{945D27}

% ~~~~~~~~~~~~~~~~~~~~~~~~~~~~~~~~~~~~~~~~~~~~~~~~~~
%Arrow Commands:

% Thank you Bernard, gernot, and Sigur who I copied this from:
% https://tex.stackexchange.com/questions/364096/command-for-longhookrightarrow
\newcommand{\hooklongrightarrow}{\lhook\joinrel\longrightarrow}
\newcommand{\hooklongleftarrow}{\longleftarrow\joinrel\rhook}
\newcommand{\hookxlongrightarrow}[2][]{\lhook\joinrel\xrightarrow[#1]{#2}}
\newcommand{\hookxlongleftarrow}[2][]{\xleftarrow[#1]{#2}\joinrel\rhook}

% Thank you egreg who I copied from:
% https://tex.stackexchange.com/questions/260554/two-headed-version-of-xrightarrow
\newcommand{\longrightarrowdbl}{\longrightarrow\mathrel{\mkern-14mu}\rightarrow}
\newcommand{\longleftarrowdbl}{\leftarrow\mathrel{\mkern-14mu}\longleftarrow}

\newcommand{\xrightarrowdbl}[2][]{%
  \xrightarrow[#1]{#2}\mathrel{\mkern-14mu}\rightarrow
}
\newcommand{\xleftarrowdbl}[2][]{%
  \leftarrow\mathrel{\mkern-14mu}\xleftarrow[#1]{#2}
}




% ~~~~~~~~~~~~~~~~~~~~~~~~~~~~~~~~~~~~~~~~~~~~~~~~~~

\newcommand{\hOne}{%
   \color{Black}%
   \fontsize{14}{16}\selectfont%
}
\newcommand{\hTwo}{%
   \color{MidnightBlue}%
   \fontsize{13}{15}\selectfont%
}
\newcommand{\hThree}{%
   \color{PineGreen!85!Orange}
   \fontsize{13}{15}\selectfont%
}
\newcommand{\hFour}{%
   \color{Cerulean}
   \fontsize{12}{14}\selectfont%
}
\newcommand{\myComment}{%
   \color{RawerSienna}%
   \fontsize{12}{14}\selectfont%
}
\newcommand{\pracOne}{
   \color{BrickRed}%
   \fontsize{13}{15}\selectfont%
}
\newcommand{\pracTwo}{
   \color{Orange}%
   \fontsize{12}{14}\selectfont%
}
\newcommand{\exOne}{%
   \color{Purple}%
   \fontsize{14}{16}\selectfont%
}
\newcommand{\exTwo}{%
   \color{RedViolet}%
   \fontsize{13}{15}\selectfont%
}
\newcommand{\exP}{%
   \color{VioletRed}%
   \fontsize{12}{14}\selectfont%
}
% ~~~~~~~~~~~~~~~~~~~~~~~~~~~~~~~~~~~~~~~~~~~~~~~~

\newcommand{\cyPen}[1]{{\vphantom{.}\color{Cerulean}#1}}

\newenvironment{myIndent}{%
   \begin{adjustwidth}{2.5em}{0em}%
}{%
   \end{adjustwidth}%
}

\newenvironment{myDindent}{%
   \begin{adjustwidth}{5em}{0em}%
}{%
   \end{adjustwidth}%
}

\newenvironment{myTindent}{%
   \begin{adjustwidth}{7.5em}{0em}%
}{%
   \end{adjustwidth}%
}

\newenvironment{myConstrict}{%
   \begin{adjustwidth}{2.5em}{2.5em}%
}{%
   \end{adjustwidth}%
}

\newcommand{\udefine}[1]{{%
   \setulcolor{Red}%
   \setul{0.14em}{0.07em}%
   \ul{#1}%
}}

\newcommand{\uuline}[2][.]{%
{\vphantom{a}\color{#1}%
\rlap{\rule[-0.18em]{\widthof{#2}}{0.06em}}%
\rlap{\rule[-0.32em]{\widthof{#2}}{0.06em}}}%
#2}

\newcommand*{\markDate}[1]{%
   {\huge \color{Black} \textbf{#1} \newline}%
}

\newcommand{\pprime}{{\prime\prime}}
\newcommand{\suchthat}{ \hspace{0.5em}s.t.\hspace{0.5em}}
\newcommand{\rea}[1]{\mathrm{Re}(#1)}
\newcommand{\ima}[1]{\mathrm{Im}(#1)}
\newcommand{\comp}{\mathsf{C}}
\newcommand{\myHS}{ \hspace{0.5em}}

\newcommand{\myId}{\mathrm{Id}}
\newcommand{\myIm}{\mathrm{im}}
\newcommand{\myObj}{\mathrm{Obj}}
\newcommand{\myHom}{\mathrm{Hom}}
\newcommand{\myEnd}{\mathrm{End}}
\newcommand{\myAut}{\mathrm{Aut}}

\newcommand{\mcateg}[1]{\mathsf{#1}}

% Thank you Gonzalo Medina and Moriambar who wrote this on stack exchange:
%https://tex.stackexchange.com/questions/74125/how-do-i-put-text-over-symbols%
\newcommand{\myequiv}[1]{\stackrel{\mathclap{\mbox{\footnotesize{$#1$}}}}{\equiv}}

% Thank you chs who wrote this on stack exchange:
%https://tex.stackexchange.com/questions/89821/how-to-draw-a-solid-colored-circle%
\newcommand{\filledcirc}[1][.]{\ensuremath{\hspace{0.05em}{\color{#1}\bullet}\mathllap{\circ}\hspace{0.05em}}}

%Thank you blerbl who wrote this on stack exchange:
%https://tex.stackexchange.com/questions/25348/latex-symbol-for-does-not-divide
\newcommand{\ndiv}{\hspace{-0.3em}\not|\hspace{0.35em}}

\newcounter{PropNumber}
\newcommand{\propCount}{%
   \stepcounter{PropNumber}%
   \thePropNumber%
}

\newcommand{\mySepOne}[1][.]{%
   {\noindent\color{#1}{\rule{6.5in}{1mm}}}\\%
}
\newcommand{\mySepTwo}[1][.]{%
   {\noindent\color{#1}{\rule{6.5in}{0.5mm}}}\\%
}

\newenvironment{myClosureOne}[2][.]{%
   \color{#1}%
   \begin{tabular}{|p{#2in}|} \hline \\%
}{%
   \\ \hline \end{tabular}%
}

\newcommand{\retTwo}{\hfill\bigbreak}


\title{My Notes on Paolo Aluffi's Algebra Chapter 0}
\author{Isabelle Mills}

\begin{document}
   \maketitle{}
   \setul{0.14em}{0.07em}
   \calibri

   \markDate{1/7/2024}

   \hOne
   A \udefine{multiset} is a collection of elements 
   which like a set is unordered but unlike a set\\ can
   contain duplicate elements.
   
   \begin{myIndent}
      \exOne
      One way to define a multiset is as a function
      \( f: A\rightarrow\mathbb{N} \) such that each
      \(\alpha \in A \)\\ is mapped to the number of times that 
      \(\alpha\) appears in the multiset. Then, given\\ the multisets 
      \(f_{1}: A\rightarrow \mathbb{N}\) and \(f_{2}: 
      B \rightarrow \mathbb{N}\), we can define the following\\
      operations:
      \exTwo
      \begin{itemize}
         \item \(\alpha \in f_{1} \Longleftrightarrow \alpha \in A\)\\ [-10pt]

         \item \(f_{1} \subseteq f_{2} \Longleftrightarrow 
               \forall \alpha \in f_{1}, \hspace{0.25em} \alpha \in f_{2} 
               \text{ and } f_{1}(\alpha) \leq f_{2}(\alpha)\)\\ [-12pt]

         \item \(f_{1} \cup f_{2}: (A \cup B) \longrightarrow
               \mathbb{N} \) such that for \(\alpha
               \in A \cup B\), if \(\alpha \in A \cap B \),
               then\\ \((f_{1} \cup f_{2})(\alpha) = f_{1}(\alpha)
               + f_{2}(\alpha)\). As for if \(\alpha \notin 
               A \cap B \), then \((f_{1} \cup f_{2})(\alpha)\)\\
               equals whatever $\alpha$ was mapped to 
               in the multiset it originally came from.\\ [-12pt]
         
         \item \(f_{1} \cap f_{2}: (A \cap B) \longrightarrow
               \mathbb{N} \) such that for \(\alpha
               \in A \cap B\), we have that\\
               \((f_{1} \cap f_{2})(\alpha) = \min(f_{1}(\alpha)
               , f_{2}(\alpha))\)\\ [-12pt]
         
         \item \(f_{1} \setminus f_{2}: ((A \setminus B)
               \cup \{\alpha \in A \cap B \mid f_{1}(\alpha) >
               f_{2}(\alpha)\}) \longrightarrow \mathbb{N} \) 
               such that for\\ each \(\alpha \in f_{1} \setminus f_{2}\), 
               if \(\alpha \in f_{2} \), then \((f_{1} \setminus f_{2})(\alpha) 
               = f_{1}(\alpha) - f_{2}(\alpha)\). As for if\\ \(\alpha 
               \notin f_{2} \), then \((f_{1} \setminus f_{2})(\alpha) = 
               f_{1}(\alpha)\)\\ [-12pt]
         
      \end{itemize}
   
      \begin{myIndent}\begin{myIndent}\begin{myIndent}
      \begin{myIndent}\begin{myIndent}
         \myComment
         \hfill \break
         A practical example of a multiset is the prime
         \\factorization of any positive integer.
      \end{myIndent}\end{myIndent}\end{myIndent}
      \end{myIndent}\end{myIndent}
   
   \end{myIndent}

   \hrulefill

   \hOne
   \hfill \break
   We say that two sets $A$ and $B$ are \udefine{isomorphic} if and only
   if there exists a bijection\\ between $A$ and $B$. We denote this by
   writing $A \cong B$. Additionally, we can refer\\ to any bijection $f$
   between $A$ and $B$ as an \udefine{isomorphism} between the two sets.

   \hfill \break
   A function $f: A \rightarrow B$ is a \udefine{monomorphism} (a.k.a 
   a \udefine{monic}) if for all sets $Z$ and all\\ functions $a^{\prime}$
   and $a^{\pprime}: Z \rightarrow A$, we have that $f \circ a^{\prime}
   = f \circ a^{\pprime} \Longrightarrow a^{\prime} = a^{\pprime}$.

   % Note to future self, I just made it so that if you
   % type ctrl + shift + e followed by space, then it will
   % surround a highlighted text with begin{}...end{}
   \begin{myIndent}
      \hTwo
      Proposition \propCount: A function is injective if and only if
      it is a monomorphism.

      \begin{myIndent}
         \hThree
         Proof: Let's say we have a function $f: A \rightarrow B$.\retTwo

         First, let us assume $f$ is injective.
         
         \begin{myIndent}
            Then let us assume we have two functions $a^{\prime}$ 
            and $a^{\pprime}$ from some set $Z$\\ to $A$ such that 
            $f \circ a^{\prime} = f \circ a^{\pprime}$. Because $f$ is injective, we know it has\\ a left-hand inverse
            $g: B \rightarrow A$ such that $g \circ f = \myId_A$.
            Composing\\ $g$ with the previous equation, we get that:

            {\center $a^{\prime} = \mathrm{Id}_A \circ a^{\prime} = g\circ(f \circ a^{\prime}) = g \circ (f \circ 
            a^{\pprime}) = \mathrm{Id}_A \circ a^{\pprime} = a^{\pprime}$\retTwo\par}

            Thus, we've shown that $f$ is a monomorphism.
         \end{myIndent}
         \newpage

         Next, we shall assume $f$ is a monomorphism.

         \begin{myIndent}
            Based on this, we can say that for any two functions 
            $a^{\prime}$ and $a^{\pprime}$ mapping\\ a set
            $Z$ to $A$, we have that $f \circ a^{\prime}
            = f \circ a^{\pprime} \Longrightarrow a^{\prime} = 
            a^{\pprime}$. However,\\ now note that if we make $Z$ a 
            \udefine{singleton}, meaning it only contains one\\ element, 
            then $a^{\prime}$ and $a^{\pprime}$ can each
            only take on one value. So, we can\\ effectively rewrite
            $f \circ a^{\prime} = f \circ a^{\pprime}
            \Rightarrow a^{\prime} = a^{\pprime}$ as:

            {\center $f(a^{\prime}) = f(a^{\pprime}) \Rightarrow 
            a^{\prime} = a^{\pprime}$\retTwo\par}

            This is the definition of an injective function. $\blacksquare$\retTwo
         \end{myIndent}
      \end{myIndent}
   \end{myIndent}

   \markDate{1/8/2024}

   \hOne
   A function $f: A \rightarrow B$ is an \udefine{epimorphism} 
   (a.k.a an \udefine{epi}) if for all sets $Z$ and all\\ functions $b^{\prime}$
   and $b^{\pprime}: B \rightarrow Z$, we have that 
   $b^{\prime} \circ f = b^{\pprime} \circ f \Rightarrow 
   b^{\prime} = b^{\pprime}$.

   
   \begin{myIndent}
      \hTwo
      Proposition \propCount: A function is a surjection if and
      only if it is an epimorphism.
      
      \hThree
      \begin{myIndent}
         Proof: Let's say we have a function $f: A \rightarrow B$.
         \hfill \bigbreak

         First, let us assume $f$ is surjective.
         \begin{myIndent}
               Then let's assume we have two functions $b^\prime$ and $b^\pprime$ from $B$ to some\\ set $Z$ such that $b^\prime \circ f = b^\pprime \circ f$. Because $f$ is surjective, we know it\\ has a right-hand inverse $h: B \rightarrow A$ such that $f \circ h = \myId_B$. Composing\\ $h$ with the previous equation, we get that:

               {\center $b^{\prime} = b^{\prime} \circ \myId_B = (b^\prime \circ f) \circ h = (b^\pprime \circ f) \circ h = b^\pprime \circ \myId_B = b^\pprime$\retTwo\par}

               So $f$ is an epimorphism.\retTwo
         \end{myIndent}

         Next, assume $f$ is not surjective.
         \begin{myIndent}
            Then there exists $\beta \in B$ such that for all $\alpha \in A$, we have that\\ $f(\alpha) \neq \beta$. Notably, this mean $|B| \geq 1$.\retTwo
            
            If $A = \emptyset$, then define $b^\prime$ to be the function from $B$ to $\{0\}$ and $b^\pprime$ to be\\ the function from $B$ to $\{1\}$. Then, $b^\prime \circ f = f = b^\pprime \circ f$ but $b^\prime \neq b^\pprime$.\retTwo
            
            Meanwhile if $A \neq \emptyset$, then there exists $f(\alpha) \in B \setminus \{\beta\}$. So, $|B| \geq 2$,\\ meaning we can set $b^\prime$ equal to $\myId_B$ and define $b^\pprime$ as a function mapping\\ each element of $B \setminus \{\beta \}$ to itself and $\beta$ to any of the other elements in\\ $B$. Now,
            $b^\prime \circ f = f = b^\pprime \circ f$ but $b^\prime \neq b^\pprime$.\retTwo
            
            Hence, we have shown that $f$ is not an epimorphism.\retTwo
         \end{myIndent}
      \end{myIndent}
   \end{myIndent}

   Sometimes, to indicate that a function $f: A \rightarrow B$ is a monomorphism,\\ epimorphism, or isomorphism, we use the following notation:
   \begin{itemize}
      \item Monomorphism: $f: A \hooklongrightarrow B$
      \item Epimorphism: $f: A \longrightarrowdbl B$
      \item Isomorphism: $f: A \xrightarrow{\phantom{a}\sim\phantom{a}} B$
   \end{itemize}

   \newpage
   \markDate{3/24/2024}

   A \udefine{relation} on a set $S$ is a subset $R$ of the cartesian product $S \times S$. Specifically,\\ we use the notation $x \hspace{0.2em}R\hspace{0.2em} y$ to mean that $(x, y) \in R$. Certain types of relations\\ are especially important and thus are represented with their own symbol.
   \begin{itemize}
      \item An \udefine{equivalence relation}, typically denoted $\hspace{0.1em}\sim$, on a set $S$ has the properties:\\
      \begin{tabular}{l c l c l}
         $\circ$ $\forall a \in S,\myHS a \sim a$ &\quad& $\circ$ $a \sim b \Longrightarrow b \sim a$ &\quad& $\circ$ $a \sim b$ and $b \sim c \Longrightarrow a \sim c$
      \end{tabular}

      \item An \udefine{order relation}, typically denoted $\hspace{0.1em}<$, on a set $S$ has the properties:\\
      \begin{tabular}{l}
         $\circ$ $\forall a, b \in S$, exactly one of the following is true: $a < b$, $b < a$, or $a = b$.\\
         $\circ$ $a < b$ and $b < c$ implies that $a < c$.\retTwo
      \end{tabular}
   \end{itemize}

   Given a set $S$, an equivalence relation $\sim$, and an element $a \in S$, we define the\\ \udefine{equivalence class} of $a$ with respect to $\sim$ to be the set $[a]_\sim = \{b \in S \mid a \sim b\}$.\\ Also, we define the quotient of $S$ with respect to the equivalence relation $\sim$ as\\ the set of equivalence classes with respect to $\sim$.

   {\centering $S/{\sim} = \{[a]_\sim \mid a \in S\}$\retTwo\par}

   \mySepTwo

   Given any function $f: A \longrightarrow B$, define $a \sim b \Longleftrightarrow f(a) = f(b)$.

   {\begin{myIndent} \hTwo
      Proposition \propCount: Every function $f$ can be decomposed as follows:
      % https://q.uiver.app/#q=WzAsNCxbMCwwLCJBIl0sWzEsMCwiKEEvXFxzaW0pIl0sWzIsMCwiXFxtYXRocm17aW19ZiJdLFszLDAsIkIiXSxbMCwxLCJnIiwyLHsic3R5bGUiOnsiaGVhZCI6eyJuYW1lIjoiZXBpIn19fV0sWzEsMiwiXFxzaW0iXSxbMiwzLCIiLDAseyJzdHlsZSI6eyJ0YWlsIjp7Im5hbWUiOiJob29rIiwic2lkZSI6InRvcCJ9fX1dLFswLDMsImYiLDAseyJjdXJ2ZSI6LTN9XV0=
      \[\begin{tikzcd}[column sep=2.25em]
         A & {(A/\sim)} & {\mathrm{im}f} & B
         \arrow["g"', two heads, from=1-1, to=1-2]
         \arrow["\widetilde{f}"', "\sim", from=1-2, to=1-3]
         \arrow["h"', hook, from=1-3, to=1-4]
         \arrow["f", curve={height=-32pt}, from=1-1, to=1-4]
      \end{tikzcd}\]
      {\begin{myTindent}\begin{myTindent}\begin{myTindent} \hFour
         \phantom{.}\\ [-20pt] (in other words, $f = h \circ \widetilde{f} \circ g$)\retTwo
      \end{myTindent}\end{myTindent}\end{myTindent}}
      ...where $g$ is the surjection mapping $a$ to $[a]_\sim$ for all $a \in A$, $h$ is the inclusion function\\ (which is injective) from the  image of $f$ to $B$, and $\widetilde{f}$ is a bijective function defined\\ as the mapping $[a]_\sim$ to $f(a)$ where $a \in [a]_\sim$.\retTwo

      {\begin{myIndent} \hThree
         Proof:\\
         $(A/{\sim})$ is defined as the range of $g$. So $g$ is automatically surjective. Also,\\ inclusion functions like $h$ are always injective.\retTwo

         Now we show $\widetilde{f}$ is well defined and bijective.
         {\begin{myIndent} \hFour
            \begin{enumerate}
               \item Assume $a^\prime, a^\pprime \in A$ such that $[a^\prime] = [a^\pprime]$. Then by how we defined $\sim$,\\ $f(a^\prime) = f(a^\pprime)$. So $[a^\prime] = [a^\pprime] \Longrightarrow \widetilde{f}([a^\prime]) = \widetilde{f}([a^\pprime])$, meaning $\widetilde{f}$ is well\\ defined.
               
               \newpage

               \item Assume $\widetilde{f}([a^\prime]) = \widetilde{f}([a^\pprime])$. Then $f(a^\prime) = f(a^\pprime)$, meaning $a' \sim a^\pprime$.\\ Hence $[a^\prime] = [a^\pprime]$, meaning  $\widetilde{f}$ is injective.\\
               
               \item Given any $b \in \myIm f$, there exists $a \in A$ such that $f(a) = b$. Then\\ $\widetilde{f}([a]_\sim) = f(a) = b$. So $\widetilde{f}$ is surjective.\retTwo
            \end{enumerate}
         \end{myIndent}}

         Finally, it's clear that $f = h \circ \widetilde{f} \circ g$. So we're done.
      \end{myIndent}}
   \end{myIndent}}

   \mySepTwo

   \markDate{3/25/2024}

   A \udefine{category} $\mcateg{C}$ consists of a class $\myObj(\mcateg{C})$ of \udefine{objects} of the category, and for every\\ two objects $A, B$ of $\mcateg{C}$, a set $\myHom_\mcateg{C}(A, B)$ of \udefine{morphisms} with the following\\ properties:
   \begin{itemize}
      \item For every object $A$ of $\mcateg{C}$, there exists a morphism $1_A \in \myHom_\mcateg{C}(A, A)$ called\\ the identity on $A$.
      \item Morphisms can be composed, meaning $f \in \myHom_\mcateg{C}(A, B)$ and $g \in \myHom_\mcateg{C}(B, C)$\\ means that $gf \in \myHom_\mcateg{C}(A, C)$
      \item Composition is associative, meaning if $f \in \myHom_\mcateg{C}(A, B)$, $g \in \myHom_\mcateg{C}(B, C)$,\\ and $h \in \myHom_\mcateg{C}(C, D)$, then $(hg)f = h(gf)$.
      \item The identity morphisms are identities with respect to composition, meaning\\ for all $f \in \myHom_\mcateg{C}(A, B)$, $f1_A = f$ and $1_Bf = f$.
      \item $\myHom_\mcateg{C}(A, B)$ and $\myHom_\mcateg{C}(C, D)$ are disjoint unless $A = C$ and $B = D$.
      
      {\begin{myIndent} \hTwo
         We use the word "class" because by Russell's paradox, there are many sets\\ which aren't well defined. For example, there is no set of sets. So we instead\\ make a class of all sets. \retTwo
         Also, we write category names in sans-serif font to better distinguish them.
      \end{myIndent}}
   \end{itemize}

   A morphism of an object $A$ of a category $\mcateg{C}$ to itself is called an \udefine{endomorphism}.\\ Thus we denote $\myHom_\mcateg{C}(A, A)$ as $\myEnd_\mcateg{C}(A)$.
   {\begin{myIndent} \hTwo
      Note that by the composition rules of a category, if $f, g \in \myEnd_\mcateg{C}(A)$,\\ then $fg, gf \in \myEnd_\mcateg{C}(A)$.\retTwo
   \end{myIndent}}

   We can denote a morphism $f \in \myHom_\mcateg{C}(A, B)$ as $f: A \rightarrow B$.\retTwo

   \exOne
   Examples of Categories:
   \begin{myIndent} \exTwo
      \begin{itemize}
         \item We define the category of sets: $\mcateg{Set}$, such that $\myObj(\mcateg{Set})$ is the class of all sets\\ and for $A$ and $B$ in $\myObj(\mcateg{Set})$, $\myHom_\mcateg{Set}(A, B)$ is the set of all functions from $A$\\ to $B$ (abbreviated as $B^A$).\newpage
         
         \item If $S$ is a set and $\sim$ is an equivalence relation on $S$, then we can define a\\ category whose objects are the elements of $S$, and for $a, b \in S$, $\myHom(a, b)$\\ equals $\{(a, b)\}$ when $a \sim b$ and $\emptyset$ otherwise.
         {\begin{myIndent} \exP
            Note that for this category, we need to define what it means to compose\\ morphisms. So let's say that if $f = \{(a, b)\}$ and $g = \{(b, c)\}$, then\\ $gf = \{(a, c)\}$.\retTwo
         \end{myIndent}}

         \item Let $\mcateg{C}$ be a category and let $A$ be an object of $\mcateg{C}$. Then we can define a category $\mcateg{C}_A$ as follows:
         \begin{itemize}
            \item[$\circ$] $\myObj(\mcateg{C}_A) =$ all morphisms from any object of $\mcateg{C}$ to $A$
            \item[$\circ$] If $f_1: Z_1 \longrightarrow A$ and $f_2: Z_2 \longrightarrow A$ are objects of $\mcateg{C}_A$, then $\myHom_{\mcateg{C}_A}(f, g)$\\ is the set of morphisms $\sigma: Z_1 \rightarrow Z_2$ such that $f_1 = f_2\sigma$.
         \end{itemize}
         {\begin{myDindent} \exP
            Thus the morphisms of $\mcateg{C}_A$ are \ul{commutative diagrams} with the\\ objects $Z_1$, $Z_2$, and $A$.\\ [-12pt]

            \begin{tabular}{p{2.7in} p{2in}}
               \raisebox{-1em}{
               \begin{tabular}{p{2.5in}}
                  To prove that this is a category,\newline first consider that each object\newline $f: Z\longrightarrow A$ has an identity\newline morphism:\newline
               \end{tabular}}
               &
               % https://q.uiver.app/#q=WzAsMyxbMSwyLCJBIl0sWzAsMCwiWiJdLFsyLDAsIloiXSxbMSwwLCJmXzEiLDJdLFsyLDAsImZfMSJdLFsxLDIsIjFfWiJdXQ==
               {\begin{tikzcd}[column sep=tiny]
                  Z && Z \\
                  \\
                  & A
                  \arrow["{f}"', from=1-1, to=3-2]
                  \arrow["{f}", from=1-3, to=3-2]
                  \arrow["{1_Z}", from=1-1, to=1-3]
               \end{tikzcd}}
            \end{tabular}\\
            \begin{tabular}{p{4.5in}}
               \begin{tabular}{p{4.3in}}
                  Also, the morphisms of $\mcateg{C}_A$ compose. If the diagram with $\sigma$ is\newline in $\myHom_{\mcateg{C}_A}(f_1, f_2)$ and the diagram with $\tau$ is in $\myHom_{\mcateg{C}_A}(f_2, f_3)$,\newline then we define their composition in $\myHom_{\mcateg{C_A}}(f_2, f_3)$ as the\newline diagram with the composed morphism $\tau\sigma$ in $\mcateg{C}$.\\ [6pt]
   
                  % https://q.uiver.app/#q=WzAsNCxbMiwyLCJBIl0sWzIsMCwiWl8yIl0sWzAsMCwiWl8xIl0sWzQsMCwiWl8zIl0sWzIsMCwiZl8xIiwyXSxbMSwwLCJmXzIiLDJdLFszLDAsImZfMyJdLFsyLDEsIlxcc2lnbWEiXSxbMSwzLCJcXHRhdSJdXQ==
                  {\begin{tikzcd}[column sep=tiny]
                     {Z_1} && {Z_2} && {Z_3} \\
                     \\
                     && A
                     \arrow["{f_1}"', from=1-1, to=3-3]
                     \arrow["{f_2}"', from=1-3, to=3-3]
                     \arrow["{f_3}", from=1-5, to=3-3]
                     \arrow["\sigma", from=1-1, to=1-3]
                     \arrow["\tau", from=1-3, to=1-5]
                  \end{tikzcd}
                  $\xRightarrow{\phantom{aaaaaaaaaaaaaaa}}$
                  % https://q.uiver.app/#q=WzAsMyxbMiwyLCJBIl0sWzAsMCwiWl8xIl0sWzQsMCwiWl8zIl0sWzEsMCwiZl8xIiwyXSxbMiwwLCJmXzMiXSxbMSwyLCJcXHRhdVxcc2lnbWEiXV0=
                  \begin{tikzcd}[column sep=tiny]
                     {Z_1} &&&& {Z_3} \\
                     \\
                     && A
                     \arrow["{f_1}"', from=1-1, to=3-3]
                     \arrow["{f_3}", from=1-5, to=3-3]
                     \arrow["\tau\sigma", from=1-1, to=1-5]
                  \end{tikzcd}}\retTwo
                  
                  As is hopefully apparent, the identity morphisms compose as is\\ required for $\mcateg{C}_A$ to be a category.\\ [6pt]

                  Finally, composing morphisms of $\mcateg{C}_A$ is associative because\\ $(\upsilon\tau)\sigma = \upsilon(\tau\sigma)$ in the category $\mcateg{C}$.
                  
                  \begin{center}
                     % https://q.uiver.app/#q=WzAsNSxbMywzLCJBIl0sWzAsMCwiWl8xIl0sWzIsMCwiWl8yIl0sWzQsMCwiWl8zIl0sWzYsMCwiWl80Il0sWzEsMCwiZl8xIiwyXSxbMiwwLCJmXzIiLDJdLFszLDAsImZfMyJdLFs0LDAsImZfNCJdLFsxLDIsIlxcc2lnbWEiLDFdLFsyLDMsIlxcdGF1IiwxXSxbMyw0LCJcXHVwc2lsb24iLDFdLFsxLDMsIlxcdGF1XFxzaWdtYSIsMCx7ImN1cnZlIjotM31dLFsyLDQsIlxcdGF1XFx1cHNpbG9uIiwwLHsiY3VydmUiOi0zfV0sWzEsNCwiXFx1cHNpbG9uXFx0YXVcXHNpZ21hIiwwLHsiY3VydmUiOi01fV1d
                     \begin{tikzcd}[column sep=small, row sep=scriptsize]
                        {Z_1} && {Z_2} && {Z_3} && {Z_4} \\
                        \\
                        \\
                        &&& A
                        \arrow["{f_1}"', from=1-1, to=4-4]
                        \arrow["{f_2}", from=1-3, to=4-4]
                        \arrow["{f_3}"', from=1-5, to=4-4]
                        \arrow["{f_4}", from=1-7, to=4-4]
                        \arrow["\sigma", from=1-1, to=1-3]
                        \arrow["\tau", from=1-3, to=1-5]
                        \arrow["\upsilon", from=1-5, to=1-7]
                        \arrow["\tau\sigma", curve={height=-18pt}, from=1-1, to=1-5]
                        \arrow["\upsilon\tau", curve={height=-18pt}, from=1-3, to=1-7]
                        \arrow["\upsilon\tau\sigma", curve={height=-50pt}, from=1-1, to=1-7]
                     \end{tikzcd}
                  \end{center}
               \end{tabular}
            \end{tabular}\retTwo
         \end{myDindent}}

         \newpage

         \item Categories like the one in the previous example are called \ul{slice categories}. We\\ can similarly define \ul{coslice categories} as follows:
         \begin{myIndent}
            Let $\mcateg{C}$ be a category and let $A$ be an object of $\mcateg{C}$. Then we can define\\ a category $\mcateg{C}^A$ such that:
            \begin{itemize}
               \item[$\circ$] $\myObj(\mcateg{C}^A) =$ all morphisms from $A$ to any object of $\mcateg{C}$
               \item[$\circ$] If $f_1: A \longrightarrow Z_1$ and $f_2: A \longrightarrow Z_2$ are objects of $\mcateg{C}^A$, then $\myHom_{\mcateg{C}^A}(f, g)$\\ is the set of morphisms $\sigma: Z_1 \rightarrow Z_2$ such that $\sigma f_1 = f_2$.
            \end{itemize}
            {\begin{myDindent}\exP
               In other words, we're now considering commutative diagrams\\ of the form:
               
               \begin{center}
                  % https://q.uiver.app/#q=WzAsMyxbMSwwLCIgQSJdLFswLDIsIlpfMSJdLFsyLDIsIlpfMiJdLFswLDEsImZfMSIsMl0sWzAsMiwiZl8yIl0sWzEsMiwiXFxzaWdtYSIsMl1d
                  \begin{tikzcd}
                     & { A} \\
                     \\
                     {Z_1} && {Z_2}
                     \arrow["{f_1}"', from=1-2, to=3-1]
                     \arrow["{f_2}", from=1-2, to=3-3]
                     \arrow["\sigma"', from=3-1, to=3-3]
                  \end{tikzcd}
               \end{center}

            \end{myDindent}}
         \end{myIndent}
      \end{itemize}
   \end{myIndent}\retTwo

\hOne
\mySepTwo
{\pracOne\retTwo
   \textbf{Problem 3.8}: A \ul{subcategory} $\mcateg{C}^\prime$ of a category $\mcateg{C}$ consists of a collection of objects of $\mcateg{C}$ with\\ morphisms $\myHom_{\mcateg{C}^\prime}(A, B) \subseteq \myHom_{\mcateg{C}}(A, B)$ for all objects $A, B$ in $\myObj(\mcateg{C^\prime})$ such that $\mcateg{C}^\prime$ has\\ all the necessary identities and compositions to be a category. A subcategory $\mcateg{C}^\prime$ is \ul{full} if\\ $\myHom_{\mcateg{C}^\prime}(A, B) = \myHom_{\mcateg{C}}(A, B)$ for all $A, B$ in $\myObj(\mcateg{C}^\prime)$.\retTwo

   \begin{myIndent}
      \pracTwo
      Let $\mcateg{Set}^\prime$ be the category of infinite sets.
      \begin{itemize}
         \item $\myObj(\mcateg{Set}^\prime)$ is the class of all infinite sets.
         \item For all $A, B$ in $\myObj(\mcateg{Set}^\prime)$, $\myHom_{\mcateg{Set}^\prime}(A, B)$ is the set of all functions from $A$ to $B$.\retTwo
      \end{itemize}
   
      Now given the infinite sets $A$ and $B$, any morphism $f \in \myHom_{\mcateg{Set}}(A, B)$ is also a\\ morphism of $\myHom_{\mcateg{Set^\prime}}(A, B)$. So $\mcateg{Set^\prime}$ is a full subcategory of $\mcateg{Set}$.\retTwo
   \end{myIndent}

   \pracOne\textbf{Problem 3.1}: Let $\mcateg{C}$ be a category. Then consider $\mcateg{C}^{op}$ with
   \begin{itemize}
      \item $\myObj(\mcateg{C}^{op}) = \myObj(\mcateg{C})$
      \item for $A, B$ in $\myObj(\mcateg{C}^{op})$, $\myHom_{\mcateg{C}^{op}}(A, B) = \myHom_{\mcateg{C}}(B, A)$.\\
   \end{itemize}

   \begin{myIndent}
      \pracTwo Let $A$, $B$, and $C$ be objects of $\mcateg{C}^{op}$. Given $g \in \myHom_{\mcateg{C}^{op}}(A, B)$ and $h \in \myHom_{\mcateg{C}^{op}}(B, C)$, define the composition $hg \in \myHom_{\mcateg{C}^{op}}(A, C)$ to be the morphism $gh \in \myHom_{\mcateg{C}}(C, A)$.\retTwo

      To see why this is well defined note that if $g \in \myHom_{\mcateg{C}^{op}}(A, B)$, then $g \in \myHom_{\mcateg{C}}(B, A)$.\\ Similarly, if $h \in \myHom_{\mcateg{C}^{op}}(B, C)$, then $h \in \myHom_{\mcateg{C}}(C, B)$. As $\mcateg{C}$ is a category, there must\\ exist a  morphism $gh \in \myHom_{\mcateg{C}}(C, A)$, which in turn means that the morphism we\\ defined as the composition $hg \in \myHom_{\mcateg{C^{op}}}(A, C)$ exists.
      % https://q.uiver.app/#q=WzAsMyxbMCwxLCJDIl0sWzIsMCwiQiJdLFs0LDAsIkEiXSxbMCwxLCJoIl0sWzEsMiwiZyJdLFswLDIsImdoIiwyLHsiY3VydmUiOjJ9XV0=
      \[\begin{tikzcd}
         && B && A \\
         C
         \arrow["h", from=2-1, to=1-3]
         \arrow["g", from=1-3, to=1-5]
         \arrow["gh"', curve={height=12pt}, from=2-1, to=1-5]
      \end{tikzcd}\]
      
      \newpage

      So by how we defined composition of morphisms in $\mcateg{C}^{op}$, we know $\mcateg{C}^{op}$ satisfies the\\ composition property of a category. Now what's left to show is that $\mcateg{C}^{op}$ has the\\ other properties of a category. \retTwo
      
      For any object $A$ in $\myObj(\mcateg{C}^{op})$, $\myEnd_{\mcateg{C}^{op}}(A) = \myEnd_{\mcateg{C}}(A)$. So, $A$ inherits a morphism $1_A$\\ from $\mcateg{C}$.
      \begin{myIndent}
         Consider $g \in \myHom_{\mcateg{C}^{op}}(A, B)$. Then $g1_A$ in $\myHom_{\mcateg{C}^{op}}(A, B)$ is equal to $1_A g = g$ in\\ $\myHom_{\mcateg{C}}(B, A)$. So in $\mcateg{C}^{op}$, we have that $g1_A = g$.\retTwo

         Similarly, consider $h \in \myHom_{\mcateg{C}^{op}}(B, A)$. Then $1_Ah \in \myHom_{\mcateg{C}^{op}}(B, A)$ is equal\\ to $h1_A = h$ in $\myHom_{\mcateg{C}}(A, B)$. So in $\mcateg{C}^{op}$, we have that $1_Ah = h$.\retTwo
      \end{myIndent}

      Finally, observe that given the morphisms $f \in \myHom_{\mcateg{C}^{op}}(A, B)$, $g \in \myHom_{\mcateg{C}^{op}}(B, C)$, and $h \in \myHom_{\mcateg{C}^{op}}(C, D)$, we know that in $\mcateg{C}$:
      % https://q.uiver.app/#q=WzAsNCxbMCwxLCJEIl0sWzIsMCwiQyJdLFs0LDEsIkIiXSxbNiwwLCJBIl0sWzAsMSwiaCJdLFsxLDIsImciXSxbMiwzLCJmIl0sWzAsMiwiZ2giLDAseyJjdXJ2ZSI6M31dLFsxLDMsImZnIiwwLHsiY3VydmUiOi0zfV1d
      \[\begin{tikzcd}
         && C &&&& A \\
         D &&&& B
         \arrow["h", from=2-1, to=1-3]
         \arrow["g", from=1-3, to=2-5]
         \arrow["f", from=2-5, to=1-7]
         \arrow["gh", curve={height=18pt}, from=2-1, to=2-5]
         \arrow["fg", curve={height=-18pt}, from=1-3, to=1-7]
      \end{tikzcd}\]

      $(gf) \in \myHom_{\mcateg{C}^{op}}(A, C)$ refers to the morphism $fg \in \myHom_{\mcateg{C}}(C, A)$. So,\\ $h(gf)\in \myHom_{\mcateg{C}^{op}}(A, D)$ refers to the morphism $(fg)h \in \myHom_{\mcateg{C}}(D, A)$. At the\\ same time, $(hg) \in \myHom_{\mcateg{C}^{op}}(B, D)$ refers to the morphism $gh \in \myHom_{\mcateg{C}}(D, B)$. So,\\ $(hg)f \in \myHom_{\mcateg{C}}(D, A)$ refers to the morphism $f(gh) \in \myHom_{\mcateg{C}}(D, A)$. Thus as\\ $(fg)h = f(gh)$ in $\mcateg{C}$, we have that $h(gf) = (hg)f$ in $\mcateg{C}^{op}$.
   \end{myIndent}
}

\mySepTwo

\markDate{3/26/2024}

A morphism $f \in \myHom_{\mcateg{C}}(A, B)$ is an \udefine{isomorphism} if it has a two sided inverse under\\ composition (i.e. $\exists g \in \myHom_\mcateg{C}(B, A)$ such that $gf = 1_A$ and $fg = 1_B$).\retTwo


{\begin{myIndent} \hTwo
   Proposition \propCount: The inverse of an isomorphism is unique.
   {\begin{myIndent} \hThree
      Proof:\\
      Suppose $g_1, g_2: B \longrightarrow A$ both act as inverses of $f: A \longrightarrow B$. Then:

      {\centering $ g_1 = g_1 1_B = g_1 (fg_2) = (g_1 f)g_2 = 1_A g_2 = g_2$\retTwo \par}

      
      \begin{myDindent}\hFour
         \begin{myClosureOne}{4.5}
            Corollary: If $f$ has a left-hand inverse $g_1$ and a righthand inverse $g_2$,\\ then $f$ must be an isomorphism and $g_1 = g_2$ must be the unique\\ inverse of $f$.
            \begin{myDindent}
               (Our proof from before also shows this.)
            \end{myDindent}
         \end{myClosureOne}\\ [3pt]
      \end{myDindent}
   \end{myIndent}}

   Since the inverse of $f$ is unique, we denote it $f^{-1}$.

   \newpage

   Proposition \propCount:
   \begin{itemize}
      \item[(A)] Each identity $1_A$ is an isomorphism with itself being its own inverse.
      \item[(B)] If $f$ is an isomorphism, then $f^{-1}$ is an isomorphism and $(f^{-1})^{-1} = f$.
      \item[(C)] If $f \in \myHom_\mcateg{C}(A, B)$ and $g \in \myHom_\mcateg{C}(B, C)$ are isomorphisms, then the\\ composition $gf$ is an isomorphism and $(gf)^{-1} = f^{-1}g^{-1}$.
      
      {\begin{myIndent} \hThree
         To prove any of these, just show that the proposed inverses are in fact\\ an inverse. For example:
         \begin{itemize}
            \item[$\circ$] $1_A 1_A = 1_A$
            \item[$\circ$] $(gf)(f^{-1}g^{-1}) = g(ff^{-1})g^{-1} = g 1_B g^{-1} = g g^{-1} = 1_C$\retTwo
         \end{itemize}
      \end{myIndent}}
   \end{itemize}
\end{myIndent}}

Two objects $A$ and $B$ of a category are \udefine{isomorphic} if there is an isomorphism\\ $f: A \longrightarrow B$. We denote this by writing $A \cong B$.\retTwo

An \udefine{automorphism} of an object $A$ of a category $\mcateg{C}$ is an isomorphism from $A$ to itself. The set of automorphisms of $A$ is denoted $\myAut_\mcateg{C}(A)$.
\begin{myIndent}
   \hTwo
   Note:
   \begin{itemize}
      \item $\myAut_\mcateg{C}(A) \subseteq \myEnd_\mcateg{C}(A)$
      \item If $f, g \in \myAut_\mcateg{C}(A)$, then $fg$ and $gf$ are in $\myAut_\mcateg{C}(A)$.
      \item $1_A \in \myAut_\mcateg{C}(A)$
      \item For each $f \in \myAut_\mcateg{C}(A)$, there exists $f^{-1} \in \myAut_\mcateg{C}(A)$.
      
      {\begin{myTindent}\hFour
         Spoiler: The last three points mean that $\myAut_\mcateg{C}(A)$ forms a group.\retTwo
      \end{myTindent}}
   \end{itemize}
\end{myIndent}

The definitions of surjections and injections don't translate into category theory\\ because the objects of a category don't necessarily have elements. However, the\\ definitions of monomorphisms and epimorphisms do hold in category theory.\retTwo

Let $\mcateg{C}$ be a category and $f: A \rightarrow B$ a morphism.
\begin{myIndent}
   \begin{itemize}
      \item $f$ is a \udefine{monomorphism} if for any object $Z$ of $\mcateg{C}$ and morphisms\\ $\alpha^\prime, \alpha^\pprime \in \myHom_\mcateg{C}(Z, A)$, we have that $f\alpha^\prime = f\alpha^\pprime \Longrightarrow \alpha^\prime = \alpha^\pprime$.\retTwo
      
      \item $f$ is a \udefine{epimorphism} if for any object $Z$ of $\mcateg{C}$ and morphisms\\ $\beta^\prime, \beta^\pprime \in \myHom_\mcateg{C}(B, Z)$, we have that $\beta^\prime f = \beta^\pprime f \Longrightarrow \beta^\prime = \beta^\pprime$.\retTwo
   \end{itemize}
\end{myIndent}

\exOne\mySepTwo

$f$ being both a monomorphism and epimorphism does not necessarily imply that\\ $f$ is isomorphism.
{\begin{myIndent} \exTwo
   For example, consider a category whose objects are all the elements of $\mathbb{Z}$, and\\ where for $a, b \in \mathbb{Z}$, $\myHom(a, b)$ equals $\{(a, b)\}$ if $a \leq b$ and $\emptyset$ otherwise.
   {\begin{myIndent}\exP
      Also we define the composition of $\{(a, b)\}$ and $\{(b, c)\}$ to be $\{(a, c)\}$.
   \end{myIndent}}

   \newpage

   Let $f: a \longrightarrow b$ be a morphism and consider any object $z$ of the category. Since\\ there is only at most one morphism possible in $\myHom(z, a)$, $f$ is automatically a\\ monomorphism. Similarly, $f$ is automatically an epimorphism because there is\\ only at most one morphism possible in $\myHom(b, z)$. That said, the only isomorphisms\\ are the morphisms: $(a, a) \in \myEnd(a)$ for each $a \in \mathbb{Z}$.\retTwo
\end{myIndent}}

Another thing the above category demonstrates is that monomorphisms don't\\ necessarily have left-hand inverses and epimorphisms don't necessarily have\\ right-hand inverses.

\mySepTwo

\pracOne
\textbf{Problem 4.3}: Let $\mcateg{C}$ be a category and $f \in \myHom_\mcateg{C}(A, B)$ be a morphism. Prove that if $f$\\ has a right-inverse, then $f$ is an epimorphism.\retTwo

{\begin{myIndent} \pracTwo
   Assume $f$ has a right-inverse $g: B \longrightarrow A$. Then consider two morphisms\\ $\beta^\prime, \beta^\pprime: B \longrightarrow Z$ such that $\beta^\prime f = \beta^\pprime f$. Thus:

   {\center $ \beta^\prime = \beta^\prime1_B = (\beta^\prime f) g = (\beta^\pprime f) g = \beta^\pprime 1_B = \beta^\pprime$ \retTwo\par}

   
   \begin{myTindent}\begin{myTindent}
      \begin{myClosureOne}{3.35}
         \\ [-22pt] By similar reasoning, we can show that $f$ having a\\ left-inverse implies that $f$ is a monomorphism.\\ [-10pt]
      \end{myClosureOne}
   \end{myTindent}\end{myTindent}
\end{myIndent}}

\textbf{Problem 4.4}:
\begin{itemize}
   \item Prove that the composition of two monomorphisms is a monomorphism.
   {\begin{myIndent}\pracTwo
      Let $f: A \longrightarrow B$ and $g: B \longrightarrow C$ be monomorphisms. Then consider two\\ morphisms $\alpha^\prime, \alpha^\pprime: Z \longrightarrow A$ such that $(gf)\alpha^\prime = (gf)\alpha^\pprime$.
      \begin{myIndent}
         Since $g$ is a monomorphism, $g(f\alpha^\prime) = g(f\alpha^\pprime) \Longrightarrow f\alpha^\prime = f\alpha^\pprime$.\\
         Then as $f$ is a monomorphism, $f\alpha^\prime = f\alpha^\pprime \Longrightarrow \alpha^\prime = \alpha^\pprime$.\\
         So $gf$ is a monomorphism.
      \end{myIndent}
      \begin{myDindent}\begin{myTindent}
         \begin{myClosureOne}{3.35}
            \\ [-22pt] By similar reasoning, we can show that the\\ composition of two epimorphisms is an\\ epimorphism.\\ [-10pt]
         \end{myClosureOne}\\
      \end{myTindent}\end{myDindent}
   \end{myIndent}}

   \item Deduce that we can define a subcategory $\mcateg{C}_{\mathrm{mono}}$ of a category $\mcateg{C}$ such that:
   \begin{itemize}
      \item[$\circ$] $\myObj(\mcateg{C_{\mathrm{mono}}}) = \myObj(\mcateg{C})$
      \item[$\circ$] For each $A, B$ in $\myObj(\mcateg{C_{\mathrm{mono}}}), \myHS \myHom_{\mcateg{C_{\mathrm{mono}}}}(A, B)$ is the subset of $\myHom_{\mcateg{C}}(A, B)$\\ consisting of only monomorphisms.
   \end{itemize}

   {\begin{myIndent}\pracTwo
      Having followed the recipe above for making $\mcateg{C}_{\mathrm{mono}}$, we need to show that $\mcateg{C}_{\mathrm{mono}}$\\ satisfies the properties of a category.\retTwo

      By the previous part of the problem, we know that all morphisms in $C_{\mathrm{mono}}$ compose\\ with each other to give other morphisms in $C_{\mathrm{mono}}$. Also, because morphism\\ composition in $\mcateg{C}$ is associative, we also have that morphism composition in $\mcateg{C}_{\mathrm{mono}}$\\ is associative. So, what we have left to show is that each object in $\mcateg{C}_{\mathrm{mono}}$ has an\\ identity morphism.

      \newpage

      By problem 4.3, we know that isomorphisms are automatically both monomorphisms\\ and epimorphisms because they have both a right-inverse and a left-inverse. This\\ means that since the identity morphisms of $\mcateg{C}$ are isomorphisms, we know that they\\ are also morphisms in $\mcateg{C}_{\mathrm{mono}}$. So each object $A$ in $\myObj(\mcateg{C}_{\mathrm{mono}})$ has an identity\\ morphism $1_A$. Additionally, for every morphism $f: A \longrightarrow B$ in $\mcateg{C}_{\mathrm{mono}}$, we have that\\ $1_B f = f$ and $f 1_A = f$ because that's how those morphisms would compose in $\mcateg{C}$.\retTwo

      So $\mcateg{C}_{\mathrm{mono}}$ satisfies the properties of a category. Hence we conclude that we can define\\ it as a subcategory of $\mcateg{C}$.
   \end{myIndent}}
\end{itemize}

\hOne\mySepTwo

Let $\mcateg{C}$ be a category. We say that an object $I$ of $\mcateg{C}$ is \udefine{initial} in $\mcateg{C}$ if for every object\\ $A$ of $\mcateg{C}$, there exists exactly one morphism $I \longrightarrow A$ in $\mcateg{C}$. Meanwhile, we say that an\\ object $F$ of $\mcateg{C}$ is \udefine{final} in $\mcateg{C}$ if for every object $A$ of $\mcateg{C}$, there exists exactly one morphism\\ $A \longrightarrow F$ in $\mcateg{C}$.
\begin{myIndent}
   One can use the word \udefine{terminal} to describe either $I$ or $F$.
\end{myIndent}

{\begin{center}\exOne
   \begin{myClosureOne}{5}
      \\ [-24pt]
      Examples:
      \begin{myIndent} \exTwo
         In the category $\mcateg{Set}$, $\emptyset$ is initial because there is a single\newline morphism: the empty function $\emptyset$, going from $\emptyset$ to every\newline other set. Also, every other object of $\mcateg{Set}$ is not initial since\newline they all have at least two morphisms towards any set of\newline size 2.\retTwo

         Meanwhile, every singleton $\{a\}$ in the category of $\mcateg{Set}$ is\newline final since for every other set $S$, there is exactly one\newline morphism from $S$ to $\{a\}$. Specifically, that morphism is\newline the function assigning all elements of $S$ to $a$.
      \end{myIndent}
      \\ [-16.4pt]
   \end{myClosureOne}\retTwo
\end{center}}

{\begin{myIndent} \hTwo
   Proposition \propCount: Let $\mcateg{C}$ be a category.
   \begin{itemize}
      \item If $I_1$ and $I_2$ are both initial objects in $\mcateg{C}$, then $I_1 \cong I_2$.
      \item If $F_1$ and $F_2$ are both final objects in $\mcateg{C}$, then $F_1 \cong F_2$.
   \end{itemize}
   Furthermore, these isomorphisms are uniquely determined.

   {\begin{myIndent} \hThree
      Proof:\\
      By the definition of a category, all objects have an identity morphism. So if $F$\\ is final, then the unique morphism $F \longrightarrow F$ must be the identity morphism\\ $1_F$.\retTwo

      Now assume $F_1$ and $F_2$ are both final in $\mcateg{C}$. Then there is a unique morphism\\ $f: F_1 \longrightarrow F_2$ and a unique morphism $g: F_2 \longrightarrow F_1$. Now, $gf$ is a morphism\\ from $F_1$ to $F_1$. So, $gf$ must equal $1_{F_1}$. By similar reasoning, $fg = 1_{F_2}$. This\\ tells us that $g = f^{-1}$ and $f$ is an isomorphism. So $F_1 \cong F_2$.

      \newpage

      The proof for initial objects is entirely analogous. \retTwo
   \end{myIndent}}
\end{myIndent}}

\markDate{3/30/2024}

A big reason we care about category theory is that it allows us to characterize\\ concepts at "satisfying a universal property" or "being the solution to a universal\\ problem" without worrying about the specifics of how we construct that concept.\retTwo

We say a construction satisfies a universal property if it may be viewed as a terminal\\ object of a category. A general outline for what might be said is:
{\begin{myIndent}\hTwo
   "Object $X$ is univeral with respect to the following property: for any $Y$ such that\dots, there exists a unique morphism $Y \rightarrow X$ such that\dots."\retTwo
\end{myIndent}}

Notably, the above statement doesn't say what the category is that $X$ and $Y$ are in.\\ This is normal because category theorists are fucking lazy. In fact, category theorists\\ actually often write even less because they think it's other people's responsibility to\\ try and decode whatever cryptic, terse statements they write.\retTwo

\exOne
Example:\\
Let $\sim$ be an equivalence relation defined on a set $A$. Then "the quotient $A/{\sim}$ is\\ universal with respect to the property of mapping $A$ to a set in such a way that\\ equivalent elements have the same image."\retTwo
\begin{myIndent}\exTwo
   In this statement, the other objects in the category are functions $\varphi: A \longrightarrow S$\\ satisfying that $a^\prime \sim a^\pprime \Longrightarrow \varphi(a^\prime) = \varphi(a^\pprime)$. Thus, we can conclude that the object\\ satisfying the universal property is not literally the quotient $A/{\sim}$ but instead some\\ function from $A$ to $A/{\sim}$.\retTwo

   Now, the objects of the category in above statement are morphisms of the category\\ $\mcateg{Set}$ which are coming out of a specific object $A$ in $\mcateg{Set}$ and satisfying an additional\\ property. Thus, (according to Aluffi) the only category that our above statement\\ could reasonably be alluding to is a subcategory of $\mcateg{C}^A$ (see page 7). Letting $\mcateg{C}$ be\\ the category of the statement above, we have that:
   \begin{itemize}
      \item $\myObj(\mcateg{C}) = $ all functions / morphisms $\varphi$ from $A$ to any object of $\mcateg{Set}$ satisfying\\ that $a^\prime \sim a^\pprime \Longrightarrow \varphi(a^\prime) = \varphi(a^\pprime)$.
      
      \item For $f_1: A \rightarrow S_1$ and $f_2: A \rightarrow S_2$ in $\myObj(\mcateg{C}), \myHS \myHom_{\mcateg{C}}(f_1, f_2) = $ the set of\\ morphisms $\sigma: S_1 \rightarrow S_2$ such that $\sigma \circ f_1 = f_2$.\retTwo
   \end{itemize}

   Finally, what function from $A$ to $A / {\sim}$ could the above statement possibly care about\\ other than the canonical projection $\pi$ mapping $a$ to $[a]_\sim$? So, to prove this statement\\ we now prove that $\pi$ is a terminal object of $\mcateg{C}$.

   \newpage

   {\begin{myIndent}\exP
      Firstly, unless ${\sim} = A \times A$, we do not have that $\pi$ is a final object of $\mcateg{C}$. This is\\ because all the final objects of $\mcateg{C}$ are functions mapping $A$ to a singleton. So, we\\ instead want to show that $\pi$ is an initial object of $\mcateg{C}$. \\ [-2pt]

      Let $\varphi: A \longrightarrow S$ be an object of $\mcateg{C}$. If there is a function $\overbar{\varphi}$ such that $\overbar{\varphi} \circ \pi = \varphi$,\\ [-2pt] then $\overbar{\varphi}([a]_\sim) = \varphi(a)$ for all $[a]_\sim \in A/ {\sim}$. Thus $\overbar{\varphi}$ must be unique if it exists.\\ [-2pt] Meanwhile, $[a_1]_\sim = [a_2]_\sim \Longrightarrow a_1 \sim a_2 \Longrightarrow \varphi(a_1) = \varphi(a_2)$. Hence $\overbar{\varphi}$ is well-\\defined.\retTwo

      So, $\pi$ is an initial object of $\mcateg{C}$.\retTwo
   \end{myIndent}}
\end{myIndent}

As for how the previous statement can be useful, consider a function $f: A \rightarrow B$\\ and equivalence relation $\sim$ defined such that $a_1 \sim a_2 \Longleftrightarrow f(a_1) = f(a_2)$. We can\\ show that the function from $A$ to $\myIm f$ mapping $a$ to $f(a)$ also satisfies the same\\ universal property as the canonical projection from $A$ to $A/{\sim}$. So the two functions\\ must be isomorphic in the category of the statement.
\begin{myTindent}\exP
   In turn, this gives us another proof for proposition 3 on pages 4 and 5.
   % https://q.uiver.app/#q=WzAsNCxbMSwyLCJBIl0sWzAsMCwiQS97XFxzaW19Il0sWzIsMCwiXFxtYXRocm17aW19ZiJdLFs0LDAsIkIiXSxbMCwxLCJcXHBpIl0sWzAsMl0sWzEsMiwiXFx3aWRldGlsZGV7Zn0iLDAseyJjdXJ2ZSI6LTF9XSxbMiwxLCJcXHdpZGV0aWxkZXtmfV57LTF9IiwwLHsiY3VydmUiOi0xfV0sWzIsM11d
   \[\begin{tikzcd}
      {A/{\sim}} && {\mathrm{im}f} && B \\
      \\
      & A
      \arrow[from=3-2, to=1-1]
      \arrow[from=3-2, to=1-3]
      \arrow["{\widetilde{f}}", curve={height=-6pt}, from=1-1, to=1-3]
      \arrow["{\widetilde{f}^{-1}}", curve={height=-6pt}, from=1-3, to=1-1]
      \arrow[from=1-3, to=1-5]
   \end{tikzcd}\]\retTwo
\end{myTindent}

\hOne
\mySepTwo




\end{document}