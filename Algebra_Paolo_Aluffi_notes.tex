% Note for any github stalkers. I am currently in the process
% of learning LaTeX. I don't know what I'm doing yet. Sorry
% if my code absolutely sucks.


\documentclass{book}

\usepackage{fontspec} % used to import Calibri
\usepackage{anyfontsize} % used to adjust font size

% needed for inch and other length measurements
% to be recognized
\usepackage{calc}

% for colors and text effects as is hopefully obvious
\usepackage[dvipsnames]{xcolor}
\usepackage{soul}

% control over margins
\usepackage[margin=1in]{geometry}
\usepackage[strict]{changepage}

\usepackage{mathtools}
\usepackage{amsfonts}
\usepackage{amssymb} % originally imported to get the proof square
\usepackage{xfrac}
\usepackage[overcommands]{overarrows} % Get my preferred vector arrows...
\usepackage{relsize}

% Just am using this to get a dashed line in a table...
% Also you apparently want this to be inactive if you aren't
% using it because it slows compilation.
\usepackage{arydshln} \ADLinactivate 
\newenvironment{allowTableDashes}{\ADLactivate}{\ADLinactivate}

\usepackage{graphicx}
\graphicspath{{./158_Images/}}

\usepackage{tikz}
   \usetikzlibrary{arrows.meta}
   \usetikzlibrary{graphs, graphs.standard}

\usepackage{quiver} %commutative diagrams


\newfontfamily{\calibri}{Calibri}
\setlength{\parindent}{0pt}
\definecolor{RawerSienna}{HTML}{945D27}

% ~~~~~~~~~~~~~~~~~~~~~~~~~~~~~~~~~~~~~~~~~~~~~~~~~~
%Arrow Commands:

% Thank you Bernard, gernot, and Sigur who I copied this from:
% https://tex.stackexchange.com/questions/364096/command-for-longhookrightarrow
\newcommand{\hooklongrightarrow}{\lhook\joinrel\longrightarrow}
\newcommand{\hooklongleftarrow}{\longleftarrow\joinrel\rhook}
\newcommand{\hookxlongrightarrow}[2][]{\lhook\joinrel\xrightarrow[#1]{#2}}
\newcommand{\hookxlongleftarrow}[2][]{\xleftarrow[#1]{#2}\joinrel\rhook}

% Thank you egreg who I copied from:
% https://tex.stackexchange.com/questions/260554/two-headed-version-of-xrightarrow
\newcommand{\longrightarrowdbl}{\longrightarrow\mathrel{\mkern-14mu}\rightarrow}
\newcommand{\longleftarrowdbl}{\leftarrow\mathrel{\mkern-14mu}\longleftarrow}

\newcommand{\xrightarrowdbl}[2][]{%
  \xrightarrow[#1]{#2}\mathrel{\mkern-14mu}\rightarrow
}
\newcommand{\xleftarrowdbl}[2][]{%
  \leftarrow\mathrel{\mkern-14mu}\xleftarrow[#1]{#2}
}




% ~~~~~~~~~~~~~~~~~~~~~~~~~~~~~~~~~~~~~~~~~~~~~~~~~~

\newcommand{\hOne}{%
   \color{Black}%
   \fontsize{14}{16}\selectfont%
}
\newcommand{\hTwo}{%
   \color{MidnightBlue}%
   \fontsize{13}{15}\selectfont%
}
\newcommand{\hThree}{%
   \color{PineGreen!85!Orange}
   \fontsize{13}{15}\selectfont%
}
\newcommand{\hFour}{%
   \color{Cerulean}
   \fontsize{12}{14}\selectfont%
}
\newcommand{\myComment}{%
   \color{RawerSienna}%
   \fontsize{12}{14}\selectfont%
}
\newcommand{\pracOne}{
   \color{BrickRed}%
   \fontsize{13}{15}\selectfont%
}
\newcommand{\pracTwo}{
   \color{Orange}%
   \fontsize{12}{14}\selectfont%
}
\newcommand{\exOne}{%
   \color{Purple}%
   \fontsize{14}{16}\selectfont%
}
\newcommand{\exTwo}{%
   \color{RedViolet}%
   \fontsize{13}{15}\selectfont%
}
\newcommand{\exP}{%
   \color{VioletRed}%
   \fontsize{12}{14}\selectfont%
}
% ~~~~~~~~~~~~~~~~~~~~~~~~~~~~~~~~~~~~~~~~~~~~~~~~

\newcommand{\cyPen}[1]{{\vphantom{.}\color{Cerulean}#1}}

\newenvironment{myIndent}{%
   \begin{adjustwidth}{2.5em}{0em}%
}{%
   \end{adjustwidth}%
}

\newenvironment{myDindent}{%
   \begin{adjustwidth}{5em}{0em}%
}{%
   \end{adjustwidth}%
}

\newenvironment{myTindent}{%
   \begin{adjustwidth}{7.5em}{0em}%
}{%
   \end{adjustwidth}%
}

\newenvironment{myConstrict}{%
   \begin{adjustwidth}{2.5em}{2.5em}%
}{%
   \end{adjustwidth}%
}

\newcommand{\udefine}[1]{{%
   \setulcolor{Red}%
   \setul{0.14em}{0.07em}%
   \ul{#1}%
}}

\newcommand{\uuline}[2][.]{%
{\vphantom{a}\color{#1}%
\rlap{\rule[-0.18em]{\widthof{#2}}{0.06em}}%
\rlap{\rule[-0.32em]{\widthof{#2}}{0.06em}}}%
#2}

\newcommand*{\markDate}[1]{%
   {\huge \color{Black} \textbf{#1} \newline}%
}

\newcommand{\pprime}{{\prime\prime}}
\newcommand{\suchthat}{ \hspace{0.5em}s.t.\hspace{0.5em}}
\newcommand{\rea}[1]{\mathrm{Re}(#1)}
\newcommand{\ima}[1]{\mathrm{Im}(#1)}
\newcommand{\comp}{\mathsf{C}}
\newcommand{\myHS}{ \hspace{0.5em}}

\newcommand{\myId}{\mathrm{Id}}
\newcommand{\myIm}{\mathrm{im}}
\newcommand{\myObj}{\mathrm{Obj}}
\newcommand{\myHom}{\mathrm{Hom}}
\newcommand{\myEnd}{\mathrm{End}}

\newcommand{\mcateg}[1]{\mathsf{#1}}

% Thank you Gonzalo Medina and Moriambar who wrote this on stack exchange:
%https://tex.stackexchange.com/questions/74125/how-do-i-put-text-over-symbols%
\newcommand{\myequiv}[1]{\stackrel{\mathclap{\mbox{\footnotesize{$#1$}}}}{\equiv}}

% Thank you chs who wrote this on stack exchange:
%https://tex.stackexchange.com/questions/89821/how-to-draw-a-solid-colored-circle%
\newcommand{\filledcirc}[1][.]{\ensuremath{\hspace{0.05em}{\color{#1}\bullet}\mathllap{\circ}\hspace{0.05em}}}

%Thank you blerbl who wrote this on stack exchange:
%https://tex.stackexchange.com/questions/25348/latex-symbol-for-does-not-divide
\newcommand{\ndiv}{\hspace{-0.3em}\not|\hspace{0.35em}}

\newcounter{PropNumber}
\newcommand{\propCount}{%
   \stepcounter{PropNumber}%
   \thePropNumber%
}

\newcommand{\mySepOne}[1][.]{%
   {\noindent\color{#1}{\rule{6.5in}{1mm}}}\\%
}
\newcommand{\mySepTwo}[1][.]{%
   {\noindent\color{#1}{\rule{6.5in}{0.5mm}}}\\%
}

\newenvironment{myClosureOne}[2][.]{%
   \color{#1}%
   \begin{tabular}{|p{#2in}|} \hline \\%
}{%
   \\ \hline \end{tabular}%
}

\newcommand{\retTwo}{\hfill\bigbreak}


\title{My Notes on Paolo Aluffi's Algebra Chapter 0}
\author{Isabelle Mills}

\begin{document}
   \maketitle{}
   \setul{0.14em}{0.07em}
   \calibri

   \markDate{1/7/2024}

   \hOne
   A \udefine{multiset} is a collection of elements 
   which like a set is unordered but unlike a set\\ can
   contain duplicate elements.
   
   \begin{myIndent}
      \exOne
      One way to define a multiset is as a function
      \( f: A\rightarrow\mathbb{N} \) such that each
      \(\alpha \in A \)\\ is mapped to the number of times that 
      \(\alpha\) appears in the multiset. Then, given\\ the multisets 
      \(f_{1}: A\rightarrow \mathbb{N}\) and \(f_{2}: 
      B \rightarrow \mathbb{N}\), we can define the following\\
      operations:
      \exTwo
      \begin{itemize}
         \item \(\alpha \in f_{1} \Longleftrightarrow \alpha \in A\)\\ [-10pt]

         \item \(f_{1} \subseteq f_{2} \Longleftrightarrow 
               \forall \alpha \in f_{1}, \hspace{0.25em} \alpha \in f_{2} 
               \text{ and } f_{1}(\alpha) \leq f_{2}(\alpha)\)\\ [-12pt]

         \item \(f_{1} \cup f_{2}: (A \cup B) \longrightarrow
               \mathbb{N} \) such that for \(\alpha
               \in A \cup B\), if \(\alpha \in A \cap B \),
               then\\ \((f_{1} \cup f_{2})(\alpha) = f_{1}(\alpha)
               + f_{2}(\alpha)\). As for if \(\alpha \notin 
               A \cap B \), then \((f_{1} \cup f_{2})(\alpha)\)\\
               equals whatever $\alpha$ was mapped to 
               in the multiset it originally came from.\\ [-12pt]
         
         \item \(f_{1} \cap f_{2}: (A \cap B) \longrightarrow
               \mathbb{N} \) such that for \(\alpha
               \in A \cap B\), we have that\\
               \((f_{1} \cap f_{2})(\alpha) = \min(f_{1}(\alpha)
               , f_{2}(\alpha))\)\\ [-12pt]
         
         \item \(f_{1} \setminus f_{2}: ((A \setminus B)
               \cup \{\alpha \in A \cap B \mid f_{1}(\alpha) >
               f_{2}(\alpha)\}) \longrightarrow \mathbb{N} \) 
               such that for\\ each \(\alpha \in f_{1} \setminus f_{2}\), 
               if \(\alpha \in f_{2} \), then \((f_{1} \setminus f_{2})(\alpha) 
               = f_{1}(\alpha) - f_{2}(\alpha)\). As for if\\ \(\alpha 
               \notin f_{2} \), then \((f_{1} \setminus f_{2})(\alpha) = 
               f_{1}(\alpha)\)\\ [-12pt]
         
      \end{itemize}
   
      \begin{myIndent}\begin{myIndent}\begin{myIndent}
      \begin{myIndent}\begin{myIndent}
         \myComment
         \hfill \break
         A practical example of a multiset is the prime
         \\factorization of any positive integer.
      \end{myIndent}\end{myIndent}\end{myIndent}
      \end{myIndent}\end{myIndent}
   
   \end{myIndent}

   \hrulefill

   \hOne
   \hfill \break
   We say that two sets $A$ and $B$ are \udefine{isomorphic} if and only
   if there exists a bijection\\ between $A$ and $B$. We denote this by
   writing $A \cong B$. Additionally, we can refer\\ to any bijection $f$
   between $A$ and $B$ as an \udefine{isomorphism} between the two sets.

   \hfill \break
   A function $f: A \rightarrow B$ is a \udefine{monomorphism} (a.k.a 
   a \udefine{monic}) if for all sets $Z$ and all\\ functions $a^{\prime}$
   and $a^{\pprime}: Z \rightarrow A$, we have that $f \circ a^{\prime}
   = f \circ a^{\pprime} \Longrightarrow a^{\prime} = a^{\pprime}$.

   % Note to future self, I just made it so that if you
   % type ctrl + shift + e followed by space, then it will
   % surround a highlighted text with begin{}...end{}
   \begin{myIndent}
      \hTwo
      Proposition \propCount: A function is injective if and only if
      it is a monomorphism.

      \begin{myIndent}
         \hThree
         Proof: Let's say we have a function $f: A \rightarrow B$.\retTwo

         First, let us assume $f$ is injective.
         
         \begin{myIndent}
            Then let us assume we have two functions $a^{\prime}$ 
            and $a^{\pprime}$ from some set $Z$\\ to $A$ such that 
            $f \circ a^{\prime} = f \circ a^{\pprime}$. Because $f$ is injective, we know it has\\ a left-hand inverse
            $g: B \rightarrow A$ such that $g \circ f = \myId_A$.
            Composing\\ $g$ with the previous equation, we get that:

            {\center $a^{\prime} = \mathrm{Id}_A \circ a^{\prime} = g\circ(f \circ a^{\prime}) = g \circ (f \circ 
            a^{\pprime}) = \mathrm{Id}_A \circ a^{\pprime} = a^{\pprime}$\retTwo\par}

            Thus, we've shown that $f$ is a monomorphism.
         \end{myIndent}
         \newpage

         Next, we shall assume $f$ is a monomorphism.

         \begin{myIndent}
            Based on this, we can say that for any two functions 
            $a^{\prime}$ and $a^{\pprime}$ mapping\\ a set
            $Z$ to $A$, we have that $f \circ a^{\prime}
            = f \circ a^{\pprime} \Longrightarrow a^{\prime} = 
            a^{\pprime}$. However,\\ now note that if we make $Z$ a 
            \udefine{singleton}, meaning it only contains one\\ element, 
            then $a^{\prime}$ and $a^{\pprime}$ can each
            only take on one value. So, we can\\ effectively rewrite
            $f \circ a^{\prime} = f \circ a^{\pprime}
            \Rightarrow a^{\prime} = a^{\pprime}$ as:

            {\center $f(a^{\prime}) = f(a^{\pprime}) \Rightarrow 
            a^{\prime} = a^{\pprime}$\retTwo\par}

            This is the definition of an injective function. $\blacksquare$\retTwo
         \end{myIndent}
      \end{myIndent}
   \end{myIndent}

   \markDate{1/8/2024}

   \hOne
   A function $f: A \rightarrow B$ is an \udefine{epimorphism} 
   (a.k.a an \udefine{epi}) if for all sets $Z$ and all\\ functions $a^{\prime}$
   and $a^{\pprime}: B \rightarrow Z$, we have that 
   $a^{\prime} \circ f = a^{\pprime} \circ f \Rightarrow 
   a^{\prime} = a^{\pprime}$.

   
   \begin{myIndent}
      \hTwo
      Proposition \propCount: A function is a surjection if and
      only if it is an epimorphism.
      
      \hThree
      \begin{myIndent}
         Proof: Let's say we have a function $f: A \rightarrow B$.
         \hfill \bigbreak

         First, let us assume $f$ is surjective.
         \begin{myIndent}
               Then let's assume we have two functions $a^\prime$ and $a^\pprime$ from $B$ to some\\ set $Z$ such that $a^\prime \circ f = a^\pprime \circ f$. Because $f$ is surjective, we know it\\ has a right-hand inverse $h: B \rightarrow A$ such that $f \circ h = \myId_B$. Composing\\ $h$ with the previous equation, we get that:

               {\center $a^{\prime} = a^{\prime} \circ \myId_B = (a^\prime \circ f) \circ h = (a^\pprime \circ f) \circ h = a^\pprime \circ \myId_B = a^\pprime$\retTwo\par}

               So $f$ is an epimorphism.\retTwo
         \end{myIndent}

         Next, assume $f$ is not surjective.
         \begin{myIndent}
            Then there exists $\beta \in B$ such that for all $\alpha \in A$, we have that\\ $f(\alpha) \neq \beta$. Importantly, as $f(\alpha) \in B$, we know $|B| \neq 1$. So set $a^\prime$\\ equal to $\myId_B$ and define $a^\pprime$ as a function mapping each element\\ of $B \setminus \{\beta \}$ to itself and $\beta$ to any of the other elements in $B$. Now,\\
            $a^\prime \circ f = f = a^\pprime \circ f$ but $a^\prime \neq a^\pprime$. So $f$ is not epimorphic. $\blacksquare$\retTwo
         \end{myIndent}
      \end{myIndent}
   \end{myIndent}

   Sometimes, to indicate that a function $f: A \rightarrow B$ is a monomorphism,\\ epimorphism, or isomorphism, we use the following notation:
   \begin{itemize}
      \item Monomorphism: $f: A \hooklongrightarrow B$
      \item Epimorphism: $f: A \longrightarrowdbl B$
      \item Isomorphism: $f: A \xrightarrow{\phantom{a}\sim\phantom{a}} B$
   \end{itemize}

   \newpage
   \markDate{3/24/2024}

   A \udefine{relation} on a set $S$ is a subset $R$ of the cartesian product $S \times S$. Specifically,\\ we use the notation $x \hspace{0.2em}R\hspace{0.2em} y$ to mean that $(x, y) \in R$. Certain types of relations\\ are especially important and thus are represented with their own symbol.
   \begin{itemize}
      \item An \udefine{equivalence relation}, typically denoted $\hspace{0.1em}\sim$, on a set $S$ has the properties:\\
      \begin{tabular}{l c l c l}
         $\circ$ $\forall a \in S,\myHS a \sim a$ &\quad& $\circ$ $a \sim b \Longrightarrow b \sim a$ &\quad& $\circ$ $a \sim b$ and $b \sim c \Longrightarrow a \sim c$
      \end{tabular}

      \item An \udefine{order relation}, typically denoted $\hspace{0.1em}<$, on a set $S$ has the properties:\\
      \begin{tabular}{l}
         $\circ$ $\forall a, b \in S$, exactly one of the following is true: $a < b$, $b < a$, or $a = b$.\\
         $\circ$ $a < b$ and $b < c$ implies that $a < c$.\retTwo
      \end{tabular}
   \end{itemize}

   Given a set $S$, an equivalence relation $\sim$, and an element $a \in S$, we define the\\ \udefine{equivalence class} of $a$ with respect to $\sim$ to be the set $[a]_\sim = \{b \in S \mid a \sim b\}$.\\ Also, we define the quotient of $S$ with respect to the equivalence relation $\sim$ as\\ the set of equivalence classes with respect to $\sim$.

   {\centering $S/{\sim} = \{[a]_\sim \mid a \in S\}$\retTwo\par}

   \mySepTwo

   Given any function $f: A \longrightarrow B$, define $a \sim b \Longleftrightarrow f(a) = f(b)$.

   {\begin{myIndent} \hTwo
      Proposition \propCount: Every function $f$ can be decomposed as follows:
      % https://q.uiver.app/#q=WzAsNCxbMCwwLCJBIl0sWzEsMCwiKEEvXFxzaW0pIl0sWzIsMCwiXFxtYXRocm17aW19ZiJdLFszLDAsIkIiXSxbMCwxLCJnIiwyLHsic3R5bGUiOnsiaGVhZCI6eyJuYW1lIjoiZXBpIn19fV0sWzEsMiwiXFxzaW0iXSxbMiwzLCIiLDAseyJzdHlsZSI6eyJ0YWlsIjp7Im5hbWUiOiJob29rIiwic2lkZSI6InRvcCJ9fX1dLFswLDMsImYiLDAseyJjdXJ2ZSI6LTN9XV0=
      \[\begin{tikzcd}[column sep=2.25em]
         A & {(A/\sim)} & {\mathrm{im}f} & B
         \arrow["g"', two heads, from=1-1, to=1-2]
         \arrow["\widetilde{f}"', "\sim", from=1-2, to=1-3]
         \arrow["h"', hook, from=1-3, to=1-4]
         \arrow["f", curve={height=-32pt}, from=1-1, to=1-4]
      \end{tikzcd}\]
      {\begin{myTindent}\begin{myTindent}\begin{myTindent} \hFour
         \phantom{.}\\ [-20pt] (in other words, $f = h \circ \widetilde{f} \circ g$)\retTwo
      \end{myTindent}\end{myTindent}\end{myTindent}}
      ...where $g$ is the surjection mapping $a$ to $[a]_\sim$ for all $a \in A$, $h$ is the inclusion function\\ (which is injective) from the  image of $f$ to $B$, and $\widetilde{f}$ is a bijective function defined\\ as the mapping $[a]_\sim$ to $f(a)$ where $a \in [a]_\sim$.\retTwo

      {\begin{myIndent} \hThree
         Proof:\\
         $(A/{\sim})$ is defined as the range of $g$. So $g$ is automatically surjective. Also,\\ inclusion functions like $h$ are always injective.\retTwo

         Now we show $\widetilde{f}$ is well defined and bijective.
         {\begin{myIndent} \hFour
            \begin{enumerate}
               \item Assume $a^\prime, a^\pprime \in A$ such that $[a^\prime] = [a^\pprime]$. Then by how we defined $\sim$,\\ $f(a^\prime) = f(a^\pprime)$. So $[a^\prime] = [a^\pprime] \Longrightarrow \widetilde{f}([a^\prime]) = \widetilde{f}([a^\pprime])$, meaning $\widetilde{f}$ is well\\ defined.
               
               \newpage

               \item Assume $\widetilde{f}([a^\prime]) = \widetilde{f}([a^\pprime])$. Then $f(a^\prime) = f(a^\pprime)$, meaning $a' \sim a^\pprime$.\\ Hence $[a^\prime] = [a^\pprime]$, meaning  $\widetilde{f}$ is injective.\\
               
               \item Given any $b \in \myIm f$, there exists $a \in A$ such that $f(a) = b$. Then\\ $\widetilde{f}([a]_\sim) = f(a) = b$. So $\widetilde{f}$ is surjective.\retTwo
            \end{enumerate}
         \end{myIndent}}

         Finally, it's clear that $f = h \circ \widetilde{f} \circ g$. So we're done.
      \end{myIndent}}
   \end{myIndent}}

   \mySepTwo

   \markDate{3/25/2024}

   A \udefine{category} $\mcateg{C}$ consists of a class $\myObj(\mcateg{C})$ of \udefine{objects} of the category, and for every\\ two objects $A, B$ of $\mcateg{C}$, a set $\myHom_\mcateg{C}(A, B)$ of \udefine{morphisms} with the following\\ properties:
   \begin{itemize}
      \item For every object $A$ of $\mcateg{C}$, there exists a morphism $1_A \in \myHom_\mcateg{C}(A, A)$ called\\ the identity on $A$.
      \item Morphisms can be composed, meaning $f \in \myHom_\mcateg{C}(A, B)$ and $g \in \myHom_\mcateg{C}(B, C)$\\ means that $gf \in \myHom_\mcateg{C}(A, C)$
      \item Composition is associative, meaning if $f \in \myHom_\mcateg{C}(A, B)$, $g \in \myHom_\mcateg{C}(B, C)$,\\ and $h \in \myHom_\mcateg{C}(C, D)$, then $(hg)f = h(gf)$.
      \item The identity morphisms are identities with respect to composition, meaning\\ for all $f \in \myHom_\mcateg{C}(A, B)$, $f1_A = f$ and $1_Bf = f$.
      \item $\myHom_\mcateg{C}(A, B)$ and $\myHom_\mcateg{C}(C, D)$ are disjoint unless $A = C$ and $B = D$.
      
      {\begin{myIndent} \hTwo
         We use the word "class" because by Russell's paradox, there are many sets\\ which aren't well defined. For example, there is no set of sets. So we instead\\ make a class of all sets. \retTwo
         Also, we write category names in sans-serif font to better distinguish them.
      \end{myIndent}}
   \end{itemize}

   A morphism of an object $A$ of a category $\mcateg{C}$ to itself is called an \udefine{endomorphism}.\\ Thus we denote $\myHom_\mcateg{C}(A, A)$ as $\myEnd_\mcateg{C}(A)$.
   {\begin{myIndent} \hTwo
      Note that by the composition rules of a category, if $f, g \in \myEnd_\mcateg{C}(A)$,\\ then $fg, gf \in \myEnd_\mcateg{C}(A)$.\retTwo
   \end{myIndent}}

   We can denote a morphism $f \in \myHom_\mcateg{C}(A, B)$ as $f: A \rightarrow B$.\retTwo

   \exOne
   Examples of Categories:
   \begin{myIndent} \exTwo
      \begin{itemize}
         \item We define the category of sets: $\mcateg{Set}$, such that $\myObj(\mcateg{Set})$ is the class of all sets\\ and for $A$ and $B$ in $\myObj(\mcateg{Set})$, $\myHom_\mcateg{Set}(A, B)$ is the set of all functions from $A$\\ to $B$ (abbreviated as $B^A$).\newpage
         
         \item If $S$ is a set and $\sim$ is an equivalence relation on $S$, then we can define a\\ category whose objects are the elements of $S$, and for $a, b \in S$, $\myHom(a, b)$\\ equals $\{a, b\}$ when $a \sim b$ and $\emptyset$ otherwise.
         {\begin{myIndent} \exP
            Note that for this category, we need to define what it means to compose\\ morphisms. So let's say that if $f = \{a, b\}$ and $g = \{b, c\}$, then $gf = \{a, c\}$.\retTwo
         \end{myIndent}}

         \item Let $\mcateg{C}$ be a category and let $A$ be an object of $\mcateg{C}$. Then we can define a category $\mcateg{C}_A$ as follows:
         \begin{itemize}
            \item[$\circ$] $\myObj(\mcateg{C}_A) =$ all morphisms from any object of $\mcateg{C}$ to $A$
            \item[$\circ$] If $f_1: Z_1 \longrightarrow A$ and $f_2: Z_2 \longrightarrow A$ are objects of $\mcateg{C}_A$, then $\myHom_{\mcateg{C}_A}(f, g)$\\ is the set of morphisms $\sigma: Z_1 \rightarrow Z_2$ such that $f_1 = f_2\sigma$.
         \end{itemize}
         {\begin{myDindent} \exP
            Thus the morphisms of $\mcateg{C}_A$ are \ul{commutative diagrams} with the\\ objects $Z_1$, $Z_2$, and $A$.\\ [-12pt]

            \begin{tabular}{p{2.7in} p{2in}}
               \raisebox{-1em}{
               \begin{tabular}{p{2.5in}}
                  To prove that this is a category,\newline first consider that each object\newline $f: Z\longrightarrow A$ has an identity\newline morphism:\newline
               \end{tabular}}
               &
               % https://q.uiver.app/#q=WzAsMyxbMSwyLCJBIl0sWzAsMCwiWiJdLFsyLDAsIloiXSxbMSwwLCJmXzEiLDJdLFsyLDAsImZfMSJdLFsxLDIsIjFfWiJdXQ==
               {\begin{tikzcd}[column sep=tiny]
                  Z && Z \\
                  \\
                  & A
                  \arrow["{f}"', from=1-1, to=3-2]
                  \arrow["{f}", from=1-3, to=3-2]
                  \arrow["{1_Z}", from=1-1, to=1-3]
               \end{tikzcd}}
            \end{tabular}\retTwo
            \begin{tabular}{p{4.5in}}
               \begin{tabular}{p{4.3in}}
                  Also, the morphisms of $\mcateg{C}_A$ compose. If the diagram with $\sigma$ is\newline in $\myHom_{\mcateg{C}_A}(f_1, f_2)$ and the diagram with $\tau$ is in $\myHom_{\mcateg{C}_A}(f_2, f_3)$,\newline then we define their composition in $\myHom_{\mcateg{C_A}}(f_2, f_3)$ as the\newline diagram with $\tau\sigma$.\retTwo
   
                  % https://q.uiver.app/#q=WzAsNCxbMiwyLCJBIl0sWzIsMCwiWl8yIl0sWzAsMCwiWl8xIl0sWzQsMCwiWl8zIl0sWzIsMCwiZl8xIiwyXSxbMSwwLCJmXzIiLDJdLFszLDAsImZfMyJdLFsyLDEsIlxcc2lnbWEiXSxbMSwzLCJcXHRhdSJdXQ==
                  {\begin{tikzcd}[column sep=tiny]
                     {Z_1} && {Z_2} && {Z_3} \\
                     \\
                     && A
                     \arrow["{f_1}"', from=1-1, to=3-3]
                     \arrow["{f_2}"', from=1-3, to=3-3]
                     \arrow["{f_3}", from=1-5, to=3-3]
                     \arrow["\sigma", from=1-1, to=1-3]
                     \arrow["\tau", from=1-3, to=1-5]
                  \end{tikzcd}
                  $\xRightarrow{\phantom{aaaaaaaaaaaaaaa}}$
                  % https://q.uiver.app/#q=WzAsMyxbMiwyLCJBIl0sWzAsMCwiWl8xIl0sWzQsMCwiWl8zIl0sWzEsMCwiZl8xIiwyXSxbMiwwLCJmXzMiXSxbMSwyLCJcXHRhdVxcc2lnbWEiXV0=
                  \begin{tikzcd}[column sep=tiny]
                     {Z_1} &&&& {Z_3} \\
                     \\
                     && A
                     \arrow["{f_1}"', from=1-1, to=3-3]
                     \arrow["{f_3}", from=1-5, to=3-3]
                     \arrow["\tau\sigma", from=1-1, to=1-5]
                  \end{tikzcd}}\retTwo
                  
                  As is hopefully apparent, the identity morphisms compose as is required for $\mcateg{C}_A$ to be a category.\retTwo

                  Finally, composing morphisms of $\mcateg{C}_A$ is associative.
                  
                  \begin{center}
                     % https://q.uiver.app/#q=WzAsNSxbMywzLCJBIl0sWzAsMCwiWl8xIl0sWzIsMCwiWl8yIl0sWzQsMCwiWl8zIl0sWzYsMCwiWl80Il0sWzEsMCwiZl8xIiwyXSxbMiwwLCJmXzIiLDJdLFszLDAsImZfMyJdLFs0LDAsImZfNCJdLFsxLDIsIlxcc2lnbWEiLDFdLFsyLDMsIlxcdGF1IiwxXSxbMyw0LCJcXHVwc2lsb24iLDFdLFsxLDMsIlxcdGF1XFxzaWdtYSIsMCx7ImN1cnZlIjotM31dLFsyLDQsIlxcdGF1XFx1cHNpbG9uIiwwLHsiY3VydmUiOi0zfV0sWzEsNCwiXFx1cHNpbG9uXFx0YXVcXHNpZ21hIiwwLHsiY3VydmUiOi01fV1d
                     \begin{tikzcd}[column sep=small]
                        {Z_1} && {Z_2} && {Z_3} && {Z_4} \\
                        \\
                        \\
                        &&& A
                        \arrow["{f_1}"', from=1-1, to=4-4]
                        \arrow["{f_2}", from=1-3, to=4-4]
                        \arrow["{f_3}"', from=1-5, to=4-4]
                        \arrow["{f_4}", from=1-7, to=4-4]
                        \arrow["\sigma", from=1-1, to=1-3]
                        \arrow["\tau", from=1-3, to=1-5]
                        \arrow["\upsilon", from=1-5, to=1-7]
                        \arrow["\tau\sigma", curve={height=-18pt}, from=1-1, to=1-5]
                        \arrow["\upsilon\tau", curve={height=-18pt}, from=1-3, to=1-7]
                        \arrow["\upsilon\tau\sigma", curve={height=-50pt}, from=1-1, to=1-7]
                     \end{tikzcd}
                  \end{center}
               \end{tabular}
            \end{tabular}\retTwo
         \end{myDindent}}

         \newpage

         \item Categories like the one in the previous example are called \ul{slice categories}. We\\ can similarly define \ul{coslice categories} as follows:
         \begin{myIndent}
            Let $\mcateg{C}$ be a category and let $A$ be an object of $\mcateg{C}$. Then we can define\\ a category $\mcateg{C}^A$ such that:
            \begin{itemize}
               \item[$\circ$] $\myObj(\mcateg{C}^A) =$ all morphisms from $A$ to any object of $\mcateg{C}$
               \item[$\circ$] If $f_1: A \longrightarrow Z_1$ and $f_2: A \longrightarrow Z_2$ are objects of $\mcateg{C}^A$, then $\myHom_{\mcateg{C}^A}(f, g)$\\ is the set of morphisms $\sigma: Z_1 \rightarrow Z_2$ such that $\sigma f_1 = f_2$.
            \end{itemize}
            {\begin{myDindent}\exP
               In other words, we're now considering commutative diagrams\\ of the form:
               
               \begin{center}
                  % https://q.uiver.app/#q=WzAsMyxbMSwwLCIgQSJdLFswLDIsIlpfMSJdLFsyLDIsIlpfMiJdLFswLDEsImZfMSIsMl0sWzAsMiwiZl8yIl0sWzEsMiwiXFxzaWdtYSIsMl1d
                  \begin{tikzcd}
                     & { A} \\
                     \\
                     {Z_1} && {Z_2}
                     \arrow["{f_1}"', from=1-2, to=3-1]
                     \arrow["{f_2}", from=1-2, to=3-3]
                     \arrow["\sigma"', from=3-1, to=3-3]
                  \end{tikzcd}
               \end{center}

            \end{myDindent}}
         \end{myIndent}
      \end{itemize}
   \end{myIndent}\retTwo

\hOne
\mySepTwo
{\pracOne\retTwo
   \textbf{Problem 3.8}: A \ul{subcategory} $\mcateg{C}^\prime$ of a category $\mcateg{C}$ consists of a collection of objects of $\mcateg{C}$ with\\ morphisms $\myHom_{\mcateg{C}^\prime}(A, B) \subseteq \myHom_{\mcateg{C}}(A, B)$ for all objects $A, B$ in $\myObj(\mcateg{C^\prime})$ such that $\mcateg{C}^\prime$ has\\ all the necessary identities and compositions to be a category. A subcategory $\mcateg{C}^\prime$ is \ul{full} if\\ $\myHom_{\mcateg{C}^\prime}(A, B) = \myHom_{\mcateg{C}}(A, B)$ for all $A, B$ in $\myObj(\mcateg{C}^\prime)$.\retTwo

   \begin{myIndent}
      \pracTwo
      Let $\mcateg{Set}^\prime$ be the category of infinite sets.
      \begin{itemize}
         \item $\myObj(\mcateg{Set}^\prime)$ is the class of all infinite sets.
         \item For all $A, B$ in $\myObj(\mcateg{Set}^\prime)$, $\myHom_{\mcateg{Set}^\prime}(A, B)$ is the set of all functions from $A$ to $B$.\retTwo
      \end{itemize}
   
      Now given the infinite sets $A$ and $B$, any morphism $f \in \myHom_{\mcateg{Set}}(A, B)$ is also a\\ morphism of $\myHom_{\mcateg{Set^\prime}}(A, B)$. So $\mcateg{Set^\prime}$ is a full subcategory of $\mcateg{Set}$.\retTwo
   \end{myIndent}

   \pracOne\textbf{Problem 3.1}: Let $\mcateg{C}$ be a category. Then consider $\mcateg{C}^{op}$ with
   \begin{itemize}
      \item $\myObj(\mcateg{C}^{op}) = \myObj(\mcateg{C})$
      \item for $A, B$ in $\myObj(\mcateg{C}^{op})$, $\myHom_{\mcateg{C}^{op}}(A, B) = \myHom_{\mcateg{C}}(B, A)$.\\
   \end{itemize}

   \begin{myIndent}
      \pracTwo Let $A$, $B$, and $C$ be objects of $\mcateg{C}^{op}$. Given $g \in \myHom_{\mcateg{C}^{op}}(A, B)$ and $h \in \myHom_{\mcateg{C}^{op}}(B, C)$, define the composition $hg \in \myHom_{\mcateg{C}^{op}}(A, C)$ to be the morphism $gh \in \myHom_{\mcateg{C}}(C, A)$.\retTwo

      To see why this is well defined note that if $g \in \myHom_{\mcateg{C}^{op}}(A, B)$, then $g \in \myHom_{\mcateg{C}}(B, A)$.\\ Similarly, if $h \in \myHom_{\mcateg{C}^{op}}(B, C)$, then $h \in \myHom_{\mcateg{C}}(C, B)$. As $\mcateg{C}$ is a category, there must\\ exist a  morphism $gh \in \myHom_{\mcateg{C}}(C, A)$, which in turn means that the morphism we\\ defined as the composition $hg \in \myHom_{\mcateg{C^{op}}}(A, C)$ exists.
      % https://q.uiver.app/#q=WzAsMyxbMCwxLCJDIl0sWzIsMCwiQiJdLFs0LDAsIkEiXSxbMCwxLCJoIl0sWzEsMiwiZyJdLFswLDIsImdoIiwyLHsiY3VydmUiOjJ9XV0=
      \[\begin{tikzcd}
         && B && A \\
         C
         \arrow["h", from=2-1, to=1-3]
         \arrow["g", from=1-3, to=1-5]
         \arrow["gh"', curve={height=12pt}, from=2-1, to=1-5]
      \end{tikzcd}\]
      
      \newpage

      So by how we defined composition of morphisms in $\mcateg{C}^{op}$, we know $\mcateg{C}^{op}$ satisfies the\\ composition property of a category. Now what's left to show is that $\mcateg{C}^{op}$ has the\\ other properties of a category. \retTwo
      
      For any object $A$ in $\myObj(\mcateg{C}^{op})$, $\myEnd_{\mcateg{C}^{op}}(A) = \myEnd_{\mcateg{C}}(A)$. So, $A$ inherits a morphism $1_A$\\ from $\mcateg{C}$.
      \begin{myIndent}
         Consider $g \in \myHom_{\mcateg{C}^{op}}(A, B)$. Then $g1_A$ in $\myHom_{\mcateg{C}^{op}}(A, B)$ is equal to $1_A g = g$ in\\ $\myHom_{\mcateg{C}}(B, A)$. So in $\mcateg{C}^{op}$, we have that $g1_A = g$.\retTwo

         Similarly, consider $h \in \myHom_{\mcateg{C}^{op}}(B, A)$. Then $1_Ah \in \myHom_{\mcateg{C}^{op}}(B, A)$ is equal\\ to $h1_A = h$ in $\myHom_{\mcateg{C}}(A, B)$. So in $\mcateg{C}^{op}$, we have that $1_Ah = h$.\retTwo
      \end{myIndent}

      Finally, observe that given the morphisms $f \in \myHom_{\mcateg{C}^{op}}(A, B)$, $g \in \myHom_{\mcateg{C}^{op}}(B, C)$, and $h \in \myHom_{\mcateg{C}^{op}}(C, D)$, we know that in $\mcateg{C}$:
      % https://q.uiver.app/#q=WzAsNCxbMCwxLCJEIl0sWzIsMCwiQyJdLFs0LDEsIkIiXSxbNiwwLCJBIl0sWzAsMSwiaCJdLFsxLDIsImciXSxbMiwzLCJmIl0sWzAsMiwiZ2giLDAseyJjdXJ2ZSI6M31dLFsxLDMsImZnIiwwLHsiY3VydmUiOi0zfV1d
      \[\begin{tikzcd}
         && C &&&& A \\
         D &&&& B
         \arrow["h", from=2-1, to=1-3]
         \arrow["g", from=1-3, to=2-5]
         \arrow["f", from=2-5, to=1-7]
         \arrow["gh", curve={height=18pt}, from=2-1, to=2-5]
         \arrow["fg", curve={height=-18pt}, from=1-3, to=1-7]
      \end{tikzcd}\]

      $(gf) \in \myHom_{\mcateg{C}^{op}}(A, C)$ refers to the morphism $fg \in \myHom_{\mcateg{C}}(C, A)$. So,\\ $h(gf)\in \myHom_{\mcateg{C}^{op}}(A, D)$ refers to the morphism $(fg)h \in \myHom_{\mcateg{C}}(D, A)$. At the\\ same time, $(hg) \in \myHom_{\mcateg{C}^{op}}(B, D)$ refers to the morphism $gh \in \myHom_{\mcateg{C}}(D, B)$. So,\\ $(hg)f \in \myHom_{\mcateg{C}}(D, A)$ refers to the morphism $f(gh) \in \myHom_{\mcateg{C}}(D, A)$. Thus as\\ $(fg)h = f(gh)$ in $\mcateg{C}$, we have that $(hg)f = h(gf)$ in $\mcateg{C}^{op}$.
   \end{myIndent}
}

\mySepTwo

\markDate{3/26/2024}

\end{document}