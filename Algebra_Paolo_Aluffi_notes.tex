% Note for any github stalkers. I am currently in the process
% of learning LaTeX. I don't know what I'm doing yet. Sorry
% if my code absolutely sucks.


\documentclass{book}

\usepackage{fontspec} % used to import Calibri
\usepackage{anyfontsize} % used to adjust font size

% needed for inch and other length measurements
% to be recognized
\usepackage{calc}

% for colors and text effects as is hopefully obvious
\usepackage[dvipsnames]{xcolor}
\usepackage{soul}

% control over margins
\usepackage[margin=1in]{geometry}
\usepackage[strict]{changepage}

\usepackage{mathtools}
\usepackage{amsfonts}
\usepackage{amssymb} % originally imported to get the proof square
\usepackage{xfrac}
\usepackage[overcommands]{overarrows} % Get my preferred vector arrows...
\usepackage{relsize}

% Just am using this to get a dashed line in a table...
% Also you apparently want this to be inactive if you aren't
% using it because it slows compilation.
\usepackage{arydshln} \ADLinactivate 
\newenvironment{allowTableDashes}{\ADLactivate}{\ADLinactivate}

\usepackage{graphicx}
\graphicspath{{./158_Images/}}

\usepackage{tikz}
   \usetikzlibrary{arrows.meta}
   \usetikzlibrary{graphs, graphs.standard}

\newfontfamily{\calibri}{Calibri}
\setlength{\parindent}{0pt}
\definecolor{RawerSienna}{HTML}{945D27}

% ~~~~~~~~~~~~~~~~~~~~~~~~~~~~~~~~~~~~~~~~~~~~~~~~~~
%Arrow Commands:

% Thank you Bernard, gernot, and Sigur who I copied this from:
% https://tex.stackexchange.com/questions/364096/command-for-longhookrightarrow
\newcommand{\Hooklongrightarrow}{\lhook\joinrel\longrightarrow}
\newcommand{\Hooklongleftarrow}{\longleftarrow\joinrel\rhook}
\newcommand{\Hookxlongrightarrow}[2][]{\lhook\joinrel\xrightarrow[#1]{#2}}
\newcommand{\Hookxlongleftarrow}[2][]{\xleftarrow[#1]{#2}\joinrel\rhook}

% Thank you egreg who I copied from:
% https://tex.stackexchange.com/questions/260554/two-headed-version-of-xrightarrow
\newcommand{\longrightarrowdbl}{\longrightarrow\mathrel{\mkern-14mu}\rightarrow}
\newcommand{\longleftarrowdbl}{\leftarrow\mathrel{\mkern-14mu}\longleftarrow}

\newcommand{\xrightarrowdbl}[2][]{%
  \xrightarrow[#1]{#2}\mathrel{\mkern-14mu}\rightarrow
}
\newcommand{\xleftarrowdbl}[2][]{%
  \leftarrow\mathrel{\mkern-14mu}\xleftarrow[#1]{#2}
}




% ~~~~~~~~~~~~~~~~~~~~~~~~~~~~~~~~~~~~~~~~~~~~~~~~~~

\newcommand{\hOne}{%
   \color{Black}%
   \fontsize{14}{16}\selectfont%
}
\newcommand{\hTwo}{%
   \color{MidnightBlue}%
   \fontsize{13}{15}\selectfont%
}
\newcommand{\hThree}{%
   \color{PineGreen!85!Orange}
   \fontsize{13}{15}\selectfont%
}
\newcommand{\hFour}{%
   \color{Cerulean}
   \fontsize{12}{14}\selectfont%
}
\newcommand{\myComment}{%
   \color{RawerSienna}%
   \fontsize{12}{14}\selectfont%
}
\newcommand{\teachComment}{
   \color{Orange}%
   \fontsize{12}{14}\selectfont%
}
\newcommand{\exOne}{%
   \color{Purple}%
   \fontsize{14}{16}\selectfont%
}
\newcommand{\exTwo}{%
   \color{RedViolet}%
   \fontsize{13}{15}\selectfont%
}
\newcommand{\exP}{%
   \color{VioletRed}%
   \fontsize{12}{14}\selectfont%
}
% ~~~~~~~~~~~~~~~~~~~~~~~~~~~~~~~~~~~~~~~~~~~~~~~~

\newcommand{\cyPen}[1]{{\vphantom{.}\color{Cerulean}#1}}

\newenvironment{myIndent}{%
   \begin{adjustwidth}{2.5em}{0em}%
}{%
   \end{adjustwidth}%
}

\newenvironment{myDindent}{%
   \begin{adjustwidth}{5em}{0em}%
}{%
   \end{adjustwidth}%
}

\newenvironment{myTindent}{%
   \begin{adjustwidth}{7.5em}{0em}%
}{%
   \end{adjustwidth}%
}

\newenvironment{myConstrict}{%
   \begin{adjustwidth}{2.5em}{2.5em}%
}{%
   \end{adjustwidth}%
}

\newcommand{\udefine}[1]{{%
   \setulcolor{Red}%
   \setul{0.14em}{0.07em}%
   \ul{#1}%
}}

\newcommand{\uuline}[2][.]{%
{\vphantom{a}\color{#1}%
\rlap{\rule[-0.18em]{\widthof{#2}}{0.06em}}%
\rlap{\rule[-0.32em]{\widthof{#2}}{0.06em}}}%
#2}

\newcommand*{\markDate}[1]{%
   {\huge \color{Black} \textbf{#1} \newline}%
}

\newcommand{\pprime}{{\prime\prime}}
\newcommand{\suchthat}{ \hspace{0.5em}s.t.\hspace{0.5em}}
\newcommand{\rea}[1]{\mathrm{Re}(#1)}
\newcommand{\ima}[1]{\mathrm{Im}(#1)}
\newcommand{\comp}{\mathsf{C}}
\newcommand{\myHS}{ \hspace{0.5em}}

\newcommand{\myId}{\mathrm{Id}}

% Thank you Gonzalo Medina and Moriambar who wrote this on stack exchange:
%https://tex.stackexchange.com/questions/74125/how-do-i-put-text-over-symbols%
\newcommand{\myequiv}[1]{\stackrel{\mathclap{\mbox{\footnotesize{$#1$}}}}{\equiv}}

% Thank you chs who wrote this on stack exchange:
%https://tex.stackexchange.com/questions/89821/how-to-draw-a-solid-colored-circle%
\newcommand{\filledcirc}[1][.]{\ensuremath{\hspace{0.05em}{\color{#1}\bullet}\mathllap{\circ}\hspace{0.05em}}}

%Thank you blerbl who wrote this on stack exchange:
%https://tex.stackexchange.com/questions/25348/latex-symbol-for-does-not-divide
\newcommand{\ndiv}{\hspace{-0.3em}\not|\hspace{0.35em}}

\newcounter{PropNumber}
\newcommand{\propCount}{%
   \stepcounter{PropNumber}%
   \thePropNumber%
}

\newcommand{\mySepOne}[1][.]{%
   {\noindent\color{#1}{\rule{6.5in}{1mm}}}\\%
}
\newcommand{\mySepTwo}[1][.]{%
   {\noindent\color{#1}{\rule{6.5in}{0.5mm}}}\\%
}

\newenvironment{myClosureOne}[2][.]{%
   \color{#1}%
   \begin{tabular}{|p{#2in}|} \hline \\%
}{%
   \\ \hline \end{tabular}%
}

\newcommand{\retTwo}{\hfill\bigbreak}


\title{My Notes on Paolo Aluffi's Algebra Chapter 0}
\author{Isabelle Mills}

\begin{document}
   \maketitle{}
   \calibri

   \markDate{1/7/2024}

   \hOne
   A \udefine{multiset} is a collection of elements 
   which like a set is unordered but unlike a set\\ can
   contain duplicate elements.
   
   \begin{myIndent}
      \hTwo
      One way to define a multiset is as a function
      \( f: A\rightarrow\mathbb{N} \) such that each
      \(\alpha \in A \)\\ is mapped to the number of times that 
      \(\alpha\) appears in the multiset. Then, given the\\ multisets 
      \(f_{1}: A\rightarrow \mathbb{N}\) and \(f_{2}: 
      B \rightarrow \mathbb{N}\), we can define the following
      operations:
      \hThree
      \begin{itemize}
         \item \(\alpha \in f_{1} \Longleftrightarrow \alpha \in A\)\\ [-10pt]

         \item \(f_{1} \subseteq f_{2} \Longleftrightarrow 
               \forall \alpha \in f_{1}, \hspace{0.25em} \alpha \in f_{2} 
               \text{ and } f_{1}(\alpha) \leq f_{2}(\alpha)\)\\ [-12pt]

         \item \(f_{1} \cup f_{2}: (A \cup B) \longrightarrow
               \mathbb{N} \) such that for \(\alpha
               \in A \cup B\), if \(\alpha \in A \cap B \),
               then\\ \((f_{1} \cup f_{2})(\alpha) = f_{1}(\alpha)
               + f_{2}(\alpha)\). As for if \(\alpha \notin 
               A \cap B \), then \((f_{1} \cup f_{2})(\alpha)\)\\
               equals whatever $\alpha$ was mapped to 
               in the multiset it originally came from.\\ [-12pt]
         
         \item \(f_{1} \cap f_{2}: (A \cap B) \longrightarrow
               \mathbb{N} \) such that for \(\alpha
               \in A \cap B\), we have that\\
               \((f_{1} \cap f_{2})(\alpha) = \min(f_{1}(\alpha)
               , f_{2}(\alpha))\)\\ [-12pt]
         
         \item \(f_{1} \setminus f_{2}: ((A \setminus B)
               \cup \{\alpha \in A \cap B \mid f_{1}(\alpha) >
               f_{2}(\alpha)\}) \longrightarrow \mathbb{N} \) 
               such that for\\ each \(\alpha \in f_{1} \setminus f_{2}\), 
               if \(\alpha \in f_{2} \), then \((f_{1} \setminus f_{2})(\alpha) 
               = f_{1}(\alpha) - f_{2}(\alpha)\). As for if\\ \(\alpha 
               \notin f_{2} \), then \((f_{1} \setminus f_{2})(\alpha) = 
               f_{1}(\alpha)\)\\ [-12pt]
         
      \end{itemize}
   
      \begin{myIndent}\begin{myIndent}\begin{myIndent}
      \begin{myIndent}\begin{myIndent}
         \myComment
         \hfill \break
         A practical example of a multiset is the prime
         \\factorization of any positive integer.
      \end{myIndent}\end{myIndent}\end{myIndent}
      \end{myIndent}\end{myIndent}
   
   \end{myIndent}

   \hrulefill

   \hOne
   \hfill \break
   We say that two sets $A$ and $B$ are \udefine{isomorphic} if and only
   if there exists a bijection\\ between $A$ and $B$. We denote this by
   writing $A \cong B$. Additionally, we can refer\\ to any bijection $f$
   between $A$ and $B$ as an \udefine{isomorphism} between the two sets.

   \hfill \break
   A function $f: A \rightarrow B$ is a \udefine{monomorphism} (a.k.a 
   a \udefine{monic}) if for all sets $Z$ and all\\ functions $a^{\prime}$
   and $a^{\pprime}: Z \rightarrow A$, we have that $f \circ a^{\prime}
   = f \circ a^{\pprime} \Longrightarrow a^{\prime} = a^{\pprime}$.

   % Note to future self, I just made it so that if you
   % type ctrl + shift + e followed by space, then it will
   % surround a highlighted text with begin{}...end{}
   \begin{myIndent}
      \hTwo
      Proposition \propCount: A function is injective if and only if
      it is a monomorphism.

      \begin{myIndent}
         \hThree
         Proof: Let's say we have a function $f: A \rightarrow B$.\retTwo

         First, let us assume $f$ is injective.
         
         \begin{myIndent}
            Then let us assume we have two functions $a^{\prime}$ 
            and $a^{\pprime}$ from some set $Z$\\ to $A$ such that 
            $f \circ a^{\prime} = f \circ a^{\pprime}$. Because $f$ is injective, we know it has\\ a left-hand inverse
            $g: B \rightarrow A$ such that $g \circ f = \myId_A$.
            Composing\\ $g$ with the previous equation, we get that:

            {\center $a^{\prime} = \mathrm{Id}_A \circ a^{\prime} = g\circ(f \circ a^{\prime}) = g \circ (f \circ 
            a^{\pprime}) = \mathrm{Id}_A \circ a^{\pprime} = a^{\pprime}$\retTwo\par}

            Thus, we've shown that $f$ is a monomorphism.
         \end{myIndent}
         \newpage

         Next, we shall assume $f$ is a monomorphism.

         \begin{myIndent}
            Based on this, we can say that for any two functions 
            $a^{\prime}$ and $a^{\pprime}$ mapping\\ a set
            $Z$ to $A$, we have that $f \circ a^{\prime}
            = f \circ a^{\pprime} \Longrightarrow a^{\prime} = 
            a^{\pprime}$. However,\\ now note that if we make $Z$ a 
            \udefine{singleton}, meaning it only contains one\\ element, 
            then $a^{\prime}$ and $a^{\pprime}$ can each
            only take on one value. So, we can\\ effectively rewrite
            $f \circ a^{\prime} = f \circ a^{\pprime}
            \Rightarrow a^{\prime} = a^{\pprime}$ as:

            {\center $f(a^{\prime}) = f(a^{\pprime}) \Rightarrow 
            a^{\prime} = a^{\pprime}$\retTwo\par}

            This is the definition of an injective function. $\blacksquare$\retTwo
         \end{myIndent}
      \end{myIndent}
   \end{myIndent}

   \markDate{3/23/2024}

   \hOne
   A function $f: A \rightarrow B$ is an \udefine{epimorphism} 
   (a.k.a an \udefine{epi}) if for all sets $Z$ and all\\ functions $a^{\prime}$
   and $a^{\pprime}: B \rightarrow Z$, we have that 
   $a^{\prime} \circ f = a^{\pprime} \circ f \Rightarrow 
   a^{\prime} = a^{\pprime}$.

   
   \begin{myIndent}
      \hTwo
      Proposition \propCount: A function is a surjection if and
      only if it is an epimorphism.
      
      \hThree
      \begin{myIndent}
         Proof: Let's say we have a function $f: A \rightarrow B$.
         \hfill \bigbreak

         First, let us assume $f$ is surjective.
         \begin{myIndent}
               Then let's assume we have two functions $a^\prime$ and $a^\pprime$ from $B$ to some\\ set $Z$ such that $a^\prime \circ f = a^\pprime \circ f$. Because $f$ is surjective, we know it\\ has a right-hand inverse $h: B \rightarrow A$ such that $f \circ h = \myId_B$. Composing\\ $h$ with the previous equation, we get that:

               {\center $a^{\prime} = a^{\prime} \circ \myId_B = (a^\prime \circ f) \circ h = (a^\pprime \circ f) \circ h = a^\pprime \circ \myId_B = a^\pprime$\retTwo\par}

               So $f$ is an epimorphism.\retTwo
         \end{myIndent}

         Next, assume $f$ is not surjective.
         \begin{myIndent}
            Then there exists $\beta \in B$ such that for all $\alpha \in A$, we have that\\ $f(\alpha) \neq \beta$. Importantly, as $f(\alpha) \in B$, we know $|B| \neq 1$. So set $a^\prime$\\ equal to $\myId_B$ and define $a^\pprime$ as a function mapping each element\\ of $B \setminus \{\beta \}$ to itself and $\beta$ to any of the other elements in $B$. Now,\\
            $a^\prime \circ f = f = a^\pprime \circ f$ but $a^\prime \neq a^\pprime$. So $f$ is not epimorphic. $\blacksquare$\retTwo
         \end{myIndent}
      \end{myIndent}
   \end{myIndent}

   
   \begin{tabular}{p{2in} p{4in}}
      $\Hooklongrightarrow$ 
      $\Hooklongleftarrow$ 
      $\Hookxlongrightarrow{hellow}$ 
      $\Hookxlongleftarrow{hellow}$\newline
      $\longrightarrowdbl$ 
      $\longleftarrowdbl$ 
      $\xrightarrowdbl{hellow}$ 
      $\xleftarrowdbl{hellow}$ &

      aaaaaaaaaaaaaaaaaa\newline
      aaaaaaaaaaaaaaaaaa\\ \\

      aaaaaaaaaaaaaaaaaa\newline
      aaaaaaaaaaaaaaaaaa &

      aaaaaaaaaaaaaaaaaa\newline
      aaaaaaaaaaaaaaaaaa
   \end{tabular}\\

   aaaaaaaaa\\
   aaaaaaaaa


   \newpage

   A \udefine{relation} on a set $S$ is a subset $R$ of the cartesian product $S \times S$. Specifically, we\\ use the notation $x \hspace{0.2em}R\hspace{0.2em} y$ to mean that $(x, y) \in R$. Certain types of relations are\\ especially important and thus are represented with their own symbol.
   \begin{itemize}
      \item An \udefine{equivalence relation}, typically denoted $\sim$ on a set $S$ has the properties:\\
      \begin{tabular}{l c l c l}
         $\circ$ $\forall x \in S,\myHS x \sim x$ &\quad& $\circ$ $x \sim y \Longrightarrow y \sim x$ &\quad& $\circ$ $x \sim y$ and $y \sim z \Longrightarrow x \sim z$
      \end{tabular}

      \item An \udefine{order relation}, typically denoted $<$ on a set $S$ has the properties:\\
      \begin{tabular}{l}
         $\circ$ $\forall x, y \in S$, exactly one of the following is true: $x < y$, $y < x$, or $x = y$.\\
         $\circ$ $x < y$ and $y < z$ implies that $x < z$.
      \end{tabular}
   \end{itemize}

\end{document}


(also\\ written as \(f: A \xlongrightarrow{\sim} B\))