\documentclass{book}

\usepackage{fontspec} % used to import Calibri
\usepackage{anyfontsize} % used to adjust font size

% needed for inch and other length measurements
% to be recognized
\usepackage{calc}

% for colors and text effects as is hopefully obvious
\usepackage[dvipsnames]{xcolor}
\usepackage{soul}

% control over margins
\usepackage[margin=1in]{geometry}
\usepackage[strict]{changepage}

\usepackage{mathtools}
\usepackage{amsfonts}
\usepackage{bm}

\usepackage[scr=rsfso, scrscaled=.96]{mathalpha}

% This is how I'm getting the nice caligraphy font :(
\DeclareMathAlphabet{\eulerscr}{U}{eus}{m}{n}
\newcommand{\mathcalli}[1]{\text{\scalebox{1.11}{$\eulerscr{#1}$}}}


\usepackage{amssymb} % originally imported to get the proof square
\usepackage{xfrac}
\usepackage[overcommands]{overarrows} % Get my preferred vector arrows...
\usepackage{relsize}

% Just am using this to get a dashed line in a table...
% Also you apparently want this to be inactive if you aren't
% using it because it slows compilation.
\usepackage{arydshln} \ADLinactivate 
\newenvironment{allowTableDashes}{\ADLactivate}{\ADLinactivate}

\usepackage{graphicx}
\graphicspath{{./158_Images/}}

\usepackage{tikz}
   \usetikzlibrary{arrows.meta}
   \usetikzlibrary{graphs, graphs.standard}

\usepackage{quiver} %commutative diagrams






\usepackage[hidelinks]{hyperref}
\newcommand{\inLinkRap}[2]{{\color{blue}\hyperlink{#1}{\textit{#2}}}}







\newfontfamily{\calibri}{Calibri}
\setlength{\parindent}{0pt}
\definecolor{RawerSienna}{HTML}{945D27}

% ~~~~~~~~~~~~~~~~~~~~~~~~~~~~~~~~~~~~~~~~~~~~~~~~~~
%Arrow Commands:

% Thank you Bernard, gernot, and Sigur who I copied this from:
% https://tex.stackexchange.com/questions/364096/command-for-longhookrightarrow
\renewcommand{\hookrightarrow}{\lhook\joinrel\rightarrow}
\renewcommand{\hookleftarrow}{\leftarrow\joinrel\rhook}
\newcommand{\hooklongrightarrow}{\lhook\joinrel\longrightarrow}
\newcommand{\hooklongleftarrow}{\longleftarrow\joinrel\rhook}
\newcommand{\hookxlongrightarrow}[2][]{\lhook\joinrel\xrightarrow[#1]{#2}}
\newcommand{\hookxlongleftarrow}[2][]{\xleftarrow[#1]{#2}\joinrel\rhook}

% Thank you egreg who I copied from:
% https://tex.stackexchange.com/questions/260554/two-headed-version-of-xrightarrow
\newcommand{\longrightarrowdbl}{\longrightarrow\mathrel{\mkern-14mu}\rightarrow}
\newcommand{\longleftarrowdbl}{\leftarrow\mathrel{\mkern-14mu}\longleftarrow}

\newcommand{\xrightarrowdbl}[2][]{%
  \xrightarrow[#1]{#2}\mathrel{\mkern-14mu}\rightarrow
}
\newcommand{\xleftarrowdbl}[2][]{%
  \leftarrow\mathrel{\mkern-14mu}\xleftarrow[#1]{#2}
}

\newcommand{\mRoman}[1]{%
   \textrm{\MakeUppercase{\romannumeral #1}}%
}



% ~~~~~~~~~~~~~~~~~~~~~~~~~~~~~~~~~~~~~~~~~~~~~~~~~~

\newcommand{\hOne}{%
   \color{Black}%
   \fontsize{14}{16}\selectfont%
}
\newcommand{\hTwo}{%
\color{Black}%
   \fontsize{13}{15}\selectfont%
}
% \newcommand{\scratchWork}{%
%    \color{PineGreen!85!Orange}
%    \fontsize{12}{14}\selectfont%
% }
\newcommand{\hThree}{%
   \color{Black}%
   \fontsize{12}{14}\selectfont%
}
\newcommand{\myComment}{%
   \color{RawerSienna}%
   \fontsize{12}{14}\selectfont%
}
\newcommand{\pracOne}{
   \color{BrickRed}%
   \fontsize{13}{15}\selectfont%
}
\newcommand{\pracTwo}{
   \color{Orange}%
   \fontsize{12}{14}\selectfont%
}
\newcommand{\why}{%
   \color{Orange}%
   \fontsize{12}{14}\selectfont%
	Why:
}
\newcommand{\exOne}{%
   \color{Purple}%
   \fontsize{14}{16}\selectfont%
}
\newcommand{\exTwo}{%
   \color{Purple}%
   \fontsize{13}{15}\selectfont%
}
\newcommand{\exThree}{%
   \color{Purple}%
   \fontsize{12}{14}\selectfont%
}
\newcommand{\exP}{%
   \color{Purple}%
   \fontsize{12}{14}\selectfont%
}
\newcommand{\exTwoP}{%
   \color{RedViolet}%
   \fontsize{13}{15}\selectfont%
}
\newcommand{\exThreeP}{%
   \color{RedViolet}%
   \fontsize{12}{14}\selectfont%
}
\newcommand{\exFourP}{%
   \color{RedViolet}%
   \fontsize{11}{13}\selectfont%
}
\newcommand{\exPP}{%
   \color{RedViolet}%
   \fontsize{12}{14}\selectfont%
}
\newcommand{\exPPP}{%
   \color{VioletRed}%
   \fontsize{12}{14}\selectfont%
}

% Homework standard below (God the bloat in the header is absurd...)
% ~~~~~~~~~~~~~~~~~~~~~~~~~~~~~~~~~~~~~~~~~~~~~~~~
\newcommand{\Hstatement}{%
   \color{MidnightBlue!90!Black}%
   \fontsize{12}{13}\selectfont%
}
\newcommand{\HexOne}{%
   \color{Purple}%
   \fontsize{12}{13}\selectfont%
}
\newcommand{\HexTwoP}{%
   \color{RedViolet}%
   \fontsize{12}{13}\selectfont%
}
\newcommand{\HexPPP}{%
   \color{VioletRed}%
   \fontsize{11}{12}\selectfont%
}

% ~~~~~~~~~~~~~~~~~~~~~~~~~~~~~~~~~~~~~~~~~~~~~~~~

\newcommand{\cyPen}[1]{{\vphantom{.}\color{Cerulean}#1}}
\newcommand{\redPen}[1]{{\vphantom{.}\color{Red}#1}}

\newenvironment{myIndent}{%
   \begin{adjustwidth}{2.5em}{0em}%
}{%
   \end{adjustwidth}%
}

\newenvironment{myDindent}{%
   \begin{adjustwidth}{5em}{0em}%
}{%
   \end{adjustwidth}%
}

\newenvironment{myTindent}{%
   \begin{adjustwidth}{7.5em}{0em}%
}{%
   \end{adjustwidth}%
}

\newenvironment{myConstrict}{%
   \begin{adjustwidth}{2.5em}{2.5em}%
}{%
   \end{adjustwidth}%
}

\newcommand{\udefine}[1]{{%
   \setulcolor{Red}%
   \setul{0.14em}{0.07em}%
   \ul{#1}%
}}

\newcommand{\uprop}[1]{{%
   \setulcolor{Purple}%
   \setul{0.14em}{0.07em}%
   \ul{#1} 
}}

\newcommand{\blab}[1]{\textbf{#1}}
\newcommand{\blect}[1]{{\color{MidnightBlue}\textbf{#1}}}

\newcommand{\uuline}[2][.]{%
{\vphantom{a}\color{#1}%
\rlap{\rule[-0.18em]{\widthof{#2}}{0.06em}}%
\rlap{\rule[-0.32em]{\widthof{#2}}{0.06em}}}%
#2}

\newcommand{\pprime}{{\prime\prime}}
\newcommand{\suchthat}{ \hspace{0.3em}s.t.\hspace{0.3em}}
\newcommand{\rea}[1]{\mathrm{Re}(#1)}
\newcommand{\ima}[1]{\mathrm{Im}(#1)}
\newcommand{\comp}{\mathsf{C}}
\newcommand{\trans}{\mathsf{T}}
\newcommand{\myHS}{ \hspace{0.5em}}
\newcommand{\gap}{\phantom{2}}

\newcommand{\GenLin}{\ensuremath{\mathrm{GL}}}
\newcommand{\Cay}{\ensuremath{\mathrm{Cay}}}

\newcommand{\myId}{\mathrm{Id}}
\newcommand{\myIm}{\mathrm{im}}
\newcommand{\Obj}{\mathrm{Obj}}
\newcommand{\Hom}{\mathrm{Hom}}
\newcommand{\End}{\mathrm{End}}
\newcommand{\Aut}{\mathrm{Aut}}

\newcommand{\df}{\mathrm{d}}
\newcommand{\Df}{\mathrm{D}}

\newcommand{\mcateg}[1]{{\bm{\mathsf{#1}}}}

\newcommand{\mdeg}{\mathrm{mdeg}\phantom{.}}

\newcommand{\divides}{\mathop{\mid}}

\newcommand{\card}{\mathrm{card}}
\newcommand{\supp}{\mathrm{supp}}
\newcommand{\diam}{\mathrm{diam}}
\newcommand{\conv}{\mathrm{conv}}
\newcommand{\opnorm}{\mathrm{op}}
\newcommand{\loc}{\mathrm{loc}}
\newcommand{\sgn}{\mathrm{sgn}}
\newcommand{\acc}{\mathrm{acc}}

\newcommand{\mSpan}{\mathrm{span}}
\newcommand{\Interior}{\mathop{\mathrm{Int}}}

\newcommand{\mMat}[1]{\mathbf{#1}}

\newcommand{\NBV}{\ensuremath{\mathrm{NBV}}}
\newcommand{\Acc}{\mathrm{Acc}}
\newcommand{\BV}{\ensuremath{\mathrm{BV}}}
\newcommand{\Var}{\ensuremath{\mathrm{Var}}}

\newcommand{\Alt}{\mathrm{Alt}}
\newcommand{\Sym}{\mathrm{Sym}}

\newcommand{\weakst}{weak$^*$ }

\newcommand{\radtimes}{\mathop{\widehat{\times}}}

\newcommand{\mMod}[1]{\phantom{a}(\mathrel{\mathrm{mod}} #1)}
\newcommand{\Fun}{\mathrm{Fun}}
\newcommand{\act}{\mathrm{act}}
\newcommand{\Fix}{\mathrm{Fix}}

\DeclareMathOperator{\lcm}{lcm}



% Thank you Gonzalo Medina and Moriambar who wrote this on stack exchange:
%https://tex.stackexchange.com/questions/74125/how-do-i-put-text-over-symbols%
\newcommand{\myequiv}[1]{\stackrel{\mathclap{\mbox{\footnotesize{$#1$}}}}{\equiv}}

% Thank you chs who wrote this on stack exchange:
%https://tex.stackexchange.com/questions/89821/how-to-draw-a-solid-colored-circle%
\newcommand{\filledcirc}[1][.]{\ensuremath{\hspace{0.05em}{\color{#1}\bullet}\mathllap{\circ}\hspace{0.05em}}}

%Thank you blerbl who wrote this on stack exchange:
%https://tex.stackexchange.com/questions/25348/latex-symbol-for-does-not-divide
\newcommand{\ndiv}{\hspace{-0.3em}\not|\hspace{0.35em}}

\newcommand{\mySepOne}[1][.]{%
   {\noindent\color{#1}{\rule{6.5in}{1mm}}}\\%
}
\newcommand{\mySepTwo}[1][.]{%
   {\noindent\color{#1}{\rule{6.5in}{0.5mm}}}\\%
}
\newcommand{\mySepThree}[1][.]{%
   {\noindent\color{#1}{\rule{6in}{0.25mm}}}\\%
}

\newenvironment{myClosureOne}[2][.]{%
   \color{#1}%
   \begin{tabular}{|p{#2in}|} \hline \\%
}{%
   \\ \hline \end{tabular}%
}

\newcommand{\retTwo}{\hfill\bigbreak}

\newcommand{\dispDate}[1]{{
   \color{Black}%
   \fontsize{20}{18}\selectfont%
   #1\retTwo
}}


\begin{document}
\setul{0.14em}{0.07em}
\calibri

\exTwo\ul{Lemma:}\\ [-20pt]
\begin{itemize}
	\item[(a)] For all $g^\prime \in G$ we have that $\Fix(g^\prime g (g^\prime)^{-1}) = g^\prime \cdot \Fix(g) \coloneqq \{g^\prime \cdot x \in X : g \cdot x = x\}$.
	
	\begin{myIndent}\exThreeP
		Proof:
		
		{\centering\begin{tabular}{l}
			$x \in \Fix(g^\prime g (g^\prime)^{-1}) \Longleftrightarrow (g^\prime g (g^\prime)^{-1}) \cdot x = x$\\ [3pt]
			$\phantom{x \in \Fix(g^\prime g (g^\prime)^{-1})} \Longleftrightarrow g \cdot ((g^\prime)^{-1} \cdot x) = (g^\prime)^{-1} \cdot x$\\ [3pt]
			$\phantom{x \in \Fix(g^\prime g (g^\prime)^{-1})} \Longleftrightarrow (g^\prime)^{-1} \cdot x \in \Fix(g) \Longleftrightarrow x \in g^\prime \cdot \Fix(g)$.
		\end{tabular} \retTwo\par}
	\end{myIndent}

	\item[(b)] For all $g \in G$ we have that $G_{g \cdot x} = gG_xg^{-1}$
	
	\begin{myIndent}\exThreeP
		Proof:

		{\centering\begin{tabular}{l}
			$g^\prime \in G_{g \cdot x} \Longleftrightarrow g^\prime \cdot (g \cdot x) = g \cdot x$\\ [3pt]
			$\phantom{g^\prime \in G_{g \cdot x}} \Longleftrightarrow (g^{-1}g^\prime g) \cdot x = x \Longleftrightarrow g^{-1}g^\prime g \in G_x \Longleftrightarrow g^\prime \in gG_xg^{-1}$. $\blacksquare$
		\end{tabular}\retTwo\par}
	\end{myIndent}
\end{itemize}

\ul{Corollary:} Suppose $G \curvearrowright X$ and $|X| < \infty$. Then $g \mapsto |\Fix(g)|$ is a class function,\\ meaning that $|\Fix(g^\prime g (g^\prime)^{-1})| = |\Fix(g)|$ (or in other words $|\Fix(g)|$ is constant on\\ any given conjugate classes).

\begin{myIndent}\exThreeP
	Proof:\\
	$|\Fix(g^\prime g (g^\prime)^{-1})| = |g^\prime \cdot \Fix(g)|$ by the last lemma. And since $x \mapsto g^\prime \cdot x$ is an element\\ of $S_X$, we know that $|g^\prime \cdot \Fix(g)| = |\Fix(g)|$. $\blacksquare$\retTwo
\end{myIndent}

\hTwo The \udefine{$G$-orbit} of $x \in X$ is the set of all points in $X$ that are \udefine{$G$-similar} to $x$. Or to put into other words, we define $G \cdot x \coloneqq \{g \cdot x \in X : g \in G\}$ and say that $x^\prime$ is $G$-similar to $x$ if $x^\prime = g \cdot x \in G \cdot x$ for some $g \in G$. Also, in that case we denote $x^\prime \sim x$.\retTwo

\exTwo\ul{Lemma:} $\sim$ is an equivalence relation.
\begin{myIndent}\exThreeP
	Proof:\\ [-20pt]
	\begin{itemize}
		\item $x \sim x$ as $1_G \cdot x = x$.
		\item $x \sim y \Longrightarrow y \sim x$ as $x = g \cdot y \Longrightarrow g^{-1} \cdot x = y$.
		\item If $x \sim y$ and $y \sim z$ then let $g_1, g_2 \in G$ be such that $x = g_1 \cdot y$ and $y = g_2 \cdot z$. Then $x = (g_1g_2) \cdot z$. So $x \sim z$. $\blacksquare$\retTwo
	\end{itemize}
\end{myIndent}

\hTwo It's now clear that the $G$-orbit of $x$: $G \cdot x$, is the equivalence class of $x$ with respect to $\sim$. Thus, we define $X/G \coloneqq \{G \cdot x : x \in X\}$. Also note that $X/G$ is a partition of $X$. As a result, we know that $|X| = \hspace{-0.8em}\sum\limits_{G \cdot x \in X/G}\hspace{-0.8em} |G \cdot x|$.\retTwo

\exTwo\ul{Theorem (Orbit-stabilizer):} The map $G/G_x \to G \cdot x$ given by $gG_x \mapsto g \cdot x$ is a bijection. Hence $|G \cdot x| = [G : G_x]$ (where the latter is the number of left cosets of $G$ in $G_x$).
\begin{myIndent}\exThreeP
	Proof:\\
	We first show this map is well-defined. Suppose $g_1G_x = g_2G_x$. Then $g_2 = g_1h$ for some $g \in G_x$. And in turn $g_2 \cdot x = (g_1h) \cdot x = g_1 \cdot (h \cdot x) = g_1 \cdot x$.\retTwo

	Next we show injectivity. Assume $g_1 \cdot x = g_2 \cdot x$. Then $g_2^{-1} \cdot (g_1 \cdot x) = x$. So $g_2^{-1}g_1 \in G_x$. Or in other words, $g_1G_x = g_2G_x$.\retTwo

	Finally, surjectivity is obvious from the fact that $G \cdot x$ is the set of $y \in X$ such that there exists $g \in G$ with $g \cdot x = y$. $\blacksquare$\newpage
\end{myIndent}

\hTwo Note that $|G \cdot x| = 1$ iff $\forall g \in G,\gap g \cdot x = x$ iff $x \in \Fix(G)$ where:

{\centering$\Fix(G) = X^G \coloneqq \{x \in X : \forall g \in G,\gap g \cdot x = x\}$.\retTwo\par}

This leads to the equation:

{\centering$|X| = \hspace{-0.8em}\sum\limits_{\begin{smallmatrix}
	G \cdot x \in X/G \\ |G \cdot x| = 1
\end{smallmatrix}}\hspace{-0.8em} |G \cdot x| + 
\hspace{-0.8em}\sum\limits_{\begin{smallmatrix}
	G \cdot x \in X/G \\ |G \cdot x| > 1
\end{smallmatrix}}\hspace{-0.8em} |G \cdot x| = |\Fix(G)| + 
\hspace{-0.8em}\sum\limits_{\begin{smallmatrix}
	G \cdot x \in X/G \\ |G \cdot x| > 1
\end{smallmatrix}}\hspace{-0.8em} [G:G_x]$.\retTwo\par}

\Hstatement\mySepTwo

\hTwo I need to do the rest of the math 200a homework still. So I'm going to take a break from taking lecture notes to do the homework.\retTwo

\Hstatement\blab{Set 1 Problem 3:} Find the automorphism group of the Cayley graph of $\mathbb{Z}$ with respect to $\{-1, +1\}$.

\begin{myIndent}\HexOne
	To start off, note that $\{n, m\}$ is an edge of $\Cay(\mathbb{Z}, \{-1, 1\})$ iff $n - m = \pm 1$. This yields the infinite graph which I've attempted to draw below.\retTwo

	{\centering{\color{black}\raisebox{0em}{\tikz[scale=0.8,inner sep=4pt]{
		\tikzstyle{myCir}=[circle, fill, thick, color=black];
		\tikzstyle{myLine}=[thick, color=black];

		\node[myCir, label=above:$0$] (0) at (0, 0) {};
		\node[myCir, label=above:$1$] (1) at (2, 0) {} edge[myLine] (0);
		\node[myCir, label=above:$2$] (2) at (4, 0) {} edge[myLine] (1);
		\node[myCir, label=above:$3$] (3) at (6, 0) {} edge[myLine] (2);
		\node[myCir, label=above:$-1$] (-1) at (-2, 0) {} edge[myLine] (0);
		\node[myCir, label=above:$-2$] (-2) at (-4, 0) {} edge[myLine] (-1);
		\node[myCir, label=above:$-3$] (-3) at (-6, 0) {} edge[myLine] (-2);

		\node (+i) at (7.7, 0) {$\cdots$} edge[myLine] (3);
		\node (-i) at (-7.7, 0) {$\cdots$} edge[myLine] (-3);
	}}}\retTwo\par}

	Now from this graph it is clear that reversing the graph is a symmetry. Specifically, define $\tau(n) = -n$. Then $\tau(n) - \tau(m) = -n - (-m) = m-n = -(n-m)$. Hence, $n-m =\pm 1$ iff $\tau(n) - \tau(m) = \mp 1$ and we thus know that $\tau$ preserves the edges of our graph and is thus a symmetry.\retTwo

	Another obvious symmetry of our graph are index shifts. Specifically define $\sigma(n) = n + 1$. Then $\tau(n) - \tau(m) = n - m$ for all $n, m \in \mathbb{N}$ and it is thus obvious that $\tau$ preserves the edges of graph and is a symmetry.

	\begin{myDindent}\HexPPP
		I glossed over this point before but technically we also need to show $\sigma$ and $\tau$ are\\ bijections. To do this, just note that $\sigma^{-1}$ is given by $n \mapsto n-1$ and $\tau^{-1} = \tau$. So,\\ both maps are invertible.\retTwo
	\end{myDindent}

	Now we claim that every automorphism of $\Cay(\mathbb{Z}, \{-1, 1\})$ is some composition of $\tau$ and $\sigma$. To prove this, let $\theta$ be any arbitrary automorphism. We know that $\theta(0) = k$ for some $k \in \mathbb{Z}$. And in turn we have that $(\sigma^{-k} \circ \theta)(0) = 0$. Next note that $(\sigma^{-k} \circ \theta)(1)$ equals either $+1$ or $-1$. In the former case, we can trivially say that $\tau^0 \circ \sigma^{-k} \circ \theta$ fixes both $0$ and $1$. As for the latter case, since $\tau(0) = 0$ and $\tau(-1) = +1$, we can say that $\tau^{1} \circ \sigma^{-k} \circ \theta$ fixes both $0$ and $1$. Either way, this shows there exists a graph automorphism $\psi = \sigma^k \circ \tau^i$ (where $k \in \{\mathbb{Z}\}$ and $i \in \{0, 1\}$) such that $\psi^{-1} \circ \theta$ fixes both $0$ and $1$.\retTwo

	Observation: If $\phi \in \Aut(\Cay(\mathbb{Z}, \{-1, 1\}))$ with $\phi(0) = 0$ and $\phi(1) = 1$, then $\phi = \myId$.
	\begin{myIndent}\HexTwoP
		To prove this, we do induction separately on the positive integers and then on the negative integers.
		\begin{itemize}
			\item Suppose $n \geq 1$ and we've already shown that $\phi(k) = k$ for all $0 \leq k \leq n$. Then since $\phi$ is a graph automorphism, we must have that $\phi(n + 1) = \phi(n) \pm 1$. But since $\phi$ is a bijection and we already know that $\phi(n-1) = n-1 =\phi(n) - 1$, this means we can only have that $\phi(n+1) = \phi(n) + 1 = n + 1$. By induction this means that $\phi(n) = n$ for all $n \geq 0$.\retTwo
			
			\item Next suppose $n \leq 0$ and we've shown for all $k \geq n$ that $\phi(k) = k$. Then like before we must have that $\phi(n - 1) = \phi(n) \pm 1 = n \pm 1$ since $\phi$ is a graph automorphism. But since $\phi$ is a bijection and we already know $\phi(n + 1) = n + 1$,\newpage we can only have $\phi(n - 1) = n - 1$. By induction this means that $\phi(n) = n$ for all $n \in \mathbb{Z}$.\retTwo
		\end{itemize}
	\end{myIndent}

	Thus $\psi^{-1} \circ \theta = \myId$. Or in other words $\theta = \psi = \sigma^k \tau^i$ where $k \in \mathbb{Z}$ and $i \in \{0, 1\}$. This shows that $\Aut(\Cay(\mathbb{Z}, \{-1, 1\})) = \langle \sigma, \tau\rangle$.\retTwo

	Now the homework sheet specifically tells us to list out all the elements of the group of automorphisms. To do this, we need to show that $\sigma^{k_1} \circ \tau^{i_1} \neq \sigma^{k_2} \circ \tau^{i_2}$ if either $k_1 \neq k_2$ or $i_1 \neq i_2$. 
	\begin{myIndent}\HexTwoP
		To start off, note that $\sigma^{k_1}$ and $\sigma^{k_2}$ are easily checked to not equal each other when\\ $k_1 \neq k_2$. We merely note that $\sigma^{k_1}(0) = k_1 \neq k_2 = \sigma^{k_2}(0)$.\retTwo
		
		Also, it is easy to see that $\langle \sigma \rangle = \{\sigma^k : k \in \mathbb{Z}\}$ is a cyclic subgroup of our collection of symmetries and that $\tau$ is not in that subgroup. After all the only $k \in \mathbb{Z}$ such that $\sigma^k(0) = \tau(0)$ is $k = 0$. However, $\sigma^0(1) = 1 \neq -1 = \tau(1)$. It now follows that $\langle \sigma \rangle$ and $\langle \sigma \rangle \tau$ are two disjoint cosets which partition our collection of symmetries.\retTwo

		Finally, we need to show that if $k_1 \neq k_2$ then $\sigma^{k_1} \circ \tau \neq \sigma^{k_2} \circ \tau$. To do this, suppose $\sigma^{k_1} \circ \tau = \sigma^{k_2} \circ \tau$. Then by composing $\tau$ on the right side we get that $\sigma^{k_1} = \sigma^{k_2}$. And by prior work, we thus know that $k_1 = k_2$.\retTwo
	\end{myIndent}

	Thus $\Aut(\Cay(\mathbb{Z}, \{-1, 1\})) = \{\sigma^k \circ \tau^i : k \in \mathbb{Z} \text{ and } i \in \{0, 1\}\}$ and we know that the representation $\theta = \sigma^k \circ \tau^i$ is unique.\retTwo

	As for showing how to compose elements note that:

	{\centering $\tau \circ \sigma \circ \tau(n) = \tau \circ \sigma(-n) = \tau(-n + 1) = n - 1 = \sigma^{-1}(n)$.\retTwo\par}

	And since conjugation is a group automorphism, we know that:
	\begin{itemize}
		\item $(\sigma^m \circ \tau) \circ \sigma^n = \sigma^m \circ (\tau \circ \sigma^n \circ \tau) \circ \tau = \sigma^m \circ (\tau \circ \sigma \circ \tau)^n \circ \tau = \sigma^m \circ \sigma^{-n} \circ \tau$\\ [3pt]
		$\phantom{(\sigma^m \circ \tau) \circ \sigma^n = \sigma^m \circ (\tau \circ \sigma^n \circ \tau) \circ \tau = \sigma^m \circ (\tau \circ \sigma \circ \tau)^n \circ \tau} = \sigma^{m-n} \circ \tau$,\\

		\item $(\sigma^m \circ \tau) \circ (\sigma^n \circ \tau) = \sigma^m \circ (\tau \circ \sigma^n \circ \tau) = \sigma^m \circ (\tau \circ \sigma \circ \tau)^n = \sigma^m \circ \sigma^{-n} = \sigma^{m - n}$,\\

		\item $\sigma^m \circ (\sigma^n \circ \tau) = \sigma^{m + n} \circ \tau$ and $\sigma^m \circ \sigma^n = \sigma^{m + n}$. $\blacksquare$\retTwo
	\end{itemize}
\end{myIndent}

\blab{Set 1 Problem 2:} Suppose $G$ is a finite group and that for every positive integer $n$:\\ [-16pt]

{\center$|\{g \in G : g^n = e\}| \leq n$\\ [9pt]\par}

(where $e$ is the identity element of $G$). Use the following steps to prove that $G$ is a cyclic group.

\begin{enumerate}
	\item[(a)] Prove that if there is an element of order $d$ in $G$, then there are exactly $\phi(d)$ elements of\\ order $d$ in $G$ where $\phi(d)$ is the Euler $\phi$-function {\color{BrickRed}(where as a reminder $\phi(d)$ equals the\\ number of integers between $1$ and $d$ inclusive which are coprime to $d$)}.
	
	\begin{myIndent}\HexOne
		Suppose $g \in G$ with $o(g) = d$ and then consider the cyclic subgroup $\langle g \rangle \subseteq G$. We\\ [1pt] know that $o(g^k) = \frac{o(g)}{\gcd(o(g), k)} = \frac{d}{\gcd(d, k)} = d$ iff $\gcd(k, d) = 1$. So by considering $g^k$\\ for each $k \in \{1, \ldots, d\}$ with $\gcd(d, k) = d$ we get that there are at least $\phi(d)$\\ [3pt] distinct elements of $G$ with order $d$.\newpage

		That said, all $g^k$ where $k \in \{0, \ldots, d - 1\}$ are distinct elements of $\{g \in G : g^d = e\}$. And since $|\{g \in G : g^d = e\}| \leq d$, this proves that $h \in G$ can satisfy that $h^d = e$ only if $h = g^k$ for some integer $k$. And also because $h^d$ equaling $e$ is a necessary condition for us to have $o(h) = d$, we know that the $\phi(d)$ elements of $G$ we found before are the only elements of $G$ with order $d$.\retTwo
	\end{myIndent}
	
	\item[(b)] For every positive number $d$, let $\psi(d)$ be the number of elements of $G$ that have order $d$. Show that $\psi(d) \leq \phi(d)$ and that $\psi(d) \neq 0$ implies that $d \divides |G|$.
	
	\begin{myIndent}\HexOne
		We know that $\phi(d) \geq 1$ for all positive $d$ since $\gcd(1, d) = 1$. So, if $\psi(d) = 0$, then we trivially know that $\psi(d) \leq \phi(d)$. Meanwhile, if $\psi(d) > 0$ then we showed in part (a) that $\psi(d) = \phi(d)$. Hence in either case we have that $\psi(d) \leq \phi(d)$.\retTwo

		Also, the fact that $d \divides |G|$ if $\psi(d) \neq 0$ is just a result of Lagrange's theorem (since the order of any subgroup of $G$ must divides $|G|$ and $\phi(d) \neq 0$ implies there is a cyclic subgroup of $G$ with order $d$).\retTwo
	\end{myIndent}
	
	\item[(c)] Prove that $\psi(d) = \phi(d)$ if $d$ is a positive divisor of $|G|$. Deduce that $G$ is a cyclic group.
	
	\begin{myIndent}\HexOne
		Let $n = |G|$ and note that $\sum_{d \divides n} \psi(d) = n$ since every element of $G$ has some order dividing $n$. At the same time, it is a somewhat well known result that $\sum_{d \divides n} \phi(d) = n$ for all $n \in \mathbb{N}$.
		\begin{myIndent}\HexTwoP
			I can't find a proof of this result anywhere in my notes so I guess I'll prove it here.\retTwo

			Let $S = \{1, \ldots, n\}$ and define $S_d \coloneqq \{k \in S : \gcd(k, n) = d\}$ for each\\ $d$. Clearly, the $S_d$ form a partition of $S$ as we range over all the divisors\\ of $n$. Also note that there is a bijective correspondence between $S_d$ and\\ $E_{n/d} \coloneqq \{k \in \{1, \ldots, \frac{n}{d}\} : \gcd(k, \frac{n}{d}) = 1\}$.
			\begin{myIndent}\HexPPP
				Specifically note that $\gcd(m, n) = d \Longrightarrow \frac{m}{d}, \frac{n}{d} \in \mathbb{Z}$ with $\gcd(\frac{m}{d}, \frac{n}{d}) = 1$. And if we also have that $m \leq n$ then clearly $\frac{m}{d} \leq \frac{n}{d}$. So, $m \in S_d \Longrightarrow \frac{m}{d} \in E_{n/d}$. Meanwhile, if $\gcd(m, \frac{n}{d}) = 1$, then we know that $\gcd(dm, n) = d$. And also if $m \leq \frac{n}{d}$, then we know that $md \leq n$ Hence $m \in E_{n/d} \Longrightarrow dm \in S_d$. It now follows that the map $m \mapsto \frac{m}{d}$ is an invertible map from $S_d$ to $E_{n/d}$.\retTwo
			\end{myIndent}

			Now $|S_d| = |E_{n/d}| = \phi(\frac{n}{d})$. Also, we know that $n = |S| = \sum_{d \divides n} |S_d|$. So we have shown that $n = \sum_{d \divides n} \phi(\frac{n}{d}) = \sum_{d \divides n} \phi(d)$.\retTwo
		\end{myIndent}

		Since $\psi(d) \leq \phi(d)$ for all $d$, we thus have that:

		{\centering $n = \sum_{d \divides n} \psi(d) \leq \sum_{d \divides n} \phi(d) = n$.\retTwo\par}

		And this proves that $\sum_{d \divides n} \psi(d) = \sum_{d \divides n} \phi(d)$. Going even further, since\\ $0 \leq \psi(d) \leq \phi(d)$ for all $d$, the two sums can only equal if $\psi(d) = \phi(d)$ for all $d$ being summed over. In particular, we must have that $\phi(n) = \psi(n) \geq 1$. So, there is some element of order $n = |G|$ in $G$. This is equivalent to saying that $G$ is cyclic. $\blacksquare$\retTwo
	\end{myIndent}
\end{enumerate}

\blab{Set 1 Problem 1:} Suppose $G_1$ and $G_2$ are two groups. We say $G_1$ and $G_2$ are \udefine{algebraically\\ independent} if there are no proper normal subgroups $N_1$ and $N_2$ of $G_1$ and $G_2$ respectively\\ such that $G_1/N_1 \cong G_2/N_2$.\newpage
\begin{enumerate}
	\item[(a)] Prove that $G_1$ and $G_2$ are algebraically independent if and only if $G_1 \times G_2$ satisfies the following property: suppose $H$ is a subgroup of $G_1 \times G_2$ and the projection of $H$ to the $i$-th component is $G_i$ for $i = 1, 2$. Then $H = G_1 \times G_2$.
	\begin{myIndent}\HexOne
		\begin{myDindent}\exPPP
			As a reminder, the group $G_1 \times G_2$ is just the cartesian product\\ of the two groups equipped with the law of composition that\\ $(g_1, g_2)(g_1^\prime, g_2^\prime) = (g_1g_1^\prime, g_2g_2^\prime)$.\retTwo
		\end{myDindent}
		
		$(\Longrightarrow)$\\
		Suppose $G_1$ and $G_2$ are algebraically independent and then consider any subgroup $H \subseteq G_1 \times G_2$ such that $\pi_1(H) = G_1$ and $\pi_2(H) = G_2$ (where $\pi_1$ and $\pi_2$ are the projection maps). Also let $e_1$ and $e_2$ denote the identity elements of $G_1$ and $G_2$ respectively.\retTwo

		To start off, let $N_1 \coloneqq H \cap (\{e_1\} \times G_2)$ and $N_2 \coloneqq H \cap (G_1 \times \{e_2\})$. Then set $N_1^\prime \coloneqq \pi_2(N_1)$ and $N_2^\prime \coloneqq \pi_1(N_2)$. Both $N_1$ and $N_2$ are easily seen to be subgroups of $G_1 \times G_2$ as they are both intersections of groups. From there it also easy to see that $N_1^\prime$ and $N_2^\prime$ are subgroups of $G_2$ and $G_1$ respectively on account of being images of $N_2$ and $N_1$ via the homomorphisms $\pi_2$ and $\pi_1$. And of course there are obvious group isomorphisms showing that $N_1^\prime \cong N_1$ and $N_2^\prime \cong N_2$.
		\retTwo

		Our first big step is to show that $N_1^\prime$ and $N_2^\prime$ are normal subgroups (which in turn means that $G_1/N_2^\prime$ and $G_2/N_1^\prime$ are well-defined quotient groups).
		\begin{myIndent}\HexTwoP
			Suppose $g_1 \in N_2^\prime$ and let $g^\prime_1$ be any element of $G$. Since $\pi_1(H) = G_1$, we know\\ [2pt] there is some $g^\prime_2 \in G_2$ such that $(g^\prime_1, g^\prime_2) \in H$. And since $H$ is closed under\\ [2pt] inverses, we also know that $((g_1^\prime)^{-1}, (g_2^\prime)^{-1}) \in H$. Therefore $g^\prime_1 g (g^\prime_1)^{-1} \in N_2^\prime$\\ [2pt] since $(g^\prime_1 g (g^\prime_1)^{-1},  g^\prime_2 e_2 (g^\prime_2)^{-1}) = (g^\prime_1 g (g^\prime_1)^{-1}, e_2) \in H$. This proves that $N_2^\prime$ is\\ [2pt] normal in $G_1$. Analogous reasoning shows that $N_1^\prime$ is normal in $G_2$.\retTwo
		\end{myIndent}

		Next we define a group homomorphism $\phi$ from $G_1$ to $G_2/N_1^\prime$ as follows:

		{\centering Given any $g_1 \in G$, let $\phi(g_1) = g_2N_1^\prime$ where $(g_1, g_2) \in H$.   \retTwo\par}

		\begin{myIndent}\HexTwoP
			To show this is well defined, suppose $g_2, g_2^\prime \in G_2$ both satisfy that\\ [1pt] $(g_1, g_2) \in H$ and $(g_1, g_2^\prime) \in H$. Then $(e_1, g_2^{-1}g_2^\prime) \in H$, which in turns\\ [1pt] means that $g_2^{-1}g_2^\prime \in N_1^\prime$. This is equivalent to saying that $g_2^{-1} g_2^\prime N_1^\prime = N_1^\prime$\\ which in turn is equivalent to saying that $g_2^\prime N_1^\prime = g_2 N_1^\prime$.\retTwo

			Also, to see that $\phi$ is a homomorphism, suppose $(g_1, g_2), (g_1^\prime, g_2^\prime) \in H$.\\ [1pt] Then $(g_1g_1^\prime, g_2g_2^\prime) \in H$ and so $\phi(g_1g_1^\prime) = g_2g_2^\prime N_1^\prime$. But we also have that\\ [1pt] $\phi(g_1)\phi(g_2) = g_2N_1^\prime g_2^\prime N_1^\prime = g_2g_2^\prime N_1^\prime$. So $\phi(g_1g_2) = \phi(g_1)\phi(g_2)$.\retTwo
		\end{myIndent}

		Now we claim $\phi$ is surjective. After all, $\pi_2(H) = G_2$ so for all $g_2 \in G_2$ there exists\\ [1pt] $g_1 \in G_1$ such that $(g_1, g_2) \in H$. And then in turn $\phi(g_1) = g_2 N_1^\prime$. We also claim that\\ [1pt] the kernel of $\phi$ is $N_2^\prime$. After all, suppose $\phi(g_1) = N_1^\prime$. Then we know that there is some\\ [1pt] $g_2 \in G_2$ such that $(g_1, g_2) \in H$ and $(e_1, g_2) \in H$. But since $(e_1, g_2) \in H$, we also\\ [1pt] know that $(e_1, g_2^{-1}) \in H$, and thus $(e_1g_1, g_2^{-1}g_2) = (g_1, e_2) \in H$. So, $g_1 \in N_2^\prime$ and\\ [1pt] we've shown that $\ker(\phi) \subseteq N_2^\prime$. Going the other direction and showing $N_2^\prime \subseteq \ker(\phi)$\\ [1pt] is as simple as noting that $e_2N_1^\prime = N_1^\prime$.\newpage

		By the first isomorphism theorem, we are thus able to conclude that $\frac{G}{N_2^\prime} \cong \frac{G}{N_1^\prime}$.\retTwo

		{\color{red}I ran out of time so everything after this point is not being graded\dots\retTwo}

		Since $G_1$ and $G_2$ are algebraically independent, this implies that $N_2^\prime = G_1$ and\\ $N_1^\prime = G_2$. But now since $G_1 \times \{e_2\}$ and $\{e_1\} \times G_2$ are both contained in $H$ are\\ easily seen to together generate all of $G_1 \times G_2$, we know that $H = G_1 \times G_2$.\\ This proves the property in the problem statement.\retTwo

		$(\Longleftarrow)$\\
		Suppose $G_1$ and $G_2$ are not algebraically independent and let $N_1$ and $N_2$ be\\ proper normal subgroups of $G_1$ and $G_2$ such that $G_1 / N_1 \cong G_2 / N_2$. Then let\\ $\phi : G_1 / N_1 \to G_2 / N_2$ be a group isomorphism.\retTwo

		We define the set $H \coloneqq \{(g_1, g_2) \in G_1 \times G_2 : \phi(g_1N_1) = g_2N_2\}$ and claim that\\ this is a subgroup of $G_1 \times G_2$.
		\begin{myIndent}\HexTwoP
			\begin{itemize}
				\item Note that $(e_1, e_2) \in H$ since we must have that $\phi(N_1) = N_2$.\\ [-6pt]
				\item Suppose $(g_1, g_2) \in H$. Then $\phi(g_1 N_1) = g_2 N_2$. But note that:
				
				{\centering$N_2 = \phi(N_1) = \phi(g_1^{-1}g_1 N_1) = \phi(g_1^{-1} N_1)\phi(g_1 N_1) = \phi(g_1^{-1} N_1) g_2N_2$.\retTwo\par}

				Therefore $\phi(g_1^{-1}N_1) = (g_2N_2)^{-1} = g_2^{-1} N_2$ and we've shown that\\ $(g_1, g_2) \in H$.\\ [-6pt]

				\item Suppose $(g_1, g_2), (g_1^\prime, g_2^\prime) \in H$. Then we have that $\phi(g_1N_1) = g_2N_2$ and\\ $\phi(g_1^\prime N_1) = g_2^\prime N_2$. And since $\phi$ is a group homomorphism, we get that:
				
				{\centering $\phi(g_1g_1^\prime N_1) \phi(g_1 N_1)\phi(g_1^\prime N_1) = (g_2N_2)(g_2^\prime N_2) = g_2g_2^\prime N_2$. \retTwo\par}

				This shows that $(g_1g_1^\prime, g_2g_2^\prime) \in H$.\retTwo
			\end{itemize}
		\end{myIndent}

		Next observe that $\pi_1(H) = G_1$. After all, for any $g_1 \in G_1$ we can just pick\\ [1pt] $g_2 \in \phi(g_1 N)$ and then we'll know that $(g_1, g_2) \in H$. We also know that\\ [1pt] $\pi_2(H) = G_2$. After all, since $\phi$ is surjective, we know that for any $g_2 \in G_2$\\ [1pt] there exists a coset $g_1^\prime N_1 \in G_1/N_1$ such that $\phi(g_1^\prime N_1) = g_2N_2$. And now by\\ [1pt] just choosing any $g_1 \in g_1^\prime N_1$ we get that $(g_1, g_2) \in H$.\retTwo

		That said, $H \neq G_1 \times G_2$. To see this, just pick any $g_1 \in N_1$ and $g_2 \notin N_2$.\\ Then $\phi(g_1 N_1) \neq g_2N_2$ and we have that $(g_1, g_2) \notin H$. $\blacksquare$\retTwo
	\end{myIndent}
	
	\item[(b)] Suppose $G_1$ and $G_2$ are two finite groups and $\gcd(|G_1|, |G_2|) = 1$. Then $G_1$ and $G_2$ are algebraically independent.
	
	\begin{myIndent}\HexOne
		Let $H$ be any subgroup of $G_1 \times G_2$ such that $\pi_1(H) = G_1$ and $\pi_2(H) = G_2$.\\ [2pt] Since $\pi_1$ and $\pi_2$ are group homomorphisms from $G_1 \times G_2$ to $G_1$ and $G_2$ respectively,\\ [2pt] we know that both $|G_1| = |\pi_1(H)|$and $|G_2| = |\pi_2(H)|$ divide $|H|$. Hence,\\ [2pt] $\lcm(|G_1|, |G_2|)$ divides $|H|$. Meanwhile, we have by Lagrange's theorem that\\ [2pt] $|H|$ divides $|G_1 \times G_2| = |G_1||G_2|$.\retTwo

		But now because $\gcd(|G_1|, |G_2|) = 1$, we have that $\lcm(|G_1|, |G_2|) = |G_1||G_2|$ So, we must have $|H| = |G_1||G_2|$. And this proves that $H = G_1 \times G_2$.\newpage

		By part (a), we can now conclude that $G_1$ and $G_2$ are algebraically independent. $\blacksquare$\retTwo
	\end{myIndent}
\end{enumerate}

\mySepTwo

\hTwo

% ~~~~~~~~~~~~~~~~~~~~~~~~~~~~~~~~~~~~~~~~~~~~~~

\hypertarget{Folland Proposition 10.1}{}
\hypertarget{bib citation 12}{}
\hypertarget{math 241a lecture 3}{}
\hypertarget{math 200a lecture 4}{}
\hypertarget{math 220a lecture 3}{}
\hypertarget{bib citation 13}{}
\hypertarget{page idk 1 reference}{}

\end{document}




