\documentclass{book}

\usepackage{fontspec} % used to import Calibri
\usepackage{anyfontsize} % used to adjust font size

% needed for inch and other length measurements
% to be recognized
\usepackage{calc}

% for colors and text effects as is hopefully obvious
\usepackage[dvipsnames]{xcolor}
\usepackage{soul}

% control over margins
\usepackage[margin=1in]{geometry}
\usepackage[strict]{changepage}

\usepackage{mathtools}
\usepackage{amsfonts}
\usepackage{bm}

\usepackage[scr=rsfso, scrscaled=.96]{mathalpha}

% This is how I'm getting the nice caligraphy font :(
\DeclareMathAlphabet{\eulerscr}{U}{eus}{m}{n}
\newcommand{\mathcalli}[1]{\text{\scalebox{1.11}{$\eulerscr{#1}$}}}


\usepackage{amssymb} % originally imported to get the proof square
\usepackage{xfrac}
\usepackage[overcommands]{overarrows} % Get my preferred vector arrows...
\usepackage{relsize}

% Just am using this to get a dashed line in a table...
% Also you apparently want this to be inactive if you aren't
% using it because it slows compilation.
\usepackage{arydshln} \ADLinactivate 
\newenvironment{allowTableDashes}{\ADLactivate}{\ADLinactivate}

\usepackage{graphicx}
\graphicspath{{./158_Images/}}

\usepackage{tikz}
   \usetikzlibrary{arrows.meta}
   \usetikzlibrary{graphs, graphs.standard}

\usepackage{quiver} %commutative diagrams






\usepackage[hidelinks]{hyperref}
\newcommand{\inLinkRap}[2]{{\color{blue}\hyperlink{#1}{\textit{#2}}}}







\newfontfamily{\calibri}{Calibri}
\setlength{\parindent}{0pt}
\definecolor{RawerSienna}{HTML}{945D27}

% ~~~~~~~~~~~~~~~~~~~~~~~~~~~~~~~~~~~~~~~~~~~~~~~~~~
%Arrow Commands:

% Thank you Bernard, gernot, and Sigur who I copied this from:
% https://tex.stackexchange.com/questions/364096/command-for-longhookrightarrow
\renewcommand{\hookrightarrow}{\lhook\joinrel\rightarrow}
\renewcommand{\hookleftarrow}{\leftarrow\joinrel\rhook}
\newcommand{\hooklongrightarrow}{\lhook\joinrel\longrightarrow}
\newcommand{\hooklongleftarrow}{\longleftarrow\joinrel\rhook}
\newcommand{\hookxlongrightarrow}[2][]{\lhook\joinrel\xrightarrow[#1]{#2}}
\newcommand{\hookxlongleftarrow}[2][]{\xleftarrow[#1]{#2}\joinrel\rhook}

% Thank you egreg who I copied from:
% https://tex.stackexchange.com/questions/260554/two-headed-version-of-xrightarrow
\newcommand{\longrightarrowdbl}{\longrightarrow\mathrel{\mkern-14mu}\rightarrow}
\newcommand{\rightarrowdbl}{\rightarrow\mathrel{\mkern-14mu}\rightarrow}
\newcommand{\longleftarrowdbl}{\leftarrow\mathrel{\mkern-14mu}\longleftarrow}

\newcommand{\xrightarrowdbl}[2][]{%
  \xrightarrow[#1]{#2}\mathrel{\mkern-14mu}\rightarrow
}
\newcommand{\xleftarrowdbl}[2][]{%
  \leftarrow\mathrel{\mkern-14mu}\xleftarrow[#1]{#2}
}

\newcommand{\mRoman}[1]{%
   \textrm{\MakeUppercase{\romannumeral #1}}%
}

\newcommand{\locUConverges}{%
	\xrightarrow{\ell.u.}
}
\newcommand{\UConverges}{%
	\xrightarrow{u.}
}





% ~~~~~~~~~~~~~~~~~~~~~~~~~~~~~~~~~~~~~~~~~~~~~~~~~~

\newcommand{\hOne}{%
   \color{Black}%
   \fontsize{14}{16}\selectfont%
}
\newcommand{\hTwo}{%
\color{Black}%
   \fontsize{13}{15}\selectfont%
}
% \newcommand{\scratchWork}{%
%    \color{PineGreen!85!Orange}
%    \fontsize{12}{14}\selectfont%
% }
\newcommand{\hThree}{%
   \color{Black}%
   \fontsize{12}{14}\selectfont%
}
\newcommand{\myComment}{%
   \color{RawerSienna}%
   \fontsize{12}{14}\selectfont%
}
\newcommand{\pracOne}{
   \color{BrickRed}%
   \fontsize{13}{15}\selectfont%
}
\newcommand{\pracTwo}{
   \color{Orange}%
   \fontsize{12}{14}\selectfont%
}
\newcommand{\why}{%
   \color{Orange}%
   \fontsize{12}{14}\selectfont%
	Why:
}
\newcommand{\exOne}{%
   \color{Purple}%
   \fontsize{14}{16}\selectfont%
}
\newcommand{\exTwo}{%
   \color{Purple}%
   \fontsize{13}{15}\selectfont%
}
\newcommand{\exThree}{%
   \color{Purple}%
   \fontsize{12}{14}\selectfont%
}
\newcommand{\exP}{%
   \color{Purple}%
   \fontsize{12}{14}\selectfont%
}
\newcommand{\exTwoP}{%
   \color{RedViolet}%
   \fontsize{13}{15}\selectfont%
}
\newcommand{\exThreeP}{%
   \color{RedViolet}%
   \fontsize{12}{14}\selectfont%
}
\newcommand{\exFourP}{%
   \color{RedViolet}%
   \fontsize{11}{13}\selectfont%
}
\newcommand{\exPP}{%
   \color{RedViolet}%
   \fontsize{12}{14}\selectfont%
}
\newcommand{\exPPP}{%
   \color{VioletRed}%
   \fontsize{12}{14}\selectfont%
}

% Homework standard below (God the bloat in the header is absurd...)
% ~~~~~~~~~~~~~~~~~~~~~~~~~~~~~~~~~~~~~~~~~~~~~~~~
\newcommand{\Hstatement}{%
   \color{MidnightBlue!90!Black}%
   \fontsize{12}{13}\selectfont%
}
\newcommand{\HexOne}{%
   \color{Purple}%
   \fontsize{12}{13}\selectfont%
}
\newcommand{\HexTwoP}{%
   \color{RedViolet}%
   \fontsize{12}{13}\selectfont%
}
\newcommand{\HexPPP}{%
   \color{VioletRed}%
   \fontsize{11}{12}\selectfont%
}

% ~~~~~~~~~~~~~~~~~~~~~~~~~~~~~~~~~~~~~~~~~~~~~~~~

\newcommand{\cyPen}[1]{{\vphantom{.}\color{Cerulean}#1}}
\newcommand{\redPen}[1]{{\vphantom{.}\color{Red}#1}}

\newenvironment{myIndent}{%
   \begin{adjustwidth}{2.5em}{0em}%
}{%
   \end{adjustwidth}%
}

\newenvironment{myDindent}{%
   \begin{adjustwidth}{5em}{0em}%
}{%
   \end{adjustwidth}%
}

\newenvironment{myTindent}{%
   \begin{adjustwidth}{7.5em}{0em}%
}{%
   \end{adjustwidth}%
}

\newenvironment{myConstrict}{%
   \begin{adjustwidth}{2.5em}{2.5em}%
}{%
   \end{adjustwidth}%
}

\newcommand{\udefine}[1]{{%
   \setulcolor{Red}%
   \setul{0.14em}{0.07em}%
   \ul{#1}%
}}

\newcommand{\uprop}[1]{{%
   \setulcolor{Purple}%
   \setul{0.14em}{0.07em}%
   \ul{#1} 
}}

\newcommand{\blab}[1]{\textbf{#1}}
\newcommand{\blect}[1]{{\color{MidnightBlue}\textbf{#1}}}

\newcommand{\uuline}[2][.]{%
{\vphantom{a}\color{#1}%
\rlap{\rule[-0.18em]{\widthof{#2}}{0.06em}}%
\rlap{\rule[-0.32em]{\widthof{#2}}{0.06em}}}%
#2}

\newcommand{\pprime}{{\prime\prime}}
\newcommand{\suchthat}{ \hspace{0.3em}s.t.\hspace{0.3em}}
\newcommand{\rea}[1]{\mathrm{Re}(#1)}
\newcommand{\ima}[1]{\mathrm{Im}(#1)}
\newcommand{\comp}{\mathsf{C}}
\newcommand{\trans}{\mathsf{T}}
\newcommand{\myHS}{ \hspace{0.5em}}
\newcommand{\gap}{\phantom{2}}

\newcommand{\GenLin}{\ensuremath{\mathrm{GL}}}
\newcommand{\Cay}{\ensuremath{\mathrm{Cay}}}

\newcommand{\myId}{\mathrm{Id}}
\newcommand{\myIm}{\mathrm{im}}
\newcommand{\Obj}{\mathrm{Obj}}
\newcommand{\Hom}{\mathrm{Hom}}
\newcommand{\End}{\mathrm{End}}
\newcommand{\Aut}{\mathrm{Aut}}

\newcommand{\df}{\mathrm{d}}
\newcommand{\Df}{\mathrm{D}}

\newcommand{\mcateg}[1]{{\bm{\mathsf{#1}}}}

\newcommand{\mdeg}{\mathrm{mdeg}\phantom{.}}

\newcommand{\dividesDeprecated}{\mathop{\mid}}
\newcommand{\divides}{\mathrel{\mid}}

\newcommand{\card}{\mathrm{card}}
\newcommand{\supp}{\mathrm{supp}}
\newcommand{\diam}{\mathrm{diam}}
\newcommand{\conv}{\mathrm{conv}}
\newcommand{\opnorm}{\mathrm{op}}
\newcommand{\loc}{\mathrm{loc}}
\newcommand{\sgn}{\mathrm{sgn}}
\newcommand{\acc}{\mathrm{acc}}

\newcommand{\mSpan}{\mathrm{span}}
\newcommand{\Interior}{\mathop{\mathrm{Int}}}

\newcommand{\mMat}[1]{\mathbf{#1}}

\newcommand{\NBV}{\ensuremath{\mathrm{NBV}}}
\newcommand{\Acc}{\mathrm{Acc}}
\newcommand{\BV}{\ensuremath{\mathrm{BV}}}
\newcommand{\Var}{\ensuremath{\mathrm{Var}}}

\newcommand{\Alt}{\mathrm{Alt}}
\newcommand{\Sym}{\mathrm{Sym}}

\newcommand{\weakst}{weak$^*$ }

\newcommand{\radtimes}{\mathop{\widehat{\times}}}

\newcommand{\mMod}[1]{\phantom{a}(\mathrel{\mathrm{mod}} #1)}
\newcommand{\Fun}{\mathrm{Fun}}
\newcommand{\act}{\mathrm{act}}
\newcommand{\Fix}{\mathrm{Fix}}
\newcommand{\Sub}{\mathrm{Sub}}
\newcommand{\Cl}{\mathrm{Cl}}
\newcommand{\GL}{\mathrm{GL}}
\newcommand{\SL}{\mathrm{SL}}
\newcommand{\PSL}{\mathrm{PSL}}
\newcommand{\core}{\mathrm{core}}
\newcommand{\Syl}{\mathrm{Syl}}
\newcommand{\Iso}{\mathrm{Iso}}
\newcommand{\Homeo}{\mathrm{Homeo}}
\newcommand{\Inn}{\mathrm{Inn}}
\newcommand{\Out}{\mathrm{Out}}
\newcommand{\ab}{\mathrm{ab}}
\newcommand{\Max}{\mathrm{Max}}
\newcommand{\lt}{\mathrm{lt}}
\newcommand{\Nil}{\mathrm{Nil}}
\newcommand{\Ideal}{\mathrm{Ideal}}
\newcommand{\Spec}{\mathrm{Spec}}
\newcommand{\Res}{\mathrm{Res}}
\newcommand{\prim}{\mathrm{prim}}
\newcommand{\exConv}{\mathrm{ex}}
\newcommand{\mRank}{\mathrm{rank}}

\DeclareMathOperator{\lcm}{lcm}
\DeclareMathOperator{\Log}{Log}
\DeclareMathOperator{\ord}{ord}
\DeclareMathOperator{\adj}{adj}
\DeclareMathOperator{\symdif}{\triangle}
\DeclareMathOperator{\Average}{Average}
\DeclareMathOperator*{\AverageAst}{Average}

% Thank you Gonzalo Medina and Moriambar who wrote this on stack exchange:
%https://tex.stackexchange.com/questions/74125/how-do-i-put-text-over-symbols%
\newcommand{\myequiv}[1]{\stackrel{\mathclap{\mbox{\footnotesize{$#1$}}}}{\equiv}}

% Thank you chs who wrote this on stack exchange:
%https://tex.stackexchange.com/questions/89821/how-to-draw-a-solid-colored-circle%
\newcommand{\filledcirc}[1][.]{\ensuremath{\hspace{0.05em}{\color{#1}\bullet}\mathllap{\circ}\hspace{0.05em}}}

%Thank you blerbl who wrote this on stack exchange:
%https://tex.stackexchange.com/questions/25348/latex-symbol-for-does-not-divide
\newcommand{\ndiv}{\hspace{-0.3em}\not|\hspace{0.35em}}

\newcommand{\mySepOne}[1][.]{%
   {\noindent\color{#1}{\rule{6.5in}{1mm}}}\\%
}
\newcommand{\mySepTwo}[1][.]{%
   {\noindent\color{#1}{\rule{6.5in}{0.5mm}}}\\%
}
\newcommand{\mySepThree}[1][.]{%
   {\noindent\color{#1}{\rule{6in}{0.25mm}}}\\%
}

\newenvironment{myClosureOne}[2][.]{%
   \color{#1}%
   \begin{tabular}{|p{#2in}|} \hline \\%
}{%
   \\ \hline \end{tabular}%
}

\newcommand{\retTwo}{\hfill\bigbreak}

\newcommand{\dispDate}[1]{{
   \color{Black}%
   \fontsize{20}{18}\selectfont%
   #1\retTwo
}}


\begin{document}
\setul{0.14em}{0.07em}
\calibri

\hTwo\dispDate{1/26/2025}

\blect{Math 220b Notes:}\retTwo

Given a collection of points $a_1, \ldots, a_N \in \mathbb{C}$ and positive integers $m_1, \ldots, m_N$, we can\\ always find an entire function $f: \mathbb{C} \to \mathbb{C}$ which only has zeros at the $a_n$ and with\\ multiplicities given by $m_n$. After all, we can just consider the polynomial:

{\center$f(z) = \prod_{n=1}^n (z - a_n)^{m_n}$.\retTwo\par}

A natural folowup question to ask is if we can find an entire function with infinitely many prescribed zeros of varying multiplicities. As it turns out the answer is yes.
\begin{myIndent}\color{BrickRed}
	A natural first idea one might have for proving this is to try setting:
	
	{\centering$f(z) = \prod_{a=1}^\infty (1 - \frac{z}{a_n})^{m_n}$.\retTwo\par}
	
	The issue with this approach though is that that product may diverge. As an\\ example, consider setting $a_n = -n$ and $m_n = 1$ for all $n \in \mathbb{N}$. Then the product\\ $f(z) \coloneqq \prod_{n=1}^\infty (1 - \frac{z}{a_n})^{m_n} = \prod_{n=1}^\infty (1 + \frac{z}{n})$ diverges to $\infty$ at $z = 1$. After all, we\\ get that $f(1) = 2(\frac{3}{2})(\frac{4}{3})(\frac{5}{4})\cdots$.\retTwo

	By noting that $\ord(\prod_{k=1}^\infty g_k, a) = \sum_{k=1}^\infty \ord(g_k, a)$, a natural way of modifying our first idea is to try and multiply each $(1 - \frac{z}{a_n})^{m_n}$ term by some other function with no zeros. That way, we don't add any unwanted zeros to the function we are\\ constructing and we can maybe coerce the product into converging.\retTwo
	
	It turns out this modified approach will work. Although surprisingly, by the\\ homework problem on \inLinkRap{Math 220b Set 2 Problem 5}{pages 549-550}, we know these other function we're\\ multiplying onto $(1 - \frac{z}{a_n})^{m_n}$  would have to have the form $e^{g_n}$ where $g_n \in O(\mathbb{C})$\\ for all $n$.\retTwo
\end{myIndent}

Given any $p \in \mathbb{Z}_{\geq 0}$, we define the \udefine{$p$th. Weierstrass primary/elementary factor} to be:

{\centering$E_p(z) = \left\{\begin{matrix}
	(1 - z) & \text{ if } p = 0 \\ (1 - z)\exp(z + \frac{z^2}{2} + \cdots + \frac{z^p}{z}) & \text{ if } p \neq 0
\end{matrix}\right.$.\retTwo\par}

Note that if $a \in \mathbb{C} - \{0\}$, then $E_p(\frac{z}{a})$ is entire and it's only zero is a a simple one at $z = a$.

\begin{myIndent}\color{BrickRed}
	For motivation on why we are defining $E_p(z)$, note that the Taylor series for\\ $-\Log(1 - z)$ about $0$ is $\sum_{n=1}^\infty \frac{z^n}{n}$. Therefore, we'll ideally have that $E_p(z)$ is\\ approximately equal to $1 = \frac{1 - z}{1 - z}$ when $p$ is large and $|z|$ is small. The next lemma\\ will make this idea more concrete.\retTwo
\end{myIndent}

\exTwo\ul{Lemma A:} $|E_p(z) - 1| \leq |z|^{p + 1}$ if $|z| \leq 1$.
\begin{myIndent}\exThreeP
	Proof:\\
	If $p = 0$, then $|E_p(z) - 1| = |-z| = |z|$. So our proposed inequality trivialy holds.\retTwo

	Suppose $p > 0$ and set $u(z) = z + \frac{z^2}{2} + \cdots + \frac{z^p}{z}$. Then $E_p(z) = (1 - z)e^{u(z)}$ and we want to express $E_p(z)$ as a power series $\sum_{k=0}^\infty a_k z^k$ (which will have infinite radius of\\ [1pt] convergence since $E_p(z)$ is entire).\newpage

	\ul{Claim 1:} $a_0 = 1$.
	\begin{myIndent}\exPPP
		Note that $a_0 = E_p(0) = (1 - 0)e^{u(0)} = 1$.\retTwo
	\end{myIndent}

	\ul{Claim 2:} $a_1 = a_2 = \cdots = a_p = 0$.
	\begin{myIndent}\exPPP
		By differentiating $u(z)$, we get that $u^\prime(z) = 1 + z + \ldots + z^{p-1}$. In turn:
		
		{\centering$(1 - z)u^\prime(z) = 1 - z^p$.\retTwo\par}

		But that implies that:
		
		{\centering$E_p^\prime(z) = -e^{u(z)} + (1 - z)u^\prime(z)e^{u(z)} = -e^{u(z)} + (1 - z^p)e^{u(z)} = -z^p e^{u(z)}$\retTwo\par}

		Therefore, if we look at the taylor expansion $\sum_{k=1}^\infty ka_k z^{k-1}$ for $E_p^\prime(z)$ about $0$ and note that $e^{u(0)} = 1 \neq 0$, we must have that the lowest degree term in that power series is the $z^p$ term. In other words, $a_1 = a_2 = \cdots = a_p = 0$.\retTwo
	\end{myIndent}

	\ul{Claim 3:} $a_k \leq 0$ in $\mathbb{R}$ for all $k \geq p + 1$.
	\begin{myIndent}\exPPP
		We showed in claim 2 that:
		
		{\centering$\sum_{k=p+1}^\infty ka_k z^{k-1} = -z^p e^{u(z)}$.\retTwo\par}

		In turn, $\sum_{k=p+1}^\infty ka_k z^{k - (p + 1)} = -e^{u(z)} = -\exp(z + \frac{z^2}{2} + \cdots + \frac{z^p}{p})$. Or in other words, we have that $a_k$ equals $\frac{-1}{k}$ times the $(k - (p + 1))$th. coefficient in the Taylor expansion of $\exp(z + \frac{z^2}{2} + \cdots + \frac{z^p}{p})$ for all $k \geq p + 1$.\retTwo

		If we can show that all coefficients in the Taylor expansion of $\exp(z + \frac{z^2}{2} + \cdots + \frac{z^p}{p})$ are positive, we will be done. Fortunately, note that:

		{\centering $\exp(z + \frac{z^2}{2} + \cdots + \frac{z^p}{p}) = e^z e^{\frac{z^2}{2}} \cdots e^{\frac{z^p}{p}} = \prod\limits_{n = 1}^p\left(\sum\limits_{\ell = 0}^\infty \frac{1}{\ell! n^\ell}z^{n\ell}\right)$.\retTwo\par}

		By taking successive Cauchy products of those series (i.e. foiling), we'll get that the coefficients of the power series for $\exp(z + \frac{z^2}{2} + \cdots + \frac{z^p}{p})$ are all sums of products of the $\frac{1}{\ell! n^\ell}$ (which are all positive). Hence, those coefficients are positive.
	\end{myIndent}
\end{myIndent}

\pracTwo\mySepTwo
Unnecessary tangent: Here's how we can actually calculate the power series expansion $\sum_{k=0}^\infty c_k z^k$ for $e^{u(z)}$.
\begin{myIndent}
	To start off, the derivative of $e^{u(z)}$ is $(1 + z + \ldots + z^{p-1})e^{u(z)}$. Therefore, we must have that $\sum_{k=0}^\infty (k+1)c_{k+1} z^k = (1 + z + \ldots + z^{p-1})\sum_{k=0}^\infty c_{k} z^k$. Also, when you consider that $e^{u(0)} = 1$, we thus get the following recurrence relation:
	\begin{itemize}
		\item $c_0 = 1$
		\item $c_{k+1} = \frac{1}{k+1}\hspace{-0.5em}\sum\limits_{\ell = 0}^{\min(k, p-1)}\hspace{-0.9em} c_{k - \ell}$\retTwo
	\end{itemize}

	Next, we can easily calculate from that relation that $c_1 = \cdots = c_p = 1$. Furthermore, note that for any $k \geq p - 1$ that:
	
	{\centering$c_{k+2} - \frac{k+1}{k+2}c_{k+1} = \frac{1}{k+2}c_{k+1} - \frac{1}{k+2}c_{k-p+1}$.\retTwo\par}

	In other words, $c_{k+2} = c_{k+1} - \frac{1}{k+2}c_{k-p+1}$. So finally (and this is how much simplified I'm capable of getting it before my attention span runs out), we have that the coefficients $c_k$ are given by the following recurrence relation:\newpage
	\begin{itemize}
		\item $c_0 = c_1 = \cdots = c_p = 1$;
		\item $c_{k+1} = c_k - \frac{1}{k+1}c_{k - p}$ if $k \geq p$.
	\end{itemize}
\end{myIndent}
\pracTwo\mySepTwo

\begin{myIndent}\exThreeP
	\ul{Claim 4:} $\sum_{k=p+1}^\infty |a_k| = 1$
	\begin{myIndent}\exPPP
		Note that $0 = E_p(1) = 1 + \sum_{k=p+1}^\infty a_k$. Therefore:
		
		{\centering$\sum_{k=p+1}^\infty |a_k| = -\sum_{k=p+1}^\infty a_k = 1$\retTwo\par}
	\end{myIndent}

	Finally, note that when $|z| \leq 1$, we have that:

	{\centering\begin{tabular}{l}
		$|E_p(z) - 1| = |z^{p+1}\sum\limits_{k = p + 1}^\infty a_k z^{k - p - 1}|$\\ [14pt]
		$\phantom{|E_p(z) - 1|} \leq |z|^{p+1}\sum\limits_{k = p + 1}^\infty |a_k||z^{k-p-1}| \leq |z|^{p+1}\sum\limits_{k = p + 1}^\infty 1|a_k| = |z|^{p+1}$. $\blacksquare$
	\end{tabular}\retTwo\par}
\end{myIndent}

\exTwo\ul{Lemma B:} Given any sequence $\{a_n\}_{n \in \mathbb{N}} \subseteq \mathbb{C} - \{0\}$ such that $|a_n| \to \infty$, we have for all\\ $r > 0$ that $\sum_{n=1}^\infty (\frac{r}{|a_n|})^n < \infty$.
\begin{myIndent}\exThreeP
	Proof:\\
	Since $|a_n| \to \infty$, we know there exists $N$ such that $|a_n| \geq 2r$ for all $n \geq N$. In turn, for all $n \geq N$ we have that $(\frac{r}{|a_n|})^{n} \leq \frac{1}{2^n}$. And since $\sum_{n=0}^\infty \frac{1}{2^n} < \infty$, we can conclude by the comparison test that $\sum_{n=1}^\infty (\frac{r}{|a_n|})^n < \infty$. $\blacksquare$\retTwo
\end{myIndent}

\hTwo With that we are ready to prove our desired theorem:\retTwo

\exTwo\ul{Weierstraß Factorization Problem:} Let $\{a_n\}_{n \in \mathbb{N}}$ be any sequence of distinct elements in $\mathbb{C}$ with $|a_n| \to \infty$, and also let $\{m_n\}_{n \in \mathbb{N}} \subseteq \mathbb{Z}_{> 0}$.

\begin{myIndent}\myComment
	Note that the condition that $|a_n| \to \infty$ is equivalent to guarenteeing that the set\\ $\{a_n : n \in \mathbb{N}\}$ has no limit points in $\mathbb{C}$.\retTwo
\end{myIndent}

We claim there exists an entire function $f: \mathbb{C} \to \mathbb{C}$ which only has zeros at the $a_n$ and with multiplicities given by $m_n$.

\begin{myIndent}\exThreeP
	Proof:\\
	To start off, it will be convenient (for the sake of notation) to replace $\{a_k\}_{k \in \mathbb{N}}$ with the sequence:

	{\centering $\underbrace{a_1, \ldots, a_1}_{m_1 \text{ times}}, \underbrace{a_2, \ldots, a_2}_{m_2 \text{ times}}, \underbrace{a_3, \ldots, a_3}_{m_3 \text{ times}} \ldots$ \retTwo\par}

	Importantly, doing this relabeling won't change the fact that $|a_n| \to \infty$ as $n \to \infty$. One other note is that we can without loss of generality assume $a_n \neq 0$ for any $n$. After all, if not we can just remove those terms from our sequence, solve the factorization problem to get an entire function $h(z)$, and then finally set $f(z) = z^m h(z)$ where $m$ is the number of zero terms we removed from the sequence.\retTwo

	Now our goal is to pick a sequence $\{p_n\}_{n \in \mathbb{N}} \subseteq \mathbb{Z}_{\geq 0}$ such that $f(z) = \prod_{n=1}^\infty E_{p_n}(\frac{z}{a_n})$\\ converges absolutely locally uniformly. Equivalently, we need $\sum_{n=1}^\infty |E_{p_n}(\frac{z}{a_n}) - 1|$ to\\ converge locally uniformly.\newpage

	Set $p_n = n - 1$ for all $n$. This will guarentee that $\sum_{n=1}^\infty |E_{p_n}(\frac{z}{a_n}) - 1|$ converges normally (and hence locally uniformly).
	\begin{myIndent}\exPPP
		Indeed let $K \subseteq \mathbb{C}$ be compact. Then there exists $r > 0$ such that $K \subseteq \overline{\Delta}(0, r)$. Since $|a_n| \to \infty$, there exists $N$ such that $|a_n| \geq r$ if $n \geq N$. And now by lemma A, we have that:

		{\centering $|E_{p_n}(\frac{z}{a_n}) - 1| \leq |\frac{z}{a_n}|^{p_n+1} \leq (\frac{r}{|a_n|})^{n}$ for all $z \in K$ \retTwo\par}

		Finally normal converge follows from lemma B.\retTwo
	\end{myIndent}

	It follows that $f(z) \coloneqq \prod_{n=1}^\infty E_{p_n}(\frac{z}{a_n})$ is an entire function with a zero at each $a_n$. $\blacksquare$\retTwo
\end{myIndent}

\hTwo Note that by no means did we show that the $p_n$ used in the above proof are unique. On the contrary, it is easy to justify that we can always modify at least finitely many of the $p_n$.\retTwo

By combining the above proof with the homework problem on \inLinkRap{Math 220b Set 2 Problem 5}{pages 549-550}, we can rewrite our prior result in the following more versatile form:\retTwo

\exTwo\ul{Theorem:} If $f$ is entire with countably many zeros, then $f(z) = z^m e^{h(z)} \prod_{n \in \mathbb{N}} E_{p_n}(\frac{z}{a_n})$ for some $h \in O(\mathbb{C})$, $\{a_n\}_{n \in \mathbb{N}} \subseteq \mathbb{C} - \{0\}$, $m \in \mathbb{Z}_{\geq 0}$ and $\{p_n\}_{n \in \mathbb{N}} \subseteq \mathbb{Z}_{\geq 0}$.



\newpage


































% \exTwo\ul{Theorem 3.18:} In a locally convex topological vector space $\mathcalli{X}$, $E \subseteq \mathcalli{X}$ is weakly (von\\ Neumann) bounded iff $E$ is originally (von Neumann) bounded.

% \begin{myIndent}\exThreeP
% 	Proof:\\
% 	Since the weak topology on $\mathcalli{X}$ is coarser then the original topology, it's trivial that being originally bounded means that a set is weakly bounded.\retTwo

% 	Meanwhile, suppose $E \subseteq \mathcalli{X}$ is weakly bounded and $U$ is any neighborhood of $0$ in the original topology on $\mathcalli{X}$. Since $\mathcalli{X}$ is locally convex, we can find a convex balanced open set $V$ in the original topology such that $0 \in V \subseteq \overline{V} \subseteq U$.
% 	\begin{myIndent}\exPPP
% 		Specifically, first let $V_1$ be an open set in $\mathcalli{X}$ such that $e \in V_1 \subseteq \overline{V_1} \subseteq U$. (We can do this by the above corollary). Secondly, we can use local convexity to find a convex open set $V_2$ such that $e \in V_2 \subseteq V_1$. Thirdly, by the work on \inLinkRap{Page 230 definition of balanced sets}{pages 230-232} we can find a balanced convex open set $V$ such that $e \in V \subseteq V_2$ Finally, as $V \subseteq V_2 \subseteq V_1 \subseteq \overline{V_1} \subseteq U$, we have that $\overline{V} \subseteq U$.\retTwo

% 		\begin{myIndent}\color{RawerSienna}
% 			I should note by the way that $\overline{V}$ is the closure of $V$ in the original topology.\retTwo
% 		\end{myIndent}
% 	\end{myIndent}

% 	Let $K \coloneqq \{\lambda \in \mathcalli{X}^* : |\lambda(x)| \leq 1 \text{ for all } x \in V\}$. Then we claim that:

% 	{\centering $\overline{V} = \{x \in \mathcalli{X} : |\lambda(x)| \leq 1 \text{ for all } \lambda \in K\}$ \retTwo\par}

% 	\begin{myIndent}\exPPP
% 		By the definition of $K$ we know that $V \subseteq \{x \in \mathcalli{X} : |\lambda(x)| \leq 1 \text{ for all } \lambda \in K\}$. Also, we know the latter set is closed since it is the intersection of a bunch of closed sets. Hence, we've proven that $\overline{V} \subseteq \{x \in \mathcalli{X} : |\lambda(x)| \leq 1 \text{ for all } \lambda \in K\}$.\retTwo

% 		To show the other inclusion, suppose $x \in X - \overline{V}$.



% 	\end{myIndent}


% \end{myIndent}








% Another vector-valued integral is the \udefine{Pettis integral}. To establish this integral, I shall follow Rudin's functional analysis book.\retTwo

% \exTwo\ul{T}



% ~~~~~~~~~~~~~~~~~~~~~~~~~~~~~~~~~~~~~~~~~~~~~~

\hypertarget{Page 378 Reference}{}
\hypertarget{Math 200a Set 4 Problem 3}{}


\hypertarget{Generalization page 189}{} 


\hypertarget{Math 200a problem set 2 what is a commutator and derived subgroup}{}



\hypertarget{Folland Proposition 11.2}{}
\hypertarget{Folland Lemma 7.15 reference}{}
\hypertarget{Folland Proposition 11.4(b)}{}

\hypertarget{Alireza lemma page 257}{}

\hypertarget{page 337 reference}{}

\hypertarget{Folland Proposition 10.1}{}


\hypertarget{Folland proposition 11.1}{}

\hypertarget{Ergodic reading group notes 3}{}

\hypertarget{existence and uniqueness diff eq notes}{}
\hypertarget{math 241a lecture 5}{}
\hypertarget{idk reference 2}{}

\hypertarget{idk reference 5}{}
\hypertarget{idk reference 6}{}

\end{document}


% \blect{Math 220 Homework:}\\

% \blab{Exercise III.2.2:} Prove that if $b_n, a_n$ are real and positive, $0 < b = \lim_{n \to \infty} b_n$, and $a = \limsup_{n \to \infty} a_n$, then $ab = \limsup_{n \to \infty} (a_nb_n)$.

% \begin{myIndent}\HexOne

% \end{myIndent}



% \hTwo Suppose $|G| = pq$ where $p < q$ are prime numbers. Then $s_q = 1$. Hence there exists a unique Sylow $q$-subgroup $Q$. Furthermore, $Q \lhd G$ and $Q$ is cylic with order $q$.\retTwo

% Next, let $P$ by a Sylow $p$-subgroup. Then because $Q \lhd G$, we have that $PQ < G$. Also, $|P \cap Q| \dividesDeprecated \gcd(p, q) = 1$. So, $P \cap Q = \{1\}$ and from there it follows that $|PQ| = pq = |G|$. So $G / Q = PQ / Q \cong P / (P \cap Q) \cong P$.
