\documentclass{book}

\usepackage{fontspec} % used to import Calibri
\usepackage{anyfontsize} % used to adjust font size

% needed for inch and other length measurements
% to be recognized
\usepackage{calc}

% for colors and text effects as is hopefully obvious
\usepackage[dvipsnames]{xcolor}
\usepackage{soul}

% control over margins
\usepackage[margin=1in]{geometry}
\usepackage[strict]{changepage}

\usepackage{mathtools}
\usepackage{amsfonts}
\usepackage{bm}

\usepackage[scr=rsfso, scrscaled=.96]{mathalpha}

% This is how I'm getting the nice caligraphy font :(
\DeclareMathAlphabet{\eulerscr}{U}{eus}{m}{n}
\newcommand{\mathcalli}[1]{\text{\scalebox{1.11}{$\eulerscr{#1}$}}}


\usepackage{amssymb} % originally imported to get the proof square
\usepackage{xfrac}
\usepackage[overcommands]{overarrows} % Get my preferred vector arrows...
\usepackage{relsize}

% Just am using this to get a dashed line in a table...
% Also you apparently want this to be inactive if you aren't
% using it because it slows compilation.
\usepackage{arydshln} \ADLinactivate 
\newenvironment{allowTableDashes}{\ADLactivate}{\ADLinactivate}

\usepackage{graphicx}
\graphicspath{{./158_Images/}}

\usepackage{tikz}
   \usetikzlibrary{arrows.meta}
   \usetikzlibrary{graphs, graphs.standard}

\usepackage{quiver} %commutative diagrams






\usepackage[hidelinks]{hyperref}
\newcommand{\inLinkRap}[2]{{\color{blue}\hyperlink{#1}{\textit{#2}}}}







\newfontfamily{\calibri}{Calibri}
\setlength{\parindent}{0pt}
\definecolor{RawerSienna}{HTML}{945D27}

% ~~~~~~~~~~~~~~~~~~~~~~~~~~~~~~~~~~~~~~~~~~~~~~~~~~
%Arrow Commands:

% Thank you Bernard, gernot, and Sigur who I copied this from:
% https://tex.stackexchange.com/questions/364096/command-for-longhookrightarrow
\renewcommand{\hookrightarrow}{\lhook\joinrel\rightarrow}
\renewcommand{\hookleftarrow}{\leftarrow\joinrel\rhook}
\newcommand{\hooklongrightarrow}{\lhook\joinrel\longrightarrow}
\newcommand{\hooklongleftarrow}{\longleftarrow\joinrel\rhook}
\newcommand{\hookxlongrightarrow}[2][]{\lhook\joinrel\xrightarrow[#1]{#2}}
\newcommand{\hookxlongleftarrow}[2][]{\xleftarrow[#1]{#2}\joinrel\rhook}

% Thank you egreg who I copied from:
% https://tex.stackexchange.com/questions/260554/two-headed-version-of-xrightarrow
\newcommand{\longrightarrowdbl}{\longrightarrow\mathrel{\mkern-14mu}\rightarrow}
\newcommand{\longleftarrowdbl}{\leftarrow\mathrel{\mkern-14mu}\longleftarrow}

\newcommand{\xrightarrowdbl}[2][]{%
  \xrightarrow[#1]{#2}\mathrel{\mkern-14mu}\rightarrow
}
\newcommand{\xleftarrowdbl}[2][]{%
  \leftarrow\mathrel{\mkern-14mu}\xleftarrow[#1]{#2}
}

\newcommand{\mRoman}[1]{%
   \textrm{\MakeUppercase{\romannumeral #1}}%
}



% ~~~~~~~~~~~~~~~~~~~~~~~~~~~~~~~~~~~~~~~~~~~~~~~~~~

\newcommand{\hOne}{%
   \color{Black}%
   \fontsize{14}{16}\selectfont%
}
\newcommand{\hTwo}{%
\color{Black}%
   \fontsize{13}{15}\selectfont%
}
% \newcommand{\scratchWork}{%
%    \color{PineGreen!85!Orange}
%    \fontsize{12}{14}\selectfont%
% }
\newcommand{\hThree}{%
   \color{Black}%
   \fontsize{12}{14}\selectfont%
}
\newcommand{\myComment}{%
   \color{RawerSienna}%
   \fontsize{12}{14}\selectfont%
}
\newcommand{\pracOne}{
   \color{BrickRed}%
   \fontsize{13}{15}\selectfont%
}
\newcommand{\pracTwo}{
   \color{Orange}%
   \fontsize{12}{14}\selectfont%
}
\newcommand{\why}{%
   \color{Orange}%
   \fontsize{12}{14}\selectfont%
	Why:
}
\newcommand{\exOne}{%
   \color{Purple}%
   \fontsize{14}{16}\selectfont%
}
\newcommand{\exTwo}{%
   \color{Purple}%
   \fontsize{13}{15}\selectfont%
}
\newcommand{\exThree}{%
   \color{Purple}%
   \fontsize{12}{14}\selectfont%
}
\newcommand{\exP}{%
   \color{Purple}%
   \fontsize{12}{14}\selectfont%
}
\newcommand{\exTwoP}{%
   \color{RedViolet}%
   \fontsize{13}{15}\selectfont%
}
\newcommand{\exThreeP}{%
   \color{RedViolet}%
   \fontsize{12}{14}\selectfont%
}
\newcommand{\exFourP}{%
   \color{RedViolet}%
   \fontsize{11}{13}\selectfont%
}
\newcommand{\exPP}{%
   \color{RedViolet}%
   \fontsize{12}{14}\selectfont%
}
\newcommand{\exPPP}{%
   \color{VioletRed}%
   \fontsize{12}{14}\selectfont%
}

% Homework standard below (God the bloat in the header is absurd...)
% ~~~~~~~~~~~~~~~~~~~~~~~~~~~~~~~~~~~~~~~~~~~~~~~~
\newcommand{\Hstatement}{%
   \color{MidnightBlue!90!Black}%
   \fontsize{12}{13}\selectfont%
}
\newcommand{\HexOne}{%
   \color{Purple}%
   \fontsize{12}{13}\selectfont%
}
\newcommand{\HexTwoP}{%
   \color{RedViolet}%
   \fontsize{12}{13}\selectfont%
}
\newcommand{\HexPPP}{%
   \color{VioletRed}%
   \fontsize{11}{12}\selectfont%
}

% ~~~~~~~~~~~~~~~~~~~~~~~~~~~~~~~~~~~~~~~~~~~~~~~~

\newcommand{\cyPen}[1]{{\vphantom{.}\color{Cerulean}#1}}
\newcommand{\redPen}[1]{{\vphantom{.}\color{Red}#1}}

\newenvironment{myIndent}{%
   \begin{adjustwidth}{2.5em}{0em}%
}{%
   \end{adjustwidth}%
}

\newenvironment{myDindent}{%
   \begin{adjustwidth}{5em}{0em}%
}{%
   \end{adjustwidth}%
}

\newenvironment{myTindent}{%
   \begin{adjustwidth}{7.5em}{0em}%
}{%
   \end{adjustwidth}%
}

\newenvironment{myConstrict}{%
   \begin{adjustwidth}{2.5em}{2.5em}%
}{%
   \end{adjustwidth}%
}

\newcommand{\udefine}[1]{{%
   \setulcolor{Red}%
   \setul{0.14em}{0.07em}%
   \ul{#1}%
}}

\newcommand{\uprop}[1]{{%
   \setulcolor{Purple}%
   \setul{0.14em}{0.07em}%
   \ul{#1} 
}}

\newcommand{\blab}[1]{\textbf{#1}}
\newcommand{\blect}[1]{{\color{MidnightBlue}\textbf{#1}}}

\newcommand{\uuline}[2][.]{%
{\vphantom{a}\color{#1}%
\rlap{\rule[-0.18em]{\widthof{#2}}{0.06em}}%
\rlap{\rule[-0.32em]{\widthof{#2}}{0.06em}}}%
#2}

\newcommand{\pprime}{{\prime\prime}}
\newcommand{\suchthat}{ \hspace{0.3em}s.t.\hspace{0.3em}}
\newcommand{\rea}[1]{\mathrm{Re}(#1)}
\newcommand{\ima}[1]{\mathrm{Im}(#1)}
\newcommand{\comp}{\mathsf{C}}
\newcommand{\trans}{\mathsf{T}}
\newcommand{\myHS}{ \hspace{0.5em}}
\newcommand{\gap}{\phantom{2}}

\newcommand{\GenLin}{\ensuremath{\mathrm{GL}}}
\newcommand{\Cay}{\ensuremath{\mathrm{Cay}}}

\newcommand{\myId}{\mathrm{Id}}
\newcommand{\myIm}{\mathrm{im}}
\newcommand{\Obj}{\mathrm{Obj}}
\newcommand{\Hom}{\mathrm{Hom}}
\newcommand{\End}{\mathrm{End}}
\newcommand{\Aut}{\mathrm{Aut}}

\newcommand{\df}{\mathrm{d}}
\newcommand{\Df}{\mathrm{D}}

\newcommand{\mcateg}[1]{{\bm{\mathsf{#1}}}}

\newcommand{\mdeg}{\mathrm{mdeg}\phantom{.}}

\newcommand{\card}{\mathrm{card}}
\newcommand{\supp}{\mathrm{supp}}
\newcommand{\diam}{\mathrm{diam}}
\newcommand{\conv}{\mathrm{conv}}
\newcommand{\opnorm}{\mathrm{op}}
\newcommand{\sgn}{\mathrm{sgn}}
\newcommand{\mSpan}{\mathrm{span}}
\newcommand{\Interior}{\mathop{\mathrm{Int}}}

\newcommand{\mMat}[1]{\mathbf{#1}}

\newcommand{\NBV}{\ensuremath{\mathrm{NBV}}}
\newcommand{\Acc}{\mathrm{Acc}}
\newcommand{\BV}{\ensuremath{\mathrm{BV}}}
\newcommand{\Var}{\ensuremath{\mathrm{Var}}}

\newcommand{\Alt}{\mathrm{Alt}}
\newcommand{\Sym}{\mathrm{Sym}}

\newcommand{\weakst}{weak$^*$ }

\newcommand{\radtimes}{\mathop{\widehat{\times}}}

\newcommand{\mMod}[1]{\phantom{a}(\mathrel{\mathrm{mod}} #1)}

% Thank you Gonzalo Medina and Moriambar who wrote this on stack exchange:
%https://tex.stackexchange.com/questions/74125/how-do-i-put-text-over-symbols%
\newcommand{\myequiv}[1]{\stackrel{\mathclap{\mbox{\footnotesize{$#1$}}}}{\equiv}}

% Thank you chs who wrote this on stack exchange:
%https://tex.stackexchange.com/questions/89821/how-to-draw-a-solid-colored-circle%
\newcommand{\filledcirc}[1][.]{\ensuremath{\hspace{0.05em}{\color{#1}\bullet}\mathllap{\circ}\hspace{0.05em}}}

%Thank you blerbl who wrote this on stack exchange:
%https://tex.stackexchange.com/questions/25348/latex-symbol-for-does-not-divide
\newcommand{\ndiv}{\hspace{-0.3em}\not|\hspace{0.35em}}

\newcommand{\mySepOne}[1][.]{%
   {\noindent\color{#1}{\rule{6.5in}{1mm}}}\\%
}
\newcommand{\mySepTwo}[1][.]{%
   {\noindent\color{#1}{\rule{6.5in}{0.5mm}}}\\%
}
\newcommand{\mySepThree}[1][.]{%
   {\noindent\color{#1}{\rule{6in}{0.25mm}}}\\%
}

\newenvironment{myClosureOne}[2][.]{%
   \color{#1}%
   \begin{tabular}{|p{#2in}|} \hline \\%
}{%
   \\ \hline \end{tabular}%
}

\newcommand{\retTwo}{\hfill\bigbreak}

\newcommand{\dispDate}[1]{{
   \color{Black}%
   \fontsize{20}{18}\selectfont%
   #1\retTwo
}}


\begin{document}
\setul{0.14em}{0.07em}
\calibri


\hTwo\dispDate{9/29/2025}

All homework problem's for math 220a will be coming from John Conway's book \textit{functions of one complex variable} (See \inLinkRap{bib citation 13}{item 13} in the bibliography).\retTwo

\Hstatement\blab{Exercise I.6.4:} Let $\Lambda$ be a circle lying in $S^2$. Then (by the definition of a circle) there is a unique\\ plane $P$ in $\mathbb{R}^3$ such that $P \cap S^2 = \Lambda$. So, take $P = \{(x_1, x_2, x_3) \in \mathbb{R}^3 : x_1\beta_1 + x_2\beta_2 + x_3\beta_3 = \ell\}$\\ where $(\beta_1, \beta_2, \beta_3)$ is a unit vector orthogonal to $P$ and $\ell \in \mathbb{R}$. Then show that if $\Lambda$ contains $(0, 0, 1)$, it's projection to $\mathbb{C}_\infty$ is a straight line, and meanwhile if $\Lambda$ doesn't contain $(0, 0, 1)$ then it's projection to $\mathbb{C}_\infty$ is another circle.

\begin{myIndent}\HexOne
	To start off, any point $z = x + iy \in \mathbb{C}$ corresponds to a point in $\Lambda$ iff:

	{\centering $\frac{2x}{|z|^2 + 1}\beta_1 + \frac{2y}{|z|^2 + 1}\beta_2 + \frac{|z|^2 - 1}{|z|^2 + 1}\beta^3 = \ell$ \retTwo\par}

	After some rearranging this becomes $2\beta_1 x + 2\beta_2 y + (x^2 + y^2 - 1)\beta_3 = \ell(x^2 + y^2 + 1)$. Or in other words $(\beta_3 - \ell)x^2 + 2\beta_1 x + (\beta_3 - \ell)y^2 + 2\beta_2 y = \ell + \beta_3$. And now we break off into two cases.

	\begin{enumerate}
		\item Suppose $\beta_3 = \ell$. Then we know that $0\beta_1 + 0\beta_2 + 1\beta_3 = \ell$. So $(0, 0, 1) \in \Lambda$. At the same time, all the square terms in our condition cancel and we are left with the requirement $2\beta_1 x + 2\beta_2 y = \ell + \beta_3$. This is the equation of a line. Hence, we've shown that the projection of $\Lambda$ onto $\mathbb{C}_\infty$ is a straight line unioned with the point at infinity.\retTwo
		
		\item Suppose $\beta_3 \neq \ell$. Then we know that $0\beta_1 + 0\beta_2 + 1\beta_3 \neq \ell$ and hence $(0, 0, 1) \notin \Lambda$. At the same time, we can now divide our equation from before by $\beta - \ell$ in order to get that $x^2 + 2\frac{\beta_1}{\beta_3 - \ell}x + y^2 + 2\frac{\beta_2}{\beta_3 - \ell}y = \frac{\ell + \beta_3}{\beta_3 - \ell}$. And by completing the square we have:
		
		{\centering $(x + \frac{\beta_1}{\beta_3 - \ell})^2 + (y + \frac{\beta_2}{\beta_3 - \ell})^2 = \frac{\beta_3 + \ell}{\beta_3 - \ell} + \frac{\beta_1^2 + \beta_2^2}{(\beta_3 - \ell)^2} = \frac{1 - \ell^2}{(\beta_3 - \ell)^2}$.\retTwo\par}

		So, the projection of $\Lambda$ onto $\mathbb{C}_\infty$ is a circle of radius $\frac{\sqrt{1 - \ell^2}}{\beta_3 - \ell}$ centered at $\frac{-\beta_1}{\beta_3 - \ell} + i\frac{-\beta_2}{\beta_3 - \ell}$.

		\begin{myIndent}\HexPPP
			As a side note, let $\mathbf{b} = (b_1, b_2, b_3)$ and consider any $\mathbf{y} = (y_1, y_2, y_3) \in \Lambda$. Then note by the Cauchy Schwarts inequality that $\ell^2 = (\mathbf{b} \cdot \mathbf{y})^2 \leq \|\mathbf{b}\|^2 \|\mathbf{y}\|^2 = 1 \cdot 1 = 1$. Hence,\\ $\sqrt{1 - \ell^2}$ is a well defined real number.\retTwo
		\end{myIndent}
	\end{enumerate}
\end{myIndent}

\hTwo I'll continue with this class on \inLinkRap{math 220a lecture 2}{page \_\_\_}

\mySepTwo

\blect{Math 200a (lecture 2):}\retTwo

Given a group $G$ and $a \in G$, we denote the \udefine{order} of $a$ as $o(a) \coloneqq o(\langle a \rangle)$
Now let $C_n$ be a cyclic group of order $n$. In other words, $C_n = \langle a \rangle$ where $o(a) = n$. What is $\Aut(C_n)$?\\ [-6pt]

\begin{myIndent}\Hstatement
	\blab{Set 1 problem 5:}
	\begin{itemize}
		\item[(a)] Suppose $\theta \in \Aut(C_n)$ and pick any $m \in \mathbb{Z}$ with $\theta(a) = a^m$. Then $\gcd(m, n) = 1$.
		
		\begin{myIndent}\HexOne
			Since $\theta$ is a bijection, we know that $\theta(a^k)$ for each $k \in \{0, \ldots, n-1\}$ is distinct. But also since $\theta$ is a group homomorphism, we know that $\theta(a^0) = a^0 = 1$ and $\theta(a^k) = (\theta(a))^k = (a^m)^k$ for $k > 0$. Hence, we must have that $o(a^m) = n$.\\ [-1pt] But now recall also that $o(a^m) = \frac{o(a)}{\gcd(o(a), m)} = \frac{n}{\gcd(n, m)}$. So, we must have that\\ [-3pt] $\gcd(n, m) = 1$.\newpage
		\end{myIndent}

		\item[(b)] Suppose $m \in \mathbb{Z}$ satisfies that $\gcd(m, n) = 1$ and define $\theta_m(a): C_n \to C_n$ by\\ $\theta_m(a^k) = (a^k)^m$. Then $\theta \in \Aut(C_n)$.
		
		\begin{myIndent}\HexOne
			To see that $\theta_m$ is a group homomorphism, just note that:
			
			{\centering\begin{tabular}{l}
				$\theta_m(a^{k_1}a^{k_2}) = \theta_m(a^{k_1 + k_2})$\\
				$\phantom{\theta_m(a^{k_1}a^{k_2})} = (a^{k_1 + k_2})^m = a^{m(k_1 + k_2)}$\\
				$\phantom{\theta_m(a^{k_1}a^{k_2}) = (a^{k_1 + k_2})^m} = a^{mk_1 + mk_2} = (a^{k_1})^m (a^{k_2})^m = \theta_m(a^{k_1})\theta_m(a^{k_2})$
			\end{tabular}\retTwo\par}

			Meanwhile note that $o(a^m) = \frac{o(a)}{\gcd(o(a), m)} = \frac{n}{1} = n$.\retTwo
			
			Therefore, because $\theta_m(a^k) = (\theta_m(a))^k = (a^m)^k$ since $\theta_m$ is a group\\ homomorphism, we know by pigeonhole principle that $\theta_m$ is both injective\\ and surjective.\retTwo
		\end{myIndent}

		\item[(c)] The mapping $(\mathbb{Z}/n\mathbb{Z})^\times \to \Aut(C_n)$ is an isomorphism.
		
		\begin{myIndent}\HexOne
			
		\end{myIndent}
	\end{itemize}
\end{myIndent}








% ~~~~~~~~~~~~~~~~~~~~~~~~~~~~~~~~~~~~~~~~~~~~~~

\hypertarget{Folland Proposition 10.1}{}
\hypertarget{bib citation 12}{}
\hypertarget{math 241a lecture 2}{}
\hypertarget{math 200a lecture 2}{}
\hypertarget{math 220a lecture 2}{}
\hypertarget{bib citation 13}{}

\end{document}




