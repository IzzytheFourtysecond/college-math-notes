\documentclass{book}

\usepackage{fontspec} % used to import Calibri
\usepackage{anyfontsize} % used to adjust font size

% needed for inch and other length measurements
% to be recognized
\usepackage{calc}

% for colors and text effects as is hopefully obvious
\usepackage[dvipsnames]{xcolor}
\usepackage{soul}

% control over margins
\usepackage[margin=1in]{geometry}
\usepackage[strict]{changepage}

\usepackage{mathtools}
\usepackage{amsfonts}
\usepackage{bm}

\usepackage[scr=rsfso, scrscaled=.96]{mathalpha}

% This is how I'm getting the nice caligraphy font :(
\DeclareMathAlphabet{\eulerscr}{U}{eus}{m}{n}
\newcommand{\mathcalli}[1]{\text{\scalebox{1.11}{$\eulerscr{#1}$}}}


\usepackage{amssymb} % originally imported to get the proof square
\usepackage{xfrac}
\usepackage[overcommands]{overarrows} % Get my preferred vector arrows...
\usepackage{relsize}

% Just am using this to get a dashed line in a table...
% Also you apparently want this to be inactive if you aren't
% using it because it slows compilation.
\usepackage{arydshln} \ADLinactivate 
\newenvironment{allowTableDashes}{\ADLactivate}{\ADLinactivate}

\usepackage{graphicx}
\graphicspath{{./158_Images/}}

\usepackage{tikz}
   \usetikzlibrary{arrows.meta}
   \usetikzlibrary{graphs, graphs.standard}

\usepackage{quiver} %commutative diagrams






\usepackage[hidelinks]{hyperref}
\newcommand{\inLinkRap}[2]{{\color{blue}\hyperlink{#1}{\textit{#2}}}}







\newfontfamily{\calibri}{Calibri}
\setlength{\parindent}{0pt}
\definecolor{RawerSienna}{HTML}{945D27}

% ~~~~~~~~~~~~~~~~~~~~~~~~~~~~~~~~~~~~~~~~~~~~~~~~~~
%Arrow Commands:

% Thank you Bernard, gernot, and Sigur who I copied this from:
% https://tex.stackexchange.com/questions/364096/command-for-longhookrightarrow
\renewcommand{\hookrightarrow}{\lhook\joinrel\rightarrow}
\renewcommand{\hookleftarrow}{\leftarrow\joinrel\rhook}
\newcommand{\hooklongrightarrow}{\lhook\joinrel\longrightarrow}
\newcommand{\hooklongleftarrow}{\longleftarrow\joinrel\rhook}
\newcommand{\hookxlongrightarrow}[2][]{\lhook\joinrel\xrightarrow[#1]{#2}}
\newcommand{\hookxlongleftarrow}[2][]{\xleftarrow[#1]{#2}\joinrel\rhook}

% Thank you egreg who I copied from:
% https://tex.stackexchange.com/questions/260554/two-headed-version-of-xrightarrow
\newcommand{\longrightarrowdbl}{\longrightarrow\mathrel{\mkern-14mu}\rightarrow}
\newcommand{\longleftarrowdbl}{\leftarrow\mathrel{\mkern-14mu}\longleftarrow}

\newcommand{\xrightarrowdbl}[2][]{%
  \xrightarrow[#1]{#2}\mathrel{\mkern-14mu}\rightarrow
}
\newcommand{\xleftarrowdbl}[2][]{%
  \leftarrow\mathrel{\mkern-14mu}\xleftarrow[#1]{#2}
}

\newcommand{\mRoman}[1]{%
   \textrm{\MakeUppercase{\romannumeral #1}}%
}



% ~~~~~~~~~~~~~~~~~~~~~~~~~~~~~~~~~~~~~~~~~~~~~~~~~~

\newcommand{\hOne}{%
   \color{Black}%
   \fontsize{14}{16}\selectfont%
}
\newcommand{\hTwo}{%
\color{Black}%
   \fontsize{13}{15}\selectfont%
}
% \newcommand{\scratchWork}{%
%    \color{PineGreen!85!Orange}
%    \fontsize{12}{14}\selectfont%
% }
\newcommand{\hThree}{%
   \color{Black}%
   \fontsize{12}{14}\selectfont%
}
\newcommand{\myComment}{%
   \color{RawerSienna}%
   \fontsize{12}{14}\selectfont%
}
\newcommand{\pracOne}{
   \color{BrickRed}%
   \fontsize{13}{15}\selectfont%
}
\newcommand{\pracTwo}{
   \color{Orange}%
   \fontsize{12}{14}\selectfont%
}
\newcommand{\why}{%
   \color{Orange}%
   \fontsize{12}{14}\selectfont%
	Why:
}
\newcommand{\exOne}{%
   \color{Purple}%
   \fontsize{14}{16}\selectfont%
}
\newcommand{\exTwo}{%
   \color{Purple}%
   \fontsize{13}{15}\selectfont%
}
\newcommand{\exThree}{%
   \color{Purple}%
   \fontsize{12}{14}\selectfont%
}
\newcommand{\exP}{%
   \color{Purple}%
   \fontsize{12}{14}\selectfont%
}
\newcommand{\exTwoP}{%
   \color{RedViolet}%
   \fontsize{13}{15}\selectfont%
}
\newcommand{\exThreeP}{%
   \color{RedViolet}%
   \fontsize{12}{14}\selectfont%
}
\newcommand{\exFourP}{%
   \color{RedViolet}%
   \fontsize{11}{13}\selectfont%
}
\newcommand{\exPP}{%
   \color{RedViolet}%
   \fontsize{12}{14}\selectfont%
}
\newcommand{\exPPP}{%
   \color{VioletRed}%
   \fontsize{12}{14}\selectfont%
}

% Homework standard below (God the bloat in the header is absurd...)
% ~~~~~~~~~~~~~~~~~~~~~~~~~~~~~~~~~~~~~~~~~~~~~~~~
\newcommand{\Hstatement}{%
   \color{MidnightBlue!90!Black}%
   \fontsize{12}{13}\selectfont%
}
\newcommand{\HexOne}{%
   \color{Purple}%
   \fontsize{12}{13}\selectfont%
}
\newcommand{\HexTwoP}{%
   \color{RedViolet}%
   \fontsize{12}{13}\selectfont%
}
\newcommand{\HexPPP}{%
   \color{VioletRed}%
   \fontsize{11}{12}\selectfont%
}

% ~~~~~~~~~~~~~~~~~~~~~~~~~~~~~~~~~~~~~~~~~~~~~~~~

\newcommand{\cyPen}[1]{{\vphantom{.}\color{Cerulean}#1}}
\newcommand{\redPen}[1]{{\vphantom{.}\color{Red}#1}}

\newenvironment{myIndent}{%
   \begin{adjustwidth}{2.5em}{0em}%
}{%
   \end{adjustwidth}%
}

\newenvironment{myDindent}{%
   \begin{adjustwidth}{5em}{0em}%
}{%
   \end{adjustwidth}%
}

\newenvironment{myTindent}{%
   \begin{adjustwidth}{7.5em}{0em}%
}{%
   \end{adjustwidth}%
}

\newenvironment{myConstrict}{%
   \begin{adjustwidth}{2.5em}{2.5em}%
}{%
   \end{adjustwidth}%
}

\newcommand{\udefine}[1]{{%
   \setulcolor{Red}%
   \setul{0.14em}{0.07em}%
   \ul{#1}%
}}

\newcommand{\uprop}[1]{{%
   \setulcolor{Purple}%
   \setul{0.14em}{0.07em}%
   \ul{#1} 
}}

\newcommand{\blab}[1]{\textbf{#1}}
\newcommand{\blect}[1]{{\color{MidnightBlue}\textbf{#1}}}

\newcommand{\uuline}[2][.]{%
{\vphantom{a}\color{#1}%
\rlap{\rule[-0.18em]{\widthof{#2}}{0.06em}}%
\rlap{\rule[-0.32em]{\widthof{#2}}{0.06em}}}%
#2}

\newcommand{\pprime}{{\prime\prime}}
\newcommand{\suchthat}{ \hspace{0.3em}s.t.\hspace{0.3em}}
\newcommand{\rea}[1]{\mathrm{Re}(#1)}
\newcommand{\ima}[1]{\mathrm{Im}(#1)}
\newcommand{\comp}{\mathsf{C}}
\newcommand{\trans}{\mathsf{T}}
\newcommand{\myHS}{ \hspace{0.5em}}
\newcommand{\gap}{\phantom{2}}

\newcommand{\GenLin}{\ensuremath{\mathrm{GL}}}
\newcommand{\Cay}{\ensuremath{\mathrm{Cay}}}

\newcommand{\myId}{\mathrm{Id}}
\newcommand{\myIm}{\mathrm{im}}
\newcommand{\Obj}{\mathrm{Obj}}
\newcommand{\Hom}{\mathrm{Hom}}
\newcommand{\End}{\mathrm{End}}
\newcommand{\Aut}{\mathrm{Aut}}

\newcommand{\df}{\mathrm{d}}
\newcommand{\Df}{\mathrm{D}}

\newcommand{\mcateg}[1]{{\bm{\mathsf{#1}}}}

\newcommand{\mdeg}{\mathrm{mdeg}\phantom{.}}

\newcommand{\divides}{\mathop{\mid}}

\newcommand{\card}{\mathrm{card}}
\newcommand{\supp}{\mathrm{supp}}
\newcommand{\diam}{\mathrm{diam}}
\newcommand{\conv}{\mathrm{conv}}
\newcommand{\opnorm}{\mathrm{op}}
\newcommand{\loc}{\mathrm{loc}}
\newcommand{\sgn}{\mathrm{sgn}}
\newcommand{\acc}{\mathrm{acc}}

\newcommand{\mSpan}{\mathrm{span}}
\newcommand{\Interior}{\mathop{\mathrm{Int}}}

\newcommand{\mMat}[1]{\mathbf{#1}}

\newcommand{\NBV}{\ensuremath{\mathrm{NBV}}}
\newcommand{\Acc}{\mathrm{Acc}}
\newcommand{\BV}{\ensuremath{\mathrm{BV}}}
\newcommand{\Var}{\ensuremath{\mathrm{Var}}}

\newcommand{\Alt}{\mathrm{Alt}}
\newcommand{\Sym}{\mathrm{Sym}}

\newcommand{\weakst}{weak$^*$ }

\newcommand{\radtimes}{\mathop{\widehat{\times}}}

\newcommand{\mMod}[1]{\phantom{a}(\mathrel{\mathrm{mod}} #1)}
\newcommand{\Fun}{\mathrm{Fun}}
\newcommand{\act}{\mathrm{act}}
\newcommand{\Fix}{\mathrm{Fix}}
\newcommand{\Sub}{\mathrm{Sub}}
\newcommand{\Cl}{\mathrm{Cl}}
\newcommand{\GL}{\mathrm{GL}}
\newcommand{\SL}{\mathrm{SL}}
\newcommand{\core}{\mathrm{core}}
\newcommand{\Syl}{\mathrm{Syl}}
\newcommand{\Iso}{\mathrm{Iso}}
\newcommand{\Homeo}{\mathrm{Homeo}}
\newcommand{\Inn}{\mathrm{Inn}}


\DeclareMathOperator{\lcm}{lcm}
\DeclareMathOperator{\Log}{Log}
\DeclareMathOperator{\symdif}{\triangle}
\DeclareMathOperator{\Average}{Average}
\DeclareMathOperator*{\AverageAst}{Average}

% Thank you Gonzalo Medina and Moriambar who wrote this on stack exchange:
%https://tex.stackexchange.com/questions/74125/how-do-i-put-text-over-symbols%
\newcommand{\myequiv}[1]{\stackrel{\mathclap{\mbox{\footnotesize{$#1$}}}}{\equiv}}

% Thank you chs who wrote this on stack exchange:
%https://tex.stackexchange.com/questions/89821/how-to-draw-a-solid-colored-circle%
\newcommand{\filledcirc}[1][.]{\ensuremath{\hspace{0.05em}{\color{#1}\bullet}\mathllap{\circ}\hspace{0.05em}}}

%Thank you blerbl who wrote this on stack exchange:
%https://tex.stackexchange.com/questions/25348/latex-symbol-for-does-not-divide
\newcommand{\ndiv}{\hspace{-0.3em}\not|\hspace{0.35em}}

\newcommand{\mySepOne}[1][.]{%
   {\noindent\color{#1}{\rule{6.5in}{1mm}}}\\%
}
\newcommand{\mySepTwo}[1][.]{%
   {\noindent\color{#1}{\rule{6.5in}{0.5mm}}}\\%
}
\newcommand{\mySepThree}[1][.]{%
   {\noindent\color{#1}{\rule{6in}{0.25mm}}}\\%
}

\newenvironment{myClosureOne}[2][.]{%
   \color{#1}%
   \begin{tabular}{|p{#2in}|} \hline \\%
}{%
   \\ \hline \end{tabular}%
}

\newcommand{\retTwo}{\hfill\bigbreak}

\newcommand{\dispDate}[1]{{
   \color{Black}%
   \fontsize{20}{18}\selectfont%
   #1\retTwo
}}


\begin{document}
\setul{0.14em}{0.07em}
\calibri

\pracOne\mySepTwo
Before getting to the next theorem, I need to learn some more about quotient topologies.\retTwo

\ul{Lemma 1:} If $f: X \to Y$ is a continuous map, then $f \times f: X \times X \to Y \times Y$ (given by $f \times f(x_1, x_2) = (f(x_1), f(x_2))$) is a continuous map.
\begin{myIndent}\pracTwo
	If $\pi_1, \pi_2$ are the natural projection maps from $Y \times Y$ to $Y$, then we have that $f \times f$ is continuous iff $\pi_1 \circ (f \times f)$ and $\pi_2 \circ (f \times f)$ is continuous. But clearly if $U \subseteq Y$ is an open set then $(\pi_1 \circ (f \times f))^{-1} = f^{-1}(U) \times X$ is open in $X \times X$. And a similar statement holds for $\pi_2 \circ (f \times f)$. Hence $f \times f$ is continuous.\retTwo
\end{myIndent}

\ul{Lemma 2:} If $p: X \to Y$ is an open continuous map, then $p \times p: X \times X \to Y \times Y$ is also an open continuous map.
\begin{myIndent}\pracTwo
	We know from the last lemma that $p$ is continuous. Meanwhile, suppose $U \subseteq X \times X$ is open. Then, we know there is a collection $\{V_\alpha \times W_\alpha\}_{\alpha \in A}$ of rectangles in $X \times X$ such that $V_\alpha, W_\alpha$ are open in $X$ and $U = \bigcup_{\alpha \in A}(V_\alpha \times W_\alpha)$. Then in turn:

	{\center $p \times p(U) = p \times p(\bigcup\limits_{\alpha \in A} (V_\alpha \times W_\alpha)) = \bigcup\limits_{\alpha \in A}p \times p(V_\alpha \times W_\alpha) = \bigcup\limits_{\alpha \in A}(p(V_\alpha) \times p(W_\alpha))$ \retTwo\par}

	But since $p$ is open, we know $p(V_\alpha) \times p(W_\alpha)$ is open in $Y \times Y$. So, $p \times p(U)$ is a union of open sets. $\blacksquare$\retTwo
\end{myIndent}

\ul{Corollary 3:} If $p: X \to Y$ is an open quotient map then $p \times p: X \times X \to Y \times Y$ is also an open quotient map.\retTwo

As for why this will be relevant, suppose $G$ is a topological group and $H \lhd G$. Then if we consider the natural projection map $\pi: G \to G/H$ and equip $G/H$ with the quotient topology (so that $\pi$ is a quotient map), then we have that $\pi$ is an open map.
\begin{myIndent}\pracTwo
	Why?\\
	Suppose $U \subseteq G$ is open. Then $\pi(U)$ is open in $g/H$ if and only if $\pi^{-1}(\pi(U))$ is open in\\ [1pt] $G$. But note that $\pi^{-1}(\pi(U)) = \{x \in G : \exists y \in U \suchthat x \in yH\} = UH = \bigcup_{h \in H}Uh$.\\ [1pt] Since each $Uh$ is open (since $G$ is a topological group), we thus have that $\pi^{-1}(\pi(U))$\\ [1pt] is open. $\blacksquare$\retTwo
\end{myIndent}

\ul{Theorem 4:} Suppose $X, Y, \overline{X}, \overline{Y}$ are topological spaces, $p_1: X \to \overline{X}$ and $p_2: Y \to \overline{Y}$ are quotient maps, and $f, \overline{f}$ are functions such that the following diagram commutes:
% https://q.uiver.app/#q=WzAsNCxbMCwwLCJYIl0sWzAsMSwiXFxvdmVybGluZXtYfSJdLFsxLDEsIlxcb3ZlcmxpbmV7WX0iXSxbMSwwLCJZIl0sWzAsMywiZiJdLFsxLDIsIlxcb3ZlcmxpbmV7Zn0iXSxbMCwxLCJwXzEiLDJdLFszLDIsInBfMiJdXQ==
\[\begin{tikzcd}
	X & Y \\
	{\overline{X}} & {\overline{Y}}
	\arrow["f", from=1-1, to=1-2]
	\arrow["{p_1}"', from=1-1, to=2-1]
	\arrow["{p_2}", from=1-2, to=2-2]
	\arrow["{\overline{f}}", from=2-1, to=2-2]
\end{tikzcd}\]

If $f$ is continuous then so is $\overline{f}$.
\begin{myIndent}\pracTwo
	Proof:\\
	Let $U \subseteq \overline{Y}$ be open. Then $p_1^{-1}(\overline{f}^{-1}(U)) = (p_2 \circ f)^{-1}(U)$ is open since $f$ and $p_2$ are\\ [-1pt] continuous. And since $p_1$ is a quotient map, we thus know that $\overline{f}^{-1}(U)$ is open. This shows that $\overline{f}$ is continuous. $\blacksquare$
	\begin{myIndent}\myComment
		Also I'm dumb because this theorem can be simply restated as an application of the universal property of quotient maps.\newpage
	\end{myIndent}
\end{myIndent}

\ul{Corollary 5:} Suppose $G$ is a topological group and $H \lhd G$. Then $G/H$ is a topological group when equipped with the quotient topology (induced by the natural projection map $\pi : G \to G/H$).
\begin{myIndent}\pracTwo
	By lemma 2 and corollary 3, we know that $\pi \times \pi : G \times G \to G/H \times G/H$ is a quotient map. Also, it is easy to verify that the following diagrams commute:
	% https://q.uiver.app/#q=WzAsOCxbMCwwLCJHXFx0aW1lcyBHIl0sWzAsMiwiRy9IIFxcdGltZXMgRy9IIl0sWzIsMiwiRy9IIl0sWzIsMCwiRyJdLFs3LDAsIkciXSxbNSwwLCJHIl0sWzUsMiwiRy9IIl0sWzcsMiwiRy9IIl0sWzAsMSwiXFxwaSBcXHRpbWVzIFxccGkiXSxbMSwyLCIoeEgsIHlIKSBcXG1hcHN0byB4eUgiXSxbMCwzLCIoeCwgeSkgXFxtYXBzdG8geHkiLDJdLFszLDIsIlxccGkiLDJdLFs2LDcsInhIIFxcbWFwc3RvICh4SCleey0xfSJdLFs0LDcsIlxccGkiLDJdLFs1LDYsIlxccGkiXSxbNSw0LCJ4IFxcbWFwc3RvIHheey0xfSIsMl1d
	\[\begin{tikzcd}
		{G\times G} && G &&& G && G \\
		\\
		{G/H \times G/H} && {G/H} &&& {G/H} && {G/H}
		\arrow["{(x, y) \mapsto xy}"', from=1-1, to=1-3]
		\arrow["{\pi \times \pi}", from=1-1, to=3-1]
		\arrow["\pi"', from=1-3, to=3-3]
		\arrow["{x \mapsto x^{-1}}"', from=1-6, to=1-8]
		\arrow["\pi", from=1-6, to=3-6]
		\arrow["\pi"', from=1-8, to=3-8]
		\arrow["{(xH, yH) \mapsto xyH}", from=3-1, to=3-3]
		\arrow["{xH \mapsto (xH)^{-1}}", from=3-6, to=3-8]
	\end{tikzcd}\]

	Therefore, by theorem 4 we know that the group operations on $G/H$ are continuous and so $G/H$ is a topological group. $\blacksquare$\retTwo
\end{myIndent}

\mySepTwo

\exTwo\ul{Proposition 11.3:} Let $G$ be a topological group.
\begin{enumerate}
	\item[(a)] If $G$ is $T_1$, then $G$ is Hausdorff.
	
	\begin{myIndent}\exThreeP
		Proof:\\
		If $G$ is $T_1$ and $x \neq y \in G$, then we can find an open set $U$ containing $e$ but\\ not containing $xy^{-1}$. And in turn, by \inLinkRap{Folland proposition 11.1}{Proposition 11.1(c) and (d)} there is a symmetric neighborhood $V$ of $e$ such that $xy^{-1} \notin VV$.\retTwo

		Now we know $Vx$ and $Vy$ are neighborhoods of $x$ and $y$. Also, we know\\ $Vx \cap Vy = \emptyset$. After all, if this weren't true than there would exist $v_1, v_2 \in V$\\ with $v_1 x = v_2 y$. And in turn we'd have that $xy^{-1} = v_1^{-1}v_2$. But since $V$ is\\ symmetric and $xy^{-1} \notin VV$, this is a contradiction.\retTwo
	\end{myIndent}

	\item[(b)] If $G$ is not $T_1$, then let $H$ be the closure of $\{e\}$. Then $H$ is a normal subgroup,\\ and if $G/H$ is given the quotient topology (i.e. $A \subseteq G/H$ is open if and only if\\ $B = \{x \in G : xH \in A\}$ is open), then $G/H$ is a Hausdorff topological group.
	
	\begin{myIndent}\exThreeP
		Proof:\\
		$H$ is a subgroup by Proposition 11.1(d). Also, to see why $H$ is normal, suppose $H^\prime$ is a conjugate of $H$ with $H^\prime \neq H$. Because the group product is continuous, we know that $H^\prime$ is also closed. And since $e \in H^\prime$, we know that $H^\prime \cap H$ is a closed proper subset of $H$ containing $\{e\}$. But this contradicts that $H$ is the closure of $\{e\}$. Hence, we conclude $H$ is normal.\retTwo

		Now that we know $H$ is a normal subgroup, we can consider the quotient group $G/H$ equipped with the quotient topology. I already proved that $G/H$ is a topological group. Also note that if $\overline{e}$ is the identity of $G/H$ then $\{\overline{e}\}$ is closed in $G/H$. (This is just  a result of the definition of a quotient topology plus the fact that $H$ is closed). In turn, we can see that every singleton in $G/H$ is closed. So, $G/H$ is $T_1$. And by part (a) we have that $G/H$ is Hausdorff. $\blacksquare$\newpage
	\end{myIndent}
\end{enumerate}

\hTwo The prior theorem shows that it is not much of a restriction to assume our topological groups are Hausdorff. Hence, we now define that $G$ is a \udefine{locally compact group} if $G$ is an LCH topological group.\retTwo

Suppose $G$ is a locally compact group. Then a Borel measure $\mu$ on $G$ is called \udefine{left invariant} (or \udefine{right invariant}) if $\mu(xE) = \mu(E)$ (or $\mu(Ex) = \mu(E)$) for all $x \in G$ and $E \in \mathcalli{B}_G$. Similarly, a linear functional $I$ on $C_c(G)$ is called \udefine{left-} or \udefine{right-invariant} if $I(L_x f) = I(f)$ or $I(R_x f) = I(f)$ for all $f \in C_c(G)$. Finally, a \udefine{left} (or \udefine{right}) \udefine{Haar measure} on $G$ is a nonzero left-invariant (or right-invariant) Radon measure $\mu$ on $G$.\retTwo

Let $C_c^+ \coloneqq \{f \in C_c(G) : f \geq 0 \text{ and } \|f\|_u > 0\}$.\retTwo

\exTwo\ul{Proposition 11.4:} Let $G$ be a locally compact group.
\begin{itemize}
	\item[(a)] A Radon measure $\mu$ on $G$ is a left Haar measure iff the measure $\widetilde{\mu}$ defined by\\ $\widetilde{\mu}(E) = \mu(E^{-1})$ is a right Haar measure.
	
	\begin{myIndent}\exThreeP
		Proof:\\
		If $\mu$ is a left Haar measure then:
		
		{\centering$\widetilde{\mu}(Ex) = \mu((Ex)^{-1}) = \mu(x^{-1} E^{-1}) = \mu(E^{-1}) = \widetilde{\mu}(E)$.\retTwo\par} 

		Meanwhile if $\widetilde{\mu}$ is a right Haar measure then:

		{\centering$\mu(xE) = \mu((E^{-1}x^{-1})^{-1}) = \widetilde{\mu}(E^{-1}x^{-1}) = \widetilde{\mu}(E^{-1}) = \mu((E^{-1})^{-1}) = \mu(E)$.\retTwo\par}
		
		\begin{myIndent}\myComment
			Also as a side note, you can see that $\widetilde{\mu}$ is a well-defined measure since it is merely the \udefine{pushforward measure} of $\mu$ by the inversion map. (Back on \inLinkRap{Folland Proposition 10.1}{page 193} I was calling this the image measure\dots)\retTwo
		\end{myIndent}
	\end{myIndent}

	\item[(b)] A nonzero Radon measure $\mu$ on $G$ is a left Haar measure iff $\int f \df \mu = \int L_y f \df \mu$ for all $f \in C_c^+$ and $y \in G$.
	
	\begin{myIndent}\exThreeP
		$(\Longrightarrow)$\\
		If $\mu$ is a left Haar measure then it is obvious that $\int f \df \mu = \int L_y f \df \mu$ whenever $f$ is a simple function. And by the monotone convergence theorem we can extend this to all $f \in C_c^+$.\retTwo

		$(\Longleftarrow)$\\
		Note that $\supp(L_yf) = y \cdot \supp(f)$ for all $y \in G$ and $f \in C_c(G)$.
		\begin{myIndent}\exPPP
			After all, $L_yf(x) \neq 0$ iff $f(y^{-1}x) \neq 0$. So if $A = \{x : L_yf(x) \neq 0\}$ and $B = \{x : f(x) \neq 0\}$ then $x \in A$ iff $y^{-1}x \in B$ and that happens iff $x \in yB$. And finally, since translation by $y$ is a homeomorphism on $G$, we have that:
		
			{\centering$\supp(L_y f) = \overline{A} = \overline{yB} = y\overline{B} = y \cdot \supp(f)$.\retTwo\par}
		\end{myIndent}

		Thus, for any open set $U \subseteq G$ we know that if $f \in C_c(G)$ with $\supp(f) \subseteq U$ then $L_y f \in C_c(G)$ with $\supp(L_y f) \subseteq y U$. Similarly, if $f \in C_c(G)$ with $\supp(f) \subseteq yU$ then $L_{y^{-1}}f \in C_c(G)$ with $\supp(L_{y^{-1}}f) \subseteq U$. And when you consider for all open sets $V \subseteq G$ that $\mu(V) = \sup\{\int f \df \mu : f\in C_c(G), 0 \leq f \leq 1, \supp(f) \subseteq V\}$,\\ it becomes clear that $\mu(yU) = \mu(U)$ for all $y \in G$ and open sets $U \subseteq G$.\newpage

		As for the case that $E$ is a general Borel subset of $G$, we can just approximate $E$ and $xE$ using open sets.\retTwo
	\end{myIndent}

	\item[(c)] If $\mu$ is a left Haar measure on $G$, then $\mu(U) > 0$ for every nonempty open $U \subseteq G$ and $\int f \df \mu > 0$ for all $f \in C_c^+$.
	
	\begin{myIndent}\exThreeP
		Proof:\\
		Since $\mu \neq 0$, we can show by the regularity properties of $\mu$ that there exists a compact set $K \subseteq G$ with $\mu(K) > 0$. Then, for any open nonempty set $U \subseteq G$ we have that $K$ can be covered by finitely many left translates of $U$. Hence, $\mu(U) > 0$.\retTwo

		Next, if $f \in C_c^+$ then let $U \coloneqq \{x : f(x) > \frac{1}{2}\|f\|_u\}$. $U$ is open since $f$ is continuous. Therefore, since $\mu(U) > 0$ we have that $\int f \df \mu \geq \frac{1}{2}\|f\|_u \mu(U) > 0$.\retTwo
	\end{myIndent}

	\item[(d)] If $\mu$ is a left Haar measure on $G$ then $\mu(G) < \infty$ iff $G$ is compact.
	
	\begin{myIndent}\exThreeP
		Proof:\\
		The $(\Longleftarrow)$ direction is obvious from the definition of a Radon measure. Meanwhile, suppose $G$ is not compact and let $V$ be a compact neighborhood of $e$. Then we know that $G$ cannot be covered by finitely many translates of $V$ (lest $G$ be a finite union of compact sets). So, we may find a sequence $\{x_n\}_{n \in \mathbb{N}}$ such that $x_n \in \bigcup_{j=1}^{n-1} x_j V$ for all $n$.\retTwo

		Next, by proposition 11.1 we can find a symmetric neighborhood $U$ of $e$ with\\ $UU \subseteq V$. Importantly, if $m > n$ and $x_n U \cap x_m U \neq \emptyset$ then we would have\\ that $x_m \in x_n UU \subseteq x_n V$. But that contradicts how we picked our $x_n$. Hence, we\\ know $\{x_n U\}_{n \in \mathbb{N}}$ is a disjoint sequence of sets. And since $\mu(x_n U) = \mu(U) > 0$,\\ we know that $\mu(G) \geq \mu(\bigcup_{n \in \mathbb{N}} x_n U) = \sum_{n \in \mathbb{N}} \mu(x_n U) = \infty$. $\blacksquare$\retTwo
	\end{myIndent}
\end{itemize}

\hTwo\hypertarget{page 353 reference}{I'll} continue with actually constructing a Haar measure later on \inLinkRap{idk reference 4}{page \_\_\_}.

\mySepTwo

\blect{Math 200a Notes:}\retTwo

If $\sigma \in S_n$ we define $\supp(\sigma) \coloneqq \{i \in \{1, \ldots, n\} : \sigma(i) \neq i\}$. Note that if we consider the obvious injection $S_n \hookrightarrow S_{n+1} \hookrightarrow \cdots$ then $\supp(\sigma)$ doesn't change. Also, we let $\Fix(\sigma) = \{1, \ldots, n\} - \supp(\sigma)$.
\begin{myIndent}\myComment
	Note that this definition of $\Fix(\sigma)$ is equivalent to the set of fixed points of $\sigma$ with respect to the obvious group action $S_n \curvearrowright \{1, \ldots, n\}$.\retTwo
\end{myIndent}

We say $\sigma_1, \sigma_2 \in S_n$ are \udefine{disjoint} if $\supp(\sigma_1) \cap \supp(\sigma_2) = \emptyset$.\retTwo

\exTwo\ul{Lemma:} If $\sigma_1, \sigma_2$ are disjoint then $\sigma_1 \circ \sigma_2 = \sigma_2 \circ \sigma_1$ and $o(\sigma_1 \circ \sigma_2) = \lcm(o(\sigma_1), o(\sigma_2))$.
\begin{myIndent}\exThreeP
	Proof:\\
	For all $i \in \{1, \ldots, n\}$, if $i \in \Fix(\sigma) \cap \Fix(\sigma_2)$ then $\sigma_1\circ \sigma_2(i) = i = \sigma_2 \circ \sigma_1(i)$.\newpage
	
	Meanwhile, if $i \in \Fix(\sigma_1)$ and $i \notin \Fix(\sigma_2)$, then $\sigma_2 \circ \sigma_1(i) = \sigma_2(i)$. But now note that:
	
	{\centering\begin{tabular}{l}
		$i \notin \Fix(\sigma_2) \Longrightarrow i \in \supp(\sigma_2) \Longrightarrow \sigma_2(i) \in \supp(\sigma_2)$\\ [2pt]
		$\phantom{i \notin \Fix(\sigma_2) \Longrightarrow i \in \supp(\sigma_2)} \Longrightarrow \sigma_2(i) \notin \supp(\sigma_1)$\\ [2pt]
		$\phantom{i \notin \Fix(\sigma_2) \Longrightarrow i \in \supp(\sigma_2)} \Longrightarrow \sigma_2(i) \in \Fix(\sigma_1) \Longrightarrow \sigma_1(\sigma_2(i)) = \sigma_2(i)$.
	\end{tabular}\retTwo\par}

	So, $\sigma_1 \circ \sigma_2(i) = \sigma_2 \circ \sigma_1(i)$.\retTwo

	The case where $i \notin \Fix(\sigma_1)$ and $i \in \Fix(\sigma_2)$ is similar. And since $\sigma_1$ and $\sigma_2$ are disjoint, we never have the fourth case. This proves $\sigma_1 \circ \sigma_2 = \sigma_2 \circ \sigma_1$.\retTwo

	Next, let $o(\sigma_i) = d_i$ for both $i$ and let $\ell = \lcm(d_1, d_2)$. Then we know from before that $(\sigma_1 \circ \sigma_2)^\ell = \sigma_1^{\ell} \sigma_2^{\ell} = \myId$. Hence, we must have that $o(\sigma_1 \sigma_2)$ divides $\ell$. On the other hand, suppose $(\sigma_1 \circ \sigma_2)^k = 1$. Then by considering the obvious action $\langle \sigma_2 \rangle \curvearrowright \{1, \ldots, n\}$, we know for all $j \in \supp(\sigma_2)$ that $\langle \sigma_2\rangle \cdot j \subseteq \supp(\sigma_2)$ and in turn $\langle \sigma_2 \rangle \cdot j \subseteq \Fix(\sigma_1)$. It follows that $j = \sigma_1^k \circ \sigma_2^k(j) = \sigma_2^k(j)$ for all $j \in \supp(\sigma_2)$. Hence $\sigma_2^k = \myId$ and we've shown that $d_2 \divides k$.\retTwo

	By analogous reasoning we can show that $d_1 \divides k$. So, $\ell \divides k$. $\blacksquare$\retTwo
\end{myIndent}

\hTwo For the sake of convenience I'm just going to refer to $\sigma_2 \circ \sigma_1$ as $\sigma_2 \sigma_1$ from now on.\retTwo

If $k \geq 2$, we say $\sigma \in S_n$ is a \udefine{$k$-cycle} if there are distinct elements $a_1, \ldots, a_k \in \{1, \ldots, n\}$ such that $\supp(\sigma) = \{a_1, \ldots, a_k\}$, $\sigma(a_i) = \sigma(a_{i+1})$ for $i < k$, and $\sigma(a_k) = a_1$. We typically denote a $k$-cycle as $(a_1\gap a_2\gap \cdots\gap a_k)$.\retTwo

A one-cycle is trivial since it is just the identity permutation. So we typically don't count them as "cycles".\retTwo

\exTwo\ul{Theorem:} For all $\sigma \in S_n$ there are disjoint nontrivial cycles $\sigma_1, \ldots, \sigma_m$ such that\\ $\sigma= \sigma_1 \cdots \sigma_m$. Also, this decomposition is unique up to permuting the terms. We call\\ this decomposition a \udefine{cycle decomposition} of $\sigma$.
\begin{myIndent}\exThreeP
	Proof:\\
	Given any permutation $\sigma \in S_n$, we can say that $\langle \sigma \rangle \curvearrowright \{1, \ldots, n\}$ via the action\\ $\sigma \cdot i = \sigma(i)$. Using this fact we can easily show the existence of a cycle decomposition\\ of $\sigma$.
	\begin{myIndent}\exPPP
		Given any $a \in \{1, \ldots, n\}$ we know there is a unique smallest integer $k$ such that $\sigma^k(a) = a$. So, define $\sigma_{(a)} = (a\gap \sigma(a)\gap\sigma^2(a)\gap\cdots\gap \sigma^{k-1}(a))$. By doing this for all $a \in \{1, \ldots, n\}$, we can see that $\supp(\sigma_{(a)}) = \langle \sigma \rangle \cdot a$ (if $a$ has a non-singleton orbit). Also, it is easy to see that $\supp(\sigma_{(a)}) = \supp(\sigma_{(b)})$ implies that $\sigma_{(a)} = \sigma_{(b)}$. Hence $\Sigma \coloneqq \{\sigma_{(a)} : a \in \{1, \ldots, n\} \text{ and } \sigma(a) \neq a \}$ is a collection of disjoint cycles. And it is easy to see to that $\sigma$ equals the product of the elements of $\Sigma$ in any order.\retTwo
	\end{myIndent}

	As for proving the uniqueness of this cycle decomposition, note that if $\sigma = \tau_1 \tau_2 \cdots \tau_\ell$ where each $\tau_i$ is a disjoint cycle, then $a^\prime \in \supp(\tau_i)$ implies that $\tau_i = \sigma_{(a^\prime)}$ from before.
	\begin{myIndent}\exPPP
		To see this, first note that if $a^\prime \in \supp(\tau_i)$ then we can show from the disjointness of the $\tau_i$ that $\sigma^k(a^\prime) = \tau_i^k(a^\prime)$ for all integers $k$. But then this alligns with how we defined the permutation $\sigma_{(a^\prime)} \in \Sigma$.\newpage
	\end{myIndent}

	This says that each $\tau_i$ is equal to some $\sigma_{(a)} \in \Sigma$. Because each of the $\tau_i$ is disjoint, we know this correspondance is injective. Also, it's surjective since otherwise the products of the $\tau_i$ and the $\sigma_{(a)}$ wouldn't be the common permutation $\sigma$. $\blacksquare$\retTwo

	\begin{myIndent}\myComment
		I realize I handwaved this entire proof. But at least I thought about it and wrote down something. Professor Alireza meanwhile skipped proving this.\retTwo
	\end{myIndent}
\end{myIndent}

\hTwo The \udefine{cycle type} of $\sigma$ is $(\ell_1 \geq \ell_2 \geq \cdots \geq \ell_m)$ where $\{\ell_1, \ldots, \ell_m\}$ is the set of sizes of the orbits of $\langle \sigma \rangle \curvearrowright \{1, \ldots, n\}$.\retTwo

\exTwo\ul{Lemma:} $o(\sigma) = \lcm(\ell_1, \ldots, \ell_m)$ where $(\ell_1 \geq \cdots \geq \ell_m)$ is the cycle type of $\sigma$.

\begin{myIndent}\exThreeP
	Proof:\\
	Take a cycle decomposition of $\sigma$ and then inductively apply the lemma at the beginning of this section.\retTwo
\end{myIndent}

\exTwo\ul{Lemma:} Let $a_1, \ldots, a_{m}, a_{m+1}, \ldots, a_{m +n}$ be distinct elements.
\begin{enumerate}
	\item[(a)] $(a_1\gap\cdots\gap a_m)(a_m\gap\cdots\gap a_{m+n}) = (a_1\gap \cdots \gap a_{m + n})$.

	\item[(b)] For any $\sigma \in S_n$, $\sigma(a_1\gap a_2\gap\cdots\gap a_m)\sigma^{-1} = (\sigma(a_1)\gap \sigma(a_2)\gap \cdots \gap \sigma(a_m))$.
	
	\begin{myIndent}\exThreeP
		Proof:\\
		Showing part (a) is as simple as just going through and showing the permutations on either side of the identity map things to the same places.\retTwo

		To show part (b), recall from \inLinkRap{Alireza lemma page 257}{page 257} that $\Fix(\sigma\tau\sigma^{-1}) = \sigma \cdot \Fix(\tau)$. (where\\ $A \subseteq \{1, \ldots, n\}$ implies that $\sigma \cdot A \coloneqq \{\sigma(a) : a \in A\}$). In turn:
		
		{\centering$\supp(\sigma \tau \sigma^{-1}) = \sigma \cdot \supp(\tau)$\retTwo\par}

		And in particular, $\supp(\sigma(a_1\gap \cdots\gap a_m)\sigma^{-1}) = \{\sigma(a_1), \ldots, \sigma(a_m)\}$. So we know that the permutations on both sides of our proposed equation have the same support.\retTwo
		
		Next note for any $k < m$ that $\sigma(a_1 \gap \ldots \gap a_m)\sigma^{-1}(\sigma(a_k)) = \sigma(a_{k+1})$ (and similarly plugging in $\sigma(a_m)$ gives $\sigma(a_1)$). So the permutations agree on their supports and thus everywhere. $\blacksquare$\retTwo
	\end{myIndent}
\end{enumerate}

\ul{Proposition:} $\sigma_1, \sigma_2 \in S_n$ are conjugates if and only if they have the same cycle types.

\begin{myIndent}\exThreeP
	$(\Longrightarrow)$\\
	Suppose $\sigma = \tau_1\cdots \tau_m$ is a cycle decomposition such that $\tau_i$ has cycle length $\ell_i$ and\\ $\ell_1 \geq \ell_2 \geq \cdots \geq \ell_m$. Then the cycle type of $\sigma_1$ is $(\ell_1 \geq \cdots \geq \ell_m \geq 1 \geq \cdots \geq 1)$\\ (with $n = \sum_{i=1}^m \ell_i$ many $1$s at the end). Also, if $\sigma_2 = \sigma \sigma_1 \sigma^{-1}$ then:

	{\centering $\sigma_2 = \sigma \tau_1 \cdots \tau_m \sigma^{-1} = (\sigma \tau_1 \sigma^{-1})(\sigma \tau_2 \sigma^{-1})\cdots(\sigma \tau_m \sigma^{-1})$ \retTwo\par}

	Additionally, it's not hard to see that each $(\sigma \tau_i \sigma^{-1})$ is a disjoint cycle of the same length as $\tau_i$. Hence $(\sigma \tau_1 \sigma^{-1})(\sigma \tau_2 \sigma^{-1})\cdots(\sigma \tau_m \sigma^{-1})$ is a cycle decomposition of $\sigma_2$ and the order type of $\sigma_2$ is also $(\ell_1 \geq \cdots \geq \ell_m \geq 1 \cdots \geq 1)$.\newpage

	$(\Longleftarrow)$\\
	Suppose the common cycle type is $(\ell_1 \geq \ell_2 \geq \cdots \geq \ell_m)$. Then there exists $n < m$ and cycle decompositions:
	\begin{itemize}
		\item $\sigma_1 = (a_1^{(1)}\gap\cdots\gap a_{\ell_1}^{(1)})(a_1^{(2)}\gap\cdots\gap a_{\ell_2}^{(2)})\cdots (a_1^{(n)}\gap\cdots \gap a_{\ell_n}^{(n)})$
		\item $\sigma_2 = (b_1^{(1)}\gap\cdots\gap b_{\ell_1}^{(1)})(b_1^{(2)}\gap\cdots\gap b_{\ell_2}^{(2)})\cdots (b_1^{(n)}\gap\cdots \gap b_{\ell_n}^{(n)})$
		\begin{myIndent}\myComment
			(Note all $a_i^{(j)}$ are distinct and similarly all $b_i^{(j)}$ are distinct.)\retTwo
		\end{myIndent}
	\end{itemize}

	For every $i$ such that $\ell_i = 1$ we pick a different $a_1^{(i)}$ among the fixed points of $\sigma_1$. Similarly, we pick a different $b_1^{(i)}$ among the fixed points of $\sigma_2$ for each $i$ with $\ell_i = 1$. Then finally, we define the permutation $\tau \in S_n$ by $\tau(a_j^{(i)}) = b_j^{(i)}$. Then:

	{\centering\begin{tabular}{l}
		$\tau \sigma_1 \tau^{-1} = (\tau(a_1^{(1)})\gap \cdots \gap \tau(a_{\ell_1}^{(1)}))\cdots(\tau(a_1^{(m)})\gap \cdots \gap \tau(a_{\ell_m}^{(m)}))$\\ [3pt]
		$\phantom{\tau \sigma_1 \tau^{-1}} = (b_1^{(1)} \gap \cdots \gap b_{\ell_1}^{(1)})\cdots(b_1^{(m)}\gap \cdots \gap b_{\ell_m}^{(m)}) = \sigma_2$. $\blacksquare$
	\end{tabular}\retTwo\par}
\end{myIndent}

\ul{Corollary:} The number of conjugate classes of $S_n$ is equal to the number of integer\\ partitions of $n$ (see my paper math 188 notes).

\hTwo\mySepTwo

By noting that $(a_1\gap a_2 \gap \cdots \gap a_n) = (a_1\gap a_2)(a_2 \gap a_3)\cdots (a_{n-1}\gap a_n)$ we can see that every\\ permutation can be written as a product of \udefine{transpositions} (i.e. $2$-cycles). That said,\\ except in trivial cases there are many different ways to write a permutation as a\\ product of transpositions. For example, $(1\gap 2)(2 \gap 3)(1 \gap 2) = (1 \gap 3)$. So, is it still possible\\ to characterize permutations somehow by how they are expressed as products of\\ transpositions?\retTwo

Note that $S_n \curvearrowright \mathbb{Z}[x_1, \ldots, x_n]$ by $(\sigma \cdot f)(x_1, \ldots, x_n) = f(x_{\sigma(1)},\ldots, x_{\sigma(n)})$. Also, this\\ group action importantly has the properties that $\sigma \cdot (f + g) = (\sigma \cdot f) + (\sigma \cdot g)$ and\\ $\sigma \cdot (fg) = (\sigma \cdot f)(\sigma \cdot g)$.\retTwo

If we let $\Delta(x_1, \ldots, x_n) \coloneqq \prod_{i < j} (x_i - x_j)$, then we we have that

{\centering$\Delta^2(x_1, \ldots, x_n) = (-1)^{\frac{n(n-1)}{2}} \prod_{i \neq j}(x_i - x_j)$.\retTwo\par}

Hence, $(\sigma \cdot \Delta^2) = \Delta^2$ for all $\sigma \in S_n$ (i.e. $\Delta^2$ is a \udefine{symmetric polynomial}). And since $(\sigma \cdot \Delta)^2 = (\sigma \cdot \Delta^2) = \Delta^2$, we must have that $\sigma \cdot \Delta = \pm \Delta$ for each $\sigma \in S_n$. So, there exists a map $\varepsilon : S_n \to \{-1, 1\}$ such that $\sigma \cdot \Delta = \varepsilon(\sigma) \Delta$.\retTwo

Note that $\sigma \cdot (\sigma^\prime \cdot \Delta) = (\sigma \sigma^\prime) \cdot \Delta = \varepsilon(\sigma \sigma^\prime) \Delta$. But we also have that:

{\centering $\sigma \cdot (\sigma^\prime \cdot \Delta) = \sigma \cdot (\varepsilon(\sigma^\prime)\Delta) = \varepsilon(\sigma^\prime) (\sigma \cdot \Delta) = \varepsilon(\sigma^\prime)\varepsilon(\sigma)\Delta$. \retTwo\par}

Therefore $\varepsilon(\sigma \sigma^\prime) = \varepsilon(\sigma)\varepsilon(\sigma^\prime)$. And when you also consider that $\varepsilon(\myId) = +1$, this proves that $\varepsilon$ is a group homomorphism.\retTwo

One more note is that if $\tau$ is a transposition then $\varepsilon(\tau) = -1$. To see this, just note that if $\tau = (i\gap j)$ where $i < j$, then $\tau \cdot \Delta = (-1)^{2(j-i - 1) + 1}\Delta = -\Delta$.\newpage

Thus, if $\sigma \in S_n$ is decomposed into a product of transpositions, we must have that the number of transpositions in that product is even if $\varepsilon(\sigma) = +1$ and odd if $\varepsilon(\sigma) = -1$. In other words, \udefine{even and odd permutations} are well-defined.\retTwo

We define $A_n \coloneqq \ker(\varepsilon) = \{\text{all even permutations of } S_n\}$. Note that $A_n \lhd S_n$ (since it is a kernel of a group homomorphism).\retTwo

\exTwo\ul{Theorem:} If $n \geq 2$ then $[S_n : A_n] = 2$.
\begin{myIndent}\exThreeP
	Proof:\\
	By the first isomorphism theorem we have that:
	
	{\centering$S_n / A_n = S_n / \ker(\varepsilon) \cong \myIm(\varepsilon) = \{-1, +1\}$.\retTwo\par}
	
	Hence $[S_n : A_n] = 2$. $\blacksquare$
\end{myIndent}

\hTwo\mySepTwo

The last theorem that can be tested on the midterm tomorrow is that $A_n$ is a simple group (if $n \geq 5$). So I'll try to prove this and then do some exercises.\retTwo

\exTwo\ul{Lemma:} $A_n$ is generated by the set of $3$-cycles.
\begin{myIndent}\exThreeP
	Proof:\\
	It is enough to show that the product of $2$ transpositions is in the subgroup generated by $3$ cycles. Fortunately, if $a, b, c, d$ are distinct elements in $\{1, \ldots, n\}$ then:
	\begin{itemize}
	\item $(a\gap b)(a\gap b) = \myId$
	\item $(a \gap b)(b \gap c) = (a\gap b \gap c)$
	\item $(a \gap b)(c \gap d) = (a \gap b)(b \gap c)(b \gap c)(c \gap d) = (a\gap b \gap c)(b \gap c \gap d)$. $\blacksquare$\retTwo
	\end{itemize}
\end{myIndent}

\ul{Lemma:} If $N \lhd A_n$ and $N$ has a $3$-cycle then $N = A_n$.
\begin{myIndent}\exThreeP
	Proof:\\
	Suppose $(a_1\gap a_2 \gap a_3) \in N$. Then for any distinct $b_1, b_2, b_3 \in \{1, \ldots, n\}$ let $\sigma \in S_n$ be a permutation such that $\sigma(a_1) = b_1$, $\sigma(a_2) = b_2$, and $\sigma(a_3) = b_3$. Since $N$ is normal, we have that $(b_1\gap b_2 \gap b_3) = \sigma (a_1 \gap a_2 \gap a_3)\sigma^{-1} \in N$. So, $N$ contains all $3$-cycles. And by the last lemma this means that $A_n < N$. $\blacksquare$\retTwo
\end{myIndent}

\ul{Theorem:} If $n \geq 5$ then $A_n$ is simple.
\begin{myIndent}\exThreeP
	Proof:\\
	We shall first show that $A_5$ is simple. Note that $|A_5| = \frac{5!}{2} = 60 = 5 \cdot 2^2 \cdot 3$. So suppose for the sake of contradiction that there exists a subgroup $N \lhd A_5$ with $1 \lneqq N \lneqq A_5$.\retTwo

	We can't have that $3$ divides $|N|$.
	\begin{myIndent}\exPPP
		If $3$ divides $|N|$ then we would know by Cauchy's theorem that $N$ contains an\\ element of order $3$, say $\sigma$. So, let $(\ell_1 \geq \cdots \geq \ell_m)$ be the cycle type of $\sigma$. We must\\ have that $\ell_1 + \ldots + \ell_m = 5$ and $\lcm(\ell_1, \ldots, \ell_m) = 3$. But the only cycle type\\ satisfying those requirements is $(3 \geq 1 \geq 1)$. Hence, $\sigma$ is a $3$-cycle. That in turn\\ implies by the last lemma that $N = A_n$ (which is a contradiction).\retTwo
	\end{myIndent}

	Similarly, we can't have that $5$ divides $|N|$.\newpage
	\begin{myIndent}\exPPP
		Let $P \in \Syl_5(N)$. If $5$ divides $N$ then we also have that $P \in \Syl_5(A_5)$. And since there exists $x \in A_5$ such that $P^\prime = xPx^{-1} \subseteq N$ for all $P^\prime \in \Syl_p(A_5)$, we can in turn say that $N$ contains every element of $A_n$ with order dividing $5$.\retTwo

		But now note that the only elements of $S_5$ with order $5$ are $5$ cycles. Also, all $5$ cycles are easily seen to be even permutations. So, $\{\sigma \in S_5 : \sigma \text{ is a } 5\text{-cycle}\}$ is a subset of $N$. Also, it is an easy counting exercise to see that the number of such permutations is $\sfrac{5!}{5} = 24$. Hence, we've proven that $24 < |N|$.\retTwo

		Since $|N|$ divides $60$ and $N \lneqq A_5$, this forces $|N| = 30$. But we already showed that $3$ can't divide $|N|$. So, we have a contradiction.\retTwo
	\end{myIndent}

	This narrows us down to the case that $|N| = 2$ or $|N| = 4$. To address this case, we bring up the following lemma:\retTwo

	\ul{Lemma:} If $|G| < \infty$, $N \lhd G$, and $N$ is a $p$-group, then $N \subseteq \bigcap_{P \in \Syl_p(G)} P \eqqcolon O_p(G)$.
	\begin{myIndent}\exPPP
		Proof:\\
		By Sylow's 2nd theorem we know $N \subseteq P_0$ for some $P_0 \in \Syl_p(G)$. Then since $N$ is normal, $N = \bigcap_{x \in G} xN x^{-1} \subseteq \bigcap_{x \in G} xP_0 x^{-1} = \bigcap_{P \in \Syl_p(G)} P$. $\blacksquare$\retTwo
	\end{myIndent}

	Now pick $\sigma \in A_5$ with $o(\sigma) = 2$. The only cycle types of permutations in $S_5$ that yield permutations of order $2$ are $(2 \geq 1 \geq 1 \geq 1)$ and $(2 \geq 2 \geq 1)$. However, permutations of the former cycle type are odd. Hence, we may assume $\sigma = (a\gap b)(c \gap d)$.\retTwo

	Next, note that $P_2 \coloneqq \{\myId, (1\gap 2)(3 \gap 4), (1\gap 3)(2 \gap 4), (1\gap 4)(2 \gap 3)\}$ is a Sylow $2$-subgroup of $A_5$.
	\begin{myIndent}\exPPP
		It is clear that $P_2$ is closed under inverses. We only need to verify that it really is closed under compositions.
		\begin{itemize}
			\item \begin{tabular}{l}
				$(1\gap 2)(3 \gap 4)(1 \gap 3)(2 \gap 4) =(1 \gap 4) (2\gap 3)$ and $(1\gap 2)(3 \gap 4)(1 \gap 4)(2 \gap 3) =(1 \gap 3) (2\gap 4)$,
			\end{tabular}

			\item \begin{tabular}{l}
				$(1\gap 3)(2 \gap 4)(1 \gap 2)(3 \gap 4) =(1 \gap 4) (2\gap 3)$ and $(1\gap 3)(2 \gap 4)(1\gap 4)(2 \gap 3) = (1\gap 2)(3 \gap 4)$,
			\end{tabular}

			\item I'm bored and don't want to manually verify the last two relations.\retTwo
		\end{itemize}
	\end{myIndent}

	Thus by our lemma, we know that $N \subseteq \bigcap_{P \in \Syl_2(G)} P = \bigcap_{x \in G} xP_2x^{-1}$.\retTwo

	But now note that:
	
	{\centering$((a\gap b)(c \gap e)) P_2 ((a\gap b)(c \gap e))^{-1} = \{\myId, (b \gap a)(e \gap d), (b \gap e)(a\gap d), (b \gap d)(a \gap e)\}$\retTwo\par}

	Therefore $\bigcap_{x \in G} xP_2x^{-1}$ is trivial and we have a contradiction since $N = \{\myId\}$. This\\ finishes the proof that $A_5$ is simple. 

	\mySepThree 

	Now we proceed by induction on $n$. Suppose $1 \neq N \lhd A_n$ and for all $i \in \{1, \ldots, n\}$ let $G_i \coloneqq \{\sigma \in A_n : \sigma(i) = i\}$. Then there is an obvious group isomorphism such that $G_i \cong A_{n-1}$ for all $i$. Also, $N \cap G_i \lhd G_i$. So, by induction we know for each $i$ that either $N \cap G_i = \{1\}$ or $N \cap G_i = G_i$.\newpage

	But if $N \cap G_i = G_i$ for any $i$ then we are already done since that would imply that $N$ contains a $3$-cycle. So, without loss of generality we may now assume that $N \cap G_i = \{\myId\}$ for all $i$. As a consequence, if $\sigma, \sigma^\prime \in N$ satisfy that $\sigma(i) = \sigma^\prime(i)$ for any $i$ then we must have that $\sigma = \sigma^\prime$ since $\sigma (\sigma^\prime)^{-1} \in N \cap G_i$.\retTwo

	Suppose $\sigma \in N - \{\myId\}$. Then we have two cases:
	\begin{itemize}
		\item Suppose $\sigma$ has a cycle of size $\geq 3$. In other words, there exists distinct numbers $a, b, c \in \{1, \ldots, n\}$ such that $\sigma = (a \gap b \gap c \gap \cdots)\cdots$. But now if $d, e \in \{1, \ldots, n\}$ are any other $2$ distinct numbers, we have that $\sigma^\prime \coloneqq (c \gap d \gap e)\sigma (c \gap d \gap e)^{-1} \in N$ with $\sigma^\prime(a) = b = \sigma(a)$ and $\sigma^\prime(b) = d \neq c = \sigma(b)$. This is a contradiction.
		
		\item Meanwhile, suppose $\sigma$ has only cycles of length $2$. Since $\sigma$ is even, we thus know there are distinct numbers $a, b, c, d \in \{1, \ldots, n\}$ such that $\sigma = (a\gap b)(c \gap d)\cdots$. But now if $e, f$ are two other distinct elements of $\{1, \ldots, n\}$ (supposing $n \geq 6$), then consider $\sigma^\prime \coloneqq (c \gap e \gap f)\sigma (c \gap e \gap f)^{-1}$. Like before, $\sigma^\prime(a) = b = \sigma(a)$. But $\sigma^\prime(d) = e \neq c = \sigma(d)$. Hence we again have a contradiction. $\blacksquare$\retTwo
	\end{itemize}
\end{myIndent}

\hTwo\mySepTwo

Here is an application of the prior theorem. Suppose $G$ is a finite group with $|G| = 2m$ where $2 \not\divides m$. Then there exists a characteristic subgroup $N < G$ such that $[G : N] = 2$.

\begin{myIndent}\pracTwo
	Proof:\\
	Consider the action of $G \curvearrowright G$ by left translations and let $\phi: G \to S_G$ be the induced group homomorphism. Note that we can always identify $S_G$ with $S_{|G|}$ by just numbering the elements of $G$. Also note that for all $g \in G$ the cycle type of $\phi(g)$ is:
	
	{\centering$(o(g) \geq o(g) \geq \cdots \geq o(g))$ (where there are $|G| / o(g)$ many orbits).\retTwo\par}

	Next, by Cauchy's theorem there exists $g_0 \in G$ such that $o(g_0) = 2$. But now the cycle type of $\phi(g_0)$ is $(2 \geq 2 \geq \cdots \geq 2)$ (with $|G| / 2 = m$ many orbits.) Hence, $\phi(g_0)$ is an odd permutation and we know that $\varepsilon \circ \phi: G \to \{\pm 1\}$ is a surjective group homomorphism.\retTwo

	Let $N \coloneqq \ker(\varepsilon \circ \phi)$. Then $[G : N] = |\myIm(\varepsilon \circ \phi)| = 2$. Also, because of how we defined $N$ we know that $g \in N$ if and only if the cycle type of $(o(g) \geq o(g) \geq \cdots \geq o(g))$ gives an even permutation.\retTwo

	But now note that for all $\theta \in \Aut(G)$ we have that $o(\theta(g)) = o(g)$. Thus $g \in N$ iff $\theta(g) \in N$ for all $\theta \in \Aut(G)$. And this implies that $N$ is a characteristic subgroup of $G$. $\blacksquare$
\end{myIndent}

\Hstatement\mySepTwo

{\color{Red} (I didn't do these problems before they were due but I'm doing them now\dots)\retTwo}

\blab{Set 4 Problem 5:} In this problem we show that $\Inn(S_6) \neq \Aut(S_6)$.
\begin{enumerate}
	\item[(a)] Show that $S_5$ has $6$ Sylow $5$-subgroups and then use the action of $S_5$ on $\Syl_5(S_5)$ to show that $S_6$ has a subgroup $H$ which is isomorphic to $S_5$.
	
	\begin{myIndent}\HexOne
		Since $s_5 \coloneqq |\Syl_5(S_5)|$ equals $1 \mMod{5}$ and divides $\frac{|S_5|}{5} = \frac{5!}{5} = 24$, that already restricts $s_5$ to equaling either $1$ or $6$. That said, as I will prove later on \inLinkRap{idk reference 5}{page \_\_\_} in my lecture notes, $S_5$ does not have a normal subgroup of size $5$. Therefore, this forces $s_5 = 6$.\newpage

		Now consider the action $S_5 \curvearrowright \Syl_5(S_5)$ by conjugation and let:
		
		{\centering$\phi : S_5 \to S_{\Syl_5(S_5)} \cong S_6$ be the induced homomorphism.\retTwo\par}
		
		Note that $\sigma \in \ker(\phi)$ iff $\sigma P \sigma^{-1} = P$ for all $P \in \Syl_5(G)$. In particular, if we fix $P \in \Syl_5(G)$ then we know that $\ker(\phi) \subseteq N_G(P)$ and hence:
		
		{\centering$|\ker(\phi)| \leq \frac{S_5!}{s_5} = \frac{120}{6} = 20$.\retTwo\par}

		Thus $\ker(\phi) \lhd S_5$ and $[S_5 : \ker(\phi)] > 2$. By the aforementioned proof on \inLinkRap{idk reference 5}{page \_\_\_}, this means that $\ker(\phi) = \{\myId\}$. Hence $\phi$ is an injective homomorphism. And by letting $H \coloneqq \myIm(\phi)$ we have that $S_5 \cong H < S_6$.\retTwo
	\end{myIndent}

	\item[(b)] Show for every $\sigma \in S_6$ that $\Fix(\sigma H \sigma^{-1}) = \emptyset$ (where $S_6$ is acting on $\{1, \ldots, 6\}$ by\\ evaluation).
	
	\begin{myIndent}\HexOne
		We know that $\Fix(\sigma H \sigma^{-1}) = \bigcap_{\tau \in H}\Fix(\sigma \tau \sigma^{-1}) = \bigcap_{\tau \in H}(\sigma \cdot\Fix(\tau))$. But now recall that the action $S_5 \curvearrowright \Syl_5(S_5)$ is transitive and hence there is some $\tau^\prime \in H$ such that $\Fix(\tau^\prime) = \emptyset$. In turn, $\sigma \cdot \Fix(\tau^\prime) = \emptyset$ and so $\Fix(\sigma H \sigma^{-1}) = \emptyset$ for all $\sigma \in S_n$.

		\begin{myIndent}\exPPP
			Another way of thinking of this is that conjugation preserves cycle type. So if $\tau \in S_6$ has no fixed points (i.e. $1$-cycles), then $\sigma \tau \sigma^{-1}$ also has no $1$-cycles.\retTwo
		\end{myIndent}
	\end{myIndent}
	
	\item[(c)] Consider the action $S_6 \curvearrowright S_6 / H$ by left-translation. Show that this induces a group\\ homomorphism $\theta : S_6 \to S_6$ and that $\Fix(\theta(H)) \neq \emptyset$ (again with respect to the action of $S_6$ on $\{1, \ldots, 6\}$ by evaluation).
	
	\begin{myIndent}\HexOne
		Since $|H| = |S_5|$, we know $[S_6 : H] = \frac{|S_6|}{|S_5|} = \frac{6!}{5!} = 6$. Hence, by numbering the cosets of $H$ we can say that $S_{S_6 / H} \cong S_6$. For convenience, I'll assume the coset $H$ corresponds to $1 \in \{1, \ldots, 6\}$.\retTwo

		Now consider the induced homomorphism $\theta$ described in the problem statement. For any $\tau \in S_6$ we have that $\theta(\tau)$ describes the map $\sigma H \mapsto \tau \sigma H$. But note that if $\tau \in H$ then $\theta(\tau)$ fixes $H$. After applying our correspondance $S_{S_6 / H} \cong S_6$ this translates to saying that $1 \in \Fix(\theta(H))$.\retTwo
	\end{myIndent}

	\item[(d)] Deduce that $\Aut(S_6) \neq \Inn(S_6)$.
	
	\begin{myIndent}\HexOne
		In part (b) we proves that if $\psi \in \Inn(S_6)$ then $\Fix(\psi(H)) = \emptyset$. Thus the fact that $\theta$ from part (c) doesn't satisfy that property means $\theta$ is definitely not an inner automorphism. If we can now prove that $\theta$ is in fact an automorphism despite that, then we will be done.\retTwo

		Fortunately, note that $\tau \in \ker(\theta)$ iff $\tau \sigma H = \sigma H$ for all $\sigma \in S_6$. This is equivalent to saying that $\tau \in \bigcap_{\sigma \in S_6} \sigma H \sigma^{-1}$. And in particular, this proves that $\ker(\theta) \subseteq H$. But now since $\ker(\theta ) \lhd S_6$ and $[S_6 : \ker(\theta)] > [S_6 : H] > 2$, we know from one more application of the fact on \inLinkRap{idk reference 5}{page \_\_\_} that $\ker(\theta) = \{\myId\}$. Hence $\theta$ is injective. And by pigeonhole principle this also proves that $\theta$ is surjective. $\blacksquare$\retTwo
	\end{myIndent}
\end{enumerate}

\blab{Set 4 Problem 6:} Prove that a group $G$ of order $36$ is not simple.\newpage

\begin{myIndent}\HexOne
	Suppose for the sake of contradiction that $G$ is simple. Then since $s_3 \coloneqq |\Syl_3(G)|$ divides $12$ and equals $1 \mMod{3}$, we know $s_3$ must equal either $1$ or $4$. But we can't have that $s_3 = 1$ since that would violate the simplicity of $G$. Hence, we know that $s_3 = 4$.\retTwo

	Next, consider the action $G \curvearrowright \Syl_3(G)$ and let $\phi: G \to S_{\Syl_3(G)} \cong S_4$ be the induced homomorphism. Since that action is transitive, we know that $\ker(\phi) \neq G$. However, we also know that $\ker(\phi) \lhd G$. So by the simplicity of $G$ we must have that $\ker(\phi) = 1$. And this implies by the first isomorphism theorem that there is some subgroup $H = \myIm(\phi)$ of $S_4$ with $G \cong H$. But this is a contradiction since $|G| = 36 > 24 = |S_4|$. $\blacksquare$\retTwo
\end{myIndent}

\hTwo\mySepTwo

\dispDate{10/29/2025}

Ehh the midterm went mediocrely. I guess I'm one step closer towards flunking out of school. Anyways, right now I want to go back to studying Haar measures. So \hypertarget{idk reference 4}{I'm} going to resume what I was doing on \inLinkRap{page 353 reference}{page 353}.\retTwo

If $G$ is a group and $E, V \subseteq G$, then we can in some sense measure the size of $E$ relative to $V$ by asking what is the minimum cardinality of $A \subseteq G$ such that $E \subseteq \bigcup_{x \in A} xV$. Also, if $E$ is a compact or precompact set and $V$ is open, then we can guarentee that this minimum cardinality is finite.\retTwo

Clearly, the above construction defines a translation invariant notion of size. But it's not a measure. Can we modify this approach to actually get a measure?\retTwo

Firstly, we'll switch to working with functions in $C_c^+(G)$ since functions are easier to work with than measures and we can hopefully apply the Riesz representation theorem if we get a positive result when working with functions.\retTwo

Suppose $f, \phi \in C_c^+(G)$. Then $U = \{x : \phi(x) > \frac{1}{2}\|\phi\|_u\}$ is an open nonempty set. Also, $\supp(f)$ is compact. So there are finitely many $x_1, \ldots, x_n \in G$ with $\supp(f) \subseteq \bigcup_{j=1}^n x_jU$. And we in turn know that:

{\centering $f \leq \frac{2\|f\|_u}{\|\phi\|_u}\sum_{j=1}^n L_{x_j} \phi$. \retTwo\par}

Hence, given any $f, \phi \in C_c^+(G)$ it is well-defined to set:

{\centering$(f : \phi) \coloneqq \inf\left\{ \sum_{j=1}^n c_j : f \leq \sum_{j=1}^n c_j L_{x_j}\phi \text{ for some } n \in \mathbb{N} \text{ and } x_1, \ldots, x_n \in G\right\}$.\retTwo\par}

\begin{myTindent}\pracTwo
	Note that if $f \leq \sum_{j=1}^n c_j L_{x_j}\phi$ then we have by triangle inequality that\\ $\|f\|_u \leq \|\phi\|_u\sum_{j=1}^n c_j$. Thus, we clearly have that $0 < \frac{\|f\|_u}{\|\phi\|_u} \leq (f : \phi)$.\retTwo
\end{myTindent}

\exTwo\ul{Lemma 11.5:} Suppose that $f, g, \phi \in C_c^+$. Then:
\begin{enumerate}
	\item[(a)] $(f : \phi) = (L_x f : \phi)$ for any $x \in G$.
	
	\begin{myIndent}\exThreeP
		This is because $f \leq \sum_{j=1}^n c_j L_{x_j} \phi$ iff $L_x f \leq \sum_{j=1}^n c_j L_{xx_j}\phi$.\newpage
	\end{myIndent}

	\item[(b)] $(cf : \phi) = c(f : \phi)$ for any $c > 0$.

	\begin{myIndent}\exThreeP
		This is because $f \leq \sum_{j=1}^n c_j L_{x_j} \phi$ iff $cf \leq \sum_{j=1}^n cc_j L_{x_j}\phi$.\retTwo
	\end{myIndent}

	\item[(c)] $(f + g, \phi) \leq (f : \phi) + (g : \phi)$.
	
	\begin{myIndent}\exThreeP
		Suppose $f \leq \sum_{j=1}^m c_j L_{x_j}\phi$ and $g \leq \sum_{j={m + 1}}^{m + n}c_j L_{x_j}\phi$. Then $f + g \leq \sum_{j=1}^n c_j L_{x_j}\phi$. And by minimizing $\sum_{j=1}^m c_j$ and $\sum_{j={m + 1}}^{m + n}c_j$ this claim follows.\retTwo
	\end{myIndent}

	\item[(d)] $(f : \phi) \leq (f : g)(g : \phi)$.
	
	\begin{myIndent}\exThreeP
		If $f \leq \sum_{j=1}^n c_j L_{x_j} g$ and $g \leq \sum_{i=1}^n d_i L_{y_i}\phi$, then:
		
		{\centering$f \leq \sum_{j} c_j L_{x_j}(\sum_{i=1} d_i L_{y_i}\phi) = \sum_{j} c_j (\sum_{i} d_i L_{x_j}L_{y_i}\phi) = \sum_{j, i} c_j d_i L_{x_j y_i}\phi$\retTwo\par}

		And since $\sum_{j, i} c_j d_i = (\sum_{j} c_j)(\sum_{i}d_i)$ our claim follows. $\blacksquare$\retTwo
	\end{myIndent}
\end{enumerate}

\hTwo Now let us fix $f_0 \in C_c^+(G)$ from here on out. It doesn't matter which function $f_0$\\ specifically is. We just want to have a base-line function to compare all other functions\\ in $C_c^+(G)$ to.\retTwo

Next, if $\phi \in C_c^{+}(G)$ then let us define $ I_\phi(f) \coloneqq \frac{(f : \phi)}{(f_0 : \phi)}$ for all $f \in C_c^{+}(G)$.

\begin{myIndent}\hThree
	By the last lemma, we know that $I_\phi$ is a sublinear functional that is invariant to left\\ translations. Also, note that:

	{\centering $I_\phi(f) = \frac{(f : \phi)}{(f_0 : \phi)} \leq \frac{(f : f_0)(f_0 : \phi)}{(f_0 : \phi)} \leq (f : f_0)$ and $I_\phi(f) = \frac{(f : \phi)}{(f_0 : \phi)} \geq \frac{(f : \phi)}{(f_0 : f)(f : \phi)} \geq (f_0 : f)^{-1}$.\retTwo\par}

	(To put the last line more succinctly, $I_\phi(f) \in [(f_0 : f)^{-1}, (f : f_0)]$ for all $f, \phi \in C_c^+(G)$\dots)\retTwo
\end{myIndent}

Folland's next claim is that $I_\phi$ is "approximately" linear when $\supp(\phi)$ is small.\retTwo

\exTwo\ul{Lemma 11.7:} If $f_1, f_2 \in C_c^+(G)$ and $\varepsilon > 0$ then there is a neighborhood $V$ of $e \in G$ such that $I_\phi(f_1) + I_\phi(f_2) \leq I_\phi(f_1 + f_2) + \varepsilon$ whenever $\supp(\phi) \subseteq V$. 
\begin{myIndent}\exThreeP
	Proof:\\
	Fix $g \in C_c^+$ with $g = 1$ on $\supp(f_1 + f_2)$. Then for any $\delta > 0$ define $h_\delta = f_1 + f_2 + \delta g$\\ [-1pt] and for both $i$ define:
	
	{\centering$h^{(i)}_\delta(x) = \left\{\begin{matrix}
		\sfrac{f_i(x)}{h_\delta(x)} & \text{if } x \in \supp(f) \\
		0 & \text{otherwise.}
	\end{matrix}\right.$\retTwo\par}

	Then each $h^{(i)}_\delta \in C_c^+(G)$. So by \inLinkRap{Folland Proposition 11.2}{proposition 11.2} we know there is some neighborhood\\ $V_\delta$ of $e$ with $|h_\delta^{(i)}i(x) - h_\delta^{(i)}(y)| < \delta$ when $y^{-1}x \in V$ and $i \in \{1, 2\}$.
	\begin{myIndent}\exPPP
		Specifically, let $V_\delta$ be a nighborhood of $e$ such that $\|R_z h_\delta^{(i)} - h_\delta^{(i)}\|_u < \delta$ for all\\ $z \in V_\delta$. Then $|h_\delta^{(i)}(x) - h_\delta^{(i)}(y)| < \delta$ if $x = yz$ for some $z \in V_\delta$. Or in other\\ words, if $y^{-1}x = z \in V_\delta$ then $|h_\delta^{(i)}(x) - h_\delta^{(i)}(y)| < \delta$.\retTwo
	\end{myIndent}

	If $\phi \in C_c^+(G)$ with $\supp(\phi) \subseteq V_\delta$ and $h_\delta \leq \sum_{j=1}^n c_j L_{x_j}\phi$, then $|h_\delta^{(i)}(x) - h_\delta^{(i)}(x_j)| < \delta$ whenever $x_j^{-1} x \in \supp(\phi)$. Hence, we can say for all $x \in G$ that:

	{\centering $f_i(x) = h_\delta(x)h_\delta^{(i)} \leq \sum\limits_{j=1}^n c_j \phi(x_j^{-1} x)h_\delta^{(i)}(x) \leq \sum\limits_{j=1}^n c_j \phi(x_j^{-1} x)(h_\delta^{(i)}(x_j) + \delta)$ \newpage\par}

	But now this proves that $(f_i : \phi) \leq \sum_{j=1}^n c_j(h_\delta^{(i)}(x_j) + \delta)$ for both $i$.\retTwo
	
	Also, since $h_\delta^{(1)} + h_\delta^{(2)} \leq 1$, we can thus conclude that:
	
	{\centering$(f_1 : \phi) + (f_2 : \phi) \leq \sum_{j=1}^n c_j (1 + 2\delta)$.\retTwo\par}
	
	And by bringing $\sum_{j=1}^n c_j$ arbitrarily close to $(h : \phi)$ and dividing by $(f_0 : \phi)$, we can now say that $I_\phi(f_1) + I_\phi(f_2) \leq (1 + 2\delta)I_\phi(h_\delta) \leq (1 + 2\delta)(I_\phi(f_1 + f_2) + \delta I_\phi(g))$.\retTwo

	Now we want $(1 + 2\delta)(I_\phi(f_1 + f_2) + \delta I_\phi(g)) < I_\phi(f_1 + f_2) + \varepsilon$. Equivalently this means we want $2\delta I_\phi(f_1 + f_2) + \delta(1 + 2\delta) I_\phi(g) < \varepsilon$. And fortunately, this will be guarenteed if:

	{\centering $2\delta (f_1 + f_2 : f_0) + \delta(1 + 2\delta)(g : f_0) < \varepsilon$ \retTwo\par}

	So, by choosing $\delta$ small enough we can guarentee that $I_\phi(f_1) + I_\phi(f_2) \leq I_\phi(f_1 + f_2)$ whenever $\supp(\phi) \subseteq V_\delta$. $\blacksquare$\retTwo
\end{myIndent}

\exTwo\ul{Theorem 11.8:} Every locally compact group $G$ contains a left Haar measure.
\begin{myIndent}\exThreeP
	Proof:\\
	For each $f \in C_C^+(G)$ let $X_f \coloneqq [(f_0 : f)^{-1}, (f : f_0)]$. Then let $X = \prod_{f \in C_c^+(G)} X_f$. By\\ [-1pt] Tychonoff's theorem we know that $X$ is a compact Hausdorff space. Also, we have that\\ $I_\phi \in X$ for all $\phi \in C_c^+(G)$ since $I_\phi(f) \in [(f_0 : f)^{-1}, (f : f_0)]$ for all $f, \phi \in C_c^+(G)$.\retTwo

	Now for each  neighborhood $V$ of $e$ let $K(V)$ be the closure in $X$ of $\{I_\phi : \supp(\phi) \subseteq V\}$. Then note that $\bigcap_{j=1}^n K(V_j) \supseteq K(\bigcap_{j=1}^n V_j) \neq \emptyset$  for all finite collections $\{V_1, \ldots, V_n\}$ of neighborhoods of $e$. Hence since $X$ is compact and the collection of sets $K(V)$ has the finite intersection property, we know there exists an element $I$ in the intersection of all the $K(V)$'s.\retTwo

	Next since $I$ is either an accumulation point of or inside $\{I_\phi : \supp(\phi) \subseteq V\}$ for\\ all neighborhoods $V$ of $e$, we know that any neighborhood of $I$ in $X$ must intersect\\ $\{I_\phi : \supp(\phi) \subseteq V\}$ for all neighborhoods $V$ of $e$ in $G$. Consequently, for any\\ neighborhood $V$ of $e$ and any $f_1, \ldots, f_n \in C_c^+(G)$ and $\varepsilon > 0$ there exists $\phi \in C_c^+(G)$\\ with $\supp(\phi) \subseteq V$ such that $|I(f_j) - I_\phi(f_j)| < \varepsilon$ for each $j$.

	\begin{myIndent}\exPPP
		Why?\\
		By definition of the product topology, the following is an open neighborhood of $I$\\ in $X$:
		
		{\centering$U \coloneqq \{I^\prime \in X : |I(f_j) - I^\prime(f_j)| < \varepsilon \text{ for } j =1, \ldots, n\}$\retTwo\par}
		
		Then any $I_\phi \in U \cap \{I_\phi : \supp(\phi) \subseteq V\}$ satisfies that $|I(f_j) - I_\phi(f_j)| < \varepsilon$\\ for each $j \in \{1, \ldots, n\}$.\retTwo
	\end{myIndent}

	Consequently, we can now show using lemmas 11.5 and 11.7 that $I$ is left-invariant and satisfies that $I(af + bg) = aI(f) + bI(g)$ for all $f, g \in C_c^+$ and $a, b > 0$. 
	\begin{myIndent}\exPPP
		To show that $I(af + bg) = aI(f) + bI(g)$, consider any $\varepsilon > 0$ and then pick a neighborhood $V$ of $e$ such that whenever $\supp(\phi) \subseteq V$ we have that:

		{\centering $I_\phi(af + bg) \leq aI_\phi(f) + bI_\phi(g) \leq I_\phi(af + bg) + \varepsilon$ \retTwo\par}

		Then by our prior reasoning there exists $\phi \in C_c^+(G)$ with $\supp(\phi) \subseteq V$ such that $|I(f) - I_\phi(f)| < \varepsilon$, $|I(g) - I_\phi(g)| < \varepsilon$, and $|I(af + bg) - I_\phi(af + bg)| < \varepsilon$. And hence we can get that $|I(af + bg) - aI(f) - bI(g)| < 4\varepsilon$.\newpage

		By taking $\varepsilon \to 0$ this then proves that $I(af + bg) = aI(f) + bI(g)$.\retTwo

		Similarly, to prove that $I$ is left-invariant let $\varepsilon > 0$ and pick $\phi$ such that\\ $|I(f) - I_\phi(f)| < \varepsilon$ and $|I(L_x f) - I_\phi(L_x f)| < \varepsilon$. Then $|I(f) - I(L_x f)| < 2\varepsilon$.\retTwo
	\end{myIndent}

	We can extend $I$ to all of $C_c(G, [0, \infty))$ by defining $I(0) = 0$. This importantly preserves the linearity and left-invariance of $I$. Then similarly to the proof of lemma 7.15 on \inLinkRap{Folland Lemma 7.15 reference}{pages 57-58}, by setting $I(f) = I(f^+) - I(f^-)$ where $f^+$ and $f^-$ are the positive and negative parts of $f$ we can extend $I$ to being a positive real linear functional on $C_c(G, \mathbb{R})$. And this extension is clearly still left-invariant since $(L_x f)^+ = L_x f^+$ and $(L_x f)^- = L_x f^-$.\retTwo

	Finally, we extend $I$ to being a positive left-invariant linear functional on all of $C_c(G)$ by just setting $I(f) = I(\rea{f}) + iI(\ima{f})$. Then by applying the Riesz-representation theorem plus \inLinkRap{Folland Proposition 11.4(b)}{proposition 11.4(b)} we get a left Haar measure on $G$. $\blacksquare$\retTwo
\end{myIndent}

\hTwo I will go into more depth on Haar measures later on \inLinkRap{idk reference 6}{page \_\_\_}.

\mySepTwo

\dispDate{10/30/2025}

\blect{Math 220a Notes:}\retTwo

We'll now introduce a notion of symmetric points with respect to a given circle $\Gamma \subseteq \mathbb{C}_\infty$. Firstly, if $\Gamma = \mathbb{R}_\infty \coloneqq \mathbb{R} \cup \{\infty\}$, then clearly the intuitive definition of a symmetric point of $z \in \mathbb{C}$ would be the point $z^* \coloneqq \overline{z} \in \mathbb{C}$.\retTwo

\exTwo\ul{Proposition:} If $S$ is a Möbius transformation and $S(\mathbb{R}) = \mathbb{R}$, then $S(z^*) = (S(z))^*$ for all $z \in \mathbb{C}$ with $S(z) \neq \infty$.
\begin{myIndent}\exThreeP
	Proof:\\
	$S$ is uniquely determined by the points $z_1, z_2, z_3 \in \mathbb{C}_\infty$ such that $S(z_1) = 1$, $S(z_2) = 0$, and $S(z_3) = \infty$. But note that since $S(\mathbb{R}) = \mathbb{R}$ and $S$ maps circles to circles, we know that $z_1, z_2, z_3 \in \mathbb{R}_\infty$. Hence, by our construction in the existence proof at the top of \inLinkRap{page 337 reference}{page 337}, we know that there are real $a, b, c, d \in \mathbb{R}$ such that $S(z) = \frac{az + b}{cz + d}$. And now if $z \in \mathbb{C} - \{z_3\}$, we have that:

	{\centering$S(z)^* = \overline{S(z)} =  \overline{\left(\frac{az+ b}{cz + d}\right)} = \frac{a\overline{z}+ b}{c\overline{z} + d} = S(\overline{z}) = S(z^*)$. $\blacksquare$\retTwo\par}
\end{myIndent}

\hTwo Next we consider letting $\Gamma \subseteq \mathbb{C}_\infty$ be any circle. Then we fix $z_1, z_2, z_3 \in \Gamma$ and define the Möbius transformation $T(z) \coloneqq (z, z_1, z_2, z_3)$. Given any $z \in \mathbb{C}_\infty - \{z_3\}$ we define the \udefine{symmetric point} of $z$ with respect to $\Gamma$ to be $z^* = T^{-1}(\overline{T(z)})$.
\begin{myIndent}\pracTwo
	In other words, $z^*$ is a symmetric point of $z$ relative to $\Gamma$ iff $(z^*, z_1, z_2, z_3) = \overline{(z, z_1, z_2, z_3)}$.\retTwo
\end{myIndent}

\exTwo\ul{Claim:} Our definition of symmetric point relative to $\Gamma$ is independent of our choice\\ of $z_1, z_2, z_3 \in \Gamma$. In other words, for any $z \in \mathbb{C}_\infty$, if $\{z_1, z_2, z_3\}$ and $\{\widetilde{z_1}, \widetilde{z_2}, \widetilde{z_3}\}$ are\\ two triplets of distinct points in $\Gamma$ with $z_3 \neq z$ and $\widetilde{z_3} \neq z$, then after defining\\ $T_1(z) \coloneqq (z, z_1, z_2, z_3)$ and $T_2(z) \coloneqq (z, \widetilde{z_1}, \widetilde{z_2}, \widetilde{z_3})$ we have that $T_1^{-1}(\overline{T_1(z)}) = T_2^{-1}(\overline{T_2(z)})$.
\begin{myIndent}\exThreeP
	Proof:\newpage
	Note that $T_1 \circ T_2^{-1} \eqqcolon M$ sends $\mathbb{R}$ to $\mathbb{R}$, as does $M^{-1} = T_2 \circ T_1^{-1}$. Importantly, we can\\ thus apply our prior proposition to see that $M^{-1}(\overline{w}) = \overline{M^{-1}(w)}$ when $w \in \mathbb{C}$ and\\ [3pt] $M^{-1}(w) = T_2(T_1^{-1}(w)) \neq \infty$. In particular, if $w = M(T_2(z)) = T_1(z)$ then we have\\ [2pt] that $M^{-1}(\overline{M(T_2(z))}) = \overline{M^{-1}(M(T_2(z)))}$ if $T_1(z) \neq \infty$ and $T_2(T_1^{-1}(T_1(z))) \neq \infty$.\retTwo

	So for all $z \in \mathbb{C}_\infty - \{z_3, \widetilde{z_3}\}$ we have that:

	{\centering\begin{tabular}{l}
		$T_1^{-1}(\overline{T_1(z)}) = T_2^{-1}(T_2(T_1^{-1}(\overline{T_1(T_2^{-1}(T_2(z)))})))$\\ [6pt]
		$\phantom{T_1^{-1}(\overline{T_1(z)})} = T_2^{-1}(M^{-1}(\overline{M(T_2(z))})) = T_2^{-1}(\overline{M^{-1}(M(T_2(z)))}) = T_2^{-1}(T_2(z))$. $\blacksquare$
	\end{tabular}\retTwo\par}

	\begin{myDindent}\myComment
		As a side note, for any $z \in \mathbb{C}_\infty$ we can always choose $z_3 \in \Gamma$ such that $z \neq z_3$. Hence, we can say that every point $z$ in $\mathbb{C}_\infty$ has a well-defined symmetric point $z^*$ relative to $\Gamma$.\retTwo
	\end{myDindent}
\end{myIndent}

\hTwo Note that since $z^*$ is the symmetric point of $z$ relative to $\Gamma$ iff $(z^*, z_1, z_2, z_3) = \overline{(z, z_1, z_2, z_3)}$, we clearly have that $(z^*)^* = z$ (relative to $\Gamma$).\retTwo

\exTwo\ul{(Conway) Theorem III.3.19 (The Symmetry Principle):} If a Möbius transformation $T$ takes\\ a circle $\Gamma_1$ onto the circle $\Gamma_2$, then any pair of points symmetric with respect to $\Gamma_1$ are\\ mapped by $T$ onto a pair of points symmetric with respect to $\Gamma_2$.

\begin{myIndent}\exThreeP
	Proof:\\
	Let $z_2, z_3, z_4$ be distinct points in $\Gamma_1$ such that $z \neq z_4$. Then if $z$ and $z^{*}$ are symmetric with respect to $\Gamma_1$ we have that: 

	{\centering $(Tz^*, Tz_2, Tz_3, Tz_4) = (z^*, z_2, z_3, z_4) = \overline{(z, z_2, z_3, z_4)} = \overline{(Tz, Tz_2, Tz_3, Tz_4)}$. \retTwo\par}

	And since $Tz_2, Tz_3, Tz_4$ are three distinct points of $\Gamma_2$, we have that $Tz^*$ and $Tz$ are symmetric with respect to $\Gamma_2$. $\blacksquare$\retTwo
\end{myIndent}

Can we get an explicit formula for $z^*$?
\begin{myIndent}\hThree
	Firstly we need a quick observation. If $z_1, z_2, z_3, z_4 \in \mathbb{C}$ with $z_1 \neq z_4$ then:
	
	{\centering$\overline{(z_1, z_2, z_3, z_4)} = (\overline{z_1}, \overline{z_2}, \overline{z_3}, \overline{z_4})$.\retTwo\par}

	\begin{myIndent}\pracTwo
		Why?\\
		From the existence proof in the theorem on \inLinkRap{page 337 reference}{page 337} we know that:
		
		{\centering$(z_1, z_2, z_3, z_4) = \frac{(z_1 - z_3)(z_2 - z_4)}{(z_1 - z_4)(z_2 - z_3)}$ and $(\overline{z_1}, \overline{z_2}, \overline{z_3}, \overline{z_4}) = \frac{(\overline{z_1} - \overline{z_3})(\overline{z_2} - \overline{z_4})}{(\overline{z_1} - \overline{z_4})(\overline{z_2} - \overline{z_3})}$ .\retTwo\par}
	\end{myIndent}

	Now let $\Gamma = \{z : |z - a| = R\}$ be a circle. Then for any three distinct points $z_2, z_3, z_4$ in $\Gamma$ and $z \in \mathbb{C}$ with $z \neq z_4$ we have (since Mobius transforms preserve cross ratios) that:

	{\centering\begin{tabular}{l}
		$(z^*, z_2, z_3, z_4) = \overline{(z, z_2, z_3, z_4)}$\\ [6pt]
		$\phantom{(z^*, z_2, z_3, z_4)} = \overline{(z - a, z_2 - a, z_3 - a, z_4 - a)} = \overline{(\frac{R^2}{z - a}, \frac{R^2}{z_2 - a}, \frac{R^2}{z_3 - a}, \frac{R^2}{z_4 - a})}$\\ [6pt]
		$\phantom{(z^*, z_2, z_3, z_4) = \overline{(z - a, z_2 - a, z_3 - a, z_4 - a)}} =(\frac{R^2}{\overline{z} - \overline{a}}, \frac{R^2}{\overline{z_2} - \overline{a}}, \frac{R^2}{\overline{z_3} - \overline{a}}, \frac{R^2}{\overline{z_4} - \overline{a}})$\\ [6pt]
		$\phantom{(z^*, z_2, z_3, z_4) = \overline{(z - a, z_2 - a, z_3 - a, z_4 - a)}} =(\frac{R^2}{\overline{z} - \overline{a}}, \frac{(z_2 - a)(\overline{z_2} - \overline{a})}{\overline{z_2} - \overline{a}}, \frac{(z_3 - a)(\overline{z_3} - \overline{a})}{\overline{z_3} - \overline{a}}, \frac{(z_4 - a)(\overline{z_4} - \overline{a})}{\overline{z_4} - \overline{a}})$\\ [6pt]
		$\phantom{(z^*, z_2, z_3, z_4) = \overline{(z - a, z_2 - a, z_3 - a, z_4 - a)}} = (\frac{R^2}{\overline{z} - \overline{a}}, z_2 - a, z_3 - a, z_4 - a)$\\ [6pt]
		$\phantom{(z^*, z_2, z_3, z_4) = \overline{(z - a, z_2 - a, z_3 - a, z_4 - a)}} = (\frac{R^2}{\overline{z} - \overline{a}} + a, z_2, z_3, z_4)$
	\end{tabular}\newpage\par}

	Hence $z^* = a + \frac{R^2}{\overline{z} - \overline{a}}$ (when $\Gamma = \{z : |z - a| = R\}$). And in particular we have that $(z^* - a)(\overline{z} - \overline{a}) = R^2$.

	\begin{myIndent}\myComment
		Perhaps unsurprisingly this indicates that the symmetric point of the center of the circle is $\infty$.\retTwo
	\end{myIndent}
\end{myIndent}

\hTwo\mySepTwo

Suppose $\Gamma = \mathbb{R}$ and let $z_1, z_2, z_3 \in \mathbb{R}$. Then put $T(z) = (z, z_1, z_2, z_3) = \frac{az + b}{cz + d}$. As\\ mentioned before we can choose $a, b, c, d$ to be real-valued. Thus:

{\centering $Tz = \frac{az + b}{cz + d} = \frac{az + b}{|cz + d|^2}(c\overline{z} + d) = \frac{1}{|cz + d|^2} (ac|z|^2 + bd + bc\overline{z} + adz)$ \retTwo\par}

And specifically focusing on the imaginary component, we have that:

{\centering$\ima{(z, z_1, z_2, z_3)} = \frac{(ad - bc)}{|cz + d|^2}\ima{z}$.\retTwo\par}

This shows that $\{z : \ima{(z, z_1, z_2, z_3)} < 0\}$ is equal to either $\{z : \ima{z} < 0\}$ or\\ $\{z : \ima{z} > 0\}$ depending on whether $ad - bc > 0$ or $ad - bc < 0$ respectively.\retTwo

Next suppose $\Gamma$ is an arbitrary circle and that $z_1, z_2, z_3 \in \Gamma$. Then if $S$ is any Möbius transformation we have that:

{\centering\begin{tabular}{l}
	$\{z : \ima{(z, z_1, z_2, z_3)} > 0\} = \{z : \ima{(S(z), S(z_1), S(z_2), S(z_3))} > 0\}$\\ [4pt]
	$\phantom{\{z : \ima{(z, z_1, z_2, z_3)} > 0\}} = S^{-1}(\{z : \ima{(z, S(z_1), S(z_2), S(z_3))} > 0\})$ 
\end{tabular}\retTwo\par}

And in particular, if $S$ maps $\Gamma$ onto $\mathbb{R}_\infty$ then $\{z : \ima{(z, z_1, z_2, z_3)} > 0\} = S^{-1}(H)$ where $H \subseteq \mathbb{C}$ is either the upper half plane or lower half plane.\retTwo

For a circle, we can indicate an \udefine{orientation} (i.e. a direction going around the circle) of\\ the circle by picking an ordered triple $(z_1, z_2, z_3)$ on $\Gamma$. Intuitively, you can think of an orientation as saying that you travel in $\Gamma$ in the direction going from $z_1$ to $z_2$ without passing through $z_3$ in the middle.\retTwo

If $(z_1, z_2, z_3)$ is an orientation of $\Gamma$ then we define the \udefine{right side of $\Gamma$} (with respect to our orientation) to be $\{z : \ima{(z, z_1, z_2, z_3)} > 0\}$. Similarly, we define the \udefine{left side of $\Gamma$} to be $\{z : \ima{(z, z_1, z_2, z_3)} < 0\}$.\retTwo

\exTwo\ul{(Conway) Theorem III.3.21 (The Orientation Principle):} Let $\Gamma_1$ and $\Gamma_2$ be two circles in\\ $\mathbb{C}_\infty$ and let $T$ be a Möbius transformation such that $T(\Gamma_1) = \Gamma_2$. Let $(z_1, z_2, z_3)$ be an\\ orientation for $\Gamma_1$. Then $T$ takes the right side and the left side of $\Gamma_1$ onto the right side\\ and left side of $\Gamma_2$ with respect to the orientation $(Tz_1, Tz_2, Tz_3)$.
\begin{myIndent}\exThreeP
	Why? Just note that $(z, z_1, z_2, z_3) = (Tz, Tz_1, Tz_2, Tz_3)$.\retTwo
\end{myIndent}

\pracOne\mySepTwo

One more comment I'll make before moving on:
\begin{myIndent}
	Recall that if $\Gamma \subseteq \mathbb{C}_\infty$ is a circle and $z_1, z_2, z_3 \in \Gamma$, then $f(z) \coloneqq \ima{(z, z_1, z_2, z_3)}$ is equal to zero iff $z \in \Gamma$. It follows if $G_+$ denotes the left side of $\Gamma$ relative to $(z_1, z_2, z_3)$ and $G_-$ denotes the right side, $G_+$ and $G_-$ partition $\mathbb{C}_\infty - \Gamma$.\retTwo

	Also note that since $f$ is continuous, we can show that both $G_+$ and $G_-$ are clopen in $\mathbb{C}_\infty - \Gamma$. Since $\mathbb{C}_\infty - \Gamma$ is easily shown to have at most two path components, it follows that both $G_+$ and $G_-$ must be the connected components of $\mathbb{C}_\infty - \Gamma$.\retTwo

	A consequence of this as well as the orientation principle is that if if $T$ maps a circle $\Gamma_1 = \{z : |z - a| = R\}$ to another circle $\Gamma_2 = \{z : |z - a^\prime| = R^\prime\}$, then either $T$ maps the interior $\Gamma_1$ to the whole interior of $\Gamma_2$ or $T$ maps the interior of $\Gamma_1$ to the whole exterior of $\Gamma_2$.\retTwo
\end{myIndent}

Actually I have another comment to make as well.
\begin{myIndent}
	By our construction on \inLinkRap{page 337 reference}{page 337}, we know that $(z, z_1, z_2, z_3) = (z, z_1, z_3, z_2)^{-1}$. In particular, since $\frac{1}{w} = \overline{w}{|w|^2}$, this means that the orientation $(z_1, z_3, z_2)$ has the oppositive left vs right sides as does the orientation $(z_1, z_2, z_3)$.\retTwo
\end{myIndent}

\hTwo\mySepTwo

The next topic covered by math 220a is proving Cauchy's formula and then doing a bunch of stuff with that such as proving that holomorphic functions are analytic.  Since I already took notes on a bunch of this last Spring, I'm going to intentionally skip over a lot of this.\retTwo

\exTwo\ul{(Conway) Proposition IV.2.1:} Let $\varphi : [a, b] \times [c, d] \to \mathbb{C}$ be a continuous function and define $g : [c, d] \to \mathbb{C}$ by $g(t) = \int_a^b \varphi(s, t)\df s$. Then $g$ is continuous. Moreover, if $\frac{\partial \varphi}{\partial t}(s, t)\df s$ and is continuous on $[a, b] \times [c, d]$ then $g$ is continuously differentiable with $g^\prime(t) = \int_a^b \frac{\partial \varphi}{\partial t}(s, t)\df s$.
\begin{myIndent}\exThreeP
	Why? Just use the fact that $\varphi$ or $\frac{\partial \varphi}{\partial t}$ is continuous on a compact domain in order to get an upper bound and then apply the theorem from math 240a (see \inLinkRap{Generalization page 189}{page 189}).\retTwo
\end{myIndent}



% ~~~~~~~~~~~~~~~~~~~~~~~~~~~~~~~~~~~~~~~~~~~~~~

\hypertarget{Generalization page 189}{} 


\hypertarget{Folland Proposition 11.2}{}
\hypertarget{Folland Lemma 7.15 reference}{}
\hypertarget{Folland Proposition 11.4(b)}{}

\hypertarget{Alireza lemma page 257}{}

\hypertarget{page 337 reference}{}

\hypertarget{Folland Proposition 10.1}{}


\hypertarget{Folland proposition 11.1}{}

\hypertarget{Ergodic reading group notes 3}{}

\hypertarget{existence and uniqueness diff eq notes}{}
\hypertarget{math 241a lecture 5}{}
\hypertarget{idk reference 2}{}

\hypertarget{idk reference 5}{}
\hypertarget{idk reference 6}{}

\end{document}


% \blect{Math 220 Homework:}\\

% \blab{Exercise III.2.2:} Prove that if $b_n, a_n$ are real and positive, $0 < b = \lim_{n \to \infty} b_n$, and $a = \limsup_{n \to \infty} a_n$, then $ab = \limsup_{n \to \infty} (a_nb_n)$.

% \begin{myIndent}\HexOne

% \end{myIndent}



% \hTwo Suppose $|G| = pq$ where $p < q$ are prime numbers. Then $s_q = 1$. Hence there exists a unique Sylow $q$-subgroup $Q$. Furthermore, $Q \lhd G$ and $Q$ is cylic with order $q$.\retTwo

% Next, let $P$ by a Sylow $p$-subgroup. Then because $Q \lhd G$, we have that $PQ < G$. Also, $|P \cap Q| \divides \gcd(p, q) = 1$. So, $P \cap Q = \{1\}$ and from there it follows that $|PQ| = pq = |G|$. So $G / Q = PQ / Q \cong P / (P \cap Q) \cong P$.

