\documentclass{book}

\usepackage{fontspec} % used to import Calibri
\usepackage{anyfontsize} % used to adjust font size

% needed for inch and other length measurements
% to be recognized
\usepackage{calc}

% for colors and text effects as is hopefully obvious
\usepackage[dvipsnames]{xcolor}
\usepackage{soul}

% control over margins
\usepackage[margin=1in]{geometry}
\usepackage[strict]{changepage}

\usepackage{mathtools}
\usepackage{amsfonts}
\usepackage{bm}

\usepackage[scr=rsfso, scrscaled=.96]{mathalpha}

% This is how I'm getting the nice caligraphy font :(
\DeclareMathAlphabet{\eulerscr}{U}{eus}{m}{n}
\newcommand{\mathcalli}[1]{\text{\scalebox{1.11}{$\eulerscr{#1}$}}}


\usepackage{amssymb} % originally imported to get the proof square
\usepackage{xfrac}
\usepackage[overcommands]{overarrows} % Get my preferred vector arrows...
\usepackage{relsize}

% Just am using this to get a dashed line in a table...
% Also you apparently want this to be inactive if you aren't
% using it because it slows compilation.
\usepackage{arydshln} \ADLinactivate 
\newenvironment{allowTableDashes}{\ADLactivate}{\ADLinactivate}

\usepackage{graphicx}
\graphicspath{{./158_Images/}}

\usepackage{tikz}
   \usetikzlibrary{arrows.meta}
   \usetikzlibrary{graphs, graphs.standard}

\usepackage{quiver} %commutative diagrams






\usepackage[hidelinks]{hyperref}
\newcommand{\inLinkRap}[2]{{\color{blue}\hyperlink{#1}{\textit{#2}}}}







\newfontfamily{\calibri}{Calibri}
\setlength{\parindent}{0pt}
\definecolor{RawerSienna}{HTML}{945D27}

% ~~~~~~~~~~~~~~~~~~~~~~~~~~~~~~~~~~~~~~~~~~~~~~~~~~
%Arrow Commands:

% Thank you Bernard, gernot, and Sigur who I copied this from:
% https://tex.stackexchange.com/questions/364096/command-for-longhookrightarrow
\renewcommand{\hookrightarrow}{\lhook\joinrel\rightarrow}
\renewcommand{\hookleftarrow}{\leftarrow\joinrel\rhook}
\newcommand{\hooklongrightarrow}{\lhook\joinrel\longrightarrow}
\newcommand{\hooklongleftarrow}{\longleftarrow\joinrel\rhook}
\newcommand{\hookxlongrightarrow}[2][]{\lhook\joinrel\xrightarrow[#1]{#2}}
\newcommand{\hookxlongleftarrow}[2][]{\xleftarrow[#1]{#2}\joinrel\rhook}

% Thank you egreg who I copied from:
% https://tex.stackexchange.com/questions/260554/two-headed-version-of-xrightarrow
\newcommand{\longrightarrowdbl}{\longrightarrow\mathrel{\mkern-14mu}\rightarrow}
\newcommand{\rightarrowdbl}{\rightarrow\mathrel{\mkern-14mu}\rightarrow}
\newcommand{\longleftarrowdbl}{\leftarrow\mathrel{\mkern-14mu}\longleftarrow}

\newcommand{\xrightarrowdbl}[2][]{%
  \xrightarrow[#1]{#2}\mathrel{\mkern-14mu}\rightarrow
}
\newcommand{\xleftarrowdbl}[2][]{%
  \leftarrow\mathrel{\mkern-14mu}\xleftarrow[#1]{#2}
}

\newcommand{\mRoman}[1]{%
   \textrm{\MakeUppercase{\romannumeral #1}}%
}



% ~~~~~~~~~~~~~~~~~~~~~~~~~~~~~~~~~~~~~~~~~~~~~~~~~~

\newcommand{\hOne}{%
   \color{Black}%
   \fontsize{14}{16}\selectfont%
}
\newcommand{\hTwo}{%
\color{Black}%
   \fontsize{13}{15}\selectfont%
}
% \newcommand{\scratchWork}{%
%    \color{PineGreen!85!Orange}
%    \fontsize{12}{14}\selectfont%
% }
\newcommand{\hThree}{%
   \color{Black}%
   \fontsize{12}{14}\selectfont%
}
\newcommand{\myComment}{%
   \color{RawerSienna}%
   \fontsize{12}{14}\selectfont%
}
\newcommand{\pracOne}{
   \color{BrickRed}%
   \fontsize{13}{15}\selectfont%
}
\newcommand{\pracTwo}{
   \color{Orange}%
   \fontsize{12}{14}\selectfont%
}
\newcommand{\why}{%
   \color{Orange}%
   \fontsize{12}{14}\selectfont%
	Why:
}
\newcommand{\exOne}{%
   \color{Purple}%
   \fontsize{14}{16}\selectfont%
}
\newcommand{\exTwo}{%
   \color{Purple}%
   \fontsize{13}{15}\selectfont%
}
\newcommand{\exThree}{%
   \color{Purple}%
   \fontsize{12}{14}\selectfont%
}
\newcommand{\exP}{%
   \color{Purple}%
   \fontsize{12}{14}\selectfont%
}
\newcommand{\exTwoP}{%
   \color{RedViolet}%
   \fontsize{13}{15}\selectfont%
}
\newcommand{\exThreeP}{%
   \color{RedViolet}%
   \fontsize{12}{14}\selectfont%
}
\newcommand{\exFourP}{%
   \color{RedViolet}%
   \fontsize{11}{13}\selectfont%
}
\newcommand{\exPP}{%
   \color{RedViolet}%
   \fontsize{12}{14}\selectfont%
}
\newcommand{\exPPP}{%
   \color{VioletRed}%
   \fontsize{12}{14}\selectfont%
}

% Homework standard below (God the bloat in the header is absurd...)
% ~~~~~~~~~~~~~~~~~~~~~~~~~~~~~~~~~~~~~~~~~~~~~~~~
\newcommand{\Hstatement}{%
   \color{MidnightBlue!90!Black}%
   \fontsize{12}{13}\selectfont%
}
\newcommand{\HexOne}{%
   \color{Purple}%
   \fontsize{12}{13}\selectfont%
}
\newcommand{\HexTwoP}{%
   \color{RedViolet}%
   \fontsize{12}{13}\selectfont%
}
\newcommand{\HexPPP}{%
   \color{VioletRed}%
   \fontsize{11}{12}\selectfont%
}

% ~~~~~~~~~~~~~~~~~~~~~~~~~~~~~~~~~~~~~~~~~~~~~~~~

\newcommand{\cyPen}[1]{{\vphantom{.}\color{Cerulean}#1}}
\newcommand{\redPen}[1]{{\vphantom{.}\color{Red}#1}}

\newenvironment{myIndent}{%
   \begin{adjustwidth}{2.5em}{0em}%
}{%
   \end{adjustwidth}%
}

\newenvironment{myDindent}{%
   \begin{adjustwidth}{5em}{0em}%
}{%
   \end{adjustwidth}%
}

\newenvironment{myTindent}{%
   \begin{adjustwidth}{7.5em}{0em}%
}{%
   \end{adjustwidth}%
}

\newenvironment{myConstrict}{%
   \begin{adjustwidth}{2.5em}{2.5em}%
}{%
   \end{adjustwidth}%
}

\newcommand{\udefine}[1]{{%
   \setulcolor{Red}%
   \setul{0.14em}{0.07em}%
   \ul{#1}%
}}

\newcommand{\uprop}[1]{{%
   \setulcolor{Purple}%
   \setul{0.14em}{0.07em}%
   \ul{#1} 
}}

\newcommand{\blab}[1]{\textbf{#1}}
\newcommand{\blect}[1]{{\color{MidnightBlue}\textbf{#1}}}

\newcommand{\uuline}[2][.]{%
{\vphantom{a}\color{#1}%
\rlap{\rule[-0.18em]{\widthof{#2}}{0.06em}}%
\rlap{\rule[-0.32em]{\widthof{#2}}{0.06em}}}%
#2}

\newcommand{\pprime}{{\prime\prime}}
\newcommand{\suchthat}{ \hspace{0.3em}s.t.\hspace{0.3em}}
\newcommand{\rea}[1]{\mathrm{Re}(#1)}
\newcommand{\ima}[1]{\mathrm{Im}(#1)}
\newcommand{\comp}{\mathsf{C}}
\newcommand{\trans}{\mathsf{T}}
\newcommand{\myHS}{ \hspace{0.5em}}
\newcommand{\gap}{\phantom{2}}

\newcommand{\GenLin}{\ensuremath{\mathrm{GL}}}
\newcommand{\Cay}{\ensuremath{\mathrm{Cay}}}

\newcommand{\myId}{\mathrm{Id}}
\newcommand{\myIm}{\mathrm{im}}
\newcommand{\Obj}{\mathrm{Obj}}
\newcommand{\Hom}{\mathrm{Hom}}
\newcommand{\End}{\mathrm{End}}
\newcommand{\Aut}{\mathrm{Aut}}

\newcommand{\df}{\mathrm{d}}
\newcommand{\Df}{\mathrm{D}}

\newcommand{\mcateg}[1]{{\bm{\mathsf{#1}}}}

\newcommand{\mdeg}{\mathrm{mdeg}\phantom{.}}

\newcommand{\dividesDeprecated}{\mathop{\mid}}
\newcommand{\divides}{\mathrel{\mid}}

\newcommand{\card}{\mathrm{card}}
\newcommand{\supp}{\mathrm{supp}}
\newcommand{\diam}{\mathrm{diam}}
\newcommand{\conv}{\mathrm{conv}}
\newcommand{\opnorm}{\mathrm{op}}
\newcommand{\loc}{\mathrm{loc}}
\newcommand{\sgn}{\mathrm{sgn}}
\newcommand{\acc}{\mathrm{acc}}

\newcommand{\mSpan}{\mathrm{span}}
\newcommand{\Interior}{\mathop{\mathrm{Int}}}

\newcommand{\mMat}[1]{\mathbf{#1}}

\newcommand{\NBV}{\ensuremath{\mathrm{NBV}}}
\newcommand{\Acc}{\mathrm{Acc}}
\newcommand{\BV}{\ensuremath{\mathrm{BV}}}
\newcommand{\Var}{\ensuremath{\mathrm{Var}}}

\newcommand{\Alt}{\mathrm{Alt}}
\newcommand{\Sym}{\mathrm{Sym}}

\newcommand{\weakst}{weak$^*$ }

\newcommand{\radtimes}{\mathop{\widehat{\times}}}

\newcommand{\mMod}[1]{\phantom{a}(\mathrel{\mathrm{mod}} #1)}
\newcommand{\Fun}{\mathrm{Fun}}
\newcommand{\act}{\mathrm{act}}
\newcommand{\Fix}{\mathrm{Fix}}
\newcommand{\Sub}{\mathrm{Sub}}
\newcommand{\Cl}{\mathrm{Cl}}
\newcommand{\GL}{\mathrm{GL}}
\newcommand{\SL}{\mathrm{SL}}
\newcommand{\PSL}{\mathrm{PSL}}
\newcommand{\core}{\mathrm{core}}
\newcommand{\Syl}{\mathrm{Syl}}
\newcommand{\Iso}{\mathrm{Iso}}
\newcommand{\Homeo}{\mathrm{Homeo}}
\newcommand{\Inn}{\mathrm{Inn}}
\newcommand{\Out}{\mathrm{Out}}
\newcommand{\ab}{\mathrm{ab}}
\newcommand{\Max}{\mathrm{Max}}
\newcommand{\lt}{\mathrm{lt}}
\newcommand{\Nil}{\mathrm{Nil}}
\newcommand{\Ideal}{\mathrm{Ideal}}
\newcommand{\Spec}{\mathrm{Spec}}
\newcommand{\Res}{\mathrm{Res}}
\newcommand{\exConv}{\mathrm{ex}}


\DeclareMathOperator{\lcm}{lcm}
\DeclareMathOperator{\Log}{Log}
\DeclareMathOperator{\symdif}{\triangle}
\DeclareMathOperator{\Average}{Average}
\DeclareMathOperator*{\AverageAst}{Average}

% Thank you Gonzalo Medina and Moriambar who wrote this on stack exchange:
%https://tex.stackexchange.com/questions/74125/how-do-i-put-text-over-symbols%
\newcommand{\myequiv}[1]{\stackrel{\mathclap{\mbox{\footnotesize{$#1$}}}}{\equiv}}

% Thank you chs who wrote this on stack exchange:
%https://tex.stackexchange.com/questions/89821/how-to-draw-a-solid-colored-circle%
\newcommand{\filledcirc}[1][.]{\ensuremath{\hspace{0.05em}{\color{#1}\bullet}\mathllap{\circ}\hspace{0.05em}}}

%Thank you blerbl who wrote this on stack exchange:
%https://tex.stackexchange.com/questions/25348/latex-symbol-for-does-not-divide
\newcommand{\ndiv}{\hspace{-0.3em}\not|\hspace{0.35em}}

\newcommand{\mySepOne}[1][.]{%
   {\noindent\color{#1}{\rule{6.5in}{1mm}}}\\%
}
\newcommand{\mySepTwo}[1][.]{%
   {\noindent\color{#1}{\rule{6.5in}{0.5mm}}}\\%
}
\newcommand{\mySepThree}[1][.]{%
   {\noindent\color{#1}{\rule{6in}{0.25mm}}}\\%
}

\newenvironment{myClosureOne}[2][.]{%
   \color{#1}%
   \begin{tabular}{|p{#2in}|} \hline \\%
}{%
   \\ \hline \end{tabular}%
}

\newcommand{\retTwo}{\hfill\bigbreak}

\newcommand{\dispDate}[1]{{
   \color{Black}%
   \fontsize{20}{18}\selectfont%
   #1\retTwo
}}


\begin{document}
\setul{0.14em}{0.07em}
\calibri

\hTwo\dispDate{12/14/2025}

\blect{Math 241a Notes:}\retTwo

Suppose $V$ is a finite dimensional vector space and $\pi: G \to \GL(V)$ is a representation. Then $\pi$ is called \udefine{irreducible} if the only $\pi(G)$-invariant subspaces are $\{0\}$ and $V$. $\pi$ is called \udefine{completely reducible} if $V = \bigoplus V_i$ where $V_i$ is a $\pi(G)$-invariant irreducible subspace.\retTwo

Also if $V = \mathbb{C}^n$ or $\mathbb{R}^n$, then I shall denote $\GL(V)$ as $\GL_n(\mathbb{C})$ or $\GL_n(\mathbb{R})$ respectively. Similarly, I shall denote $U(V)$ as $U(n)$.\retTwo

\exTwo\ul{Proposition 2.2.11:} If $G$ is a group and $\pi: G \to U(n)$ is a unitary representation, then:
\begin{itemize}
	\item[(i)] every $\pi(G)$-invariant subspace has a $\pi(G)$-invariant orthogonal complement.
	\begin{myIndent}\exThreeP
		Proof:\\
		Suppose $V$ is invariant and $w \in V^\perp$. Then as $\pi(g)$ is unitary (which means\\ $\pi(g)^* = \pi(g)^{-1}$) for each $g \in G$, we know:
		
		{\centering$\langle \pi(g)w, v\rangle = \langle w, \pi(g)^* v\rangle = \langle w, \pi(g^{-1})v \rangle = 0$.\retTwo\par}
		
		It follows that $V^\perp$ is $G$-invariant.\retTwo
	\end{myIndent}

	\item[(ii)] $\pi$ is completely reducible.
	\begin{myIndent}\exThreeP
		Proof:\\
		We can prove this by induction. If $\mathbb{C}^n$ isn't irreducible then we can write\\ $\mathbb{C}^n = V \oplus V^\perp \cong \mathbb{C}^{k} \oplus \mathbb{C}^{n-k}$ where both $V$ and $V^\perp$ are $G$-invariant. Then\\ we just repeat this reasoning on the smaller subspaces. $\blacksquare$\retTwo
	\end{myIndent}
\end{itemize}

\ul{Proposition 2.2.12:} If $G$ is a compact group, $V$ is a finite dimensional real or complex\\ Hausdorff topological vector space, and $\pi: G \to \GL(V)$ is a (strong operator) continuous representation, then $\pi$ is completely reducible.
\begin{myIndent}\exThreeP
	Proof:\\
	Using \inLinkRap{G-invariant inner product page 485}{corollary 2.2.8 on page 485}, let $\langle\phantom{.},\phantom{.}\rangle$ be a $G$-invariant inner product on $V$. Then $\pi$ is a unitary representation with respect to this inner product. So, we can apply the prior proposition. $\blacksquare$\retTwo
\end{myIndent}

\hTwo\mySepTwo

Let $\mathcalli{X}$ be a real or complex vector space and let $A \subseteq \mathcalli{X}$ be convex.
\begin{itemize}
	\item Given any $x, y \in \mathcalli{X}$ we let $[x, y] \coloneqq \{ty + (1-t)x : 0 \leq t \leq 1\}$. Also, we let $(x, y) \coloneqq \{ty + (1-t)x : 0 < t < 1\}$. 
	\item We say $x \in A$ is an \udefine{extreme point} if for any $y, z \in A$ we have that $x \in [y, z]$ iff\\ $x = y$ or $x = z$. We denote the set of such points as $\exConv(A)$.
	\item We say $\emptyset \neq B \subseteq A$ is an \udefine{extreme set} if for any $y, z \in A$ we have that:
	
	{\centering$(y, z) \cap B \neq \emptyset \Longrightarrow [y, z] \cap B \neq \emptyset$.\newpage\par}
\end{itemize}

Given a set $E \subseteq \mathcalli{X}$ where $\mathcalli{X}$ is a topological real or complex vector space, we define $\overline{\conv}(A)$ to be the smallest closed convex set containing $A$. This is well-defined because arbitrary intersections of closed convex sets are closed and convex.







% ~~~~~~~~~~~~~~~~~~~~~~~~~~~~~~~~~~~~~~~~~~~~~~

\hypertarget{Page 378 Reference}{}
\hypertarget{Math 200a Set 4 Problem 3}{}


\hypertarget{Generalization page 189}{} 


\hypertarget{Math 200a problem set 2 what is a commutator and derived subgroup}{}



\hypertarget{Folland Proposition 11.2}{}
\hypertarget{Folland Lemma 7.15 reference}{}
\hypertarget{Folland Proposition 11.4(b)}{}

\hypertarget{Alireza lemma page 257}{}

\hypertarget{page 337 reference}{}

\hypertarget{Folland Proposition 10.1}{}


\hypertarget{Folland proposition 11.1}{}

\hypertarget{Ergodic reading group notes 3}{}

\hypertarget{existence and uniqueness diff eq notes}{}
\hypertarget{math 241a lecture 5}{}
\hypertarget{idk reference 2}{}

\hypertarget{idk reference 5}{}
\hypertarget{idk reference 6}{}

\end{document}


% \blect{Math 220 Homework:}\\

% \blab{Exercise III.2.2:} Prove that if $b_n, a_n$ are real and positive, $0 < b = \lim_{n \to \infty} b_n$, and $a = \limsup_{n \to \infty} a_n$, then $ab = \limsup_{n \to \infty} (a_nb_n)$.

% \begin{myIndent}\HexOne

% \end{myIndent}



% \hTwo Suppose $|G| = pq$ where $p < q$ are prime numbers. Then $s_q = 1$. Hence there exists a unique Sylow $q$-subgroup $Q$. Furthermore, $Q \lhd G$ and $Q$ is cylic with order $q$.\retTwo

% Next, let $P$ by a Sylow $p$-subgroup. Then because $Q \lhd G$, we have that $PQ < G$. Also, $|P \cap Q| \dividesDeprecated \gcd(p, q) = 1$. So, $P \cap Q = \{1\}$ and from there it follows that $|PQ| = pq = |G|$. So $G / Q = PQ / Q \cong P / (P \cap Q) \cong P$.

