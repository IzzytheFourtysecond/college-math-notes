\documentclass{book}

\usepackage{fontspec} % used to import Calibri
\usepackage{anyfontsize} % used to adjust font size

% needed for inch and other length measurements
% to be recognized
\usepackage{calc}

% for colors and text effects as is hopefully obvious
\usepackage[dvipsnames]{xcolor}
\usepackage{soul}

% control over margins
\usepackage[margin=1in]{geometry}
\usepackage[strict]{changepage}

\usepackage{mathtools}
\usepackage{amsfonts}
\usepackage{bm}

\usepackage[scr=rsfso, scrscaled=.96]{mathalpha}

% This is how I'm getting the nice caligraphy font :(
\DeclareMathAlphabet{\eulerscr}{U}{eus}{m}{n}
\newcommand{\mathcalli}[1]{\text{\scalebox{1.11}{$\eulerscr{#1}$}}}


\usepackage{amssymb} % originally imported to get the proof square
\usepackage{xfrac}
\usepackage[overcommands]{overarrows} % Get my preferred vector arrows...
\usepackage{relsize}

% Just am using this to get a dashed line in a table...
% Also you apparently want this to be inactive if you aren't
% using it because it slows compilation.
\usepackage{arydshln} \ADLinactivate 
\newenvironment{allowTableDashes}{\ADLactivate}{\ADLinactivate}

\usepackage{graphicx}
\graphicspath{{./158_Images/}}

\usepackage{tikz}
   \usetikzlibrary{arrows.meta}
   \usetikzlibrary{graphs, graphs.standard}

\usepackage{quiver} %commutative diagrams






\usepackage[hidelinks]{hyperref}
\newcommand{\inLinkRap}[2]{{\color{blue}\hyperlink{#1}{\textit{#2}}}}







\newfontfamily{\calibri}{Calibri}
\setlength{\parindent}{0pt}
\definecolor{RawerSienna}{HTML}{945D27}

% ~~~~~~~~~~~~~~~~~~~~~~~~~~~~~~~~~~~~~~~~~~~~~~~~~~
%Arrow Commands:

% Thank you Bernard, gernot, and Sigur who I copied this from:
% https://tex.stackexchange.com/questions/364096/command-for-longhookrightarrow
\renewcommand{\hookrightarrow}{\lhook\joinrel\rightarrow}
\renewcommand{\hookleftarrow}{\leftarrow\joinrel\rhook}
\newcommand{\hooklongrightarrow}{\lhook\joinrel\longrightarrow}
\newcommand{\hooklongleftarrow}{\longleftarrow\joinrel\rhook}
\newcommand{\hookxlongrightarrow}[2][]{\lhook\joinrel\xrightarrow[#1]{#2}}
\newcommand{\hookxlongleftarrow}[2][]{\xleftarrow[#1]{#2}\joinrel\rhook}

% Thank you egreg who I copied from:
% https://tex.stackexchange.com/questions/260554/two-headed-version-of-xrightarrow
\newcommand{\longrightarrowdbl}{\longrightarrow\mathrel{\mkern-14mu}\rightarrow}
\newcommand{\longleftarrowdbl}{\leftarrow\mathrel{\mkern-14mu}\longleftarrow}

\newcommand{\xrightarrowdbl}[2][]{%
  \xrightarrow[#1]{#2}\mathrel{\mkern-14mu}\rightarrow
}
\newcommand{\xleftarrowdbl}[2][]{%
  \leftarrow\mathrel{\mkern-14mu}\xleftarrow[#1]{#2}
}

\newcommand{\mRoman}[1]{%
   \textrm{\MakeUppercase{\romannumeral #1}}%
}



% ~~~~~~~~~~~~~~~~~~~~~~~~~~~~~~~~~~~~~~~~~~~~~~~~~~

\newcommand{\hOne}{%
   \color{Black}%
   \fontsize{14}{16}\selectfont%
}
\newcommand{\hTwo}{%
\color{Black}%
   \fontsize{13}{15}\selectfont%
}
% \newcommand{\scratchWork}{%
%    \color{PineGreen!85!Orange}
%    \fontsize{12}{14}\selectfont%
% }
\newcommand{\hThree}{%
   \color{Black}%
   \fontsize{12}{14}\selectfont%
}
\newcommand{\myComment}{%
   \color{RawerSienna}%
   \fontsize{12}{14}\selectfont%
}
\newcommand{\pracOne}{
   \color{BrickRed}%
   \fontsize{13}{15}\selectfont%
}
\newcommand{\pracTwo}{
   \color{Orange}%
   \fontsize{12}{14}\selectfont%
}
\newcommand{\why}{%
   \color{Orange}%
   \fontsize{12}{14}\selectfont%
	Why:
}
\newcommand{\exOne}{%
   \color{Purple}%
   \fontsize{14}{16}\selectfont%
}
\newcommand{\exTwo}{%
   \color{Purple}%
   \fontsize{13}{15}\selectfont%
}
\newcommand{\exThree}{%
   \color{Purple}%
   \fontsize{12}{14}\selectfont%
}
\newcommand{\exP}{%
   \color{Purple}%
   \fontsize{12}{14}\selectfont%
}
\newcommand{\exTwoP}{%
   \color{RedViolet}%
   \fontsize{13}{15}\selectfont%
}
\newcommand{\exThreeP}{%
   \color{RedViolet}%
   \fontsize{12}{14}\selectfont%
}
\newcommand{\exFourP}{%
   \color{RedViolet}%
   \fontsize{11}{13}\selectfont%
}
\newcommand{\exPP}{%
   \color{RedViolet}%
   \fontsize{12}{14}\selectfont%
}
\newcommand{\exPPP}{%
   \color{VioletRed}%
   \fontsize{12}{14}\selectfont%
}

% Homework standard below (God the bloat in the header is absurd...)
% ~~~~~~~~~~~~~~~~~~~~~~~~~~~~~~~~~~~~~~~~~~~~~~~~
\newcommand{\Hstatement}{%
   \color{MidnightBlue!90!Black}%
   \fontsize{12}{13}\selectfont%
}
\newcommand{\HexOne}{%
   \color{Purple}%
   \fontsize{12}{13}\selectfont%
}
\newcommand{\HexTwoP}{%
   \color{RedViolet}%
   \fontsize{12}{13}\selectfont%
}
\newcommand{\HexPPP}{%
   \color{VioletRed}%
   \fontsize{11}{12}\selectfont%
}

% ~~~~~~~~~~~~~~~~~~~~~~~~~~~~~~~~~~~~~~~~~~~~~~~~

\newcommand{\cyPen}[1]{{\vphantom{.}\color{Cerulean}#1}}
\newcommand{\redPen}[1]{{\vphantom{.}\color{Red}#1}}

\newenvironment{myIndent}{%
   \begin{adjustwidth}{2.5em}{0em}%
}{%
   \end{adjustwidth}%
}

\newenvironment{myDindent}{%
   \begin{adjustwidth}{5em}{0em}%
}{%
   \end{adjustwidth}%
}

\newenvironment{myTindent}{%
   \begin{adjustwidth}{7.5em}{0em}%
}{%
   \end{adjustwidth}%
}

\newenvironment{myConstrict}{%
   \begin{adjustwidth}{2.5em}{2.5em}%
}{%
   \end{adjustwidth}%
}

\newcommand{\udefine}[1]{{%
   \setulcolor{Red}%
   \setul{0.14em}{0.07em}%
   \ul{#1}%
}}

\newcommand{\uprop}[1]{{%
   \setulcolor{Purple}%
   \setul{0.14em}{0.07em}%
   \ul{#1} 
}}

\newcommand{\blab}[1]{\textbf{#1}}
\newcommand{\blect}[1]{{\color{MidnightBlue}\textbf{#1}}}

\newcommand{\uuline}[2][.]{%
{\vphantom{a}\color{#1}%
\rlap{\rule[-0.18em]{\widthof{#2}}{0.06em}}%
\rlap{\rule[-0.32em]{\widthof{#2}}{0.06em}}}%
#2}

\newcommand{\pprime}{{\prime\prime}}
\newcommand{\suchthat}{ \hspace{0.3em}s.t.\hspace{0.3em}}
\newcommand{\rea}[1]{\mathrm{Re}(#1)}
\newcommand{\ima}[1]{\mathrm{Im}(#1)}
\newcommand{\comp}{\mathsf{C}}
\newcommand{\trans}{\mathsf{T}}
\newcommand{\myHS}{ \hspace{0.5em}}
\newcommand{\gap}{\phantom{2}}

\newcommand{\GenLin}{\ensuremath{\mathrm{GL}}}
\newcommand{\Cay}{\ensuremath{\mathrm{Cay}}}

\newcommand{\myId}{\mathrm{Id}}
\newcommand{\myIm}{\mathrm{im}}
\newcommand{\Obj}{\mathrm{Obj}}
\newcommand{\Hom}{\mathrm{Hom}}
\newcommand{\End}{\mathrm{End}}
\newcommand{\Aut}{\mathrm{Aut}}

\newcommand{\df}{\mathrm{d}}
\newcommand{\Df}{\mathrm{D}}

\newcommand{\mcateg}[1]{{\bm{\mathsf{#1}}}}

\newcommand{\mdeg}{\mathrm{mdeg}\phantom{.}}

\newcommand{\divides}{\mathop{\mid}}

\newcommand{\card}{\mathrm{card}}
\newcommand{\supp}{\mathrm{supp}}
\newcommand{\diam}{\mathrm{diam}}
\newcommand{\conv}{\mathrm{conv}}
\newcommand{\opnorm}{\mathrm{op}}
\newcommand{\loc}{\mathrm{loc}}
\newcommand{\sgn}{\mathrm{sgn}}
\newcommand{\acc}{\mathrm{acc}}

\newcommand{\mSpan}{\mathrm{span}}
\newcommand{\Interior}{\mathop{\mathrm{Int}}}

\newcommand{\mMat}[1]{\mathbf{#1}}

\newcommand{\NBV}{\ensuremath{\mathrm{NBV}}}
\newcommand{\Acc}{\mathrm{Acc}}
\newcommand{\BV}{\ensuremath{\mathrm{BV}}}
\newcommand{\Var}{\ensuremath{\mathrm{Var}}}

\newcommand{\Alt}{\mathrm{Alt}}
\newcommand{\Sym}{\mathrm{Sym}}

\newcommand{\weakst}{weak$^*$ }

\newcommand{\radtimes}{\mathop{\widehat{\times}}}

\newcommand{\mMod}[1]{\phantom{a}(\mathrel{\mathrm{mod}} #1)}
\newcommand{\Fun}{\mathrm{Fun}}
\newcommand{\act}{\mathrm{act}}
\newcommand{\Fix}{\mathrm{Fix}}
\newcommand{\Sub}{\mathrm{Sub}}
\newcommand{\Cl}{\mathrm{Cl}}
\newcommand{\GL}{\mathrm{GL}}
\newcommand{\SL}{\mathrm{SL}}
\newcommand{\core}{\mathrm{core}}
\newcommand{\Syl}{\mathrm{Syl}}
\newcommand{\Iso}{\mathrm{Iso}}
\newcommand{\Homeo}{\mathrm{Homeo}}
\newcommand{\Inn}{\mathrm{Inn}}


\DeclareMathOperator{\lcm}{lcm}
\DeclareMathOperator{\symdif}{\triangle}
\DeclareMathOperator{\Average}{Average}
\DeclareMathOperator*{\AverageAst}{Average}

% Thank you Gonzalo Medina and Moriambar who wrote this on stack exchange:
%https://tex.stackexchange.com/questions/74125/how-do-i-put-text-over-symbols%
\newcommand{\myequiv}[1]{\stackrel{\mathclap{\mbox{\footnotesize{$#1$}}}}{\equiv}}

% Thank you chs who wrote this on stack exchange:
%https://tex.stackexchange.com/questions/89821/how-to-draw-a-solid-colored-circle%
\newcommand{\filledcirc}[1][.]{\ensuremath{\hspace{0.05em}{\color{#1}\bullet}\mathllap{\circ}\hspace{0.05em}}}

%Thank you blerbl who wrote this on stack exchange:
%https://tex.stackexchange.com/questions/25348/latex-symbol-for-does-not-divide
\newcommand{\ndiv}{\hspace{-0.3em}\not|\hspace{0.35em}}

\newcommand{\mySepOne}[1][.]{%
   {\noindent\color{#1}{\rule{6.5in}{1mm}}}\\%
}
\newcommand{\mySepTwo}[1][.]{%
   {\noindent\color{#1}{\rule{6.5in}{0.5mm}}}\\%
}
\newcommand{\mySepThree}[1][.]{%
   {\noindent\color{#1}{\rule{6in}{0.25mm}}}\\%
}

\newenvironment{myClosureOne}[2][.]{%
   \color{#1}%
   \begin{tabular}{|p{#2in}|} \hline \\%
}{%
   \\ \hline \end{tabular}%
}

\newcommand{\retTwo}{\hfill\bigbreak}

\newcommand{\dispDate}[1]{{
   \color{Black}%
   \fontsize{20}{18}\selectfont%
   #1\retTwo
}}


\begin{document}
\setul{0.14em}{0.07em}
\calibri

\hTwo \hypertarget{math 200a lecture 6}{\blect{Math 200a (lecture 6)}}\retTwo

\exTwo\ul{Sylow's 2nd. Theorem:} Suppose $P_0 \in \Syl_p(G)$ and $Q$ is a $p$-subgroup of $G$. Then there exists $x \in G$ such that $Q \subseteq xP_0 x^{-1}$.

\begin{myIndent}\exThreeP
	Proof:\\
	$Q \curvearrowright G / P_0$ by left-translation. Then by the theorem in the middle of \inLinkRap{Alireza theorem page 271}{page 271}, we\\ have that $|G / P_0| \equiv |(G/P_0)^Q| \mMod{p}$. But since $P_0 \in \Syl_p(G)$, we have that\\ $|G/P_0| \not\equiv 0 \mMod{p}$. Hence, there must exist some $gP_0 \in (G / P_0)^Q$.\retTwo

	In turn, $xgP = gP$ for all $x \in Q$. Or in other words, $g \in gPg^{-1}$ for all $x \in Q$. Hence, $Q \subseteq gPg^{-1}$. $\blacksquare$\retTwo
\end{myIndent}

\ul{Corollary:} If $P_1, P_2 \in \Syl_p(G)$ then there exists $g \in G$ such that $gP_1g^{-1} = P_2$.

\begin{myIndent}\exThreeP
	Proof:\\
	By Sylow's 2nd theorem we know there exists $g \in G$ such that $P_2 \subseteq gP_1g^{-1}$. And since $|P_2| = |P_1|$ we deduce $P_2 = gP_1g^{-1}$. $\blacksquare$\retTwo
\end{myIndent}

\hTwo Note the following observations:\\ [-20pt]
\begin{itemize}
	\item If $\theta \in \Aut(G)$ and $P \in \Syl_p(G)$ then $\theta(P) \in \Syl_p(G)$.
	\item $G \curvearrowright \Syl_p(G)$ by conjugation and this actions is transitive (by the last corollary).
	\item A subgroup $H < G$ is called a \udefine{characteristic} subgroup if $\forall \theta \in \Aut(G)$ we have\\ that $\theta(H) = H$. By the last two observations, if $\Syl_p(G) = \{P\}$, then $P$ is a\\ characteristic subgroup of $G$ {\color{BrickRed}(which automatically means $P$ is normal since\\ conjugation is an automorphism of $G$)}.\retTwo
\end{itemize}

\exTwo\ul{Corollary:} If $P \lhd G$ and $P \in \Syl_p(G)$, then $P$ is a characteristic subgroup of $G$.

\begin{myIndent}\exThreeP
	Proof:\\
	Since $P \lhd G$, $P \in \Syl_p(G)$, and $G \curvearrowright \Syl_p(G)$ transitively via conjugation, we must have that $\Syl_p(G) = \{P\}$. Hence $P$ is a characteristic subgroup of $G$. $\blacksquare$\retTwo
\end{myIndent}

\hTwo\mySepTwo

\exTwo\ul{Lemma:} If $P \in \Syl_p(G)$, then $\Syl_p(N_G(P)) = \{P\}$.

\begin{myIndent}\exThreeP
	Proof:\\
	We know $|P| = p^{\nu_p(|G|)}$. Also, $P < N_G(P) < G$ means that $|P|$ divides $|N_G(P)|$ and $|N_G(P)|$ divides $|G|$. Thus $\nu_p(|G|) = \nu_p(|N_G(P)|)$ and so $P \in \Syl_p(N_G(P))$. Finally, since $P \lhd N_G(P)$, we know from the last corollary that $\Syl_p(N_G(P)) = \{P\}$. $\blacksquare$\retTwo
\end{myIndent}

\ul{Lemma:} If $P_0 \in \Syl_p(G)$ and we consider $P_0 \curvearrowright \Syl_p(G)$ by conjugation, then\\ $(\Syl_p(G))^{P_0} = \{P_0\}$.

\begin{myIndent}\exThreeP
	Proof:\\
	$P \in (\Syl_p(G))^{P_0}$ if and only if for all $x\in P_0$, $xPx^{-1} = P$. That's to say, iff $P_0 \subseteq N_G(P)$. But that would mean $P_0 \in \Syl_p(N_G(P)) = \{P\}$. So $(\Syl_p(G))^{P_0} = \{P_0\}$. $\blacksquare$\newpage
\end{myIndent}

\ul{Sylow's 3rd. Theorem:} If $G$ is a finite group, $|\Syl_p(G)| \equiv 1 \mMod{p}$.

\begin{myIndent}\exThreeP
	Proof:\\
	Suppose $P_0 \in \Syl_p(G)$. Then $|\Syl_p(G) \equiv |(\Syl_p(G))^{P_0} \mMod{p}$. But from the prior lemma we know $|(\Syl_p(G))^{P_0}| = 1$. $\blacksquare$\retTwo

	\myComment So as a recap, suppose $G$ is a finite group and $p$ is a prime number dividing $|G|$. Then:
	\begin{itemize}
		\item Sylow's first theorem guarentees that $\Syl_p(G) \neq \emptyset$.
		\item Sylow's third theorem guarentees that $|\Syl_p(G)| \equiv 1 \mMod{p}$.
		\item Sylow's second theorem guarentees that $|\Syl_p(G)|$ equals the number of conjugates of $P_0$ where $P_0 \in \Syl_p(G)$. Thus (see  \inLinkRap{page 271 reference}{page 271}), we have for any $P_0 \in \Syl_p(G)$ that $|\Syl_p(G)| = [G : N_G(P_0)]$. And in particular, since $P_0 < N_G(P_0) < G$, we have that $|\Syl_p(G)| = \frac{|G|}{|N_G(P_0)|} = \frac{[G : P_0]|P_0|}{[N_G(P_0) : P_0]|P_0|} = \frac{[G : P_0]}{[N_G(P_0) : P_0]}$. So $|\Syl_p(G)|$ divides $[G : P_0]$.\retTwo
	\end{itemize}
\end{myIndent}

\ul{Proposition:} $P \in \Syl_p(G) \Longrightarrow N_G(N_G(P)) = N_G(P)$.

\begin{myIndent}\exThreeP
	Proof:\\
	The $\supseteq$ inclusion is obvious. Meanwhile, $x \in N_G(N_G(P))$ implies that\\ $xN_G(P)x^{-1} = N_G(P)$. But note that if $\theta \in \Aut(G)$ and $H < G$, then\\ $\theta(N_G(H)) = N_G(\theta(H))$.

	\begin{myIndent}\exPPP
		If $x \in N_G(H)$ then we know that $xHx^{-1} = H$. So:
		
		{\centering $\phi(x)\phi(H)\phi(x)^{-1} = \phi(xHx^{-1}) = \phi(H)$.\retTwo\par}
		
		This shows that $\phi(x) \in N_G(\phi(H))$ and hence $\phi(N_G(H)) \subseteq N_G(\phi(H))$ whenever $\phi \in \Aut(G)$. Using this fact, now note that for any $\phi \in \Aut(G)$, we have that:
		
		{\centering$N_G(H) = \phi^{-1}(\phi(N_G(H))) \subseteq \phi^{-1}(N_G(\phi(H))) \subseteq N_G(\phi^{-1}(\phi(H))) = N_G(H)$\retTwo\par}

		So, $N_G(H) = \phi^{-1}(N_G(\phi(H)))$. And by composing $\phi$ we get that:
		
		{\centering$\phi(N_G(H)) = N_G(\phi(H))$.\retTwo\par}
	\end{myIndent}

	It follows that $N_G(xPx^{-1}) = xN_G(P)x^{-1} = N_G(P)$ whenever $x \in N_G(N_G(P))$. But in that case we have that $\Syl_p(N_G(xPx^{-1})) = \Syl_p(N_G(P))$. And as $P$ and $xPx^{-1}$ are both Sylow $p$-groups, we conclude $xPx^{-1} = P$. So $x \in N_G(P)$
\end{myIndent}

\hTwo\mySepTwo

I probably should have been taught this in math 100a but never was. So, I guess I'll just refresh myself now. The book I'm following along with is \textit{Abstract Algebra} by Dummit and Foote.\retTwo

Suppose $G$ is a group and $H, K$ are subgroups of $G$. Then we define:

{\centering$HK \coloneqq \{ hk \in G : h \in H \text{ and } k \in K\}$.\retTwo\par}

\exTwo\ul{Proposition 3.2.13:} If $H$ and $K$ are finite subgroups of a group, then $|HK| = \frac{|H||K|}{|H\cap K|}$.

\begin{myIndent}\exThreeP
	Proof:\\
	Note that $HK = \bigcup_{h \in H} hK$. Thus $|HK|$ equals $|K|$ times the number of distinct left cosets $hK$ where $h \in H$. But note that for any $h_1, h_2 \in H$:
	
	{\centering$h_1 K = h_2 K \Longleftrightarrow h_2^{-1}h_1 \in H \cap K \Longleftrightarrow h_1 (H \cap K) = h_2 (H \cap K)$.\newpage\par}

	Hence $|HK| = |K| \cdot [H : H \cap K] = |K|\frac{|H|}{|H \cap K|}$ by Lagrange's theorem. $\blacksquare$\retTwo
\end{myIndent}

\ul{Proposition 3.2.14:} If $H$ and $K$ are subgroups of $G$, then $HK < G$ iff $HK = KH$.

\begin{myIndent}\exThreeP
	$(\Longrightarrow)$\\
	Suppose $HK < G$. Since $K < HK$ and $H < HK$, we thus know that $KH \subseteq HK$.\\ Meanwhile, suppose $h \in H$ and $k \in K$. Since $HK$ is a group, we know $(hk)^{-1} \in HK$.\\ So there exists $h^\prime \in H$ and $k^\prime \in K$ such that $(hk)^{-1} = h^\prime k^\prime$. But then $hk = (k^\prime)^{-1}(h^\prime)^{-1}$\\ which is in $KH$. So $HK \subseteq KH$.\retTwo

	$(\Longleftarrow)$\\
	Assume $HK = KH$ and let $a, b \in HK$. Then there exists $h_1, h_2 \in H$ and $k_1, k_2 \in K$ such that $a = h_1k_1$ and $b = h_2k_2$. Now it's clear that $1_G \in HK$. So, if we can show that $ab^{-1} \in HK$, then we will know that $HK$ is a group.\retTwo
	
	Fortunately, $ab^{-1} = h_1k_1k_2^{-1}h_2^{-1}$. And since $KH = HK$, we know there is $h_3 \in H$ and $k_3 \in K$ such that $(k_1k_2^{-1})h_2^{-1} = h_3 k_3$. Thus, $ab^{-1} = (h_1h_3)(k_3) \in HK$. $\blacksquare$\retTwo
\end{myIndent}

\ul{Corollary 3.2.15:} If $H$ and $K$ are subgroups of $G$ and $H < N_G(K)$, then $HK$ is a subgroup of $G$. In particular, if $K \lhd G$ then $HK < G$ for any $H < G$.

\begin{myIndent}\exThreeP
	Proof:\\
	Let $h \in H$ and $k \in K$. Then $hkh^{-1} \in K$. So $hk = (hkh^{-1})h \in KH$ and we've proven that $HK = KH$. $\blacksquare$\retTwo
\end{myIndent}

\ul{Second Isomorphism Theorem:} Let $G$ be a group, let $A$ and $B$ be subgroups of $G$, and assume $A < N_G(B)$. Then $AB < G$, $B \lhd AB$, $A \cap B \lhd A$, and $AB/B \cong A/(A\cap B)$.

\begin{myIndent}\exThreeP
	Proof:\\
	By the last corollary we know that $AB < G$. Also, since $A < N_G(B)$ and $B < N_G(B)$, it follows $AB < N_G(B)$. Hence $B \lhd AB$.\retTwo

	Now we know there is a well-defined group homomorphism $\phi: A \to AB/B$ given by $\phi(a) = aB$. Clearly $\phi$ is surjective. Meanwhile, it's easy to see that $\ker(\phi) = A \cap B$. So by the first isomorphism theorem, we have that $A \cap B \lhd A$ and that:
	
	{\centering $AB/B \cong A/(A\cap B)$. $\blacksquare$\retTwo\par}
\end{myIndent}

\hTwo Here is one more miscellaneous result before getting back to the lecture:\retTwo

\exTwo\ul{Lemma:} If $N_1, N_2 \lhd G$, then $\forall x \in N_1$ and $\forall y \in N_2$ we have that $xyx^{-1}y^{-1} \in N_1 \cap N_2$.

\begin{myIndent}\exThreeP
	Proof:\\
	$(xyx^{-1}) \in N_2$ and $(yx^{-1}y^{-1}) \in N_1$ since both $N_1$ and $N_2$ are normal. Hence:
	
	{\centering$(xyx^{-1})y^{-1} = x(yx^{-1}y^{-1}) \in N_1 \cap N_2$. $\blacksquare$\retTwo\par}
\end{myIndent}

\ul{Corollary:} If $N_1, N_2 \lhd G$ and $N_1 \cap N_2 = \{1\}$, then $xy = yx$ for all $x \in N_1$ and $y \in N_2$.

\hTwo\mySepTwo

So here are some uses of Sylow's theorems:
\begin{itemize}
	\item Suppose $p < q$ are distinct primes with $p \not\divides q- 1$. If $|G| = pq$ then $G \cong C_{pq}$.
	
	\begin{myIndent}\pracTwo
		Let $s_q$ and $s_p$ be shorthand for $|\Syl_q(G)|$ and $|\Syl_p(G)|$. Now we know by Sylow's third theorem that $s_q \equiv 1 \mMod{q}$.\newpage
		
		Also, we know that $s_q \divides p$ by Sylow's second theorem. And since $p < q$, we must have that $s_q = 1$. Hence $\Syl_q(G) = \{Q\}$ for some $Q \lhd G$ such that $|Q| = q$ and $Q$ is cylic of order $q$.\retTwo

		Next, note once again by Sylow's second theorem that $s_p \divides q$. Hence, we must have that either $s_p = 1$ or $s_p = q$. That said, we know $q - 1 \not\equiv 0 \mMod{p}$ by assumption and that $s_p \equiv 1 \mMod{p}$ by Sylow's third theorem. So, we must have that $s_p = 1$ and it follows that $\Syl_p(G) = \{P\}$ for some $P \lhd G$ such that $|P| = p$ and $P$ is cyclic of order $p$.\retTwo

		Now $|P \cap Q| \divides \gcd(|P|, |Q|) = 1$. So $P \cap Q = \{1\}$. And by our prior corollary, this means that $xy = yx$ for all $x \in P$ and $y \in Q$.\retTwo

		Now consider the map $f : P \times Q \to G$ given by $(x, y) \mapsto xy$. We claim this is a group isomorphism.
		\begin{itemize}
			\item[$\bullet$] Note that:
			
			{\centering\begin{tabular}{l}
				$f(x_1, y_1)f(x_2, y_2) = x_1y_1x_2y_2 = x_1x_2y_1y_2$\\
				$\phantom{f(x_1, y_1)f(x_2, y_2) = x_1y_1x_2y_2} = f(x_1x_2, y_1y_2) = f((x_1, y_1)(x_2,y_2))$.
			\end{tabular}\retTwo\par}

			Thus $f$ is a group homomorphism.\retTwo

			\item[$\bullet$] Suppose $f(x, y) = 1$. Then $xy = 1$ which means that $x = y^{-1}$. But now $x, y^{-1} \in P \cap Q = \{1\}$. So $(x, y) = (1, 1)$ and we've shown that $f$ is injective.\retTwo
			
			\item[$\bullet$] $|\myIm(f)| = |PQ| = \frac{|P||Q|}{|P \cap Q|} = \frac{pq}{1} = |G|$. So $f$ is surjective.\retTwo
		\end{itemize}

		It follows that $G \cong P \times Q \cong \mathbb{Z}/p\mathbb{Z} \times \mathbb{Z}/q\mathbb{Z} \cong \mathbb{Z}/pq\mathbb{Z}$. (The last equivalence is from Chinese remainder theorem\dots)
	\end{myIndent}
\end{itemize}

I am not fully caught up with this class yet. But, I'll stop here for now so that I can go back to taking functional analysis notes. For more math 200a notes go to \inLinkRap{math 200a lecture 7}{page \_\_\_}.

\mySepTwo

\dispDate{10/14/2025}


% ~~~~~~~~~~~~~~~~~~~~~~~~~~~~~~~~~~~~~~~~~~~~~~

\hypertarget{Ergodic reading group notes 3}{}

\hypertarget{more function analysis lectures 3-5}{}
\hypertarget{existence and uniqueness diff eq notes}{}

\hypertarget{page 251 reference}{}
\hypertarget{page 271 reference}{}
\hypertarget{Alireza theorem page 271}{}

\hypertarget{math 241a lecture 4}{}
\hypertarget{math 200a lecture 7}{}
\hypertarget{math 220a lecture 8}{}

\end{document}




